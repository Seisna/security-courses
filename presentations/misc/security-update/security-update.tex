\documentclass[20pt,landscape,a4paper,footrule]{foils}
\usepackage{zencurity-slides}


\begin{document}
\selectlanguage{danish}


\mytitlepage{Security Update}{End of 2018}

Security areas to look into when 2019 comes around.

Very short presentation intended to create security awareness by listing some of the security problems we face. Then listing some ideas, methods and tools to improve the situation.


\slide{Goals for today}

\begin{list1}
\item Discuss IT-security
\item Create Awareness
\item Client security
\item Server security
\item Short pentest introduction
\end{list1}


\slide{Paranoia defined}

\hlkimage{15cm}{paranoia-definition.png}

Source: google paranoia definition

\slide{Confidentiality Integrity Availability}

\hlkimage{10cm}{cia-triad-uk.pdf}

\begin{list1}
\item We want to protect something
\item Confidentiality - data holdes hemmelige
\item Integrity - data ændres ikke uautoriseret
\item Availability - data og systemet er tilgeængelig når de skal bruges
\end{list1}

\slide{What is data?}
\hlkimage{10cm}{Linus3-04041999.jpg}

\begin{list1}
\item Personal data you dont want to loose:
\begin{list2}
\item Wedding pictures
\item Pictures of your children
\item Sextapes
\item Personal finances
\end{list2}
\end{list1}

Source: picture of my son less than 24 hours old - precious!


\slide{The current situation}

\hlkimage{8cm}{homer-end-is-near.jpg}
\begin{list1}
\item Internet security sucks, laptops suck at security
\item Mobile devices suck even more at security - less CPU/MEM/storage
\item We depend on cloud services and underfunded infrastructure - OpenSSL
\item We depend on others and the whole internet - DDoS
\item New vulnerabilities, while we are already dealing with those from yesterday
\end{list1}


\slide{Heartbleed CVE-2014-0160}

\hlkimage{22cm}{heartbleed-com.png}

Source: \link{http://heartbleed.com/}


\slide{Heartbleed hacking}

\begin{alltt}\footnotesize
  06b0: 2D 63 61 63 68 65 0D 0A 43 61 63 68 65 2D 43 6F  -cache..Cache-Co
  06c0: 6E 74 72 6F 6C 3A 20 6E 6F 2D 63 61 63 68 65 0D  ntrol: no-cache.
  06d0: 0A 0D 0A 61 63 74 69 6F 6E 3D 67 63 5F 69 6E 73  ...action=gc_ins
  06e0: 65 72 74 5F 6F 72 64 65 72 26 62 69 6C 6C 6E 6F  ert_order&billno
  06f0: 3D 50 5A 4B 31 31 30 31 26 70 61 79 6D 65 6E 74  =PZK1101&payment
  0700: 5F 69 64 3D 31 26 63 61 72 64 5F 6E 75 6D 62 65  _id=1&{\bf card_numbe}
  0710: XX XX XX XX XX XX XX XX XX XX XX XX XX XX XX XX  {\bf r=4060xxxx413xxx}
  0720: 39 36 26 63 61 72 64 5F 65 78 70 5F 6D 6F 6E 74  {\bf 96&card_exp_mont}
  0730: 68 3D 30 32 26 63 61 72 64 5F 65 78 70 5F 79 65  {\bf h=02&card_exp_ye}
  0740: 61 72 3D 31 37 26 63 61 72 64 5F 63 76 6E 3D 31  {\bf ar=17&card_cvn=1}
  0750: 30 39 F8 6C 1B E5 72 CA 61 4D 06 4E B3 54 BC DA  {\bf 09}.l..r.aM.N.T..
\end{alltt}

\begin{list2}
\item Obtained using Heartbleed proof of concepts - Gave full credit card details
\item "can XXX be exploited" - yes, clearly! PoCs ARE needed\\
without PoCs even Akamai wouldn't have repaired completely!
\item The internet was ALMOST fooled into thinking getting private keys from Heartbleed was not possible - scary indeed.
\end{list2}


\slide{Exploits og sårbarheder}

\begin{list1}
\item Exploits som Heartbleed udnytter sårbarheder
\item Kræver nogle forudsætninger
\item Sårbarheder i software
\item Sårbarheder pga dårlige indstillinger
\item Dårlige passwords
\end{list1}

\slide{Clients}

\begin{list1}
\item Vores client systemer er idag oftest:
\item Laptops
\item Mobile enheder
\item Delte stationære
\end{list1}

\slide{Malware charateristics}

\begin{list1}
\item Malware is advanced and sophisticated
\item Modular frameworks
\item Use strong cryptography to hide, and hide your data ransomware
\item Use 0-day exploits - unknown to others
\item Use rootkits to stay under radar and avoid anti-virus
\item Mutate and change to avoid detection
\item In general less noisy
\end{list1}

\slide{Botnets and malware sold with support}

\hlkimage{21cm}{dagens-tilbud-trojanere.pdf}

\begin{list1}
\item Malware programmers act like software houses
\item "Buy this version with updates and support"
\item Rent a bot net with 100.000 computers
\end{list1}


\slide{Phishing - Receipt for Your Payment to mark561@bt....com}
\hlkimage{21cm}{paypal-phish.png}
\vskip 1cm
\centerline{\bf\LARGE Do you recognize Phishing?}

\slide{Client sikkerhed - Back to basics}

\hlkimage{12cm}{backup-1.png}


\begin{list1}
\item Opdateringer er altid nødvendige, vi skal opfordre til opdateringer
\item Firewall aktiveret altid
\item Phishing og exploits via email - spearphishing er stor risiko
\item Procedurer og rapportering, gør det nemt
\end{list1}

\vskip 5mm

\centerline{\bf Backups gør at vi kan opdatere uden risiko, genskabe data}

\vskip 5mm

PS Måske bruges en cloud service istedet for bånd, men offline er godt

\slide{Risk management defined}

\hlkimage{23cm}{shon-harris-risk-management.png}

Source: Shon Harris \emph{CISSP All-in-One Exam Guide}

\vskip 2cm
\centerline{We all take risks every day - sometimes even calculated risk}

\slide{Server sikkerhed}

\begin{list1}
\item Vi mener servere, og tilhørende infrastruktur
\item Servere - server OS og applikationer
\item KVM: IPMI, Dell DRAC, HP ILO, adskilte management netværk
\item Netværksenheder brug VPN til at tilgå alle administrative interfaces
\item Firewalls - naturligvis, men flere zoner, DMZ
\end{list1}

\slide{Solutions}

\begin{list1}
\item Automate your job, Ansible is our preferred tool
\item Backup your life, help others backup, Duplicity is my choice
\item Use hackertools to detect and identify
\item Categorise, sort, prioritize, group problems - solve more
\item Measure, collect and present - make it pretty
\item Learn from devops, Elasticsearch Logstash Kibana, Grafana
\item \link{http://ssd.eff.org} Learn self-defense for yourself, practice infosec war
\end{list1}


\slide{Daily operations}

\begin{list1}
\item Asset ownership
Routers, servers, network devices, services
\item User management
Lots of things, including Active Directory
\item Patch management
\item Host and services ownership, who owns the services
\end{list1}

\slide{Monthly and quarterly checks}

\begin{list1}
\item Crypto stuff
HTTPS certificates, TLS in general, use 4k keys, \\
TLS 1.2+, \url{https://www.ssllabs.com/}
\item VPN settings: IPsec, VPN, L2TP, key lengths, roll shared PSK
\item Routing, internet service provider contacts, whois
\item Firewall review - read all rules, disable unneeded
\item Review DNS and mail settings, DMARC, DNSSEC, DKIM
\end{list1}


\slide{Use standard to take Charge of Your Security}

Multiple standards:
\begin{list2}
\item ISO/IEC 27001 - information security management system standards\\
\link{http://en.wikipedia.org/wiki/ISO/IEC_27001}
\item SSAE 16 No. 16, Reporting on Controls at a Service Organization\\
Statement on Standards for Attestation Engagements (SSAE) \link{http://ssae16.com/}
\item ISAE 3402 Assurance Reports on Controls at a Service Organization\\
International Standard on Assurance Engagements (ISAE)
\link{http://isae3402.com/}
\item Independent assessment from a trusted security firm - which must often also be certified
\item Physical Security, power, HVAC - trusted partners reviewing security
\item Implementing standards is also security
\end{list2}

Over the years organisations have become more mature with regards to security

Implementing security controls were easier when you owned all resources

\slide{Balanced security}

\hlkimage{21cm}{afbalanceret-sikkerhed.pdf}

\begin{list1}
\item Better to have the same level of security
\item If you have bad security in some part - guess where attackers will end up
\item Hackers are not required to take the hardest path into the network
\item Realize there is no such thing as 100\% security
\end{list1}


\slide{Defense in depth - layered security}

\hlkimage{8cm}{security-layers-1-uk.pdf}

\centerline{\hlkbig\color{security6blue} Multiple layers of security! Isolation!}


\slide{First advice use the modern operating systems}

\begin{list1}
\item Newer versions of Microsoft Windows, Mac OS X and Linux
\begin{list2}
\item Buffer overflow protection
\item Stack protection, non-executable stack
\item Heap protection, non-executable heap
\item \emph{Randomization of parameters} stack gap m.v.
\end{list2}
\item Note: these still have errors and bugs, but are better than older versions

\end{list1}

\vskip 1cm

\centerline{Always try to make life worse and more costly for attackers}

\slide{Network tools - examples}

\hlkimage{16cm}{kibana-solido.png}
\begin{list1}
\item Net: Zeek \link{http://www.bro-ids.org} Suricata \link{http://suricata-ids.org}
\item DNS: DSC and PacketQ \link{https://github.com/dotse/packetq/wiki}
\item Syslog: Elasticsearch, Logstash, and Kibana, called ELK stack or Elastic stack
\item Packetbeat \link{https://www.elastic.co/products/beats/packetbeat}
\end{list1}
\centerline{Collect and present data more easily - non-programmers}





\slide{Security engineering som job rolle}

\begin{alltt}\small
On any given day, you may be challenged to:

	Create new ways to solve existing production security issues
	Configure and install firewalls and intrusion detection systems
	Perform vulnerability testing, risk analyses and security assessments
	Develop automation scripts to handle and track incidents
	Investigate intrusion incidents, conduct forensic investigations and incident responses
	Collaborate with colleagues on authentication, authorization and encryption solutions
	Evaluate new technologies and processes that enhance security capabilities
	Test security solutions using industry standard analysis criteria
	Deliver technical reports and formal papers on test findings
	Respond to information security issues during each stage of a project’s lifecycle
	Supervise changes in software, hardware, facilities, telecommunications and user needs
	Define, implement and maintain corporate security policies
	Analyze and advise on new security technologies and program conformance
	Recommend modifications in legal, technical and regulatory areas that affect IT security
\end{alltt}

Source: \url{https://www.cyberdegrees.org/jobs/security-engineer/}\\
also
\url{https://en.wikipedia.org/wiki/Security_engineering}

\slide{Pentesting as example}

\begin{list1}
\item Penetration testing
\item Kontrol af sikkerheden
\item Bruger aktive værktøjer
\item Brug Nmap pakken til at checke åbne porte
\end{list1}


\slide{Trinity breaking in}

\hlkimage{20cm}{trinity-nmapscreen-hd-cropscale-418x250.jpg}
\link{http://nmap.org/movies.html}\\
Meget realistisk \link{https://www.youtube.com/watch?v=0PxTAn4g20U}

\slide{Hackertools are for everyone!}

\hlkimage{2cm}{hackers_JOLIE+1995.jpg}


\begin{list2}
\item Hackers work all the time to break stuff, Use hackertools:
\item Nmap, Nping \link{http://nmap.org}
\item Wireshark - \link{http://www.wireshark.org/}
\item Aircrack-ng \link{http://www.aircrack-ng.org/}
\item Metasploit Framework \link{http://www.metasploit.com/}
\item Burpsuite \link{http://portswigger.net/burp/}
\item Skipfish \link{http://code.google.com/p/skipfish/}
\item Kali Linux \link{http://www.kali.org}
\end{list2}

\vskip 5mm
\centerline{Most popular hacker tools \link{http://sectools.org/}}



\slide{Nping - ligesom ping, blot med flere protokoller}

\begin{alltt}\footnotesize
  root@KaliVM:~# nping --tcp -p 80 www.zencurity.com

  Starting Nping 0.7.70 ( https://nmap.org/nping ) at 2018-09-07 19:06 CEST
  SENT (0.0300s) TCP 10.137.0.24:3805 > 185.129.60.130:80 S ttl=64 id=18933 iplen=40  seq=2984847972 win=1480
  RCVD (0.0353s) TCP 185.129.60.130:80 > 10.137.0.24:3805 SA ttl=56 id=49674 iplen=44  seq=3654597698 win=16384 <mss 1460>
  SENT (1.0305s) TCP 10.137.0.24:3805 > 185.129.60.130:80 S ttl=64 id=18933 iplen=40  seq=2984847972 win=1480
  RCVD (1.0391s) TCP 185.129.60.130:80 > 10.137.0.24:3805 SA ttl=56 id=50237 iplen=44  seq=2347926491 win=16384 <mss 1460>
  SENT (2.0325s) TCP 10.137.0.24:3805 > 185.129.60.130:80 S ttl=64 id=18933 iplen=40  seq=2984847972 win=1480
  RCVD (2.0724s) TCP 185.129.60.130:80 > 10.137.0.24:3805 SA ttl=56 id=9842 iplen=44  seq=2355974413 win=16384 <mss 1460>
  SENT (3.0340s) TCP 10.137.0.24:3805 > 185.129.60.130:80 S ttl=64 id=18933 iplen=40  seq=2984847972 win=1480
  RCVD (3.0387s) TCP 185.129.60.130:80 > 10.137.0.24:3805 SA ttl=56 id=1836 iplen=44  seq=3230085295 win=16384 <mss 1460>
  SENT (4.0362s) TCP 10.137.0.24:3805 > 185.129.60.130:80 S ttl=64 id=18933 iplen=40  seq=2984847972 win=1480
  RCVD (4.0549s) TCP 185.129.60.130:80 > 10.137.0.24:3805 SA ttl=56 id=62226 iplen=44  seq=3033492220 win=16384 <mss 1460>

  Max rtt: 40.044ms | Min rtt: 4.677ms | Avg rtt: 15.398ms
  Raw packets sent: 5 (200B) | Rcvd: 5 (220B) | Lost: 0 (0.00%)
  Nping done: 1 IP address pinged in 4.07 seconds
\end{alltt}

\vskip 1cm
\centerline{Byg pakkerne med kommandolinien}

\slide{Nmap GUI - Zenmap}

\hlkimage{12cm}{nmap-zenmap.png}

\begin{quote}
Nmap ("Network Mapper") is a free and open source (license) utility for network discovery and security auditing.
\end{quote}

\centerline{Today a package of programs for Windows, Mac, BSD, Linux, ... source}






\slide{Aktiv testing What happens now?}

\begin{list1}
\item Think like a hacker
\item Recon phase -- gather information reconnaissance
\begin{list2}
\item Traceroute, Whois, DNS lookups
\item Ping sweep, port scan
\item OS detection -- TCP/IP and banner grabbing
\item Service scan -- rpcinfo, netbios, ...
\item telnet/netcat interact with services
\end{list2}
\end{list1}

\slide{Kali Linux the pentest toolbox}

\hlkimage{\linewidth-8cm}{kali-linux.png}

\begin{list1}
\item  Kali \link{http://www.kali.org/}
\item 100.000s of videos on youtube alone, searching for kali and \$TOOL
\item Also versions for Raspberry Pi, mobile and other small computers
\end{list1}


\slide{Nmap port sweep efter webservere}

\begin{alltt}\small
root@cornerstone:~#{\bfseries  nmap -p80,443 172.29.0.0/24}

Starting Nmap 6.47 ( http://nmap.org ) at 2015-02-05 07:31 CET
Nmap scan report for 172.29.0.1
Host is up (0.00016s latency).
PORT    STATE    SERVICE
{\color{darkgreen}80/tcp  open     http}
443/tcp filtered https
MAC Address: 00:50:56:C0:00:08 (VMware)

Nmap scan report for 172.29.0.138
Host is up (0.00012s latency).
PORT    STATE  SERVICE
{\color{darkgreen}80/tcp  open   http}
443/tcp closed https
MAC Address: 00:0C:29:46:22:FB (VMware)

\end{alltt}

\slide{Scan for Heartbleed and SSLv2/SSLv3}

\hlkimage{8cm}{nmap-sslv2.png}

\begin{list1}
\item \verb+nmap -p 443 --script ssl-heartbleed <target>+\\
\link{https://nmap.org/nsedoc/scripts/ssl-heartbleed.html}
\item \verb+masscan 0.0.0.0/0 -p0-65535  --heartbleed+\\
\link{https://github.com/robertdavidgraham/masscan}
\end{list1}


\myquestionspage


\end{document}
