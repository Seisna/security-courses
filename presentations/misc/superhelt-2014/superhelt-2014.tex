\documentclass[20pt,landscape,a4paper,footrule]{foils}
\usepackage{solido-network-slides}
%\usepackage{ulem}

% Basic things that we need are below
\selectlanguage{danish}

%\externaldocument{unix-audit-security-oevelser}
\externaldocument{\jobname-exercises}

\begin{document}

%Ledernes KonferenceCenter
%Rug�rdsvej 590
%5210 Odense NV
%Kl. 17-19
%Spor 1: It-sikkerhed og it-udvikling ved it-debatt�r Henrik Kramsh�j fra Solido Networks


\mytitlepage
{IT-sikkerhed 2014}{PROSA Superhelteseminar}


\slide{Generic advice}

Recommendations \hlkrightimage{8cm}{Encrypt_all_the_things.png}
\begin{list2}
\item Lock your devices, phones, tables and computers
\item Update software and apps
\item Do NOT use the same password everywhere
\item Watch out when using open wifi-networks
\item Multiple browsers: one for Facebook, and separate for home banking apps?
\item Multiple laptops? One for private data, one for work?
\item Think of the data you produce, why do people take naked pictures and SnapChat them?
\item Use pseudonyms and aliases, do not use your real name everywhere
\item Enable encryption: IMAP{\bf S} POP3{\bf S}
  HTTP{\bf S} TOR OpenPGP VPN SSL/TLS
\end{list2}



\slide{Planen idag}
\vskip 2 cm

\hlkimage{5cm}{dont-panic.png}
\centerline{\color{titlecolor}\LARGE Don't Panic!}

\begin{list1}
\item Kl 17:30-19:30 - med pause
\item Mindre enetale, mere foredrag 2.0 med sociale medier, informationsdeling og interaktion
\end{list1}
\centerline{Send gerne sp�rgsm�l senere}
%PS er her hele weekenden
{\footnotesize PS er her nogle timer efter foredraget til sp�rgsm�l og snak}



\slide{The current situation}

\hlkimage{10cm}{homer-end-is-near.jpg}
\begin{list1}
\item Internet security sucks
\item Personal computers like laptops suck at security
\item Mobile devices suck even more at security - less CPU/MEM/storage
\item We depend on cloud services and underfunded infrastructure - OpenSSL
\item We depend on others and the whole internet - DDoS
\end{list1}



\slide{Goals: Internet Ninjas}

\hlkimage{10cm}{ninjas.png}
\begin{list1}
\item Superhelte er ninajer
\item Kender internet, teknologierne og mulighederne
\item Rettidig omhu og defense in depth
\item Konsekvenserne ved d�rlig sikkerhed
\end{list1}


\slide{Form�l: sund paranoia}


\hlkimage{10cm}{password-window.png}
\centerline{Opbevaring af passwords}



\slide{Security is magic}

%\vskip 2cm

\hlkimage{4cm}{wizard_in_blue_hat.png}

\hlkrightimage{5cm}{003scawebgoshindomanicon.png}
.
%{\large Superheltegerninger}

\begin{list1}
\item Think security always appropriate paranoia
\item Follow news about software security
\item Support communties, join and learn
\end{list1}



\slide{Hackerv�rkt�jer er ogs� til dig!}
% m�ske til reference afsnit?
\hlkimage{2cm}{hackers_JOLIE+1995.jpg}


\begin{list2}
\item Hackers work all the time to break stuff, Use hackertools:
\item Nmap, Nping \link{http://nmap.org}
\item Wireshark - \link{http://http://www.wireshark.org/}
\item Aircrack-ng \link{http://www.aircrack-ng.org/}
\item Metasploit Framework \link{http://www.metasploit.com/}
%\item Paros proxy \link{http://www.parosproxy.org}
\item Burpsuite \link{http://portswigger.net/burp/}
\item Skipfish \link{http://code.google.com/p/skipfish/}
\item Kali Linux \link{http://www.kali.org}
\end{list2}

\vskip 5mm
\centerline{Most popular hacker tools \link{http://sectools.org/}}

\slide{Kali Linux the new backtrack}

\hlkimage{\linewidth-8cm}{kali-linux.png}

\begin{list1}
\item BackTrack \link{http://www.backtrack-linux.org}
\item  Kali \link{http://www.kali.org/}
\item 100.000s of videos on youtube: "kali hack" ~60.000, "backtrack hack" ~125.000
\end{list1}

\slide{it's a Unix system, I know this}


\hlkimage{24cm}{twitter-unix-security.png}

\begin{list1}
\item Skal du igang med sikkerhed?
\item Installer et netv�rk, evt. bare en VMware, Virtualbox, Parallels, Xen, GNS3, ...
\item Brug BackTrack, se evt. youtube videoer om programmerne
\end{list1}

Quote fra Jurassic Park
\link{http://www.youtube.com/watch?v=dFUlAQZB9Ng}


\slide{Metasploit and Armitage Still rocking the internet}


\hlkimage{20cm}{metasploit-about.png}

\begin{list1}
\item \link{http://www.metasploit.com/}
\item Armitage GUI fast and easy hacking for Metasploit\\
\link{http://www.fastandeasyhacking.com/}
\item Kursus Metasploit Unleashed\\
\link{http://www.offensive-security.com/metasploit-unleashed/Main_Page}
\item Bog: Metasploit: The Penetration Tester's Guide, No Starch Press\\
ISBN-10: 159327288X
\end{list1}



\slide{Fri adgang til hackerv�rkt�jer}

\begin{list1}
\item I 1993 skrev Dan Farmer og Wietse Venema artiklen\\
\emph{Improving the Security of Your Site by Breaking Into it}
\item I 1995 udgav de softwarepakken SATAN\\
\emph{Security Administrator Tool for Analyzing Networks}
\begin{quote}\large
We realize that SATAN is a two-edged sword - like many tools,\\
it can be used for good and for evil purposes. We also \\
realize that intruders (including wannabees) have much \\
more capable (read intrusive) tools than offered with SATAN.
\end{quote}
\item Se \link{http://sectools.org} og \link{http://www.packetstormsecurity.org/}

\end{list1}
Kilde:
\link{http://www.fish2.com/security/admin-guide-to-cracking.html}

\slide{Heartbleed hacking}

\begin{alltt}\footnotesize
  06b0: 2D 63 61 63 68 65 0D 0A 43 61 63 68 65 2D 43 6F  -cache..Cache-Co
  06c0: 6E 74 72 6F 6C 3A 20 6E 6F 2D 63 61 63 68 65 0D  ntrol: no-cache.
  06d0: 0A 0D 0A 61 63 74 69 6F 6E 3D 67 63 5F 69 6E 73  ...action=gc_ins
  06e0: 65 72 74 5F 6F 72 64 65 72 26 62 69 6C 6C 6E 6F  ert_order&billno
  06f0: 3D 50 5A 4B 31 31 30 31 26 70 61 79 6D 65 6E 74  =PZK1101&payment
  0700: 5F 69 64 3D 31 26 63 61 72 64 5F 6E 75 6D 62 65  _id=1&card_numbe
  0710: XX XX XX XX XX XX XX XX XX XX XX XX XX XX XX XX   r=4060xxxx413xxx
  0720: 39 36 26 63 61 72 64 5F 65 78 70 5F 6D 6F 6E 74  96&card_exp_mont
  0730: 68 3D 30 32 26 63 61 72 64 5F 65 78 70 5F 79 65  h=02&card_exp_ye
  0740: 61 72 3D 31 37 26 63 61 72 64 5F 63 76 6E 3D 31  ar=17&card_cvn=1
  0750: 30 39 F8 6C 1B E5 72 CA 61 4D 06 4E B3 54 BC DA  09.l..r.aM.N.T..
\end{alltt}

\begin{list2}
\item Obtained using Heartbleed proof of concepts - Gave full credit card details
\item "can XXX be exploited" - yes, clearly! PoCs ARE needed\\
without PoCs even Akamai wouldn't have repaired completely!
\item The internet was ALMOST fooled into thinking getting private keys from Heartbleed was not possible - scary indeed.
\end{list2}

{\footnotesize Nothing more about HB in this presentation - we have better things to discuss}


\slide{Attack overview}

\hlkimage{22cm}{sicherheitstacho.png}

{\small\link{http://www.sicherheitstacho.eu/?lang=en}}



\slide{ Netflow NFSen}

\hlkimage{22cm}{nfsen-udp-flood.png}

\centerline{An extra 100k packets per second from this netflow source (source is a router)}


\slide{Alert (TA14-017A) UDP-based Amplification Attacks}

\begin{alltt}\small
Protocol   Bandwidth Amplification Factor        Vulnerable Command
DNS        28 to 54          see: TA13-088A [1]
NTP        556.9             see: TA14-013A [2]
SNMPv2       6.3             GetBulk request
NetBIOS      3.8             Name resolution
SSDP        30.8             SEARCH request
CharGEN    358.8             Character generation request
QOTD       140.3             Quote request
BitTorrent   3.8             File search
Kad         16.3             Peer list exchange
Quake Network Protocol 63.9  Server info exchange
Steam Protocol  5.5          Server info exchange
\end{alltt}

Source: US-CERT\\
\link{http://www.us-cert.gov/ncas/alerts/TA14-017A}


\slide{Detecting DDoS}

\hlkimage{15cm}{nfsen-ddos-profile-1.png}

We created a DDoS profile with the common types.

We can ask RDDtools about max, average etc.
\begin{alltt}\footnotesize
rrdtool graph x -s -24h DEF:v=DDoS/mx-cph-01.rrd:packets:MAX VDEF:vm=v,MAXIMUM PRINT:vm:%.lf
\end{alltt}



\slide{But DNS is bad! DNS Amplification?!}

\begin{quote}
This is the official homepage for PacketQ, a simple tool to make SQL-queries against PCAP-files, making packet analysis and building statistics simple and quick. PacketQ was previously known as DNS2db but was renamed in 2011 when it was rebuilt and could handle protocols other than DNS among other things.

Look how easy it's to count DNS-packets in a PCAP-file.
\end{quote}

\begin{alltt}
\small
# packetq -s "select count(*) as count_dns from dns" packets.pcap
[ \{ "table_name": "result",
      "head": [
      \{ "name": "count_dns","type": "int" \} ],   {\bf "data": [ [95501] ] \}} ]
\end{alltt}

\link{https://github.com/dotse/packetq/wiki}\\

\slide{Using PacketQ}

\hlkimage{22cm}{using-packetq.png}

Discussion: bridging the gaps between Devops and Security? Good thing, easy?

\link{http://securityblog.switch.ch/2013/01/22/using-packetq/}




\slide{Big data tools}

\hlkimage{20cm}{kibana-solido.png}
\begin{list1}
\item Moloch \link{https://github.com/aol/moloch}
\item DSC and PacketQ \link{https://github.com/dotse/packetq/wiki}
\item Elasticsearch, Logstash, and Kibana
\end{list1}



\slide{Moloch}

\hlkimage{20cm}{moloch-sessions.png}

Picture from \link{https://github.com/aol/moloch}

% Suricata, Logstash, Elasticsearch, D3JShttp://d3js.org/
\slide{Suricata with Dashboards}

\hlkimage{12cm}{kibana-suricata.png}

Picture from Twitter\\
\link{https://twitter.com/nullthreat/status/445969209840128000}\\

New link March 2014: 10Gbits\\
{\small\link{http://pevma.blogspot.se/2014/03/suricata-prepearing-10gbps-network.html}}

\link{http://suricata-ids.org/2014/03/25/suricata-2-0-available/}


\slide{Security devops}

\begin{list1}
\item We need devops skillz in security
\item automate, security is also big data
\item integrate tools, transfer, sort, search, pattern matching, statistics, ...
\item tools, languages, databases, protocols, data formats
\item Example introductions:
\begin{list2}
\item Seven languages/database/web frameworks in Seven Weeks
\item Elasticsearch the definitive guide\\
\link{http://www.elasticsearch.org/guide/en/elasticsearch/guide/current/index.html}
\item \link{http://www.elasticsearch.org/overview/kibana/}
\item \link{http://www.elasticsearch.org/overview/logstash/}
\end{list2}
\end{list1}

\centerline{We are all Devops now, even security people!}



\slide{Network Security Through Data Analysis}

\hlkimage{8cm}{network-security-through-data-analysis.png}

Low page count, but high value! Recommended.

Network Security Through Data Analysis: Building Situational Awareness\\
By Michael Collins\\
Publisher: O'Reilly Media
Released: February 2014 Pages: 348



\slide{BRO IDS}

\hlkimage{14cm}{bro-ids.png}

\begin{quote}
While focusing on network security monitoring, Bro provides a comprehensive platform for more general network traffic analysis as well. Well grounded in more than 15 years of research, Bro has successfully bridged the traditional gap between academia and operations since its inception.
\end{quote}

\link{http://www.zeek.org/}

\slide{BRO more than an IDS}

\begin{quote}
	The key point that helped me understand was the explanation that Bro is a
               domain-specific language for networking applications and that Bro-IDS
               (http://bro-ids.org/) is an application written with Bro.
\end{quote}

Why I think you should try Bro\\
\link{https://isc.sans.edu/diary.html?storyid=15259}\\

\slide{Bro scripts}

\begin{alltt}\small
global dns_A_reply_count=0;
global dns_AAAA_reply_count=0;
...
event dns_A_reply(c: connection, msg: dns_msg, ans: dns_answer, a: addr)
	\{
	++dns_A_reply_count;
	\}

event dns_AAAA_reply(c: connection, msg: dns_msg, ans: dns_answer, a: addr)
	\{
	++dns_AAAA_reply_count;
	\}
\end{alltt}

source: dns-fire-count.bro from\\
{\small \link{https://github.com/LiamRandall/bro-scripts/tree/master/fire-scripts}}




\slide{Security Onion}

\hlkimage{12cm}{security-onion.png}
\begin{list2}
\item Security Onion is a Linux distro for IDS, NSM, and log management
\item \link{securityonion.blogspot.dk}
\item \link{http://blog.securityonion.net/p/securityonion.html}
\end{list2}


\slide{Problem: DDoS traffic before filtering}

\hlkimage{26cm}{ddos-before-filtering}

\begin{list1}
\item Problem: We receive unauthenticated chaotic traffic
\item Only two links shown, at least 3Gbit incoming for this single IP
\end{list1}

\slide{Goal: DDoS traffic after filtering}
\hlkimage{18cm}{ddos-after-filtering}

\begin{list1}
\item Solution: Discard early, discard on edge, reduce noise
\item Only use CPU resources for potentially real traffic
\item Link toward server (next level firewall actually) about ~350Mbit outgoing
\item Drinking from a water hose, eating an elephant
\item Reduce problems until manageable, divide and conquer
\end{list1}

\slide{Defense in depth - multiple layers of security}

\hlkimage{23cm}{network-layers-1.pdf}



\slide{Stateless firewall filter limit protocols}

\begin{alltt}\footnotesize
term limit-icmp \{
    from \{
        protocol icmp;
    \}
    then \{
        policer ICMP-100M;
        accept;
    \}
\}
term limit-udp \{
    from \{
        protocol udp;
    \}
    then \{
        policer UDP-1000M;
        accept;
    \}
\}
\end{alltt}

Routers also have extensive Class-of-Service (CoS) tools today

\slide{Strict filtering for some servers, still stateless!}

\begin{alltt}\footnotesize
term some-server-allow \{
    from \{
        destination-address \{
            109.238.xx.0/xx;
        \}
        protocol tcp;
        destination-port [ 80 443 ];
    \}
    then accept;
\}
term some-server-block-unneeded \{
    from \{
        destination-address \{
            109.238.xx.0/xx;
        \}
        protocol-except icmp;
    \}
    then \{
        count some-server-block;
        discard;
    \}
\}
\end{alltt}

Wut - no UDP, yes UDP service is not used on these servers


\slide{Summary: Goal of protection mechanisms}

\begin{list1}
\item DDoS attacks increase in size
\item +100Gb happens regularly
\item Even 200Gb is becoming more common
\item No vendor can deliver a single device with 100\%
\item Slice the attacks - Divide and conquer
\item Use the available features and resources in combination - optimize your infrastructure
\end{list1}

\slide{Allowed traffic to next layer}

\begin{list1}
\item Basic filtering and routers can eliminate a lot

\item Characteristics after employing the techniques:
\item Known bad sources removed
\item Maximum 100Mbit ICMP
\item Maximum 1000Mbit UDP
\item Only port 80/tcp and 443/tcp to some range
\end{list1}
\centerline{LESS traffic to consider on firewall/next device}

\slide{DDoS in 2014}

\hlkimage{16cm}{arbor-2014-ntp.png}

\begin{list2}
\item Several big players you need to research before needing them!
\item Arbor Networks sells software solutions for carriers\\
http://www.arbornetworks.com/

\item Prolexic sells DDoS services, DNS and BGP based\\
http://www.prolexic.com/

\item CloudFlare proxy based\\
http://www.cloudflare.com/
\end{list2}




\slide{Afbalanceret sikkerhed}

\hlkimage{21cm}{afbalanceret-sikkerhed.pdf}

\begin{list1}
\item Det er bedre at have et ensartet niveau
\item Hvor rammer angreb hvis du har Fort Knox et sted og kaos andre steder
\item Hackere v�lger ikke med vilje den sv�reste vej ind
\end{list1}




\slide{Udviklingsstandarder}

\begin{list1}
\item Hvad g�r I for at undg� problemer som de her n�vnte?
- kan man g�re mere?
\item Man b�re v�re klar over hvilke teknologier man bruger
\item Standardiser p� et mindre antal produkter, biblioteker, sprog
\item Regler og procedurer skal hele tiden opdateres:
\begin{list2}
\item Kvalitetssikring
\item Retningslinier for tilladte tags
\item Retningslinier for brug af SQL
\end{list2}

\item Ved at fokusere p� antallet af produkter kan man m�ske
  indskr�nke mulighederne for fejl, h�j kvalitet er ofte mere sikkert

\item {\bf nye produkter kan v�re farlige til man l�rer dem at kende!}
\end{list1}

\slide{Retningslinier}

\begin{list2}
\item Hvis der ikke findes retningslinier for udvikling s� etabler disse
\item eksempel:\\
javascript m� gerne benyttes til at validere forms for at give hurtig
feedback til brugeren
\item serveren der modtager input fra brugeren validerer alle data
  sikkerhedsm�ssigt
\item Retningslinierne er medvirkende til at foretage
en afbalanceret investering i sikkerheden
\item undg� dyre hovsa l�sninger
\item undg� huller i sikkerheden, ens niveau

\item Der findes vejledninger til b�de gamle og nye sprog/systemer, \\
eks Ruby On Rails Security Guide
\link{http://guides.rubyonrails.org/security.html}
\end{list2}



\slide{Deadly sins bogen}

\hlkimage{8cm}{24-deadly.jpg}

\begin{list1}
\item \emph{24 Deadly Sins of Software Security}
Michael Howard, David LeBlanc, John Viega 2. udgave, f�rste hed 19 Deadly Sins
\end{list1}

\slide{OWASP top ten}

\hlkimage{16cm}{owasp.jpg}

\begin{quote}
The OWASP Top Ten provides a minimum standard for web application
security. The OWASP Top Ten represents a broad consensus about what
the most critical web application security flaws are.
\end{quote}

\begin{list1}
\item The Open Web Application Security Project (OWASP)
\item OWASP har gennem flere �r udgivet en liste over de 10 vigtigste
  sikkerhedsproblemer for webapplikationer
\item \link{http://www.owasp.org}
\end{list1}

\slide{OWASP WebGoat}

\hlkimage{14cm}{WebGoat.png}

\begin{list1}
\item WebGoat fra OWASP, \link{http://www.owasp.org}
\item Tr�ningsmilj� til webhacking
\item Downloades som Zipfil og kan afvikles direkte p� en Windows laptop
\end{list1}

\link{https://www.owasp.org}

\slide{WebGoat overview}

\hlkimage{17cm}{WebGoat-overview.png}

\centerline{Perfect for learning web hacking/protection}

\slide{Burpsuite}

\begin{quote}
Burp Suite is an integrated platform for performing security testing of web applications. Its various tools work seamlessly together to support the entire testing process, from initial mapping and analysis of an application's attack surface, through to finding and exploiting security vulnerabilities.

Burp gives you full control, letting you combine advanced manual techniques with state-of-the-art automation, to make your work faster, more effective, and more fun.
\end{quote}

Burp suite indeholder b�de proxy, spider, scanner og andre v�rkt�jer i samme pakke - NB: EUR 249 per user per year.

\link{http://portswigger.net/burp/}


\slide{More Web application hacking}

\hlkimage{5cm}{images/web-app-hackers-handbook.jpeg}

\begin{list1}
\item \emph{The Web Application Hacker's Handbook: Discovering and Exploiting Security Flaws}
Dafydd Stuttard, Marcus Pinto, Wiley 2007 ISBN: 978-0470170779
\end{list1}



\slide{Protect yourself: Why think of security?}

\hlkimage{8cm}{1984-not-instruction-manual.jpg}


\begin{quote}
	Privacy is necessary for an open society in the electronic age. Privacy is not secrecy. A private matter is something one doesn't want the whole world to know, but a secret matter is something one doesn't want anybody to know. Privacy is the power to selectively reveal oneself to the world. ~A Cypherpunk's Manifesto by Eric Hughes, 1993
\end{quote}

Copied from \link{https://cryptoparty.org/wiki/CryptoParty}
\slide{Face reality}

\begin{list2}
\item Criminals sell your credit card information and identity theft
\item Trade infected computers like a commodity
\item Governments write laws that allows them to introduce back-doors - and use these
\item Governments do blanket surveillance of their population
\item Governments implement censorship, threaten citizens and journalist
\item Governments will introduce back-doors in products we use
\item Danish police and TAX authorities have the legals means, see \emph{Rockerloven}
\end{list2}

\vskip 1cm
\centerline{You are not paranoid when there are people actively attacking you!}


\slide{Use protection - always}

\hlkimage{14cm}{protect-from-governments.jpg}
%{\LARGE Protecting yourself against criminals or the government is the same thing!}

\slide{Turkey: Erdogan bans Twitter }

\hlkimage{12cm}{twitter-turkey.png}

\slide{Censorship on the internet - lol}

\centerline{\color{titlecolor} The Net interprets censorship as damage and routes around it.}

\hlkimage{22cm}{John-Gilmore-quotes.png}

\link{http://en.wikiquote.org/wiki/John_Gilmore}\\
\link{http://en.wikipedia.org/wiki/John_Gilmore_(activist)}

\slide{Directly connection Tor Users from Turkey}

\hlkimage{20cm}{userstats-relay-country-2013-12-25-off-2014-03-25-tr.png}

Image from \link{https://metrics.torproject.org}\\
via \link{https://twitter.com/runasand}

\slide{Directly connection Tor Users from Turkey +10.000}

\hlkimage{20cm}{userstats-relay-country-2013-12-25-off-2014-03-26-tr.png}

Image from \link{https://metrics.torproject.org} via
\link{https://twitter.com/ioc32/status/448791582423408640}

\slide{Tor project install}

\hlkimage{12cm}{tor-project.png}

Der findes diverse tools til Tor, Torbutton on/off knap til Firefox osv.

Det anbefales at bruge Torbrowser bundles fra \link{https://www.torproject.org/}

\slide{Torbrowser - anonym browser}

\hlkimage{20cm}{torbrowser-main-window.png}

\centerline{\color{titlecolor} Mere anonym browser - Firefox in disguise}

\slide{Whonix - Tor to the max!}

\hlkimage{17cm}{400px-Whonix.jpg}

\begin{quote}
Whonix is an operating system focused on anonymity, privacy and security. It's based on the Tor anonymity network[5], Debian GNU/Linux[6] and security by isolation. DNS leaks are impossible, and not even malware with root privileges can find out the user's real IP. \link{https://www.whonix.org/}

\end{quote}

\centerline{Torbrowser er godt, Whonix giver lidt ekstra sikkerhed}


\slide{Multiple browsers}

\hlkimage{20cm}{multi-browser-strategy.png}

\begin{list2}
\item Strict Security settings in the general browser, Firefox or Chrome?
\item More lax security settings for "trusted sites" - like home banking
\item Security plugins like HTTPS Everywhere and NoScripts for generic browsing
\end{list2}

\slide{HTTPS Everywhere}

\hlkimage{5cm}{HTTPS_Everywhere_new_logo.jpg}
\begin{quote}
HTTPS Everywhere is a Firefox extension produced as a collaboration between The Tor Project and the Electronic Frontier Foundation. It encrypts your communications with a number of major websites.
\end{quote}

\centerline{\link{http://www.eff.org/https-everywhere}}



\slide{Secure your mobile}

\hlkimage{20cm}{the-guardian-project.pdf}

\centerline{Dont forget your mobile platforms \link{https://guardianproject.info/}}



\slide{www.uncensoreddns.org}

\hlkimage{20cm}{censurfridns-1.png}

\slide{DNSSEC trigger}

\hlkimage{7cm}{dnssec-trigger.png}

Lots of DNSSEC tools, I recommend DNSSEC-trigger a local name server for your laptop

\begin{list2}
\item DNSSEC Validator for firefox\\ \link{https://addons.mozilla.org/en-us/firefox/addon/dnssec-validator/}
\item OARC tools \link{https://www.dns-oarc.net/oarc/services/odvr}
\item \link{http://www.nlnetlabs.nl/projects/dnssec-trigger/}
\end{list2}


\slide{F�lg med Twitter news}

%\hlkimage{18cm}{twitter-security-feed.png}
\hlkimage{10cm}{twitter-safety.png}

\begin{list1}
\item Twitter has become an important new resource for lots of stuff
\item Twitter has replaced RSS for me
\end{list1}


\slide{Be careful - sp�rgsm�l?}

\hlkimage{5cm}{michael-conrad.jpg}
\centerline{\Large Hey, Lets be careful out there!}
\vskip 2 cm

\begin{center}
\myname

%\myweb
\end{center}

\vskip 2cm
Billede: Michael Conrad \link{http://www.hillstreetblues.tv/}

\slide{VikingScan.org - free portscanning}

\hlkimage{18cm}{vikingscan.png}
%\vskip 1cm
%\centerline{\link{http://www.vikingscan.org}}


\end{document}
\input{references.tex}
