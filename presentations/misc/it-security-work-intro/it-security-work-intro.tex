\documentclass[Screen16to9,17pt]{foils}
%\documentclass[16pt,landscape,a4paper,footrule]{foils}
\usepackage{zencurity-slides}

% Du kommer forbi d. 16/3 kl. 11 GBG.E342 og snakker:

%   Det moderne IT-sikkerhedslandskab
% * Dit firma – hvad lavede / laver i?
% * Pen-testing
% * IT-sikkerhed
% * Fremtidsperspektiver for branchen (både software og IT-sikkerhed)
% Ca. 30 minutter.


\begin{document}
%\rm
\selectlanguage{english}

\mytitlepage
{Det moderne IT-sikkerhedslandskab}{KEA}

\hlkprofil

\slide{Hvad laver Zencurity Aps -- Security engineering a job role}

\begin{alltt}\small
On any given day, you may be challenged to:
        Create new ways to solve existing production security issues
        Configure and install firewalls and intrusion detection systems
        Perform vulnerability testing, risk analyses and security assessments
        Develop automation scripts to handle and track incidents
        Investigate intrusion incidents, conduct forensic investigations and incident responses
        Collaborate with colleagues on authentication, authorization and encryption solutions
        Evaluate new technologies and processes that enhance security capabilities
        Test security solutions using industry standard analysis criteria
        Deliver technical reports and formal papers on test findings
        Respond to information security issues during each stage of a project’s lifecycle
        Supervise changes in software, hardware, facilities, telecommunications and user needs
        Define, implement and maintain corporate security policies
        Analyze and advise on new security technologies and program conformance
        Recommend modifications in legal, technical and regulatory areas that affect IT security
\end{alltt}

Source: \url{https://www.cyberdegrees.org/jobs/security-engineer/}\\
also
\url{https://en.wikipedia.org/wiki/Security_engineering}



\slide{Core Concepts}

Information Security is a huge domain:

\begin{quote}
The (ISC)² CBK is a collection of topics relevant to cybersecurity professionals around the world. It establishes a common framework of information security terms and principles which enables cybersecurity and IT/ICT professionals worldwide to discuss, debate and resolve matters pertaining to the profession with a common understanding, taxonomy and lexicon.
\end{quote}
Source: \link{https://www.isc2.org/Certifications/CBK}

List of 8 domains in CISSP CBK: Security and Risk Management, Asset Security,
Security Architecture and Engineering, Communications and Network Security, Identity and Access Management, Security Assessment and Testing, Security Operations, Software Development Security

- then add all the news about new tools, exploits, and networking



\slide{Hacker tools}

\begin{list1}
\item \emph{Improving the Security of Your Site by Breaking Into it}\\ by
Dan Farmer and Wietse Venema in 1993
\item Later in 1995 released the software SATAN\\
\emph{Security Administrator Tool for Analyzing Networks}
\item Caused some commotion, panic and discussions, every script kiddie can hack, the internet will melt down!
\vskip 5mm
\begin{quote}
We realize that SATAN is a two-edged sword -- like
many tools, it can be used for good and for evil
purposes. We also realize that intruders (including
wannabees) have much more capable (read intrusive)
tools than offered with SATAN.
\end{quote}
\end{list1}

\vskip 1cm
Source:
\link{http://www.fish2.com/security/admin-guide-to-cracking.html}


\slide{Learning at different levels}

\hlkimage{10cm}{trinity-nmapscreen-hd-cropscale-418x250.jpg}

To illustrate this, I will use the example of:\\
Nmap - a very famous port scanner.

Unfortunately there are about 100 options, and the man page is some 3100 lines ...

\slide{Prerequisite knowledge}

Plan: You want to learn Nmap!

{\large Often we combine our knowledge with skills into competence, \\
which enable us to perform some job, task or function.}

\begin{list2}
\item Knowledge level: What is a port scanner\\
Need to know TCP/IP, IP address, ports and services -- example HTTP 80/tcp, TCP session setup
\end{list2}

So get this sorted out first, otherwise you cannot understand what Nmap does, and output returned

\slide{Skills are needed}

\hlkimage{9cm}{nmap-zenmap.png}


\begin{list2}
\item Skills level: Running a port scanner\\
Need to have operating system -- luckily Nmap supports Mac, Windows, Linux, ...
\item My recommendation: create a virtual machine with Kali Linux
\end{list2}

\slide{Combined it becomes a Competence}

\begin{alltt}\footnotesize
full-tcp-scan: nmap -p 1-65535 -A -oA full-tcp-scan -iL targets
dns-scan: nmap -sU -p 53 --script=dns-recursion -iL targets -oA dns-recursive
bgpscan: nmap -A -p 179 -oA bgpscan -iL \$LINKNET
dns-recursive: nmap -sU -p 53 --script=dns-recursion -iL targets -oA dns-recursive
php-scan: nmap -sV --script=http-php-version -p80,443 -oA php-scan -iL targets
scan-vtep-tcp: nmap -A -p 1-65535 -oA scan-vtep-tcp 192.0.2.77 192.0.2.78
snmp-10.x.y.0.gnmap: nmap -sV -A -p 161 -sU --script=snmp-info -oA snmp-10xy 10.x.y.0/19
snmpscan: nmap -sU -p 161 -oA snmpscan --script=snmp-interfaces -iL targets
sshscan: nmap -A -p 22 -oA sshscan -iL targets
vncscan: nmap -A -p 5900-5905 -oA vncscan -iL targets
\end{alltt}

\begin{list2}
\item Competence level: Running a quality port scan of an enterprise\\
Need to have plan for scanning, know which scan functions to use
\end{list2}

My recommendation: work through a 4 hour course with Nmap as the subject


\slide{OSI Model and Internet Protocols}

\hlkimage{10cm,angle=90}{images/compare-osi-ip.pdf}

\slide{Recommended technologies to learn}


So to accomplish the goal of using Nmap efficiently you need some basics

Networking: Basic Protocols from the Internet Protocols suite IP/TCP, or TCP/IP
\begin{list2}
\item Network Layer: Ethernet, Address Resolution Protocol (ARP), IPv4 and ICMP\\
Later add Wi-Fi and IPv6
\item Transport Layer: Transmission Control Protocol (TCP) and User Datagram Protocol (UDP)
\item Common upper layer: Dynamic Host Configuration Protocol (DHCP), Domain Name System (DNS),
Hypertext Transfer Protocol (HTTP)\\
Later add the encrypted/secure versions like Hypertext Transfer Protocol Secure (HTTPS) which uses Transport Layer Security (TLS)
\end{list2}

Pro tip: always say Ethernet frames and IP packets. No one uses datagram anymore.

Pro tip: If you \emph{really know DNS} you can make a huge impact in the malware area!


\slide{Recommended tools to learn}

\begin{list2}
\item Open Source I really love open source. There is just too much great open source software, to ignore, and security budgets are tight in DK!
\item Linux/Unix knowledge is necessary
-- because a lot of the newest tools are written for Linux/Unix/BSD
\item Git and Github -- where you can find lots of tools, libraries, applications
\item Programming experience is an advantage for automating stuff -- Python is a nice generic tool for this
\item Ansible provisioning -- installing and configuring software for production
\item Elasticsearch -- how to run a \emph{service}, full fledged applications exist for Elasticsearch
\end{list2}


\slide{Nice Rack you got there!}

\hlkimage{8cm}{rackskab.jpg}

\slide{Physical Inspection is Needed}

Yes, go through the server room!

Things we find:
\begin{list2}
\item Single firewall, running with a single power supply -- single point of failure
\item No Uninterruptable Power Supply -- having NO UPS is bad if availability is important
\item Bad cabling, disaster can strike, and no one can help you
\item Bad cooling can take down your whole company
\end{list2}

Advice: Start documenting your setup -- buy a label maker today

\slide{Core Switch Administration}

Then I also found switch administration with admin/admin *sigh*

\hlkimage{10cm}{edgemax-admin-admin.png}

\begin{list2}
\item Is this the main switch for the whole office?! Yes - unfortunately
\item I was also called up one time about a large core switch that had lost configuration, nothing worked
\end{list2}


\slide{Cisco ACI (2019)}

%\hlkimage{}{}

\begin{quote}{\bf
Vulnerability Analysis}
\begin{list2}
\item Remote Code Execution on Leaf Switches over IPv6 via Local SSH Server (CVE-2019-1836, CVE2019-1803, and CVE-2019-1804) -- SSH access with specific source port, private key left on firmware image, and on all switches
\item Cisco Nexus 9000 Series Fabric Switches ACI Mode Fabric Infrastructure VLAN Unauthorized Access
Vulnerability (CVE-2019-1890)
\item Cisco Nexus 9000 Series Fabric Switches Application Centric Infrastructure Mode Link Layer Discovery
Protocol Buffer Overflow Vulnerability (CVE-2019-1901)
\item Cisco Application Policy Infrastructure Controller REST API Privilege Escalation Vulnerability (CVE2019-1889)
\end{list2}
\end{quote}
Source: \url{https://static.ernw.de/whitepaper/ERNW_Whitepaper68_Vulnerability_Assessment_Cisco_ACI_signed.pdf}

Further all processes run as root user -- good job Cisco


\slide{HP Switches are very user friendly}

%\hlkimage{}{}

\begin{alltt}\scriptsize
$ telnet 172.16.1.21
Connected to 172.16.1.21.
HP J9772A 2530-48G-PoEP Switch
Software revision YA.16.08.0015
(C) Copyright 2020 Hewlett Packard Enterprise Development LP
RESTRICTED RIGHTS LEGEND
...
Press any key to continue
Your previous successful login (as manager) was on 1990-02-19 21:27:21
from 172.16.1.250
SW04Stuen# show configuration
Running configuration:
; J9772A Configuration Editor; Created on release #YA.16.08.0015
; Ver #14:01.44.00.04.19.02.13.98.82.34.61.18.28.f3.84.9c.63.ff.37.27:45
\end{alltt}

\begin{list2}
\item Nothing to see here -- just log me in without a password, thank you HP\\
No NTP servers, no log servers, default credentials, using bad default SNMP public, configured with VLANs
\end{list2}

\slide{In 2022 Don't keep your Exchange server on the LAN!}

\hlkimage{13cm}{exchange-lan.pdf}

Another service which is being attacked in recent years is Microsoft Exchange\\
This customer had their Exchange server directly on the LAN?!

\begin{list2}
\item There is a high risk that a single vulnerability in Microsoft Exchange would \\
open this network to complete compromise
\end{list2}


\slide{Microsoft Exchange vulnerabilities with CVSS 7 or higher}

\hlkimage{16cm}{images/exchange-2021-cvss-7.png}
\hlkimage{16cm}{images/exchange-2022-cvss-7.png}

\begin{list2}
\item Lesson: Don't put Exchange on your LAN!
\end{list2}

\slide{Open Source -- Linux hackerlab}

\hlkimage{6cm}{hacklab-1.png}

\begin{list2}
\item Create your own playground, a hackerlab
\item kramse-labs -- Guide to preparing your laptop for training with Kramse\\
\link{https://github.com/kramse/kramse-labs}
\item Recommend two VMs, Debian and Kali Linux
\item Don't forget to find the Debian Handbook and Kali Linux Revealed, free PDFs
\end{list2}

% {\bf Start a download of Kali now, if you want to play with the tools.}\\
% Recommend virtual machine download 64-bit\\
%  \url{https://www.kali.org/get-kali/#kali-virtual-machines}

{\bf I consider Linux/Unix knowledge a must for working in Networking and Security}


\slide{Tools: Open Source and Python}
\hlkimage{7cm}{maltrail.png}

\begin{list2}
\item Open Source is already written *doh*
\item Can provide solutions or parts of a solution
\item Often feature-rich, mature, tested, maintained, and even when \emph{not} can be brought back to life
\item Picture from Maltrail \link{https://github.com/stamparm/maltrail}\\
Maltrail is a malicious traffic detection system, utilizing publicly available lists containing malicious and/or generally suspicious trails, along with static trails compiled from various AV reports and custom user defined lists,
\end{list2}




\slide{Why Ansible}

%\hlkimage{}{}

Platform options Ansible:
\begin{alltt}
CloudEngine OS, CNOS, Dell OS6, Dell OS9 Dell OS10, ENOS, EOS, ERIC_ECCLI, EXOS,
FRR, ICX, IOS, IOS-XR, IronWare, Junos OS, Meraki, Pluribus NETVISOR, NOS, NXOS,
RouterOS, SLX-OS, VOSS, VyOS, WeOS 4

plus routers based on Linux, OpenBSD, FreeBSD etc.
\end{alltt}


One management system with many uses, free to download and use
\begin{list2}
\item Generic configuration management -- and you end up running support systems, network near systems
\item Ansible for Network Automation\\
\link{https://docs.ansible.com/ansible/latest/network/index.html}
\item Allows you to install, configure and run your network management systems -- like LibreNMS, Nipap
\end{list2}

\slide{Python and YAML}

\hlkimage{7cm}{git-logo.png}

\begin{list2}
\item We need to store configurations of devices and systems
\item Run Ansible playbooks
\item Problem: Remember what we did, when, how
\item Solution: use git for the playbooks
\item Not the only version control system, but my preferred one
\item Git can also be used by Oxidized which I also love \link{https://github.com/ytti/oxidized}
\end{list2}



\slide{Why Elasticsearch}

%\hlkimage{}{}

\begin{quote}
The Elastic Common Schema (ECS) is an open source specification, developed with support from the Elastic user community. ECS defines a common set of fields to be used when storing event data in Elasticsearch, such as logs and metrics.
\end{quote}

One storage system with many uses, free to download and use
\begin{list2}
\item Logstash - can take logs and SNMP traps easily
\item Packetbeat \link{https://www.elastic.co/beats/packetbeat}
\item Elastiflow
\link{https://github.com/robcowart/elastiflow}
\item Has defined an Elastic Common Scheme (ECS)\\
\link{https://www.elastic.co/guide/en/ecs/current/ecs-reference.html}
\end{list2}




\slide{The Future in IT-Security, software development -- IT in general}

%\hlkimage{}{}

\begin{quote}{\bf
It-anvendelse i virksomheder (tema) 2022 It-sikkerhed i mikrovirksomheder}

Dokumentation og dermed delbar viden om it-sikkerhedstiltag og -regler er et væsentlig element i virksomhedernes arbejde med digital sikkerhed. {\bf Over halvdelen af Danmarks godt 14.000 mikrovirksomheder (5-9 ansatte) havde i 2022 ingen dokumentation} om forholdsregler, aktiviteter og procedurer vedr. it-sikkerhed. Der var således 53 pct. af virksomhederne med 5-9 ansatte, der ikke havde dokumenteret deres it-sikkerhedstiltag, mv. Til sammenligning var den tilsvarende andel 45 pct. blandt virksomheder med minimum 10 ansatte og blot 8 pct. blandt de største virksomheder med minimum 250 ansatte.
\end{quote}
Source: \url{https://www.dst.dk/da/Statistik/nyheder-analyser-publ/nyt/NytHtml?cid=50382}

\begin{list2}
\item We need more resources, a lot more!
\end{list2}


\slide{Hackers don't give a shit}

Your system is only for testing, development, ...

Your network is a research network, under construction, \\
being phased out, ...

Try something new, go to your management

Bring all the exceptions, all of them, update the risk \\
analysis figures - if this happens it is about 1mill DKK

Ask for permission to go full monty on your security

{\bf Think like attackers - don't hold back}


\hlkimage{10cm}{kiwicon-2009-hackers-dont-give-shit.jpg}


\myquestionspage



\slide{Books: Communications and Network Security course}

Primary literature
\begin{list2}
\item \emph{Applied Network Security Monitoring Collection, Detection, and Analysis}, 2014 Chris Sanders \\
ISBN: 9780124172081 - shortened ANSM
\item \emph{Practical Packet Analysis - Using Wireshark to Solve Real-World Network Problems}, 3rd edition 2017, \\
Chris Sanders ISBN: 9781593278021 - shortened PPA
\item \emph{Linux Basics for Hackers Getting Started with Networking, Scripting, and Security in Kali}. OccupyTheWeb, December 2018, 248 pp. ISBN-13: 978-1-59327-855-7 - shortened LBfH
\item The \emph{Lecture Plan}\\
\link{https://zencurity.gitbook.io/kea-it-sikkerhed/net-og-komm-sikkerhed/lektionsplan}
\item Presentations -- slides for each lecture, 14 evenings in total for this course\\{\footnotesize
\link{https://github.com/kramse/security-courses/tree/master/courses/networking/communication-and-network-security}}
\end{list2}

Price check -- all three books can be bought in hardcopy for approx 1.000-1.100DKK


\slide{Other books I use in courses - some are free}

\begin{list2}
\item \emph{The Debian Administrator’s Handbook}, Raphaël Hertzog and Roland Mas\\
\url{https://debian-handbook.info/}
\item \emph{Kali Linux Revealed  Mastering the Penetration Testing Distribution}\\
Raphaël Hertzog, Jim O'Gorman\\
\link{https://www.kali.org/download-kali-linux-revealed-book/}

\item \emph{Gray Hat Hacking: The Ethical Hacker's Handbook}, 5. ed. Allen Harper and others ISBN: 978-1-260-10841-5
\item \emph{Web Application Security}, Andrew Hoffman, 2020, ISBN: 9781492053118 - download for free through Nginx:\\
\link{https://www.nginx.com/resources/library/web-application-security/}

\item \emph{Hacking, 2nd Edition: The Art of Exploitation}, Jon Erickson, February 2008, ISBN-13: 9781593271442
\item All my training and educational materials are open source, including exercises booklets with small exercises that you can do with virtual machines like Debian and Kali Linux using lots of open source tools.\\
{\bf \link{https://github.com/kramse/security-courses}}
\end{list2}




\slide{Equipment -- wanna work with networks}

Laptops, one is enough to get started

.
\hlkrightpic{85mm}{-2cm}{sample-network.png}

\begin{list1}
\item I have a network with me when needed, \\
which has the following systems:
\begin{list2}
\item OpenBSD router
\item Switches Juniper EX2200-C and small TP-Link
\item UniFi AP wireless access-point
\end{list2}
\end{list1}

Above or similar can often be found lying around in offices, ask if you can take it.


\slide{Wifi Hardware}

I recommend getting an extra wireless network card for your laptop.

A wireless USB network card with external antenna can be used for many purposes.

\begin{list2}
\item The following are two recommended models:
\item TP-link TL-WN722N hardware version 2.0 cheap but only support 2.4GHz
\item Alfa AWUS036ACH 2.4GHz + 5GHz Dual-Band and high performing
\item Both usually work great in Kali Linux
\item Newer, better, cheaper may exist -- YMMV
\end{list2}

I have some available for people to try if you dont want to buy them.

And if you have the money, USB Ethernet for playing with raw frames in your VM\\
I use 200DKK StarTech USB Ethernet -- works for me


\slide{Hacker lab setup -- tips}

\hlkimage{8cm}{hacklab-1.png}

\begin{list2}
\item Hardware: any modern laptop with CPU and virtualisation\\
Don't forget to enable it in the BIOS
\item Software: your favourite operating system Windows, Mac, Linux, ...
\item Virtualisation software: VMware, Virtual box, pick your poison
\item Hacker software: Kali as a Virtual Machine \link{https://www.kali.org/}
\item Soft targets: Metasploitable, Linux, Microsoft Windows, Microsoft Exchange, Windows server, ...
\end{list2}







\end{document}
