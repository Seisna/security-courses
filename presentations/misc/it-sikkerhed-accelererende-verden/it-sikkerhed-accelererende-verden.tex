\documentclass[Screen16to9,17pt]{foils}
\usepackage{zencurity-slides}

% PROSA i odense 5. sept 2024
% Basal it-sikkerhed i en altid accelererende verden
% Kan du følge med!? Går du og bliver utryg over IT-sikkerheden, fordi du ved den halter!?

% I dette foredrag vil vi tage et vue udover IT-sikkerhed startende med basale modeller som Confidentiality Integrity og Availability (CIA) over ISC2 Common Body og Knowledge til moderne frameworks som Mitre ATT&CK og CIS Controls.

% Målet er at bringe nogle modeller i spil som kan hjælpe med at strukturere arbejdet med IT-sikkerhed, så du kan få svar på hvad du vil beskytte, hvad skal du beskytte det imod - på overordnet plan.

% Dernæst vil Henrik med min erfaring udpege de vigtigste punkter som for ham vil give mere langsigtet beskyttelse mod truslerne i virksomheder og organisationer.

% Det er emner som netværksisolering, backup og defense in depth.

% Placering: Foredraget foregår i Prosa's egne lokaler i Odense, hvor der vil blive serveret en sandwich og en øl/vand i løbet af aftenen.

% Målgruppe: Alle der er interesserede i IT-sikkerhed i moderne organisationer.

% Nøgleord: It-sikkerhed, frameworks, SMV sikkerhed, ssudvikling

% Underviser: Henrik Kramselund er it-sikkerhedsekspert og ejer af it-sikker­heds­firmaet Zencurity. Han underviser desuden på KEAs diplomuddannelse i it-sikkerhed og har gennem årene arrangeret en lang række kurser i PROSA.

% Medbring gerne noteredskaber, blok eller laptop.


\begin{document}
\selectlanguage{danish}
\mytitlepage{Basal it-sikkerhed i en altid accelererende verden}{\the\year}


\vskip 1cm
\centerline{\footnotesize slides are also available on Github}


\slide{Plan for today}

\begin{list1}
\item Talk about the current world of information security
\item What a crazy place we are in with a flood of vulnerabilities
\item Resource shortage -- man power, skillz etc.
\end{list1}

\slide{Goals: }

\hlkimage{6cm}{thomas-galler-hZ3uF1-z2Qc-unsplash.jpg}

\begin{list1}
\item Point to a way out of the forever \emph{patch everything now}-cycle
\item Give a high level view -- don't try to remember everything
\item An inspiration to do the basics, before buying into the latest craze\\ General AI and large language models (LLM) won't solve most problems, machine learning does solve some
\item We have solutions -- some have been around since the 70s and 80s
\end{list1}


\slide{Every Year}

\hlkimage{4cm}{happy-new-year-roven-images-601197-unsplash.jpg}


\begin{list2}
\item Same problems last year? Same problems EVERY year
\item Last year was a nightmare of break-ins and data leaks
\item Data leaks, GDPR, ransomware, ...
\end{list2}

\vskip 1cm
{\LARGE\bf Try not to panic, but there are lots of threats}

Are we loosing the battle?

\slide{Har du haft snakken med din CISO?}

%\hlkimage{}{}

\begin{quote}\large
{\bf IT-sikkerhed:}\\
Vi vil gerne bede om 10 millioner til IT-sikkerhed i budget næste år

{\bf CTO/CIO/CISO:}\\
Umuligt

{\bf IT-sikkerhed:}\\
OK, fair nok. Så skal vi bare bede om {\bf 100 millioner til Ransomware}, tak.

Husk også at uddanne CFO i bitcoin transaktioner.
\end{quote}

\begin{list2}
\item Er ovenstående urealistisk?
\end{list2}

\slide{Demant 2019}

%\hlkimage{}{}

\begin{quote}
- For året 2019 rapporterede vi et tab i omsætning på {\bf 575 millioner kroner}. Det i sig selv er alvorligt. Hvad angår vores opmærksomhed på it-sikkerhedsområdet, har it-hændelsen været med til at understrege nødvendigheden af, at tage dette felt seriøst. Angreb mod it-infrastruktur er uden tvivl en af de største trusler mod en virksomhed, og det kan gå galt, hvis man ikke er i stand til at lukke ned for skaden og bruge sin back-up.

...

- På det konkrete plan har vi fået et mantra der lyder {\bf ’Active Directory is king, and backup is Queen’}. Men mere overordnet har vi også lært at Fokus skal helt op på øverste niveau i virksomheden, at man skal skaffe høj faglig indsigt i sikkerhed og trusler, og at det er et arbejde, der skal være under konstant observation og udvikling.

\end{quote}
Kilde: \link{https://dit.dk/Nyheder/2021/Demant}


\begin{list2}
\item Vi taler altså om tab i størrelsesorden tre-cifrede millionbeløb!
\end{list2}

\slide{Overlapping Security Incidents}

\hlkrightpic{9cm}{1cm}{datalaek-2019.png}

New data breaches nearly every week, these from danish news site \link{version2.dk}

Problem, we need to receive data from others

Data from others may contain malware

Have a job posting, yes\\
- then HR will be expecting CVs sent as .doc files


\slide{Paranoia defined}

\hlkimage{8cm}{paranoia-definition.png}
Source: google paranoia definition

Use appropriate paranoia, and yes, hot pink red blinking is an appropriate threat level


\slide{Hackers don't give a shit}

\hlkrightpic{11cm}{-3cm}{kiwicon-2009-hackers-dont-give-shit.jpg}

Your system is only for testing, development, ...

Your network is a research network, under construction, \\
being phased out, ...

Try something new, go to your management

Bring all the exceptions, all of them, update the risk \\
analysis figures - if this happens it is about 1mill DKK

{\bf Think like attackers - don't hold back}



\slide{Work together}

\hlkimage{9cm}{Shaking-hands_web.jpg}

\begin{list1}
\item Team up!
\item We need to share security information freely
\item We often face the same threats, so we can work on solving these together
\end{list1}

\slide{Goals of Security}

\begin{list1}
\item Prevention - means that an attack will fail
\item Detection - determine if attack is underway, or has occured - report it
\item Recovery - stop attack, assess damage, repair damage
\end{list1}

Policy and Mechanism

\begin{quote}
{\bf Definition 1-1.} A \emph{security policy} is a statement of what is, and what is not, allowed.

{\bf Definition 1-2.} A \emph{security mechanism} is a method, tool or procedure for enforcing a security policy.
\end{quote}

Quote from Matt Bishop, Computer Security section 1.3


\slide{Balanced security}

\hlkimage{21cm}{afbalanceret-sikkerhed.pdf}

\begin{list1}
\item Better to have the same level of security
\item If you have bad security in some part - guess where attackers will end up
\item Hackers are not required to take the hardest path into the network
\item Realize there is no such thing as 100\% security
\end{list1}



\slide{Confidentiality Integrity Availability}

\hlkimage{8cm}{cia-triad-uk.pdf}

\begin{list1}
\item We want to protect something
\item Confidentiality - data kept secret
\item Integrity - no unauthorized changes to data
\item Availability - data and systems are available to authorized uses when they need them
\end{list1}



\slide{Trusted Computing Base}

{\bf Definition 20-6.} A \emph{trusted computing base} (TCB) consists of all protection mechanisms within a computer system -- including hardware, firmware, and software -- that are responsible for enforcing a security policy

Quote from Matt Bishop, Computer Security

Keeping this small, simple and understandable help keeping systems more secure.

Example the Qubes OS depend on few security-critical components:\\
\link{https://www.qubes-os.org/doc/security-critical-code/}



\slide{Intrusion Kill Chains}

\hlkimage{13cm}{crafting-cip-kill-chain.png}

\begin{list2}
\item See also \emph{Intelligence-Driven Computer Network Defense Informed by Analysis of Adversary Campaigns and Intrusion Kill Chains}, Eric M. Hutchins , Michael J. Cloppert, Rohan M. Amin, Ph.D. Lockheed Martin Corporation, 2011\\{\footnotesize
 \link{https://www.lockheedmartin.com/content/dam/lockheed-martin/rms/documents/cyber/LM-White-Paper-Intel-Driven-Defense.pdf}}
\end{list2}


\slide{Vulnerabilities - CVE}

\begin{list1}
\item Common Vulnerabilities and Exposures (CVE):
  \begin{list2}
  \item classification
  \item identification
  \end{list2}
\item When discovered each vuln gets a CVE ID
\item CVE maintained by MITRE - not-for-profit
org for research and development in the USA.
\item National Vulnerability Database search for CVE.
\item Sources: \link{http://cve.mitre.org/} og \link{http://nvd.nist.gov}
\item also checkout OWASP Top-10 \link{http://www.owasp.org/}
\end{list1}

\slide{Local vs. remote exploits}

\begin{list1}
\item {\bfseries Local vs. remote}
angiver om et exploit er rettet mod
en sårbarhed lokalt på maskinen, eksempelvis
opnå højere privilegier, eller beregnet
til at udnytter sårbarheder over netværk
\item {\bfseries Remote root exploit}
- den type man frygter mest, idet
det er et exploit program der når det afvikles giver
angriberen fuld kontrol, root user er administrator
på Unix, over netværket.
\item {\bfseries Zero-day exploits} dem som ikke offentliggøres -- dem
  som hackere holder for sig selv. Dag 0 henviser til at ingen kender
  til dem før de offentliggøres og ofte er der umiddelbart ingen
  rettelser til de sårbarheder
\end{list1}

\slide{Separation of duty}

\begin{quote}
{\bf Separation of duties} (SoD; also known as Segregation of Duties) is the concept of having more than one person required to complete a task. In business the separation by sharing of more than one individual in one single task is an internal control intended to prevent fraud and error.
\end{quote}

Quote from \url{https://en.wikipedia.org/wiki/Separation_of_duties}

\begin{quote}
{\bf Separation of function}. Developers do not develop new programs on production systems because of the potential threat to production data.
\end{quote}
\emph{Computer Security}, Matt Bishop, 2019

Danish: Funktionsadskillelse

\slide{Lipners Integrity Matrix Model}

\hlkimage{20cm}{lipner-with-integrity.png}

\emph{Non-Discretionary Controls for Commercial Applications}, Steven B. Lipner, IEEE Symposium on Security and Privacy, and Fifth Seminar on the DoD Computer Security Initiative, 1982

\slide{Mitre ATT\&CK Framework }

\hlkimage{9cm}{mitre-attack.png}

\begin{list2}
\item Source: \url{https://attack.mitre.org/} Great resource for attack categorization
\item examples of attack methods used by real actors
\item Hint browse the ATT\&CK 101 Blog Post\\
\url{https://medium.com/mitre-attack/att-ck-101-17074d3bc62}
\end{list2}


\slide{The dangers of logging in as the root user}

%\hlkimage{}{}

\begin{quote}
A huge advantage that Unix and Linux operating systems have over Windows is that Unix and Linux do
a much better job of keeping privileged administrative accounts separated from normal user accounts.
Indeed, one reason that older versions of Windows were so susceptible to security issues, such as
drive-by virus infections, was the common practice of setting up user accounts with administrative
privileges, without having the protection of the User Access Control (UAC) that’s in newer versions of
Windows.
\end{quote}
Source: \emph{Mastering Linux Security and Hardening} (MLSH), third edition

\begin{list2}
\item Agreed, but I may be biased
\item Mac OS X made it very simple to run administrative tasks, so you didn't need to run as root
\item Modern Linux user interfaces make similar attempts with \verb+pkexec+, \verb+kdesudo+,  \verb+gksudo+ etc.
\item Windows is getting better, many organisations in DK are removing administrative access to regular users, even KEA
\end{list2}

\slide{The advantages of using sudo}

%\hlkimage{}{}

\begin{quote}
Used properly, the sudo utility can greatly enhance the security of your systems, and it can make an
administrator’s job much easier. With sudo , you can do the following:
\begin{list2}
\item Assign certain users full administrative privileges, while assigning other users only the privi-
leges they need to perform tasks that are directly related to their respective jobs.
\item Allow users to perform administrative tasks by entering their own normal user passwords so
that you don’t have to distribute the root password to everybody and their brother.
\item Make it harder for intruders to break into your systems. If you implement sudo and disable
the root user account, would-be intruders won’t know which account to attack because they
won’t know which one has admin privileges.
\item Create sudo policies that you can deploy across an entire enterprise network, even if that
network has a mix of Unix, BSD, and Linux machines.
\item Improve your auditing capabilities because you’ll be able to see what users are doing with
their admin privileges.
\end{list2}
\end{quote}
Source: \emph{Mastering Linux Security and Hardening} (MLSH), third edition

\slide{Why use Sudo conclusion}

Main thing about sudo is that you do NOT give out the root password to anybody! They will use their own credentials and can be limited to single commands, scripts and even parameters. You could have a single sudoers file for your own organisation, that includes groups of servers, user groups etc.

Sidenote: sudo also has a number of CVEs unfortunately


\slide{Principle of Least Privilege}

\begin{list1}
\item {\bf Definition 14-1} The \emph{principle of least privilege} states that a subject should be given only those privileges that it needs in order to complete the task.
\item Also drop privileges when not needed anymore, relinquish rights immediately
\item Example, need to read a document - but not write.
\item Database systems can often provide very fine grained access to data
\end{list1}



\slide{Fokus on the basics}

\begin{list2}
\item User management - including administrative users
\item Asset management
\item Laptop security
\item Penetration testing
\item Firewalls and segmentation
\item VPN everywhere
\item TLS and VPN settings, encryption
\item DNS and email security
\item Syslog and monitorering
\item Incident Response and response
\end{list2}

\vskip 5mm
\centerline{Vi skal allesammen hjælpe hinanden! Ovenstående er listen jeg giver studerende og virksomheder}



\slide{Fokus: User management}

\hlkimage{8cm}{humans2.png}

\begin{list2}
\item Relevant for alle organisationer
\item Er måden vi sikrer godkendte brugere kan udføre opgaver
\item Kodeord bruges til at forhindre uautoriseret adgang
\item Har I styr på brugerid?
\item Hvor er brugere oprettet?
\item Hvor hurtigt kan I fjerne "een bruger" eller "deaktivere en bruger" alle steder!
\item Er det et kludetæppe - ja, mange steder er det
\end{list2}


\slide{Local administrator?}

\hlkimage{10cm}{dragon-drawing-6.jpg}

\begin{list2}
\item Findes der systemer som er helt åbne, med lokal administrator
\item Er det stadig nødvendigt
\item Vi bør bruge Principle of Least privilege
\item Vi ved hvordan, for det fortalte Jerome Saltzer og Michael Schroeder i deres 1975 artikel\\ \emph{The Protection of Information in Computer Systems}\\
\url{https://en.wikipedia.org/wiki/Saltzer_and_Schroeder%27s_design_principles}
\end{list2}


\slide{Passwords vælges ikke tilfældigt}

\hlkimage{20cm}{50-most-used-passwords.png}

Source:
\link{https://wpengine.com/unmasked/}


\slide{Your data has already have been owned by criminals}

\hlkimage{9cm}{pwned.png}

\begin{list1}
\item Your data is already being sold, and resold on the Internet
\item Stop reusing passwords, use a password safe to generate and remember
\item Check you own email addresses on \link{https://haveibeenpwned.com/}
\item They have an API you can integrate to avoid re-using already leaked passwords\\
{\footnotesize\link{https://www.troyhunt.com/introducing-306-million-freely-downloadable-pwned-passwords/}}
\end{list1}


\slide{Opbevaring af passwords}

\hlkimage{6cm}{password-window.png}

\begin{list2}
\item Use password managers! Available as cloud connected, local only, teams based
\item You will have to investigate which one to choose, but find one!
\end{list2}


\slide{Fokus: Laptop sikkerhed}

\hlkimage{13cm}{kelly-sikkema-212376-unsplash.jpg}

\begin{list2}
\item Relevant for alle
\item Hvordan sikrer vi at vi ikke mister værdierne, hardware og data typisk
\end{list2}


\slide{Secure Laptops}

\hlkimage{10cm}{librem-15-v3-turns99.png}

\begin{list2}
\item Laptops (og mobile enheder)
\item Hvad kendetegner en laptop? og en telefon?
\item Hardware naturligvis, en Macbook koster officielt mere end en brugt mellemklassebil
\item - og husk brugen af laptops -- de er dyre, men indholdet er ofte mere værd!
\item Er laptops sikre, og hvad betyder det?
\end{list2}



\slide{Are your data secure - data at rest}

\hlkimage{15cm}{images/data-integrity-1.pdf}

\begin{list1}
\item Stolen laptop, tablet, phone - can anybody read your data?
\item Do you trust "remote wipe"
\item How do you in fact wipe data securely off devices, and SSDs?
\item Encrypt disk and storage devices before using them in the first place!
\end{list1}



\slide{Start Attacking from the Inside}

\hlkimage{6cm}{erik-odiin-568459-unsplash.jpg}


\begin{list2}
\item Now imagine you were in control of a company laptop
\item Do you have a large internal world wide network?\\
Having a large open network may cost you {\bf 1.9 billion DKK - ref Maersk case}
\item Try scanning everything, start in a small corner, expand
\item Scan all you danish segments, one by one, then the nordic, then the world
\item Yes, things may break - FINE, BREAKING is GOOD
\end{list2}

\centerline{\bf Better to break while we are ready to un-break}


\slide{Nmap the world}

\hlkimage{19cm}{trinity-nmapscreen-hd-cropscale-418x250.jpg}


\slide{Hackertools are for everyone!}

{\Large\bf Hackers work all the time trying to break stuff}

Blue teams can use hackertools, and become more effecient:
\begin{list2}
\item Nmap, Nping \link{http://nmap.org}
\item Wireshark - \link{http://www.wireshark.org/}
\item Aircrack-ng \link{http://www.aircrack-ng.org/}
\item Metasploit Framework \link{http://www.metasploit.com/}
\item Burpsuite \link{http://portswigger.net/burp/}
\item Kali Linux \link{http://www.kali.org}
\end{list2}

\vskip 5mm
\centerline{Most popular hacker tools \link{https://tools.kali.org/} and \link{http://sectools.org/}}


\slide{Kali Linux the pentest toolbox}

\hlkimage{14cm}{kali-linux.png}

\begin{list1}
\item  Kali \link{http://www.kali.org/}
\item 100.000s of videos on youtube alone, searching for kali and \$TOOL
\item Also versions for Raspberry Pi, mobile and other small computers
\end{list1}


\slide{Hackerlab setup}

\hlkimage{11cm}{hacklab-1.png}

\begin{list2}
\item Create hacker labs, encourage hacker labs!
\item Software Host OS: Windows, Mac, Linux
\item Virtualisation software: VMware, Virtual box, HyperV pick your poison
\item Hackersoftware: Kali Virtual Machine \link{https://www.kali.org/} ,
\end{list2}

\slide{Hacking is not magic}

\hlkimage{11cm}{ninjas.png}

\begin{list2}
\item Hacking only requires some ninja training
\item We have been doing this since 1995 when SATAN was released
\item Listen, Plan, Act, Do hacking
\item Be curious, and honest -- let our students play with fire in special networks
\end{list2}

\slide{Fokus: Firewalls og segmentering}

\hlkimage{10cm}{virksomhedens-netvaerk.pdf}

\begin{list2}
\item Hvis du har et netværk, så bør du have en firewall
\item En firewall tillader autoriseret trafik og blokerer resten
\item Hvornår har du sidst set din løsning efter?
\item Hvor lang tid tager det at se en 5.000 linier Cisco ASA config igennem?
\end{list2}

\slide{Imagine Attacks from the Inside}

\hlkimage{6cm}{erik-odiin-568459-unsplash.jpg}

\begin{list2}
\item Now imagine you were in control of a company laptop
\item Do you have a large internal world wide network?\\
NotPetya cost Maersk about 1.9 billion DKK
%\item Try scanning everything, start in a small corner, expand
%\item Scan all you danish segments, one by one, then the nordic, then the world
%\item Yes, things may break - FINE, BREAKING is GOOD

\item entry thought to be via software update of M.E.Doc [uk] an Ukrainian tax preparation program
\item Attackers are very creative and have a large attack surface to most companies
\end{list2}



\slide{Together with Firewalls - Virtual LAN (VLAN)}

\hlkimage{8cm}{vlan-portbased.pdf}

\begin{list1}
\item Managed switches often allow splitting into zones called virtual LANs
\item Most simple version is port based
\item Like putting ports 1-4 into one LAN and remaining in another LAN
\item Packets must traverse a router or firewall to cross between VLANs
\end{list1}

\slide{Virtual LAN (VLAN) IEEE 802.1q}

\hlkimage{15cm}{vlan-8021q.pdf}

\begin{list1}
\item Using IEEE 802.1q  VLAN tagging on Ethernet frames
\item Virtual LAN, to pass from one to another, must use a router/firewall
\item Allows separation/segmentation and protects traffic from many security issues
\item Used in most, if not all, Wi-Fi networks -- each SSID has a VLAN behind it
\end{list1}

\slide{Network Access Control -- Connecting clients more securely}

Talking about standard, another useful one:\\
IEEE 802.1x -- Port Based Network Access Control

\hlkimage{7cm}{802.1X_wired_protocols.png}

\begin{list1}
\item Authentication protocol ensures user validation before port access
\item Can authenticate using username and then password or certificate
\item Typically RADIUS and 802.1x which can use LDAP or Active Directory
\item Already used in Wi-Fi networks, so can be turned on for wired Ethernet ports
\end{list1}




\slide{Vulnerabilities are everywhere!}

\hlkimage{18cm}{cve-details-new-updated.png}
Source: CVEdetails.com on 2024-09-02

\begin{list2}
\item This is crazy! \url{https://www.cvedetails.com/}
\end{list2}

\slide{Vulnerabilitiesby type \& year}

\hlkimage{17cm}{cve-details-year.png}
Source: CVEdetails.com on 2024-09-02 Graph on the web site is interactive \url{https://www.cvedetails.com/}

\slide{LG TVs 2024 -- CVE-2023-6317 up to CVE-2023-6320}

\hlkimage{10cm}{LG-shodan.png}

\begin{quote}{\large\bf
90,000+ LG TVs Vulnerable to Authorization Attacks\\
Due to WebOS Vulnerabilities}

Bitdefender Labs has revealed a critical security flaw in over 90,000 LG smart TVs running the company’s proprietary WebOS platform.

If exploited, the vulnerability could allow attackers to gain unauthorized access to the TV’s functions and potentially the user’s home network.

\end{quote}
Source: \url{https://cybersecuritynews.com/lg-tvs-vuauthorization-attacks/}


\slide{D-Link NAS devices accessible via “backdoor” account CVE-2024-3273}

%\hlkimage{}{}

\begin{quote}{\large\bf
92,000+ internet-facing D-Link NAS devices accessible via “backdoor”}

A vulnerability (CVE-2024-3273) in four old D-Link NAS models could be exploited to compromise internet-facing devices, a threat researcher has found.

The existence of the flaw was confirmed by D-Link last week, and an exploit for opening an interactive shell has popped up on GitHub.

“The vulnerability lies within the \verb+nas_sharing.cgi+ uri, which is vulnerable due to two main issues: a backdoor facilitated by hardcoded credentials, and a command injection vulnerability via the system parameter,” says the discoverer, who goes by the online handle “netsecfish”.

{\bf The “backdoor” account has messagebus as the username and doesn’t require a password.}
\end{quote}
Source: \url{https://www.helpnetsecurity.com/2024/04/08/cve-2024-3273/}


\slide{XZ backdoor}

This month, only days ago it surfaced that someone injected backdoors into some software named XZ.
\emph{Inside the failed attempt to backdoor SSH globally — that got caught by chance}

\begin{quote}
What happened here is now well documented elsewhere, so I shall not recap it much, but essentially somebody appears to have hijacked the open source XZ project by social engineering the volunteer developer into handing over maintainer access after they cited some mental health issues, used the package XZ Utils to piggy back into systemd loading liblzma, which in turn loaded XZ, allowing sshd to be hooked to trojan it on Linux distributions that use systemd.

The {\bf trojan allows somebody a private key to hijack sshd to execute commands}, amongst other functions. It is highly advanced.
\end{quote}
Source:
\url{https://doublepulsar.com/inside-the-failed-attempt-to-backdoor-ssh-globally-that-got-caught-by-chance-bbfe628fafdd}

\begin{list2}
    \item Post by AndresFreundTec \url{https://mastodon.social/@AndresFreundTec/112180083704606941}
\end{list2}


\slide{Protection, building secure and robust networks}

\hlkimage{14cm}{sample-ip-network.pdf}


\begin{list2}
\item We should prefer security mechanisms that does NOT require us to keep patching every month
\item Can we change our networks to avoid this? Yes!
\end{list2}


\slide{Defense in depth}

%\hlkimage{10cm}{Bartizan.png}
\hlkimage{15cm}{medieval-clipart-5}
\centerline{Picture originally from: \url{http://karenswhimsy.com/public-domain-images}}


\slide{Cilium overview}

\hlkimage{12cm}{cilium-overview.png}

\begin{quote}
Kubernetes provides Network Policies for controlling traffic going in and out of the pods. Cilium implements the Kubernetes Network Policies for L3/L4 level and extends with L7 policies for granular API-level security for common protocols such as HTTP, Kafka, gRPC, etc
\end{quote}
Source: picture and text from \link{https://cilium.io/blog/2018/09/19/kubernetes-network-policies/}


\slide{Security is more than blocking!}

\hlkimage{22cm}{cilium-features.png}

\begin{list2}
\item A lot of features relate to \emph{security}
\end{list2}



\slide{Fokus: VPN alle steder}

\hlkimage{12cm}{ks-kyung-784757-unsplash.jpg}

\begin{list2}
\item VPN er relevant for alle der har data af værdi
\item Sikrer data der flyttes
\item Virtual Private Network dækker over klienter der kobler op, og site-2-site
\end{list2}


\slide{Fokus: DNS og email}

\hlkimage{4cm}{brian-patrick-tagalog-680954-unsplash.jpg}

\begin{list2}
\item Vi er afhængige af email, modtagelse og afsendelse
\item Når vi modtager skal det helst gå hurtigt
\item Når vi sender skal vi ikke ende i spam mappen
\item Phishing, hvem kan sende \emph{fra vores domæne}
\end{list2}


\slide{Various key attack types, clients and employees}

\begin{list2}
\item Phishing - sending fake emails, to collect credentials
\item Spear phishing - targetted attacks
\item Person in the middle - sniffing and changing data in transit
\item Drive-by attacks - web pages infected with malware, often ad servers
\item Malware transferred via USB or email
\item Credential Stuffing, Password related, like re-use of password, see slide about being pwned
\end{list2}

\vskip 1cm
\centerline{\Large\bf If we all wait a bit, and not click links immediately}

\vskip 1cm
Hackers try to create "urgency", click this or loose money


\slide{Storing query logs, old school or needed?}

\hlkimage{5cm}{bro-sample-ssl-scripts.png}

\begin{list2}
\item DNS query logs, keep it for at least a week?\\
- with DSC and PacketQ \link{https://github.com/DNS-OARC/PacketQ}
\item SSL/TLS log with Zeek/Suricata\\
{\footnotesize\link{https://www.zeek.org/sphinx-git/script-reference/scripts.html}}
\item Log with Elasticsearch?\\
{\footnotesize\link{https://www.elastic.co/guide/en/elasticsearch/guide/current/index.html}}
%\item Even netflow session logging, full 1:1 - NFSen, Suricata Flow mode?
%\item Moloch \link{https://github.com/aol/moloch}
\item  Uetisk? eller smart hvis man vil spore hvor malware kom ind
\item {\bf Vi må nok som medarbejdere acceptere mere logning, men selvfølgelig ikke som privatpersoner og borgere}
\end{list2}


\slide{Network visibility: Netflow with NFSen}

\hlkimage{15cm}{nfsen-udp-flood.png}

\centerline{An extra 100k packets per second from this netflow source (source is a router)}

Logging can show what happens/happened.


\slide{Fokus: Incident Response og reaktion}

\hlkimage{10cm}{margarida-csilva-121801-unsplash.jpg}

\begin{list2}
\item Fortsat fra logningen ... hvad så nu!
\item Hvis du har en sikkerhedshændelse skal den håndteres
\item jo hurtigere og mere effektivt det håndteres jo bedre
\end{list2}

Lifeguard training photo by Margarida CSilva on Unsplash

\slide{Øv krisesituationer}

\hlkimage{14cm}{sheldon-nunes-1226991-unsplash.jpg}

\begin{list2}
\item Lav rollespil
\item Lav tabletop exercises
\end{list2}


\slide{Building Secure Infrastructures}

\begin{list1}
\item A real-life setup of an infrastructure from scratch can be daunting!
\item You need:
\begin{list2}
\item Policies
\item Procedures
\item Incident Response
\end{list2}
\item Running systems which require
\begin{list2}
\item Configurations
\item Settings
\item Supporting infrastructure -- networks
\item Supporting infrastructure -- logging, dashboarding, monitoring
\end{list2}
\item Building something \emph{secure} is {\bf hard work!}
\end{list1}



\slide{Existing infrastructures}

\begin{list1}
\item or even worse you inherited an infrastructure
\item Multiple networks, with different vendors, rules
\item Multiple generations of services, applications, technologies
\item Built over decades
\item Varying to no documentation
\item Organizational challenges
\item Ingrained culture -- frozen in time
\end{list1}

How do you get started improving security?


\slide{Security Controls and Frameworks}

\begin{list1}
\item Multiple exist
\vskip 1cm
\begin{list2}
\item CIS controls Center for Internet Security (CIS) \link{https://www.cisecurity.org}
\item PCI Best Practices for Maintaining PCI DSS Compliance v2.0 Jan 2019
\item NIST Cybersecurity Framework (CSF)\\
Framework for Improving
Critical Infrastructure Cybersecurity\\ \link{https://www.nist.gov/cyberframework}\\
\link{http://csrc.nist.gov/publications/PubsSPs.html}
\item National Security Agency (NSA)\\ \link{http://www.nsa.gov/research/publications/index.shtml}
\item NSA security configuration guides\\ \link{http://www.nsa.gov/ia/guidance/security_configuration_guides/index.shtml}
\item Information Systems Audit and Control Association (ISACA)\\
\link{http://www.isaca.org/Knowledge-Center/Risk-IT-IT-Risk-Management/Pages/default.aspx}
\end{list2}
\end{list1}


\slide{Risk management defined}

\hlkimage{20cm}{shon-harris-risk-management.png}

Source: Shon Harris \emph{CISSP All-in-One Exam Guide}


\slide{Center for Internet Security CIS Controls}

\begin{quote}
  The CIS ControlsTM are a prioritized set of actions that collectively form a defense-in-depth set
of best practices that mitigate the most common attacks against systems and networks. The
CIS Controls are developed by a community of IT experts who apply their first-hand experience
as cyber defenders to create these globally accepted security best practices. The experts who
develop the CIS Controls come from a wide range of sectors including retail, manufacturing,
healthcare, education, government, defense, and others.
\end{quote}

Source: \link{https://www.cisecurity.org/} CIS-Controls-Version-7-1.pdf

\slide{Center for Internet Security CIS Controls 7.1}

\begin{list2}
\item
The five critical tenets of an effective cyber defense system as reflected
in the CIS Controls are:
\item {\bf Offense informs defense:} Use knowledge of actual attacks that have
compromised systems to provide the foundation to continually learn
from these events to build effective, practical defenses. Include only
those controls that can be shown to stop known real-world attacks.
\item {\bf Prioritization:} Invest first in Controls that will provide the greatest risk
reduction and protection against the most dangerous threat actors
and that can be feasibly implemented in your computing environment.
The CIS Implementation Groups discussed below are a great place for
organizations to start identifying relevant Sub-Controls.
\item {\bf Measurements and Metrics:} Establish common metrics to provide a
shared language for executives, IT specialists, auditors, and security
officials to measure the effectiveness of security measures within
an organization so that required adjustments can be identified and
implemented quickly.
\item {\bf Continuous diagnostics and mitigation:} Carry out continuous
measurement to test and validate the effectiveness of current security
measures and to help drive the priority of next steps.
\item {\bf Automation:} Automate defenses so that organizations can achieve
reliable, scalable, and continuous measurements of their adherence to
the Controls and related metrics. \hskip 2cm Source: CIS-Controls-Version-7-1.pdf
\end{list2}


\slide{Inventory and Control of Hardware Assets}

CIS controls 1-6 are Basic, everyone must do them.


\begin{quote}
CIS Control 1:\\
Inventory and Control of Hardware Assets\\
Actively manage (inventory, track, and correct) all hardware devices on the network so that only authorized devices are given access, and unauthorized and unmanaged devices are found and prevented from gaining access.
\end{quote}

\begin{list1}
\item What is connected to our networks?
\item What firmware do we need to install on hardware?
\item Where IS the hardware we own?
\item What hardware is still supported by vendor?
\end{list1}

Source: Center for Internet Security CIS Controls 7.1 CIS-Controls-Version-7-1.pdf


\slide{Inventory and Control of Software Assets}

\begin{quote}
CIS Control 2:\\
Inventory and Control of Software Assets\\
Actively manage (inventory, track, and correct) all software on the network so that only authorized software is installed and can execute, and that all unauthorized and unmanaged software is found and prevented from installation or execution.
\end{quote}

\begin{list1}
\item What licenses do we have? Paying too much?
\item What versions of software do we depend on?
\item What software needs to be phased out, upgraded?
\item What software do our employees need to support?
\end{list1}

Source: Center for Internet Security CIS Controls 7.1 CIS-Controls-Version-7-1.pdf


\slide{Continuous Vulnerability Management}

\begin{quote}
CIS Control 3:\\
Continuous Vulnerability Management\\
Continuously acquire, assess, and take action on new information in order to identify vulnerabilities, remediate, and minimize the window of opportunity for attackers.
\end{quote}

\slide{Spørgsmål og mere debat}

\hlkimage{7cm}{idog.jpg}

\begin{center}
\hlkbig

\myname

\end{center}

\end{document}



\end{document}
