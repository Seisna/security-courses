\documentclass[a4paper,11pt,notitlepage]{report}
% Henrik Lund Kramshoej, February 2001
% hlk@security6.net,
% My standard packages
\usepackage{zencurity-network-exercises}

\begin{document}

\rm
\selectlanguage{english}

\mytitle{Simulating DDoS Attacks}{breaking the firewall infrastructure}

\pagenumbering{roman}


\setcounter{tocdepth}{0}

\normal

{\color{titlecolor}\tableofcontents}
%\listoffigures - not used
%\listoftables - not used

\normal
\pagestyle{fancyplain}



\chapter*{Simulating DDoS packets}

A small workshop teaching people how to produce DDoS simulation traffic - usefull for testing their own infrastructures.

We will have a server connected on 10Gbit on a switch with multiple 1Gbit port for attackers. Attackers will be connected through 1Gbit ports using USB Ethernet - we have loaners.

Work together to produce enough to take down this server!

WHILE attack is ongoing there will be both the possibility to monitor traffic, monitor port, and decide on changes to prevent the attacks from working.

We will work through common attack types, like:

\begin{itemize}
\item TCP SYN flooding
\item TCP other flooding
\item UDP flooding NTP, etc.
\item ICMP flooding
\item Misc - stranger attacks and illegal combinations of flags etc.
\end{itemize}

then we will discuss and implement changes suggested and retry attacks.

You will go away from this with tools for producing packets, hping3 and during the workshop we will discuss some configurations for protecting - PF rules, switch rules, server firewall rules.




\chapter*{\color{titlecolor}Preface}
\markboth{Preface}{}

This material is prepared for use in my trainings and was prepared by
Henrik Kramselund, \link{http://www.zencurity.com} .
It describes the networking setup and
applications for trainings and workshops where hands-on exercises are needed.

\vskip 1cm
Further a presentation is used which is available as PDF from kramse@Github\\
Look for \jobname\ in the repo security-courses.

These exercises are expected to be performed in a training setting with network connected systems. The exercises use a number of tools which can be copied and reused after training. A lot is described about setting up your workstation in the repo

\link{https://github.com/kramse/kramse-labs}

\section*{\color{titlecolor}Prerequisites}

This material expect that participants have a working knowledge of
TCP/IP from a user perspective. Basic concepts such as web site addresses and email should be known as well as IP-addresses and common protocols like DHCP.

\vskip 1cm
Have fun and learn
\eject

% =================== body of the document ===============
% Arabic page numbers
\pagenumbering{arabic}
\rhead{\fancyplain{}{\bf \chaptername\ \thechapter}}

% Main chapters
%---------------------------------------------------------------------
% gennemgang af emnet
% check questions

\chapter*{\color{titlecolor}Exercise content}
\markboth{Exercise content}{}

Most exercises follow the same procedure and has the following content:
\begin{itemize}
\item {\bf Objective:} What is the exercise about, the objective
\item {\bf Purpose:} What is to be the expected outcome and goal of doing this exercise
\item {\bf Suggested method:} suggest a way to get started
\item {\bf Hints:} one or more hints and tips or even description how to
do the actual exercises
\item {\bf Solution:} one possible solution is specified
\item {\bf Discussion:} Further things to note about the exercises, things to remember and discuss
\end{itemize}

Please note that the method and contents are similar to real life scenarios and does not detail every step of doing the exercises. Entering commands directly from a book only teaches typing, while the exercises are designed to help you become able to learn and actually research solutions.


\chapter{Discover active systems ping and port sweep 15 min}
\label{ex:nmap-pingsweep}
\hlkimage{5cm}{nmap-zenmap.png}

{\bf Objective:}\\
Use nmap to discover active systems and ports

{\bf Purpose:}\\
Know how to use nmap to scan networks for active systems. These ports receive traffic from \emph{the internet} and can be used for DDoS attacks.

Tip: Yes, filtering traffic further out removes it from processing in routers, firewalls, load balancers, etc. So making a stateless filter on the edge may be recommended.

{\bf Suggested method:}\\
Try different scans,
\begin{itemize}
\item Ping sweep to find active systems
\item Port sweeps to find active systems with specific ports
\end{itemize}

{\bf Hints:} \\
Try nmap in sweep mode - and you may run this from Zenmap

{\bf Solution:}\\
Use the command below as examples:
\begin{itemize}
\item Ping sweep ICMP and port probes: \verb+nmap -sP 10.0.45.*+
\item Port sweeps 80/tcp and 443/tcp: \verb+nmap -p 80 10.0.45.*+
\item Port sweeps UDP scans can be done: \verb+nmap -sU -p 161 10.0.45.*+
\end{itemize}

{\bf Discussion:}\\
Quick scans quickly reveal interesting hosts, ports and services

Also now make sure you understand difference between single host scan
10.0.45.64/32, a whole subnet /24 ~250 hosts 10.0.45.0/24 and other more advanced targeteting like 10.0.45.0/25 and 10.0.45.1-10

We will now assume port 80/443 are open, as well as a few UDP services - maybe we can use them in amplification attacks later.

Hint: there is a GUI for Nmap called Zenmap. This depends on Python 2.7 and while it can be installed using Alien to convert RPM package and installing dependencies, it is not recommended anymore.

Instead you can use the Kaboxer - Kali Applications Boxer,\\
\link{https://www.kali.org/blog/introducing-kaboxer/} and \verb+apt install zenmap-kbx+

\chapter{TCP SYN flooding 30min}
\label{ex:syn-flood}

{\bf Objective:}\\
Start a web server attack using SYN flooding tool hping3.

{\bf Purpose:}\\
See how easy it is to produce packets on a network using hacker programs.

The tool we will use is very flexible and can produce ICMP, UDP and TCP using very few options. This tool is my primary one for doing professsional DDoS testing.

\begin{alltt}\footnotesize
-1 --icmp
       ICMP  mode,  by  default  hping3  will  send  ICMP echo-request, you can set other ICMP
       type/code using --icmptype --icmpcode options.

-2 --udp
       UDP mode, by default hping3 will send udp to target host's port 0.  UDP header  tunable
       options are the following: --baseport, --destport, --keep.
\end{alltt}

TCP mode is default, so no option needed.


{\bf Suggested method:}\\
Connect to the LAB network using Ethernet! Borrow a USB network card if you dont have one.

Start your Kali VM in bridged mode, try a basic TCP flooding attack against the server provided by the instructor, or your own Debian server.

Try doing the most common attacks TCP SYN flood using hping3:

\begin{alltt}
hping3 --flood -p 80 -S 10.0.45.12
\end{alltt}

You should see something like this:
\begin{alltt}\footnotesize
HPING 10.0.45.12: NO FLAGS are set, 40 headers + 0 data bytes
hping in flood mode, no replies will be shown
^C
--- 10.0.45.12 hping statistic ---
352339 packets transmitted, 0 packets received, 100% packet loss
round-trip min/avg/max = 0.0/0.0/0.0 ms
\end{alltt}

You can try different ports with TCP flooding, try port 22/tcp or HTTP(S) port 80/tcp and 443/tcp


{\bf Hints:}\\
The tool we use can do a lot of different things, and you can control the speed. You can measure at the server being attacked or what you are sending, commonly using ifpps or such programs can help.

By changing the speed we can find out how much traffic is needed to bring down a service. This measurement can then be re-checked later and see if improvements really worked.

This allows you to use the tool to test devices and find the breaking point, which is more interesting than if you can overload, because you always can.
\begin{alltt}\footnotesize
-i --interval
       Wait  the  specified  number  of  seconds or micro seconds between sending each packet.
       --interval X set wait to X seconds, --interval uX set wait to X micro seconds.  The de‐
       fault  is  to  wait one second between each packet. Using hping3 to transfer files tune
       this option is really important in order to increase transfer rate. Even  using  hping3
       to  perform  idle/spoofing  scanning  you should tune this option, see HPING3-HOWTO for
       more information.

--fast Alias for -i u10000. Hping will send 10 packets for second.

--faster
       Alias for -i u1. Faster then --fast ;) (but not as fast as your computer can send pack‐
       ets due to the signal-driven design).

--flood
       Sent  packets  as fast as possible, without taking care to show incoming replies.  This
       is ways faster than to specify the -i u0 option.
\end{alltt}

{\bf Solution:}\\
When your team has sent +1 million packets per second into the network, from one or two laptops - you are done.

{\bf Discussion:}\\
Gigabit Ethernet can send up to 1.4 million packets per second, pps.

There is a presentation about DDoS protection with low level technical measures to implement at\\
{\footnotesize \link{https://github.com/kramse/security-courses/tree/master/presentations/network/introduction-ddos-testing}}

Receiving systems, and those en route to the service, should be checked for resources like CPU load, bandwidth, logging. Logging can also overload the logging infrastructure, so take care when configuring this in your own networks.


\chapter{TCP other flooding 15min}


{\bf Objective:}\\
Start a webserver attack using TCP flooding tool hping3.

{\bf Purpose:}\\
Run various other common attacks

TCP mode is default, so no option needed.


{\bf Suggested method:}\\
Connect to the LAB network using Ethernet! Borrow a USB network card if you dont have one.

Start your Kali VM in bridged mode, try a basic TCP flooding attack against the server provided by the instructor, or your own Debian server.

\begin{alltt}
hping3 --flood -p 80 -R 10.0.45.12
\end{alltt}

You should see something like this:
\begin{alltt}\footnotesize
HPING 10.0.45.12: NO FLAGS are set, 40 headers + 0 data bytes
hping in flood mode, no replies will be shown
^C
--- 10.0.45.12 hping statistic ---
352339 packets transmitted, 0 packets received, 100% packet loss
round-trip min/avg/max = 0.0/0.0/0.0 ms
\end{alltt}


{\bf Hints:}\\
Common attacks use the SYN, as shown in previous exercise, but other popular
TCP attacks are RST, PUSH, URG, FIN, ACK attacks - setting one or more flags in the packets.

\begin{alltt}
-L  --setack     set TCP ack
-F  --fin        set FIN flag
-S  --syn        set SYN flag
-R  --rst        set RST flag
-P  --push       set PUSH flag
-A  --ack        set ACK flag
-U  --urg        set URG flag
-X  --xmas       set X unused flag (0x40)
-Y  --ymas       set Y unused flag (0x80)
\end{alltt}





{\bf Solution:}\\
When your team has sent +1 million packets per second into the network, from one or two laptops - you are done.

{\bf Discussion:}\\
If an attacker varies the packets they can be harder to filter out, and the attacks succeed.

\chapter{UDP flooding NTP, etc. 15min}


{\bf Objective:}\\
Start a webserver attack using UDP flooding tool hping3.

{\bf Purpose:}\\
See how easy it is to produce packets on a network using hacker programs.

The tool we will use is very flexible and can produce ICMP, UDP and TCP using very few options. This tool is my primary one for doing professsional DDoS testing.

This time we will select UDP mode:

\begin{alltt}\footnotesize
-2 --udp
       UDP mode, by default hping3 will send udp to target host's port 0.  UDP header  tunable
       options are the following: --baseport, --destport, --keep.
\end{alltt}

{\bf Suggested method:}\\
Connect to the LAB network using Ethernet! Borrow a USB network card if you dont have one.

Start your Kali VM in bridged mode, try a basic TCP flooding attack against the server provided by the instructor, or your own Debian server.

\begin{alltt}\footnotesize
hping3 --flood -2 -p 53 10.0.45.12
\end{alltt}



{\bf Hints:}\\

Try doing the most common attacks:
\begin{itemize}
\item UDP flooding, try port 53/udp DNS, 123/udp NTP and port 161/udp SNMP
\end{itemize}

{\bf Solution:}\\
When your team has sent +1 million packets per second into the network, from one or two laptops - you are done.

{\bf Discussion:}\\
Many networks don't send and receive a lot of UDP traffic. If you measure a baseline of the protocols needed on a daily basis you might be able to configure a profile for normal usage, and filter out bad traffic in case of attacks.

A starting point might be to allow full bandwidth for TCP, 10\% UDP and 1\% ICMP. This will ensure that even if an attacker is sending more than 1\% ICMP only a fraction reaches your network and systems.

This is especially effective for protocols like ICMP which is not used for large data transfers.

\chapter{ICMP flooding 15min}

{\bf Objective:}\\
Start a webserver attack using ICMP flooding tool hping3.

{\bf Purpose:}\\
See how easy it is to produce packets on a network using hacker programs.

The tool we will use is very flexible and can produce ICMP, UDP and TCP using very few options. This tool is my primary one for doing professsional DDoS testing.

This time we will select UDP mode:

\begin{alltt}\footnotesize
-1 --icmp
       ICMP  mode,  by  default  hping3  will  send  ICMP echo-request, you can set other ICMP
       type/code using --icmptype --icmpcode options.
\end{alltt}

{\bf Suggested method:}\\
Connect to the LAB network using Ethernet! Borrow a USB network card if you dont have one.

Start your Kali VM in bridged mode, try a basic TCP flooding attack against the server provided by the instructor, or your own Debian server.

Try doing the most common attack:
\begin{itemize}
\item ICMP flooding with echo
\end{itemize}

\begin{alltt}\footnotesize
hping3 --flood -1 10.0.45.12
\end{alltt}





{\bf Hints:}\\
Common attacks use ICMP ECHO, but other types can be sent in the packets.

\begin{alltt}\footnotesize
ICMP
  -C  --icmptype   icmp type (default echo request)
  -K  --icmpcode   icmp code (default 0)
      --force-icmp send all icmp types (default send only supported types)
      --icmp-gw    set gateway address for ICMP redirect (default 0.0.0.0)
      --icmp-ts    Alias for --icmp --icmptype 13 (ICMP timestamp)
      --icmp-addr  Alias for --icmp --icmptype 17 (ICMP address subnet mask)
      --icmp-help  display help for others icmp options
\end{alltt}


{\bf Solution:}\\
When your team has sent +1 million packets per second into the network, from one or two laptops - you are done.

{\bf Discussion:}\\
If you have a 10G network connection, do you REALLY need 10Gbps of ICMP traffic?

Probably not, and routers can often filter this in wirespeed.

Routers have extensive Class-of-Service (CoS) tools today and a starting point might be as shown in Juniper Junos policer config:

\begin{alltt}\footnotesize
term limit-icmp \{
    from \{
        protocol icmp;
    \}
    then \{
        policer ICMP-100M;
        accept;
    \}
\}
term limit-udp \{
    from \{
        protocol udp;
    \}
    then \{
        policer UDP-1000M;
        accept;
    \}
\}
\end{alltt}

This effectively limit the damage an attacker can do. Your firewall and IDS devices will be free to spend more processing on the remaining protocols.



\chapter{Misc - stranger attacks 15min}

Various other attacks are possible, sending illegal combinations of flags etc.



{\bf Objective:}\\
Start a webserver attack using the packet generator and flooding tool t50.

{\bf Purpose:}\\
See how easy it is to produce packets on a network using hacker programs.

The tool we will use is very flexible and can produce ICMP, UDP and TCP using very few options. This tool is another primary one for doing professsional DDoS testing.

Apart from TCP,UDP and ICMP this tool can also produce packets for dynamic routing testting, OSPF, EIGRP and other esoteric RSVP, IPSEC, RIP and GRE.

\begin{alltt}\footnotesize
  $ t50 -help
  T50 Experimental Mixed Packet Injector Tool v5.8.3
  Originally created by Nelson Brito <nbrito@sekure.org>
  Previously maintained by Fernando Mercês <fernando@mentebinaria.com.br>
  Maintained by Frederico Lamberti Pissarra <fredericopissarra@gmail.com>

  Usage: t50 <host[/cidr]> [options]
  Common Options:
      --threshold NUM           Threshold of packets to send     (default 1000)
      --flood                   This option supersedes the 'threshold'
      --encapsulated            Encapsulated protocol (GRE)      (default OFF)
   -B,--bogus-csum              Bogus checksum                   (default OFF)
      --shuffle                 Shuffling for T50 protocol       (default OFF)
   -q,--quiet                   Disable INFOs
      --turbo                   Extend the performance           (default OFF)
   -l,--list-protocols          List all available protocols
   -v,--version                 Print version and exit
   -h,--help                    Display this help and exit
...
   Some considerations while running this program:
    1. There is no limitation of using as many options as possible.
    2. Report t50 bugs at https://gitlab.com/fredericopissarra/t50.git.
    3. Some header fields with default values MUST be set to '0' for RANDOM.
    4. Mandatory arguments to long options are mandatory for short options too.
    5. Be nice when using t50, the author DENIES its use for DoS/DDoS purposes.
    6. Running t50 with '--protocol T50' option sends ALL protocols sequentially.

\end{alltt}


{\bf Suggested method:}\\
Connect to the LAB network using Ethernet! Borrow a USB network card if you dont have one.

Start your Kali VM in bridged mode, try a basic TCP flooding attack against the server provided by the instructor, or your own Debian server.

Run the help page, and browse options.
\begin{alltt}\footnotesize
t50 -h
\end{alltt}





{\bf Hints:}\\
The tools we use can do a lot of different things and using the command line options can produce high speed packet attacks without having to program in C ourselves.

Try doing a special attack:
\begin{itemize}
\item t50 with '--protocol T50' option sends ALL protocols, so try:\\
\verb+t50 --protocol T50 10.0.45.12+
\end{itemize}


{\bf Solution:}\\
When your team has sent +1 million packets per second into the network, from one or two laptops - you are done.

{\bf Discussion:}\\
Gigabit Ethernet can send up to 1.4 million packets per second, pps.

There is a presentation about DDoS protection with low level technical measures to implement at\\
{\footnotesize \link{https://github.com/kramse/security-courses/tree/master/presentations/network/introduction-ddos-testing}}

Receiving systems, and those en route to the service, should be checked for resources like CPU load, bandwidth, logging. Logging can also overload the logging infrastructure, so take care when configuring this in your own networks.


\chapter{Bonus: DNS, NTP and SNMP amplification 15min}
\label{ex:}


{\bf Objective:}\\
Start a webserver attack using reflection attacks.

{\bf Purpose:}\\
See how easy it is to produce packets on a network using hacker programs.

See that attackers might not use their own connection directly, and using other peoples systems they can increase the attack size by sending requests with spoofed source.

{\bf Suggested method:}\\
Connect to the LAB network using Ethernet! Borrow a USB network card if you dont have one.

Start your Kali VM in bridged mode, try a basic TCP flooding attack against the server provided by the instructor, or your own Debian server.

Try sending requests
\begin{alltt}\footnotesize


\end{alltt}

Try doing the most common UDP reflection attacks:
\begin{itemize}
\item DNS port 53/udp
\item NTP 123/udp
\item SNMP port 161/udp
\end{itemize}


{\bf Hints:}\\


{\bf Solution:}\\
When your team has sent +1 million packets per second into the network, from one or two laptops - you are done.

{\bf Discussion:}\\

A lot of DDoS protection happens at the edge. This is the recommendation, but if you have a large network, you might have amplification sources inside your network or the ISPs network.

In our case we can imagine preventing a direct UDP attack being blocked, and then circumventing this by sending requests into the network - so it will be amplified on the inside.

Anti-spoofing of IP addresses are the main ways to prevent this. Unfortunately this is not implemented on the internet at large - too many networks allow spoofing of addresses.

You should make sure the networks you control implement network ingress and egress filtering. One resource is the BCP38  Network Ingress Filtering:
            Defeating Denial of Service Attacks which employ
                       IP Source Address Spoofing

\link{https://tools.ietf.org/html/bcp38}


\end{document}
