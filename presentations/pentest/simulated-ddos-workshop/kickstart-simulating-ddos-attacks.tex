\documentclass[a4paper,11pt,notitlepage,landscape]{report}
% Henrik Kramselund, February 2001
% hlk@zencurity.com,
% My standard packages
\usepackage{zencurity-one-page}
%\usepackage{lscape}
\usepackage{graphicx}

\begin{document}

\rm
\selectlanguage{english}

\newcommand{\subject}[1]{Simulating DDoS Attacks Workshop }

\lhead{\fancyplain{}{\color{titlecolor}\bfseries\LARGE Kickstart: \subject} Tuesday 8th from 11:00 - 13:00 @KEA Village in Party}

\normal
\hlkimage{4cm}{penguinping.jpg}
This material is prepared for use in \emph{\subject} and was prepared by
Henrik Kramselund, \url{hlk@zencurity.com} \url{xhek@kea.dk}.
It contains the very basic information to get started!

This workshop and exercises are expected to be performed in a training setting with network connected systems. The exercises use a number of tools which can be copied and reused after training. A lot is described about setting up your workstation in the Github repositories.

{\bf The main site is: \link{https://github.com/kramse/security-courses/tree/master/presentations/pentest/simulated-ddos-workshop}}\\
To get kickstarted in this workshop:

\begin{list2}
\item[\faSquareO] Slides and exercises booklet -- recommend downloading latest version of workshop PDF and exercises PDF single files\\
\link{https://github.com/kramse/security-courses/tree/master/presentations/pentest/simulated-ddos-workshop}
\item[\faSquareO] You don't need a virtual machine, but these tools are easily installed on Debian and Kali Linux. Read about setup of exercise systems here\\
\link{https://github.com/kramse/kramse-labs}
\item[\faSquareO] Select and install attack software -- your package system could have hping3 or t50 or both
\item[\faSquareO] We will use hping3, t50 and I will use Penguinping \link{https://penguinping.org/}
\item[\faSquareO] Borrow Ethernet card if you don't have Ethernet. You can do a little Wi-Fi attacks, but it will affect neighbors \smiley
\item[\faSquareO] After the technical workshop we will have a discussion workshop about {\bf Co-creating a DDoS Taxonomy} from 13:00-15:00
\end{list2}

I hope we will have a fun and enjoyable time in this workshop.

\end{document}
