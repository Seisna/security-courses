\documentclass[Screen16to9,17pt]{foils}
%\documentclass[16pt,landscape,a4paper,footrule]{foils}
\usepackage{zencurity-slides}
\externaldocument{pentest-II-exercises}

% Penetration testing II - webbaserede angreb

% Mål:
% Introduktion til penetrationstest.

% Forudsætninger:
% Der forventes kendskab til TCP/IP på brugerniveau.

% Beskrivelse:
% Websystemer idag er kendetegnede ved at være meget dynamiske, komplekse og store - og derfor ofte fyldt med sikkerhedsfejl. På dette foredrag vil vi se nærmere på teknologierne og de alment forekommende fejl i websystemer og andre emner:

% Vi gennemgår OWASP Top-10 og giver eksempler på værktøjer til at undersøge websystemer.

% * Websystemer introduktion
% * HTTP protokoller, servere og sikkerhed
% * Open Web Application Security Project OWASP Top-10
% * Hello world of insecure CGI programming
% * Command og SQL injection
% * Brug af proxyprogrammer Tamper Data og Burp Suite demonstreres
% * Webcrawlere og web scannere Nikto, w3af, Skipfish

%Der vil være demonstrationer af sårbarheder på alle foredragene - typisk med open source programmer, således at deltagerne kan afprøve de selvsamme demoer hjemme.

% Husk:
% sqlmap
% Skipfish
% w3af

% HK versionen
% Websystemer i dag er kendetegnede ved at være meget dynamiske, komplekse og store - og derfor ofte fyldt med sikkerhedsfejl.
% På dette foredrag vil vi se nærmere på teknologierne og de alment forekommende fejl i websystemer og tilhørende emner.
% Vi gennemgår OWASP Top-10 og giver eksempler på værktøjer til at undersøge websystemer.

% * HTTP protokoller, servere og sikkerhed
% * Proxy programmer Tamper Data og Burp Suite
% * Open Web Application Security Project OWASP Top-10
% * Hello world of insecure CGI programming
% * Command og SQL injection, sqlmap
% * PHP sikkerhed, Rails, Python - introduktion og gode råd
% * Webcrawlere og web scannere Nikto, w3af, Skipfish




\begin{document}

\selectlanguage{english}

%{Penetration testing II\\\normalsize webbaserede angreb}

\mytitlepage
{Pentest II Web Attacks}

%\begin{alltt}
%\tiny
%\centerline{$Id: pentest-II-foredrag.tex,v 1.2 2006/11/08 14:41:45 hlk Exp $}
%\end{alltt}

\LogoOn

\slide{Plan}

\begin{list1}
\item Subjects
\begin{list2}
\item Introduce pentest in web applications and related technologies
\item Basic web application security concepts
\item Applying basic security assessment Skills
\item Point towards some of the most important resources in this subject
\item Open Worldwide (previously Web) Application Security Project OWASP Top-10 and JuiceShop
\item Introduce scanning with Nikto, Burp Suite and other programs
\item Command and SQL injection
\item Make it possible to get started hacking on web applications
\end{list2}
\item Demos and recommendations for exercises
\begin{list2}
\item Running various tools
\item Note: exercise booklets on Github contain much more than we can go through, continue on your own
\end{list2}
\end{list1}

{\bf Whenever it say exercise, it will be a demo!}

\slide{Goals}
\vskip 1 cm

\hlkimage{3cm}{dont-panic.png}
\centerline{\color{titlecolor}\LARGE Don't Panic!}


\begin{list1}
\item Introduce modern web application security testing
\item Introduce some of the basic tools in this genre of hacker tools
\item Show a hacker lab for web hacking and run some tools
\item Point you towards resources, so you can get started with web hacking  tools
\end{list1}

\slide{Agreements for testing networks}

\begin{quote}\small
Danish Criminal Code\\
Straffelovens paragraf 263 Stk. 2. Med bøde eller fængsel indtil 1 år og 6 måneder straffes den, der uberettiget skaffer sig adgang til en andens oplysninger eller programmer, der er bestemt til at bruges i et informationssystem.
\end{quote}

Hacking can result in:
\begin{list2}
\item Getting your devices confiscated by the police
\item Paying damages to persons or businesses
\item If older getting a fine and a record -- even jail perhaps
\item Getting a criminal record, making it hard to travel to some countries and working in security
\item Fear of terror has increased the focus -- so dont step over bounds!
\end{list2}

Asking for permission and getting an OK before doing invasive tests, always!


\slide{Why talk about hacking web}

\hlkimage{8cm}{pawel-janiak-dxFi8Ea670E-unsplash.jpg}

\begin{list2}
\item Everything is web today
\item HTTPS/TLS used for data transport in most applications
\item Web APIs are used across the industry
\item Even \emph{desktop} and \emph{mobile} application are rarely more than a specialized browser\\ like \url{electronjs.org} -- Chromioum and Node.js
\hfill {\footnotesize Photo by Pawel Janiak on Unsplash}
\end{list2}


\slide{Materials -- where to start}

\begin{list2}
\item This presentation -- slides for today, start here
\item Setup instructions for creating a Kali virtual machine:\\
\link{https://github.com/kramse/kramse-labs}
\item Nmap Workshop exercises\\{\footnotesize
\link{https://github.com/kramse/security-courses/blob/master/courses/pentest/nmap-workshop/nmap-workshop-exercises.pdf}}
\item KEA Pentest course\\{\footnotesize
\link{https://github.com/kramse/security-courses/blob/master/courses/pentest/kea-pentest/}}
\item KEA Security in Web Development course\\{\footnotesize
\link{https://github.com/kramse/security-courses/tree/master/courses/system-and-software/security-in-web-development}}
\item KEA Software Security course exercises\\{\footnotesize
\link{https://github.com/kramse/security-courses/tree/master/courses/system-and-software/software-security}}
\end{list2}

OWASP Juice Shop is a running example so there is some overlap

\slide{Books and Educational Materials}

\begin{list2}
\item \emph{Web Application Security}, Andrew Hoffman, 2020, ISBN: 9781492053118
\item \emph{The Web Application Hacker's Handbook: Finding and Exploiting Security Flaws}
Dafydd Stuttard, Marcus Pinto, Wiley September 2011 ISBN: 978-1118026472
\item \emph{Kali Linux Revealed Mastering the Penetration Testing Distribution}\\
\link{https://www.kali.org/}
\end{list2}

It is recommended to buy the \emph{Pwning OWASP Juice Shop} Official companion guide to the OWASP Juice Shop, but it is also available online for free.

From \link{https://leanpub.com/juice-shop](https://leanpub.com/juice-shop}\\
-- suggested price USD 5.99


We teach using these books at Copenhagen School of Design and Technology (KEA)

\slide{Web Application Security}

\hlkimage{6cm}{hoffman-web-application-security.jpg}
\emph{Web Application Security}, Andrew Hoffman, 2020, ISBN: 9781492053118 called WAS



\slide{Hacker tools}

\hlkimage{3cm}{hackers_JOLIE+1995.jpg}

\begin{list2}
\item Everyone use similar tools, see also \link{http://www.sectools.org/}
\item Portscanning Nmap, Nping -- test ports and services, Nping is great for firewall admins \link{https://nmap.org}
\item Wireshark avanceret netværkssniffer - \link{http://http://www.wireshark.org/}
\item Skipfish \link{http://code.google.com/p/skipfish/}
\item Burp suite \link{http://portswigger.net/burp/}


\item and scripting, PowerShell, Unix shell, Perl, Python, Ruby, \ldots
\end{list2}

Picture: Acid Burn / Angelina Jolie Hackers 1995


\slide{What happens now?}

\begin{list1}
\item Think like a hacker
\item Reconnaissance
\begin{list2}
\item ping sweep, port scan
\item OS detection -- TCP/IP or banner grabbing
\item Service scan -- rpcinfo, netbios, ...
\item telnet/netcat interact with services
\end{list2}
\item Exploit/test: Metasploit, Nikto, exploit programs
\item Cleanup/hardening not shown today, but:
\begin{list2}
\item Make a report or document findings
\item Change, improve and harden systems
\item Go through report with stakeholders, track progress
\item Update programs, settings, configurations, architecture
\end{list2}
\item You also need to show others that you are in control of security
\end{list1}


\slide{Hacker lab setup}

\hlkimage{8cm}{hacklab-1.png}

\begin{list2}
\item Hardware: modern laptop CPU with virtualisation\\
Dont forget to enable hardware virtualisation in the BIOS
\item Virtualisation software: VMware, Virtual box, HyperV pick your poison
\item Linux server system: Debian amd64 64-bit \link{https://www.debian.org/}
\item Setup instructions can be found at \link{https://github.com/kramse/kramse-labs}
\item Target: OWASP Juice Shop Project

\end{list2}


\slide{OWASP Juice Shop Project}

I will also use the OWASP Juice Shop Tool Project as a running example. This is an application which is modern AND designed to have security flaws.

Read more about this project at:\\
\link{https://www.owasp.org/index.php/OWASP_Juice_Shop_Project}\\
\link{https://github.com/bkimminich/juice-shop}

It is recommended to buy the Pwning OWASP Juice Shop Official companion guide to the OWASP Juice Shop from \link{https://leanpub.com/juice-shop} - suggested price USD 5.99

I run this as a docker container on Debian

\exercise{ex:sw-basicDebianVM}
\exercise{ex:sw-basicVM}
\exercise{ex:sw-startjuice}
\exercise{ex:js-burp}

\slide{What is Infrastructure -- Software}


\hlkimage{10cm}{alexander-schimmeck-SeeM4AnkEHE-unsplash.jpg}

\begin{list2}
\item Enterprises today have a lot of computing systems supporting the business needs
\item These are very diverse and were often discrete systems
\item Many things are being web enabled
\end{list2}

\hfill Photo by Alexander Schimmeck on Unsplash

\slide{Business Challenges}

\hlkimage{7cm}{adam-bignell-9tI2z5VZIZg-unsplash.jpg}

\begin{list2}
\item Accumulation of software
\item Legacy systems
\item Partners
\item Various types of data
\item Employee churn, replacement \hfill Photo by Adam Bignell on Unsplash
\end{list2}


\slide{Software Challenges}

\hlkimage{7cm}{john-barkiple-l090uFWoPaI-unsplash.jpg}

\begin{list2}
\item Complexity
\item Various languages
\item Various programming paradigms, client server, monolith, Model View Controller
\item Conflicting data types and available structures \hfill Photo by John Barkiple on Unsplash
\end{list2}



\slide{OSI and Internet Protocols}

\hlkimage{10cm,angle=90}{images/compare-osi-ip.pdf}


\slide{Wireshark -- unencrypted traffic}

\hlkimage{13cm}{images/wireshark-http.png}

See more at \link{https://en.wikipedia.org/wiki/Hypertext_Transfer_Protocol}

\slide{Primary HTTP mthods}

\begin{list2}
\item [GET]
Requests a representation of the specified resource. Requests using GET should only retrieve data and should have no other effect. (This is also true of some other HTTP methods.)[1] The W3C has published guidance principles on this distinction, saying, "Web application design should be informed by the above principles, but also by the relevant limitations."[13] See safe methods below.
\item [POST]
Requests that the server accept the entity enclosed in the request as a new subordinate of the web resource identified by the URI. The data POSTed might be, for example, an annotation for existing resources; a message for a bulletin board, newsgroup, mailing list, or comment thread; a block of data that is the result of submitting a web form to a data-handling process; or an item to add to a database.[14]
\item [PUT]
Requests that the enclosed entity be stored under the supplied URI. If the URI refers to an already existing resource, it is modified; if the URI does not point to an existing resource, then the server can create the resource with that URI.[15]
\end{list2}

Source: \link{https://en.wikipedia.org/wiki/Hypertext_Transfer_Protocol}


\slide{TLS Server Name Indication (SNI) example}

\hlkimage{12cm}{wireshark-sni-twitter.png}



\slide{Initial Overview of Software Security}

\begin{list2}
\item Security Testing Versus Traditional Software Testing
\item Functional testing does not prevent security issues!
\item SQL Injection example, injecting commands into database
\item Attackers try to break the application, server, operating system, etc.
\item Use methods like user input, memory corruption / buffer overflow, poor exception handling, broken authentication, ...
\end{list2}

\vskip 2cm
\centerline{\LARGE Where to start?}


\slide{Input Validation}

Missing or flawed input validation is the number one cause of many of the most severe vulnerabilities:
\begin{list2}
\item Buffer overflows - writing into control structures of programs, taking over instructions and program flow
\item SQL injection - executing commands and queries in database systems
\item Cross-site scripting - web based attack type
\item Recommend centralizing validation routines
\item Perform validation in secure context, controller on server
\item Secure component boundaries
\end{list2}

\slide{Weak Structural Security}

Several sources describe more design flaws:
\begin{list2}
\item Large Attack surface
\item Running a Process at Too High a Privilege Level, dont run everything as root or administrator
\item No Defense in Depth, use more controls, make a strong chain
\item Not Failing Securely
\item Mixing Code and Data
\item Misplaced trust in External Systems
\item Insecure Defaults
\item Missing Audit Logs
\end{list2}

\slide{OWASP Top Ten}

\hlkimage{16cm}{owasp.jpg}

\begin{quote}
The OWASP Top Ten provides a minimum standard for web application
security. The OWASP Top Ten represents a broad consensus about what
the most critical web application security flaws are.
\end{quote}

\begin{list1}
\item The Open Web Application Security Project (OWASP)
\item OWASP produces lists of the most common types of errors in web applications
\item \link{http://www.owasp.org}
\item Create Secure Software Development Lifecycle
\end{list1}



\slide{Vulnerabilities - CVE}

\begin{list1}
\item Common Vulnerabilities and Exposures (CVE):
  \begin{list2}
  \item classification
  \item identification
  \end{list2}
\item When discovered each vuln gets a CVE ID
\item CVE maintained by MITRE - not-for-profit
org for research and development in the USA.
\item National Vulnerability Database search for CVE.
\item Sources: \link{http://cve.mitre.org/} og \link{http://nvd.nist.gov}
\item also checkout OWASP Top-10 \link{http://www.owasp.org/}
\end{list1}

\slide{Sample vulnerabilities}

\begin{list2}
\item \small CVE-2000-0884\\
IIS 4.0 and 5.0 allows remote attackers to read documents outside of
the web root, and possibly execute arbitrary commands, via malformed
URLs that contain UNICODE encoded characters, aka the "Web Server
Folder Traversal" vulnerability.

\item \small CVE-2002-1182\\
IIS 5.0 and 5.1 allows remote attackers to cause a denial of service
(crash) via malformed WebDAV requests that cause a large amount of
memory to be assigned.

\item Exim RCE CVE-2019-10149 June\\ \url{https://www.qualys.com/2019/06/05/cve-2019-10149/return-wizard-rce-exim.txt}

\item Exim RCE CVE-2019-15846 September\\
\url{https://exim.org/static/doc/security/CVE-2019-15846.txt}

\item CVE-2020 Netlogon Elevation of Privilege \\
\link{https://msrc.microsoft.com/update-guide/vulnerability/CVE-2020-1472}
\item Log4J RCE (CVE-2021-44228) - and follow up like CVE-2021-45046, also look at scanners like:\\
\link{https://github.com/fullhunt/log4j-scan}

\end{list2}
Source:\\
\link{http://cve.mitre.org/}



\slide{November 2021: Log4Shell}
\begin{quote}\small
It would not be possible to discuss 2021 in the context of vulnerabilities without the mention of Log4Shell. {\bf A widely used Java-based logging library caused headaches for Security professionals worldwide}. Many scrambled to quantify their use of Log4j within their estates.

A zero-day exploit quickly followed, confirming the worst - {\bf Remote Code Execution (RCE) was indeed possible.} However, what made the nature of the vulnerability even more challenging was the ability to exploit a backend logging system from an unaffected front end host. For example, an attacker can craft a weaponised log entry on a mobile app or webserver not running Log4j. The attacker could make their way through to backend middleware itself running Log4j, which significantly extends the attack surface of the vulnerability.

The NCSC even took the step of recommending the update was immediately applied, whether or not Log4Shell was known to be in use. As is commonly the case with critical vulnerabilities, two successive Log4j patches were subsequently released in the week following the original addressing Denial of Service (DoS) and a further RCE. This further increased workloads of Security and IT teams just as they thought the worst of 2021 had been and gone.
\end{quote}
Source - for this description:\\
\link{https://chessict.co.uk/resources/blog/posts/2022/january/2021-top-security-vulnerabilities/}

See also \link{https://en.wikipedia.org/wiki/Log4Shell}


\slide{Shellshock CVE-2014-6271 - and others}

\hlkimage{12cm}{shellshock-ubuntu.png}

Source:
\link{https://en.wikipedia.org/wiki/Shellshock_(software_bug)}

\centerline{Data sent through multiple levels may run into a bash shell during processing}

\slide{Shellshock - multiple vulnerabilities}

Here is an example of a system that has a patch for CVE-2014-6271 but not CVE-2014-7169:

\hlkimage{12cm}{shellshock-CVE-2014-7169.png}


\verb+X='() { (a)=>\' bash -c "echo date"+

Source: \link{https://en.wikipedia.org/wiki/Shellshock_(software_bug)}

Today we still see platform allowing injection of scripting and commands directly!

\slide{Heartbleed CVE-2014-0160}

\hlkimage{20cm}{heartbleed-com.png}

Source: \link{http://heartbleed.com/}

\slide{Key points after heartbleed}

\hlkimage{16cm}{ssl-tls-breaks-timeline.png}
Source: picture source\\ {\footnotesize\link{https://www.duosecurity.com/blog/heartbleed-defense-in-depth-part-2}}
\begin{list2}
\item Writing SSL software and other secure crypto software is hard
\item Configuring SSL is hard\\
check you own site \link{https://www.ssllabs.com/ssltest/}
\item SSL is hard, finding bugs "all the time"
\link{http://armoredbarista.blogspot.dk/2013/01/a-brief-chronology-of-ssltls-attacks.html}
\end{list2}

\slide{Hacking is magic}

\hlkimage{5cm}{wizard_in_blue_hat.png}

\vskip 1 cm

\centerline{Hacking looks like magic -- especially buffer overflows}


\slide{Hacking is not magic}

\hlkimage{14cm}{ninjas.png}

\centerline{Hacking only demands ninja training and knowledge others don't have}

It is like a puzzle, we need this, this and that. Make it happen in a repeatable way.


\slide{Really do Nmap your world}

\hlkimage{8cm}{nmap-zenmap.png}
\centerline{\bf When learning Nmap use the Zenmap GUI!}

\begin{list2}
\item Nmap is a port scanner, but does more. Can be thought of as a generic vuln scanner by now
\item Use the Kali version with \verb+apt install zenmap-kbx+
\end{list2}


\slide{Basic Portscan}

What is port scanning
\begin{list2}
\item Testing all ports from 0/1 up to 65535
\item Goal is to identify open ports -- vulnerable services
\item Typically TCP and UDP scans
\item TCP scanning is more reliable than UDP scanning
\item TCP handshake is easy to see, due to session setup -- services must respond to SYN with SYN-ACK. Otherwise client programs like browsers will not work
\item UDP applications respond differently -- if at all\\
They might respond to queries and probes in the correct format, \\
If no firewall the operating systems will respond with ICMP on closed ports
\item Use Zenmap while learning Nmap
\end{list2}


\slide{Nmap port sweep web services}

\begin{alltt}\footnotesize
# {\bf nmap -A -p 80,443 www.kramse.org}
Starting Nmap 7.94 ( https://nmap.org ) at 2023-11-12 11:30 CET
Nmap scan report for www.kramse.org (185.129.63.130)

PORT    STATE SERVICE  VERSION
{\color{darkgreen}80/tcp  open  http     nginx}
|_http-title: Did not follow redirect to https://www.kramse.org/
{\color{darkgreen}443/tcp open  ssl/http nginx}
|_http-title: Kramse Portal
| ssl-cert: Subject: commonName=kramse.org
| Subject Alternative Name: DNS:kramse.dk, DNS:kramse.org, DNS:kramselund.com,
DNS:kramselund.dk, DNS:www.kramse.dk, DNS:www.kramse.org, DNS:www.kramselund.com,
DNS:www.kramselund.dk
| Not valid before: 2023-09-10T21:00:34
|_Not valid after:  2023-12-09T21:00:33
|_ssl-date: TLS randomness does not represent time
MAC Address: 52:54:00:71:F7:D1 (QEMU virtual NIC)
Warning: OSScan results may be unreliable because we could not find at least 1 open and 1 closed port
Device type: general purpose
Running (JUST GUESSING): OpenBSD 6.X|4.X (92%)
OS CPE: cpe:/o:openbsd:openbsd:6 cpe:/o:openbsd:openbsd:4.0
Aggressive OS guesses: OpenBSD 6.2 - 6.4 (92%), OpenBSD 4.0 (90%), OpenBSD 6.1 (87%), OpenBSD 6.0 - 6.4 (85%)
\end{alltt}


\slide{Nping check TCP socket connection}

\begin{alltt}\tiny
# {\bf nping --tcp -p80 www.zencurity.dk}

Starting Nping 0.7.40 ( https://nmap.org/nping ) at 2017-02-26 17:15 CET
SENT (0.0412s) {\color{darkgreen}TCP 185.27.115.6:25250 > 185.129.60.130:80 S} ttl=64 id=5872 iplen=40  seq=3020958725 win=1480
RCVD (0.0416s) {\color{darkgreen}TCP 185.129.60.130:80 > 185.27.115.6:25250 SA} ttl=63 id=4918 iplen=44  seq=394075685 win=16384
SENT (1.0417s) TCP 185.27.115.6:25250 > 185.129.60.130:80 S ttl=64 id=5872 iplen=40  seq=3020958725 win=1480
RCVD (1.0420s) TCP 185.129.60.130:80 > 185.27.115.6:25250 SA ttl=63 id=34525 iplen=44  seq=830276468 win=16384
SENT (2.0431s) TCP 185.27.115.6:25250 > 185.129.60.130:80 S ttl=64 id=5872 iplen=40  seq=3020958725 win=1480
RCVD (2.0435s) TCP 185.129.60.130:80 > 185.27.115.6:25250 SA ttl=63 id=62810 iplen=44  seq=1289199807 win=16384
SENT (3.0446s) TCP 185.27.115.6:25250 > 185.129.60.130:80 S ttl=64 id=5872 iplen=40  seq=3020958725 win=1480
RCVD (3.0449s) TCP 185.129.60.130:80 > 185.27.115.6:25250 SA ttl=63 id=43831 iplen=44  seq=2100284412 win=16384
SENT (4.0460s) TCP 185.27.115.6:25250 > 185.129.60.130:80 S ttl=64 id=5872 iplen=40  seq=3020958725 win=1480
RCVD (4.0463s) TCP 185.129.60.130:80 > 185.27.115.6:25250 SA ttl=63 id=38950 iplen=44  seq=2839712282 win=16384

Max rtt: 0.332ms | Min rtt: 0.257ms | Avg rtt: 0.301ms
Raw packets sent: 5 (200B) | Rcvd: 5 (230B) | Lost: 0 (0.00%)
Nping done: 1 IP address pinged in 4.08 seconds
\end{alltt}

This tool from the Nmap package can verify if firewalls are open etc.
SA (Syn Ack) is when the firewall and network works, AND web server is started etc.
If web server not running, would be RESET instead
\link{http://nmap.org}

\slide{sslscan}

\begin{alltt}\small
root@kali:~# {\bf sslscan --ssl2 web.kramse.dk}
Version: 1.10.5-static
OpenSSL 1.0.2e-dev xx XXX xxxx

Testing SSL server web.kramse.dk on port 443
...
  SSL Certificate:
Signature Algorithm: sha256WithRSAEncryption
RSA Key Strength:    2048

Subject:  *.kramse.dk
Altnames: DNS:*.kramse.dk, DNS:kramse.dk
Issuer:   AlphaSSL CA - SHA256 - G2
\end{alltt}

Source:
Originally sslscan from http://www.titania.co.uk
 but use the version on Kali

SSLscan can check your own sites, while Qualys SSLLabs only can test from hostname

\slide{Demo: Nmap and sslscan}

\hlkimage{6cm}{exercise}

\begin{list1}
\item Port scan with Nmap
\item sslscan HTTPS -- do NOT use SSL anymore, since 2014 everything is TLS
\item Transport Layer Security \url{https://en.wikipedia.org/wiki/Transport_Layer_Security}
\item Test your own sites using testing pages like \url{https://internet.nl/}
\end{list1}


\slide{Scan for Heartbleed and SSLv2/SSLv3}

Nmap includes Nmap scripting engine (NSE)

\hlkimage{5cm}{nmap-sslv2.png}

\begin{list1}
\item \verb+nmap -p 443 --script ssl-heartbleed <target>+\\
\link{https://nmap.org/nsedoc/scripts/ssl-heartbleed.html}
\item Almost every new popular vulnerability will have Nmap recipe
\end{list1}



\slide{Generic Network Fault Injection}

\begin{list1}
\item Inserting proxies can allow modification of data in transit
\item Can be used for random bit corruption
\item Can often reproduce the data
\item Automate gathering of evidence
\item You can use a simple Random TCP/UDP fault injector, but usually you would get better results by conforming to some structure
\item Various test cases must tried with potential bad data, examples:
\begin{list2}
\item loooong input in all fields and input -- buffer overflows
\item SQL injection -- trying database commands as input
\item Cross-site scripting -- does inserting JavaScript result in this being included in pages returned
\item Random bytes - recommend using real fuzzers that understand target protocol
\item Metacharacters like null bytes, or \verb+.,:;/\textbackslash+
\end{list2}
\end{list1}

\slide{Exploits}

\begin{alltt}
$buffer = "";
$null = "\textbackslash{}x00"; \pause
$nop = "\textbackslash{}x90";
$nopsize = 1; \pause
$len = 201; // what is needed to overflow, maybe 201, maybe more!
$the_shell_pointer = 0xdeadbeef; // address where shellcode is
# Fill buffer
for ($i = 1; $i < $len;$i += $nopsize) \{
    $buffer .= $nop;
\}\pause
$address = pack('l', $the_shell_pointer);
$buffer .= $address;\pause
exec "$program", "$buffer";
\end{alltt}
\vskip 1 cm
\centerline{Demo exploit in Perl}
%Eksempel på webserver buffer overflow, nosejob?

\slide{local vs. remote exploits}

\begin{list1}
\item {\bfseries local vs. remote}
angiver om et exploit er rettet mod
en sårbarhed lokalt på maskinen, eksempelvis
opnå højere privilegier, eller beregnet
til at udnytter sårbarheder over netværk
\item {\bfseries remote root exploit}
- den type man frygter mest, idet
det er et exploit program der når det afvikles giver
angriberen fuld kontrol, root user er administrator
på UNIX, over netværket.
\item {\bfseries zero-day exploits} dem som ikke offentliggøres - dem
  som hackere holder for sig selv. Dag 0 henviser til at ingen kender
  til dem før de offentliggøres og ofte er der umiddelbart ingen
  rettelser til de sårbarheder
\end{list1}


\slide{Apache Tomcat Null Byte sårbarhed}

\hlkimage{20cm}{tomcat-exploit-bid-6721.png}

\begin{list1}
\item BID 6721 Apache Tomcat Null Byte Directory/File Disclosure Vulnerability
\item \link{http://www.securityfocus.com/bid/6721/}
\item CVE-2003-0042
\end{list1}


\slide{Apache Tomcat vulnerability -- vulnerable 3.3.1}

\hlkimage{25cm}{tomcat-exploit-bid-6721-result.png}

\centerline{Vulnerable version of Tomcat}

\slide{Apache Tomcat vulnerability - updated Tomcat 5.5.20}

\hlkimage{25cm}{tomcat-exploit-bid-6721-non-result.png}

\centerline{after \emph{upgrade} server is not vulnerable}



\slide{The Exploit Database - todays buffer overflow and exploits}

\hlkimage{13cm}{exploit-db.png}

\centerline{\link{http://www.exploit-db.com/}}


\slide{OWASP top ten}

\hlkimage{16cm}{owasp.jpg}

\begin{quote}
The OWASP Top Ten provides a minimum standard for web application
security. The OWASP Top Ten represents a broad consensus about what
the most critical web application security flaws are.
\end{quote}

\begin{list1}
\item The Open Worldwide (Web) Application Security Project (OWASP)
\item OWASP Top 10 is used across computer security
\item \link{http://www.owasp.org}
\end{list1}





\slide{Proof of concept programs exist - god or bad?}

\centerline{Some of the tools released shortly after Heartbleed announcement}
\begin{list2}
\item \link{https://github.com/FiloSottile/Heartbleed} tool i Go\\
site \link{http://filippo.io/Heartbleed/}
\item \link{https://github.com/titanous/heartbleeder} tool i Go
\item \link{http://s3.jspenguin.org/ssltest.py} PoC
\item \link{https://gist.github.com/takeshixx/10107280} test tool med STARTTLS support
\item \link{http://possible.lv/tools/hb/} test site
\item \link{https://twitter.com/richinseattle/status/453717235379355649} Practical Heartbleed attack against session keys links til, \link{https://www.mattslifebytes.com/?p=533} og "Fully automated here "\\ \link{https://www.michael-p-davis.com/using-heartbleed-for-hijacking-user-sessions/}

\item Metasploit er også opdateret på master repo\\ \link{https://twitter.com/firefart/status/453758091658792960}\\
\link{https://github.com/rapid7/metasploit-framework/blob/master/modules/auxiliary/scanner/ssl/openssl_heartbleed.rb}
\end{list2}



\slide{Nikto web scanner}

\hlkimage{2cm}{nikto.jpg}

\begin{quote}\small
{\bf Description}
Nikto is an Open Source (GPL) web server scanner which performs
comprehensive tests against web servers for multiple items, including
over 3200 potentially dangerous files/CGIs, versions on over 625
servers, and version specific problems on over 230 servers. Scan items
and plugins are frequently updated and can be automatically updated
(if desired).
\end{quote}

\begin{list1}
\item Quick to run, checks quite a few things. I still use and find stuff with Nikto -- and you can expand it easily
\item \verb+nikto -host 127.0.0.1 -port 8080+
\item Nikto web server scanner originally from \link{http://cirt.net/nikto2}
\end{list1}


\slide{Demo: Nikto}

\begin{alltt}\footnotesize
Script started on Tue Nov  7 17:43:54 2006
$  nikto -host 127.0.0.1 -port 8080 ^M
---------------------------------------------------------------------------
- Nikto 1.35/1.34     -     www.cirt.net
+ Target IP:       127.0.0.1
+ Target Hostname: localhost.pentest.dk
+ Target Port:     8080
+ Start Time:      Tue Nov  7 17:43:59 2006
...
+ /examples/ - Directory indexing enabled, also default JSP examples. (GET)
+ /examples/jsp/snp/snoop.jsp - Displays information about page
retrievals, including other users. (GET)
+ /examples/servlets/index.html - Apache Tomcat default JSP pages
present. (GET)
\end{alltt}
%$

\begin{list1}
\item It is still very usefull, and free
\item Demo nikto -- should find a few things, at least server header
\item When something is \emph{found} should be verified could be a false positive
\end{list1}


\slide{Attack proxies: webscarab og Zap}

\hlkimage{5cm}{webscarab_logo.png}

\begin{list1}
\item Proxies, men inkluderer fuzzing og session id undersøgelse
\item Webscarab JAVA framework til udvikling af værktøjer til HTTP og HTTPS undersøgelse\\
\link{https://www.owasp.org/index.php/Category:OWASP_WebScarab_Project}

\item OWASP anbefaler Zed Attack Proxy (ZAP) idag\\
\link{https://www.owasp.org/index.php/OWASP_Zed_Attack_Proxy_Project}\\
\link{https://code.google.com/p/zap-extensions/}
\end{list1}



\slide{Zed Attack Proxy (ZAP)}

\begin{quote}
The Zed Attack Proxy (ZAP) is an easy to use integrated penetration testing tool for finding vulnerabilities in web applications.

It is designed to be used by people with a wide range of security experience and as such is ideal for developers and functional testers who are new to penetration testing.

ZAP provides automated scanners as well as a set of tools that allow you to find security vulnerabilities manually.
\end{quote}

Source: ZAP homepage \link{https://wiki.owasp.org/index.php/OWASP_Zed_Attack_Proxy_Project} and
\url{https://www.zaproxy.org/}

\slide{Mini proxy: Tamper Data}

\hlkimage{16cm}{tamper-data.png}

Add-on for Firefox catch and modify data before it is sent to server\\
\link{https://addons.mozilla.org/en-US/firefox/addon/tamper-data/}




\slide{Burpsuite}

\begin{quote}
Burp Suite is an integrated platform for performing security testing of web applications. Its various tools work seamlessly together to support the entire testing process, from initial mapping and analysis of an application's attack surface, through to finding and exploiting security vulnerabilities.

Burp gives you full control, letting you combine advanced manual techniques with state-of-the-art automation, to make your work faster, more effective, and more fun.
\end{quote}

Burp suite contains lots of functionality proxy, spider, scanner and can be extended with Java, Python programs

\link{http://portswigger.net/burp/}\\
\link{https://pro.portswigger.net/bappstore/}


\slide{Skipfish}

\hlkimage{12cm}{skipfish-screen.png}

Source: Michal Zalewski \link{http://code.google.com/p/skipfish/}

More scanners:\\
{\small \link{https://www.owasp.org/index.php/Category:Vulnerability_Scanning_Tools}}

% Udvide med flere?
% http://resources.infosecinstitute.com/14-popular-web-application-vulnerability-scanners/
% https://www.owasp.org/index.php/Category:Vulnerability_Scanning_Tools


\slide{Session IDs}

\begin{list2}
\item Session IDs tie the user with the state on the server
\item Must be randomly assigned, otherwise an attacker can guess a valid ID
\item Common problems, time based or predictable in some way
\item Check code for generating IDs or measure - Phase Space Analysis
\end{list2}

\slide{10. Cross-Site Scripting (XSS)}

\begin{quote}\small
  Cross-Site Scripting (XSS) vulnerabilities are some of the most common vulnerabili‐
  ties throughout the internet, and have appeared as a direct response to the increasing
  amount of user interaction in today’s web applications.
  At its core, an XSS attack functions by taking advantage of the fact that web applica‐
  tions execute scripts on users’ browsers. Any type of dynamically created script that is
  executed puts a web application at risk if the script being executed can be contamina‐
  ted or modified in any way—in particular by an end user.
\end{quote}
Source: \emph{Web Application Security}, Andrew Hoffman, 2020, ISBN: 9781492053118

XSS attacks are categorized a number of ways, with the big three being:
\begin{list2}
\item Stored (the code is stored on a database prior to execution)
\item Reflected (the code is not stored in a database, but reflected by a server)
\item DOM-based (code is both stored and executed in the browser)
\end{list2}

Note: attacks the user, his/her data mostly.

\slide{Cross-site scripting}

\begin{list1}
\item When logged into a site, we have a session identifier -- being presented with each HTTP request
\item If an attacker can \emph{activate} javascript that access and re-send this to another site we have a cross-site scripting attack
\item We often check using some parameter with code, like this:
\begin{alltt}\footnotesize
<A HREF="http://example.com/comment.cgi?
mycomment=<SCRIPT>malicious code</SCRIPT>
">Click here</A>
\end{alltt}
\item If this code is returned, and active, as part of the returned page and HTML we have a cross-site scripting possibility
\item This code is executed in the user browser, with the same permissions as the user
\end{list1}

\slide{11. Cross-Site Request Forgery (CSRF)}

\begin{quote}
Sometimes we already know an API endpoint exists that would allow us to perform an operation we wish to perform, but we do not have access to that endpoint because it requires privileged access (e.g., an admin account).

In this chapter, we will be building Cross-Site Request Forgery (CSRF) exploits that result in an admin or privileged account performing an operation on our behalf rather than using a JavaScript code snippet.

CSRF attacks take advantage of the way browsers operate and the trust relationship between a website and the browser. By finding API calls that rely on this relationship to ensure security—but yield too much trust to the browser—we can craft links and forms that with a little bit of effort can cause a user to make requests on his or her own behalf—unknown to the user generating the request.
\end{quote}

\begin{list2}
  \item Often seen in small CPE routers in homes, if the user activates and evil link, their router might be reconfigured or taken over
\end{list2}


\slide{Configuration errors -- often overlooked}

\begin{list1}
\item Using the wrong program in the wrong places
\begin{list2}
\item When using a program, does it meet requirements
\item Is it a suitable program for the environment
\item Do we know how to maintain, update, secure this
\end{list2}
\item What happens if you put a generic command shell in the cgi-bin folder, executable programs
\item We again and again see people do things like that
\item People also often keep running things on HTTP, even with password logins
\end{list1}

\slide{PHP shell escapes}

\begin{list1}
\item PHP in the old days was horrible insecure. It was possible to include files from HTTP urls, file URL open etc. Often user input was copied directly into a shell, which is why we also spend some time talking about shells and Linux/Unix
\end{list1}
\begin{alltt}
<pre>
<?php passthru("{\bf netstat -an && ifconfig -a}"); ?>
</pre>
\end{alltt}
\begin{list1}
\item Other tools have similar shell escapes:
\begin{list2}
\item Perl: \verb+print `/usr/bin/finger $input{'command'}`;+
\item UNIX shell: \verb+`echo hello`+
\item Microsoft SQL: \verb+exec master..xp_cmdshell 'net user test testpass /ADD'+
\end{list2}
\end{list1}
%$

\vskip 1 cm

\centerline{\bf Result: web server send back data through HTTP/HTTPS}


\slide{SQL injection}

\begin{list1}
\item SQL Injection FAQ
\link{http://www.sqlsecurity.com}:
\item \begin{alltt}\small
Set myRecordset = myConnection.execute
("SELECT * FROM myTable
WHERE someText ='" & request.form("inputdata") & "'")
med input: ' exec master..xp_cmdshell 'net user test testpass /ADD' --
\end{alltt}
\item modtager og udfører serveren:
\item \begin{alltt}
SELECT * FROM myTable
WHERE someText ='' exec master..xp_cmdshell
'net user test testpass /ADD'--'
\end{alltt}
\item -- er kommentar i SQL
\item Derefter er det kun platformen, OS, og rettighederne der afgør problemets omfang
\end{list1}

\centerline{Dette er den klassiske SQL injection mod Windows, fra ~2000}



\slide{Sqlmap}

\begin{quote}\small
sqlmap is an open source penetration testing tool that automates the process of detecting and exploiting SQL injection flaws and taking over of database servers. It comes with a powerful detection engine, many niche features for the ultimate penetration tester and a broad range of switches lasting from database fingerprinting, over data fetching from the database, to accessing the underlying file system and executing commands on the operating system via out-of-band connections.

Features
\end{quote}

\begin{list1}
\item Automatic SQL injection and database takeover tool
\link{http://sqlmap.org/}
\end{list1}


\slide{sqlmap features}

\hlkimage{15cm}{sqlmap-features-1.png}

Not a complete list!

Source: \link{http://sqlmap.org/}

\slide{HTML Entity Encoding -- escaping characters}

%\hlkimage{}{}

\begin{quote}
Another preventative measure that can be applied is to perform HTML entity escap‐
ing on all HTML tags present in user-supplied data. Entity encoding allows you to
specify characters to be displayed in the browser, but in a way that they cannot be
interpreted as JavaScript.
The “big five” for entity encoding are shown in Table 22-1.

\begin{list2}
\item \verb+&+ \verb+&amp;+ ampersand
\item \verb+<+ \verb+&lt;+ -- less than
\item \verb+>+ \verb+&gt;+ -- greater than
\item \verb+"+ \verb+&#034;+
\item \verb+'+ \verb+&#039;+
\end{list2}
\end{quote}

Burp has encoding/decoding built-in for checking, and abusing this.

Recommend using facilities in your chosen language, like
\link{https://www.php.net/manual/en/function.htmlspecialchars.php} and
\link{https://www.php.net/manual/en/function.htmlentities.php}


\slide{Content Security Policy (CSP) for XSS Prevention}

%\hlkimage{}{}

\begin{quote}
The {\bf CSP is a security configuration tool that is supported by all major browsers.} It provides settings that a developer can take advantage of to either {\bf relax or harden security rules regarding what type of code can run inside your application.}

CSP protections come in several forms, including {\bf what external scripts can be loaded, where they can be loaded, and what DOM APIs are allowed to execute the script}.

Let’s evaluate some CSP configurations that aid in mitigating XSS risk.
\end{quote}

\begin{list2}
\item Very highly recommended, easy to check, quite easy to implement
\item Tools available for you to generate even complex CSPs
\item and lots of references:\\
\link{https://developers.google.com/web/fundamentals/security/csp}\\
\link{https://developer.mozilla.org/en-US/docs/Web/HTTP/CSP}
\end{list2}


\slide{Implementing CSP}

%\hlkimage{}{}

\begin{quote}
To enable CSP, you need to configure your web server to return the Content-Security-Policy HTTP header. (Sometimes you may see mentions of the X-Content-Security-Policy header, but that's an older version and you don't need to specify it anymore.)

Alternatively, the <meta> element can be used to configure a policy, for example:

\begin{alltt}
<meta http-equiv="Content-Security-Policy"
      content="default-src 'self'; img-src https://*; child-src 'none';">
\end{alltt}
\end{quote}
Source: \link{https://developer.mozilla.org/en-US/docs/Web/HTTP/CSP}

\begin{list2}
\item I usually see this as part of the headers
\item Other headers can also be checked using Mozilla Observatory
\end{list2}

\slide{Scanning for HTTP settings}

%\hlkimage{}{}

Nmap can also report these settings:
\begin{alltt}
|     X-XSS-Protection: 1;mode=block
|     Content-Security-Policy: script-src 'self' 'unsafe-inline' 'unsafe-eval'
|     X-Content-Type-Options: nosniff
\end{alltt}

\begin{list2}
\item If you have many sites, Nmap can also report this
\item Use the documentation from Mozilla Observatory for full explainations about these settings
\end{list2}


\slide{Header Verification -- first check}

%\hlkimage{}{}

\begin{quote}
Because the {\bf origin of many CSRF requests is separate from your web application}, we can mitigate the risk of CSRF attacks by checking the origin of the request. In the world of HTTP, there are two headers we are interested in when checking the origin of a request: {\bf referer and origin . These headers are important because they cannot be modified programmatically with JavaScript in all major browsers}. As such, a properly implemented browser’s referer or origin header has a low chance of being spoofed.
\end{quote}

\begin{list2}
    \item Origin header
The origin header is only sent on HTTP POST requests. It is a simple header
that indicates where a request originated from. Unlike referer , this header is
247also present on HTTPS requests, in addition to HTTP requests.

\item Referer header
The referer header is set on all requests, and also indicates where a request ori‐
ginated from. The only time this header is not present is when the referring link
has the attribute rel=noreferer set.
\end{list2}

These headers are a first line of defense, but there is a case where they will fail.

\slide{Better: CSRF Tokens}

\hlkimage{16cm}{was-csrf-token.png }
Source: \emph{Web Application Security}, Andrew Hoffman, 2020, ISBN: 9781492053118

\begin{list2}
\item Where do you come from, where do your browser go
\end{list2}

\slide{Anti-CRSF Coding Best Practices}

%\hlkimage{}{}

\begin{quote}
There are many methods of eliminating or mitigating CRSF risk in your web applica‐
tion that start at the code or design phase.
Several of the most effective methods are:
\begin{list2}
\item Refactoring to stateless GET requests
\item Implementation of application-wide CSRF defenses
\item Introduction of request-checking middleware
\end{list2}
Implementing these simple defenses in your web application will dramatically reduce
the risk of falling prey to CSRF-targeting hackers.
\end{quote}




\slide{Example CSRF Protection Django}

%\hlkimage{}{}

\begin{quote}{\bf
Cross Site Request Forgery protection}\\
The CSRF middleware and template tag provides easy-to-use protection against Cross Site Request Forgeries. This type of attack occurs when a malicious website contains a link, a form button or some JavaScript that is intended to perform some action on your website, using the credentials of a logged-in user who visits the malicious site in their browser. A related type of attack, ‘login CSRF’, where an attacking site tricks a user’s browser into logging into a site with someone else’s credentials, is also covered.

The {\bf first defense against CSRF attacks is to ensure that GET requests (and other ‘safe’ methods, as defined by RFC 7231\#section-4.2.1) are side effect free}. Requests via ‘unsafe’ methods, such as POST, PUT, and DELETE, can then be protected by following the steps below.
\end{quote}
Source: \link{https://docs.djangoproject.com/en/4.1/ref/csrf/}

\begin{list2}
\item Many existing options can help with CSRF protection, so make sure to include this in your architecture and planning
\end{list2}



\slide{Prepared Statements -- what it is}

\begin{quote}
One development that most SQL implementations have begun to support is prepared statements. Prepared statements reduce a significant amount of risk when using user-supplied data in an SQL query. Beyond this, prepared statements are very easy to learn and make debugging SQL queries much easier.

Prepared statements work by compiling the query first, with placeholder values for variables. These are known as bind variables, but are often just referred to as placeholder variables. After compiling the query, the placeholders are replaced with the values provided by the developer. As a result of this two-step process, the intention of the query is set before any user-submitted data is considered.
\end{quote}
Source: \emph{Web Application Security} page 261

\begin{list2}
\item Doing the above should result in applications which are secure by design
\item Adhering to the best security principles
\item Implementing security from design to deployment ensure good security
\end{list2}

\slide{Prepared Statements -- what it provides and how to use}

In MySQL, prepared statements are quite simple:
\begin{minted}[fontsize=\small]{sql}
PREPARE q FROM 'SELECT name, barCode from products WHERE price <= ?';
SET @price = 12;
EXECUTE q USING @price;
DEALLOCATE PREPARE q;
\end{minted}

\begin{quote}
With a prepared statement, because the intention is set in stone prior to the user-submitted data being presented to the SQL interpreter, the query itself cannot change. {\bf This means that a SELECT operation against users cannot be escaped and modified into a DELETE operation by any means.} An {\bf additional query cannot occur after the SELECT operation} if the user escapes the original query and begins a new one. {\bf Prepared statements eliminate most SQL injection risk} and are supported by almost every major SQL database: MySQL, Oracle, PostgreSQL, Microsoft SQL Server, etc.
\end{quote}
Source: \emph{Web Application Security} page 262


\slide{Exploiting insecure deserialization vulnerabilities}

%\hlkimage{}{}

\begin{quote}
Exploiting insecure deserialization vulnerabilities
In this section, we'll teach you how to exploit some common scenarios using examples from PHP, Ruby, and Java deserialization. We hope to demonstrate how exploiting insecure deserialization is actually much easier than many people believe. This is even the case during blackbox testing if you are able to use pre-built gadget chains.
\end{quote}

\link{https://portswigger.net/web-security/deserialization/exploiting}


\begin{list2}
    \item OWASP has multiple pages about these problems:\\
    \link{https://owasp.org/www-community/vulnerabilities/Deserialization_of_untrusted_data}
\item Note:
\begin{quote}{\bf
Platform}\\
Languages: C, C++, Java, Python, Ruby (and probably others)\\
Operating platforms: Any
\end{quote}
\end{list2}

\slide{Avoiding problems: Development standards }

\begin{list1}
\item What can we do to avoid the problems
\item Identify what technologies are used
\item Standardize on a selected set of technologies, fewer tools, libraries and languages
\item Rules for development:
\begin{list2}
\item Quality assurance, Rules for allowed functions, algorithms, Rules for SQL statements, only allow prepared statements
\end{list2}
\item Having a focus on the number of products and technologies used we can gain more experience with these and have less errors, higher quality is often more secure
\item OWASP has many different guides and examples, like Cheat sheets\\
\link{https://www.owasp.org/index.php/PHP_Security_Cheat_Sheet}
\end{list1}


\slide{Secure Software Development Lifecycle}

\begin{list2}
\item SSDL represents a structured approach toward implementing and performing secure software development
\item Security issues evaluated and addressed early
\item During business analysis
\item through requirements phase
\item during design and implementation
\end{list2}

Source: The Art of Software Security Testing Identifying Software Security Flaws
Chris Wysopal ISBN: 9780321304865

\slide{OWASP Web Security Testing Guide}

\begin{quote}
The Web Security Testing Guide (WSTG) Project produces the premier cybersecurity
testing resource for web application developers and security professionals.

The WSTG is a comprehensive guide to testing the security of web applications and
web services. Created by the collaborative efforts of cybersecurity professionals
and dedicated volunteers, the WSTG provides a framework of best practices used by
penetration testers and organizations all over the world.
\end{quote}

\begin{list2}
\item Project from OWASP:\\
\link{https://owasp.org/www-project-web-security-testing-guide/}
\item Use the Tab \emph{Release Versions} to download version 4.2 in PDF
\item Also available as a checklist \verb+OWASPv4_Checklist.xlsx+
\end{list2}


\slide{Security in the Software Development Life Cycle (SDLC)}

\hlkimage{6cm}{OWASP-SDLC.jpg}

\begin{quote}\small{\bf
When to Test?}\\
Most people today don’t test software until it has already been created and is in the deployment phase of its life cycle (i.e., code has been created and instantiated into a working web application). This is generally a very ineffective and cost-prohibitive practice. One of the best methods to prevent security bugs from appearing in production applications is to improve the Software Development Life Cycle (SDLC) by including security in each of its phases.
\end{quote}
Source: OWASP Web Security Testing Guide


\slide{Putting it all together}

\hlkimage{16cm}{securing-devops.png}
Source:
\emph{Securing DevOps: Security in the cloud} by Julien Vehent
Manning 2018, 384 pages


\slide{Conclusion Pentest web attacks}

\begin{quote}\Large
Install, configure, monitor
\end{quote}

\hlkimage{5cm}{network-layers-2022.pdf}
~
\begin{list2}
\item You really should try testing, and investigate your existing devices
all of them
\item This is just one small part of your security posture
\item Harden servers, networks, configurations
\item Also block outgoing connections -- shellshock, log4shell, ... next CVE coming

\end{list2}

\myquestionspage


\end{document}
