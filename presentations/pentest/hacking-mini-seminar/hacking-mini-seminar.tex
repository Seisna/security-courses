\documentclass[20pt,landscape,a4paper,footrule]{foils}
\usepackage{zencurity-slides}

%\externaldocument{unix-audit-security-oevelser}
\externaldocument{\jobname-exercises}

% Hacking workshop - Kom og hack IT-systemer
% Kl 9 – 12


% Formålet med sessionen er at demonstrere hvordan hackere arbejder, og hvordan samme teknikker kan bruges til at verificere sikkerheden på netværk og systemer.


% Medbring PC
% Der vil være opgaver hvor det forventes at deltagerene er aktive for at få det fulde udbytte. Så medbring din PC.
% De indledende opgaver kan udføres med en almindelig browser, som Firefox der bør være installeret før dagen. Til de mest tekniske vil det største udbytte opnås med en virtuel maskine hvor Kali Linux kan installeres. Dette kan gøres med virtualiseringsprogrammer som VMware, Virtual Box - men spørg lige IT-afdelingen hvad de anbefaler.

% Vi anbefaler at der arbejdes i hold på 2 personer, men opgaverne kan udføres alene. Der behøves kun virtualisering på een PC per hold.

% IT-sikkerhed
% Der vil være et lukket netværk hvor opgaverne udføres mod specielt opsatte sårbare systemer. Der vil blive arbejdet med sikkerhedsprogrammer og hackerprogrammer, men ikke malware og virus! PC kan således trygt medbringes.

% Basic things that we need are below
\begin{document}
\selectlanguage{danish}

\mytitlepage{Hacking Seminar}


\slide{Formålet }

\vskip 2 cm

\hlkimage{5cm}{dont-panic.png}
\centerline{\color{titlecolor}\LARGE Don't Panic!}

\begin{list1}
\item Hvad er truslerne, hvad er nogle trends indenfor sikkerhed
\item Hvordan foregår hacking
\item Hvad kan vi gøre for at dæmme op for truslerne
\end{list1}

\vskip 1cm
\centerline{Jeg bruger Kali 2.0 Linux hackerplatformen som eksempel}

\slide{Demo: Metasploit Armitage }

\hlkimage{16cm}{armitage-overview.png}

\slide{Metasploit and Armitage Still rocking the internet}


\hlkimage{14cm}{metasploit-about.png}

\begin{list1}
\item Udviklingsværktøjerne til exploits er i dag meget raffinerede!
\item \link{http://www.metasploit.com/}
\item Armitage GUI fast and easy hacking for Metasploit\\
\link{http://www.fastandeasyhacking.com/}
\item Kursus Metasploit Unleashed\\
\link{http://www.offensive-security.com/metasploit-unleashed/Main_Page}
\item Bog: Metasploit: The Penetration Tester's Guide, No Starch Press\\
ISBN-10: 159327288X
\end{list1}


\slide{Hackerværktøjer}

\begin{list1}
\item \emph{Improving the Security of Your Site by Breaking Into it} af
Dan Farmer og Wietse Venema i 1993
\item De udgav i 1995 så en softwarepakke med navnet SATAN
\emph{Security Administrator Tool for Analyzing Networks}
\item De forårsagede en del panik og furore, alle kan hacke, verden bryder sammen

\vskip 1cm
\begin{quote}
We realize that SATAN is a two-edged sword -- like
many tools, it can be used for good and for evil
purposes. We also realize that intruders (including
wannabees) have much more capable (read intrusive)
tools than offered with SATAN.
\end{quote}
\end{list1}

\vskip 1cm
Kilde:
\link{http://www.fish2.com/security/admin-guide-to-cracking.html}




\slide{Security Breaches: Ashley Madison}


\begin{quote}
When hackers swiped an estimated 36 million accounts associated with AshleyMadison.com, a site which helps married people cheat on their partners, there was a rush to find out what had been stolen.
\end{quote}


\begin{list2}
\item Ashley Madison, inkl password hacking
\item Popular passwords
123456,
,password
,12345
,qwerty
,12345678,football
\item Also hacking kills? Suicides and family break-ups?
\item accidental outing of gays/ Gay persecution, death?
\end{list2}

Source:\\
{\footnotesize\link{http://www.zdnet.com/article/these-are-the-worst-passwords-from-the-ashley-madison-hack/}}

\slide{Security Breaches: Top 20 Ashley Madison Passwords}

\begin{quote}
Previously thought to be impossible thanks to the slow pace and high stress it puts on a computer's CPU, the CynoSure Prime group managed to crack over 11 million passwords from the total of 36 million, mainly due to a programming error in how the passwords were hashed.
\end{quote}


\begin{list2}
\item Main passwords hashed with slow (good) bcrypt  algoritm, but hackers found tokens hashed with MD5\\
\link{http://cynosureprime.blogspot.dk/2015/09/how-we-cracked-millions-of-ashley.html}
\item Also check out Twitter Mark Burnett @m8urnett and his 10 million password dump\\
\link{http://wpengine.com/unmasked/}
\item Systems exist which can try 135 billion MD5 hashes PER SECOND with 8 GPUs
\end{list2}

Source: Catalin Cimpanu, Softpedia 15 September 2015\\
{\footnotesize\link{http://news.softpedia.com/news/top-20-ashely-madison-passwords-491799.shtml}}


\slide{Ransomware web ransomware}

\begin{quote}
Tre Randomware familier rammer Danmark
Ransomware, som rammer Danmark er groft fordelt i tre malware familier: Cryptowall, CTB-Locker og FileCoder. De spredes via to centrale metoder: spamkampagner med vedhæftede filer og ved brug af "drive-by" angreb.
\end{quote}

Source:  https://www.csis.dk/da/csis/news/4676/

Andre kilder:\\
The World Is Now Richer with 21 Million New Types of Malware, 230,000 Each Day\\
{\tiny\link{http://news.softpedia.com/news/the-world-is-now-richer-with-21-million-new-types-of-malware-230-000-each-day-491947.shtml}}


\slide{Most vulnerable operating systems in 2014}

\hlkimage{18cm}{GFI-vulns-2014-OS-chart.jpg}

\begin{quote}
An average of 19 vulnerabilities per day were reported in 2014, according to the data from the National Vulnerability Database (NVD).
\end{quote}

Source:\\
{\footnotesize
\link{http://www.gfi.com/blog/most-vulnerable-operating-systems-and-applications-in-2014/}}


\slide{Most vulnerable applications in 2014}

\hlkimage{16cm}{gfi-vulns-application-chart.jpg}

\begin{quote}\small
Not surprisingly at all, web browsers continue to have the most security vulnerabilities because they are a popular gateway to access a server and to spread malware on the clients.
\end{quote}

Source:\\
{\footnotesize
\link{http://www.gfi.com/blog/most-vulnerable-operating-systems-and-applications-in-2014/}}


\slide{Release of vuln information without patch}

\begin{list1}
\item Google project Zero
\item Follow a
"90-day disclosure deadline statement... which in this instance has passed."
\item Released Zero-day information about Microsoft and Apple OS X vulnerabilities
\item MS patched some in \emph{first Patch Tuesday of 2015, which came out on Jan. 13.}
\end{list1}

Sources:\\
{\tiny
\link{http://googleonlinesecurity.blogspot.fr/2014/07/announcing-project-zero.html}\\
\link{http://searchsecurity.techtarget.com/news/2240238448/Googles-Project-Zero-reveals-another-Windows-zero-day-vulnerability}\\
\link{http://www.engadget.com/2015/01/02/google-posts-unpatched-microsoft-bug/}\\
\link{http://www.eweek.com/security/google-project-zero-continues-its-microsoft-zero-day-assault.html}\\
\link{http://www.zdnet.com/article/googles-project-zero-reveals-three-apple-os-x-zero-day-vulnerabilities/}
}

Trend with more vulnerabilities per day, and\\
even big vendors cannot react quickly enough


\slide{Samba remote code execution}


\begin{alltt}\small
  ===========================================================
  == Subject:     Unexpected code execution in smbd.
  ==
  == CVE ID#:     CVE-2015-0240
  ==
  == Versions:    Samba 3.5.0 to 4.2.0rc4
  ==
  == Summary:     Unauthenticated code execution attack on
  ==		smbd file services.
  ==
  ===========================================================
\end{alltt}

\centerline{Great, even our old tools still has multiple bugs}

Source:\\
\link{https://www.samba.org/samba/security/CVE-2015-0240}


\slide{DNS attacks, February 2015 - ongoing for +10 years!}

\hlkimage{15cm}{krebs-lenovo-google-dns-hack.png}
\begin{list1}
%\item DNS is the Domain Name System, \link{https://en.wikipedia.org/wiki/Dns}
\item DNS insecurity has huge impact on your security!
\item Most are denial of service, by may create Mitm or confidentiality concerns
\item Select DNS providers with care
\end{list1}


Sources:\\
{\tiny
\link{https://krebsonsecurity.com/2015/02/webnic-registrar-blamed-for-hijack-of-lenovo-google-domains/}\\
\link{http://www.version2.dk/artikel/google-og-lenovo-defaced-som-foelge-af-overset-sikkerhedsproblemstilling-91295}}



\slide{Example, Using tools similar to PacketQ}

\hlkimage{20cm}{using-packetq.png}

Are you using your brain and existing tools? Building own specialised tools?\\
Discussion: bridging the gaps between Devops and Security? Good thing, easy?

{\footnotesize
\link{http://securityblog.switch.ch/2013/01/22/using-packetq/}\\
\link{http://jpmens.net/2013/05/27/server-agnostic-logging-of-dns-queries-responses/}
}

\slide{Storing query logs, old school or needed?}

\hlkimage{7cm}{bro-sample-ssl-scripts.png}

Looking at DNS PacketQ it was an Older link, but thinking the time is now for doing:

\begin{list2}
\item DNS query logs, keep it for at least a week? - with DSC and PacketQ
\item SSL/TLS full logs over sessions, certs, keys - with Bro/Suricata\\
\link{https://www.zeek.org/sphinx-git/script-reference/scripts.html}
\item Log and search with Elasticsearch?\\
\link{https://www.elastic.co/guide/en/elasticsearch/guide/current/index.html}
\item Even netflow session logging, full 1:1 - NFSen, Suricata Flow mode?
%\item Moloch \link{https://github.com/aol/moloch}
\end{list2}

%\centerline{Why go to this extreme, storing information about past sessions?}

\slide{February 2015: Finding infected sources}

\begin{quote}
"We were contacted by a client to help with their incident response in tracking down an
infection on a clients machine with the new CTB-Locker ransomware (Curve-Tor-Bitcoin Locker)
aka Critroni which had no signatures available at the time of infection for this variant.

LANGuardian includes a file share activity monitoring module which provided a very
detailed forensic analysis of the ransomware and the paths it had taken in order to
encrypt the clients system and also the fileserver in which it was connected to, the
initial infection came from the opening of an attachment in an e-mail."
\end{quote}

\vskip 1cm

\centerline{It has become critical to identify vulnerable or infected ASAP!}

Source:
{\tiny\link{https://www.netfort.com/support-team-stories-detecting-the-source-of-ransomware/}}

Dont forget Suricata \link{http://suricata-ids.org/} and Security Onion\\ {\small\link{https://github.com/Security-Onion-Solutions/security-onion/wiki/Installation}}

\slide{Kibana 4}

\hlkimage{14cm}{kibanascreenshothomepagebannerbigger.jpg}

\centerline{Highly recommended for a lot of data visualisation}

Non-programmers can create, save, and share dashboards

Source:
\link{https://www.elastic.co/products/kibana}



\slide{DDoS in 2015}

\hlkimage{20cm}{arbor-networks-10-year-ddos.png}

\centerline{Expect amplification attacks and 3-digit attacks for some years}

Source:\\
Arbor Networks: Worldwide Infrastructure Security Report, Volume X January 2015



\slide{Brug hackerværktøjer!}

\hlkimage{12cm}{nfsen-udp-flood.png}

\centerline{An extra 100k packets per second from this netflow source (source is a router)}

\begin{list1}
\item Hackerværktøjer -- bruger I dem? -- efter dette kursus gør I
\item Portscannere kan afsløre huller i forsvaret
\item I vil kunne finde mange potentielle problemer proaktivt ved
  regelmæssig brug

\end{list1}



\slide{Detecting DDoS example tool Nfsen}

\hlkimage{15cm}{nfsen-ddos-profile-1.png}

We created a DDoS profile with the common types.

We can ask RRDtools about max, average etc.
\begin{alltt}\small
rrdtool graph x -s -24h DEF:v=DDoS/mx-cph-01.rrd:packets:MAX
VDEF:vm=v,MAXIMUM PRINT:vm:%.lf
\end{alltt}




\slide{DDoS traffic before filtering}
\hlkimage{26cm}{ddos-before-filtering}

\centerline{Only two links shown, at least 3Gbit incoming for this single IP}

\slide{DDoS traffic after filtering}
\hlkimage{18cm}{ddos-after-filtering}
\centerline{Link toward server (next level firewall actually) about ~350Mbit outgoing}

\begin{list1}
\item Problem: We receive unauthenticated chaotic traffic

\item Solution: Discard early, discard on edge, reduce noise

\item Only use CPU resources for potentially real traffic
\end{list1}

\centerline{Single firewall layer typically cannot cope!}

\slide{Defense in depth - multiple layers of security}

\hlkimage{17cm}{network-layers-1.pdf}

% 802.1q
\slide{VLAN Virtual LAN}

\hlkimage{10cm}{vlan-portbased.pdf}

\begin{list1}
\item Nogle switche tillader at man opdeler portene
\item Denne opdeling kaldes VLAN og portbaseret er det mest simple
\item Port 1-4 er et LAN
\item De resterende er et andet LAN
\item Data skal omkring en firewall eller en router for at krydse fra VLAN1 til VLAN2
\end{list1}

\slide{IEEE 802.1q}

\hlkimage{19cm}{vlan-8021q.pdf}

\begin{list1}
\item Nogle switche tillader konfiguration med 802.1q VLAN tagging på Ethernet niveau
\item Data skal omkring en firewall eller en router for at krydse fra VLAN1 til VLAN2
\item VLAN trunking giver mulighed for at dele VLANs ud på flere switches
\item Der findes administrationsværktøjer der letter dette arbejde: \\
Prøv evt. openNAC, FreeNAC, Cisco VMPS
\end{list1}

\slide{Focus for the near future}

\begin{list2}
\item Walk through your infrastructure\\
get a detailed view of data, flows, protocols, bandwidth, ports and services

\item Create a list of critical phone numbers and contacts, enter it in your phone
\item Automate updates for both clients and servers, goal update everything in hours
\item Learn to run Nmap and Metasploit scripts - identify vulnerable servers
\end{list2}

\vskip 2cm
\centerline{consider the fact we have multiple overlapping critical security incidents now!}

\vskip 2cm
How many incidents can your organisation handle in parallel?

Can multiple people in your organisation initate updates?



\slide{Sceneskift - Hacking er magi}

\hlkimage{7cm}{wizard_in_blue_hat.png}

\vskip 1 cm

\centerline{Hacking ligner indimellem  magi}


\slide{Hacking er ikke magi}

\hlkimage{17cm}{ninjas.png}

\vskip 1 cm
\centerline{Hacking kræver blot lidt ninja-træning}

\slide{Movie:Kryptonite lock - old}

\hlkimage{16cm}{youtube-bic-lock.png}

\begin{list1}
\item Just search for: kryptonite lock bic pen
\item \link{https://www.youtube.com/watch?v=LahDQ2ZQ3e0}
\end{list1}



\slide{Hacking eksempel - det er ikke magi}

\begin{list1}
\item MAC filtrering på trådløse netværk
\item Alle netkort har en MAC adresse - BRÆNDT ind i kortet fra fabrikken
\item Mange trådløse Access Points kan filtrere MAC adresser
\item Kun kort som er på listen over godkendte adresser tillades adgang til netværket
\pause
\item Det virker dog ikke \smiley
\item De fleste netkort tillader at man overskriver denne adresse midlertidigt
\item Derudover har der ofte været fejl i implementeringen af MAC filtrering
\end{list1}


\slide{MAC filtrering}

\hlkimage{15cm}{stupid-security.jpg}



\slide{Hvad skal der ske?}

\begin{list1}
\item Tænk som en hacker
\item Rekognoscering
\begin{list2}
\item ping sweep, port scan
\item OS detection -- TCP/IP eller banner grab
\item Servicescan -- rpcinfo, netbios, ...
\item telnet/netcat interaktion med services
\end{list2}
\item Udnyttelse/afprøvning: Metasploit, Nikto, exploit programs
\item Oprydning/hærdning vises måske ikke, men I bør i praksis:
\begin{list2}
\item Lav en rapport
\item Ændre, forbedre og hærde systemer
\item Gennemgå rapporten, registrer ændringer
\item Opdater programmer, konfigurationer, arkitektur, osv.
\end{list2}
\item I skal jo også VISE andre at I gør noget ved sikkerheden.
\end{list1}


\slide{Hackerlab opsætning}

\hlkimage{10cm}{hacklab-1.png}

\begin{list2}
\item Hardware: en moderne laptop med CPU der kan bruge virtualiseting\\
Husk at slå virtualisering til i BIOS
\item Software: dit favoritoperativsystem, Windows, Mac, Linux
\item Virtualiseringssoftware: VMware, Virtual box, vælg selv
\item Hackersoftware: Kali som Virtual Machine \link{https://www.kali.org/}
\item Soft targets: Metasploitable, Windows 2000, Windows XP, ...
\end{list2}

\slide{Hackerværktøjer}
% måske til reference afsnit?
\hlkimage{3cm}{hackers_JOLIE+1995.jpg}

\begin{list2}
\item Alle bruger nogenlunde de samme værktøjer, se også \link{http://www.sectools.org/}
\item Portscanner Nmap, Nping -- tester porte, godt til firewall admins \link{https://nmap.org}
\item Generel sårbarhedsscanner Metasploit Framework \link{https://www.metasploit.com/}
\item Specialscannere, eksempelvis web sårbarhedsscanner -- eksempelvis Nikto, Skipfish
\item Specielle scannere -- wifi Aircrack-ng, web Burpsuite \link{http://portswigger.net/burp/}
\item Wireshark avanceret netværkssniffer -- \link{https://www.wireshark.org/}
\item og scripting, PowerShell, Unix shell, Perl, Python, Ruby, \ldots
\end{list2}

Billedet: Angelina Jolie fra Hackers 1995



\slide{Kali Linux the new backtrack}

\hlkimage{20cm}{kali-linux.png}

\begin{list1}
\item BackTrack -- \link{http://www.backtrack-linux.org}
\item Kali -- \link{https://www.kali.org/} version 2.0 netop udkommet!
\item Wireshark -- \link{https://www.wireshark.org} avanceret netværkssniffer
\end{list1}



\slide{OSI og Internet modellerne}

\hlkimage{14cm,angle=90}{images/compare-osi-ip.pdf}


\slide{Wireshark - grafisk pakkesniffer}

\hlkimage{20cm}{images/wireshark-website.png}

\centerline{\link{http://www.wireshark.org}}
\vskip 5mm
\centerline{både til Windows og UNIX}

\slide{Wireshark usage}
\hlkimage{16cm}{wireshark-http.png}

Wireshark: Filters, hexdump, protocol dissection, overview, coloring, advanced features

\demo{Wireshark}

\slide{The Exploit Database -- dagens buffer overflow}

\hlkimage{20cm}{exploit-db.png}

\centerline{\link{http://www.exploit-db.com/}}

\slide{Heartbleed CVE-2014-0160}

\hlkimage{22cm}{heartbleed-com.png}

Source: \link{http://heartbleed.com/}


\slide{Heartbleed hacking}

\begin{alltt}\footnotesize
  06b0: 2D 63 61 63 68 65 0D 0A 43 61 63 68 65 2D 43 6F  -cache..Cache-Co
  06c0: 6E 74 72 6F 6C 3A 20 6E 6F 2D 63 61 63 68 65 0D  ntrol: no-cache.
  06d0: 0A 0D 0A 61 63 74 69 6F 6E 3D 67 63 5F 69 6E 73  ...action=gc_ins
  06e0: 65 72 74 5F 6F 72 64 65 72 26 62 69 6C 6C 6E 6F  ert_order&billno
  06f0: 3D 50 5A 4B 31 31 30 31 26 70 61 79 6D 65 6E 74  =PZK1101&payment
  0700: 5F 69 64 3D 31 26 63 61 72 64 5F 6E 75 6D 62 65  _id=1&{\bf card_numbe}
  0710: XX XX XX XX XX XX XX XX XX XX XX XX XX XX XX XX  {\bf r=4060xxxx413xxx}
  0720: 39 36 26 63 61 72 64 5F 65 78 70 5F 6D 6F 6E 74  {\bf 96&card_exp_mont}
  0730: 68 3D 30 32 26 63 61 72 64 5F 65 78 70 5F 79 65  {\bf h=02&card_exp_ye}
  0740: 61 72 3D 31 37 26 63 61 72 64 5F 63 76 6E 3D 31  {\bf ar=17&card_cvn=1}
  0750: 30 39 F8 6C 1B E5 72 CA 61 4D 06 4E B3 54 BC DA  {\bf 09}.l..r.aM.N.T..
\end{alltt}

\begin{list2}
\item Obtained using Heartbleed proof of concepts - Gave full credit card details
\item "can XXX be exploited" - yes, clearly! PoCs ARE needed\\
without PoCs even Akamai wouldn't have repaired completely!
\item \link{https://github.com/rapid7/metasploit-framework/blob/master/modules/auxiliary/scanner/ssl/openssl_heartbleed.rb}
\end{list2}



\slide{Network mapping}

\hlkimage{23cm}{images/network-example.pdf}

\begin{list1}
\item Ved brug af traceroute og tilsvarende programmer kan man ofte
  udlede topologien i det netværk man undersøger
\end{list1}


\slide{Portscan med Zenmap GUI}

\hlkimage{16cm}{nmap-zenmap.png}

\slide{Scan for Heartbleed and SSLv2/SSLv3}

\hlkimage{8cm}{nmap-sslv2.png}

\begin{list1}
\item \verb+nmap -p 443 --script ssl-heartbleed <target>+\\
\link{https://nmap.org/nsedoc/scripts/ssl-heartbleed.html}
\item \verb+masscan 0.0.0.0/0 -p0-65535  --heartbleed+\\
\link{https://github.com/robertdavidgraham/masscan}
\end{list1}

\centerline{Almost every new vulnerability will have Nmap recipe}



\slide{Cracking passwords}

\begin{list2}
\item Hashcat is the world's fastest CPU-based password recovery tool.
\item oclHashcat-plus is a GPGPU-based multi-hash cracker using a brute-force attack (implemented as mask attack), combinator attack, dictionary attack, hybrid attack, mask attack, and rule-based attack.
\item oclHashcat-lite is a GPGPU cracker that is optimized for cracking performance. Therefore, it is limited to only doing single-hash cracking using Markov attack, Brute-Force attack and Mask attack.
\item John the Ripper password cracker old skool men stadig nyttig
\end{list2}

Source:\\
\link{http://hashcat.net/wiki/}\\
\link{http://www.openwall.com/john/}

\slide{Parallella John}

\hlkimage{20cm}{parallella-john.png}

\link{https://twitter.com/solardiz/status/492037995080712192}

Warning: FPGA hacking - not finished part of presentation

\slide{Stacking Parallella boards}
\hlkimage{16cm}{4BoardStack.jpg}

\link{http://www.parallella.org/power-supply/}




\slide{ Kali øvelser}

Vi kan ikke nå alverden men prøv selv at gentage det jeg viste hjemme
\begin{list2}
\item Wireshark
\item Wireshark med FTP
\item Nmap med Zenmap
\item Armitage og Metasploit, husk \verb+service postgresql start+
\end{list2}


\slide{Security devops}

\begin{list1}
\item We need devops skillz in security
\item automate, security is also big data
\item integrate tools, transfer, sort, search, pattern matching, statistics, ...
\item tools, languages, databases, protocols, data formats
\item Example introductions:
\begin{list2}
\item Seven languages/database/web frameworks in Seven Weeks
\item Elasticsearch the definitive guide\\
\link{http://www.elasticsearch.org/guide/en/elasticsearch/guide/current/index.html}
\item \link{http://www.elasticsearch.org/overview/kibana/}
\item \link{http://www.elasticsearch.org/overview/logstash/}
\end{list2}
\end{list1}

\centerline{We are all Devops now, even security people!}


\slide{Recommended Books: Get Started }


\begin{list1}


\item \emph{24 Deadly Sins of Software Security}
Michael Howard, David LeBlanc, John Viega 2. udgave, første hed 19 Deadly Sins
\hlkimage{3cm}{24-deadly.jpg}

\item \emph{Network Security Through Data Analysis: Building Situational Awareness}\\
By Michael Collins, O'Reilly Media, February 2014 Pages: 348
Low page count, but high value! Recommended.
\hlkimage{3cm}{network-security-through-data-analysis.png}
\end{list1}

\slide{Join Camps}

\hlkimage{18cm}{Chaos_Communication_Camp_2015_with_Thunderstorm.jpg}
\centerline{\bf Chaos Communication Camp 2015 It was Awesome!}

{\small Source Wikipedia and \link{https://www.flickr.com/photos/schwarzbrot/20447504269/}}

\myquestionspage

\end{document}
