\documentclass[a4paper,11pt,notitlepage,landscape]{report}
% Henrik Kramselund, February 2001
% hlk@zencurity.com,
% My standard packages
\usepackage{zencurity-one-page}
\usepackage{graphicx}
%\usepackage{lscape}
\usepackage{graphicx}

\begin{document}

%\rm
\selectlanguage{english}

\newcommand{\subject}[1]{Hacker Workshop PROSA }

\lhead{\fancyplain{}{\color{titlecolor}\bfseries\LARGE Kickstart: \subject}Wednesday Nov 29, 2023 from 17:00 - 20:00}

\normal
\hlkimage{2cm}{penguinping.jpg}
This material is prepared for use in \emph{\subject} It contains the very basic information to get started!

This workshop and exercises are expected to be performed in a training setting with network connected systems. The exercises use a number of open source tools which can be copied and reused after training.

To get kickstarted in this workshop it is recommended to perform the following:
\begin{list2}

\item[\faSquareO] {\bf Download and install Nmap}\\
We will use Nmap as the main tool \link{https://nmap.org/download.html} this package is easily installed on Windows, Mac OS X and Linux.\\
Note: The tool Nmap is an advanced portscanner and \emph{may} be flagged by your anti-virus program. Usually that is a good thing, we don't want anyone inside an organisation to do port scans, only authorized administrators. The tool itself is NOT malware.
\item[\faSquareO] {\bf Download slides and exercises booklet}\\
Recommend downloading latest version at the beginning of the workshop. Get main slides and exercises PDF from\\
\link{https://github.com/kramse/security-courses/tree/master/presentations/pentest/hackerworkshop-prosa}
\item[\faSquareO] {\bf Join the Pentest Wi-Fi network: SSID pentest}\\
I brought my own network for this workshop, where we can scan and perform things which are not recommended on the PROSA Wi-Fi
\end{list2}

You don't {\bf need} a virtual machine for the workshop, but if you want to do more advanced pentesting after the workshop it is recommended to install a Kali virtual machine:
\begin{list2}
\item[\faSquareO]  Read my recommendations about setup of virtual systems here\\ \link{https://github.com/kramse/kramse-labs}
\end{list2}


I hope we will have a fun and enjoyable time in this workshop.


Henrik Kramselund, \url{hlk@zencurity.com} \url{xhek@kea.dk}.

\eject

{\LARGE Sample IP network}
\hlkimage{17cm}{sample-ip-network.pdf}

\begin{itemize}
\item Network addresses 10.0.45.0/24 -- 10.0.45.0 - 10.0.45.255
\item The router is typically the first 10.0.45.1 or last usable address 10.0.45.254
\item Check you own address using the command line, or some control panel:
Linux terminal: \verb+ip address+ ,
Mac OS X terminal: \verb+ifconfig+ ,
Windows \verb+ipconfig+
\item The network configuration you get from DHCP will include name servers and router
\item You may port scan the whole network, using Nmap is fine
\item {\bf You are allowed to attack the router and servers provided by the instructor!} (Metasploit/exploits etc.)
\end{itemize}
\end{document}
