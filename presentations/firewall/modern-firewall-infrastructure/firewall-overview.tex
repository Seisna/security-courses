\documentclass[Screen16to9,17pt]{foils}
\usepackage{zencurity-slides}

\selectlanguage{danish}

% Diversitet er et fælles anliggende for alle i tech v. Henrik Kramselund, Internet-, netværks- og sikkerhedskonsulent

% Radia, Dorothy, Elizabeth og Laura er fornavnene på nogle af internet-, netværks- og sikkerhedskonsulentens rollemodeller inden for IT. Lær mere om dem og vær med, når Kramselund giver et call to action og deler erfaringer med at skabe inkluderende miljøer i tech

\addbibresource{../../../texfiles/firewall-refs.bib}

\begin{document}

\mytitlepage
{Firewall Overview: The Modern Firewall Infrastructure}
{PROSA 2022 with apperance of the TROOPERS20 presentation I didn't give}


\hlkprofiluk

\slide{Code of Conduct}

{\bf \LARGE Call to Action}

We need to be aware of the impact of our behaviour on others. When a person discovers they have made a mistake, they should acknowledge this and apologise.

\begin{quote}
Rationale
Our goals in having this Code of Conduct are:
\begin{list2}
\item {\bf To help everyone feel safe and included.} Many people will be new to our community. Some may have had negative experiences in other communities. We want to set a clear expectation that harassment and related behaviours are not tolerated here. If people do have an unpleasant experience, they will know that this is neither the norm nor acceptable to us as a community.

\item {\bf To make everyone aware of expected behaviour.} We are a diverse community; a CoC sets clear expectations in terms of how people should behave.
\end{list2}
\end{quote}

Source: {\small RIPE Code of Conduct Publication date: 05 Oct 2021
\link{https://www.ripe.net/publications/docs/ripe-766}}


\slide{"Pac-Man" rule}

\begin{quote}
\vskip -1cm The rule is:\\
{\bf When standing as a group of people, always leave room for 1 person to join your group.} \hskip 3cm More memorably, stand like Pac-Man!
\end{quote}

\hlkimage{6cm}{images/pacman-rule.png}

\begin{quote}
The new person, who has been given permission to join your group, will gather up the courage, and join you! Another important point, the group should now readjust to leave another space for a new person.\\
\link{https://ericholscher.com/blog/2017/aug/2/pacman-rule-conferences/}
\end{quote}



\slide{What we want -- Learning, coexistence, to be ourselves}

\hlkimage{13cm}{IPv6-Hackathon-image-itu-ripe.png}

{\color{red}\faHeart} Picture from RIPE NCC IPv6 hackathon ITU Copenhagen November 2017: {\color{red}\faHeart}\\{\footnotesize
\link{https://labs.ripe.net/author/becha/results-of-the-ripe-ncc-hackathon-version-6/}}

\slide{Hvordan skaber vi gode miljøer}

\hlkimage{10cm}{bornhack-2022-logo-l.png}

\begin{list2}
\item BornHack, har skabt et virkeligt godt miljø som nydes af alle fra børn/unge til gamle, og alle mennesker
\item Da vi startede i 2016 vidste vi at vi ville lave et miljø -- hvor alle skulle være velkomne.
\item Vi valgte at genbruge store dele af Code of Conduct fra EMF, og lavede BornHack Code of Conduct\\
\link{https://bornhack.dk/conduct/}
\item Det har været en stor success og I er velkomne næste gang
\item Der er også May Contain Hackers \link{https://mch2022.org/}\\
MCH2022 is the successor of a string of similar events happening every four years since 1989. \\
These are GHP, HEU, HIP, HAL, WTH, HAR, OHM and SHA.

\end{list2}



\slide{Goals for today}

\hlkimage{6cm}{thomas-galler-hZ3uF1-z2Qc-unsplash.jpg}

\begin{list2}
\item Introduce a network tool for isolation, firewalls and discuss what they \emph{filter on} -- network packets
\item See how they relate to VLANs -- logical/virtual seperation for hosts\\
Wi-Fi is also related to VLAN, but not much about wireless today
\item The Modern Firewall Infrastructure: Pruning networks without breaking them
\item Building a firewall infrastructure, talk about parts, show and tell
\end{list2}

hfill {\small Photo by Thomas Galler on Unsplash}


\slide{Plan for today: Firewall Overview}

\begin{list1}
\item Subjects
\item A) Introduction and firewall basics
\begin{list2}
\item Traffic inspection and firewalls
\item Generic IP Firewalls stateless filtering vs stateful inspection
\item Common features in firewalls
\end{list2}
\item B) The Modern Firewall Infrastructure
\begin{list2}
\item Problems in firewalls, and network security
\item Firewall pruning -- handling complexity
\item Making changes, administration
\item C) Examples of firewalls, more hands-on show it
\end{list2}
\end{list1}

\slide{Time schedule}

\begin{list2}
\item 17:00 - 18:15  Introduction and basics\\

\item 30min break  Eat and mingle, hang around, get coffee/tea\\
Go be with your family

\item 18:45 - 19:30 45min Walking through the technologies and protection\\

\item 15min break\\

\item 19:45 -20:30 45min More about components used for building secure and robust networks\\

\item 20:30 - 21:00 playtime, questions, demo, discussion
\end{list2}


\slide{Networks are trouble}

\hlkimage{9cm}{dragon-drawing-6.jpg}
\centerline{\Large Internet here be dragons}

\begin{list2}
\item Networks are constantly evolving
\item Increased threat landscape, World wide networks, attacks from everywhere
\item Vulnerabilities are found daily, Software quality - even security and firewall software has flaws
\item Even more vulnerabilities are \emph{developed} and \emph{installed}\\
Sorry developers, but some of you don't care, and it shows!
\end{list2}

\slide{Best Current Practice }

%\hlkimage{}{}

\begin{quote}
Lets get this out of the way immediately, you should already be doing
\end{quote}

\begin{list2}
\item Network segmentation and filtering -- we could write a book about this! {\myalert}
\item Monitor your network -- both bandwidth, error, netflow etc. {\myalert}
\item Take control of your network, no more admin/admin logins on core devices {\myalert}
\item Turn on authentication for protocols -- routing protocols but also any http service within your org {\myalert}
\item Configure host-based firewalls {\myalert}
\item Control DNS -- internally and externally, recursive, authoritative etc. {\myalert}
\end{list2}

\centerline{This goes for IPv4-only, IPv6-only, and mixed networks!}

\slide{Internet in a Box}

\hlkrightimage{10cm}{internet-in-a-box.jpeg}

The main purposes for bringing this box, is to show
\begin{list2}
\item These are standard devices, Juniper EX3300 cheap oldish, works great
\item Managed switches are a must! You can learn by buying cheap ones,\\
like the TP-Link T1500G-10PS  shown, VLAN, SNMP, Syslog ...
\item Multiple systems created using PC Engines APU2C4 (really D4)\\
running OpenBSD, Unbound, Suricata, Zeek, DHCP, \\
router advertisement, PF firewall - explicit and nice ICMPv6 filtering ...
\item Attack systems compact PCs or laptop
\item Creating a home lab is not expensive, \\
bought the Arista 7150 24-port 10G used on ebay
\end{list2}

You should have similar (or better) devices in your production network, and they can be
configured to do a LOT more than you use them for right now

\slide{Protection, building secure and robust networks}

\hlkimage{12cm}{sample-ip-network.pdf}


\begin{list2}
\item We should prefer security mechanisms that does NOT require us to keep patching every month
\item Can we change our networks to avoid this? Yes!
\end{list2}

\slide{Introducing firewalls}


Some of these slides are part of the course:\\
\emph{Communication and Network Security}

specifically the slideshow\\
\emph{3. Traffic Inspection and Firewalls}



\slide{Firewalls defined, multiple definitions}

\begin{quote}
In computing, a firewall is a network security system that monitors and controls incoming and outgoing network traffic based on predetermined security rules.[1] A firewall typically establishes a barrier between a trusted internal network and untrusted external network, such as the Internet.[2]
\end{quote}
Source: Wikipedia \link{https://en.wikipedia.org/wiki/Firewall_(computing)}

\begin{quote}
Firewalls are by design a choke point, natural place \\
to do network security monitoring!
\end{quote}

Source: Cheswick 1994 book and 2nd ed 2003
\emph{Firewalls and Internet Security: Repelling the Wily Hacker} , Second Edition, William R. Cheswick, Steven M. Bellovin, and Aviel D. Rubin, {\bf 2003}




\slide{Network Segmentation -- Firewalls}

\begin{quote}\small
\$ firewall\\

1. (I) {\bf An internetwork gateway that restricts data communication traffic to and from one of the connected networks} (the one said to be "inside" the firewall) and thus protects that network's system resources against threats from the other network (the one that is said to be "outside" the firewall). (See: guard, security gateway.)

2. (O) {\bf A device or system that controls the flow of traffic between networks using differing security postures.} Wack, J. et al (NIST), "Guidelines on Firewalls and Firewall Policy", Special Publication 800-41, January 2002.

Tutorial: A firewall typically protects a smaller, secure network (such as a corporate LAN, or even just one host) from a larger network (such as the Internet). The firewall is installed at the point where the networks connect, and the firewall applies policy rules to control traffic that flows in and out of the protected network.
\end{quote}
{\footnotesize Source: RFC4949 \emph{Internet Security Glossary, Version 2}\\
\link{https://datatracker.ietf.org/doc/html/rfc4949} 2007}

\slide{Continued}
\begin{quote}\small
{\bf A firewall is not always a single computer.} For example, a firewall may consist of a pair of filtering routers and one or more proxy servers running on one or more bastion hosts, all connected to a small, dedicated LAN (see: buffer zone) between the two routers. The external router blocks attacks that use IP to break security (IP address spoofing, source routing, packet fragments), while proxy servers block attacks that would exploit a vulnerability in a higher-layer protocol or service. The internal router blocks traffic from leaving the protected network except through the proxy servers. The difficult part is defining criteria by which packets are denied passage through the firewall, because a firewall not only needs to keep unauthorized traffic (i.e., intruders) out, but usually also needs to let authorized traffic pass both in and out.
\end{quote}
{\footnotesize Source: RFC4949 \emph{Internet Security Glossary, Version 2}\\
\link{https://datatracker.ietf.org/doc/html/rfc4949} 2007}

\begin{list2}
\item Network layer, we often call them packet filters, stateless
\item application level, we often call them, stateful filtering and gateways
\item Firewalls are by design a choke point, natural place \\
to do network security monitoring!
%\item Older but still interesting Cheswick chapter 2 PDF
%\emph{A Security Review of Protocols: Lower Layers}\\
%\link{http://www.wilyhacker.com/}
\end{list2}


\slide{Definition of firewalls -- Wikipedia}

Another short definition that encapsulates this is found on Wikipedia, and may suffice in many situations. Again there will typically be multiple networks, zones or areas of the networks with varying degrees of trust.
\begin{quote}
In computing, a firewall is a {\bf network security system that monitors and controls incoming and outgoing network traffic based on predetermined security rules}.[1] A firewall typically establishes a barrier between a trusted network and an untrusted network, such as the Internet.[2]
\end{quote}
Source: Wikipedia
 \link{https://en.wikipedia.org/wiki/Firewall_(computing)}

{\bf TL;DR Not necessarily a single device}


\slide{A firewall -- in the vendor eyes}

% Single line in firewall - single line out
% With text device under test, as caption
\hlkimage{19cm}{Firewall-vendor.pdf}

\begin{quote}
"{\bf Can your firewall flex in the face of change?}\\
Does it harmonize your network, workload, and application security? Does it protect apps and employees in your hybrid or multicloud environment? Make sure you're covered."
\end{quote}
Source: not shown to protect the audience from further marketing speak
% Source is Cisco.com

\slide{A firewall -- in the enterprise mindset }

% Single line in firewall,
% cloud with at-sign Internet on the left,\\
% cloud with at-sign LAN on the right
\hlkimage{19cm}{Firewall-enterprise.pdf}

\begin{list2}
\item Even though some vendors suggest they can do everything in a single box, I don't believe them!
\item Truth -- yes, we can do almost anything in software
\item Realization {\bf Your infrastructure is based on multiple components and or devices}
\end{list2}


\slide{Firewalls and Internet Security, 1994}

\hlkimage{3cm}{images/cheswick-cover2e.jpg}

\begin{list1}
\item Cheswick chapter 2 \emph{A Security Review of Protocols: Lower Layers}
\begin{list2}
\item Network layer, packet filters, application level, stateless, stateful
\end{list2}
\item Cheswick chapter 3 \emph{Security Review: The Upper Layers}
\begin{list2}
\item How to configure firewalls often boil down to, should we allow protocol X
\item If we allow SMB through an internet firewall, we are asking for trouble
\end{list2}
\item Source: Cheswick 1994 book and 2nd ed 2003
\emph{Firewalls and Internet Security: Repelling the Wily Hacker} , Second Edition, William R. Cheswick, Steven M. Bellovin, and Aviel D. Rubin, {\bf 2003} \link{http://www.wilyhacker.com/}
\end{list1}

\slide{Baseline Skills for Network Security}

\begin{list2}\small
\item Threat-Centric Security, NSM, and the NSM Cycle
\item TCP/IP Protocols
\item Common Application Layer Protocols
\item Packet Analysis
\item Windows Architecture
\item Linux Architecture
\item Basic Data Parsing (BASH, Grep, SED, AWK, Python, PowerShell etc)
\item IDS Usage (Snort, Suricata, etc.)
\item Indicators of Compromise and IDS Signature Tuning
\item Open Source Intelligence Gathering
\item Basic Analytic Diagnostic Methods
\item Basic Malware Analysis
\end{list2}

Source: \emph{Applied Network Security Monitoring Collection, Detection, and Analysis},\\
Chris Sanders and Jason Smith

\slide{Book: Applied Network Security Monitoring (ANSM)}

\hlkimage{5cm}{ansm-book.png}

\emph{Applied Network Security Monitoring: Collection, Detection, and Analysis}
1st Edition

Chris Sanders, Jason Smith
eBook ISBN: 9780124172166
Paperback ISBN: 9780124172081 496 pp.
Imprint: Syngress, December 2013


\slide{Book: Practical Packet Analysis (PPA)}
\hlkimage{6cm}{PracticalPacketAnalysis3E_cover.png}

\emph{Practical Packet Analysis,
Using Wireshark to Solve Real-World Network Problems}
by Chris Sanders, 3rd Edition
April 2017, 368 pp.
ISBN-13:
978-1-59327-802-1

\link{https://nostarch.com/packetanalysis3}


\slide{Network Protocol Knowledge Needed for Network Security}

To work with network security the following protocols are the bare minimum to know about.

\begin{list2}
\item ARP Address Resolution Protocol for IPv4
\item NDP Neighbor Discovery Protocol for IPv6
\item IPv4 \& IPv6 -- the basic packet fields source, destination,
\item ICMPv4 \& ICMPv6 Internet Control Message Protocol
\item UDP User Datagram Protocol
\item TCP Transmission Control Protocol
\item DHCP Dynamic Host Configuration Protocol
\item DNS Domain Name System
\end{list2}

These protocols are part of the Internet Protocol suite, or TCP/IP for short. The canonical document describing this is from 1981 RFC-0791 \citetitle{RFC0791}. The protocols were deployed on the internet around 1983.





\slide{Unified communications}

\hlkimage{19cm}{firma-netvaerk-wlan}


\slide{Modern Firewall Infrastructures}


\centerline{\hlkbig A firewall {\color{security6blue}blocks traffic} on a network}

\vskip 1 cm
\pause

\centerline{\hlkbig A firewall {\color{red}allows traffic} on a network}
{\small The interesting part is typically what it allows!}

\begin{list1}
\item A firewall infrastructure must:
\begin{list2}
\item Prevent attackers from entering
\item Prevent data exfiltration
\item Prevent worms, malware, virus from spreading in networks
\item Be part of an overall solution with ISP, routers, other firewalls, switched infrastructures,\\
  intrusion detection systems and the rest of the infrastructure
\end{list2}
\end{list1}

\vskip 5mm
\centerline{Difficult -- and requires design and secure operations}


\slide{Packet Filtering}

\begin{alltt}\footnotesize
0                   1                   2                   3
0 1 2 3 4 5 6 7 8 9 0 1 2 3 4 5 6 7 8 9 0 1 2 3 4 5 6 7 8 9 0 1
+-+-+-+-+-+-+-+-+-+-+-+-+-+-+-+-+-+-+-+-+-+-+-+-+-+-+-+-+-+-+-+-+
|Version|  IHL  |Type of Service|          Total Length         |
+-+-+-+-+-+-+-+-+-+-+-+-+-+-+-+-+-+-+-+-+-+-+-+-+-+-+-+-+-+-+-+-+
|         Identification        |Flags|      Fragment Offset    |
+-+-+-+-+-+-+-+-+-+-+-+-+-+-+-+-+-+-+-+-+-+-+-+-+-+-+-+-+-+-+-+-+
|  Time to Live |    Protocol   |         Header Checksum       |
+-+-+-+-+-+-+-+-+-+-+-+-+-+-+-+-+-+-+-+-+-+-+-+-+-+-+-+-+-+-+-+-+
|                       Source Address                          |
+-+-+-+-+-+-+-+-+-+-+-+-+-+-+-+-+-+-+-+-+-+-+-+-+-+-+-+-+-+-+-+-+
|                    Destination Address                        |
+-+-+-+-+-+-+-+-+-+-+-+-+-+-+-+-+-+-+-+-+-+-+-+-+-+-+-+-+-+-+-+-+
|                    Options                    |    Padding    |
+-+-+-+-+-+-+-+-+-+-+-+-+-+-+-+-+-+-+-+-+-+-+-+-+-+-+-+-+-+-+-+-+
\end{alltt}

\begin{list1}
\item Packet filtering are firewall devices filtering on single packet
\item Most \emph{specialized firewall devices} do stateful filtering and more
\item Don't forget IPv6 -- even though you haven't turned it on, it is there
\end{list1}




\slide{Modern Firewalls}

\begin{list1}
\item Basalt set et netværksfilter - det yderste fæstningsværk
\item Indeholder typisk:
  \begin{list2}
   \item Grafisk brugergrænseflade til konfiguration - er det en
   fordel?
\item TCP/IP filtermuligheder - pakkernes afsender, modtager, retning
  ind/ud, porte, protokol, ...
\item både IPv4 og IPv6
\item foruddefinerede regler/eksempler - er det godt hvis det er nemt
  at tilføje/åbne en usikker protokol?
\item typisk NAT funktionalitet indbygget
\item typisk mulighed for nogle serverfunktioner: kan agere
  DHCP-server, DNS caching server og lignende
  \end{list2}
\item En router med Access Control Lists - kaldes ofte netværksfilter,
  mens en dedikeret maskine kaldes firewall
%  funktionen er reelt den samme - der filtreres traffik
\end{list1}


\slide{Sample rules from OpenBSD PF}

\begin{alltt}\tiny
# hosts and networks
router="217.157.20.129"
webserver="217.157.20.131"
homenet="{ 192.168.1.0/24, 1.2.3.4/24 }"
wlan="10.0.42.0/24"
wireless=wi0
set skip lo0
# things not used
spoofed="{ 127.0.0.0/8, 172.16.0.0/12, 10.0.0.0/16, 255.255.255.255/32 }"
{\bf
# default block anything
block in all }
# egress and ingress filtering - disallow spoofing, and drop spoofed
block in quick from $spoofed to any
block out quick from any to $spoofed

pass in on $wireless proto tcp from \{ $wlan $homenet \} to any port = 22
pass in on $wireless proto tcp from any to $webserver port = 80

pass out
\end{alltt}


\slide{Example Firewall Products}
\begin{list2}
\item Checkpoint Firewall-1 \link{http://www.checkpoint.com}
\item Cisco ASA \link{http://www.cisco.com}
\item Juniper SRX \link{http://www.juniper.net}
\item Palo Alto \link{https://www.paloaltonetworks.com/}
\item Fortinet \link{https://www.fortinet.com/}
\end{list2}

Those listed are the most popular commercial ones I see in Denmark


\slide{Open source based firewalls}
\begin{list2}
\item Linux firewalls IP tables, use command line tool ufw Uncomplicated Firewall!
\item Firewall GUIs on top of Linux -- lots! Some are also available as commercial ones
\item OpenBSD PF
\link{http://www.openbsd.org}
\item FreeBSD IPFW og IPFW2 \link{http://www.freebsd.org}
\item Mac OS X uses OpenBSD PF
\item FreeBSD has an older version of the OpenBSD PF, should really be renamed now
\end{list2}

\slide{OPNsense}

\hlkimage{6cm}{images/screenshots_OPNsense-1024x518.png}

\begin{list1}
\item OPNsense \link{https://opnsense.org/}
\item Firewall bygget på FreeBSD med grafisk web interface
\item Oprindeligt videreudvikling af m0n0wall over \link{https://www.pfsense.org/}\\
\item De har været brugt i produktion i danske firmaer igennem mange år
\end{list1}


\slide{Uncomplicated Firewall (UFW)}

\begin{alltt}\small
root@debian01:~# apt install ufw
...
root@debian01:~# ufw allow 22/tcp
Rules updated
Rules updated (v6)
root@debian01:~# ufw enable
Command may disrupt existing ssh connections. Proceed with operation (y|n)? y
Firewall is active and enabled on system startup
root@debian01:~# ufw status numbered
Status: active

     To                         Action      From
     --                         ------      ----
[ 1] 22/tcp                     ALLOW IN    Anywhere
[ 2] 22/tcp (v6)                ALLOW IN    Anywhere (v6)
\end{alltt}

\begin{list2}
\item Extremely easy to use -- I recommend and use the (Uncomplicated Firewall) UFW
\end{list2}

\demo{UFW -- Uncomplicated Firewall}


\slide{Specialized Firewall devices are NOT Alone}

\hlkimage{15cm}{network-layers-1.png}

\centerline{Use Defense in Depth -- all layers have features}




\slide{Together with Firewalls - Virtual LAN (VLAN)}

\hlkimage{8cm}{vlan-portbased.pdf}

\begin{list1}
\item Managed switches often allow splitting into zones called virtual LANs
\item Most simple version is port based
\item Like putting ports 1-4 into one LAN and remaining in another LAN
\item Packets must traverse a router or firewall to cross between VLANs
\end{list1}

\slide{Virtual LAN (VLAN) IEEE 802.1q}

\hlkimage{16cm}{vlan-8021q.pdf}

\begin{list1}
\item Using IEEE 802.1q  VLAN tagging on Ethernet frames
\item Virtual LAN, to pass from one to another, must use a router/firewall
\item Allows separation/segmentation and protects traffic from many security issues
\end{list1}




\slide{Network Access Control -- Connecting clients more securely}

Talking about standard, another useful one:\\
IEEE 802.1x -- Port Based Network Access Control

\hlkimage{7cm}{802.1X_wired_protocols.png}

\begin{list1}
\item Authentication protocol ensures user validation before port access
\item Can authenticate using username and then password or certificate
\item Typically RADIUS and 802.1x which can use LDAP or Active Directory
\item Already used in Wi-Fi networks, so can be turned on for wired Ethernet ports
\end{list1}



\slide{Protection, building secure and robust networks}

\hlkimage{12cm}{sample-ip-network.pdf}


\begin{list2}
\item We should prefer security mechanisms that does NOT require us to keep patching every month
\item Can we change our networks to avoid this? Yes!
\end{list2}


\slide{Defense in depth}

%\hlkimage{10cm}{Bartizan.png}
\hlkimage{15cm}{medieval-clipart-5}
\centerline{Picture originally from: \url{http://karenswhimsy.com/public-domain-images}}



\slide{Firewall koncepter}

\begin{list1}
\item Rækkefølgen af regler betyder noget!
\begin{list2}
\item To typer af firewalls:
 First match - når en regel matcher, gør det som angives block/pass
 Last match  - marker pakken hvis den matcher, til sidst afgøres block/pass
\end{list2}
\item {\bf Det er ekstremt vigtigt at vide hvilken type firewall
    man bruger!}
\item OpenBSD PF er last match
\item FreeBSD IPFW er first match
\item Linux iptables/netfilter er last match
\end{list1}

\slide{First or Last match firewall?}

\hlkimage{18cm}{images/first-last-match-1.pdf}



\slide{First match - IPFW}

\begin{alltt}
\hlksmall
00100 16389  1551541 allow ip from any to any via lo0
00200     0        0 deny log ip from any to 127.0.0.0/8
00300     0        0 check-state
...
{\bfseries
65435    36     5697 deny log ip from any to any}
65535   865    54964 allow ip from any to any
\end{alltt}

\vskip 2 cm

\centerline{Den sidste regel nås aldrig!}

\slide{Last match - OpenBSD PF}

\begin{alltt}\small
ext_if="ext0"
int_if="int0"
{\bf
block in}
pass out keep state

pass quick on \{ lo $int_if \}

# Tillad forbindelser ind på port 80=http og port 53=domain
# på IP-adressen for eksterne netkort ($ext_if) syntaksen
pass in on $ext_if proto tcp to ($ext_if) port http keep state
pass in on $ext_if proto \{ tcp, udp \} to ($ext_if) port domain keep state
\end{alltt}


Pakkerne markeres med block eller pass indtil sidste regel\\
nøgleordet \emph{quick} afslutter match - god til store regelsæt

\slide{Linux iptables/netfilter eksempel}

\begin{alltt}\footnotesize
# Firewall configuration written by system-config-securitylevel
# Manual customization of this file is not recommended.
*filter
:INPUT ACCEPT [0:0]
:FORWARD ACCEPT [0:0]
:OUTPUT ACCEPT [0:0]
:RH-Firewall-1-INPUT - [0:0]
-A INPUT -j RH-Firewall-1-INPUT
-A FORWARD -j RH-Firewall-1-INPUT
-A RH-Firewall-1-INPUT -i lo -j ACCEPT
-A RH-Firewall-1-INPUT -p icmp --icmp-type any -j ACCEPT
-A RH-Firewall-1-INPUT -p 50 -j ACCEPT
-A RH-Firewall-1-INPUT -p 51 -j ACCEPT
-A RH-Firewall-1-INPUT -p udp --dport 5353 -d 224.0.0.251 -j ACCEPT
-A RH-Firewall-1-INPUT -p udp -m udp --dport 631 -j ACCEPT
-A RH-Firewall-1-INPUT -m state --state ESTABLISHED,RELATED -j ACCEPT
-A RH-Firewall-1-INPUT -m state --state NEW -m tcp -p tcp --dport 443 -j ACCEPT
-A RH-Firewall-1-INPUT -m state --state NEW -m tcp -p tcp --dport 22 -j ACCEPT
-A RH-Firewall-1-INPUT -j REJECT --reject-with icmp-host-prohibited
COMMIT
\end{alltt}

\centerline{NB: husk at aktivere IP forwarding}


\slide{Firewalls og ICMP}

\begin{alltt}
ipfw add allow icmp from any to any icmptypes 3,4,11,12
\end{alltt}

\begin{list1}
\item Ovenstående er IPFW syntaks for at tillade de interessante ICMP beskeder igennem
\item Tillad ICMP types:
\begin{list2}
\item 3 Destination Unreachable
\item 4 Source Quench Message
\item 11 Time Exceeded
\item 12 Parameter Problem Message
\end{list2}
\end{list1}

\slide{Firewall konfiguration}

\begin{list1}
\item Den bedste firewall konfiguration starter med:
\begin{list2}
\item Papir og blyant
\item En fornuftig adressestruktur
\end{list2}
\item Brug dernæst en firewall med GUI første gang!
\item Husk dernæst:
\begin{list2}
\item En firewall skal passes
\item En firewall skal opdateres
\item Systemerne bagved skal hærdes!
\end{list2}
\end{list1}




\slide{Bloker indefra og ud}

\begin{list1}
\item Der er porte og services som altid bør blokeres
\item Det kan være kendte sårbare services
\begin{list2}
\item Windows SMB filesharing - ikke til brug på Internet!
\item UNIX NFS - ikke til brug på Internet!
\end{list2}
\item Kendte problemer som minimum
\end{list1}

Demo time? Nmap and Nping?


\slide{Special features}

\begin{list2}
\item Network Address Translation - NAT
\item IPv6 funktionalitet

\item Båndbredde håndtering
\item VLAN funktionalitet - mere udbredt i forbindelse med VoIP
\item Redundante firewalls - pfsync og CARP
% pfsync giver et indblik i hvordan den slags kan laves, hvor de
% kommercielle ``bare kan det''
\item IPsec og Andre VPN features
\item inspection - diverse muligheder for at lave deep inspection i protokoller
\item Eksemplvis DNS inspection
\end{list2}

\slide{Proxy servers}

\begin{list1}
\item Filtrering på højere niveauer i OSI modellen er muligt
\item Idag findes proxy applikationer til de mest almindelige
  funktioner
\item Den typiske proxy er en caching webproxy der kan foretage HTTP
  request på vegne af arbejdsstationer og gemme resultatet
\item Reverse proxy benyttes også ofte til at terminere TLS/HTTPS og lave loadbalancing\\
Web Application Firewall (WAF) benyttes også
\item NB: nogle protokoller egner sig ikke til proxy servere
\end{list1}


\slide{IPsec og Andre VPN features}

\begin{list1}
\item De fleste firewalls giver mulighed for at lave krypterede
  tunneler
\item Nyttigt til fjernkontorer der skal have usikker traffik henover
  usikre netværk som Internet
\item Konceptet kaldes Virtual Private Network VPN
\item IPsec er de facto standarden for VPN og beskrevet i RFC'er
\end{list1}



\slide{Netflow and Session Logging}

\begin{list2}
\item Netflow is getting more important, more data share the same links
\item Accounting is important
\item Detecting DoS/DDoS and problems is essential
\item Netflow sampling is vital information - 123Mbit, but what kind of traffic
\item NFSen is an old but free application
\link{http://nfsen.sourceforge.net/}
\item Currently also investigating sFlow - hopefully more fine grained
\item sFlow, short for "sampled flow", is an industry standard for packet export at Layer 2 of the OSI model, \\
\link{https://en.wikipedia.org/wiki/SFlow}
\end{list2}

\centerline{Netflow is often from routers, we dont have any here}

Also look into Elastiflow! \link{https://github.com/robcowart/elastiflow}


\slide{Collect Network Evidence from the network}

\begin{list1}
\item Network Flows
\item Cisco standard NetFlow version 5 defines a flow as a unidirectional sequence of packets that all share the following 7 values:
\begin{list2}
\item Ingress interface (SNMP ifIndex)
\item IP protocol, Source IP address and Destination IP address
\item Source port for UDP or TCP, 0 for other protocols
\item Destination port for UDP or TCP, type and code for ICMP, or 0 for other protocols
\item IP Type of Service
\end{list2}
\item today Netflow version 9 or IPFIX
\end{list1}

Source: \\{\footnotesize
\link{https://en.wikipedia.org/wiki/NetFlow}\\
\link{https://en.wikipedia.org/wiki/IP_Flow_Information_Export}}



\slide{Netflow using NFSen}

\hlkimage{13cm}{images/nfsen-overview.png}


\slide{ Netflow NFSen}

\hlkimage{17cm}{nfsen-udp-flood.png}

\centerline{An extra 100k packets per second from this netflow source (source is a router)}

\slide{Netflow processing from the web interface}

\hlkimage{12cm}{images/nfsen-processing-1.png}

\centerline{Bringing the power of the command line forward}

\slide{ElastiFlow -- Elasticsearch based}

\hlkimage{10cm}{elastiflow.png}

\begin{quote}
  ElastiFlow™ provides network flow data collection and visualization using the Elastic Stack (Elasticsearch, Logstash and Kibana). It supports Netflow v5/v9, sFlow and IPFIX flow types (1.x versions support only Netflow v5/v9).
\end{quote}
Source: Picture and text from \link{https://github.com/robcowart/elastiflow} \\


\slide{Big Data tools: Elasticsearch}

\hlkimage{10cm}{kibana-basics-with-vega.jpg}

Elasticsearch is an open source distributed, RESTful search and analytics engine capable of solving a growing number of use cases.

\link{https://www.elastic.co}

\vskip 1cm
\centerline{We are all Devops now, even security people!}


\slide{Kibana}

\hlkimage{10cm}{kibanascreenshothomepagebannerbigger.jpg}

Highly recommended for a lot of data visualisation as non-programmers can create, save, and share dashboards

Source:
\link{https://www.elastic.co/products/kibana}

I have a set of Ansible files that can install a mostly complete Elastic stack on Debian:\\
\link{https://github.com/kramse/kramse-labs/tree/master/suricatazeek}

\slide{Network Security Through Data Analysis}

\hlkimage{4cm}{network-security-through-data-analysis.png}

Low page count, but high value! Recommended.

Network Security through Data Analysis, 2nd Edition
By Michael S Collins
Publisher: O'Reilly Media
2015-05-01: Second release, 348 Pages

New Release Date: August 2017



\slide{Bottlenecks exist, but where}


\hlkimage{12cm}{overview-routing-customer-2015.pdf}

\begin{list2}
\item Lower layer attacks Transport Layer Attacks TCP SYN flood -- packet based
\item Higher layer attacks like Slowloris and web attacks -- keep sessions running
\item Protect everything without loosing functionality or creating administrative nightmare
\end{list2}

\slide{Availability and Network flooding attacks}

The attacks we are discussing today are:
\begin{list2}
\item {\bf SYN flood} is the most basic and very common on the internet towards 80/tcp and 443/tcp
\item {\bf ICMP and UDP flooding} are the next popular targets -- more similar ones exist
\item Special packets and protocols -- anything that can create \emph{load on systems} work
\item All of them try to use up some resources
\begin{list2}
\item {\bf Memory space} in specific sections of the {\bf kernel, TCP state, firewalls state, number of concurrent sessions/connections}
\item {\bf Interrupt processing} of packets - packets per second (pps)
\item {\bf CPU processing} in firewalls, pps
\item CPU processing in server software
\item {\bf Bandwidth} - megabits per second (mbps)
\item Typically source is spoofed or amplification attacks abusing devices on the Internet
\end{list2}
\end{list2}





\slide{Packet processing in firewalls -- detailed view}

% Picture with pipeline from Juniper SRX
\hlkimage{21cm}{srx-firewall-flow-based-processing.png}
\emph{Traffic Processing on SRX Series Devices Overview}\\ {\scriptsize
\link{https://www.juniper.net/documentation/us/en/software/junos/flow-packet-processing/topics/topic-map/security-srx-devices-processing-overview.html}}




\slide{Scanning and Attacking -- \bf Pressure Points and Scope}

% Drawing showing "network"

\hlkimage{21cm}{generic-network-pressure-points.pdf}

\begin{list2}
\item In scope for me is everything that could adversely affect the network
\item Common scope IPv4: Link network /28 or /26 and a hosting network /24
\item Common scope IPv6: Link network /64 (bad) or /127 (RFC9099) and a hosting network /48 with subnets
%\item Domain Name Service outside of the network is ofc also attack vector, ignored today
\end{list2}


\slide{DDoS traffic before filtering}
\hlkimage{23cm}{ddos-before-filtering}

\centerline{Only two links shown, at least 3Gbit incoming for this single IP}

\slide{DDoS traffic after filtering}
\hlkimage{23cm}{ddos-after-filtering}
\centerline{Link toward server (next level firewall actually) about ~350Mbit outgoing}

Better to filter stateless before traffic reaches firewall, less work!



\slide{DDoS protection and flooding}

\hlkimage{12cm}{overview-routing-customer-2015.pdf}

\begin{list2}
\item Transport Layer Attacks TCP SYN flood TCP sequence numbers
\item High level attacks like Slowloris - keep TCP/HTTP connection for a long time.
\end{list2}

\slide{Båndbreddestyring og policy based routing}

\begin{list1}
\item Mange routere og firewalls idag kan lave båndbredde allokering til
  protokoller, porte og derved bestemte services
  \item Specielt relevant for DDoS beskyttelse
  \item Findes på F5 BigIP, Cisco, Junos osv.
\item Mest kendte er i Open Source:
\begin{list2}
\item OpenBSD - integreret i PF
\item FreeBSD har dummynet
\item Linux Traffic Control
\end{list2}
\item Det kaldes også traffic shaping
\end{list1}



\slide{Designing the protection -- bandwidth and rate limit}

\begin{tabularx}{\textwidth-5cm}{|p{5cm}|p{7cm}|X|} \hline
{\bf Protocol} & {\bf Mbps} & {\bf Prefix}\\\hline
TCP & Up to full bandwidth 10Gbps & 192.0.2.0/25 \\\hline
UDP & Less than 1Gbps & 192.0.2.128/25 \\\hline
ICMP & Less than 10Mbps & 192.0.2.0/24 \\\hline
\end{tabularx}


\begin{list2}
\item Create an address plan for your services
\item Monitor your traffic -- how much UDP and TCP do you have, roughly

\item Above is a simplified example -- dig deeper into your traffic
\end{list2}

\slide{Designing the protection -- address families \& protocols }

\begin{tabularx}{\textwidth-5cm}{|p{2cm}|p{2cm}|p{7cm}|X|} \hline
{\bf Address family} & {\bf Protocol} & {\bf Services and ports} & {\bf Prefix}\\\hline
IPv4 & TCP & 25, 80, 8003, 443, 4443 & 192.0.2.0/25 \\\hline
IPv4 & UDP & 53 & 192.0.2.128/25 \\\hline
IPv6 & UDP & 53  & 2001:DB8:ABCD:0053::/60 \\\hline
IPv6 & TCP & 80 443 & 2001:DB8:ABCD:1000::/56 \\\hline
\end{tabularx}

\begin{list2}
\item Direction is also very important -- servers that never initiate connections have fewer requirements
\item How much traffic do you have that uses IPv6 yet? Should an IPv6 DDoS take up all resources?
\item Maybe let IPv4 only use a part, so at least some customers can visit using IPv6?
\item Maybe you can do an allow list for allocated networks, since not all is used yet
\end{list2}



\slide{Config example: SNMP}

\begin{alltt}\footnotesize
snmp \{
    description "Router-CPH-01";
    location "Copenhagen, Denmark";
    contact "noc@zencurity.com";
    community yourcommunitynotmine \{
        authorization read-only;
        {\bf clients \{
               10.1.1.1/32;
               10.1.2.2/32;
        \}    }
    \}
\}
\end{alltt}

If you must use SNMPv2 then at least put it into separate VLAN! {\myalert}



%\slide{Stateless filtering Junos}
\slide{Stateless firewall filter throw stuff away}

\begin{alltt}\footnotesize
hlk@MX-CPH-02> show configuration firewall filter all | no-more
/* This is a sample, better to use BGP flowspec and RTBH */
inactive: term edgeblocker \{
    from \{
        source-address \{
            84.180.xxx.173/32;
...
            87.245.xxx.171/32;
        \}
        destination-address \{
            91.102.91.16/28;
        \}
        protocol [ tcp udp icmp ];
    \}
    then \{
        count edge-block;
        discard;
    \}
\}
\end{alltt}
Hint: can also leave out protocol and then it will match all protocols

\slide{Stateless firewall filter limit protocols}

\begin{alltt}\footnotesize
term limit-icmp \{
    from \{
        protocol icmp;
    \}
    then \{
        policer ICMP-100M;
        accept;
    \}
\}
term limit-udp \{
    from \{
        protocol udp;
    \}
    then \{
        policer UDP-1000M;
        accept;
    \}
\}
\end{alltt}

Routers also have extensive Class-of-Service (CoS) tools today

\slide{Strict filtering for some servers, still stateless!}

\begin{alltt}\footnotesize
term some-server-allow \{
    from \{
        destination-address \{
            109.238.xx.0/xx;
        \}
        protocol tcp;
        destination-port [ 80 443 ];
    \} then accept;
\}
term some-server-block-unneeded \{
    from \{
        destination-address \{
            109.238.xx.0/xx; \}
        protocol-except icmp;  \}
    then \{ count some-server-block; discard;
    \}
\}
\end{alltt}

Wut - no UDP, yes UDP service is not used on these servers


\slide{ Firewalls - screens, IDS like features}

When you know regular traffic you can decide:

\begin{alltt}\footnotesize
hlk@srx-kas-05# show security screen ids-option untrust-screen
icmp \{
    ping-death;
\}
ip \{
    source-route-option;
    tear-drop;
\}
tcp \{    Note: UDP flood setting also exist
    syn-flood \{
        alarm-threshold 1024;
        attack-threshold 200;
        source-threshold 1024;
        destination-threshold 2048;
        timeout 20;
    \}
    land;
\} Always select your own settings YMMV
\end{alltt}


\slide{uRPF unicast Reverse Path Forwarding}

\begin{quote}
Reverse path forwarding (RPF) is a technique used in modern routers for the purposes of ensuring loop-free forwarding of multicast packets in multicast routing and to help prevent IP address spoofing in unicast routing.
\end{quote}
Source: \link{http://en.wikipedia.org/wiki/Reverse_path_forwarding}

%\slide{Juniper RPF check}

%\begin{quote}
%{\bf Understanding Unicast Reverse Path Forwarding}\\
%IP spoofing can occur during a denial-of-service (DoS) attack. IP spoofing allows an intruder to pass IP packets to a destination as genuine traffic, when in fact the packets are not actually meant for the destination. This type of spoofing is harmful because it consumes the destination's resources.

%A unicast reverse-path-forwarding (RPF) check is a tool to reduce forwarding of IP packets that might be spoofing an address. A unicast RPF check performs a route table lookup on an IP packet's source address, and checks the incoming interface.
%\end{quote}

%Source:\\ {\footnotesize\link{http://www.juniper.net/techpubs/en_US/junos13.1/topics/topic-map/unicast-rpf.html}\\
%\link{http://www.juniper.net/techpubs/en_US/junos13.1/topics/usage-guidelines/interfaces-configuring-unicast-rpf.html}}


\slide{Strict vs loose mode RPF}

\hlkimage{24cm}{uRPF-check-1.pdf}

\slide{Strict mode RPF}

\begin{quote}
{\bf Configuring Unicast RPF Strict Mode}\\
In strict mode, unicast RPF checks whether the incoming packet has a source address that matches a prefix in the routing table, {\bf and whether the interface expects to receive a packet with this source address prefix.}
\end{quote}


\slide{uRPF Junos config with loose mode}

\begin{alltt}\footnotesize
xe-5/1/1 \{
    description "Transit: Blah (AS65512)";
    unit 0 \{
        family inet \{
            rpf-check \{
                mode loose;
            \}
            filter \{
                input all;
                output all;
            \}
            address xx.yy.xx.yy/30;
        \}
        family inet6 \{
            rpf-check \{
                mode loose;
            \}
            address 2001:xx:yy/126;
\} \} \}
\end{alltt}

See also: {\small\link{http://www.version2.dk/blog/den-danske-internettrafik-og-bgp-49401}}


\slide{Remotely Triggered Black Hole Configurations}

\hlkimage{6cm}{packetlife-RTBH.png}
Picture from packetlife.net showing  R9 as a standalone "management" router for route injection.

{\footnotesize
\link{http://packetlife.net/blog/2009/jul/6/remotely-triggered-black-hole-rtbh-routing/}\\
\link{https://ripe65.ripe.net/presentations/285-inex-ripe-routingwg-amsterdam-2012-09-27.pdf}
\link{https://www.inex.ie/rtbh}}


\slide{Remotely Triggered Black Hole at upstreams}

\begin{alltt}\footnotesize
6.  Black Hole Server (Optional)
   ###################################################################################
   #                           NOTE                                                  #
   #  The Cogent Black Hole server will allow customers to announce a /32 route      #
   #  to Cogent and have all traffic to that network blocked at Cogents backbone.    #
   #  All peers on the Cogent black hole server require a password and IP address    #
   #  from your network for Cogent to peer with.                                     #
   ###################################################################################

       [   ]  Please set up a BGP peer on the Cogent Black Hole server
       Black Hole server password:
       Black Hole server peer IP:

       North American Black Hole Peer:  66.28.8.1
       European Black Hole Peer:  130.117.20.1
\end{alltt}

Source:\\
{\footnotesize\link{http://cogentco.com/files/docs/customer_service/guide/bgpq.sample.txt}}

\centerline{Better drop single /32 host than whole network!}

\slide{More information about DDoS testing}

More DDoS and testing for DDoS can be found in this presentation:
\begin{quote}\footnotesize{\bf
Simulated DDoS Attacks, Breaking the Firewall Infrastructure
Henrik Kramselund}

DDoS Attacks have become a daily annoyance for many, and we need to create robust infrastructure. This tutorial will go through a proposed method for testing your own infrastructure using off-the-shelf tools like packet generators hping3 and t50 on Kali Linux.

The goal for the tutorial is to explain:
* How to create DDoS attack simulations
* My actual experience with doing this - testing banks, etc.
* Evaluate how good is this, value proposition for you
\end{quote}

{\small Can be found at \link{https://ripe72.ripe.net/wp-content/uploads/presentations/32-simulated-ddos-ripe.pdf} or \\
\link{https://github.com/kramse/security-courses/tree/master/presentations/pentest/simulated-ddos-ripe}}


\slide{Firewalls og IPv6}

\begin{list1}
\item Læg mærke til forskellen mellem ARP og ICMPv6
\item Hvis det er muligt lav een regel der tillader adgang til services uanset protokol
\item NB: husk at aktivere IP forwarding når I skal lave en firewall
\end{list1}

See also:
\emph{Security in a Mixed IPv4 and IPv6 World}, presentation by me\\
\link{https://github.com/kramse/security-courses/tree/master/presentations/network/ipv6-security-in-mixed-v4-v6}


\slide{IPv6 is more/less secure than IPv4}

%\hlkimage{}{}

There are two big misconceptions about IPv6 security:

\begin{quote}
\begin{list2}
\item IPv6 is more secure than IPv4
\item IPv6 is less secure than IPv4
\end{list2}

\emph{Neither} are true. Both assume that comparing IPv6 security with IPv4 security is meaningful. It is not.
\end{quote}
Source: David Holder at, \link{https://blog.apnic.net/2019/03/18/common-misconceptions-about-ipv6-security/}


\slide{IPv6 is already in your network!}

\hlkimage{20cm}{ipv6_puzzle.png}
Picture from the IPv6 Act Now web site, RIPE NCC

\begin{list2}
\item You have both, you will keep on having both
\item Unless you have very strict control and turn one or the other OFF always, you have both IPv4 and IPv6 in your network!
\item My suggestion, realize IPv6 is here, take control
\end{list2}


\slide{ IPv6 Neighbor Discovery Protocol (NDP)}

\hlkimage{18cm}{ipv6-arp-ndp.pdf}

\begin{list1}
\item Address Resolution Protocol (ARP) is replaced
\item NDP expands on the ARP concept, similar command \verb+arp -an+ compared to \verb+ndp -an+
\item Can do some things we knew from DHCPv4 still DHCPv6 exist
\item {\bf Note ICMPv6 often need to be added to firewall rules for NDP!} {\myalert}
\end{list1}

\slide{ARP vs NDP}

So at the low level, near the hardware we have protocols connecting IP addreses with MAC addresses, Ethernet and Wi-Fi are commonly found

\begin{alltt}
\small
hlk@bigfoot:basic-ipv6-new$ arp -an
? (10.0.42.1) at{\bf 0:0:24:c8:b2:4c} on en1 [ethernet]
? (10.0.42.2) at 0:c0:b7:6c:19:b on en1 [ethernet]

hlk@bigfoot:basic-ipv6-new$ ndp -an
Neighbor                      Linklayer Address  Netif Expire    St Flgs Prbs
::1                           (incomplete)         lo0 permanent R
2001:16d8:ffd2:cf0f:21c:b3ff:fec4:e1b6 0:1c:b3:c4:e1:b6 en1 permanent R
fe80::1%lo0                   (incomplete)         lo0 permanent R
fe80::200:24ff:fec8:b24c%en1 {\bf 0:0:24:c8:b2:4c}      en1 8h54m51s  S  R
fe80::21c:b3ff:fec4:e1b6%en1  0:1c:b3:c4:e1:b6     en1 permanent R
\end{alltt}

\slide{ARP and NDP problems}

\begin{list2}
\item This mapping is used in your operating system, keep a dynamic ARP/neighbor cache -- a table
\item Switches map devices to ports -- tables
\item Routers remember your IP, so it can send responses back -- tables
\item A table has a maximum size! This can cause problems {\myalert}
\item This is all done without ANY security -- you can lie, attackers can lie
\item See ARP spoofing and a sample tool \link{https://en.wikipedia.org/wiki/ARP_spoofing} \\
and \link{https://en.wikipedia.org/wiki/DSniff}
\end{list2}

\slide{Similarities between IPv4 and IPv6 security}


\begin{list2}
\item Rogue DHCP servers can be done in both, plus false router advertisements in IPv6
\item MAC address overflow can be done in both
\item Unfiltered access can be abused
\item DNS spoofing can be abused
\item Sniffing unencrypted traffic is the same
\item MITM Attacks are the same {\myalert}
\item Application attacks are the same! Example Web attack over IPv4/IPv6 - often address family doesn't matter {\myalert}
\item Flooding attacks are possible at various places
\end{list2}

\slide{Disparities between IPv4 and IPv6 security, example}

%\hlkimage{}{}

\begin{quote}
The functionality provided by IPv6's Type 0 Routing Header can be exploited in order to achieve traffic amplification over a remote path for the purposes of generating denial-of-service traffic.  This document updates the IPv6 specification to deprecate the use of IPv6 Type 0 Routing Headers, in light of this security concern.
\end{quote}
Source RFC5095


\begin{list2}
\item Routing headers -- flexible, but hard to filter\\
Updated RFC8200 \emph{Internet Protocol, Version 6 (IPv6) Specification} recommend order for those
\item IPv6 Type 0 routing header, fixed\\
Deprecated officially in RFC 5095 \link{https://www.rfc-editor.org/rfc/rfc5095.txt}
\item Other stuff better specified, like RFC5722 -- see later
\end{list2}

We are relying on vendors to create updated software, but must install those updates


\slide{Operational Security Considerations for IPv6 Networks}

%\hlkimage{}{}

\begin{alltt}\small
Internet Engineering Task Force (IETF)                         É. Vyncke
Request for Comments: 9099                                         Cisco
Category: Informational                                  K. Chittimaneni
ISSN: 2070-1721
                                                                 M. Kaeo
                                                    Double Shot Security
                                                                  E. Rey
                                                                    ERNW
                                                             August 2021
         Operational Security Considerations for IPv6 Networks
\end{alltt}
Source: \link{https://www.rfc-editor.org/rfc/rfc9099.txt}

\begin{list2}
\item Fantastic reference
\item Another from RIPE NCC, 191 slides! IPv6 Security Training Course April 2021,\\
\link{https://www.ripe.net/support/training/material/ipv6-security/ipv6security-slides.pdf}
\end{list2}


\slide{Address planning -- helps security for both IPv4 and IPv6! }

%\hlkimage{}{}

\begin{quote}
IPv6 address allocations and overall architecture are important parts
of securing IPv6.  Initial designs, even if intended to be temporary,
tend to last much longer than expected.  Although IPv6 was initially
thought to make renumbering easy, in practice, it may be extremely
difficult to renumber without a proper IP Address Management (IPAM)
system.  [RFC7010] introduces the mechanisms that could be utilized
for IPv6 site renumbering and tries to cover most of the explicit
issues and requirements associated with IPv6 renumbering.

{\bf A key task for a successful IPv6 deployment is to prepare an
addressing plan.  Because an abundance of address space is available,
structuring an address plan around both services and geographic
locations allows address space to become a basis for more structured
security policies to permit or deny services between geographic
regions}.  [RFC6177] documents some operational considerations of
using different prefix sizes for address assignments at end sites.
\end{quote}
Source: RFC 9099

\begin{list2}
\item You have space, use it!
\end{list2}


\slide{Network Architecture and Address planning }

\hlkimage{8cm}{ipv6-linked-to-ipv4.png}
Source: picture from Surfnet Preparing and IPv6 Address Plan

\begin{list2}
\item Take the opportunity to re-design your network! Create a design, consider it green field, work towards it!
\item Use /127 for point-to-point links, add loopback addresses on routers, allows filtering of access to management
\item You can also make parts IPv6-only, Veronika McKillop at TROOPERS19 \emph{Microsoft IT (secure) journey to IPv6-only}\\
\link{https://troopers.de/troopers19/agenda/h7sv7v/}
\end{list2}


\slide{OpenBSD PF IPv6 NDP}
\begin{alltt}\footnotesize
# Macros: define common values, so they can be referenced and changed easily.
int_if=vr0
ext_if=vr2
tunnel_if=gif0
table <homenet6> { 2001:16d8:ffd2:cf0f::/64 }
set skip on lo0
scrub in all
# Filtering: the implicit first two rules are
block in all

# allow ICMPv6 for NDP
# server with configured IP address and router advertisement daemon running
pass in inet6 proto ipv6-icmp all icmp6-type neighbradv keep state
pass out inet6 proto ipv6-icmp all icmp6-type routersol keep state

# client which uses autoconfiguration would use this instead
#pass in inet6 proto ipv6-icmp all icmp6-type routeradv keep state
#pass out inet6 proto ipv6-icmp all icmp6-type neighbrsol keep state

...  probably not working AS IS
\end{alltt}



\slide{IPv4 + IPv6 serviceeksempel, med tables}

PF har tabeller og simpel regel til PF konfigurationsfilen kan give adgang til både IPv4 og IPv6:
\begin{alltt}\small
table <webservers> \{ 2001:7b8:3e4:72::20 10.0.72.20 \}
...
pass in on $ext_if proto tcp to <webserver> port http
\end{alltt}


\slide{Creating an Access Control List (ACL)}

%\hlkimage{}{}

\begin{alltt}
 (config)#ipv6 access-list RA-GUARD
 (config-ipv6-acl)#sequence 3 deny icmp any any router-advertisement
 (config-ipv6-acl)#sequence 6 permit ipv6 any any
 (config-ipv6-acl)#exit
 (config)#interface FastEthernet0/5
 (config-if)#ipv6 traffic-filter RA-GUARD in
\end{alltt}
Source: example copied from RIPE NCC IPv6 Security Training materials:\\
\link{https://www.ripe.net/support/training/material/ipv6-security/ipv6security-slides.pdf}

\begin{list2}
\item Best practice, and not that hard to do
\item ACL, filtering and firewalling will create longer lasting protection
\item Paired with a nice address plan you can easily put restrictions on traffic flow, without hurting functionaliy or the business
\item Does ANY client in ANY office NEEEEEED to connect to ANY UPS, Virtualisation and printer across the world ...
\end{list2}

\slide{Get IPv6 prefix! }

%\hlkimage{}{}

\begin{quote}
You can ask RIPE NCC for an IPv6 provider independent prefix, through a LIR -- I have a LIR!
\end{quote}

\begin{list2}
\item YOU can't request directly, but need to find a RIPE NCC member to request it\\
Hint: Zencurity Aps is a member
\item It will cost you about EUR 100 per year and you will get minimum /48
\item You can move this space from provider to provider\\
more easily than migrating from their IP space to some new providers space
\item You can have this announced via multiple providers -- redundancy
\item Read more about this at:\\
\link{https://www.ripe.net/manage-ips-and-asns/ipv6/request-ipv6/how-to-request-an-ipv6-pi-assignment}
\item If you want to play with IPv6 try an IPv6 tunnel broker like\\
\link{https://tunnelbroker.net/} or \link {https://www.route48.org/}
\end{list2}




\slide{Design a robust network Isolation and segmentation}

\begin{list1}
\item Hvad kan man gøre for at få bedre netværkssikkerhed?
\begin{list2}
\item Bruge switche - der skal ARP spoofes og bedre performance
\item Opdele med firewall til flere DMZ zoner for at holde
      udsatte servere adskilt fra hinanden, det interne netværk og
      Internet
\item Overvåge, læse logs og reagere på hændelser
\end{list2}
\item Husk du skal også kunne opdatere dine servere
\end{list1}



\slide{Concrete advice for enterprise networks}


\begin{list2}
\item Portscanning - start using portscans in your networks, verify how far malware and hackers can travel, and identify soft systems needing updates or isolation
\item Have separation -- anywhere, starting with organisation units, management networks, server networks, customers, guests, LAN, WAN, Mail, web, ...
\item Use Web proxies - do not allow HTTP directly except for a short allow list, \\
do not allow traffic to and from any new TLD
\item Use only your own DNS servers, create a pair of Unbound servers, \\
point your internal DNS running on Windows to these\\
Create filtering, logging, restrictions on these Unbound DNS servers\\
\link{https://www.nlnetlabs.nl/projects/unbound/about/} and also \link{https://pi-hole.net/}
\item Only allow SMTP via your own mail servers, create a simple forwarder if you must
\end{list2}

Allow lists are better than block list, even if it takes some time to do it

\slide{Capture data and logs!}


\begin{list2}
\item Run DNS query logs -- when client1 is infected with malware from domain malwareexample.com, then search for more clients i
nfected
\item Run Zeek and gather information about all HTTPS sessions -- captures certificates by default, and we can again search for
certificate related to malwareexample.com
\item Run network logging -- session logs in enterprise networks are GREAT \\
(country wide illegal logging is of course NOT)
\end{list2}

Make sure to check with employees, inform them!

\slide{DROP SOME TRAFFIC NOW}

\begin{list2}
\item Drop some traffic on the border of everything
\item Seriously do NOT allow Windows RPC across borders
\item Border here may be from regional country office back to HQ
\item Border may be from internet to internal networks
\item Block Windows RPC ports, 135, 137, 139, 445
\item Block DNS directly to internet, do not allow clients to use any DNS, fake 8.8.8.8 if you must internally
\item Block SMTP directly to internet
\item Create allow list for internal networks, client networks should not contact other client networks but only relevant server networks
\end{list2}

You DONT need to allow direct DNS towards internet, except from your own recursive DNS servers

If you get hacked by Windows RPC in 2022, you probably deserve it, sorry for being blunt

Best would be to analyze traffic and create allow lists, some internal networks to not need internet at all


\slide{Default permit}

%\hlkimage{}{}

One of the early implementers of firewalls Marcus J. Ranum summarized in 2005 The Six Dumbest Ideas in Computer Security \link{https://www.ranum.com/security/computer_security/editorials/dumb/} which includes the always appropriate discussion about default permit versus default deny.

\begin{quote}\small {\bf
\#1) Default Permit}\\
This dumb idea crops up in a lot of different forms; it’s incredibly persistent and difficult to eradicate. Why? Because it’s so attractive. Systems based on ”Default Permit” are the computer security equivalent of empty calories: tasty, yet fattening.

The most recognizable form in which the ”Default Permit” dumb idea manifests itself is in firewall rules. Back in the very early days of computer security, network managers would set up an internet connection and decide to secure it by turning off incoming telnet, incoming rlogin, and incoming FTP. Everything else was allowed through, hence the name ”Default Permit.” This put the security practitioner in an endless arms-race with the hackers.
\end{quote}


\begin{list2}
\item Allow all current networks today on all ports for all protocols \emph{is} an allow list \\
Which tomorrow can be split into one for TCP, UDP and remaining, and measured upon
\item Measure, improve, repeat
\end{list2}



\slide{We cannot do X}

\begin{quote}
We cannot block SMTP from internal networks, since we do not know for sure if vendor X equipment needs to send the MOST important email alert at some unspecific time in the future
\end{quote}

Cool, then we can do an allow list starting today on our border firewall:
\begin{alltt}
table <smtp-exchange> \{ $exchange1 $exchange2 $exchange3 \}
table <smtp-unknown> persist file "/firewall/mail/smtp-internal-unknown.txt"
# Regular use, allowed
pass out on egress inet proto tcp from smtp-echange to any port 25/tcp
# Unknown, remove when phased out
pass out on egress inet proto tcp from smtp-internal to any port 25/tcp
\end{alltt}

Year 0 the unknown list may be 100\% of all internal networks, but new networks added to infrastructure are NOT added, so list will shrink -- evaluate the list, and compare to network logs, did networks send ANY SMTP for 1,2,3 years?


\slide{Anti-pattern dobbelt NAT i eget netværk}

\hlkimage{16cm}{nat-double.pdf}

\begin{list1}
\item Det er nødvendigt med NAT for at oversætte traffik der sendes videre
ud på internet.
\vskip 1cm
\item Der er ingen som helst grund til at benytte NAT indenfor eget netværk!
\end{list1}

\slide{Pattern use IDS to get flow, connections and data}

\begin{list1}
\item Use Intrusion Detection Systems - IDS
\item Angrebsværktøjerne efterlader spor
\item Det anbefales at have IDS og flow opsamling som minimum
\item Hostbased IDS - kører lokalt på et system og forsøger at
  detektere om der er en angriber inde
\item Network based IDS - NIDS - bruger netværket
\item Automatiserer netværksovervågning:
  \begin{list2}
  \item bestemte pakker kan opfattes som en signatur
\item analyse af netværkstrafik - FØR angreb
\item analyse af netværk under angreb - sender en alarm
  \end{list2}
\end{list1}


\slide{The Zeek Network Security Monitor}

The Zeek Network Security Monitor is not a single tool, more of a
powerful network analysis framework

\hlkimage{8cm}{zeek-ids.png}

\begin{quote}
While focusing on network security monitoring, Zeek provides a comprehensive platform for more general network traffic analysis as well. Well grounded in more than 15 years of research, Zeek has successfully bridged the traditional gap between academia and operations since its inception.
\end{quote}

Zeek is the tool formerly known as Bro, changed name in 2018. \link{https://www.zeek.org/}



\slide{Suricata IDS/IPS/NSM}
\hlkimage{6cm}{suricata.png}

\begin{quote}
Suricata is a high performance Network IDS, IPS and Network Security Monitoring engine.
\end{quote}

 \link{http://suricata-ids.org/}
 \link{http://openinfosecfoundation.org}

Workshop materials available:\\
{\small\link{https://github.com/kramse/security-courses/tree/master/courses/networking/suricatazeek-workshop}}



\slide{Anti-pattern blokering af ALT ICMP}

\begin{alltt}\small
# Simple stateful network firewall rules for allowing ICMP in IPv6 ICMPv6
# Allow ICMPv6 destination unreach
        $fwcmd6 add pass ipv6-icmp from any to any icmptypes 1
# Allow NS/NA/toobig (don't filter it out)
        $fwcmd6 add pass ipv6-icmp from any to any icmptypes 2
# Allow timex Time exceeded
        $fwcmd6 add pass ipv6-icmp from any to any icmptypes 3
# Allow parameter problem
        $fwcmd6 add pass ipv6-icmp from any to any icmptypes 4
# IPv6 ICMP - echo request (128) and echo reply (129)
        $fwcmd6 add pass ipv6-icmp from any to any icmptypes 128,129
# IPv6 ICMP - router solicitation (133) and router advertisement (134)
        $fwcmd6 add pass ipv6-icmp from any to any icmptypes 133,134
# IPv6 ICMP - neighbour discovery solicitation (135) and advertisement (136)
        $fwcmd6 add pass ipv6-icmp from any to any icmptypes 135,136
\end{alltt}

\begin{list1}
\item Lad være med at blokere for alt ICMP, så ødelægger du funktionaliteten i dit net
\end{list1}

\slide{ICMPv4 beskedtyper}

\begin{list1}
\item Type
\begin{list2}
\item 0 = net unreachable;
\item 1 = host unreachable;
\item 2 = protocol unreachable;
\item 3 = port unreachable;
\item 4 = fragmentation needed and DF set;
\item 5 = source route failed.
\end{list2}
\item Tillad disse ICMP types:
\begin{list2}
\item 3 Destination Unreachable
\item 4 Source Quench Message
\item 11 Time Exceeded
\item 12 Parameter Problem Message
\end{list2}
\end{list1}

\begin{list1}
\item Lad være med at blokere for alt ICMP, så ødelægger du funktionaliteten i dit net
\end{list1}

\slide{Anti-pattern blokering af DNS opslag på TCP}

\begin{list1}
\item Det bliver (er) nødvendigt med DNS opslag over TCP
\vskip 1cm
\item Store svar kræver TCP
\item Det betyder at firewalls skal tillade DNS opslag via TCP
\vskip 1cm
\item De nye forslag DNS over TLS (DoT) og DNS over HTTPS (DoH)
\item DNS kryptering bliver med TCP
\vskip 1cm
\item Anbefaling i enterprise netværk:\\
Brug en caching nameserver, således at det kun er den som kan lave DNS opslag ud i verden

\end{list1}



\slide{Hosting og internet-udbydere}

\hlkimage{17cm}{network-bgp-asn.png}

\begin{list2}
\item Data krydser mange internetudbydere
\item Det er stadig muligt at spoofe mange steder fra
\end{list2}


\slide{Routing and BGP Solutions }

\begin{list2}
\item Filtrering, ingress / egress:\\
"reject external packets that claim to be from the local net"
\item See also Reverse Path forwarding \link{https://en.wikipedia.org/wiki/Reverse-path_forwarding}
\item Routers and routing protocols must be more skeptical\\
Routing filters implemented everywhere, auth on routing protocols OSPF/BGP etc.
\item Has been recommended for some years, but not done in all organisations
\item BGP routing Resource Public Key Infrastructure RPKI
\item BCP38 is RFC2827: \emph{Network Ingress Filtering: Defeating Denial of Service Attacks which employ IP Source Address Spoofing}\\
\link{http://www.bcp38.info/}
\item \emph{Mutually Agreed Norms for Routing Security}, \link{https://www.manrs.org/}
\end{list2}


\slide{Mutually Agreed Norms for Routing Security (MANRS)}

\hlkimage{2cm}{MANRS_square.png}

\begin{quote}
  Mutually Agreed Norms for Routing Security (MANRS) is a global initiative, supported by the Internet Society, that provides crucial fixes to reduce the most common routing threats. 
\end{quote}

\begin{list1}
\item Problems related to incorrect routing information
\item Problems related to traffic with spoofed source IP addresses
\item Problems related to coordination and collaboration between network operators
\item {\small\link{https://www.manrs.org/isps/}}
\item {\small\link{https://www.manrs.org/wp-content/uploads/2018/09/MANRS_PDF_Sep2016.pdf}}
\end{list1}

\slide{Expected Actions in MANRS for Network Operators}

\begin{list1}
\item 1. Prevent propagation of incorrect routing information\\
Clear routing policies and systems for correctness, route import filters
\item 2. Prevent traffic with spoofed source IP addresses\\
Validate source address from end-users and infrastructure, use BCP38
\item 3. Facilitate global operational communication and coordination between network operators\\
Maintain contact information in databases like Whois, PeeringDB
{\small\link{https://www.peeringdb.com/}}

\item Advanced
\item 4. Facilitate validation of routing information on a global scale\\
Use RPSL {\small\link{ https://en.wikipedia.org/wiki/Routing_Policy_Specification_Language}}
\end{list1}


\slide{RPKI Testing}

\hlkimage{9cm}{rpki-test.png}

\begin{list2}
\item Check your own networks! Ask your ISP to check RPKI\\
\link{https://sg-pub.ripe.net/jasper/rpki-web-test/}
\item Read more about RPKI at:\\
\link{https://www.ripe.net/manage-ips-and-asns/resource-management/rpki}
\end{list2}


\end{document}
