\documentclass[Screen16to9,17pt]{foils}
%\documentclass[16pt,landscape,a4paper,footrule]{foils}
%\usepackage{zencurity-slides-troopers}
\usepackage{zencurity-slides}


% Abstract
% The Newcomers' Guide to Networking: Textbooks for the Next Generation×
% How can we attract more newcomers to our networking communities? I work in education, and I learnt about networking the old school way in the 1990s, with large 500 page textbooks, and this is still sometimes the case. Can we crowdsource interesting and useful sources of knowledge for the next generation? Can we create a "newcomers' guide to networking"?


% Notes


\addbibresource{/home/user/projects/books/firewall-publications/texfiles/firewall-refs.bib}

\begin{document}

%\rm
\selectlanguage{english}
\mytitlepage{The Newcomers' Guide to Networking: \\\emph{Textbooks} for the Next Generation}{RIPE86 Community Plenary}


% \hlkprofiluk


\slide{Problem statement}

{\Large Your co-worker has a question: \emph{You work in networking}, so can you give me some pointers for my friend that wants to \emph{get into networking}?!}

\begin{quote}
How can we {\bf attract more newcomers} to our {\bf networking communities?} I work in {\bf education}, and I learnt about networking the {\bf old school way} in the 1990s, with large {\bf 500 page textbooks}, and this is still sometimes the case. Can we {\bf crowdsource interesting and useful sources of knowledge for the next generation}? Can we create a {\bf "newcomers'} guide to networking"?
\end{quote}

\begin{list2}
\item A small call for resources, send me your best links, resources, books, papers, and interesting resources\\
I will add them here $====>$ \url{https://github.com/kramse/learning-paths}
\item May contain opnionated content, beware
\item BTW \emph{my slides are boring!}
\end{list2}


\slide{First of all Thank you!}

\begin{alltt}\scriptsize
0                   1                   2                   3
0 1 2 3 4 5 6 7 8 9 0 1 2 3 4 5 6 7 8 9 0 1 2 3 4 5 6 7 8 9 0 1
+-+-+-+-+-+-+-+-+-+-+-+-+-+-+-+-+-+-+-+-+-+-+-+-+-+-+-+-+-+-+-+-+
|Version|  IHL  |Type of Service|          Total Length         |
+-+-+-+-+-+-+-+-+-+-+-+-+-+-+-+-+-+-+-+-+-+-+-+-+-+-+-+-+-+-+-+-+
|         Identification        |Flags|      Fragment Offset    |
+-+-+-+-+-+-+-+-+-+-+-+-+-+-+-+-+-+-+-+-+-+-+-+-+-+-+-+-+-+-+-+-+
|  Time to Live |    Protocol   |         Header Checksum       |
+-+-+-+-+-+-+-+-+-+-+-+-+-+-+-+-+-+-+-+-+-+-+-+-+-+-+-+-+-+-+-+-+
|                       Source Address                          |
+-+-+-+-+-+-+-+-+-+-+-+-+-+-+-+-+-+-+-+-+-+-+-+-+-+-+-+-+-+-+-+-+
|                    Destination Address                        |
+-+-+-+-+-+-+-+-+-+-+-+-+-+-+-+-+-+-+-+-+-+-+-+-+-+-+-+-+-+-+-+-+
|                    Options                    |    Padding    |
+-+-+-+-+-+-+-+-+-+-+-+-+-+-+-+-+-+-+-+-+-+-+-+-+-+-+-+-+-+-+-+-+
\end{alltt}

\begin{list2}
\item Thank you to all the authors of the books I have learnt from. Thank you all that wrote RFCs, I enjoy them a lot
\item Unfortunately they are not beginner friendly, if I was a newcomer today I would probably NOT read them
\item Many resources are very structured, but become a bit boring, sorry
\end{list2}

\slide{Prerequisite knowledge}

\hlkimage{6cm}{nmap-logo.png}
{\Large Plan: You want to learn Port Scanning and Nmap! \\
Usefull in both defense and attack scenarios!}

\begin{list2}
\item Knowledge level: What is a port scanner\\
Need to know TCP/IP, IP address, ports and services -- example HTTP 80/tcp, TCP session setup
\end{list2}

So a newcomer should get this sorted out first, otherwise they cannot understand what Nmap does, and output returned

\slide{Skills are needed}

\hlkimage{9cm}{nmap-zenmap.png}

\begin{list2}
\item Skills level: Running a port scanner\\
Need to have operating system -- luckily Nmap supports Mac, Windows, Linux, ...
\item My recommendation: create a virtual machine with Kali Linux\\
BTW ... what is a Virtual Machine?! And so it keeps on
\end{list2}


\slide{Small set of Recommended networking technologies to learn}

So to accomplish the goal of using Nmap efficiently you need some basics

Networking: Basic Protocols from the Internet Protocols suite IP/TCP, or TCP/IP
\begin{list2}
\item Network Layer: Ethernet, Address Resolution Protocol (ARP), IPv4 and ICMP\\
Later add Wi-Fi and IPv6
\item Transport Layer: Transmission Control Protocol (TCP) and User Datagram Protocol (UDP)
\item Common upper layer: Dynamic Host Configuration Protocol (DHCP), Domain Name System (DNS),
Hypertext Transfer Protocol (HTTP)\\
Later add the encrypted/secure versions like Hypertext Transfer Protocol Secure (HTTPS) which uses Transport Layer Security (TLS)
\end{list2}

Enough to be able to go to other resources, and a connected path from ARP to TLS


\slide{What am I looking for -- easier entry ways into networking}


\hlkimage{13cm}{1-dns-hierarchy.png}

{\small Source: \emph{The DNS hierarchy} by Julia Evans}\\
\url{https://wizardzines.com/zines/dns/samples/1-dns-hierarchy.png} {\color{red}\faHeart}


\slide{Tools that help present real world data}

\hlkimage{16cm}{edudig.png}
\begin{list2}
\item at the DNS hackathon I saw this cool tool, EduDIG \url{https://edudig.se/} {\color{red}\faHeart}
\end{list2}


\slide{Books that are fun and educational}

\hlkimage{13cm}{how-the-internet-works.jpg}
{\small Source: \emph{How the Internet Really Works
An Illustrated Guide to protocols, privacy, censorship, and governance} by catnip19}\\
\url{https://catnip.article19.org/} {\color{red}\faHeart}

\slide{Real books are also welcome}


\hlkimage{6cm}{PracticalPacketAnalysis3E_cover.png}

\begin{list2}
\item \emph{Practical Packet Analysis - Using Wireshark to Solve Real-World Network Problems}, 3rd edition 2017, 368 pp\\
Chris Sanders ISBN: 9781593278021
\end{list2}

\slide{Send me your best resources for newcomers}

\begin{list2}
\item Send your best links and resource to me \verb+hlk@kramse.org+
\item I will collect it and probably make an \emph{awesome list} -- there are awesome lists for just about any topic on Github
\item Github repo added \url{https://github.com/kramse/learning-paths} -- pull requests welcome
\end{list2}

I am also very happy to discuss learning with you in breaks, I am here until friday.

\end{document}
