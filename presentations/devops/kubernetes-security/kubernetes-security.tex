\documentclass[Screen16to9,17pt]{foils}
%\documentclass[16pt,landscape,a4paper,footrule]{foils}
\usepackage{zencurity-slides}

% Kubernetes Security

% Vi ser på Kubernetes med sikkerhedsbrillerne på.

% Kubernetes er blevet en af de mest populære cloud teknologier både til self-hosted og hos cloud leverandører. Vi vil i dette foredrag se på infrastrukturen i cloud med Kubernetes som eksempel. Det er emner som netværksovervågning, stresstest, muligheder for beskyttelse og isolering samt forensic og opklaring af hændelser.

% Deltagerne vil efterfølgende kunne opsætte miljø tilsvarende underviserens og træne sikkerhed i et lukket miljø.

% Fokus vil være på best current practice med rigeligt med referencer til dokumenter og teknisk know-how, hvordan det kan gøres hos jer selv i praksis bagefter.

% Målgruppe: alle der er interesserede i overvågning af cloud, men primært Kubernetes.

% Varighed: 4 timer inkl pauser

% Nøgleord:
% Cloud security, DDoS protection, load balancing, logs and audit, cloud monitoring, performance testing

% Dato: Mandag 6. februar kl. 17.00-21.00
% Sted: Online. Direkte link bliver sendt pr. mail på dagen.
% Pris: Gratis for medlemmer af PROSA. 525 kr. for ikke-medlemmer



% Husk
% Ideer til indledning og struktur, emner i min præsentation

% Describe €THING and then security settings and features for €THING
% Make tools stand out with font awesome icon "tools"
% https://fontawesome.com/v5/icons/tools?style=regular&s=solid&f=classic

% Production K8s
% https://learning.oreilly.com/library/view/production-kubernetes/9781492092292/

% Case: Cheesehub, husk Securing Cheesehub paper og mine valg til min installation

\begin{document}

%\rm
\selectlanguage{english}

\mytitlepage
{PROSA Kubernetes Security}
{A holistic approach to running K8s in production}
\LogoOn

\slide{Time schedule}

\begin{list2}
\item 17:00 - 18:15  Introduction and basics\\

\item 30min break  Eat and mingle, hang around, get coffee/tea\\

\item 18:45 - 19:30 45min Walking through the stack\\

\item 15min break\\

\item 19:45 -20:30 45min Protection, building it secure and robust \\

\item 20:30 - 21:00 playtime, questions, demo, discussion
\end{list2}

\hlkprofiluk

\slide{Goals for today}
\vskip 1 cm

\hlkimage{6cm}{thomas-galler-hZ3uF1-z2Qc-unsplash.jpg}

\begin{list2}
\item Create an understanding of Kubernetes attack surface
\item Get an idea of the work needed to create a secure Kubernets deployment
\item Present some parts of the solution -- including specific tools \faWrench
\item Show a Kubernetes lab and run some tools
\item Point you towards resources, so you can get started with Kubernetes more securely
\item Lay out a plan for a modern featureful reasonably secure K8s bare-metal install
\end{list2}

\slide{Security is a Concern in Kubernetes Deployments}

\hlkimage{15cm}{the_new_stack_k8s_top_security.png}

Source: \link{https://thenewstack.io/top-challenges-kubernetes-users-face-deployment/}

\slide{I come from Security -- Security engineering a job role}

\begin{alltt}\small
On any given day, you may be challenged to:
        Create new ways to solve existing production security issues
        Configure and install firewalls and intrusion detection systems
        Perform vulnerability testing, risk analyses and security assessments
        Develop automation scripts to handle and track incidents
        Investigate intrusion incidents, conduct forensic investigations and incident responses
        Collaborate with colleagues on authentication, authorization and encryption solutions
        Evaluate new technologies and processes that enhance security capabilities
        Test security solutions using industry standard analysis criteria
        Deliver technical reports and formal papers on test findings
        Respond to information security issues during each stage of a project’s lifecycle
        Supervise changes in software, hardware, facilities, telecommunications and user needs
        Define, implement and maintain corporate security policies
        Analyze and advise on new security technologies and program conformance
        Recommend modifications in legal, technical and regulatory areas that affect IT security
\end{alltt}

Source: \url{https://www.cyberdegrees.org/jobs/security-engineer/}\\
also
\url{https://en.wikipedia.org/wiki/Security_engineering}



\slide{My Intentions}

\hlkimage{7cm}{thomas-drouault-IBUcu_9vXJc-unsplash.jpg}

I suspect you want to work with Kubernetes, maybe you already do, but:
\begin{list2}
\item You are responsible for all of it
\item You don't have the resources, knowledge, time, etc. for getting to know it all
\item You have to create an architecture, as well as implement it
\item ... and monitor it, ... and secure it
\end{list2}
My intention is to be your sparring partner today, list all the parts I think must be considered.

\slide{Target Audience}

\hlkimage{9cm}{nesa-by-makers-IgUR1iX0mqM-unsplash.jpg}

\begin{quote}

\end{quote}

\begin{list2}
\item Overall target audience: operators, and developers becoming operators -- devops
\item Maybe secdevops - security personnel that needs some scalable systems running on K8s
    - like myself
\item People being told to setup a K8s cluster -- for some app
\end{list2}


\slide{Disclaimer: I am not an expert in EVERYTHING Kubernetes}

\begin{quote}
\large \bf I am not an EXPERT \\
-- but I have read a lot and watched youtube videos. \smiley
\end{quote}

\begin{list1}
\item Kubernetes is such a big ecosystem
\item The list is long, and an incomplete list contains:
\begin{list2}
\item Kubernetes core components -- software written by the project RBAC etc.
\item Kubernetes supported container technologies, docker etc.
\item Kubernetes dependencies -- etcd software used for storing data
\item Operating systems -- and that lies below, storage
\item Kubernetes networking providers -- multiple exist, which one to choose -- like Cilium
\item Monitoring solutions -- logging solutions
\end{list2}
\item and on top all the applications in and around Kubernetes -- NGINX, PostgreSQL, ...
\end{list1}

\slide{Why are we here today then?!}
\hlkrightimage{10cm}{eugen-str-CrhsIRY3JWY-unsplash.jpg}

We are here to get an overview:
\begin{list2}
\item Identify what we \emph{need} for a Kubernetes production environment
\item Consider more parts of the picture than \emph{just Kubernetes core}
\item Find a solution that works, with some example technologies
\item Provide a kind of check list for your deployment
\end{list2}

\vskip 5mm
\begin{center}
In general I want to take a more holistic approach to\\
Kubernetes Security which I hope can help you
\end{center}


\slide{More Disclaimers}

\begin{list2}
\item Disclaimer: We cover security today\\
We will not cover all parts of K8s  - only the ones that we need for security, as few as possible.\\
This slideshow is not a replacement for the official docs, and we also recommend multiple\\ resources books, articles and papers detailing specific parts

\item Disclaimer: Deprecation\\
Things are built, and later deprecated. Even while preparing this a few things were deprecated! Core things even change from time to time, which is good. In general I don't trust older references, but verify with newest official docs

\item Disclaimer: Single tenant only\\
Running a K8s cluster for your organization is hard enough, I don't claim I could run a production multi-tenancy setup or public cloud YMMV
\end{list2}

\slide{Materials -- where to start}

\begin{list2}
\item This presentation -- slides for today, start here
\item \emph{Kubernetes: Up and Running}, 3rd Edition, Brendan Burns, Joe Beda, Kelsey Hightower,
August 2022\\
Introduces containers and Kubernetes, books in 3rd ed also has been fixed and updated
\item \emph{Container Security}, (CS) Liz Rice, O'Reilly, Apr 2020\\
A classic text about containers and a must read for everyone
\item \emph{Networking and Kubernetes: A Layered Approach}, James Strong, Vallery Lancey, O'Reilly, 2021\\
K8s uses a lot of intricate networking -- great for understanding this more
\item \emph{Learn Kubernetes Security} (LKS), Kaizhe Huang , Pranjal Jumde, Packt, July 2020\\
This is a very complete and detailed view at K8s security, covering many many parts
\item Advanced security practioners could perhaps enjoy security focused books like:\\
\emph{Hacking Kubernetes: Threat-Driven Analysis and Defense}, Andrew Martin, Michael Hausenblas, O'Reilly, October 2021
\end{list2}

\slide{Creating a lab is not expensive}

\hlkrightimage{8cm}{internet-in-a-box.jpeg}

I use ordinary IT-equipment like switches, PCs and servers, with some additional 10Gbps network devices -- nothing really special running Debian Linux

The main purposes for showing this box, is to show
\begin{list2}
\item These are standard network devices, bought the Arista 7150 24-port 10G used on ebay
\item This specific switch does VXLAN VTEP and BGP, which I use in my labs
\item OpenBSD which I use for routing also has built-in VXLAN and BGP
\end{list2}

You should have similar (or better) devices in your production network, and they can be
configured to do a LOT more than you use them for right now


\slide{Before your first K8s Production Deployment}

%\hlkimage{}{}


\begin{list2}
\item Run K8s locally, as in your laptop!
\item Get it up and running, deploy a sample container
\item Configure \faWrench\ \verb+kubectl+
\end{list2}

I don't remember when I ran K8s first, or how many times I have run it locally or in various example labs on the internet. Recently I ran some K8s BGP labs in my browser

Trying out K8s on your laptop:

\begin{list2}
\item Smallest demo with single node minikube running inside Debian 11
    "start and run a hello world"
\item Next one might be multiple VMs running K8s nodes -- multi node with OpenBSD VM for BGP
\item I can run all of this inside Qubes as HVM
\end{list2}


\slide{Main K8s Components}

\begin{list2}
\item Interfaces API
\item Ports and Services
\end{list2}



\slide{K8s objects}
\begin{list2}
\item like Service accounts
\end{list2}


\slide{From low level outside}

Starting from the bottom and upwards - keeping it short on purpose:
\begin{list2}
\item Hardware - DRAC, ILO, KVM, IPMI etc.
\item Connections - how to connect your clusters\\
- find some blueprints, add management network, show complete drawing!
\item Have some jump host, OpenBSD with wireguard VPN, show it works - *demo*
\item Consider access from the outside and inwards
\end{list2}

Now you have built a cluster -- great, is it secure? Does it even work, can you run it?


\slide{Document or it didn't happen}

\hlkimage{8cm}{mythical-man-month-documentary.png}

\begin{list2}
\item Make sure to document!
\item There are so many parts, so many decisions, soo many places thing can mess up
\item Have a minimum of documentation -- and the YAML files are not enough
\item Hint:  you can read the Mythical Man Month at \link{https://archive.org/details/MythicalManMonth}
\end{list2}

\slide{Why Have Formal Documents}

\hlkimage{12cm}{mythical-formal-documents.png}
Source: \emph{The Mythical Man-Month (Anniversary Edition)}
by Frederick P. Brooks Jr.

\begin{list2}
    \item
\end{list2}

\slide{Role-based Access Control (RBAC)}

\begin{quote}
In computer systems security, {\bf role-based access control (RBAC)}[1][2] or role-based security[3] is an approach to restricting system access to unauthorized users. It is used by the majority of enterprises with more than 500 employees,[4] and can implement mandatory access control (MAC) or discretionary access control (DAC).

Role-based access control (RBAC) is a policy-neutral access-control mechanism defined around {\bf roles and privileges}. The components of RBAC such as role-permissions, user-role and role-role relationships make it simple to perform user assignments. A study by NIST has demonstrated that RBAC addresses many needs of commercial and government organizations[citation needed]. RBAC can be used to facilitate administration of security in large organizations with hundreds of users and thousands of permissions. Although RBAC is different from MAC and DAC access control frameworks, it can enforce these policies without any complication.
\end{quote}
Quote from \url{https://en.wikipedia.org/wiki/Role-based_access_control}

\slide{Role-based Access Control (RBAC), \emph{Computer Security} Bishop}

\hlkimage{12cm}{bishop-rbac.png}

\slide{New to RBAC?}

To get an idea about roles, permissions within software projects, you can use Github as an example.

\begin{list2}
\item
Go to GitHub web page:\\
\link{https://help.github.com/en/articles/access-permissions-on-github}
\item Follow links to other pages, like:\\
\link{https://help.github.com/en/articles/permission-levels-for-an-organization}
\item Your K8s deployment might need more than this, but it is a nice starting point
\end{list2}


\slide{RBAC}

\begin{list2}
\item HK p205 simple RBAC example with role
also check official docs
\item Create my own, and use that
\end{list2}



\faWrench\ \verb+rbac-view+ HK p210


\slide{Open Policy Agent (OPA)}

\begin{list2}
\item
\end{list2}
HK p 216 +/- OPA directly or gatekeeper

One slide with Least Privilege med ref til 1975 artiklen Saltzer and Schroeder\\
\url{https://en.wikipedia.org/wiki/Principle_of_least_privilege}\\
\url{https://en.wikipedia.org/wiki/Saltzer_and_Schroeder%27s_design_principles}

\slide{Creating our cluster}
Kubeadm - selected for this project, works great on Debian


\slide{First users}
\begin{list2}
\item
\end{list2}

\slide{Recommended Users}

\begin{list2}
\item
\end{list2}
What to create, how many people, how large a group do you have?

Smallest:
Create admin group even if only one admin! Ref RIPE database roles

Medium:
Start out with a large group of all admins
Split into a few groups, maybe some overlap
Don't give all users all authorizations

\slide{Soft Multitenancy}

\begin{list2}
\item
\end{list2}
as described in HK chapter 7 p 177
easier than hard multitenancy

Goal prevent some avoidable accidents and problems, breaking other groups services, but easier to maintain


\slide{A note about service accounts}

\begin{list2}
\item
\end{list2}
Where do they "login" from, where do they request from! Only from inside the cluster
Note to self: check logs later

\slide{Interactions}

\begin{list2}
\item What must be allowed between components\\
Figure 3.1 from LKS perhaps, or 3.3
\end{list2}

\slide{Network parts}
\begin{list2}
\item
Container Network Interface (CNI)
\link{https://github.com/containernetworking/cni}
\end{list2}


\slide{CNI plugin alternatives}

Describe two:
\begin{list2}
\item Calico - available multiple places
\item Cilium - better security? encryption, I have followed Cilium for a while, choose one
\item What does changing CNI plugin bring? and importantly what do you need!
\end{list2}


\slide{Compare Calico and Cilium}

\begin{list2}
\item \link{https://platform9.com/blog/the-ultimate-guide-to-using-calico-flannel-weave-and-cilium/}

\item Others: WeaveNet, Flannel ...

\item Maybe figure 2.9 from LKS in color or benchmark.png
\end{list2}

\slide{Installing Cilium}

\begin{minted}[fontsize=\footnotesize]{shell}
helm repo add cilium https://helm.cilium.io/

API_SERVER_IP=<your_api_server_ip>
# Kubeadm default is 6443
API_SERVER_PORT=<your_api_server_port>
helm install cilium cilium/cilium --version 1.12.5 \
    --namespace kube-system \
    --set kubeProxyReplacement=strict \
    --set k8sServiceHost=${API_SERVER_IP} \
    --set k8sServicePort=${API_SERVER_PORT}
\end{minted}

\begin{list2}
\item
\end{list2}
Note: I ended up deciding to run without the kube-proxy, only Cilium

\slide{After install Cilium}

\begin{alltt}
root@gouda01:~#  kubectl get po -n kube-system
NAME                               READY   STATUS    RESTARTS   AGE
cilium-5dt9j                       1/1     Running   0          3m40s
cilium-mz4qm                       1/1     Running   0          3m40s
cilium-operator-5d5dd5b4f6-75nq9   1/1     Running   0          3m40s
cilium-operator-5d5dd5b4f6-mmgcd   1/1     Running   0          3m40s
coredns-787d4945fb-5pddx           1/1     Running   0          4m54s
coredns-787d4945fb-sjbdc           1/1     Running   0          4m54s
etcd-gouda01                       1/1     Running   6          5m6s
kube-apiserver-gouda01             1/1     Running   6          5m8s
kube-controller-manager-gouda01    1/1     Running   3          5m6s
kube-scheduler-gouda01             1/1     Running   8          5m6s
\end{alltt}


\slide{Cilium CLI tool}

\begin{alltt}
root@gouda01:/home/hlk/projects# cilium status
    /¯¯\
 /¯¯\__/¯¯\    Cilium:         OK
 \__/¯¯\__/    Operator:       OK
 /¯¯\__/¯¯\    Hubble:         disabled
 \__/¯¯\__/    ClusterMesh:    disabled
    \__/

Deployment        cilium-operator    Desired: 2, Ready: 2/2, Available: 2/2
DaemonSet         cilium             Desired: 2, Ready: 2/2, Available: 2/2
Containers:       cilium             Running: 2
                  cilium-operator    Running: 2
Cluster Pods:     2/2 managed by Cilium
Image versions    cilium             quay.io/cilium/cilium:v1.12.5@sha256:06ce2b0a0a472e73334a7504ee5c5d8b2e2d7b72ef728ad94e564740dd505be5: 2
                  cilium-operator    quay.io/cilium/operator-generic:v1.12.5@sha256:b296eb7f0f7656a5cc19724f40a8a7121b7fd725278b7d61dc91fe0b7ffd7c0e: 2
\end{alltt}

So now we have a firewall for our cluster!

\slide{Optional Remove Kube Proxy}


\begin{alltt}
$ kubectl -n kube-system delete ds kube-proxy
$ # Delete the configmap as well to avoid kube-proxy being reinstalled during a kubeadm upgrade (works only for K8s 1.19 and newer)
$ kubectl -n kube-system delete cm kube-proxy
$ # Run on each node with root permissions:
$ iptables-save | grep -v KUBE | iptables-restore
\end{alltt}



\begin{list2}
    \item \link{https://docs.cilium.io/en/v1.12/gettingstarted/kubeproxy-free/}
\end{list2}

\slide{Picture from Cilium docs}


insert nice picture here showing allowed and blocked traffic

% Perhaps this one:
% https://cilium.io/blog/2018/09/19/kubernetes-network-policies/


\link{https://docs.cilium.io/en/stable/gettingstarted/k8s-install-default/}

\slide{External IPs}

%\hlkimage{}{}

\begin{quote}
apiVersion: "cilium.io/v2alpha1"
kind: CiliumLoadBalancerIPPool
metadata:
  name: "blue-pool"
spec:
  cidrs:
  - cidr: "10.0.47.0/24"
  - cidr: "2a06:d380:0:85::0/64"
\end{quote}

\begin{list2}
\item \link{https://docs.cilium.io/en/latest/network/lb-ipam/}
\item \verb+10.0.42.254/29+ is my home
\item \verb+2a06:d380:0:85::0/64+ is from my own LIR
\end{list2}

Source:
\link{https://sue.eu/blogs/expose-loadbalanced-kubernetes-services-with-bgp-cilium/}
\link{https://nicovibert.com/2022/07/21/bgp-with-cilium/}
https://isovalent.com/resource-library/labs/ - has a lab even Cilium Host Firewall


\slide{Nice Editor for Policies}

\link{https://editor.cilium.io/}

\slide{Connectivity Testing}

\verb+cilium connectivity test+



\begin{alltt}
root@gouda01:~# cilium connectivity test
ℹ️  Monitor aggregation detected, will skip some flow validation steps
✨ [kubernetes] Deploying echo-same-node service...
✨ [kubernetes] Deploying DNS test server configmap...
✨ [kubernetes] Deploying same-node deployment...
✨ [kubernetes] Deploying client deployment...
✨ [kubernetes] Deploying client2 deployment...
✨ [kubernetes] Deploying echo-other-node service...
✨ [kubernetes] Deploying other-node deployment...
⌛ [kubernetes] Waiting for deployments [client client2 echo-same-node] to become ready...
⌛ [kubernetes] Waiting for deployments [echo-other-node] to become ready...
connectivity test failed: waiting for deployment echo-other-node to become ready has been interrupted: context deadline exceeded
root@gouda01:~#
\end{alltt}

\begin{alltt}
root@gouda01:~# kubectl get po -n cilium-test
NAME                                                    READY   STATUS    RESTARTS   AGE
client-6f6788d7cc-tjz4m                                 1/1     Running   0          88s
client2-bc59f56d5-8r6lh                                 1/1     Running   0          88s
echo-a-6c79c8b946-p2rg4                                 1/1     Running   0          20m
echo-b-7448778bfb-kvgb5                                 1/1     Running   0          20m
echo-b-host-b5896c6cf-php6w                             1/1     Running   0          20m
echo-other-node-5c58b8cf8b-h58rf                        0/2     Pending   0          87s
echo-same-node-9f8754876-n5j5x                          2/2     Running   0          88s
host-to-b-multi-node-clusterip-645c86f859-8sgkd         0/1     Pending   0          20m
host-to-b-multi-node-headless-7647655fd7-v9b4f          0/1     Pending   0          20m
pod-to-a-599fcbff49-rxx7t                               1/1     Running   0          20m
pod-to-a-allowed-cnp-cbcdd54f5-cgbt2                    1/1     Running   0          20m
pod-to-a-denied-cnp-7cd474c7b5-hr6dm                    1/1     Running   0          20m
pod-to-b-intra-node-nodeport-bb96b9bcf-4cxlv            1/1     Running   0          20m
pod-to-b-multi-node-clusterip-6897f7b4b5-g7xd9          0/1     Pending   0          20m
pod-to-b-multi-node-headless-fff844f67-fhwfj            0/1     Pending   0          20m
pod-to-b-multi-node-nodeport-cb9685695-cb6jp            0/1     Pending   0          20m
pod-to-external-1111-66588c5755-ff8kg                   1/1     Running   0          20m
pod-to-external-fqdn-allow-google-cnp-cc84ccb8f-g5pgw   1/1     Running   0          20m
\end{alltt}

\slide{K8s DNS -- CoreDNS}

\begin{list2}
\item and interaction with auth DNS
\end{list2}

Ref to loadbalancing, global load balancing, cloudflare?

HK p 44 CoreDNS with OPA can restrict some

\slide{Deploying Applications Securely}

\begin{list2}
\item
\end{list2}

\slide{K8s Security domains}

\begin{list2}
\item K8s master components
\item K8s worker components
\item K8s objects
\end{list2}

+ entities and threat actors end user, internal attacker, privileged attacker

Source: LKS p77-




\slide{Using RBAC and Namespaces to isolate}

\begin{list2}
\item HK Chapter 3 Container Runtime Isolation
\end{list2}


\slide{Use Resource Limits and network policies always}

\begin{list2}
\item Admission controllers
\end{list2}

Rule: always put a limit on it
LKS p 121 EventRateLimit


Rule: always specify a K8s Network Policy - be specific, who CAN access
make positive lists - as normally only a limited number of other resources will need access
- then also make sure to specify an egress policy
So both ingress and egress rules

HK chapter 5 p 134 Traffic Flow Control


Namespace resource quotas LKS p182

\slide{Network security in K8s}

\begin{list2}
\item Attack types DDoS slow and fast, volumetric and PPS based
\item Pro/cons with having another layer - manually configured, only allow few protocols and ports 80, 443, 8080
- take tables from firewall presentation
\end{list2}



\slide{Monitoring and updating}

\begin{list2}
\item
\end{list2}

\slide{Upgrades}

%\hlkimage{}{}

\begin{quote}

\end{quote}

You need to upgrade all parts, so remember which parts you use:
\begin{list2}
\item
\item Cilium upgrade
\end{list2}

\slide{}

%\hlkimage{}{}

\begin{alltt}
root@gouda01:~# cilium upgrade
🔮 Auto-detected datapath mode: tunnel
🚀 Upgrading cilium-operator to version quay.io/cilium/operator-generic:v1.12.5...
🚀 Upgrading cilium to version quay.io/cilium/cilium:v1.12.5...
⌛ Waiting for Cilium to be upgraded...

\end{alltt}

\begin{list2}
    \item
\end{list2}

Cilium recommend a pre-flight check -- pulls images beforehand:\\
\link{https://docs.cilium.io/en/v1.12/operations/upgrade/}

\slide{Resource monitoring and capacity planning}

\begin{list2}
\item Kubernetes dashboard, Metrics server
\end{list2}

\slide{Security monitoring and auditing}

Why isn't K8s secure?!
\begin{list2}
\item Include refs to LKS book chapter 13 CVEs, and HK p133 CVEs plus p 275 CVEs
- which ones would hurt our design and architecture?
\end{list2}

\slide{Tools that might help}

\begin{list2}
\item \faWrench\ \verb+Falco+ detect anomalous behaviour
\item \faWrench\ \verb+Prometheus+ \verb+grafana+ \verb+elasticsearch+ - usual suspects, keep it short
\item \faWrench\ \verb+Netflow+ \verb+Elastiflow+
\end{list2}

\slide{Network Security Monitoring}

\begin{list2}
\item IDS, host based IDS
\item HK chapter 9. Intrusion Detection
Anbefaler Suricata, Zeek - among others
\end{list2}


\slide{Monitoring K8s}

\begin{list2}
\item
\end{list2}

\slide{Monitoring K8s Networking}

\begin{list2}
\item Netværksovervågning
\item Cilium has Setting up Hubble Observability \\
\url{https://docs.cilium.io/en/stable/gettingstarted/hubble_setup/#hubble-setup}
\end{list2}

\slide{Isolation and Independence}

\begin{list2}
\item muligheder for beskyttelse og isolering
\end{list2}

\slide{Stress testing and DDoS}

\hlkimage{13cm}{penguinping-02-peak.png}

\begin{list2}
\item PenguinPing packet generator, my high speed packet generator
home page: \link{https://penguinping.org}
\item First versions are only about 230 lines of Lua code and implement basic command line to replace hping3
\item Built on top of MoonGen/libmoon \link{https://github.com/emmericp/MoonGen}
\end{list2}

\centerline{Extremely fast and allows easy customization}


\slide{Main Kubernetes Attack Surface}

\begin{list2}
\item
\end{list2}
Management

Front end -- internet connected parts

Supporting systems, DNS, NTP, etc.




\slide{Kubernetes Security is more than just the core}

\hlkimage{12cm}{rice-container-security-1.png}
Source: \emph{Container Security}, (CS) Liz Rice, O'Reilly, Apr 2020

\begin{list2}
\item Attacks from inside, attacks inside containers
\item Some just use up resources, some attacks others, some attack the system from the inside
\end{list2}

\slide{Analysis on Docker Hub malicious images: Attacks through public container images}

\hlkimage{10cm}{sysdig-malicious-images.png}
This article is relevant, talking about malicious docker images\\
\link{https://sysdig.com/blog/analysis-of-supply-chain-attacks-through-public-docker-images/}

\slide{Keeping Container Images secure}

\begin{list2}
\item \faWrench\ \verb+anchore+
Allow direct download from the internet into your cluster, typosquatting popular containers!

\item Suply chain in general
\end{list2}

\slide{Harden Container Images}

\begin{list2}
\item Change the goddamn passwords!\\
Container postgresql with user postgres and password *postgres*, REALLY!!!!!!1111

\item and NO MORE ROOT! Dont run as root, we realized this was bad in the 1990s!
Ref: HK chapter 8

\item CIS Docker Benchmarking also LKS p 132
\end{list2}

\slide{}

\hlkimage{14cm}{cis-docker-1.png}
Summary from: \link{https://www.aquasec.com/cloud-native-academy/docker-container/docker-cis-benchmark/} - recommendations from CIS Docker Benchmark

\begin{list2}
\item Latest version: CIS Docker Benchmark v1.5.0 - 12-28-2022
\end{list2}


\slide{Cloud Forensic}

\begin{list2}
\item
\end{list2}
forensic og opklaring af hændelser.


\slide{Benchmarking tools}

\begin{quote}
\faWrench\ \verb+Kube-bench+ is the industry-standard tool to automate checking Kubernetes compliance with the Center for Internet Security (CIS) Benchmark.

Kube-bench makes it easy for operators to check whether each node in their Kubernetes cluster is configured according to security best practices.
\end{quote}
Source: \link{https://info.aquasec.com/open-source}

\begin{list2}
\item CIS Kubernetes V1.24 Benchmark v1.0.0 - 09-21-2022 -- other versions exist
\item CIS Docker Benchmark v1.5.0 - 12-28-2022
\end{list2}

\slide{Benchmarking and keeping up to date}


% Try it out \faWrench\ \verb+kube-bench+ - make some bad changes, like LKS p95 allow anonymous authentication, show why layered defense works, and how kube-bench marks it. Another example token based access, initially we have token while installing, remember to remove!
\faWrench\ \verb+kube-hunter+

\slide{Let's talk about a case Cheesehub.dk}

%\hlkimage{}{}

\begin{quote}
{\bf CHEESE: Cyber Human Ecosystem of Engaged Security Education}

CHEESE (Cyber Human Ecosystem of Engaged Security Education) is an NSF-funded initiative to develop a learning ecosystem for cybersecurity education. The project consists of the following:

\begin{list2}
\item Easy to access, dynamic, web-based learning platform
\item Community-contributed catalog of cybersecurity training materials
\item Community-driven platform to request, develop, share, evaluate and learn about content
\item Learning ecosystem that is continuously updated
\end{list2}
\end{quote}

\begin{list2}
\item I found a system based on Kubernetes \\
\emph{CHEESE: Cyber Human Ecosystem of Engaged Security Education}\\
\link{https://docs.cheesehub.org/}
\item Educational sessions using containers being spun up for the students
\item Decided to give it a shot
\end{list2}


\slide{Kubernetes lab setup -- proof of concept}

\hlkimage{4cm}{hacklab-1.png}

\begin{list2}
\item Hardware: any modern PC with modern CPU and virtualisation\\
Don't forget to enable it in the BIOS
\item Software: your favourite Linux distribution, I choose Debian
\item Container software: pick your poison
\item Hacker software: Kali as a Virtual Machine \link{https://www.kali.org/}
\item Smallest Kubernetes: Minikube -  I run this on Debian 11
\item Production deployment probably would use better tools like \faWrench\ \verb+Kubeadm+
\end{list2}



\slide{Conclusion Kubernetes Security}

\hlkimage{4cm}{network-layers-2022.pdf}

\begin{list2}
\item It is hard!


\item HK chapter 10. has good advice too, including a small checklist for On-Premises Environments
links on p249 to Build K8s Bare-metal cluster with external access\\
\link{https://medium.com/swlh/on-premise-kubernetes-clusters-b36660ca6914}\\
\link{https://www.datapacket.com/blog/build-kubernetes-cluster}\\
\link{https://medium.com/@apiotrowski312/bare-metal-kubernetes-with-helm-rook-ingress-prometheus-grafana-6a74857cc74c}
\end{list2}


\myquestionspage


% ---- Advanced subjects --- below
\slide{Threat modelling -- maybe do your own threat modelling?}
- my presentation based on "minimum" and Best Current Practice

Make sure to do recommended basic security before considering advanced attacks

\slide{Secrets in K8s vaults}
Various methods for keeping secrets -- considered a developer problem, sorry


Sysdig - advanced logging and inspection
sysdig and sysdig-inspect
https://github.com/draios/sysdig
https://github.com/draios/sysdig-inspect

\slide{High availability}

Moved to advanced subject, but of course important

\slide{Multi tenancy}

Moved to advanced subject, perhaps HK chapter 7 can help you


\slide{Sandboxing}
Moved to advanced since we might not have time
HK Chapter 3. Sandboxing

\slide{Container Forensics}
Moved to advanced - time constraints

\slide{Honeypots}
Moved to advanced - time constraints

\slide{Mutually Agreed Norms for Routing Security (MANRS)}

%\hlkimage{2cm}{MANRS_square.png}

\begin{quote}
  Mutually Agreed Norms for Routing Security (MANRS) is a global initiative, supported by the Internet Society, that provides crucial fixes to reduce the most common routing threats. 
\end{quote}
Source: {\small\link{https://www.manrs.org/wp-content/uploads/2018/09/MANRS_PDF_Sep2016.pdf}}

\begin{list2}
\item Problems related to incorrect routing information
\item Problems related to traffic with spoofed source IP addresses -- can easily happen in K8s
\item Problems related to coordination and collaboration between network operators
\item Also BCP38 RFC2827 \emph{Network Ingress Filtering: Defeating Denial of Service Attacks
which employ IP Source Address Spoofing}
\end{list2}

You should all ask your internet providers if they know about MANRS, and follow it. We should ask our government and institutions to support and follow MANRS and good practices for network on the Internet

\slide{Books and educational materials}

\begin{list2}


\item \emph{Kubernetes Security and Observability} Brendan Creane, Amit Gupta % Er på Oreilly.com

\end{list2}


\slide{Other sources}

\begin{list2}
\item
% May be a copy of the same from a company called "Dicy" or something
\item \emph{A Holistic Approach to Kubernetes Security and Compliance}\\
%\link{https://www.rapid7.com/info/holistic-approach-to-kubernetes-security-and-compliance/}

\end{list2}

\slide{Linux and Internet Resources}

Let's face it, K8s is Unix, so basic Linux and Internet Security resources help too:
\begin{list2}
\item \emph{The Linux Command Line: A Complete Introduction}, 2nd Edition\\
 by William Shotts, internet edition \link{https://sourceforge.net/projects/linuxcommand}
\item \emph{Gray Hat Hacking: The Ethical Hacker's Handbook}, 5. ed. Allen Harper and others ISBN: 978-1-260-10841-5
\item \emph{Defensive Security Handbook: Best Practices for Securing Infrastructure}, Lee Brotherston, Amanda Berlin ISBN: 978-1-491-96038-7 284 pages
\item \emph{Web Application Security}, Andrew Hoffman, 2020, ISBN: 9781492053118
\item \emph{Practical Packet Analysis, Using Wireshark to Solve Real-World Network Problems}
by Chris Sanders, 3rd ed, ISBN: 978-1-59327-802-1
\end{list2}

We teach using these books and others! Diploma in IT-security at KEA Kompetence\\
 \link{https://zencurity.gitbook.io/}

\end{document}
