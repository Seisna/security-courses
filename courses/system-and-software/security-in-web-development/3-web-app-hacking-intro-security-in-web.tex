\documentclass[Screen16to9,17pt]{foils}
\usepackage{zencurity-slides}
\externaldocument{security-in-web-development-exercises}
\selectlanguage{english}

\begin{document}

\mytitlepage
{3. Web Application Hacking Intro}
{Security in Web Development Elective, KEA}

\slide{Goals for today}

\hlkimage{6cm}{thomas-galler-hZ3uF1-z2Qc-unsplash.jpg}

Todays goals:
\begin{list2}
\item Web Application Hacking intro
\item Introduce some pentesting methods against web servers and web applications
\item Show a hacker lab and run some tools
\item Make you capable of investigating the niche by introducing good resources
\end{list2}

Photo by Thomas Galler on Unsplash


\slide{Plan for today}

\begin{list1}
\item Subjects
\item Web Application Security: Recon
\begin{list2}
\item Generic Network Fault Injection
\item Attacking Authentication
\item Session IDs
\item Common web application issues
\end{list2}
\item Exercises
\begin{list2}
\item  Try a few attacks in the JuiceShop with web proxy
\end{list2}
\end{list1}

\slide{Reading Summary}

\emph{Web Application Security}, Andrew Hoffman, 2020, ISBN: 9781492053118

\begin{list1}
\item Part I. Recon, chapters 2-8, very short chapters
\item 2. Introduction to Web Application Reconnaissance
\item 3. The Structure of a Modern Web Application
\item 4. Finding Subdomains
\item 5. API Analysis
\item 6. Identifying Third-Party Dependencies
\item 7. Identifying Weak Points in Application Architectur
\item 8. Part I Summary
\item Look at: Getting started with Burp Suite\\
\link{https://portswigger.net/support/getting-started-with-burp-suite}
\end{list1}



\slide{Agreements for testing networks}

\begin{quote}\small
Danish Criminal Code\\
Straffelovens paragraf 263 Stk. 2. Med bøde eller fængsel indtil 1 år og 6 måneder straffes den, der uberettiget skaffer sig adgang til en andens oplysninger eller programmer, der er bestemt til at bruges i et informationssystem.
\end{quote}

Hacking can result in:
\begin{list2}
\item Getting your devices confiscated by the police
\item Paying damages to persons or businesses
\item If older getting a fine and a record -- even jail perhaps
\item Getting a criminal record, making it hard to travel to some countries and working in security
\item Fear of terror has increased the focus -- so dont step over bounds!
\end{list2}

Asking for permission and getting an OK before doing invasive tests, always!

\slide{Er sikkerhedstest af webservere interessant?}

\hlkimage{10cm}{web-dynamic.png}

\begin{list1}
\item Sikkerhedsproblemer i netværk er mange
\item Kan være et krav fra eksterne - eksempelvis VISA PCI krav
\end{list1}



\slide{Hacker tools}
% måske til reference afsnit?
\hlkimage{3cm}{hackers_JOLIE+1995.jpg}

\begin{list2}
\item Everyone use similar tools, see also \link{http://www.sectools.org/}
\item Portscanning Nmap, Nping -- test ports and services, Nping is great for firewall admins \link{https://nmap.org}
\item Metasploit Framework -- service scanning, exploit development and execution \link{https://www.metasploit.com/}
\item Dedicated niche scanners -- wifi Aircrack-ng, web Burp suite, Nikto, Skipfish \link{http://portswigger.net/burp/}
\item Wireshark avanced network sniffing tool -- \link{https://www.wireshark.org/}
\item and scripting, PowerShell, Unix shell, Perl, Python, Ruby, \ldots
\end{list2}

Picture: Angelina Jolie, Hackers 1995

\slide{OSI Model and Internet Protocols}

\hlkimage{10cm,angle=90}{images/compare-osi-ip.pdf}


\slide{What happens now?}

\begin{list1}
\item Think like a hacker
\item Reconnaissance
\begin{list2}
\item ping sweep, port scan
\item OS detection -- TCP/IP or banner grabbing
\item Service scan -- rpcinfo, netbios, ...
\item telnet/netcat interact with services
\end{list2}
\item Exploit/test: Metasploit, Nikto, exploit programs
\item Cleanup/hardening not shown today, but:
\begin{list2}
\item Make a report or document findings
\item Change, improve and harden systems
\item Go through report with stakeholders, track progress
\item Update programs, settings, configurations, architecture
\end{list2}
\item You also need to show others that you are in control of security
\end{list1}



\slide{Wireshark can help you learn many protocols}

\hlkimage{13cm}{images/wireshark-http.png}

See also \link{https://en.wikipedia.org/wiki/Hypertext_Transfer_Protocol}

\slide{Primary HTTP mthods}

\begin{list2}
\item [GET]{\small
Requests a representation of the specified resource. Requests using GET should only retrieve data and should have no other effect. (This is also true of some other HTTP methods.)[1] The W3C has published guidance principles on this distinction, saying, "Web application design should be informed by the above principles, but also by the relevant limitations."[13] See safe methods below.}
\item [HEAD]{\small
Asks for the response identical to the one that would correspond to a GET request, but without the response body. This is useful for retrieving meta-information written in response headers, without having to transport the entire content.}
\item [POST]{\small
Requests that the server accept the entity enclosed in the request as a new subordinate of the web resource identified by the URI. The data POSTed might be, for example, an annotation for existing resources; a message for a bulletin board, newsgroup, mailing list, or comment thread; a block of data that is the result of submitting a web form to a data-handling process; or an item to add to a database.[14]}
\item [PUT]{\small
Requests that the enclosed entity be stored under the supplied URI. If the URI refers to an already existing resource, it is modified; if the URI does not point to an existing resource, then the server can create the resource with that URI.[15]}
\end{list2}

Source: \link{https://en.wikipedia.org/wiki/Hypertext_Transfer_Protocol}


\slide{Informationsindsamling}

\begin{list1}
\item Indsamling af informationer kan være aktiv eller passiv indsamling i forhold
  til målet for angrebet
\item {\bf passiv} kunne være at lytte med på trafik eller søge i databaser
  på Internet: google, whois, archive.org m.fl.

Eksempel: start Wireshark og browser på samme client

\vskip 1cm
\item {\bf aktiv indsamling} er eksempelvis at sende ICMP pakker og
  registrere hvad man får af svar, portscan m.v.\\

Eksempel: brug SSLScan programmet og udfør mange request mod en server\\
\verb+sslscan --ssl2 server+
\end{list1}

Check dit site med \link{http://www.ssllabs.com}

\slide{Whois systemet}

\begin{list1}
\item IP adresserne administreres i dagligdagen af et antal Internet
  registries, hvor de største er:
\begin{list2}
\item RIPE (Réseaux IP Européens)  \link{http://ripe.net}
\item ARIN American Registry for Internet Numbers \link{http://www.arin.net}
\item Asia Pacific Network Information Center \link{http://www.apnic.net}
\item LACNIC (Regional Latin-American and Caribbean IP Address
  Registry) - Latin America and some Caribbean Islands
  \link{http://www.lacnic.net}
\item AfriNIC African Internet Numbers Registry \link{http://www.afrinic.net}
\end{list2}
\item disse fem kaldes for Regional Internet Registries (RIRs) i
  modsætning til Local Internet Registries (LIRs) og National Internet
  Registry (NIR)
\end{list1}

\vskip 1cm
\centerline{Firefox add-on galore, brug dem - AS nummer, IP, whois, country}



\slide{HTTPS Everywhere}

\hlkimage{5cm}{HTTPS_Everywhere_new_logo.jpg}
\begin{quote}
HTTPS Everywhere is a Firefox extension produced as a collaboration between The Tor Project and the Electronic Frontier Foundation. It encrypts your communications with a number of major websites.
\end{quote}

\centerline{\link{http://www.eff.org/https-everywhere}}

Today most browsers ensure you use HTTPS as much as possible, along with HTTP Strict Transport Security (HSTS)
\link{https://en.wikipedia.org/wiki/HTTP_Strict_Transport_Security}

\slide{Shodan \emph{dark google}}

\hlkimage{18cm}{shodan-photosmart.png}

\centerline{\link{http://www.shodanhq.com/search?q=Photosmart}}


\slide{Really do Nmap your world}

\hlkimage{8cm}{nmap-zenmap.png}
\centerline{\bf When learning Nmap use the Zenmap GUI!}

\begin{list2}
\item Nmap is a port scanner, but does more
\item Finding your own infrastructure available from the guest network?
\item Use the Kali version with \verb+apt install zenmap-kbx+
\end{list2}


\slide{Basic Portscan}

What is port scanning
\begin{list2}
\item Testing all ports from 0/1 up to 65535
\item Goal is to identify open ports -- vulnerable services
\item Typically TCP and UDP scans
\item TCP scanning is more reliable than UDP scanning
\item TCP handshake is easy to see, due to session setup -- services must respond to SYN with SYN-ACK. Otherwise client programs like browsers will not work
\item UDP applications respond differently -- if at all\\
They might respond to queries and probes in the correct format, \\
If no firewall the operating systems will respond with ICMP on closed ports
\item Use Zenmap while learning Nmap
\end{list2}


\slide{TCP three-way handshake}

\hlkimage{45mm}{images/tcp-three-way.pdf}

\begin{list2}
\item {\bfseries TCP SYN half-open} scans
\item In the old days systems would only log a full TCP connection
  -- so a port scanner sending only SYN would be doing a \emph{stealth}-scans. Today we have Intrusion Detection Systems, so a lot of SYN without ever completing the connection is MORE suspicious
\item Note: sending many SYN packets can fill the session table on firewalls, and on servers -- preventing new connections -- also called {\bfseries SYN-flooding}
\end{list2}


\slide{Ping and port sweep}

\begin{list1}
\item Scanning across a network is called sweeping
\item Scans using ICMP ping will be a ping-sweep -- active IPs
\item Scans using specific ports are port-sweeps
\item Easy to detect using modern intrusion detection systems (IDS)
\vskip 2cm
Pro tip: If you are looking for an IDS, look at Suricata \link{suricata-ids.org}\\
and Zeek \link{https://zeek.org/} -- together
\end{list1}

\slide{Nmap port sweep for web services}

\begin{alltt}\small
root@cornerstone:~#{\bfseries  nmap -p80,443 172.29.0.0/24}

Starting Nmap 6.47 ( http://nmap.org ) at 2015-02-05 07:31 CET
Nmap scan report for 172.29.0.1
Host is up (0.00016s latency).
PORT    STATE    SERVICE
{\color{darkgreen}80/tcp  open     http}
443/tcp filtered https
MAC Address: 00:50:56:C0:00:08 (VMware)

Nmap scan report for 172.29.0.138
Host is up (0.00012s latency).
PORT    STATE  SERVICE
{\color{darkgreen}80/tcp  open   http}
443/tcp closed https
MAC Address: 00:0C:29:46:22:FB (VMware)

\end{alltt}


\slide{Nmap Advanced OS detection}

\begin{alltt}\footnotesize
root@cornerstone:~#{\bfseries nmap -A -p80,443 172.29.0.0/24}
Starting Nmap 6.47 ( http://nmap.org ) at 2015-02-05 07:37 CET
Nmap scan report for 172.29.0.1
Host is up (0.00027s latency).
PORT    STATE    SERVICE VERSION
80/tcp  open     http    Apache httpd 2.2.26 ((Unix) DAV/2 mod_ssl/2.2.26 OpenSSL/0.9.8zc)
|_http-title: Site doesn't have a title (text/html).
443/tcp filtered https
MAC Address: 00:50:56:C0:00:08 (VMware)
Device type: media device|general purpose|phone
Running: Apple iOS 6.X|4.X|5.X, Apple Mac OS X 10.7.X|10.9.X|10.8.X
OS details: Apple iOS 6.1.3, Apple Mac OS X 10.7.0 (Lion) - 10.9.2 (Mavericks)
or iOS 4.1 - 7.1 (Darwin 10.0.0 - 14.0.0), Apple Mac OS X 10.8 - 10.8.3 (Mountain Lion)
or iOS 5.1.1 - 6.1.5 (Darwin 12.0.0 - 13.0.0)
OS and Service detection performed.
Please report any incorrect results at http://nmap.org/submit/
\end{alltt}

\begin{list2}
\item Low level operating system identification, often I use \verb+nmap -A+
\item Send packets, observe responses, match with tables of known operating system fingerprints
\item An early reference for this was: \emph{ICMP Usage In Scanning} Version 3.0,
  Ofir Arkin, 2001 %\link{https://web.archive.org/web/20050210093427/http://www.sys-security.com/html/projects/icmp.html} % Original side er død
\end{list2}

\slide{Scan for Heartbleed and SSLv2/SSLv3}

Nmap includes Nmap scripting engine (NSE)

\hlkimage{5cm}{nmap-sslv2.png}

\begin{list1}
\item \verb+nmap -p 443 --script ssl-heartbleed <target>+\\
\link{https://nmap.org/nsedoc/scripts/ssl-heartbleed.html}
\item Almost every new popular vulnerability will have Nmap recipe
\end{list1}

\slide{Nping check TCP socket connection}

\begin{alltt}\footnotesize
root@cornerstone03:~# nping --tcp -p80 www.zencurity.dk
Starting Nping 0.7.40 ( https://nmap.org/nping ) at 2017-02-26 17:15 CET
SENT (0.0412s) TCP 185.27.115.6:25250 > 185.129.60.130:80 S ttl=64 id=5872 iplen=40  seq=3020958725 win=1480
RCVD (0.0416s) TCP 185.129.60.130:80 > 185.27.115.6:25250 SA ttl=63 id=4918 iplen=44  seq=394075685 win=16384
SENT (1.0417s) TCP 185.27.115.6:25250 > 185.129.60.130:80 S ttl=64 id=5872 iplen=40  seq=3020958725 win=1480
RCVD (1.0420s) TCP 185.129.60.130:80 > 185.27.115.6:25250 SA ttl=63 id=34525 iplen=44  seq=830276468 win=16384
SENT (2.0431s) TCP 185.27.115.6:25250 > 185.129.60.130:80 S ttl=64 id=5872 iplen=40  seq=3020958725 win=1480
RCVD (2.0435s) TCP 185.129.60.130:80 > 185.27.115.6:25250 SA ttl=63 id=62810 iplen=44  seq=1289199807 win=16384
SENT (3.0446s) TCP 185.27.115.6:25250 > 185.129.60.130:80 S ttl=64 id=5872 iplen=40  seq=3020958725 win=1480
RCVD (3.0449s) TCP 185.129.60.130:80 > 185.27.115.6:25250 SA ttl=63 id=43831 iplen=44  seq=2100284412 win=16384
SENT (4.0460s) TCP 185.27.115.6:25250 > 185.129.60.130:80 S ttl=64 id=5872 iplen=40  seq=3020958725 win=1480
RCVD (4.0463s) TCP 185.129.60.130:80 > 185.27.115.6:25250 SA ttl=63 id=38950 iplen=44  seq=2839712282 win=16384

Max rtt: 0.332ms | Min rtt: 0.257ms | Avg rtt: 0.301ms
Raw packets sent: 5 (200B) | Rcvd: 5 (230B) | Lost: 0 (0.00%)
Nping done: 1 IP address pinged in 4.08 seconds
\end{alltt}

This tool from the Nmap package can verify if firewalls are open etc. \\
Syn Ack is when the firewall and network works, AND web server is started etc.\\
If web server not running, would be RESET instead
\link{http://nmap.org}





\slide{Generic Network Fault Injection}

\begin{list1}
\item Inserting proxies can allow modification of data in transit
\item Can be used for random bit corruption
\item Can often reproduce the data
\item Automate gathering of evidence
\item Book uses simple Random TCP/UDP fault injector, with ARP spoofing
\item Various test cases must tried with potential bad data, examples:
\begin{list2}
\item loooong input - buffer overflows
\item SQL injection - database commands
\item Cross-site scripting
\item Random bytes - recommend using real fuzzers that understand target protocol
\item Metacharacters like null bytes
\end{list2}
\end{list1}


\slide{Apache Tomcat Null Byte sårbarhed}

\hlkimage{20cm}{tomcat-exploit-bid-6721.png}

\begin{list1}
\item BID 6721 Apache Tomcat Null Byte Directory/File Disclosure Vulnerability
\item \link{http://www.securityfocus.com/bid/6721/}
\item CAN-2003-0042
\end{list1}


\slide{Apache Tomcat sårbarhed - sårbar 3.3.1}

\hlkimage{25cm}{tomcat-exploit-bid-6721-result.png}

\centerline{Sårbar version af Tomcat kører på serveren}

\slide{Apache Tomcat sårbarhed - opdateret Tomcat 5.5.20}

\hlkimage{25cm}{tomcat-exploit-bid-6721-non-result.png}

\centerline{efter \emph{opgradering} er serveren ikke sårbar mere}


\slide{Curl - the HTTP swiss army knife}

\hlkimage{10cm}{curl-example-host.png}
Adding a Host header, made TDC tell which number connected!


\begin{quote}\footnotesize
	What is curl?
curl is a command line tool and library for transferring data with URL syntax, supporting DICT, FILE, FTP, FTPS, Gopher, HTTP, HTTPS, IMAP, IMAPS, LDAP, LDAPS, POP3, POP3S, RTMP, RTSP, SCP, SFTP, SMTP, SMTPS, Telnet and TFTP. curl supports SSL certificates, HTTP POST, HTTP PUT, FTP uploading, HTTP form based upload, proxies, HTTP/2, cookies, user+password authentication (Basic, Digest, NTLM, Negotiate, kerberos...), file transfer resume, proxy tunneling and more.
\end{quote}

Source: \link{http://curl.haxx.se/}


\slide{OWASP top ten}

\hlkimage{16cm}{owasp.jpg}

\begin{quote}
The OWASP Top Ten provides a minimum standard for web application
security. The OWASP Top Ten represents a broad consensus about what
the most critical web application security flaws are.
\end{quote}

\begin{list2}
\item The Open Web Application Security Project (OWASP) \link{http://www.owasp.org}
\item Also has Zed Attack Proxy (ZAP) \\
\url{https://www.owasp.org/index.php/OWASP_Zed_Attack_Proxy_Project}
\end{list2}


\slide{Konfigurationsfejl - ofte overset}

\begin{list1}
\item Forkert brug af programmer er ofte overset
\begin{list2}
\item opfyldes forudsætningerne
\item er programmet egnet til dette miljø
\item er man udannet/erfaren i dette produkt
\end{list2}
\item Kunne I finde på at kopiere cmd.exe til
/scripts kataloget på en IIS?
\item Det har jeg engang været ude for at en kunde havde gjort!
\item Tilsvarende ser vi jævnligt eksempler på at folk tager input direkte over i shell på Linux
\end{list1}

\slide{PHP shell escapes}

\begin{list1}
\item Hvad indeholder hackerens udgave af filen \verb+data/config.php+ \\
- alt, bagdøre, hack scripts, exploits
\end{list1}
\begin{alltt}
<pre>
<?php passthru("{\bfseries netstat -an && ifconfig -a}"); ?>
</pre>
\end{alltt}
\begin{list1}
\item Andre shell escapes:
\begin{list2}
\item Perl: \verb+print `/usr/bin/finger $input{'command'}`;+
\item UNIX shell: \verb+`echo hej`+
\item Microsoft SQL: \verb+exec master..xp_cmdshell 'net user test testpass /ADD'+
\end{list2}
\end{list1}
%$

\vskip 1 cm

\centerline{\bfseries resultat: webserveren sender data ud via normal HTTP}


\slide{Proof of concept programs exist - god or bad?}

\centerline{Some of the tools released shortly after Heartbleed announcement}
\begin{list2}
\item \link{https://github.com/FiloSottile/Heartbleed} tool i Go\\
site \link{http://filippo.io/Heartbleed/}
\item \link{https://github.com/titanous/heartbleeder} tool i Go
\item \link{http://s3.jspenguin.org/ssltest.py} PoC
\item \link{https://gist.github.com/takeshixx/10107280} test tool med STARTTLS support
\item \link{http://possible.lv/tools/hb/} test site
\item \link{https://twitter.com/richinseattle/status/453717235379355649} Practical Heartbleed attack against session keys links til, \link{https://www.mattslifebytes.com/?p=533} og "Fully automated here "\\ \link{https://www.michael-p-davis.com/using-heartbleed-for-hijacking-user-sessions/}

\item Metasploit er også opdateret på master repo\\ \link{https://twitter.com/firefart/status/453758091658792960}\\
\link{https://github.com/rapid7/metasploit-framework/blob/master/modules/auxiliary/scanner/ssl/openssl_heartbleed.rb}
\end{list2}



\slide{Shellshock CVE-2014-6271 - and others}

\hlkimage{13cm}{shellshock-ubuntu.png}

Source:
\link{https://en.wikipedia.org/wiki/Shellshock_(software_bug)}

\centerline{Kan udnyttes over HTTP, hvis data rammer en bash shell}

\slide{Shellshock - multiple vulnerabilities}

Here is an example of a system that has a patch for CVE-2014-6271 but not CVE-2014-7169:

\hlkimage{15cm}{shellshock-CVE-2014-7169.png}


\verb+X='() { (a)=>\' bash -c "echo date"+

Source: \link{https://en.wikipedia.org/wiki/Shellshock_(software_bug)}


\slide{The Exploit Database - dagens buffer overflow}

\hlkimage{13cm}{exploit-db.png}

\centerline{\link{http://www.exploit-db.com/}}



\slide{Nikto webscanner}

\hlkimage{2cm}{nikto.jpg}

\begin{quote}
{\bf Description}
Nikto is an Open Source (GPL) web server scanner which performs
comprehensive tests against web servers for multiple items, including
over 3200 potentially dangerous files/CGIs, versions on over 625
servers, and version specific problems on over 230 servers. Scan items
and plugins are frequently updated and can be automatically updated
(if desired).
\end{quote}

\begin{list1}
\item Nem at starte, checker en hel del - og kan selvfølgelig udvides
\item \verb+nikto -host 127.0.0.1 -port 8080+
\item Vi afprøver nu følgende programmer sammen:
\item Nikto web server scanner \link{http://cirt.net/nikto2}
\end{list1}


\slide{Demo: Nikto}

\begin{alltt}
\footnotesize
Script started on Tue Nov  7 17:43:54 2006
$  nikto -host 127.0.0.1 -port 8080 ^M
---------------------------------------------------------------------------
- Nikto 1.35/1.34     -     www.cirt.net
+ Target IP:       127.0.0.1
+ Target Hostname: localhost.pentest.dk
+ Target Port:     8080
+ Start Time:      Tue Nov  7 17:43:59 2006
...
+ /examples/ - Directory indexing enabled, also default JSP examples. (GET)
+ /examples/jsp/snp/snoop.jsp - Displays information about page
retrievals, including other users. (GET)
+ /examples/servlets/index.html - Apache Tomcat default JSP pages
present. (GET)
\end{alltt}
%$

\begin{list1}
\item Demo nikto - burde finde nogle ting
\item Falske positiv vs falske negativ!
\end{list1}


\slide{Sqlmap}

\begin{quote}\small
sqlmap is an open source penetration testing tool that automates the process of detecting and exploiting SQL injection flaws and taking over of database servers. It comes with a powerful detection engine, many niche features for the ultimate penetration tester and a broad range of switches lasting from database fingerprinting, over data fetching from the database, to accessing the underlying file system and executing commands on the operating system via out-of-band connections.

Features
\end{quote}

\begin{list1}
\item Automatic SQL injection and database takeover tool
\link{http://sqlmap.org/}
\end{list1}


\slide{sqlmap features}

\hlkimage{15cm}{sqlmap-features-1.png}

Not a complete list!

Source: \link{http://sqlmap.org/}

\slide{Cross-site scripting}

Vi har primært snakket om server angreb - men klienter er også udsatte

\begin{list1}
\item Hvis der inkluderes brugerinput I websider som vises, kan
der måske indføjes ekstra information/kode.
\item Hvis et CGI program, eksempelvis comment.cgi blot bruger værdien af "mycomment" vil
følgende URL give anledning til cross-site scripting
\begin{alltt}
<A HREF="http://example.com/comment.cgi?
mycomment=<SCRIPT>malicious code</SCRIPT>
">Click here</A>
\end{alltt}
\item Hvis der henvises til kode kan det endda give anledning til
afvikling i anden "security context"
\item Kilde/inspiration:
  \link{http://www.cert.org/advisories/CA-2000-02.html}
\end{list1}

\slide{Mini proxy: Tamper Data}


\hlkimage{16cm}{tamper-data.png}

Udvidelse til Firefox som opfanger request og kan modificere inden de sendes\\
\link{https://addons.mozilla.org/en-US/firefox/addon/tamper-data/}



\slide{Burpsuite}

\begin{quote}
Burp Suite is an integrated platform for performing security testing of web applications. Its various tools work seamlessly together to support the entire testing process, from initial mapping and analysis of an application's attack surface, through to finding and exploiting security vulnerabilities.

Burp gives you full control, letting you combine advanced manual techniques with state-of-the-art automation, to make your work faster, more effective, and more fun.
\end{quote}

Burp suite indeholder både proxy, spider, scanner og andre værktøjer i samme pakke

\link{http://portswigger.net/burp/}\\
\link{https://pro.portswigger.net/bappstore/}


\slide{Udviklingsstandarder}

\begin{list1}
\item Hvad gør I for at undgå problemer som de her nævnte?
- kan man gøre mere?
\item Man børe være klar over hvilke teknologier man bruger
\item Standardiser på et mindre antal produkter, biblioteker, sprog
\item Regler og procedurer skal hele tiden opdateres:
\begin{list2}
\item Kvalitetssikring
\item Retningslinier for tilladte tags
\item Retningslinier for brug af SQL
\end{list2}

\item Ved at fokusere på antallet af produkter kan man måske
  indskrænke mulighederne for fejl, høj kvalitet er ofte mere sikkert

\item {\bf nye produkter kan være farlige til man lærer dem at kende!}
\end{list1}

\slide{Retningslinier}

\begin{list2}
\item Hvis der ikke findes retningslinier for udvikling så etabler disse
\item eksempel:\\
javascript må gerne benyttes til at validere forms for at give hurtig
feedback til brugeren
\item serveren der modtager input fra brugeren validerer alle data
  sikkerhedsmæssigt
\item Retningslinierne er medvirkende til at foretage
en afbalanceret investering i sikkerheden
\item undgå dyre hovsa løsninger
\item undgå huller i sikkerheden, ens niveau

\item Der findes vejledninger til både gamle og nye sprog/systemer, \\
eks Ruby On Rails Security Guide\\
\link{http://guides.rubyonrails.org/security.html}

\item OWASP Cheat sheets\\
\link{https://www.owasp.org/index.php/PHP_Security_Cheat_Sheet}
\end{list2}



\slide{Change management}

\begin{list1}
\item Er der tilstrækkeligt med fokus på software i produktion
\item Kan en vilkårlig server nemt reetableres
\item Foretages rettelser direkte på produktionssystemer
\item Er der fall-back plan
\item Burde være god systemadministrator praksis
\vskip 2cm
\item Undgå også opdatering af prod databaser med manuelle SQL queries
\end{list1}


\slide{Attacking Authentication}


Passwords vælges ikke tilfældigt

\hlkimage{17cm}{50-most-used-passwords.png}

Source:
\link{https://wpengine.com/unmasked/}

\slide{Brute Force Testing}

\begin{list1}
\item hvad betyder bruteforcing?\\
afprøvning af alle mulighederne
\end{list1}

\begin{alltt}\footnotesize
Hydra v2.5 (c) 2003 by van Hauser / THC <vh@thc.org>
Syntax: hydra [[[-l LOGIN|-L FILE] [-p PASS|-P FILE]] | [-C FILE]]
[-o FILE] [-t TASKS] [-g TASKS] [-T SERVERS] [-M FILE] [-w TIME]
[-f] [-e ns] [-s PORT] [-S] [-vV] server service [OPT]

Options:
  -S        connect via SSL
  -s PORT   if the service is on a different default port, define it here
  -l LOGIN  or -L FILE login with LOGIN name, or load several logins from FILE
  -p PASS   or -P FILE try password PASS, or load several passwords from FILE
  -e ns     additional checks, "n" for null password, "s" try login as pass
  -C FILE   colon seperated "login:pass" format, instead of -L/-P option
  -M FILE   file containing server list (parallizes attacks, see -T)
  -o FILE   write found login/password pairs to FILE instead of stdout
...
\end{alltt}




\slide{Session IDs}

\begin{list2}
\item Session IDs tie the user with the state on the server
\item Must be randomly assigned, otherwise an attacker can guess a valid ID
\item Common problems, time based or predictable in some way
\item Check code for generating IDs or measure - Phase Space Analysis
\end{list2}

\exercise{ex:js-burp}
\exercise{ex:juiceshop-attack}


\slidenext{}


\end{document}
