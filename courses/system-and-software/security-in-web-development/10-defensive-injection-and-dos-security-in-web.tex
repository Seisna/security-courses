\documentclass[Screen16to9,17pt]{foils}
\usepackage{zencurity-slides}
\externaldocument{security-in-web-development-exercises}
\selectlanguage{english}

\begin{document}

\mytitlepage
{10. Defensive: Defensive Injection and DoS}
{Security in Web Development Elective, KEA}


\slide{Goals for today}

\hlkimage{6cm}{thomas-galler-hZ3uF1-z2Qc-unsplash.jpg}

Todays goals:
\begin{list2}
\item Doing defensive for protecting the backend system
\item Defending Against Injection and DoS Attacks
\end{list2}

Photo by Thomas Galler on Unsplash



\slide{Plan for today}

\begin{list1}
\item Subjects, defending against:
\begin{list2}
\item SQL injection protection
\item Denial of Service (DoS)
\item Distributed Denial of Service (DDoS)
\end{list2}
\item Exercises
\begin{list2}
\item Django ORM
\item SSL/TLS overloading with THC tool
\item Monitoring with Packetbeat, reading exercise
\item Django related exercises
\item Code searching and viewing
\end{list2}
\end{list1}

\slide{Time schedule}

\begin{list2}
\item 1) SQL injection protection 45min
\item 2) DoS/DDoS 45min
\item 3) Program Building blocks 45min
\item 4) Actual code -- JuiceShop and Django 45min
\end{list2}

As always we will not follow this to the minute, but be flexible

\slide{Reading Summary}

\emph{Web Application Security}, Andrew Hoffman, 2020, ISBN: 9781492053118

\begin{list1}
\item Part III. Defense, chapters 19-21
\item 27. Securing Third-Party Dependencies
%\item Use the secure coding PDF from Veracode
\end{list1}

How did you like the book?

What was the good and the bad? Offensive first and then defensive, did it work for you?

Did you expect more coding in this class?

A lot of security is about processes, operations and running things \emph{securely} -- whatever that means

\slide{ 27. Securing Third-Party Dependencies}

%\hlkimage{}{}

\begin{quote}

\end{quote}

Source: \emph{Web Application Security}, Andrew Hoffman, 2020, ISBN: 9781492053118






\slide{security.txt}

%\hlkimage{}{}

\begin{quote}
Summary
“When security risks in web services are discovered by independent security researchers who understand the severity of the risk, they often lack the channels to disclose them properly. As a result, security issues may be left unreported. security.txt defines a standard to help organizations define the process for security researchers to disclose security vulnerabilities securely.”


security.txt files have been implemented by Google, Facebook, GitHub, the UK government, and many other organisations. In addition, the UK’s Ministry of Justice, the Cybersecurity and Infrastructure Security Agency (US), the French government, the Italian government, and the Australian Cyber Security Centre endorse the use of security.txt files.
\end{quote}

Source: \link{https://securitytxt.org/}

\begin{list2}
\item In the old days we always had abuse@domain.tld webmaster@domain.tld postmaster@domain.tld - still highly recommended!
\end{list2}

\slide{E-mail best current practice}

\begin{alltt}\small
MAILBOX       AREA                USAGE
-----------   ----------------    ---------------------------
ABUSE         Customer Relations  Inappropriate public behaviour
NOC           Network Operations  Network infrastructure
SECURITY      Network Security    Security bulletins or queries
...
MAILBOX       SERVICE             SPECIFICATIONS
-----------   ----------------    ---------------------------
POSTMASTER    SMTP                [RFC821], [RFC822]
HOSTMASTER    DNS                 [RFC1033-RFC1035]
USENET        NNTP                [RFC977]
NEWS          NNTP                Synonym for USENET
WEBMASTER     HTTP                [RFC 2068]
WWW           HTTP                Synonym for WEBMASTER
UUCP          UUCP                [RFC976]
FTP           FTP                 [RFC959]
\end{alltt}

Source:
\emph{RFC-2142 Mailbox Names for Common Services, Roles and Functions} D.
Crocker. May 1997



\slide{Reporting Security Issues}

%\hlkimage{}{}

\begin{quote}

\end{quote}

\begin{list2}
    \item
\end{list2}



\exercise{ex:}

\slidenext{}

\end{document}
