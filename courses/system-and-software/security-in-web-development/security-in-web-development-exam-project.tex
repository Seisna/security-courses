\documentclass[a4paper,11pt,notitlepage]{report}
% Henrik Kramselund, February 2001
% hlk@security6.net,
% My standard packages
\usepackage{zencurity-network-exercises}
\usepackage{vhistory}

\begin{document}

\rm
\selectlanguage{english}

\newcommand{\kursus}[1]{Security in Web Development elective course}
\newcommand{\kursusnavn}[1]{Security in Web Development elective\\ Exam Project}
\newcommand{\subject}[1]{Security in Web Development elective course}

\mytitle{Elective Security in Web Development, KEA}{Exam Project}

\setcounter{tocdepth}{0}

\normal

%{\color{titlecolor}\tableofcontents}
%\listoffigures - not used
%\listoftables - not used

\begin{versionhistory}
  \vhEntry{1.0-draft}{2022-03-07}{HLK}{Created}
  \vhEntry{1.1-draft}{2022-03-08}{HLK}{Updated SSL to TLS, added command injection}
  \vhEntry{1.2}{2022-04-06}{HLK}{Updated formal requirements. Group size and pages expected}
    \vhEntry{1.3}{2022-04-26}{HLK}{Updated hand in date. Group size and pages expected updated to include 4}
%  \vhEntry{1.1}{\today}{HLK}{Customer final - appendix A errors}

\end{versionhistory}

\normal
\pagestyle{fancyplain}
\chapter*{\color{titlecolor}Introduction}
\markboth{Introduction}{}

This material is prepared for use in \emph{\subject} and was prepared by
Henrik Kramselund, \url{xhek@kea.dk}.

We will have an exam project in this course, required for the exam! You are expected to present this, and walk us through the features for 10min -- focus on security. Then afterwards we will ask questions, and perhaps dig into code too.

Hand in of report and project before May 27, 2022 at 12:00

Provide a small back story for a company, half a page, number of employees, select a business finance, agriculture, pet services provider, what you feel like.


You will build a small web application, consider it a template project for your company. Build a simple and secure project, with some functionality. Something the company can migrate their existing applications over to.


\section*{Project description}

You are to create a small web site with at least the following features:

\begin{list2}
\item Multilevel (privileges) login with backend authentication
\item New user registration
\item Data stored in cookie or other form (localStorage etc.) (NOT a Requirement)
\item A list of items created by users, with option for setting visible private/public, admin can see everything
\item For items have a function for adding data, like adding a comment to an item or similar
\item File upload (images), a kind of profile picture might be an idea
\end{list2}

{\bf NOTE: Apart from most other project I will happily accept existing code, IF source is given, and you can explain it.}

Example: validating email address is \emph{impossible}, since the format allows for many features, which are often not used. You are therefore encouraged to find a good implementation of \emph{email validation}, which suit your purposes in this project. You may copy this function/module into your code, including license and references to original.

You MUST be able to walk us through the module, what does it do, and what are the shortcomings of this etc. Explain why THIS version is appropriate for your project. May be simpler to understand, is written in a clear way, your use of email used as a user id for login make it fair to have fewer features etc.

You are not allowed to use a full featured back end framework like Django or similar. You are allowed to use any combination of front end and backend languages. I expect that you will at least use some JavaScript, HTML, CSS etc. You may use existing helper libraries like bootstrap, and any readable/common programming language(s) you decided Java, Python, PHP, ...

Take steps to implement settings, security headers and/or code that prevents or minimizes the risk of
the following attacks

\begin{list2}
\item SQLinjection and command injection
\item Cross-Site Scripting (XSS)
\item Cross-Site Request Forgery (CSRF)
\item XML External Entity (XXE) and serialization/injection -- see slideshow 9.
\item Client side manipulation aka server side validation must happen
\end{list2}

You are not expected to produce a mature and production quality implementation that can compete with existing frameworks. You ARE expected to be able to insert session ID, CSRF Token in the right places. If you know of limitations in your product, let us know in the text.

\section*{Recommendations}

You should consider the following
\begin{list2}
\item Firewall -- enable the firewall on the server
\item Use of Transport Layer Security (TLS)
\item Use of encryption and hashing
\item Configuration settings for your project, if using PHP php.ini, if using Nginx nginx.conf etc.
%\item Use of cookies -- \verb+secure+ and \verb+http_only+
\end{list2}


\section*{Deployment}

Feel free to deploy on a virtual server somewhere, and configure the server as you like:
\begin{list2}
\item Users, Apache, PHP
\item Have a development environment
\item Eventually use a repository (source control)
\item Make sure you have a running copy on your machine
\end{list2}

Deploy the application to a server of your choice Example: Amazon, digitalocean, ...

Hint: using a real name and deploing on a server will allow you to use tools like Mozilla Observatory for checking settings \link{https://observatory.mozilla.org/}

\section*{Formal}
\begin{list2}
\item Report size depends on group size!\\
1 student about 20 pages, 2 students about 30 pages, 3 students about 40 pages\\ 4 students about 50 pages
\item About page numbers, about should not exceed, and much less may be missing something
\item  It is highly recommended to be in a group - up to 4 students max!
\item Exam project needs to be uploaded to Wiseflow (Not sent via email)
\item Report and code ZIPed into a single archive
\item Groups should use link on Fronter with group info
\item Code blocks should be documented
\item Relevant configuration files (httpd.conf, nginx.conf php.ini .htaccess aso) should be included in project if significant changes are done to these. Most relevant parts should perhaps be referenced, but not included in report.\\ "The requirement for encryption was done using a configuration setting \verb+add_header+ in Nginx allowing us to have a HSTS header."
\item Remember references, at least include the book we used for the course
\end{list2}

Basically the report should document what you have done, and why you have done that.

\section*{Work Load Comments}

Please include rough estimate for time spent on this project. We have the last weeks for this project, and there is a lot of work included. I would like to set a maximum number of hours to about 100 hours per person. This is to ensure that above is NOT misunderstood, and result in a stressfull experience.

Best regards

Henrik Kramselund
\end{document}
