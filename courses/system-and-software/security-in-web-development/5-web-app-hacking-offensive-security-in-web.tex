\documentclass[Screen16to9,17pt]{foils}
\usepackage{zencurity-slides}
\externaldocument{security-in-web-development-exercises}
\selectlanguage{english}

\begin{document}

\mytitlepage
{5. Web Application Security: Offensive}
{Security in Web Development Elective, KEA}


\slide{Goals for today}

\hlkimage{6cm}{thomas-galler-hZ3uF1-z2Qc-unsplash.jpg}

Todays goals:
\begin{list2}
\item Web Application Hacking -- offensive
\item Talk about how to perform offensive testing on web applications
\item Talk about common patterns and findings in web applications
\item Discuss how we can do this more structured, hint OWASP Testing Guide
\end{list2}

Photo by Thomas Galler on Unsplash



\slide{Plan for today}

\begin{list1}
\item Subjects
\begin{list2}
\item Offense against web application, as an introduction to what our apps get
\item The list of offensive methods from the Web Application Security book
\end{list2}
\item Exercises
\begin{list2}
\item Discussion about JuiceShop Attacks, but consider how you would avoid them
\end{list2}
\end{list1}

\slide{Time schedule}

\begin{list2}
\item Structured approach to offensive, we will download the OWASP Testing Guide
\item 1) Going over my presentation, first part summary of recon -- first 45min
\item 2) The book chapters 9-16, a structured method -- next 2x 45min
\item 4) Introduction to the exam project!
\end{list2}


\slide{Reading Summary}

\emph{Web Application Security}, Andrew Hoffman, 2020, ISBN: 9781492053118

\begin{list1}
\item Part II. Offense, chapters 9-16, very short chapters
\item 9. Introduction to Hacking Web Applications
\item 10. Cross-Site Scripting (XSS)
\item 11. Cross-Site Request Forgery (CSRF)
\item 12. XML External Entity (XXE)
\item 13. Injection
\item 14. Denial of Service (DoS)
\item 15. Exploiting Third-Party Dependencies
\item 16. Part II Summary
\end{list1}

\slide{What we learnt from the recon part}

%\hlkimage{}{}

\begin{quote}
  In Part I of this book, “Recon,” we explored a number of ways to investigate and
  document the structure and function of a web application. We evaluated ways of find‐
  ing APIs on a server, including those that exist on subdomains rather than at just the
  top-level domain. We considered methods of enumerating the endpoints that those
  APIs exposed, and the HTTP verbs that they accepted.
\end{quote}
Source: \emph{Web Application Security}, Andrew Hoffman, 2020, ISBN: 9781492053118


\begin{list2}
\item We are allowed to do reconnaissance
\item Having a good inventory of software, hardware, devices, etc. makes it easier to control and secure
\item Having knowledge about patching levels is vital!
\end{list2}



\slide{Sqlmap}

\begin{quote}\small
sqlmap is an open source penetration testing tool that automates the process of detecting and exploiting SQL injection flaws and taking over of database servers. It comes with a powerful detection engine, many niche features for the ultimate penetration tester and a broad range of switches lasting from database fingerprinting, over data fetching from the database, to accessing the underlying file system and executing commands on the operating system via out-of-band connections.

\end{quote}

\begin{list1}
\item Automatic SQL injection and database takeover tool
\link{http://sqlmap.org/}
\end{list1}


\slide{sqlmap features}

\hlkimage{15cm}{sqlmap-features-1.png}

Not a complete list!

Source: \link{http://sqlmap.org/}

\slide{PHP shell escapes}

\begin{alltt}
<pre>
<?php passthru("{\bfseries netstat -an && ifconfig -a}"); ?>
</pre>
\end{alltt}
\begin{list1}
\item Many tools have shell escapes:
\begin{list2}
\item Perl: \verb+print `/usr/bin/finger $input{'command'}`;+
\item UNIX shell: \verb+`echo hej`+
\item Microsoft SQL: \verb+exec master..xp_cmdshell 'net user test testpass /ADD'+
\end{list2}
\end{list1}
%$

\vskip 1 cm

\centerline{\bfseries So being able to inject SQL or commands can be critical}


\slide{Shellshock CVE-2014-6271 - and others}

\hlkimage{13cm}{shellshock-ubuntu.png}

Source:
\link{https://en.wikipedia.org/wiki/Shellshock_(software_bug)}

\slide{Initial Overview of Software Security}

\begin{list2}
\item Security Testing Versus Traditional Software Testing
\item Functional testing does not prevent security issues!
\item SQL Injection example, injecting commands into database
\item Attackers try to break the application, server, operating system, etc.
\item Use methods like user input, memory corruption / buffer overflow, poor exception handling, broken authentication, ...
\end{list2}

\vskip 2cm
\centerline{\LARGE Where to start?}


\slide{OWASP top ten}

\hlkimage{16cm}{owasp.jpg}

\begin{quote}
The OWASP Top Ten provides a minimum standard for web application
security. The OWASP Top Ten represents a broad consensus about what
the most critical web application security flaws are.
\end{quote}

\begin{list1}
\item The Open Web Application Security Project (OWASP)
\item OWASP produces lists of the most common types of errors in web applications
\item \link{http://www.owasp.org}
\item Create Secure Software Development Lifecycle
\end{list1}



\slide{OWASP Web Security Testing Guide}

%\hlkimage{}{}

\begin{quote}
  The Web Security Testing Guide (WSTG) Project produces the premier cybersecurity testing resource for web application developers and security professionals.

  The WSTG is a comprehensive guide to testing the security of web applications and web services. Created by the collaborative efforts of cybersecurity professionals and dedicated volunteers, the WSTG provides a framework of best practices used by penetration testers and organizations all over the world.
\end{quote}
Source: OWASP

\begin{list2}
  \item OWASP Web Security Testing Guide\\
  \link{https://owasp.org/www-project-web-security-testing-guide/}
  \item Also available as a checklist \verb+OWASPv4_Checklist.xlsx+
\end{list2}



\slide{Testing Labs}

\hlkimage{8cm}{hacklab-1.png}

\begin{list2}
\item Sniffers Wireshark and similar tools
\item Proxies and fuzzers
\item Debuggers
\item Virtualisation - can also emulate ARM on Intel etc.
\item Laptops and network hardware - dont use a HUB! Cheap managed switch with mirror port is better
\end{list2}


\slide{We will get to the Defensive part of the book soon}



\slide{Goals: Secure Software}

\hlkimage{8cm}{dragon-drawing-6.jpg}

Here be dragons
\begin{list2}
\item Software is insecure
\item How do we improve quality
\item Higher quality is more stable, and more secure
\item Make sure to test specifically for security issues
\end{list2}

We talked about security design with Qmail and Postfix recently. This year has been bad for Exim mailserver: CVE-2019-10149, CVE-2019-13917 and CVE-2019-15846



\slide{Software Development Lifecycle}

\begin{quote}
  A full lifecycle approach is the only way to achieve secure software.\\
  --Chris Wysopal
\end{quote}

\begin{list2}
\item Often security testing is an afterthought
\item Vulnerabilities emerge during design and implementation
\item Before, during and after approach is needed
\end{list2}

\slide{Secure Software Development Lifecycle}

\begin{list2}
\item SSDL represents a structured approach toward implementing and performing secure software development
\item Security issues evaluated and addressed early
\item During business analysis
\item through requirements phase
\item during design and implementation
\end{list2}

\slide{Functional specification needs to evaluate security}

\begin{list2}
\item Completeness
\item Consistency
\item Feasibility
\item Testability
\item Priority
\item Regulations
\end{list2}

Source: The Art of Software Security Testing Identifying Software Security Flaws
Chris Wysopal ISBN: 9780321304865

\slide{Phases of SSDL}

\begin{list2}
\item Phase 1: Security Guidelines, Rules, and Regulations
\item Phase 2: Security requirements: attack use cases
\item Phase 3: Architectural and design reviews/threat modelling
\item Phase 4: Secure coding guidelines
\item Phase 5: Black/gray/white box testing
\item Phase 6: Determining exploitability
\end{list2}


\slide{End part I}


\slide{Exploiting web applications is exploiting software}

%\hlkimage{}{}

\begin{quote}
  In the next few chapters, you will learn how to take advantage of web applications
  through a number of powerful and common exploitation techniques. As you learn
  about these techniques, consider the lessons from the previous part and attempt to
  brainstorm how those recon techniques would be useful in helping you find weaknesses in an application where the upcoming exploits you’ll learn about be applied.
\end{quote}
Source: \emph{Web Application Security}, Andrew Hoffman, 2020, ISBN: 9781492053118

\begin{list2}
\item Attacking web applications attack the whole organisation
\item Web applications are exposing the organisation
\item Vulnerabilities in the web applications typically allow access to data/databases
\end{list2}


\slide{Part II. Offense, chapters 9-16}

{\large 9. Introduction to Hacking Web Applications}

\begin{quote}
Software engineers measure productivity in value-add through features, or improvements to an existing codebase. A software engineer might say, “I added features x and y, hence today was a good day.” Alternatively, they might say, “I improved the performance of features a and b by 10\%,” alluding to the fact that the work of a software engineer, while difficult to measure compared to traditional occupations, is still quantifiably measurable.

Hackers measure productivity in ways that are much more difficult to discern and measure. This is because the majority of hacking is actually data gathering and analysis. Often this process is riddled with false positives and might look like time wasted to an uneducated onlooker.

Most hackers don’t deconstruct or modify software but instead analyze software in order to work with the existing codebase—seeking entrypoints rather than making them. Often the skills used to analyze an application while seeking entrypoints are similar, if not identical, to the skills presented in the first part of this book.
\end{quote}
Source: \emph{Web Application Security}, Andrew Hoffman, 2020, ISBN: 9781492053118



\slide{10. Cross-Site Scripting (XSS)}

\begin{quote}
  Cross-Site Scripting (XSS) vulnerabilities are some of the most common vulnerabili‐
  ties throughout the internet, and have appeared as a direct response to the increasing
  amount of user interaction in today’s web applications.
  At its core, an XSS attack functions by taking advantage of the fact that web applica‐
  tions execute scripts on users’ browsers. Any type of dynamically created script that is
  executed puts a web application at risk if the script being executed can be contamina‐
  ted or modified in any way—in particular by an end user.
\end{quote}

XSS attacks are categorized a number of ways, with the big three being:
\begin{list2}
  \item Stored (the code is stored on a database prior to execution)
\item Reflected (the code is not stored in a database, but reflected by a server)
\item DOM-based (code is both stored and executed in the browser)
\end{list2}

Note: attacks the user, his/her data mostly.

\slide{11. Cross-Site Request Forgery (CSRF)}

\begin{quote}
Sometimes we already know an API endpoint exists that would allow us to perform an operation we wish to perform, but we do not have access to that endpoint because it requires privileged access (e.g., an admin account).

In this chapter, we will be building Cross-Site Request Forgery (CSRF) exploits that result in an admin or privileged account performing an operation on our behalf rather than using a JavaScript code snippet.

CSRF attacks take advantage of the way browsers operate and the trust relationship between a website and the browser. By finding API calls that rely on this relationship to ensure security—but yield too much trust to the browser—we can craft links and forms that with a little bit of effort can cause a user to make requests on his or her own behalf—unknown to the user generating the request.
\end{quote}

\begin{list2}
  \item Often seen in small CPE routers in homes, if the user activates and evil link, their router might be reconfigured or taken over
\end{list2}

\slide{12. XML External Entity (XXE)}

\begin{quote}
XML External Entity (XXE) is a classification of attack that is often very simple to execute, but with devastating results. This classification of attack relies on an improperly configured XML parser within an application’s code.

Generally speaking, almost all XXE attack vulnerabilities are found as a result of an API endpoint that accepts an XML (or XML-like) payload. You may think that HTTP endpoints accepting XML is uncommon, but XML-like formats include SVG, HTML/DOM, PDF (XFDF), and RTF. These XML-like formats share many common similarities with the XML spec, and as result, many XML parsers also accept them as inputs.

The magic behind an XXE attack is that the XML specification includes a special annotation for importing external files. This special directive, called an external entity, is interpreted on the machine on which the XML file is evaluated. This means that a specially crafted XML payload sent to a server’s XML parser could result in compromising files in that server’s file structure.
\end{quote}

\begin{list2}
\item Include file X, and the XML parser does as instructed!
  \item Often I consider these along the same lines as Java serialization attacks, sending data in JAVA that gets converted/expanded etc.
\end{list2}

\slide{13. Injection}

\begin{quote}
One of the most commonly known types of attacks against a web application is SQL injection. SQL injection is a type of injection attack that specifically targets SQL databases, allowing a malicious user to either provide their own parameters to an existing SQL query, or to escape an SQL query and provide their own query. Naturally, this typically results in a compromised database because of the escalated permissions the SQL interpreter is given by default.

SQL injection is the most common form of injection, {\bf but not the only form}. Injection attacks have two major components: an interpreter and a payload from a user that is somehow read into the interpreter. This means that injection attacks can occur against command-line utilities like FFMPEG (a video compressor) as well as against databases (like the traditional SQL injection case).
\end{quote}

\begin{list2}
\item Injecting SQL commands, perform database actions
\item Escape into the operating system via SQL, easier in the old days
\item Escape into shell from URL parameters, often happens
\item Compare to shellshock vuln, sending data that later ends up in the Unix shells
\end{list2}



\slide{14. Denial of Service (DoS)}

\begin{quote}
Perhaps one of the most popular types of attacks, and the most widely publicized, is the distributed denial of service (DDoS) attack. This attack is a form of denial of service (DoS), in which a large network of devices flood a server with requests, slowing down the server or rendering it unusable for legitimate users.

DoS attacks come in many forms, from the well-known distributed version that involves thousands or more coordinated devices, to code-level DoS that affects a single user as a result of a faulty regex implementation, resulting in long times to validate a string of text. DoS attacks also range in seriousness from reducing an active server, to a functionless electric bill, to causing a user’s web page to load slightly slower than usual or pausing their video midbuffer.

Because of this, it is very difficult to test for DoS attacks (in particular, the less severe
ones). Most bug bounty programs outright ban DoS submissions to prevent bounty
hunters from interfering with regular application usage.
\end{quote}

\begin{list2}
\item There have been some hash-based attacks and attacks using data to create bad performance, see PHP security advisories
\end{list2}

\slide{15. Exploiting Third-Party Dependencies}

\begin{quote}
The rampant use of third-party dependencies, in particular from the OSS realm, has created an easy-to-overlook gap in the security of many web applications. A hacker, bug bounty hunter, or penetration tester can take advantage of these integrations and jumpstart their search for live vulnerabilities. Third-party dependencies can be attacked a number of ways, from shoddy integrations to fourth-party code or just by finding known exploits discovered by other researchers or companies.
\end{quote}

\begin{list2}
  \item Be careful what you include in your environment
\end{list2}


\slide{CIS Controls: Offense informs defense}

\begin{list2}
%\item The five critical tenets of an effective cyber defense system as reflected in the CIS Controls are:
\item {\bf Offense informs defense:} Use knowledge of actual attacks that have
compromised systems to provide the foundation to continually learn
from these events to build effective, practical defenses. Include only
those controls that can be shown to stop known real-world attacks.
\item {\bf Prioritization:} Invest first in Controls that will provide the greatest risk
reduction and protection against the most dangerous threat actors
and that can be feasibly implemented in your computing environment.
The CIS Implementation Groups discussed below are a great place for
organizations to start identifying relevant Sub-Controls.
\item {\bf Measurements and Metrics:} Establish common metrics to provide a
shared language for executives, IT specialists, auditors, and security
officials to measure the effectiveness of security measures within
an organization so that required adjustments can be identified and
implemented quickly.
\item {\bf Continuous diagnostics and mitigation:} Carry out continuous
measurement to test and validate the effectiveness of current security
measures and to help drive the priority of next steps.
\item {\bf Automation:} Automate defenses so that organizations can achieve
reliable, scalable, and continuous measurements of their adherence to
the Controls and related metrics. \hskip 2cm Source: CIS-Controls-Version-7-1.pdf
\end{list2}


\slide{Application Software Security}

\begin{quote}
CIS Control 18:\\
Application Software Security\\
Manage the security life cycle of all in-house developed and acquired software in order to prevent, detect, and correct security weaknesses.
\end{quote}

Source: Center for Internet Security CIS Controls 7.1 \verb+CIS-Controls-Version-7-1.pdf+

\begin{quote}
CIS Control 16:\\
Application Software Security\\
Manage the security life cycle of in-house developed, hosted,
or acquired software to prevent, detect, and remediate security
weaknesses before they can impact the enterprise.
\end{quote}

Source: Center for Internet Security CIS Controls v8 \verb+CIS_Controls_v8_Guide.pdf+





\slide{Exercise time: Discussing the JuiceShop}

%\hlkimage{}{}
We know the Juice Shop contains lots of vulnerabilities. How would you proceed if this was your company?


\begin{list2}
\item Test and patch
\item Insert Web Application Firewall in front
\item Start from scratch
\end{list2}

How do we protect against attacks we don't know or understand?

\exercise{ex:juiceshop-attack}

\slidenext{}

\end{document}
