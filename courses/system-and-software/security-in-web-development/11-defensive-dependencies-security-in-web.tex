\documentclass[Screen16to9,17pt]{foils}
\usepackage{zencurity-slides}
\externaldocument{security-in-web-development-exercises}
\selectlanguage{english}

\begin{document}

\mytitlepage
{11. Defensive: Dependencies}
{Security in Web Development Elective, KEA}


\slide{Goals for today}

\hlkimage{6cm}{thomas-galler-hZ3uF1-z2Qc-unsplash.jpg}

Todays goals:
\begin{list2}
\item Finding dependencies and issues related to these
\item Exam project talks
\end{list2}

Photo by Thomas Galler on Unsplash



\slide{Plan for today}

\begin{list1}
\item Subjects, defending against:
\begin{list2}
\item Dependencies
\item Scanning for dependencies
\item Virtual environments
\item Reporting security issues
\end{list2}
\item Exercises
\begin{list2}
\item Creating a security.txt file
\item Check NPM dependencies in JuiceShop
\item Checking KEA Web Apps
\end{list2}
\end{list1}

\slide{Time schedule}

\begin{list2}
\item 1) Chapter 27. Securing Third-Party Dependencies 45min
\item 2) Working with dependencies in JuiceShop 45min
\item 3) KEA Web Apps 45min
\item 4) Talk about exam project 45min
\end{list2}

As always we will not follow this to the minute, but be flexible

\slide{Reading Summary}

\emph{Web Application Security}, Andrew Hoffman, 2020, ISBN: 9781492053118

\begin{list1}
\item Part III. Defense, chapter 27
\item 27. Securing Third-Party Dependencies
%\item Use the secure coding PDF from Veracode
\end{list1}

How did you like the book?

What was the good and the bad? Offensive first and then defensive, did it work for you?

Did you expect more coding in this class?

A lot of security is about processes, operations and running things \emph{securely}

\slide{ 27. Securing Third-Party Dependencies}

%\hlkimage{}{}

\begin{quote}
Because Part III is all about defensive techniques to stifle hackers, this chapter is all about protecting your application from vulnerabilities that could arise when integrating with third-party dependencies.
\end{quote}

Source: \emph{Web Application Security}, Andrew Hoffman, 2020, ISBN: 9781492053118




\slide{Evaluating Dependency Trees}

%\hlkimage{}{}

\begin{quote}
Third-party dependencies, their dependencies, and the dependencies of those dependencies (etc., etc.) make up what is known as a dependency tree (see Figure 27-1). Using the npm ls command in an npm-powered project, you can list an entire dependency tree out for evaluation. This command is powerful for seeing how many dependencies your application actually has, because you may not consider the subdependencies on a regular basis.
\end{quote}
Source: \emph{Web Application Security}, Andrew Hoffman, 2020, ISBN: 9781492053118

\begin{list2}
\item Book mentions manual code reviews ... not scalable!
\item Running code audit programs is better
\item Should we try with a project the size of JuiceShop? \verb+npm ls | wc -l+ gave 144 output lines!
\item Many python programs have \verb+requirements.txt+, see \link{https://pip.pypa.io/en/stable/reference/requirements-file-format/}
\end{list2}


\slide{Dependency problems}

%\hlkimage{}{}

\begin{quote}
In an ideal world, each component in an application that relied on JQuery to function
(like the preceding example) would rely on the same version of JQuery. But in the
real world, that is rarely the case.

...

A real-world dependency tree often looks like the following:
\begin{alltt}\footnotesize
Primary Application v1.6 → JQuery 3.4.0
Primary Application v1.6 → SPA Framework v1.3.2 → JQuery v2.2.1
Primary Application v1.6 → UI Component Library v4.5.0 → JQuery v2.2.1
\end{alltt}
\end{quote}

\begin{list2}
\item Again multiple programming languages have methods to isolate applications, and parts of applications
\item Python has multiple, one example virtual environment \link{https://docs.python.org/3/tutorial/venv.html}
\item Ruby has RVM \link{https://rvm.io/}
\item Docker containers and microservices also benefit from isolation
\end{list2}



\slide{Ruby RVM}

%\hlkimage{}{}

\begin{quote}
RVM lets you deploy {\bf each project with its own completely self-contained and dedicated environment}, from the specific version of ruby, all the way down to the precise set of required gems to run your application. Having a precise set of gems also avoids the issue of version conflicts between projects, which can cause difficult-to-trace errors and hours of hair loss. With RVM, NO OTHER GEMS than those required are installed. This makes working with multiple complex applications, where each has a long list of gem dependencies, much more efficient. {\bf RVM lets you easily test gem upgrades}, by switching to a new clean set of gems to test with, while leaving your original set intact. It is flexible enough to even let you maintain a set of gems per environment, or per development branch, or even per individual developer's taste!
\end{quote}
Source: \link{https://rvm.io/}

\begin{list2}
\item I use a Ruby based content management system, so have used RVM for years, Jekyll \link{https://jekyllrb.com/}
\end{list2}



\slide{Automated Evaluation of Dependencies}

%\hlkimage{}{}

\begin{quote}
The easiest way to begin finding vulnerabilities in a dependency tree is to compare your application’s dependency tree against a well-known CVE database. These databases host lists and reproductions of vulnerabilities found in well-known OSS packages and third-party packages that are often integrated in first-party applications.

You can download a third-party scanner (like Snyk), or write a bit of script to convert your dependency tree into a list and then compare it against a remote CVE database. In the npm world, you can begin this process with a command like: \verb+npm list -- depth=[depth]+ .

You can compare your findings against a number of databases, but for longevity’s
sake you may want to start with the NIST as it is funded by the US government and
likely to stick around for a long time.
\end{quote}

\begin{list2}
\item NOT an easy task, there are multiple complex applications doing this
\item The prices for those tools should indicate how complex this is
\end{list2}



\slide{Scan for vulnerabilities in Docker}

%\hlkimage{}{}

\begin{quote}
Looking to speed up your development cycles? Quickly detect and learn how to remediate CVEs in your images by running \verb+docker scan IMAGE_NAME+. Check out How to scan images for details.

Vulnerability scanning for Docker local images allows developers and development teams to review the security state of the container images and take actions to fix issues identified during the scan, resulting in more secure deployments. {\bf Docker Scan runs on Snyk engine}, providing users with visibility into the security posture of their local Dockerfiles and local images.
\end{quote}
Source: \link{https://docs.docker.com/engine/scan/}

\begin{list2}
\item Multiple other tools exist, and I would like to NOT name them
\item Searching for "scan docker for vulnerabilities" will show lots of links, YMMV
\item We already discussed some of the things Github scans for in repositories hosted there
\end{list2}


\slide{Secure Integration Techniques}

{\Large aka Use Security Principles}

\begin{list2}
\item Separation of Concerns\\
One way to mitigate this risk is to run the third-party integration on its
own server (ideally maintained by your organization).
\item Secure Package Management\\
One way of mitigating risk from
third-party packages installed this way is to individually audit specific versions of the
dependency, then “lock” the semantic version to the audited version number.
\end{list2}

I recommend using everything you know, be skeptical, be vigilant, only include what is really needed

I often recommend making mirrors of all dependencies, so I can reinstall from scratch without using an internet connection! If a package is removed in the official repositories, it can be real bad

\slide{}

%\hlkimage{}{}

\begin{quote}
{\Large \bf How one developer just broke Node, Babel and thousands of projects in 11 lines of JavaScript}

{\bf Code pulled from NPM – which everyone was using}

UPDATED Programmers were left staring at broken builds and failed installations on Tuesday after someone toppled the Jenga tower of JavaScript.

A couple of hours ago, {\bf Azer Koçulu unpublished more than 250 of his modules from NPM}, which is a popular package manager used by JavaScript projects to install dependencies.

Koçulu yanked his source code because, we're told, one of the modules was called Kik and that apparently attracted the attention of lawyers representing the instant-messaging app of the same name.

...

Unfortunately, one of those dependencies was left-pad. The code is below. It pads out the lefthand-side of strings with zeroes or spaces. And {\bf thousands of projects including Node and Babel relied on it.}
\end{quote}
Source: \link{https://www.theregister.com/2016/03/23/npm_left_pad_chaos/}


\slide{Using modules, sure -- but sometimes don't}

%\hlkimage{}{}

\begin{minted}[fontsize=\footnotesize]{javascript}
module.exports = leftpad;

function leftpad (str, len, ch) {
  str = String(str);

  var i = -1;

  if (!ch && ch !== 0) ch = ' ';

  len = len - str.length;

  while (++i < len) {
    str = ch + str;
  }

  return str;
}
\end{minted}
Source: left-pad module from NPM

If you depend on a simple module from a random stranger, and it gets removed or taken over ...

\slide{Reporting Security Issues -- security.txt}

%\hlkimage{}{}

\begin{quote}
Summary
“When security risks in web services are discovered by independent security researchers who understand the severity of the risk, they often lack the channels to disclose them properly. As a result, security issues may be left unreported. security.txt defines a standard to help organizations define the process for security researchers to disclose security vulnerabilities securely.”

security.txt files have been implemented by Google, Facebook, GitHub, the UK government, and many other organisations. In addition, the UK’s Ministry of Justice, the Cybersecurity and Infrastructure Security Agency (US), the French government, the Italian government, and the Australian Cyber Security Centre endorse the use of security.txt files.
\end{quote}

Source: \link{https://securitytxt.org/}

\begin{list2}
\item In the old days we always had abuse@domain.tld webmaster@domain.tld postmaster@domain.tld - still highly recommended!
\end{list2}

\slide{E-mail best current practice}

\begin{alltt}\small
MAILBOX       AREA                USAGE
-----------   ----------------    ---------------------------
ABUSE         Customer Relations  Inappropriate public behaviour
NOC           Network Operations  Network infrastructure
SECURITY      Network Security    Security bulletins or queries
...
MAILBOX       SERVICE             SPECIFICATIONS
-----------   ----------------    ---------------------------
POSTMASTER    SMTP                [RFC821], [RFC822]
HOSTMASTER    DNS                 [RFC1033-RFC1035]
USENET        NNTP                [RFC977]
NEWS          NNTP                Synonym for USENET
WEBMASTER     HTTP                [RFC 2068]
WWW           HTTP                Synonym for WEBMASTER
UUCP          UUCP                [RFC976]
FTP           FTP                 [RFC959]
\end{alltt}

Source:
\emph{RFC-2142 Mailbox Names for Common Services, Roles and Functions} D.
Crocker. May 1997

\exercise{ex:security.txt}


\exercise{ex:npm-depend-js}


\slide{Summary Dependencies}

%\hlkimage{}{}

\begin{quote}{\bf
Today’s web applications often have thousands, if not more, of individual dependencies} required for application functionality to operate as normal. Ensuring the security of each script in each dependency is a massive undertaking. As such it should be assumed that any third-party integration comes with at least some amount of
expected risk (in exchange for reduced development time).

However, while this risk cannot be eliminated, it can be mitigated in a number of ways.

...

To conclude, third-party dependencies always present risk, but careful integration
with some thought behind it can mitigate a lot of the upfront risk your application
would otherwise be exposed to.
\end{quote}

Source: \emph{Web Application Security}, Andrew Hoffman, 2020, ISBN: 9781492053118



\slide{Part III Summary and Conclusion}

%\hlkimage{}{}

\begin{quote}
You have completed Web Application Security. Ideally you have learned a lot about
securing and exploiting web applications that you can take elsewhere and put to good
use. There is still much more to learn. To become a web application security expert,
you will need to be exposed to many more topics, technologies, and scenarios.
This book isn’t a comprehensive glossary of web application security lessons; instead,
the topics were specifically chosen based on a few criteria.

You also should have some background on the history of software security and the
evolution of hacking. This has been foundational in the lead-up to web application
recon, offensive techniques, and defensive mitigations.
\end{quote}

\begin{list2}
\item Congratulations!
\end{list2}


\slide{KEA Wants to know}


\begin{list2}
\item Which KEA application is the worst, security wise?
\end{list2}

\exercise{ex:kea-web-apps}

\slide{Workshop next time!}

%\hlkimage{}{}

We will do another kind of teaching/working.

Meet in the same room, but move tables -- see if we can get a large common table

I will be available in class.

\begin{list2}
\item You will be working on the exercises, and
\item We can discuss which you may skip
\item I will let you know which ones I think are essential
\item We will discuss subjects in Web Application Security
\item You will have a chance to give constructive criticism on the book, slides, exercises, teaching, ...
\end{list2}

There will be cookies! (Store bought christmas cookies like brunkager, pebernødder etc.)


\slidenext{}

\end{document}
