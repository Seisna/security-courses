\documentclass[Screen16to9,17pt]{foils}
\usepackage{kea-slides}

\externaldocument{system-security-exercises}
\selectlanguage{english}

\begin{document}

\mytitlepage
{1. Overview of Computer Security}
{KEA Kompetence Computer Systems Security \the\year}

\slide{Goals for part I}

\hlkimage{6cm}{thomas-galler-hZ3uF1-z2Qc-unsplash.jpg}

 
\begin{list2}
\item
\end{list2}

{\small\hfill  Photo by Thomas Galler on Unsplash}
  
\slide{Plan for part I}

\begin{list1}
\item Subjects
\begin{list2}
\item Confidentiality, Integrity and Availability
\item Cost-Benefit Analysis
\item Risk Analysis
\item Human Issues
\item Access Control Matrix
\end{list2}
\item Exercises
\begin{list2}
\item Risk Analysis
\item Quick port scan intro with Nmap
\item Run Armitage against Metasploitable
\end{list2}
\end{list1}



\slide{Reading Summary}

\begin{list1}
\item Bishop chapter 1: An Overview of Computer Security
\item Bishop chapter 2: Access Control Matrix
\end{list1}

Quote, page xxix,\\
\begin{quote}
the second edition continues to focus on the principles underlying the field of computer and information security. Many newer examples show how these principles are applied, or not applied, today; but the principles themselves are as important today as they were in 2002, and earlier.
\end{quote}

\slide{Goals: Increase Security Awareness}

\hlkimage{8cm}{homer-end-is-near.jpg}
\begin{list1}
\item Fact of life: Software has errors, hardware fails
\item Sometimes software can be made to fail in interesting ways
\item Humans can be social engineered
\item We are being attacked by criminals - including paranoid governments
\end{list1}


\slide{Paranoia defined}

\hlkimage{11cm}{paranoia-definition.png}

Source: google paranoia definition

\slide{Face reality}

From the definition:
\begin{quote}
suspicion and mistrust of people or their actions {\bf without evidence or justification}.
{\bf the global paranoia about hackers and viruses}
\end{quote}

\begin{list1}
\item It is not paranoia when:
\begin{list2}
\item Criminals sell your credit card information and identity theft
\item Trade infected computers like a commodity
\item Governments write laws that allows them to introduce back-doors - and use these
\item Governments do blanket surveillance of their population, implement censorship, threaten citizens and journalist
\end{list2}
\end{list1}

\vskip 1cm
\centerline{You are not paranoid when there are people actively attacking you!}

I recommend we have appropriate paranoia (DK: passende paranoia)

\slide{Overlapping Security Incidents}

\hlkrightpic{12cm}{1cm}{datalaek-2019.png}

New data breaches nearly every week, these from danish news site \link{version2.dk}

Problem, we need to receive data from others

Data from others may contain malware

Have a job posting, yes\\
- then HR will be expecting CVs sent as .doc files

\slide{}

or the other way

{\Large\bf Attackers used a LinkedIn job ad\\
and Skype call to breach bank’s defences}
\hlkimage{12cm}{redbanc-skype-malware.png}

{\footnotesize
\link{https://nakedsecurity.sophos.com/2019/01/21/attackers-used-a-linkedin-job-ad-and-skype-call-to-breach-banks-defences/}}


\slide{Good security}

\hlkimage{15cm}{god-sikkerhed.pdf}

\begin{list1}
\item You always have limited resources for protection - use them as best as possible
\end{list1}



\slide{Recommendations}

\begin{list1}
\item {\bf Keep updated!}\\ - read web sites, books, articles, mailing lists, Twitter, ...
\item {\bf Always have a chapter on security evaluation }\\ - any process must have security, like RFC Request for Comments have
\item {\bf Incident Response}\\ - you WILL have security incidents, be prepared
\item {\bf Write down security policy}\\ - including software and e-mail policies
\end{list1}

\slide{Advice}

\begin{list1}
\item Use technology
\item Learn the technology - read the freaking manual
\item Think about the data you have, upload, facebook license?! WTF!
\item Think about the data you create - nude pictures taken, where will they show up?
\begin{list2}
\item Turn off features you don't use
\item Turn off network connections when not in use
\item Update software and applications
\item Turn on encryption: IMAP{\bf S}, POP3{\bf S},
  HTTP{\bf S} also for data at rest, full disk encryption, tablet encryption
\item Lock devices automatically when not used for 10 minutes
\item Dont trust fancy logins like fingerprint scanner or face recognition on cheap devices
\end{list2}
\end{list1}

But which features to disable? Let the security principles guide you

\slide{Confidentiality, Integrity and Availability}

\hlkimage{8cm}{cia-triad-uk.pdf}

\begin{list1}
\item We want to protect something
\item Confidentiality - data kept a secret
\item Integrity - data is not subjected to unauthorized changes
\item Availability - data and systems are available when needed
\end{list1}

\slide{What is data?}
\hlkimage{5cm}{Linus3-04041999.jpg}

\begin{list1}
\item Personal data you dont want to loose:
\begin{list2}
\item Wedding pictures
\item Pictures of your children
\item Sextapes
\item Personal finances
\end{list2}
\end{list1}

Source: picture of my son less than 24 hours old - precious!

\slide{Security is a process}

\begin{list1}
\item Remember:
\begin{list2}
\item what is information and security?
\item Data kept electronically
\item Data kept in physical form
\item Dont forget the human element of security
\end{list2}
\item Incident Response and Computer Forensics reaction to incidents
\item Good security is the result of planning and long-term work
\end{list1}
\vskip 1cm
\centerline{\color{titlecolor}\LARGE Security is a process, not a product, Bruce Schneier}

Source for quote: \link{https://www.schneier.com/essays/archives/2000/04/the_process_of_secur.html}


\slide{Work together}

\hlkimage{9cm}{Shaking-hands_web.jpg}

\begin{list1}
\item Team up!
\item We need to share security information freely
\item We often face the same threats, so we can work on solving these together
\end{list1}

\slide{Goals of Security}

\begin{list1}
\item Prevention - means that an attack will fail
\item Detection - determine if attack is underway, or has occured - report it
\item Recovery - stop attack, assess damage, repair damage
\end{list1}

\slide{Policy and Mechanism}

\begin{quote}
{\bf Definition 1-1.} A \emph{security policy} is a statement of what is, and what is not, allowed.

{\bf Definition 1-2.} A \emph{security mechanism} is a method, tool or procedure for enforcing a security policy.
\end{quote}

Quote from Matt Bishop, Computer Security section 1.3


\slide{Your data has already have been owned by criminals}

\hlkimage{13cm}{pwned.png}

\begin{list1}
\item Your data is already being sold, and resold on the Internet
\item Stop reusing passwords, use a password safe to generate and remember
\item Check you own email addresses on \link{https://haveibeenpwned.com/}
\end{list1}

\centerline{Go ahead try the web site - hold up your hand if you are in those dumps}



\slide{Balanced security}

\hlkimage{21cm}{afbalanceret-sikkerhed.pdf}

\begin{list1}
\item Better to have the same level of security
\item If you have bad security in some part - guess where attackers will end up
\item Hackers are not required to take the hardest path into the network
\item Realize there is no such thing as 100\% security
\end{list1}


\slide{Cost-Benefit Analysis}
% ROI?

\begin{list1}
\item Benefits of computer security must be weighed against value of assets
\item Often more expensive to add security mechanisms to a system, than designing them in
\end{list1}


\slide{Risk management defined}

\hlkimage{18cm}{shon-harris-risk-management.png}

Source: Shon Harris \emph{CISSP All-in-One Exam Guide}


\slide{Quantitative Risk Assessment}

\begin{quote}
In {\bf quantitative risk assessment an annualized loss expectancy (ALE) may be used to justify the cost of implementing countermeasures to protect an asset.} This may be calculated by multiplying the single loss expectancy (SLE), which is the loss of value based on a single security incident, with the annualized rate of occurrence (ARO), which is an estimate of how often a threat would be successful in exploiting a vulnerability.
\end{quote}

Quote from \url{https://en.wikipedia.org/wiki/Risk_assessment}

\slide{Annualized Loss Expectancy}

\begin{quote}
The annualized loss expectancy (ALE) is the product of the annual rate of occurrence (ARO) and the single loss expectancy (SLE). It is mathematically expressed as:

${\displaystyle {ALE}={ARO}\times {SLE}}$

Suppose that an asset is valued at \$100,000, and the Exposure Factor (EF) for this asset is 25\%. The single loss expectancy (SLE) then, is 25\% * \$100,000, or \$25,000.

The annualized loss expectancy is the product of the annual rate of occurrence (ARO) and the single loss expectancy.

For an annual rate of occurrence of one, the annualized loss expectancy is 1 * \$25,000, or \$25,000.

For an ARO of three, the equation is: ALE = 3 * \$25,000. Therefore: ALE = \$75,000
\end{quote}

Example from:\\
\url{https://en.wikipedia.org/wiki/Annualized_loss_expectancy}\\
\url{https://en.wikipedia.org/wiki/Single-loss_expectancy}

\slide{Qualitative risk analysis}

\begin{quote}
  {\bf Qualitative risk analysis} is a technique used to quantify risk associated with a particular hazard. Risk assessment is used for uncertain events that could have {\bf many outcomes and for which there could be significant consequences.} Risk is a function of probability of an event (a particular hazard occurring) and the consequences given the event occurs. Probability refers to the likelihood that a hazard will occur. {\bf In a qualitative assessment, probability and consequence are not numerically estimated, but are evaluated verbally using qualifiers like high likelihood, low likelihood, etc.} Qualitative assessments are good for screening level assessments when comparing/screening multiple alternatives or for when sufficient data is not available to support numerical probability or consequence estimates. Once numbers are inserted into the analysis (either by quantifying the likelihood of a hazard or quantifying the consequences) the analysis transitions to a semi-quantitative or quantitative risk assessment.
\end{quote}

Quote from \url{https://en.wikipedia.org/wiki/Qualitative_risk_analysis}



%\exercise{ex:risk-assessment-101}

\slide{Fokus: Asset management}

\hlkimage{7cm}{old_book_lumen_design_st_01.png}

\begin{list2}
\item Specielt relevant for mellemstore til store organisationer
\item Hvilke assets har vi?
\item Hvordan sikrer vi at vi ikke mister værdierne
\end{list2}


\slide{Hvad er asset management}

\begin{quote}
CIS Control 1:\\
Inventory and Control of Hardware Assets
Actively manage (inventory, track, and correct) all hardware devices on the network so that only
authorized devices are given access, and unauthorized and unmanaged devices are found and
prevented from gaining access.
\end{quote}
Source: \link{https://www.cisecurity.org/}

\begin{list2}
\item Hardware - både indkøbte, opkoblede, udlånte, stjålne ...
\item Software - licenser, indkøb, brug, opgraderingspriser
\item Virtuelle arkiver
\item ...
\end{list2}

\slide{Hardware asset management}

\hlkimage{10cm}{racktables-shot-indexpage.png}

\begin{list2}
\item Der findes mange systermer
\item Det anbefales at bruge specialiserede systemer, a la RackTables:\\
Have a list of all devices you've got,
Have a list of all racks and enclosures,
Mount the devices into the racks,
Maintain physical ports of the devices and links between them
\end{list2}

\slide{Software asset management - virtuelle arkiver}

\hlkimage{9cm}{datalaek-2019.png}

\begin{list2}
\item Software - licenser, indkøb, brug, opgraderingspriser
\item Virtuelle maskiner - er en server et asset, eller er det data?
\item IP adresser
\item Data arkiver - GDPR
\end{list2}



\slide{Human Issues and Organizational Problems}

\begin{list1}
\item Returning to resources, it takes a lot of resources and people to secure systems:
\begin{list2}
\item Time
\item Money
\item Skilled resources for designing, implementing, administer, monitor
\item Computing resources
\end{list2}
\item Often threats are focussed on outsiders, but insider threat can be common
\item Dont try to fix people problems with tech
\end{list1}

\slide{Cuckoo's Egg 1986 A real spy story}

\hlkimage{4cm}{The_Cuckoos_Egg.jpg}
\begin{list1}
\item
\emph{Cuckoo's Egg: Tracking a Spy Through the Maze of Computer
 Espionage},\\  Clifford Stoll
\item \emph{During his time at working for KGB, Hess is estimated to have broken into 400 U.S. military computers}\\
Source: \link{https://en.wikipedia.org/wiki/Markus_Hess}
\end{list1}




\slide{Morris Internet Worm - 30 years ago}

\begin{list1}
\item Used multiple vulnerabilities:
\begin{list2}
\item Sendmail Debug functionality, we have similar things and Google Hacking
\item Buffer overflow in fingerd, we still have those
\item Weak passwords/password cracking, list of 432 words and /usr/dict/words, same problem today
\item Trust between systems rsh, rexec, think Domain Admin today
\item Found new systems using /etc/hosts.equiv, .rhosts, .forward, netstat ...
\end{list2}
\item Also known as the Morris Internet Worm
\item \emph{The Internet Worm Program: An Analysis}\\
Purdue Technical Report CSD-TR-823, Eugene H. Spafford
\item Resulted in creation of the CERT, \link{http://www.cert.org}
\end{list1}

\slide{Internet Worms history repeats itself}

\begin{list1}
\item Camouflage, tried to hide, malware today hides as well
\begin{list2}
\item Program name set to 'sh', looks like a regular shell
\item Used fork() to change process ID (PID)
\item Worms in the 2000s spread quickly, like Code Red 2001 to approx 350.000 systems in a week
\item SQL Slammer "It spread rapidly, infecting most of its 75,000 victims within ten minutes."
\end{list2}
\vskip 1cm
\item New malware today can use the same strategies
\item Except a lot of malware tries to stay hidden, less noisy
\item Using a small password list of 50 words it is possible to create your own botnet with 100.000s
\end{list1}


\slide{Access Control Matrix Model}

\begin{list1}
\item Access Control Matrix model describes rights of subjects over all entities in a matrix
\item Example The Unix system read, write, execute for files, devices, processes
\item Everything is a file, sort of
\item A directory write permission allows one to rename files
\item Unix superuser root can access all files, processes etc.
\end{list1}



\slide{Trusted Computing Base}

{\bf Definition 20-6.} A \emph{trusted computing base} (TCB) consists of all protection mechanisms within a computer system -- including hardware, firmware, and software -- that are responsible for enforcing a security policy

Quote from Matt Bishop, Computer Security

Keeping this small, simple and understandable help keeping systems more secure.

Example the Qubes OS depend on few security-critical components:\\
\link{https://www.qubes-os.org/doc/security-critical-code/}


\exercise{ex:nmap-pingsweep}
\exercise{ex:nmap-synscan}
\exercise{ex:nmap-os}



\slidenext

\end{document}
