\documentclass[Screen16to9,17pt]{foils}
\usepackage{kea-slides}
\externaldocument{system-security-exercises}
\selectlanguage{english}

% VF1 Systemsikkerhed (10 ECTS)
% -----------------------------
% System Security. Den studerende kan udføre, udvælge, anvende, og implementere praktiske
% tiltag til sikring af firmaets udstyr og har viden og færdigheder der supportere dette.

% Viden
% Den studerende har viden om:
% * Generelle governance principper / sikkerhedsprocedurer
% * Væsentlige forensic processer
% * Relevante it-trusler
% * Relevante sikkerhedsprincipper til systemsikkerhed
% * OS roller ift. sikkerhedsovervejelser
% * Sikkerhedsadministration i DBMS.

% Færdigheder
% Den studerende kan:
% * Udnytte modforanstaltninger til sikring af systemer
% * Følge et benchmark til at sikre opsætning af enhederne
% * Implementere systematisk logning og monitering af enheder
% * Analysere logs for incidents og følge et revisionsspor
% * Kan genoprette systemer efter en hændelse.

% Kompetencer
% Den studerende kan:
% * håndtere enheder på command line-niveau
% * håndtere værktøjer til at identificere og fjerne/afbøde forskellige typer af endpoint trusler
% * håndtere udvælgelse, anvendelse og implementering af praktiske mekanismer til at forhindre, detektere og reagere over for specifikke it-sikkerhedsmæssige hændelser
% * håndtere relevante krypteringstiltag

\begin{document}

\mytitlepage
{0. Introduction}
{KEA Kompetence Computer Systems Security \the\year}

\hlkprofiluk


\slide{Goals for part I}

\hlkimage{6cm}{thomas-galler-hZ3uF1-z2Qc-unsplash.jpg}

 
\begin{list2}
\item Welcome, course goals and expectations, get to know eachother
\item Create a good starting point for learning
\item Learn to find resources, files and programs/libraries
\item Concrete Expectations
\item Prepare tools for the exercises, Prepare Virtual Machines
\end{list2}

{\small\hfill  Photo by Thomas Galler on Unsplash}


\slide{Plan for part I}

\begin{list2}
\item Introduce lecturer and students
\item Expectations for this course
\item Literature list walkthrough
\item Prepare tools for the exercises
\item Kali and Debian Linux introduction
\end{list2}

Exercises
\begin{list2}
\item Kali Linux installation
\item Debian Linux installation
\end{list2}
Linux is a toolbox we will use and participants will use virtual machines


\slide{Time schedule}

\begin{list1}
\item 17:00 - 18:15
Introduction and basics
\item 18:15 - 18:45 -- 30min break
%Eat dinner with your family if you like
\item 18:45 - 19:30 -- 45min Teaching
\item 19:30 - 19:45 -- 15min break
\item 19:45 - 20:30 -- 45min Teaching
\end{list1}

\vskip 1cm
\centerline{\Large This will be the basic plan for each evening}

\slide{Course Materials}

\begin{list1}
\item This material is in multiple parts:
\begin{list2}
%\item Introduktionsmateriale med baggrundsinformation
\item Slide shows - presentation - this file
\item Exercises - PDF which is updated along the way
\end{list2}
\item Books listed in the lecture plan Additional resources from the internet
\end{list1}

Note: the presentation slides are not a substitute for reading the books, papers and doing exercises, many details are not shown

A special thanks to William D. (Bill) Young
Associate Professor of Instruction and Research Scientist,
The University of Texas at Austin

When asked if I could borrow parts from his CS361 \emph{Introduction to Computer Security} he graciously wrote:\\
"You are welcome to use them freely.  You can credit me at the beginning." 


\slide{Fronter Platform}

\hlkimage{9cm}{fronter.png}

We will use fronter a lot, both for sharing educational materials and news during the course.

You will also be asked to turn in deliverables through fronter

\link{https://kea-fronter.itslearning.com/}

\vskip 5mm
\centerline{If you haven't received login yet, let us know}

\slide{Overview Diploma in IT-security}

\hlkimage{17cm}{kea-diplom-oversigt.png}


\slide{Course Data}
\hlkimage{8cm}{pawel-janiak-dxFi8Ea670E-unsplash.jpg}

{\Large\bf Course: VF 3  Computer Systems Security (10 ECTS)}

Teaching dates: tuesdays and thursdays 17:00 - 20:30\\
1/2 2022, 3/2 2022, 8/2 2022, 10/2 2022, 15/2 2022, 17/2 2022, 22/2 2022, 24/2 2022, 1/3 022, 3/3 2022, 8/3 2022, 10/3 2022, 15/3 2022, 17/3 2022

Exam: 31/3 2022 \hskip 12cm Photo by Pawel Janiak on Unsplash

\slide{Deliverables and Exam}

\begin{list2}
\item Exam
\item Individual: Oral based on curriculum
\item Graded (7 scale)
\item Draw a question with no preparation. Question covers a topic
\item Try to discuss the topic, and use practical examples
\item Exam is 30 minutes in total, including pulling the question and grading
\item Count on being able to present talk for about 10 minutes
\item Prepare material (keywords, examples, exercises, wireshark captures) for different topics so that you can use it to help you at the exam

\vskip 5mm
\item Deliverables:
\item 2 Mandatory assignments
\item Both mandatory assignments are required in order to be entitled to the exam.
\end{list2}


\slide{Course Description}

From: STUDIEORDNING Diplomuddannelse i it-sikkerhed August 2018\\
Indhold: Den studerende kan udføre, udvælge, anvende, og implementere praktiske
tiltag til sikring af firmaets udstyr og har viden og færdigheder der supportere dette.

{\bf Viden}

Den studerende har viden om:
\begin{list2}
\item Generelle governance principper / sikkerhedsprocedurer
\item Væsentlige forensic processer
\item Relevante it-trusler
\item Relevante sikkerhedsprincipper til systemsikkerhed
\item OS roller ift. sikkerhedsovervejelser
\item Sikkerhedsadministration i DBMS.
\end{list2}

\slide{Færdigheder}

{\bf Færdigheder}

Den studerende kan:
\begin{list2}
\item Udnytte modforanstaltninger til sikring af systemer
\item Følge et benchmark til at sikre opsætning af enhederne
\item Implementere systematisk logning og monitering af enheder
\item Analysere logs for incidents og følge et revisionsspor
\item Kan genoprette systemer efter en hændelse.
\end{list2}

\slide{Kompetencer}

{\bf Kompetencer}

Den studerende kan:
\begin{list2}
\item håndtere enheder på command line-niveau
\item håndtere værktøjer til at identificere og fjerne/afbøde forskellige typer af endpoint trusler
\item håndtere udvælgelse, anvendelse og implementering af praktiske mekanismer til at forhindre, detektere og reagere over for specifikke it-sikkerhedsmæssige hændelser
\item håndtere relevante krypteringstiltag
\end{list2}

Final word is the Studieordning which can be downloaded from\\
{\footnotesize \link{https://kompetence.kea.dk/uddannelser/it-digitalt/diplom-i-it-sikkerhed}\\
\link{Studieordning_for_Diplomuddannelsen_i_IT-sikkerhed_Aug_2018.pdf}}

\slide{Expectations alignment}

\hlkimage{7cm}{Shaking-hands_web.jpg}

Form groups of 2-3 students

In groups of 2 students, brainstorm for 5 minutes on what topics you would like to have in this course

Use 5 minutes more on Agreeing on 5 topics and prioritize these 5 topics

\vskip 1cm
PS We will from time to time have exercises, groups dont need to be the same each time.


\slide{Prerequisites}

\begin{list1}
\item This course includes exercises and getting the most of the course requires the participants to carry out these practical exercises
\item We will use Linux for some exercises but previous Linux and Unix knowledge is not needed
\item It is recommended to use virtual machines for the exercises
\item Security and most internet related security work has the following requirements:
\begin{list2}
\item Network experience
\item Server experience
\item TCP/IP principles - often in more detail than a common user
\item Programming is an advantage, for automating things
\item Some Linux and Unix knowledge is in my opinion a {\bf necessary skill}\\
-- too many new tools to ignore, and lots found at sites like Github and Open Source written for Linux
\end{list2}
\end{list1}


\slide{Goals and plans}

%\hlkimage{}{}

\begin{quote}
  “A goal without a plan is just a wish.”\\
  ― Antoine de Saint-Exupéry
\end{quote}

I want this course to
\begin{list2}
\item Include everything required by studieordningen
\item Be practical -- you can do something useful
\item Kickstart your journey into System Security\\
Getting the best books and papers
\item Present a lot of useful sources, tools
\item Prepare you for production use of the knowledge
\end{list2}



\slide{What is Infrastructure}


\hlkimage{10cm}{alexander-schimmeck-SeeM4AnkEHE-unsplash.jpg}

\begin{list2}
\item Enterprises today have a lot of computing systems supporting the business needs
\item These are very diverse and often discrete systems
\end{list2}

\hfill Photo by Alexander Schimmeck on Unsplash


\slide{Business Challenges}

\hlkimage{7cm}{adam-bignell-9tI2z5VZIZg-unsplash.jpg}

\begin{list2}
\item Accumulation of software
\item Legacy systems
\item Partners
\item Various types of data
\item Employee churn, replacement \hfill Photo by Adam Bignell on Unsplash
\end{list2}



\slide{Software Challenges}

\hlkimage{7cm}{john-barkiple-l090uFWoPaI-unsplash.jpg}

\begin{list2}
\item Complexity
\item Various languages
\item Various programming paradigms, client server, monolith, Model View Controller
\item Conflicting data types and available structures
\item Steam train vs electric train \hfill Photo by John Barkiple on Unsplash

\end{list2}




\slide{Developers Challenges}

\hlkimage{10cm}{kelly-sikkema-YK0HPwWDJ1I-unsplash.jpg}

\begin{list2}
\item Work in teams across organisation - and partners, vendors, sub-contractors
\item Work with legacy systems, old technology
\item Learn new Technologies \hfill Photo by Kelly Sikkema on Unsplash
\end{list2}




\slide{Integration Challenges}

% hands

\hlkimage{10cm}{thomas-drouault-IBUcu_9vXJc-unsplash.jpg}

\begin{list2}
\item Enable communication between components
\item Need mediator, interpreter, translator
\item Recognize standard patterns \hfill Photo by Thomas Drouault on Unsplash
\end{list2}


\slide{Course overview}

We will now go through a little from the Table of Contents in the books.

and the \emph{Lektionsplan}\\
\link{https://zencurity.gitbook.io/kea-it-sikkerhed/systemsikkerhed/lektionsplan}


\slide{Primary literature}

%\hlkrightpic{5cm}{0cm}{old_book_lumen_design_st_01.png}
Primary literature - not all chapters are part of the curriculum:
\begin{list2}
\item \emph{Computer Security: Art and Science}, 2nd edition 2019! Matt Bishop ISBN: 9780321712332 1440 pages
\item \emph{Defensive Security Handbook: Best Practices for Securing Infrastructure}, Lee Brotherston, Amanda Berlin ISBN: 978-1-491-96038-7 284 pages
\item \emph{Forensics Discovery}, Dan Farmer, Wietse Venema 2004, Addison-Wesley 240 pages. Can be found at http://www.porcupine.org/forensics/forensic-discovery/ but recommend buying it. Referenced below as FD

\end{list2}



\slide{Book: Computer Security: Art and Science}
\hlkimage{6cm}{computer-security-art-and-science.jpg}

\emph{Computer Security: Art and Science}, Matt Bishop ISBN: 9780321712332

{\footnotesize\link{https://www.pearson.com/us/higher-education/program/Bishop-Computer-Security-2nd-Edition/PGM25107.html}}

\slide{Book: Defensive Security Handbook (DSH)}

\hlkimage{6cm}{defensive-security-handbook.jpg}

\emph{Defensive Security Handbook: Best Practices for Securing Infrastructure}, Lee Brotherston, Amanda Berlin ISBN: 978-1-491-96038-7

\slide{Book: Forensics Discovery (FD)}

\hlkimage{6cm}{forensic-discovery.jpg}

\emph{Forensics Discovery}, Dan Farmer, Wietse Venema 2004, Addison-Wesley.

Can be found at http://www.porcupine.org/forensics/forensic-discovery/ but recommend buying it - to support and also better formatted for reading




\slide{Supporting literature books}
\begin{list2}
\item \emph{Linux Basics for Hackers Getting Started with Networking, Scripting, and Security in Kali}\\
OccupyTheWeb, December 2018, 248 pp. ISBN-13: 978-1-59327-855-7 - shortened LBfH
\item \emph{Kali Linux Revealed  Mastering the Penetration Testing Distribution}\\
Raphaël Hertzog, Jim O'Gorman - shortened KLR
\item \emph{The Debian Administrator’s Handbook}, Raphaël Hertzog and Roland Mas\\
\url{https://debian-handbook.info/} - shortened DEB
\end{list2}



\slide{Book: Linux Basics for Hackers (LBfH)}

\hlkimage{6cm}{LinuxBasicsforHackers_cover-front.png}

\emph{Linux Basics for Hackers
Getting Started with Networking, Scripting, and Security in Kali}
by OccupyTheWeb
December 2018, 248 pp.
ISBN-13:
9781593278557

\link{https://nostarch.com/linuxbasicsforhackers}
Not curriculum but explains how to use Linux


\slide{Book: Kali Linux Revealed (KLR)}

\hlkimage{6cm}{kali-linux-revealed.jpg}

\emph{Kali Linux Revealed  Mastering the Penetration Testing Distribution}

\link{https://www.kali.org/download-kali-linux-revealed-book/}\\
Not curriculum but explains how to install Kali Linux


\slide{Book: The Debian Administrator’s Handbook (DEB)}

\hlkimage{6cm}{book-debian-administrators-handbook.jpg}

\emph{The Debian Administrator’s Handbook}, Raphaël Hertzog and Roland Mas\\
\url{https://debian-handbook.info/} - shortened DEB

Not curriculum but explains how to use Debian Linux


\exercise{ex:downloadKLR}
\exercise{ex:sw-downloadDEB}


%%% Break?

\slide{Technologies used in this course}

The following tools and environments are examples that may be introduced in this course:

\begin{list2}
\item Programming languages and frameworks Java, Python, regular expressions
\item Development environments -- choose your own IDE / Editor -- I use {\bf Atom}
\item Networking and network protocols: TCP/IP, HTTP, DNS, Netflow
%\item Formats XML, JSON, CSV, raw text, web scraping
\item Web technologies and services: REST, API, HTML5, CSS, JavaScript
\item Tools like cURL, Zeek, Git and Github
%\item Message queueing systems: MQ and Redis could be added
\item Aggregated example platforms: Elastic stack, logstash, elasticsearch, kibana, grafana, Filebeat
%\item Cloud and virtualisation Docker, Kubernetes, Azure, AWS, microservices -- can be added
\end{list2}

\centerline{This list is not complete or a promise }


\slide{Hackerlab Setup}

\hlkimage{6cm}{hacklab-1.png}

\begin{list2}
\item Hardware: modern laptop CPU with virtualisation\\
Dont forget to enable hardware virtualisation in the BIOS
\item Virtualisation software: VMware, Virtual box, HyperV pick your poison
\item Hackersoftware: Kali Virtual Machine amd64 64-bit \link{https://www.kali.org/}
\item Linux server system: Debian 10 Buster amd64 64-bit \link{https://www.debian.org/}
\item Setup instructions can be found at \link{https://github.com/kramse/kramse-labs}
\end{list2}

\centerline{It is enough if these VMs are pr team}

\slide{Aftale om test af netværk}

\vskip 1cm
{\bfseries Straffelovens paragraf 263 Stk. 2. Med bøde eller fængsel
  indtil 6 måneder
straffes den, som uberettiget skaffer sig adgang til en andens
oplysninger eller programmer, der er bestemt til at bruges i et anlæg
til elektronisk databehandling.}

Hacking kan betyde:
\begin{list2}
\item At man skal betale erstatning til personer eller virksomheder
\item At man får konfiskeret sit udstyr af politiet
\item At man, hvis man er over 15 år og bliver dømt for hacking, kan
  få en bøde - eller fængselsstraf i alvorlige tilfælde
\item At man, hvis man er over 15 år og bliver dømt for hacking, får
en plettet straffeattest. Det kan give problemer, hvis man skal finde
et job eller hvis man skal rejse til visse lande, fx USA og
Australien
\item Frit efter: \link{http://www.stophacking.dk} lavet af Det
  Kriminalpræventive Råd
\item Frygten for terror har forstærket ovenstående - så lad være!
\end{list2}




\slide{Mixed exercises}
Then we will do a mixed bag of exercises to introduce technologies, find your current knowledge level with regards to:

\begin{list2}
\item Linux
\item Linux command line
\item Git, Python and Ansible
\end{list2}


{\bf Note: today we will consider all these optional, we wont be able to do them all}

Later we will return to them!

\slide{Exercise CHAOS: Don't Panic -- have fun learning}

\hlkimage{6cm}{dont-panic.png}

\begin{quote}
“It is said that despite its many glaring (and occasionally fatal) inaccuracies, the Hitchhiker’s Guide to the Galaxy itself has outsold the Encyclopedia Galactica because it is slightly cheaper, and because it has the words ‘DON’T PANIC’ in large, friendly letters on the cover.”
\end{quote}
Hitchhiker’s Guide to the Galaxy, Douglas Adams

\slide{Your lab setup}

\begin{list2}
\item Go to GitHub, Find user Kramse, click through kramse-labs
\item Look into the instructions for the Virtual Machine -- Debian only

\item Get the lab instructions, from\\ {\footnotesize\url{https://github.com/kramse/kramse-labs/tree/master/suricatazeek}}
\end{list2}

Yes, reusing instruction for the Suricata Zeek workshop - tested and working!


\slide{Command prompts in Unix}

\begin{list1}
\item Shells :
  \begin{list2}
    \item sh - Bourne Shell
\item bash - Bourne Again Shell, often the default in Linux
\item ksh - Korn shell, original by David Korn, but often the public domain version used
\item csh - C shell, syntax similar to C language
\item Multiple others available, zsh is very popular
  \end{list2}
\item Windows have \verb+command.com+, \verb+cmd.exe+ but PowerShell is more similar to the Unix shells
\item Used for scripting, automation and programs
\end{list1}



\slide{Command prompts}


\begin{alltt}
\small
[hlk@fischer hlk]$ id
uid=6000(hlk) gid=20(staff) groups=20(staff),
0(wheel), 80(admin), 160(cvs)
[hlk@fischer hlk]$ sudo -s
[root@fischer hlk]#
[root@fischer hlk]# id {\bf
uid=0(root) gid=0(wheel)} groups=0(wheel), 1(daemon),
20(staff), 80(admin)
[root@fischer hlk]#
\end{alltt}

Note the difference between running as root and normal user. Usually books and instructions will use a prompt of hash mark \verb+#+ when the root user is assumed and dollar sign \verb+$+ when a normal user prompt.

\slide{Command syntax}


\begin{alltt}
echo [-n] [string ...]
\end{alltt}

\begin{list1}
\item Commands are written like this:
\begin{list2}
\item Always begin with the command to execute, like \verb+echo+ above
\item Options typically short form with single dash \verb+-n+
\item or long options \verb+--version+
\item Some commands allow grouing options, \verb+tar -c -v -f+ becomes \verb+tar -cvf+\\
NOTE: some options require parameters, so \verb+tar -c -f filename.tar+ not equal to \verb+tar -fc filename.tar+
\item Optional options are in brackets \verb+[ ]+
\item Output can be saved using redirection, into new file/overwrite \verb+echo hello > file.txt+ or append \verb+echo hello >> file.txt+
\item Read from files \verb+wc -l file.txt+ or pipe output into input \verb+cat file.txt | wc -l+\\
\verb+wc+ is word count, and option l is count lines
\end{list2}
\end{list1}



\slide{Unix Manual system}

\hlkimage{7cm}{images/Unix-command-1.pdf}

\begin{quote}
 It is a book about a Spanish guy called Manual. You should read it.
       -- Dilbert
\end{quote}

\begin{list1}
\item Manual system in Unix is always there!
\item Key word search \verb+man -k+ see also \verb+apropos+
\item Different sections, can be chosen
\end{list1}

See \verb+man crontab+ the command vs the file format in section 5 \verb+man 5 crontab+



\slide{A manual page}

\begin{alltt}\footnotesize
\small
NAME
     cal - displays a calendar
SYNOPSIS
     cal [-jy] [[month]  year]
DESCRIPTION
   cal displays a simple calendar.  If arguments are not specified, the cur-
   rent month is displayed.  The options are as follows:
   -j      Display julian dates (days one-based, numbered from January 1).
   -y      Display a calendar for the current year.

The Gregorian Reformation is assumed to have occurred in 1752 on the 3rd
of September.  By this time, most countries had recognized the reforma-
tion (although a few did not recognize it until the early 1900's.)  Ten
days following that date were eliminated by the reformation, so the cal-
endar for that month is a bit unusual.
\end{alltt}

\slide{The year 1752}

\begin{alltt}\footnotesize
  user@Projects:$ cal 1752
...
         April                  May                   June
  Su Mo Tu We Th Fr Sa  Su Mo Tu We Th Fr Sa  Su Mo Tu We Th Fr Sa
            1  2  3  4                  1  2      1  2  3  4  5  6
   5  6  7  8  9 10 11   3  4  5  6  7  8  9   7  8  9 10 11 12 13
  12 13 14 15 16 17 18  10 11 12 13 14 15 16  14 15 16 17 18 19 20
  19 20 21 22 23 24 25  17 18 19 20 21 22 23  21 22 23 24 25 26 27
  26 27 28 29 30        24 25 26 27 28 29 30  28 29 30
                        31
          July                 August              September
  Su Mo Tu We Th Fr Sa  Su Mo Tu We Th Fr Sa  Su Mo Tu We Th Fr Sa
            1  2  3  4                     1  {\bf        1  2 14 15 16}
   5  6  7  8  9 10 11   2  3  4  5  6  7  8  17 18 19 20 21 22 23
  12 13 14 15 16 17 18   9 10 11 12 13 14 15  24 25 26 27 28 29 30
  19 20 21 22 23 24 25  16 17 18 19 20 21 22
  26 27 28 29 30 31     23 24 25 26 27 28 29
                        30 31
...
\end{alltt}


\slide{Linux configuration in /etc}

.
\hlkrightpic{8cm}{0cm}{Unix-vfs.pdf}
\begin{list2}
\item Command line is a requirement in the \emph{studieordningen} \smiley
\item Linux and Unix uses a single virtual file system\\
\url{https://en.wikipedia.org/wiki/Unix_filesystem}
\item No drive letters like the ones in MS-DOS and Microsoft Windows
\item Everything starts at the root of the file system tree \verb+/+ - NOTE: \emph{forward slash}
\item One special directory is \verb+/etc/+ and sub directories which usually contain a lot of configuration files
\end{list2}

\slide{Installing software in Debian -- apt}

%\hlkimage{}{}

\begin{alltt}\footnotesize
DESCRIPTION
apt provides a high-level commandline interface for the package management system. It is intended as an end user interface
and enables some options better suited for interactive usage by default compared to more specialized APT tools like apt-get(8)
and apt-cache(8).

update (apt-get(8))
  update is used to download package information from all configured sources. Other commands operate on this data to e.g.
  perform package upgrades or search in and display details about all packages available for installation.

upgrade (apt-get(8))
  upgrade is used to install available upgrades of all packages currently installed on the system from the sources configured
  via sources.list(5). New packages will be installed if required to satisfy dependencies, but existing packages will never
  be removed. If an upgrade for a package requires the removal of an installed package the upgrade for this package isn't performed.

full-upgrade (apt-get(8))
  full-upgrade performs the function of upgrade but will remove currently installed packages if this is needed to upgrade the
  system as a whole.
\end{alltt}

\begin{list2}
  \item Install a program using apt, for example \verb+apt install nmap+
\end{list2}


\slide{Ansible}

\hlkimage{2cm}{Ansible_logo.png}

\begin{quote}
From my course materials:\\
Ansible is great for automating stuff, so by running the playbooks we can get a whole lot of programs installed, files modified - avoiding the Vi editor.
\end{quote}

\begin{list2}
\item Easy to read, even if you don't know much about YAML
\item \link{https://www.ansible.com/} and \link{https://en.wikipedia.org/wiki/Ansible_(software)}
\item Great documentation\\
 \link{https://docs.ansible.com/ansible/latest/collections/ansible/builtin/apt_module.html}
\end{list2}


\slide{Ansible Dependencies}

\hlkimage{10cm}{python-logo.png}

\begin{list2}
\item Ansible based on Python, only need Python installed\\
\link{https://www.python.org/}
\item Often you use Secure Shell for connecting to servers\\
\link{https://www.openssh.com/}
\item Easy to configure SSH keys, for secure connections
\end{list2}


\slide{Ansible playbooks}

Example playbook content, installing software using APT:
\begin{alltt}\small
apt:
    name: "\{\{ packages \}\}"
    vars:
      packages:
        - nmap
        - curl
        - iperf
        ...
\end{alltt}

Running it:
\begin{minted}[fontsize=\small]{shell}
cd kramse-labs/suricatazeek
ansible-playbook -v 1-dependencies.yml 2-suricatazeek.yml 3-elasticstack.yml 4-configuration.yml
\end{minted}

"YAML (a recursive acronym for "YAML Ain't Markup Language") is a human-readable data-serialization language."\\
\link{https://en.wikipedia.org/wiki/YAML}

\slide{Python and YAML -- Git}

\hlkimage{7cm}{git-logo.png}

\begin{list2}
\item We need to store configurations
\item Run playbooks
\item Problem: Remember what we did, when, how
\item Solution: use git for the playbooks
\item Not the only version control system, but my preferred one
\end{list2}

\slide{Alternative}

\hlkimage{10cm}{manual-install-es.png}

My playbooks allow installation of a whole Elastic stack in less then 10 minutes,

compare to:\\
\emph{Getting started with the Elastic Stack}\\
{\footnotesize\link{https://www.elastic.co/guide/en/elastic-stack-get-started/current/get-started-elastic-stack.html}}


\slide{Git getting started}

{\bf Hints:}\\
Browse the Git tutorials on \link{https://git-scm.com/docs/gittutorial}\\
and \link{https://guides.github.com/activities/hello-world/}

\begin{list2}
\item What is git
\item Terminology
\end{list2}

Note: you don't need an account on Github to download/clone repositories, but having an acccount allows you to save repositories yourself and is recommended.

\slide{Demo: Ansible, Python, Git!}

\begin{quote}
  Running Git will allow you to clone repositories from others easily. This is a great way to get new software packages, and share your own.

  Git is the name of the tool, and Github is a popular site for hosting git repositories.
\end{quote}


\begin{list2}
\item Go to \link{https://github.com/kramse/kramse-labs}
\item Lets explore while we talk
\end{list2}


\slide{Demo: output from running a git clone}

\begin{alltt}\footnotesize
user@Projects:tt$ {\bf git clone https://github.com/kramse/kramse-labs.git}
Cloning into 'kramse-labs'...
remote: Enumerating objects: 283, done.
remote: Total 283 (delta 0), reused 0 (delta 0), pack-reused 283
Receiving objects: 100% (283/283), 215.04 KiB | 898.00 KiB/s, done.
Resolving deltas: 100% (145/145), done.

user@Projects:tt$ {\bf cd kramse-labs/}

user@Projects:kramse-labs$ {\bf ls}
LICENSE  README.md  core-net-lab  lab-network  suricatazeek  work-station
user@Projects:kramse-labs$ git pull
Already up to date.
\end{alltt}

for reference at home later



\exercise{ex:basicVM}

\exercise{ex:basicDebianVM}


\exercise{ex:basicLinuxetc}


\exercise{ex:debian-firewall}

\exercise{ex:git-tutorial}

\exercise{ex:basicansible}


\slidenext{Buy the books!}



\end{document}
