\documentclass[Screen16to9,17pt]{foils}
\usepackage{kea-slides}

\externaldocument{system-security-exercises}
\selectlanguage{english}

\begin{document}

\mytitlepage
{5. Hybrid Policies}
{KEA Kompetence Computer Systems Security \the\year}

\slide{Goals for part I}

\hlkimage{6cm}{thomas-galler-hZ3uF1-z2Qc-unsplash.jpg}

 
\begin{list2}
\item Expand on the models this week, until we look at Github as a common example
\item Start talking about side-channels, even though we give them few thoughts in real life
\end{list2}

{\small Photo by Thomas Galler on Unsplash}


\slide{Plan for part I}

\begin{list1}
\item Subjects
\item Part I: Real-life security models
\begin{list2}
\item Chinese Wall model - Confidentiality and Integrity
\item Clinical Information Systems security model
\item Originator Controlled Access Control
\item Role-based Access Control (RBAC)
\item Break-the-glass Policies
\end{list2}
\item Part II: Side channels, strange and complex vulnerabilities
\begin{list2}
\item Side Channels and Deducibility
\item Memory errors and Row hammer
\end{list2}
\item Exercises
\begin{list2}
\item RBAC Access permissions on GitHub
\end{list2}
\end{list1}



\slide{Reading Summary}

\begin{list1}
\item Bishop chapter 8: Hybrid Policies
\item Browse: Using Memory Errors to Attack a Virtual Machine paper
\item An Experimental Study of DRAM Disturbance Errors
\item Exploiting the DRAM rowhammer bug to gain kernel privileges
\item \link{https://en.wikipedia.org/wiki/Row\_hammer}
\end{list1}


\slide{Chinese Wall model - Confidentiality and Integrity}


\begin{list1}
\item A model of a security policy that refers equally to confidentiality and integrity
\item Designed to provide controls that mitigate conflict of interest in commercial organizations
\item Natural environment stock exchange or investment house
\item {\bf Definition 8-1.} The \emph{objects} of the database are items of information related to a company.
\item {\bf Definition 8-2.} A \emph{company dataset} (CD) contains objects related to a single company.
\item {\bf Definition 8-3.} A \emph{conflict of interest} (COI) class contains the datasets of companies in competition.
\item
\item
\end{list1}

See also
\url{https://en.wikipedia.org/wiki/Chinese_wall}\\
\url{https://en.wikipedia.org/wiki/Brewer_and_Nash_model}

\slide{Chinese Wall model - Informal Description}

$P R(S)$ is the set of objects that S has read.

\hlkimage{22cm}{bishop-cw-model-1.png}

You can only read future objects that will not result in a conflict of interest

Screenshot from:
\emph{Computer Security: Art and Science}, Bishop, 2019

\slide{Chinese Wall model - Informal Description}

\hlkimage{22cm}{bishop-cw-model-2.png}

Adding publically accessible information, we can consider some object/informatioin can be considered sanitized - allowed for reading by all

Screenshot from:
\emph{Computer Security: Art and Science}, Bishop, 2019


\slide{Chinese Wall model - Writing objects}

\hlkimage{17cm}{bishop-cw-model-3.png}

\emph{Computer Security: Art and Science}, Bishop, 2019


%\slide{Chinese Wall model related to other models}

%\begin{list1}
%\item Books provides examples with regards to
%\item
%\item Bell-LaPadula Model
%\item
%\item Clark-Wilson Integrity Model
%\end{list1}

\slide{Clinical Information Systems security model Definitions}

\begin{list1}
\item A model of a
\item {\bf Definition 8-6.} A \emph{patient} is the subject of medical records, or an agent for that person who can give consent for the person to be treated.
\item {\bf Definition 8-7.} \emph{Personal health information} is information about a patient's health or treatment enabling that patient to be identified.
\item {\bf Definition 8-8.} A \emph{clinician} is a health-care professional who has access to personal health information while performing his or her job.
\end{list1}

\slide{Clinical Information Systems security model Principles}

\begin{list2}
\item {\bf Access Principle 1:} Each medical record has Access Control List (ACL)
\item {\bf Access Principle 2:} One clinician on ACL can add others
\item {\bf Access Principle 3:} Responsible clinician must inform patient of ACL, when record opened, need consent
\item {\bf Access Principle 4:} Access to records must be logged clinician, date and time, same for deletion
\item {\bf Creation Principle:} A clinician may open a record as a result of referral
\item {\bf Deletion Principle:} Clinical information cannot be deleted from a medical record until the appropriate time has passed
\item {\bf Confinement Principle:} Information from one medical record can only be appended to another if the ACL is of the second is a subset of the ACL of the first
\item {\bf Aggregation Principle:} Measures for preventing the aggregation of patient data must be effective
\item {\bf Enforcement Principle:} Any computer system that handles medical records must have a subsystem that enforces the preceeding principes. Subject to evaluation by independent auditors.
\end{list2}

Shortened by me. From:
\emph{Computer Security: Art and Science}, Bishop, 2019

\slide{Clinical system security: Interim guidelines}
\hlkimage{19cm}{anderson-nine-principles-of-data-security.png}

Source:
\emph{Clinical system security: Interim guidelines}, Ross Anderson, 1996


\slide{Originator Controlled Access Control}

\begin{list1}
\item Problem: MAC and DAC does not handle environments in which the originators of documents retain control over them even after dissemination
\begin{list2}
\item 1. The owner of an object cannot change the acccess controls of the object
\item 2. When an object is copied, the acess controle restrictions of that source are copied and bound to the target of the copy.
\item 3. The creator (originator) can alter the access control restrictions on a per-object and per-object basis.
\end{list2}
\item 1. and 2. are MAC, 3. DAC
\end{list1}

While important I dont see much talk about originator controlled models in the real world, except for DRM

\slide{Digital Rights Management}

\begin{list1}
\item The owner of content, movies, books, may wish to control its distribution.
\item {\bf Definition 8-9.} \emph{Digital rights management} (DRM) is the persistent control of digital content
\item \emph{Computer Security: Art and Science}, Bishop, 2019
\item Relates to copyright, fair use, quoting stuff
\item huge controversies in this space, newest examples:\\
"link-skatten" (artikel 11/15) og uploadfiltrene (artikel 13/17)
\item Most often based on some crypto and keys
\end{list1}

DRM also is used to prevent people from upgrading devices they \emph{own}



\slide{Role-based Access Control (RBAC)}

\begin{quote}
In computer systems security, {\bf role-based access control (RBAC)}[1][2] or role-based security[3] is an approach to restricting system access to unauthorized users. It is used by the majority of enterprises with more than 500 employees,[4] and can implement mandatory access control (MAC) or discretionary access control (DAC).

Role-based access control (RBAC) is a policy-neutral access-control mechanism defined around {\bf roles and privileges}. The components of RBAC such as role-permissions, user-role and role-role relationships make it simple to perform user assignments. A study by NIST has demonstrated that RBAC addresses many needs of commercial and government organizations[citation needed]. RBAC can be used to facilitate administration of security in large organizations with hundreds of users and thousands of permissions. Although RBAC is different from MAC and DAC access control frameworks, it can enforce these policies without any complication.
\end{quote}
Quote from \url{https://en.wikipedia.org/wiki/Role-based_access_control}

\slide{Role-based Access Control (RBAC), \emph{Computer Security} Bishop}

\hlkimage{12cm}{bishop-rbac.png}


\exercise{ex:github-perms}


\slide{Break-the-glass Policies}

\begin{list1}
\item {\bf Definition 8-19.} A \emph{break-the-glass} policy allows access controls to be overriden in a controlled manner.
\item Example uses may be:
\item Getting the root password for a server
\item Getting access to someones email inbox
\end{list1}

\slide{Covert Channels}

\begin{quote}
Definition: A covert channel is a path for the illegal flow of
information between subjects within a system, utilizing system
resources that were not designed to be used for inter-subject
communication.
\end{quote}

\begin{list1}
\item Note several features of this definition:
\begin{list2}
\item Information flows in violation of the security metapolicy
though not necessarily in violation of the policy.
\item The flow is between subjects within the system; two human
users talking over coffee is not a covert channel.
\item The flow occurs via system resources (file attributes, flags,
clocks, etc.) that were not intended as communication
channels.
\end{list2}
\end{list1}

List from William D. (Bill) Young
\link{https://www.cs.utexas.edu/~byoung/cs361/lecture14.pdf}


\slide{Side Channels and Deducibility}

\begin{list1}
\item It is possible to distinguish many types of covert channels,
depending on the attribute manipulated:
\begin{list2}
\item Timing: how much time did a computation take?
\item Implicit: what control path does the program take?
\item Termination: does a computation terminate?
\item Probability: what is the distribution of system events?
\item Resource exhaustion: is some resource depleted?
\item Power: how much energy is consumed?
\end{list2}
\item In practice, many researchers distinguish only storage and timing
channels.
\end{list1}

List from William D. (Bill) Young
\link{https://www.cs.utexas.edu/~byoung/cs361/lecture14.pdf}

\slide{Hardware Side Channels -- deducting AES keys 1}

\hlkimage{13cm}{troopers19-hw-side-channel-1.png}

Hardware was Arduino Nano 16 MHz, Not secure

Picture from
\emph{Hardware Side Channel attacks on the cheapest}, Alyssa Milburn and Albert Spruyt,
\link{https://troopers.de/downloads/troopers19/TROOPERS19_NGI_RT_Hardware_Side_Channels.pdf}

\slide{Hardware Side Channels -- deducting AES keys 2}

\hlkimage{15cm}{troopers19-hw-side-channel-2.png}

Picture from
\emph{Hardware Side Channel attacks on the cheapest}, Alyssa Milburn and Albert Spruyt,
\link{https://troopers.de/downloads/troopers19/TROOPERS19_NGI_RT_Hardware_Side_Channels.pdf}


\slide{Memory errors and Row hammer}

\emph{Flipping Bits in Memory Without Accessing Them: An Experimental Study of DRAM Disturbance Errors} Yoongu Kim, Ross Daly, Jeremie Kim, Chris Fallin, Ji Hye Lee, Donghyuk Lee, Chris Wilkerson, Konrad Lai, Onur Mutlu\\
{\footnotesize\link{http://users.ece.cmu.edu/~yoonguk/papers/kim-isca14.pdf}}

\emph{Exploiting the DRAM rowhammer bug to gain kernel privileges}
Project Zero blog, Posted by Mark Seaborn, sandbox builder and breaker, with contributions by Thomas Dullien, reverse engineer\\
{\footnotesize\link{https://googleprojectzero.blogspot.com/2015/03/exploiting-dram-rowhammer-bug-to-gain.html}}

\url{https://en.wikipedia.org/wiki/Row_hammer}



\slidenext

\end{document}
