\documentclass[Screen16to9,17pt]{foils}
\usepackage{kea-slides}

\externaldocument{system-security-exercises}
\selectlanguage{english}

\begin{document}

\mytitlepage
{3. User Accounts}
{KEA Kompetence Computer Systems Security \the\year}



\slide{Goals for part II}

\hlkimage{6cm}{thomas-galler-hZ3uF1-z2Qc-unsplash.jpg}


\begin{list2}
\item Talk about user accounts in general
\end{list2}

{\small\hfill  Photo by Thomas Galler on Unsplash}

\slide{Plan for part II}

\begin{list1}
\item Subjects
\begin{list2}
\item What are user accounts -- user ID
\item Securing Administrative User Accounts
\item Securing Normal User Accounts
\item Databases: RDBMS, PostgreSQL, Deadlocks
\end{list2}
\item Exercises
\begin{list2}
\item Databases - discussion about Relational Database Management System RDBMS Model and NoSQL
\end{list2}
\end{list1}



\slide{Reading Summary}

MLSH SectionI: Setting up a Secure Linux System
\begin{list1}
\item Chapter 1: Running Linux in a Virtual Environment
\item Chapter 2: Securing Administrative User Accounts
\item Chapter 3: Securing Normal User Accounts
\end{list1}



\slide{Separation of duty ns function}

\begin{quote}
{\bf Separation of duties} (SoD; also known as Segregation of Duties) is the concept of having more than one person required to complete a task. In business the separation by sharing of more than one individual in one single task is an internal control intended to prevent fraud and error.
\end{quote}

Quote from \url{https://en.wikipedia.org/wiki/Separation_of_duties}

\begin{quote}
{\bf Separation of function}. Developers do not develop new programs on production systems because of the potential threat to production data.
\end{quote}
\emph{Computer Security}, Matt Bishop, 2019

Danish: Funktionsadskillelse



\slide{Accuracy vs disclosure}

Lipner five commercial requirements:
\begin{list2}
\item[1.] Users will not write their own programs, but use existing
  production software.
\item[2.] Programmers develop and test applications on a nonproduction system, possibly using contrived data.
\item[3.] Moving applications from development to production requires a special process.
\item[4.] This process must be controlled and audited.
\item[5.] Managers and auditors must have access to system state and system logs
\end{list2}


Available from\\ {\footnotesize\link{https://csrc.nist.gov/CSRC/media/Publications/conference-paper/1982/05/24/proceedings-5th-seminar-dod-computer-security-initiative/documents/1982-5th-seminar-proceedings.pdf}}



\slide{The Biba Model}

Ken Biba (1977) proposed three different integrity access control
policies.

\begin{list2}
\item[1] The Low Water Mark Integrity Policy
\item[2] The Ring Policy
\item[3] Strict Integrity
\item All assume that we associate integrity labels with subjects and
objects, analogous to clearance levels in BLP.
\item Only Strict Integrity had much continuing influence. It is the one
typically referred to as the “Biba Model” or “Biba Integrity.”
\end{list2}

% https://www.cs.utexas.edu/~byoung/cs361/syllabus361.html
% https://www.cs.utexas.edu/~byoung/cs361/lecture21.pdf

Example page 178 mentions that this was implemented in FreeBSD


\slide{Lipners Integrity Matrix Model}

\hlkimage{15cm}{lipner-model-levels.png}

\emph{Non-Discretionary Controls for Commercial Applications}, Steven B. Lipner, IEEE Symposium on Security and Privacy, and Fifth Seminar on the DoD Computer Security Initiative, 1982

\slide{Lipners Integrity Matrix Model}

\hlkimage{20cm}{lipner-1982.png}

\emph{Non-Discretionary Controls for Commercial Applications}, Steven B. Lipner, IEEE Symposium on Security and Privacy, and Fifth Seminar on the DoD Computer Security Initiative, 1982

\slide{Lipners Integrity Matrix Model}

\hlkimage{20cm}{lipner-with-integrity.png}

\emph{Non-Discretionary Controls for Commercial Applications}, Steven B. Lipner, IEEE Symposium on Security and Privacy, and Fifth Seminar on the DoD Computer Security Initiative, 1982



% \slide{One source of truth}
% Maybe later

\slide{Clark-Wilson Integrity Model}

A {\bf well-formed transaction} from one consistent state to another consistent state.

\begin{list2}
\item Constrained Data Items: CDIs are the objects whose
integrity is protected
\item Unconstrained Data Items: UDIs are objects not covered by
the integrity policy
\item Transformation Procedures: TPs are the only procedures
allowed to modify CDIs, or take arbitrary user input and
create new CDIs. Designed to take the system from one valid
state to another.
\item Integrity Verification Procedures: IVPs are procedures
meant to verify maintainance of integrity of CDIs.
\end{list2}

\emph{A Comparison of Commercial and Military Computer Security Policies},
David D. Clark and David R. Wilson, 1987



\slide{Clark-Wilson Integrity Model}

\begin{quote}
The model uses a three-part relationship of subject/program/object (where program is interchangeable with transaction) known as a triple or an access control triple. Within this relationship, subjects do not have direct access to objects. Objects can only be accessed through programs
\end{quote}

\emph{A Comparison of Commercial and Military Computer Security Policies},
David D. Clark and David R. Wilson, 1987


See also
\url{https://en.wikipedia.org/wiki/Clark%E2%80%93Wilson_model}


\slide{MLSH Chapter 2: intro }

%\hlkimage{}{}

\begin{quote}{\bf
Securing Administrative User Accounts}\\
Managing users is one of the more {\bf challenging} aspects of IT administration. You need to make sure
that users can always {\bf access their stuff} and that they can {\bf perform the required tasks} to do their jobs.
You also need to ensure that users’ stuff is always {\bf secure from unauthorized users} and that users  {\bf can’t perform any tasks that don’t fit their job description}.
\end{quote}
Source: \emph{Mastering Linux Security and Hardening} (MLSH), third edition

Can we spot the:
\begin{list2}
\item Confidentiality, Integrity and Availability requirements
\item Principle of Least Privilege
\end{list2}

\slide{MLSH Chapter 2: contents }

\begin{quote}

The specific topics covered in this chapter are as follows:
\begin{list2}
\item The dangers of logging in as the root user
\item The advantages of using sudo
\item Setting up sudo privileges for full administrative users and for users with only certain dele-
gated privileges
\item Advanced tips and tricks to use sudo
\end{list2}
\end{quote}
Source: \emph{Mastering Linux Security and Hardening} (MLSH), third edition

\slide{The dangers of logging in as the root user}

%\hlkimage{}{}

\begin{quote}
A huge advantage that Unix and Linux operating systems have over Windows is that Unix and Linux do
a much better job of keeping privileged administrative accounts separated from normal user accounts.
Indeed, one reason that older versions of Windows were so susceptible to security issues, such as
drive-by virus infections, was the common practice of setting up user accounts with administrative
privileges, without having the protection of the User Access Control (UAC) that’s in newer versions of
Windows.
\end{quote}
Source: \emph{Mastering Linux Security and Hardening} (MLSH), third edition

\begin{list2}
\item Agreed, but I may be biased
\item Mac OS X made it very simple to run administrative tasks, so you didn't need to run as root
\item Modern Linux user interfaces make similar attempts with \verb+pkexec+, \verb+kdesudo+,  \verb+gksudo+ etc.
\item Windows is getting better, many organisations in DK are removing administrative access to regular users, even KEA
\end{list2}

\slide{The advantages of using sudo}

%\hlkimage{}{}

\begin{quote}
Used properly, the sudo utility can greatly enhance the security of your systems, and it can make an
administrator’s job much easier. With sudo , you can do the following:
\begin{list2}
\item Assign certain users full administrative privileges, while assigning other users only the privi-
leges they need to perform tasks that are directly related to their respective jobs.
\item Allow users to perform administrative tasks by entering their own normal user passwords so
that you don’t have to distribute the root password to everybody and their brother.
\item Make it harder for intruders to break into your systems. If you implement sudo and disable
the root user account, would-be intruders won’t know which account to attack because they
won’t know which one has admin privileges.
\item Create sudo policies that you can deploy across an entire enterprise network, even if that
network has a mix of Unix, BSD, and Linux machines.
\item Improve your auditing capabilities because you’ll be able to see what users are doing with
their admin privileges.
\end{list2}
\end{quote}
Source: \emph{Mastering Linux Security and Hardening} (MLSH), third edition

\slide{Why use Sudo conclusion}

Main thing about sudo is that you do NOT give out the root password to anybody! They will use their own credentials and can be limited to single commands, scripts and even parameters. You could have a single sudoers file for your own organisation, that includes groups of servers, user groups etc.

Sidenote: sudo also has a number of CVEs unfortunately

\slide{Setting up sudo privileges}

%\hlkimage{}{}

Chapter has multiple example of configuration. Mostly we add users to a predefined admin group, but for my personal systems I add named users with privileges - user \emph{hlk}

As an exercise, if you haven't done before -- make sure sudo works on your Kali and Debian. If it is pre-configured, check the settings, how did this happen?

Debian usually does NOT come with sudo installed, so -verb+apt install sudo+

\begin{list2}
\item Lets do a few of the commands from chapter 2, check sudo on your systems
\item Always use the command \verb+visudo+ to edit your sudoers file! Checks syntax, and you avoid locking yourself out!
\item I recommend the book: \emph{Sudo Mastery}, 2nd Edition by Michael W Lucas, \\
    \url{https://www.tiltedwindmillpress.com/product/sudo-mastery-2nd-edition/}
\end{list2}

\slide{MLSH chapter 3}

%\hlkimage{}{}

\begin{quote}
Enforcing strong password criteria
You wouldn’t think that a benign-sounding topic such as strong password criteria would be so contro-
versial, but it is. The conventional wisdom that you’ve undoubtedly heard for your entire computer
career says:

\begin{list2}
\item Make passwords of a certain minimum length.
\item Make passwords that consist of a combination of uppercase letters, lowercase letters, numbers,
and special characters.
\item Ensure that passwords don’t contain any words that are found in the dictionary or that are
based on the users’ own personal data.
\item Force users to change their passwords on a regular basis.
\end{list2}
\end{quote}
Source: \emph{Mastering Linux Security and Hardening} (MLSH), third edition

\slide{Unix systems are mostly single-user part I}

\begin{list2}
\item Fewer and fewer Unix systems, including Linux are multi-user systems today
\item Chapter is still important, but less so for now, use it as a reference later
\item Password policies are relevant for all systems, single or multi user
\item Pro-tip use Ansible or similar to configure all systems
\end{list2}

\slide{Relational Database Management System RDBMS}

\hlkimage{7cm}{RDBMS_structure.png}

\begin{list1}
\item Relational Database Management System RDBMS is a common database architecture
\item Common examples MS-SQL, MySQL/MariaDB, PostgreSQL
\item Picture: By Scifipete - Own work, CC BY-SA 3.0,\\ \url{https://commons.wikimedia.org/w/index.php?curid=11506013}
\item \url{https://en.wikipedia.org/wiki/Relational_database#RDBMS}
\end{list1}

\slide{PostgreSQL security}
\hlkimage{15cm}{postgresql-security.png}

Feature overview security features in PostgreSQL\\
\url{https://www.postgresql.org/about/featurematrix/#security}

\slide{Deadlocks}

\begin{list1}
\item {\bf Definition 7-1} A \emph{deadlock} is a state in which some set of processes block, each waiting for another process in the set to take some action.
\begin{list2}
\item[1.] The resource is not shared (mutual exclusion)
\item[2.] An entity must hold the resource and block, waiting until another resource becomes available (hold and wait)
\item[3.] A resource being help cannot be released (no preemption)
\item[4.] A set of entities must be holding resources such that each entity is waiting for a resource held by another entity in the set (circular wait)
\end{list2}
\item Often found in Relational Database Systems, if two processes want to update two tables, and each one has a write lock on one table, waiting for the write lock on the other
\item See also \link{https://en.wikipedia.org/wiki/Deadlock}
\end{list1}

\slide{Common Discusssion}

Databases - discussion about Relational Database Management System RDBMS Model and NoSQL databases, which ones do you and your company use?



\exercise{ex:mariadb-createdb}
\exercise{ex:github-perms}

\slide{Passwords vælges ikke tilfældigt}

\hlkimage{20cm}{50-most-used-passwords.png}

Source:
\link{https://wpengine.com/unmasked/}



\slide{Evernote password reset}

What happens when security breaks?
\hlkimage{16cm}{evernote-password-reset.png}

Sources:\\
\link{http://evernote.com/corp/news/password_reset.php}

\slide{Twitter password reset}

\hlkimage{13cm}{twitter-250k-users.png}

Sources:\\
\link{http://blog.twitter.com/2013/02/keeping-our-users-secure.html}

\slide{Saving passwords}

\hlkimage{7cm}{password-window.png}

\vskip 5mm
\centerline{Use some kind of Password Safe program}

\slide{January 2013: Github Public passwords?}


\hlkimage{16cm}{github-credentials.png}

 Sources:\\
{\footnotesize\link{https://twitter.com/brianaker/status/294228373377515522}\\
\link{http://www.webmonkey.com/2013/01/users-scramble-as-github-search-exposes-passwords-security-details/}\\
\link{http://www.leakedin.com/}\\
\link{http://www.offensive-security.com/community-projects/google-hacking-database/}
}

\slide{TwoFactor authentication: Example Duosecurity}

\hlkimage{10cm}{duosecurity-overview.png}
Source:  \link{https://www.duo.com/}


\slide{Yubico Yubikey}

\hlkimage{16cm}{yubico-overview.png}
\begin{quote}
A Yubico OTP is unique sequence of characters generated every time the YubiKey button is touched. The Yubico OTP is comprised of a sequence of 32 Modhex characters representing information encrypted with a 128 bit AES-128 key
\end{quote}

\link{http://www.yubico.com/products/yubikey-hardware/} and also \url{https://wiki.debian.org/Smartcards/YubiKey4}

\slide{Storing passwords}

\hlkimage{12cm}{images/passwordsafe-yubico.png}


\begin{list1}
\item Use password managers -- even though they also have security issues the overall improvement is great
\item PasswordSafe \link{https://pwsafe.org/} -- Note: research for yourself which password manager to use!
\item Apple Keychain provides an encrypted storage
\item Browsere, Firefox Master Password, Chrome passwords, ... who do YOU trust
\end{list1}

\slide{Google looks to ditch passwords for good 2013}

\hlkimage{10cm}{yubico-neo-v1-454x284.jpg}

"Google is currently running a pilot that uses a YubiKey cryptographic card developed by Yubico

The YubiKey NEO can be tapped on an NFC-enabled smartphone, which reads an encrypted one-time password emitted from the key fob."

{\footnotesize Source:
\link{http://www.zdnet.com/google-looks-to-ditch-passwords-for-good-with-nfc-based-replacement-7000010073/}
}

\slide{Low tech 2-step verification }

\centerline{\Large Printing code on paper, low level pragmatic }

\hlkimage{6cm}{google-backup-codes.png}

\begin{list1}
\item Login from new devices today often requires two-factor - email sent to user
\item Google 2-factor auth. SMS with backup codes
\item Also read about S/KEY developed at Bellcore {\bf in the late 1980s}\\ \link{http://en.wikipedia.org/wiki/S/KEY}
\end{list1}

\centerline{Conclusion passwords: integrate with authentication, not reinvent}

\slide{Integrate or develop?}

From previous slide:\\
\centerline{Conclusion passwords: integrate with authentication, not reinvent}


\begin{list1}
\item Dont:
\begin{list2}
\item Reinvent the wheel - too many times, unless you can maintain it afterwards
\item Never invent cryptography yourself
\item No copy paste of functionality, harder to maintain in the future
\end{list2}
\item Do:
\begin{list2}
\item Integrate with existing solutions
\item Use existing well-tested code: cryptography, authentication, hashing
\item Centralize security in your code
\item Fine to hide which authentication framework is being used, easy to replace later
\end{list2}
\end{list1}

\slide{Cisco IOS password 2013}

\begin{quote}
Title: Cisco's new password hashing scheme easily cracked\\

Description: In an astonishing decision that has left crytographic
experts scratching their heads, engineer's for Cisco's IOS operating
system chose to switch to a {\bf one-time SHA256 encoding - without salt} -
for storing passwords on the device. This decision leaves password
hashes vulnerable to high-speed cracking - modern graphics cards can
compute over {\bf 2 billion SHA256 hashes in a second - and is actually
considerably less secure than Cisco's previous implementation.} As users
cannot downgrade their version of IOS without a complete reinstall, and
no fix is yet available, security experts are urging users to avoid
upgrades to IOS version 15 at this time.
\end{quote}

Reference: via SANS @RISK newsletter\\
\link{http://arstechnica.com/security/2013/03/cisco-switches-to-weaker-hashing-scheme-passwords-cracked-wide-open/}

\slide{NT hashes}

\begin{list1}
  \item NT LAN manager hash værdier er noget man typisk kan samle op i
  netværk
\item det er en hash værdi af et password som man ikke burde kunne
  bruge til noget - hash algoritmer er envejs
\item opbygningen gør at man kan forsøge brute-force på 7 tegn ad
  gangen!
\item en moderne pc med l0phtcrack kan nemt knække de fleste password
  på få dage!
\item og sikkert 25-30\% indenfor den første dag - hvis der ingen
  politik er omkring kodeord!
\item ved at generere store tabeller, eksempelvis 100GB kan man dække
  mange hashværdier af passwords med almindelige bogstaver, tal og
  tegn - og derved knække passwordshashes på sekunder. Søg efter
  rainbowcrack med google
\end{list1}


\slide{Pass the hash}

Lots of tools in pentesting pass the hash, reuse existing credentials and tokens
\emph{Still Passing the Hash 15 Years Later}\\
\link{http://passing-the-hash.blogspot.dk/2013/04/pth-toolkit-for-kali-interim-status.html}

\begin{quote}
If a domain is built using only modern Windows OSs and COTS products (which know how to operate within these new constraints), and configured correctly with no shortcuts taken, then these protections represent a big step forward.
\end{quote}

Source:\\
{\small\link{http://www.harmj0y.net/blog/penetesting/pass-the-hash-is-dead-long-live-pass-the-hash/}
\link{https://samsclass.info/lulz/pth-8.1.htm}}

\slide{John the ripper}

\begin{quote}
John the Ripper is a fast password cracker, currently available for
many flavors of Unix (11 are officially supported, not counting
different architectures), Windows, DOS, BeOS, and OpenVMS. Its primary
purpose is to detect weak Unix passwords. Besides several crypt(3)
password hash types most commonly found on various Unix flavors,
supported out of the box are Kerberos AFS and Windows NT/2000/XP/2003
LM hashes, plus several more with contributed patches.
\end{quote}

\begin{list1}
\item UNIX passwords kan knækkes med alec Muffets kendte Crack program
  eller eksempelvis John The Ripper \link{http://www.openwall.com/john/}
\end{list1}



\slide{Cracking passwords}

\begin{list2}
\item Hashcat is the world's fastest CPU-based password recovery tool.
\item oclHashcat-plus is a GPGPU-based multi-hash cracker using a brute-force attack (implemented as mask attack), combinator attack, dictionary attack, hybrid attack, mask attack, and rule-based attack.
\item oclHashcat-lite is a GPGPU cracker that is optimized for cracking performance. Therefore, it is limited to only doing single-hash cracking using Markov attack, Brute-Force attack and Mask attack.
\item John the Ripper password cracker old skool men stadig nyttig
\end{list2}

Source:\\
\link{http://hashcat.net/wiki/}\\
\link{http://www.openwall.com/john/}

\slide{Parallella John}

\hlkimage{20cm}{parallella-john.png}

\link{https://twitter.com/solardiz/status/492037995080712192}

Expect specialized hardware to be used by NSA, GCHQ, and perhaps even organised crime

\slide{Stacking Parallella boards}
\hlkimage{10cm}{4BoardStack.jpg}

\link{http://www.parallella.org/power-supply/}

\slide{Encryption key length}

\hlkimage{6cm}{encryption-crack.png}

Source: \link{http://www.mycrypto.net/encryption/encryption_crack.html}

More up to date:\\
In 1998, the EFF built Deep Crack for less than \$250,000\\
\link{https://en.wikipedia.org/wiki/EFF_DES_cracker}\\
FPGA Based UNIX Crypt Hardware Password Cracker - ~100 EUR in 2006\\
\link{http://www.sump.org/projects/password/}








\exercise{ex:pwcrack-101}

\exercise{ex:config-ssh-keys}

\exercise{ex:CIS-benchmarks-teaser}


\slidenext

\end{document}
