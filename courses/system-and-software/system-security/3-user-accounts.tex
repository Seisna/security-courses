\documentclass[Screen16to9,17pt]{foils}
\usepackage{kea-slides}

\externaldocument{system-security-exercises}
\selectlanguage{english}

\begin{document}

\mytitlepage
{3. User Accounts}
{KEA Kompetence Computer Systems Security \the\year}


\slide{Goals for today}

\hlkimage{6cm}{thomas-galler-hZ3uF1-z2Qc-unsplash.jpg}

Todays goals:
\begin{list2}
\item Talk about user accounts in general
\end{list2}

  Photo by Thomas Galler on Unsplash

\slide{Plan for today}

\begin{list1}
\item Subjects
\begin{list2}
\item What are user accounts -- user ID
\item Securing Administrative User Accounts
\item Securing Normal User Accounts
\item Databases: RDBMS, PostgreSQL, Deadlocks
\end{list2}
\item Exercises
\begin{list2}
\item Databases - discussion about Relational Database Management System RDBMS Model and NoSQL
\end{list2}
\end{list1}



\slide{Reading Summary}

MLSH SectionI: Setting up a Secure Linux System
\begin{list1}
\item Chapter 1: Running Linux in a Virtual Environment
\item Chapter 2: Securing Administrative User Accounts
\item Chapter 3: Securing Normal User Accounts
\end{list1}



\slide{Separation of duty ns function}

\begin{quote}
{\bf Separation of duties} (SoD; also known as Segregation of Duties) is the concept of having more than one person required to complete a task. In business the separation by sharing of more than one individual in one single task is an internal control intended to prevent fraud and error.
\end{quote}

Quote from \url{https://en.wikipedia.org/wiki/Separation_of_duties}

\begin{quote}
{\bf Separation of function}. Developers do not develop new programs on production systems because of the potential threat to production data.
\end{quote}
\emph{Computer Security}, Matt Bishop, 2019

Danish: Funktionsadskillelse



\slide{Accuracy vs disclosure}

Lipner five commercial requirements:
\begin{list2}
\item[1.] Users will not write their own programs, but use existing
  production software.
\item[2.] Programmers develop and test applications on a nonproduction system, possibly using contrived data.
\item[3.] Moving applications from development to production requires a special process.
\item[4.] This process must be controlled and audited.
\item[5.] Managers and auditors must have access to system state and system logs
\end{list2}


Available from\\ {\footnotesize\link{https://csrc.nist.gov/CSRC/media/Publications/conference-paper/1982/05/24/proceedings-5th-seminar-dod-computer-security-initiative/documents/1982-5th-seminar-proceedings.pdf}}



\slide{The Biba Model}

Ken Biba (1977) proposed three different integrity access control
policies.

\begin{list2}
\item[1] The Low Water Mark Integrity Policy
\item[2] The Ring Policy
\item[3] Strict Integrity
\item All assume that we associate integrity labels with subjects and
objects, analogous to clearance levels in BLP.
\item Only Strict Integrity had much continuing influence. It is the one
typically referred to as the “Biba Model” or “Biba Integrity.”
\end{list2}

% https://www.cs.utexas.edu/~byoung/cs361/syllabus361.html
% https://www.cs.utexas.edu/~byoung/cs361/lecture21.pdf

Example page 178 mentions that this was implemented in FreeBSD


\slide{Lipners Integrity Matrix Model}

\hlkimage{15cm}{lipner-model-levels.png}

\emph{Non-Discretionary Controls for Commercial Applications}, Steven B. Lipner, IEEE Symposium on Security and Privacy, and Fifth Seminar on the DoD Computer Security Initiative, 1982

\slide{Lipners Integrity Matrix Model}

\hlkimage{20cm}{lipner-1982.png}

\emph{Non-Discretionary Controls for Commercial Applications}, Steven B. Lipner, IEEE Symposium on Security and Privacy, and Fifth Seminar on the DoD Computer Security Initiative, 1982

\slide{Lipners Integrity Matrix Model}

\hlkimage{20cm}{lipner-with-integrity.png}

\emph{Non-Discretionary Controls for Commercial Applications}, Steven B. Lipner, IEEE Symposium on Security and Privacy, and Fifth Seminar on the DoD Computer Security Initiative, 1982



% \slide{One source of truth}
% Maybe later

\slide{Clark-Wilson Integrity Model}

A {\bf well-formed transaction} from one consistent state to another consistent state.

\begin{list2}
\item Constrained Data Items: CDIs are the objects whose
integrity is protected
\item Unconstrained Data Items: UDIs are objects not covered by
the integrity policy
\item Transformation Procedures: TPs are the only procedures
allowed to modify CDIs, or take arbitrary user input and
create new CDIs. Designed to take the system from one valid
state to another.
\item Integrity Verification Procedures: IVPs are procedures
meant to verify maintainance of integrity of CDIs.
\end{list2}

\emph{A Comparison of Commercial and Military Computer Security Policies},
David D. Clark and David R. Wilson, 1987



\slide{Clark-Wilson Integrity Model}

\begin{quote}
The model uses a three-part relationship of subject/program/object (where program is interchangeable with transaction) known as a triple or an access control triple. Within this relationship, subjects do not have direct access to objects. Objects can only be accessed through programs
\end{quote}

\emph{A Comparison of Commercial and Military Computer Security Policies},
David D. Clark and David R. Wilson, 1987


See also
\url{https://en.wikipedia.org/wiki/Clark%E2%80%93Wilson_model}


\slide{Relational Database Management System RDBMS}

\hlkimage{7cm}{RDBMS_structure.png}

\begin{list1}
\item Relational Database Management System RDBMS is a common database architecture
\item Common examples MS-SQL, MySQL/MariaDB, PostgreSQL
\item Picture: By Scifipete - Own work, CC BY-SA 3.0,\\ \url{https://commons.wikimedia.org/w/index.php?curid=11506013}
\item \url{https://en.wikipedia.org/wiki/Relational_database#RDBMS}
\end{list1}

\slide{PostgreSQL security}
\hlkimage{15cm}{postgresql-security.png}

Feature overview security features in PostgreSQL\\
\url{https://www.postgresql.org/about/featurematrix/#security}

\slide{Deadlocks}

\begin{list1}
\item {\bf Definition 7-1} A \emph{deadlock} is a state in which some set of processes block, each waiting for another process in the set to take some action.
\begin{list2}
\item[1.] The resource is not shared (mutual exclusion)
\item[2.] An entity must hold the resource and block, waiting until another resource becomes available (hold and wait)
\item[3.] A resource being help cannot be released (no preemption)
\item[4.] A set of entities must be holding resources such that each entity is waiting for a resource held by another entity in the set (circular wait)
\end{list2}
\item Often found in Relational Database Systems, if two processes want to update two tables, and each one has a write lock on one table, waiting for the write lock on the other
\item See also \link{https://en.wikipedia.org/wiki/Deadlock}
\end{list1}

\slide{Common Discusssion}

Databases - discussion about Relational Database Management System RDBMS Model and NoSQL databases, which ones do you and your company use?



\exercise{ex:mariadb-createdb}
\exercise{ex:github-perms}


\slidenext

\end{document}
