\documentclass[Screen16to9,17pt]{foils}
\usepackage{kea-slides}

\externaldocument{system-security-exercises}
\selectlanguage{english}

\begin{document}

\mytitlepage
{2. Overview of Enterprise Attacks}
{KEA Kompetence Computer Systems Security \the\year}


\slide{Plan for today}

\begin{list1}
\item Subjects
\begin{list2}
\item Get an idea of the MITRE ATT\&CK framework
\item Do a little hacking using Kali and Metasploit
\end{list2}
\item Exercises
\begin{list2}
\item Quick port scan intro with Nmap
\item Run Armitage against Metasploitable, describe attacks\\
that succeeded in relation to MITRE ATT\&CK framework
\end{list2}
\end{list1}



\slide{Reading Summary}

\begin{list1}
\item Read ATT\&CK 101 Blog Post\\
\link{https://medium.com/mitre-attack/att-ck-101-17074d3bc62}
\item and browse MITRE ATT\&CK web site\\ \link{https://attack.mitre.org/}
\end{list1}


\slide{Vulnerabilities - CVE}

\begin{list1}
\item Common Vulnerabilities and Exposures (CVE):
  \begin{list2}
  \item classification
  \item identification
  \end{list2}
\item When discovered each vuln gets a CVE ID
\item CVE maintained by MITRE - not-for-profit
org for research and development in the USA.
\item National Vulnerability Database search for CVE.
\item Sources: \link{http://cve.mitre.org/} og \link{http://nvd.nist.gov}
\item also checkout OWASP Top-10 \link{http://www.owasp.org/}
\end{list1}

\slide{Sample vulnerabilities}

\begin{list1}
\item \small CVE-2000-0884\\
IIS 4.0 and 5.0 allows remote attackers to read documents outside of
the web root, and possibly execute arbitrary commands, via malformed
URLs that contain UNICODE encoded characters, aka the "Web Server
Folder Traversal" vulnerability.

\item \small CVE-2002-1182\\
IIS 5.0 and 5.1 allows remote attackers to cause a denial of service
(crash) via malformed WebDAV requests that cause a large amount of
memory to be assigned.

\item Source:\\
\link{http://cve.mitre.org/ - CVE}
\end{list1}

\centerline{And updates from vendors reference these too! A closed loop}

\slide{CWE Common Weakness Enumeration}

\hlkimage{18cm}{cwe-mitre-org.png}
\link{http://cwe.mitre.org/}

\slide{CWE/SANS Monster mitigations}

\hlkimage{13cm}{cwe-monster-mitigations.png}

Source:
\link{http://cwe.mitre.org/top25/index.html}


\slide{Hacker tools}

\begin{list1}
\item \emph{Improving the Security of Your Site by Breaking Into it}\\ by
Dan Farmer and Wietse Venema in 1993
\item Later in 1995 release the software SATAN\\
\emph{Security Administrator Tool for Analyzing Networks}
\item Caused some commotion, panic and discussions, every script kiddie can hack, the internet will melt down!
\vskip 5mm
\begin{quote}
We realize that SATAN is a two-edged sword -- like
many tools, it can be used for good and for evil
purposes. We also realize that intruders (including
wannabees) have much more capable (read intrusive)
tools than offered with SATAN.
\end{quote}
\end{list1}

\vskip 1cmlabel
Source:
\link{http://www.fish2.com/security/admin-guide-to-cracking.html}


\slide{Use hacker tools!}

\begin{list1}
\item Port scan can reveal holes in your defense
\item Web testing tools can crawl through your site and find problems
\item Pentesting is a verification and proactively finding problems
\item Its not a silverbullet and mostly find known problems in existing systems
\item Combined with honeypots they may allow better security
\end{list1}


\slide{Agreements for testing networks}

\begin{quote}\small
Danish Criminal Code\\
Straffelovens paragraf 263 Stk. 2. Med bøde eller fængsel indtil 1 år og 6 måneder straffes den, der uberettiget skaffer sig adgang til en andens oplysninger eller programmer, der er bestemt til at bruges i et informationssystem.
\end{quote}

Hacking can result in:
\begin{list2}
\item Getting your devices confiscated by the police
\item Paying damages to persons or businesses
\item If older getting a fine and a record -- even jail perhaps
\item Getting a criminal record, making it hard to travel to some countries and working in security
\item Fear of terror has increased the focus -- so dont step over bounds!
\end{list2}

Asking for permission and getting an OK before doing invasive tests, always!

\slide{ISC2 code of ethics}

\hlkimage{23cm}{isc2-code-of-ethics.png}

CISSP certified people sign papers to this extent.\\
\link{https://www.isc2.org/ethics/default.aspx}


\slide{What happens today?}

\begin{list1}
\item Think like a blue team member find vulnerable systems
\item Get basic tools running
\begin{list2}
\item ping sweep, port scan
\item OS detection -- TCP/IP or banner grab
\item Service scan -- rpcinfo, netbios, ...
\item telnet/netcat interact with services
\end{list2}
\item Exploit: Metasploit, Nikto, exploit programs
\item Cleanup/hardening not shown but in practice:
\begin{list2}
\item Make a report
\item Change and improve systems
\item Follow up on critical vulnerabilities
\item Change configuration, architecture etc.
\end{list2}
\item Remember to document process, need to show others what you do
\end{list1}




\slide{Trinity breaking in}

\hlkimage{14cm}{trinity-nmapscreen-hd-cropscale-418x250.jpg}
Very realistic:\\
\link{https://nmap.org/movies/}\\
\link{https://youtu.be/51lGCTgqE_w}


\slide{Kali Linux the pentest toolbox}

\hlkimage{14cm}{kali-linux.png}

\begin{list1}
\item  Kali \link{http://www.kali.org/}
\item 100.000s of videos on youtube alone, searching for kali and \$TOOL
\item Also versions for Raspberry Pi, mobile and other small computers
\end{list1}

\slide{Really do Nmap your world}

\hlkimage{8cm}{nmap-zenmap.png}

\begin{list2}
\item Nmap is a port scanner, but does more
\item Finding your own infrastructure available from the guest network?
\item See your printers having all the protocols enabled AND a wireless?
\end{list2}


\slide{Basal Portscanning}

\begin{list1}
\item Hvad er portscanning
\item Afprøvning af alle porte fra 0/1 og op til 65535
\item Målet er at identificere åbne porte -- sårbare services
\item Typisk TCP og UDP scanning
\item TCP scanning er ofte mere pålidelig end UDP scanning
\item TCP handshake er nemmere at identificere, skal svare SYN
\item UDP applikationer svarer forskelligt -- hvis overhovedet\\
Svarer på rigtige forespørgsler, uden firewall svares ICMP på lukkede porte
\item Brug GUI programmet Zenmap mens i lærer Nmap at kende
\end{list1}


\slide{TCP three-way handshake}

\hlkimage{5cm}{images/tcp-three-way.pdf}

\begin{list2}
\item {\bfseries TCP SYN half-open} scans
\item Tidligere loggede systemer kun når der var etableret en fuld TCP
  forbindelse\\
  -- dette kan/kunne udnyttes til \emph{stealth}-scans
\item Hvis en maskine modtager mange SYN pakker kan dette fylde
  tabellen over connections op -- og derved afholde nye forbindelser
  fra at blive oprette -- {\bfseries SYN-flooding}
\end{list2}


\slide{Nmap port sweep efter webservere}

\begin{alltt}\small
root@cornerstone:~#{\bfseries  nmap -p80,443 172.29.0.0/24}

Starting Nmap 6.47 ( http://nmap.org ) at 2015-02-05 07:31 CET
Nmap scan report for 172.29.0.1
Host is up (0.00016s latency).
PORT    STATE    SERVICE
{\color{darkgreen}80/tcp  open     http}
443/tcp filtered https
MAC Address: 00:50:56:C0:00:08 (VMware)

Nmap scan report for 172.29.0.138
Host is up (0.00012s latency).
PORT    STATE  SERVICE
{\color{darkgreen}80/tcp  open   http}
443/tcp closed https
MAC Address: 00:0C:29:46:22:FB (VMware)

\end{alltt}

\slide{Nmap port sweep efter SNMP port 161/UDP}

\begin{alltt}\small
root@cornerstone:~#{\bfseries nmap -sU -p 161 172.29.0.0/24}
Starting Nmap 6.47 ( http://nmap.org ) at 2015-02-05 07:30 CET
Nmap scan report for 172.29.0.1
Host is up (0.00015s latency).
PORT    STATE         SERVICE
{\color{darkgreen}161/udp open|filtered snmp}
MAC Address: 00:50:56:C0:00:08 (VMware)

Nmap scan report for 172.29.0.138
Host is up (0.00011s latency).
PORT    STATE  SERVICE
{\bf{161/udp closed snmp}}
MAC Address: 00:0C:29:46:22:FB (VMware)
...
Nmap done: 256 IP addresses (5 hosts up) scanned in 2.18 seconds
\end{alltt}

\slide{Nmap Advanced OS detection}
\begin{alltt}\footnotesize
root@cornerstone:~#{\bfseries nmap -A -p80,443 172.29.0.0/24}
Starting Nmap 6.47 ( http://nmap.org ) at 2015-02-05 07:37 CET
Nmap scan report for 172.29.0.1
Host is up (0.00027s latency).
PORT    STATE    SERVICE VERSION
80/tcp  open     http    Apache httpd 2.2.26 ((Unix) DAV/2 mod_ssl/2.2.26 OpenSSL/0.9.8zc)
|_http-title: Site doesn't have a title (text/html).
443/tcp filtered https
MAC Address: 00:50:56:C0:00:08 (VMware)
Device type: media device|general purpose|phone
Running: Apple iOS 6.X|4.X|5.X, Apple Mac OS X 10.7.X|10.9.X|10.8.X
OS details: Apple iOS 6.1.3, Apple Mac OS X 10.7.0 (Lion) - 10.9.2 (Mavericks)
or iOS 4.1 - 7.1 (Darwin 10.0.0 - 14.0.0), Apple Mac OS X 10.8 - 10.8.3 (Mountain Lion)
or iOS 5.1.1 - 6.1.5 (Darwin 12.0.0 - 13.0.0)
OS and Service detection performed.
Please report any incorrect results at http://nmap.org/submit/
\end{alltt}

\begin{list2}
\item Lavniveau måde at identificere operativsystemer på, prøv også
  \verb+nmap -A+
\item Send pakker med \emph{anderledes} indhold, observer svar
\item En tidlig og detaljeret reference: \emph{ICMP Usage In Scanning} Version 3.0,
  Ofir Arkin, 2001 %\link{https://web.archive.org/web/20050210093427/http://www.sys-security.com/html/projects/icmp.html} % Original side er død
\end{list2}


\slide{Heartbleed CVE-2014-0160}

\hlkimage{19cm}{heartbleed-com.png}

Source: \link{http://heartbleed.com/}



\slide{Scan for Heartbleed and SSLv2/SSLv3}

\hlkimage{6cm}{nmap-sslv2.png}

\begin{list1}
\item \verb+nmap -p 443 --script ssl-heartbleed <target>+\\
\link{https://nmap.org/nsedoc/scripts/ssl-heartbleed.html}
\item \verb+masscan 0.0.0.0/0 -p0-65535  --heartbleed+\\
\link{https://github.com/robertdavidgraham/masscan}
\end{list1}

\centerline{Almost every new vulnerability will have Nmap recipe}


\exercise{ex:nmap-pingsweep}
\exercise{ex:nmap-synscan}
\exercise{ex:nmap-os}



\slide{Local vs. remote exploits}

\begin{list1}
\item {\bfseries Local vs. remote}
angiver om et exploit er rettet mod
en sårbarhed lokalt på maskinen, eksempelvis
opnå højere privilegier, eller beregnet
til at udnytter sårbarheder over netværk
\item {\bfseries Remote root exploit}
- den type man frygter mest, idet
det er et exploit program der når det afvikles giver
angriberen fuld kontrol, root user er administrator
på Unix, over netværket.
\item {\bfseries Zero-day exploits} dem som ikke offentliggøres -- dem
  som hackere holder for sig selv. Dag 0 henviser til at ingen kender
  til dem før de offentliggøres og ofte er der umiddelbart ingen
  rettelser til de sårbarheder
\end{list1}



\slide{Metasploit and Armitage Still rocking the internet}


\hlkimage{14cm}{metasploit-about.png}

\begin{list1}

\item \link{http://www.metasploit.com/}
\item Armitage GUI fast and easy hacking for Metasploit\\
\link{http://www.fastandeasyhacking.com/}
\link{http://www.offensive-security.com/metasploit-unleashed/Main_Page}
\item Bog: Metasploit: The Penetration Tester's Guide, No Starch Press\\
ISBN-10: 159327288X - ældre bog, kan undværes
\end{list1}

\slide{Demo: Metasploit Armitage }

\hlkimage{10cm}{armitage-overview.png}



\slide{Get an idea of the MITRE ATT\&CK framework}

\hlkimage{14cm}{mitre-attack.png}

\link{https://attack.mitre.org/}


\exercise{ex:armitage-basic}

\slidenext

\end{document}
