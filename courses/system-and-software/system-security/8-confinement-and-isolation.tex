\documentclass[Screen16to9,17pt]{foils}
\usepackage{kea-slides}

\externaldocument{system-security-exercises}
\selectlanguage{english}

\begin{document}

\mytitlepage
{9. Confinement and isolation}
{KEA Kompetence Computer Systems Security}


\slide{Plan for today}

\begin{list1}
\item Subjects
\begin{list2}
\item Confinement and isolation
\item Virtual Machines and Sandboxes
\item Covert Channels
\item Firewall Flow Controls
\end{list2}
\item Exercises
\begin{list2}
\item Research VM escapes
\item Try running a Docker container
\end{list2}
\end{list1}



\slide{Reading Summary}

\begin{list1}
\item MLSH chapter 9:
\item MLSH chapter 10:
\item Skim: Removing ROP Gadgets from OpenBSD
\item Skim: Capsicum: practical capabilities for UNIX
\end{list1}

\slide{Information Flow Policies}

\begin{list1}
\item Define the way information moves throughout a system
\item Describes the authorized paths along which that information can flow
\begin{list2}
\item Preserve confidentiality
\item Preserve integrity
\end{list2}
\item Can be done using labels representing security class, as in Bell-PaPadula Model
\end{list1}


\slide{Goals for today: Confinement and isolation}

Todays goals:
\begin{list2}
\item Known the problem of keeping applications and data confined
\item Discuss isolation - including sandboxes, virtual machines and capabilities
\item Introduce network isolation, VLAN, firewalls
\end{list2}


\slide{The confinement problem}

\begin{list1}
\item {\bf Definition 18-1} The \emph{confinement problem} is the problem of preventing a server from leaking information that the user of the service considers confidential.
\item
\end{list1}

\slide{A covert channel}

\begin{list1}
\item {\bf Definition 18-2} A \emph{covert channel} is a path of communication that was not designed to be used for communication.
\item
\end{list1}

\slide{The rule of transitive confinement}

\begin{list1}
\item {\bf Definition 18-3} The \emph{rule of transitive confinement} states that if a confined process invokes a second process, the second process must be as confined as the caller.
\item
\end{list1}



\slide{Isolation}

\begin{quote}
A controlled environment is an environment that contains process execution in such a way that it can only interact with other entities in a manner that preserves its isolation.
\end{quote}
Quote from Matt Bishop, Computer Security 2019

\begin{list1}
\item
\item
\end{list1}




\slide{Virtual Machines}

\begin{list1}
\item {\bf Definition 18-4} A \emph{virtual machine} is a program that simulates the hardware of a (possibly abstract) computer system.
\item Also called hypervisor
\item Common technologies include VMware, Virtualbox, HyperV, Qemu, KVM
\item Qubes OS uses the Xen Project h\url{ttps://xenproject.org/}
\item Also similarities to sandboxes implemented in Java Virtual Machine (JVM) and other places
\end{list1}



\slide{Sandbox definition}

\begin{list1}
\item {\bf Definition 18-6} A \emph{sandbox} is an environment in which the actions of a process are restricted according to a security policy
\item Mentions firewall, which is why we also discuss these later today
\end{list1}



\slide{Chroot, Jails and }

\begin{list1}
\item Der findes mange typer \emph{jails} på Unix
\item Ideer fra Unix chroot som ikke er en egentlig sikkerhedsfeature
\begin{list2}
\item Unix chroot - bruges stadig, ofte i daemoner som OpenSSH
\item FreeBSD Jails
\item SELinux
\item Solaris Containers og Zones
\item VMware virtuelle maskiner, er det et jail?
\end{list2}
\item Hertil kommer et antal andre måder at adskille processer - sandkasser
\item Husk også de simple, database som \verb+_postgresql+, Tomcat som \verb+tomcat+, Postfix postsystem som \verb+_postfix+, SSHD som \verb+sshd+ osv. - simple brugere, få rettigheder
\end{list1}

\slide{JVM security policies}

\begin{minted}[fontsize=\small]{java}
// ========== WEB APPLICATION PERMISSIONS =====================================
// These permissions are granted by default to all web applications
// In addition, a web application will be given a read FilePermission
// and JndiPermission for all files and directories in its document root.
grant {
    // Required for JNDI lookup of named JDBC DataSource's and
    // javamail named MimePart DataSource used to send mail
    permission java.util.PropertyPermission "java.home", "read";
    permission java.util.PropertyPermission "java.naming.*", "read";
    permission java.util.PropertyPermission "javax.sql.*", "read";
...
};
// The permission granted to your JDBC driver
// grant codeBase "jar:file:${catalina.home}/webapps/examples/WEB-INF/lib/driver.jar!/-" \{
//      permission java.net.SocketPermission "dbhost.mycompany.com:5432", "connect";
// \};
\end{minted}

\begin{list1}
\item Eksempel fra \verb+apache-tomcat-6.0.18/conf/catalina.policy+
\end{list1}


\slide{Apple sandbox named generic rules}

\begin{minted}[fontsize=\small]{lisp}
;; named - sandbox profile
;; Copyright (c) 2006-2007 Apple Inc.  All Rights reserved.
;;
;; WARNING: The sandbox rules in this file currently constitute
;; Apple System Private Interface and are subject to change at any time and
;; without notice. The contents of this file are also auto-generated and not
;; user editable; it may be overwritten at any time.
;;
(version 1)
(debug deny)

(import "bsd.sb")

(deny default)
(allow process*)
(deny signal)
(allow sysctl-read)
(allow network*)
\end{minted}


\slide{Apple sandbox named specific rules }

\begin{minted}[fontsize=\small]{lisp}
;; Allow named-specific files
(allow file-write* file-read-data file-read-metadata
  (regex "^(/private)?/var/run/named\\pid$"
         "^/Library/Logs/named\\.log$"))

(allow file-read-data file-read-metadata
  (regex "^(/private)?/etc/rndc\\.key$"
         "^(/private)?/etc/resolv\\.conf$"
         "^(/private)?/etc/named\\.conf$"
         "^(/private)?/var/named/"))
\end{minted}

\begin{list1}
\item Eksempel fra \verb+/usr/share/sandbox+ på Mac OS X
\end{list1}


\slide{Capability-based security}

\begin{quote}
  Capability-based security is a concept in the design of secure computing systems, one of the existing security models. A capability (known in some systems as a key) is a communicable, unforgeable token of authority. It refers to a value that references an object along with an associated set of access rights. A user program on a capability-based operating system must use a capability to access an object. Capability-based security refers to the principle of designing user programs such that they directly share capabilities with each other according to the principle of least privilege, and to the operating system infrastructure necessary to make such transactions efficient and secure. Capability-based security is to be contrasted with an approach that uses hierarchical protection domains.
\end{quote}

\url{https://en.wikipedia.org/wiki/Capability-based_security}

\slide{Capsicum in FreeBSD}

\begin{quote}{\bf
Implementation status}\\
Capsicum for FreeBSD was implemented by Robert Watson and Jonathan Anderson. Capsicum first appeared in FreeBSD 9.0 as an experimental feature, compiled out of the kernel by default. As of FreeBSD 10.0, Capsicum capability mode, capabilities, and process descriptors are compiled into the kernel by default, and available for use by both base-system and third-party applications.

Significant KPI and API changes were made in FreeBSD 10.0 following several years' experience deploying Capsicum in experimental applications and the FreeBSD base system. In FreeBSD 10.0, a number of base-system applications use Capsicum "out of the box" including tcpdump, auditdistd, hastd, dhclient, kdump, rwhod, ctld, iscsid, and even uniq.
\end{quote}

Quote from:\\
\url{https://www.cl.cam.ac.uk/research/security/capsicum/freebsd.html}


\slide{Covert Channels}

\begin{list1}
\item {\bf Definition 18-7} A \emph{covert storage channel} uses an attribute of the shared resource. A \emph{covert timing channel} uses a temporal or ordering relationship among access to a shared resource
\item Example the recent Intel CPU attacks meltdown etc.
\item Book describes ways to detect and mitigate these covert channels, but it is hard
\item Chapter 17 - which we haven't read discuss information flow, which can be analyzed
\end{list1}

\exercise{ex:vm-escape}


\exercise{ex:docker-run}


\slide{Design a robust network Isolation and segmentation}

\begin{list1}
\item Hvad kan man gøre for at få bedre netværkssikkerhed?
\begin{list2}
\item Bruge switche - der skal ARP spoofes og bedre performance
\item Opdele med firewall til flere DMZ zoner for at holde
      udsatte servere adskilt fra hinanden, det interne netværk og
      Internet
\item Overvåge, læse logs og reagere på hændelser
\end{list2}
\item Husk du skal også kunne opdatere dine servere
\end{list1}



\slide{Firewall Flow Controls -- \emph{the firewall infrastructure}}

\hlkimage{13cm}{overview-routing-customer-2015.pdf}

\begin{list1}
\item Conclusion: Do as much as possible with your existing devices
\item Tuning and using features like stateless router filters works wonders
\end{list1}


\slide{Firewalls and related issues}

\begin{quote}
In computing, a firewall is a network security system that monitors and controls incoming and outgoing network traffic based on predetermined security rules.[1] A firewall typically establishes a barrier between a trusted internal network and untrusted external network, such as the Internet.[2]
\end{quote} Source: Wikipedia

\begin{list1}
\item \link{https://en.wikipedia.org/wiki/Firewall_(computing)}
\item \link{http://www.wilyhacker.com/} Cheswick chapter 2 PDF
\emph{A Security Review of Protocols:
Lower Layers}
\begin{list2}
\item Network layer, packet filters, application level, stateless, stateful
\end{list2}
\end{list1}

{\Large Firewalls are by design a choke point, natural place \\
to do network security monitoring!}


\slide{Firewall historik}

\hlkimage{3cm}{images/cheswick-cover2e.jpg}

\begin{list1}
\item Firewalls har været kendt siden starten af 90'erne
\item Første bog \emph{Firewalls and Internet Security} udkom i 1994 men kan stadig anbefales, læs den på \link{http://www.wilyhacker.com/}

\item 2003 kom den i anden udgave \emph{Firewalls and Internet Security}
William R. Cheswick, Steven M. Bellovin, Aviel D. Rubin,
Addison-Wesley, 2nd edition
%\item Idag findes mange, men findes der en god generel firewall bog?
\end{list1}


\slide{Book: Applied Network Security Monitoring (ANSM)}

\hlkimage{3cm}{ansm-book.png}

\emph{Applied Network Security Monitoring: Collection, Detection, and Analysis}
1st Edition, Chris Sanders, Jason Smith\\
eBook ISBN: 9780124172166 Paperback ISBN: 9780124172081 496 pp., Syngress, December 2013

{\small\link{https://www.elsevier.com/books/applied-network-security-monitoring/unknown/978-0-12-417208-1}}

\slide{Book: Practical Packet Analysis (PPA)}

\hlkimage{3cm}{PracticalPacketAnalysis3E_cover.png}

\emph{Practical Packet Analysis,
Using Wireshark to Solve Real-World Network Problems}
by Chris Sanders, 3rd Edition
April 2017, 368 pp.
ISBN-13:
978-1-59327-802-1

\link{https://nostarch.com/packetanalysis3}




\slide{Together with Firewalls - VLAN Virtual LAN}

\hlkimage{6cm}{vlan-portbased.pdf}

\begin{list1}
\item Nogle switche tillader at man opdeler portene
\item Denne opdeling kaldes VLAN og portbaseret er det mest simple
\item Port 1-4 er et LAN
\item De resterende er et andet LAN
\item Data skal omkring en firewall eller en router for at krydse fra VLAN1 til VLAN2
\end{list1}


\slide{Basic Network Security Pattern Isolate in VLANs}

\hlkimage{10cm}{vlan-8021q.pdf}

\begin{list1}
\item Du bør opdele dit netværk i segmenter efter trafik, IEEE 802.1q VLANs er mulighed
\item Data skal omkring en firewall eller en router for at krydse fra VLAN1 til VLAN2
\item Du bør altid holde interne og eksterne systemer adskilt!
\item Du bør isolere farlige services i jails og chroots
\item Brug port security til at sikre basale services DHCP, Spanning Tree osv.
\end{list1}



\slide{Generic IP Firewalls}

\vskip 4 cm
\centerline{\hlkbig En firewall er noget som {\color{security6blue}blokerer}
  traffik på Internet}

\vskip 1 cm
\pause

\centerline{\hlkbig En firewall er noget som {\color{red}tillader}
  traffik på Internet}

\slide{Defense in depth}

%\hlkimage{10cm}{Bartizan.png}
\hlkimage{15cm}{medieval-clipart-5}
\centerline{Picture originally from: \url{http://karenswhimsy.com/public-domain-images}}

\slide{Defense in depth - layered security}

\hlkimage{6cm}{security-layers-1-uk.pdf}

\centerline{\hlkbig\color{security6blue} Multiple layers of security! Isolation!}


\slide{Small networks}

\hlkimage{18cm}{images/kursus-netvaerk.pdf}


\slide{Firewall er ikke alene}

\hlkimage{15cm}{network-layers-1.png}

\centerline{\hlkbig Forsvaret er som altid - flere lag af sikkerhed! }

\slide{Unified communications}

\hlkimage{19cm}{firma-netvaerk-wlan}


\slide{Firewallrollen idag}

\begin{list1}
\item Idag skal en firewall være med til at:
\begin{list2}
\item Forhindre angribere i at komme ind
\item Forhindre angribere i at sende traffik ud
\item Forhindre virus og orme i at sprede sig i netværk
\item Indgå i en samlet løsning med ISP, routere, firewalls, switchede
  strukturer, intrusion detectionsystemer samt andre dele af infrastrukturen
\end{list2}
\item Det kræver overblik!
\end{list1}



\slide{Modern Firewalls}

\begin{list1}
\item Basalt set et netværksfilter - det yderste fæstningsværk
\item Indeholder typisk:
  \begin{list2}
   \item Grafisk brugergrænseflade til konfiguration - er det en
   fordel?
\item TCP/IP filtermuligheder - pakkernes afsender, modtager, retning
  ind/ud, porte, protokol, ...
\item både IPv4 og IPv6
\item foruddefinerede regler/eksempler - er det godt hvis det er nemt
  at tilføje/åbne en usikker protokol?
\item typisk NAT funktionalitet indbygget
\item typisk mulighed for nogle serverfunktioner: kan agere
  DHCP-server, DNS caching server og lignende
  \end{list2}
\item En router med Access Control Lists - kaldes ofte netværksfilter,
  mens en dedikeret maskine kaldes firewall
%  funktionen er reelt den samme - der filtreres traffik
\end{list1}


\slide{Sample rules from OpenBSD PF}

\begin{alltt}\tiny
# hosts and networks
router="217.157.20.129"
webserver="217.157.20.131"
homenet="{ 192.168.1.0/24, 1.2.3.4/24 }"
wlan="10.0.42.0/24"
wireless=wi0
set skip lo0
# things not used
spoofed="{ 127.0.0.0/8, 172.16.0.0/12, 10.0.0.0/16, 255.255.255.255/32 }"
{\bf
block in all # default block anything}
# egress and ingress filtering - disallow spoofing, and drop spoofed
block in quick from $spoofed to any
block out quick from any to $spoofed

pass in on $wireless proto tcp from \{ $wlan $homenet \} to any port = 22
pass in on $wireless proto tcp from any to $webserver port = 80

pass out
\end{alltt}




\slide{Packet filtering}

\begin{alltt}\footnotesize
0                   1                   2                   3
0 1 2 3 4 5 6 7 8 9 0 1 2 3 4 5 6 7 8 9 0 1 2 3 4 5 6 7 8 9 0 1
+-+-+-+-+-+-+-+-+-+-+-+-+-+-+-+-+-+-+-+-+-+-+-+-+-+-+-+-+-+-+-+-+
|Version|  IHL  |Type of Service|          Total Length         |
+-+-+-+-+-+-+-+-+-+-+-+-+-+-+-+-+-+-+-+-+-+-+-+-+-+-+-+-+-+-+-+-+
|         Identification        |Flags|      Fragment Offset    |
+-+-+-+-+-+-+-+-+-+-+-+-+-+-+-+-+-+-+-+-+-+-+-+-+-+-+-+-+-+-+-+-+
|  Time to Live |    Protocol   |         Header Checksum       |
+-+-+-+-+-+-+-+-+-+-+-+-+-+-+-+-+-+-+-+-+-+-+-+-+-+-+-+-+-+-+-+-+
|                       Source Address                          |
+-+-+-+-+-+-+-+-+-+-+-+-+-+-+-+-+-+-+-+-+-+-+-+-+-+-+-+-+-+-+-+-+
|                    Destination Address                        |
+-+-+-+-+-+-+-+-+-+-+-+-+-+-+-+-+-+-+-+-+-+-+-+-+-+-+-+-+-+-+-+-+
|                    Options                    |    Padding    |
+-+-+-+-+-+-+-+-+-+-+-+-+-+-+-+-+-+-+-+-+-+-+-+-+-+-+-+-+-+-+-+-+
\end{alltt}

\begin{list1}
\item Packet filtering er firewalls der filtrerer på IP niveau
\item Idag inkluderer de fleste stateful inspection
\end{list1}

\slide{Kommercielle firewalls}
\begin{list2}
\item Checkpoint Firewall-1 \link{http://www.checkpoint.com}
\item Cisco ASA \link{http://www.cisco.com}
\item Clavister firewalls \link{http://www.clavister.com}
\item Juniper SRX \link{http://www.juniper.net}
\item Palo Alto \link{https://www.paloaltonetworks.com/}
\item Fortinet \link{https://www.fortinet.com/}
\end{list2}

Ovenstående er dem som jeg oftest ser ude hos mine kunder i Danmark

Der findes også mange gode open source baserede firewalls


\slide{Hardware eller software}


\begin{list1}
\item Man hører indimellem begrebet \emph{hardware firewall}
\item Det er dog et faktum at en firewall består af:
\begin{list2}
\item Netværkskort - som er hardware
\item Filtreringssoftware - som er \emph{software}!
\end{list2}
\item Det giver ikke mening at kalde en ASA 5501 en hardware firewall\\
  og en APU2C4 med OpenBSD for en software firewall!
\item Man kan til gengæld godt argumentere for at en dedikeret
  firewall som en separat enhed kan give bedre sikkerhed
  \item Det er også fint at tale om host-firewalls, altså at servere og laptops har firewall slået til
\end{list1}







\slide{firewall koncepter}

\begin{list1}
\item Rækkefølgen af regler betyder noget!
\begin{list2}
\item To typer af firewalls:
 First match - når en regel matcher, gør det som angives block/pass
 Last match  - marker pakken hvis den matcher, til sidst afgøres block/pass
\end{list2}
\item {\bf Det er ekstremt vigtigt at vide hvilken type firewall
    man bruger!}
\item OpenBSD PF er last match
\item FreeBSD IPFW er first match
\item Linux iptables/netfilter er last match
\end{list1}

\slide{First or Last match firewall?}

\hlkimage{18cm}{images/first-last-match-1.pdf}



\slide{First match - IPFW}

\begin{alltt}
\hlksmall
00100 16389  1551541 allow ip from any to any via lo0
00200     0        0 deny log ip from any to 127.0.0.0/8
00300     0        0 check-state
...
{\bfseries
65435    36     5697 deny log ip from any to any}
65535   865    54964 allow ip from any to any
\end{alltt}

\vskip 2 cm

\centerline{Den sidste regel nås aldrig!}

\slide{Last match - OpenBSD PF}

\begin{alltt}\small
ext_if="ext0"
int_if="int0"
{\bf
block in}
pass out keep state

pass quick on \{ lo $int_if \}

# Tillad forbindelser ind på port 80=http og port 53=domain
# på IP-adressen for eksterne netkort ($ext_if) syntaksen
pass in on $ext_if proto tcp to ($ext_if) port http keep state
pass in on $ext_if proto \{ tcp, udp \} to ($ext_if) port domain keep state
\end{alltt}


Pakkerne markeres med block eller pass indtil sidste regel\\
nøgleordet \emph{quick} afslutter match - god til store regelsæt



\slide{Firewalls og ICMP}

\begin{alltt}
ipfw add allow icmp from any to any icmptypes 3,4,11,12
\end{alltt}

\begin{list1}
\item Ovenstående er IPFW syntaks for at tillade de interessant ICMP beskeder igennem
\item Tillad ICMP types:
\begin{list2}
\item 3 Destination Unreachable
\item 4 Source Quench Message
\item 11 Time Exceeded
\item 12 Parameter Problem Message
\end{list2}

\item Der er helt tilsvarende nogle  ICMPv6 beskeder som bør tillades i firewalls
\end{list1}

\slide{Firewall konfiguration}

\begin{list1}
\item Den bedste firewall konfiguration starter med:
\begin{list2}
\item Papir og blyant
\item En fornuftig adressestruktur
\end{list2}
\item Brug dernæst en firewall med GUI første gang!
\item Husk dernæst:
\begin{list2}
\item En firewall skal passes
\item En firewall skal opdateres
\item Systemerne bagved skal hærdes!
\end{list2}
\end{list1}




\slide{Bloker indefra og ud}

\begin{list1}
\item Der er porte og services som altid bør blokeres
\item Det kan være kendte sårbare services
\begin{list2}
\item Windows SMB filesharing - ikke til brug på Internet!
\item UNIX NFS - ikke til brug på Internet!
\item Windows RDP
\end{list2}
\item Kendte problemer som efter min mening altid skal blokeres
\end{list1}


\slide{Anti-pattern dobbelt NAT i eget netværk}

\hlkimage{16cm}{nat-double.pdf}

\begin{list1}
\item Det er nødvendigt med NAT for at oversætte traffik der sendes videre
ud på internet.
\vskip 1cm
\item Der er ingen som helst grund til at benytte NAT indenfor eget netværk!
\end{list1}


\slide{IPsec og Andre VPN features}

\begin{list1}
\item De fleste firewalls giver mulighed for at lave krypterede
  tunneler
\item Nyttigt til fjernkontorer der skal have usikker traffik henover
  usikre netværk som Internet
\item Konceptet kaldes Virtual Private Network VPN
\item IPsec er de facto standarden for VPN og beskrevet i RFC'er
\end{list1}


\slide{Linux TCP SACK PANIC  - CVE-2019-11477 et al}

Kernel vulnerabilities, CVE-2019-11477, CVE-2019-11478 and CVE-2019-11479

\begin{quote}\footnotesize{\bf
Executive Summary}\\
Three related flaws were found in the Linux kernel’s handling of TCP networking.  The most severe vulnerability could allow a remote attacker to trigger a kernel panic in systems running the affected software and, as a result, impact the system’s availability.

The issues have been assigned multiple CVEs: CVE-2019-11477 is considered an Important severity, whereas CVE-2019-11478 and CVE-2019-11479 are considered a Moderate severity.

The first two are related to the Selective Acknowledgement (SACK) packets combined with Maximum Segment Size (MSS), the third solely with the Maximum Segment Size (MSS).

These issues are corrected either through applying mitigations or kernel patches.  Mitigation details and links to RHSA advsories can be found on the RESOLVE tab of this article.
\end{quote}

Source: {\footnotesize\url{https://access.redhat.com/security/vulnerabilities/tcpsack}}


\slide{2019 ERNW White Paper 68 -- Cisco ACI}

%\hlkimage{}{}

\begin{quote}\small
  SECURITY ASSESSMENT OF CISCO ACI\\
  Software-defined networking (SDN) along-side with micro-segmentation has been proposed as a new paradigm to deploy applications faster and, simultaneously, protect the individual workloads against lateral movement. One of the major solutions in this area is the Application Centric Infrastructure (ACI) by Cisco. As part of our research endeavors, we performed a security assessment of Cisco ACI. In this whitepaper, we will provide an introduction to Cisco ACI, present the results of our assessment, and give recommendations on how to operate Cisco ACI more securely.
\end{quote}{\footnotesize
Source: \link{https://ernw.de/en/whitepapers/issue-68.html}}

\begin{list2}\footnotesize
\item Remote Code Execution on Leaf Switches over IPv6 via Local SSH Server (CVE-2019-1836, CVE2019-1803, and CVE-2019-1804)
\item Cisco Nexus 9000 Series Fabric Switches ACI Mode Fabric Infrastructure VLAN Unauthorized
  Access Vulnerability (CVE-2019-1890)
\item Cisco Nexus 9000 Series Fabric Switches Application Centric Infrastructure Mode Link Layer
  Discovery Protocol Buffer Overflow Vulnerability (CVE-2019-1901)
\item Cisco Application Policy Infrastructure Controller REST API Privilege Escalation Vulnerability (CVE2019-1889)
\end{list2}


\slidenext

\end{document}
