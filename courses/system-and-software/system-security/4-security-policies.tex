\documentclass[Screen16to9,17pt]{foils}
\usepackage{kea-slides}

\externaldocument{system-security-exercises}
\selectlanguage{english}

\begin{document}

\mytitlepage
{4. Security Policies}
{KEA Kompetence Computer Systems Security \the\year}

\slide{Goals for part I}

\hlkimage{6cm}{thomas-galler-hZ3uF1-z2Qc-unsplash.jpg}

 
\begin{list2}
\item Policies in all forms
\item Discuss the need
\item See examples
\end{list2}

  Photo by Thomas Galler on Unsplash

\slide{Plan for part I}

\begin{list1}
\item Subjects
\begin{list2}
\item Security policies what are they
%\item Discretionary Access Control (DAC)
%\item Mandatory Access Control (MAC)
  \item Example Acceptable Use Policies
  \item Example Academic Computer Security Policy
  \item Confidentiality Policies Bell-LaPadula Model
\end{list2}
\item Exercises
\begin{list2}
\item A look at SELinux an example Mandatory Access Control system
\item Find your AUP for the ISPs we use, you use, your company uses
\end{list2}
\end{list1}

\slide{Reading Summary}

\begin{list1}
\item DSH chapter 3: Policies
\item DSH chapter 4: Standards and Procedures
\item DSH chapter 5: User Education
\item MLSH chapter 4: Securing Your Server with a
Firewall – Part 1 - NOT firewalld part!

\item Browse: Campus Network Security: High Level Overview , Network Startup Resource Center\\
Campus Operations Best Current Practice, Network Startup Resource Center\\
Mutually Agreed Norms for Routing Security (MANRS)
\end{list1}



\slide{Reading Summary}

MLSH SectionI: Setting up a Secure Linux System
\begin{list1}
\item DSH chapter 3: Policies
\item DSH chapter 4: Standards and Procedures
\item DSH chapter 5: User Education
\item MLSH chapter 4: Securing Your Server with a
Firewall – Part 1 - NOT firewalld part!

\item Browse: Campus Network Security: High Level Overview , Network Startup Resource Center\\
Campus Operations Best Current Practice, Network Startup Resource Center\\
Mutually Agreed Norms for Routing Security (MANRS)
\end{list1}


\slide{Security policy}

\begin{quote}
A security policy defines \emph{secure} for a system or a set of systems.\\
Matt Bishop, Computer Security 2019
\end{quote}

\begin{list1}
\item Secure states
\item Transitions between states, what is allowed
\item Breach of security - system enters an unauthorized state
\item Is it possible to return from insecure to a secure state?
\item Book also defines Confidentiality, Integrity and Availability more precisely
\item \emph{Origin integrity} authentication
\item Military security policy (coinfidentiality) vs commercial security policy (integrity)
\end{list1}

\slide{Assumptions}

\begin{quote}
Any security policy, mechanism, or procedure is based on assumptions that, if incorrect, destroy the superstructure on which it is built.\\
Matt Bishop, Computer Security 2019
\end{quote}

\begin{list1}
\item Example, vendor patches
\item Important points:
\begin{list2}
\item Is patch correct? Example Spectre and heartbleed
\item Vendor test environments equal to intended environments
\item Installed correctly - including operator skills
\end{list2}
\end{list1}



\slide{Confidentiality Policies Bell-LaPadula Model}

\begin{quote}
The BellLa--Padula Model (BLP) is a state machine model used for enforcing access control in government and military applications.[1] It was developed by David Elliott Bell [2] and Leonard J. LaPadula, subsequent to strong guidance from Roger R. Schell, to formalize the U.S. Department of Defense (DoD) multilevel security (MLS) policy.[3][4][5] The model is a formal state transition model of computer security policy that describes a set of access control rules which use security labels on objects and clearances for subjects. Security labels range from the most sensitive (e.g., "Top Secret"), down to the least sensitive (e.g., "Unclassified" or "Public").
\end{quote}

Quote from:
\url{https://en.wikipedia.org/wiki/Bell%E2%80%93LaPadula_model}

Note: models like this have often been used in CISSP certification.

See also Orange Book reference:\\ \link{https://en.wikipedia.org/wiki/Trusted_Computer_System_Evaluation_Criteria}


\slide{Types of Access Control}

\begin{quote}
{\bf Definition 4-13.} If an individual user can set an access control mechanism to allow or deny access to an object, that mechanism is a \emph{discretionary access control (DAC)}, also called an \emph{identity-based access control (IBAC)}

{\bf Definition 4-14.}  When a system mechanism controls access to an object and an individual user cannot alter that access, the control is a \emph{mandatory access control (MAC)}, occasionally cale a \emph{rule-based access control}
\end{quote}

Quote from Matt Bishop, Computer Security 2019

\slide{Examples from real life systems}

Example systems implementing DAC/MAC:
\begin{list2}
\item Unix file permissions - DAC
\item SELinux - Mandatory Access Control architecture to the Linux Kernel
\item Sun's Trusted Solaris uses a mandatory and system-enforced access control mechanism
\end{list2}

See also:
\url{https://en.wikipedia.org/wiki/Discretionary_access_control}\\
\url{https://en.wikipedia.org/wiki/Mandatory_access_control}

\slide{Role-based Access Control (RBAC)}

\begin{quote}
In computer systems security, {\bf role-based access control (RBAC)}[1][2] or role-based security[3] is an approach to restricting system access to unauthorized users. It is used by the majority of enterprises with more than 500 employees,[4] and can implement mandatory access control (MAC) or discretionary access control (DAC).

Role-based access control (RBAC) is a policy-neutral access-control mechanism defined around {\bf roles and privileges}. The components of RBAC such as role-permissions, user-role and role-role relationships make it simple to perform user assignments. A study by NIST has demonstrated that RBAC addresses many needs of commercial and government organizations[citation needed]. RBAC can be used to facilitate administration of security in large organizations with hundreds of users and thousands of permissions. Although RBAC is different from MAC and DAC access control frameworks, it can enforce these policies without any complication.
\end{quote}
Quote from \url{https://en.wikipedia.org/wiki/Role-based_access_control}

\exercise{ex:github-perms}



\slide{DSH chapter 3: Policies}

%\hlkimage{}{}

\begin{quote}
Policies are one of the less glamorous areas of information security. They are, however, very useful and can be used to form the cornerstone of security improvement work in your organization. In this chapter we will discuss why writing policies is a good idea, what they should contain, and the choice of language to use.
\begin{list2}
\item Consistency
\item Distribution of knowledge
\item Setting expectations
\item Regulatory compliance and audit
\item Sets the tone
\item Management endorsement
\end{list2}
\end{quote}
Source: \emph{Defensive Security Handbook}, Lee Brotherston, Amanda Berlin ISBN: 978-1-491-96038-7



\slide{Template Policies}

%\hlkimage{}{}

\begin{quote}
For ease of reading, updating, and overall management it is probably easier to produce a set of policy documents rather than a single monolithic document.
SANS, for example, publishes a list of template policies that you can edit for your own needs. At the time of writing, its list of topics are: {\small
Acceptable Encryption Policy,
Acceptable Use Policy,
Clean Desk Policy,
Disaster Recovery Plan Policy,
Digital Signature Acceptance Policy,
Email Policy,
Ethics Policy,
Pandemic Response Planning Policy,
Password Construction Guidelines,
Password Protection Policy,
Security Response Plan Policy,
End User Encryption Key Protection Policy,
Acquisition Assessment Policy,
Bluetooth Baseline Requirements Policy,
Remote Access Policy,
Remote Access Tools Policy,
Router and Switch Security Policy,
Wireless Communication Policy,
Wireless Communication Standard,
Database Credentials Policy,
Technology Equipment Disposal Policy,
Information Logging Standard,
Lab Security Policy,
Server Security Policy,
Software Installation Policy,
Workstation Security (For HIPAA) Policy,
Web Application Security Policy}

{\scriptsize\url{https://learning.oreilly.com/library/view/defensive-security-handbook/9781491960370/ch03.html#:-:text=SANS%2C%20for%20example,Application%20Security%20Policy}}
\end{quote}
Source: \emph{Defensive Security Handbook}, Lee Brotherston, Amanda Berlin ISBN: 978-1-491-96038-7


\slide{Storage and Communication}

%\hlkimage{}{}

\begin{quote}
The nature of policies and procedures is meant to lend as much standard communication as possible to the organization as a whole. To do this, policies must be easily accessible. There are {\bf many software packages} that can not only provide a {\bf web interface for policies}, but also have {\bf built-in review, revision control, and approval processes}. Software with these features makes it much easier when there are a {\bf multitude of people and departments creating, editing, and approving policies}.

Another good rule of thumb is to, at least {\bf once per reviewal process, have two copies of all policies printed out}. As the majority of them will be used in digital format, there will be many policies that refer to and are in direct relation to {\bf downtime or disaster recovery procedures}. In cases such as these, they may not be accessible via digital media so having a backup in physical form is best.
\end{quote}
Source: \emph{Defensive Security Handbook}, Lee Brotherston, Amanda Berlin ISBN: 978-1-491-96038-7

\begin{list2}
\item I highly recommend having policies online, and NOT in a word processor document. It may be that you produce a combined/longer document \emph{from} the online system, but edit in some Wiki or similar
\end{list2}


\slide{Password policies}

Let's take a look at passwords.

\begin{quote}
Enforcing strong password criteria
You wouldn’t think that a benign-sounding topic such as strong password criteria would be so contro-
versial, but it is. The conventional wisdom that you’ve undoubtedly heard for your entire computer
career says:
\begin{list2}
\item Make passwords of a certain minimum length.
\item Make passwords that consist of a combination of uppercase letters, lowercase letters, numbers,
and special characters.
\item Ensure that passwords don’t contain any words that are found in the dictionary or that are
based on the users’ own personal data.
\item \sout{Force users to change their passwords on a regular basis.}
\end{list2}
\end{quote}
Source: \emph{Mastering Linux Security and Hardening} (MLSH), third edition


\slide{Cornerstones of Authentication}

%\hlkimage{}{}

\begin{quote}
The classic paradigm for authentication systems identifies three factors as the cornerstones of
authentication:
\begin{list2}
\item Something you know (e.g., a password).
\item Something you have (e.g., an ID badge or a cryptographic key).
\item Something you are (e.g., a fingerprint or other biometric data).
\end{list2}
MFA refers to the use of more than one of the above factors. The strength of authentication
systems is largely determined by the number of factors incorporated by the system — the more
factors employed, the more robust the authentication system.
\end{quote}
Source: NIST Special Publication 800-63-3 \emph{Digital Identity Guidelines} 2021


\slide{Password expiration no longer recommended}

%\hlkimage{}{}

\begin{quote}
{\bf Q-B05:} Is password expiration no longer recommended?\\
{\bf A-B05:} SP 800-63B Section 5.1.1.2 paragraph 9 states:

“Verifiers SHOULD NOT require memorized secrets to be changed arbitrarily (e.g., periodically). However, verifiers SHALL force a change if there is evidence of compromise of the authenticator.”

{\footnotesize Users tend to choose weaker memorized secrets when they know that they will have to change them in the near future. When those changes do occur, they often select a secret that is similar to their old memorized secret by applying a set of common transformations such as increasing a number in the password. This practice provides a false sense of security if any of the previous secrets has been compromised since attackers can apply these same common transformations. But if there is evidence that the memorized secret has been compromised, such as by a breach of the verifier’s hashed password database or observed fraudulent activity, subscribers should be required to change their memorized secrets. However, this event-based change should occur rarely, so that they are less motivated to choose a weak secret with the knowledge that it will only be used for a limited period of time.}
\end{quote}
Source: \url{https://pages.nist.gov/800-63-FAQ/#q-b05}

\begin{list2}
\item Since 2021 this new version of the \emph{NIST Special Publication 800-63: Digital Identity Guidelines} has begun to become agreed upon
\end{list2}

\slide{Multi-Factor Authentication Recommended}

%\hlkimage{}{}

\begin{quote}
Upon completion of the authentication process, the verifier generates an assertion containing the
result of the authentication and provides it to the RP.
\begin{list2}
\item Security Assertion Markup Language (SAML) assertions are specified using a mark-up
language intended for describing security assertions. They can be used by a verifier to
make a statement to an RP about the identity of a claimant. SAML assertions may
optionally be digitally signed.
\item OpenID Connect claims are specified using JavaScript Object Notation (JSON) for
describing security, and optionally, user claims. JSON user info claims may optionally be
digitally signed.
\item Kerberos tickets allow a ticket-granting authority to issue session keys to two
authenticated parties using symmetric key based encapsulation schemes.
\end{list2}
\end{quote}
Source: NIST Special Publication 800-63-3 \emph{Digital Identity Guidelines} 2021


\slide{DSH chapter 4: Standards and Procedures}

%\hlkimage{}{}

\begin{quote}
Standards and procedures are two sets of documentation that support the policies and bring them to life. In this chapter we will learn what standards and procedures are, how they relate to policies, and what they should contain.

If we consider the policies of an organization to be the “what” we are trying to achieve, standards and procedures form the “how.” As with policies, standards and procedures bring with them many advantages:
\begin{list2}
\item Consistency
\item Distribution of knowledge
\item Setting expectations
\item Regulatory compliance
\item Management endorsement
\end{list2}
\end{quote}
Source: NIST Special Publication 800-63-3 \emph{Digital Identity Guidelines} 2021

Note: and then some standards are not optional, financial, GDPR, NIS etc. does have some requirements


\slide{Information Security Management System (ISMS)}


%\hlkimage{}{}

\begin{quote}\small
An information security management system (ISMS) represents the collation of all the interrelated/interacting information security elements of an organization so as to ensure policies, procedures, and objectives can be created, implemented, communicated, and evaluated to better guarantee the organization's overall information security.
\end{quote}
Source: \url{https://en.wikipedia.org/wiki/Information_security_management}

\slide{From Mission to Instructions}

From a high level we can say that we have multiple documents, about security:
\begin{list2}
\item[] Mission Statements with overall ideas
\item[] Strategy -- strategic considerations
\item[] Policies -- generic policy documents for the whole organisation
\item[] Standards -- how do we want to implement
\item[] Procedures -- how do we actually configure systems, select settings, algorithms etc.
\item[] Detailed instructions -- example step-by-step how to configure a firewall rule
\end{list2}



\slide{DSH chapter 5: User Education}

%\hlkimage{}{}

\begin{quote}
User education and {\bf security awareness} as a whole is {\bf broken} in its current state. It is best to find a way to demonstrate with the right type of metrics that you are successfully implementing change and producing a more secure line of defense.
\begin{list2}
\item Broken Processes -- Ebbinghaus forgetting curve
\item Bridging the Gap -- Repetition is a proven, successful way to bridge the gap of compliance, teaching our users real-life skills
\end{list2}
\end{quote}
Source: \emph{Defensive Security Handbook}, Lee Brotherston, Amanda Berlin ISBN: 978-1-491-96038-7


\slide{Building Your Own Program}

%\hlkimage{}{}

\begin{quote}
Building a mature and strategic program from the ground up is achievable with executive support and cultural alignment. ...  At the \sout{end of this chapter} \emph{Appendix A. User Education Templates} you will find a template slideshow for a security awareness program.

\begin{list2}
\item Establish Objectives
\item Establish Baselines
\item Scope and Create Program Rules and Guidelines
\item Implement and Document Program Infrastructure
\item {\bf Positive} Reinforcement
\item Gamification
\item {\bf Define Incident Response Processes}
\end{list2}
\end{quote}
Source: \emph{Defensive Security Handbook}, Lee Brotherston, Amanda Berlin ISBN: 978-1-491-96038-7





\slide{Linux Aide}

\begin{list1}
\item Many sources mention Tripwire, an alternative is Aide
\item Advanced Intrusion Detection Environment
\item open source host based file and directory integrity checker
\item detects changes to files on the local system
\item Short example available from:\\
{\footnotesize\link{https://blog.rapid7.com/2017/06/30/how-to-install-and-configure-aide-on-ubuntu-linux/}}
\item \link{https://en.wikipedia.org/wiki/Advanced_Intrusion_Detection_Environment}
\end{list1}

How can this be applied as a policy, and what does it detect?


\slide{Example Academic Computer Security Policies}

\hlkimage{4cm}{old_book_lumen_design_st_01.png}

Lets discuss other policies

Campus Network Security: High Level Overview , Network Startup Resource Center

Campus Operations Best Current Practice, Network Startup Resource Center

Mutually Agreed Norms for Routing Security (MANRS)

\link{https://informationssikkerhed.ku.dk/}
\exercise{ex:example-AUP}


\exercise{ex:mlsh-pwquality}



\slide{MLSH chapter 4: Securing Your Server with a Firewall}

%\hlkimage{}{}

As a large example today, we are going to talk about firewall policies, since they play a huge part in keeping the systems secure.
\begin{quote}
Security is one of those things that’s best done in layers. {\bf Security-in-depth}, we call it. So, on any given
corporate network, you will find a {\bf firewall appliance separating the Internet from the demilitarized
zone (DMZ)}, where your Internet-facing servers are kept. You will also find a firewall appliance between
the DMZ and the internal LAN, and {\bf firewall software installed on each individual server and client}. We
want to make it as tough as possible for intruders to reach their final destinations within our networks.
\end{quote}
Source: \emph{Mastering Linux Security and Hardening} (MLSH), third edition


\begin{list2}
\item Best practice is to turn on firewall software on all systems
\item Require developers and implementers to specify stricter rules for access, no \emph{permit any any} anymore
\end{list2}


\slide{Firewalls are Hardening Your Systems}

%\hlkimage{}{}

\begin{quote}
Since the focus of this book is on hardening our Linux servers, we’ll focus this chapter on that last
level of defense: the firewalls on our servers and clients. We’ll cover both of the command-line netfilter
interfaces, which are {\bf iptables} and {\bf nftables}.

\begin{list2}
\item iptables: This replaced ipchains in Linux kernel version 2.6. It’s still used in a lot of Linux
distros but is rapidly disappearing.
\item nftables: This is the new kid on the block and is rapidly replacing iptables. As we’ll see later,
it has a lot of advantages over the older iptables.
\end{list2}
\end{quote}
Source: \emph{Mastering Linux Security and Hardening} (MLSH), third edition


\begin{list2}
\item Lets move over to the book now. We already did some Nmap, so we can play with firewall on and off
\end{list2}


\slide{Security Enhanced Linux}

\hlkimage{6cm}{315px-SELinux.png}

\begin{quote}
Security-Enhanced Linux (SELinux) is a Linux kernel security module that provides a mechanism for supporting access control security policies, including mandatory access controls (MAC).
\end{quote}
From:
\link{https://en.wikipedia.org/wiki/Security-Enhanced_Linux}

\exercise{ex:se-linux-intro}


\slidenext

\end{document}
