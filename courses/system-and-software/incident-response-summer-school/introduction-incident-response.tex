\documentclass[Screen16to9,17pt]{foils}
\usepackage{zencurity-slides}
\externaldocument{introduction-incident-response-exercises}
\selectlanguage{english}

% KEA Summer School in Cyber Security 2022
% Intro to Incident Response - Track B
%13:00-16:00
% Introduction to incident response. what is an incident and a log? We will discuss what happens when someone visits your network. Starting from initial compromise we will demonstrate how we can identify, process and handle incidents in networks. Using example tools from the Elastic stack we will present deep dives into network data, DNS, introduce computer forensics and how preparation and a structured method will enable companies to handle incidents more effeciently.

% From email to Birger 2022-05-16


\begin{document}

\mytitlepage
{Intro to Incident Response - Track B}
{Summer School in Cyber Security 2022}

\hlkprofiluk

\slide{Goals for today}

\hlkimage{6cm}{thomas-galler-hZ3uF1-z2Qc-unsplash.jpg}

\begin{quote}
Introduction to incident response. what is an incident and a log? We will discuss what happens when someone visits your network. Starting from initial compromise we will demonstrate how we can identify, process and handle incidents in networks. Using example tools from the Elastic stack we will present deep dives into network data, DNS, introduce computer forensics and how preparation and a structured method will enable companies to handle incidents more effeciently.
\end{quote}

Photo by Thomas Galler on Unsplash

\slide{Plan for today}

\begin{list1}
\item 13:00 - 13:45 Introduction and basics
\item 13:45 - 14:00 -- 15 min do exercises / break
\item 14:00 - 14:45
\item 14:45 - 14:00 -- 15 min do exercises / break
\item 15:00 - 15:30
\item 15:30 - 15:45 -- Last exercise
\item 15:45 - 16:00 -- Summary and conclusions
\end{list1}

All slides are in english, exercises in Danish!

{\bf Hint: Don't panic if the plan breaks!}

\slide{Introduction: Attack overview}

\hlkimage{20cm}{sicherheitstacho.png}
Source: \link{http://www.sicherheitstacho.eu/}


\slide{DDoS Attacks Still a Problem}

\hlkimage{13cm}{DDos_Attack_in_2021_arbor.png}

Security attacks and DDoS is very much in the media\\
Source: link{https://www.netscout.com/threatreport/global-ddos-attack-trends/}

\slide{Ransomware Attacks are Common}


\hlkimage{13cm}{ransomware_arbor.png}

Make sure to backup your data! Test your backups!\\
Source: link{https://www.netscout.com/threatreport/global-ddos-attack-trends/}

\slide{What can we do? -- Good security}

\hlkimage{12cm}{god-sikkerhed.pdf}

\begin{list1}
\item You always have limited resources for protection - use them as best as possible
\item Good security comes from structured work
\end{list1}


\slide{Balanced security}

\hlkimage{21cm}{afbalanceret-sikkerhed.pdf}

\begin{list1}
\item Better to have the same level of security
\item If you have bad security in some part - guess where attackers will end up
\item Hackers are not required to take the hardest path into the network
\item Realize there is no such thing as 100\% security
\end{list1}



\slide{Work together}

\hlkimage{10cm}{Shaking-hands_web.jpg}

\begin{list1}
\item Team up!
\item We need to share security information freely
\item We often face the same threats, so we can work on solving these together
\end{list1}

\slide{My daily job -- Security engineering a job role}

\begin{alltt}\footnotesize
On any given day, you may be challenged to:
        Create new ways to solve existing production security issues
        Configure and install firewalls and intrusion detection systems
        Perform vulnerability testing, risk analyses and security assessments
        Develop automation scripts to handle and track incidents
        Investigate intrusion incidents, conduct forensic investigations and incident responses
        Collaborate with colleagues on authentication, authorization and encryption solutions
        Evaluate new technologies and processes that enhance security capabilities
        Test security solutions using industry standard analysis criteria
        Deliver technical reports and formal papers on test findings
        Respond to information security issues during each stage of a project’s lifecycle
        Supervise changes in software, hardware, facilities, telecommunications and user needs
        Define, implement and maintain corporate security policies
        Analyze and advise on new security technologies and program conformance
        Recommend modifications in legal, technical and regulatory areas that affect IT security
\end{alltt}

Source: \url{https://www.cyberdegrees.org/jobs/security-engineer/}\\
also
\url{https://en.wikipedia.org/wiki/Security_engineering}




\slide{Risk management defined}

\hlkimage{20cm}{shon-harris-risk-management.png}

Source: Shon Harris \emph{CISSP All-in-One Exam Guide}



\slide{Security Controls and Frameworks}

\begin{list1}
\item Multiple exist
\vskip 1cm
\begin{list2}
\item CIS controls Center for Internet Security (CIS) \link{https://www.cisecurity.org}
\item PCI Best Practices for Maintaining PCI DSS Compliance v2.0 Jan 2019
\item NIST Cybersecurity Framework (CSF)\\
Framework for Improving
Critical Infrastructure Cybersecurity\\ \link{https://www.nist.gov/cyberframework}\\
\link{https://csrc.nist.gov/publications/sp800} - SP800 series
\item National Security Agency (NSA)\\
\link{https://www.nsa.gov/Research/}
\item NSA security configuration guides\\
\link{https://apps.nsa.gov/iaarchive/library/ia-guidance/security-configuration/}
\item Information Systems Audit and Control Association (ISACA)\\
\link{http://www.isaca.org/Knowledge-Center/}
\end{list2}
\end{list1}

\slide{Center for Internet Security CIS Controls}

\begin{quote}
  “A goal without a plan is just a wish.”\\
  ― Antoine de Saint-Exupéry
\end{quote}

\begin{quote}
  The CIS ControlsTM are a prioritized set of actions that collectively form a defense-in-depth set
of best practices that mitigate the most common attacks against systems and networks. The
CIS Controls are developed by a community of IT experts who apply their first-hand experience
as cyber defenders to create these globally accepted security best practices. The experts who
develop the CIS Controls come from a wide range of sectors including retail, manufacturing,
healthcare, education, government, defense, and others.
\end{quote}

Source: \link{https://www.cisecurity.org/} CIS-Controls-Version-7-1.pdf

\slide{Kom igang med CIS}

\begin{quote}
CIS-kontrollerne består af 20 praktiske, pragmatiske kontroller, som er målbare og med direkte henvisning til, hvordan de implementeres samt forslag til, hvilke KPI’er der bør opstilles for målinger.

Forskellen på CIS-kontrollerne og fx ISO27001 er, at du ikke kan blive certificeret efter CIS, men til gengæld opdateres CIS-kontrollerne løbende, og de indeholder prioriterede lister af, hvad du i praksis skal gøre for din cybersikkerhed. Det australske forsvar har fx vist, at hvis man implementerer de første fire kontroller fuldt ud, kan man mitigere op mod 90+\% af alt malware.
\end{quote}

Dansk artikel fra Deloitte, version 7 men version 8 er ude
\link{https://www2.deloitte.com/dk/da/pages/risk/articles/vi-stiller-skarpt-pa-cis-kontroller.html}


\slide{Overview Diploma in IT-security}

\hlkimage{14cm}{kea-diplom-oversigt.png}

Some of this comes from courses in Diploma in IT-Security / professionsbachelor i IT-Sikkerhed (Pba ITS)
\slide{Course Data}

\hlkimage{12cm}{pawel-janiak-dxFi8Ea670E-unsplash.jpg}


Specifically the courses\\
{\Large\bf Course: SIEM and Log Analysis (5 ECTS)}


\slide{Course Description}

From: STUDIEORDNING Diplomuddannelse i it-sikkerhed August 2018\\
VF4 SIEM og log analyse (5 ECTS)

Indhold\\
Den studerende lærer om Security information and event management (SIEM), herunder
hvordan man kan indsamle, administrere, og søge i sikkerhedshændelsesdata i et større it
system (komplekse systemer, IOT deployments, corporate IT).

{\bf Læringsmål:}
Viden -- Den studerende har viden om og forståelse for:
\begin{list2}
\item Typiske SIEM arkitekturer
\item Standard logformater og logtyper for standard systemer og komponenter
\item Typiske SIEM produkter
\item Juridiske krav til logning og bevarelse af data ifb. forensic analyse
\end{list2}

{\bf Færdigheder -- Den studerende kan:}
\begin{list2}
\item Lave en baseline-analyse af en infrastruktur
\item Bruge log-data til at identificere infrastrukturkomponenter
\item Bruge et værktøj til at analysere system log-data og netværkstrafik til at finde sikkerhedshændelser
\item Udvikle "dashboards" og alarmer der viser tegn på hændelser
\end{list2}

{\bf Kompetencer -- Den studerende kan:}
\begin{list2}
\item Designe og implementere en SIEM løsning på tværs af diverse produkter
\item Træffe beslutninger om hvilke data der skal indsamles i en givne situation
\item Identificerer fejl i logopsamlingen
\item Deltage i drøftelser på et praktisk og strategisk niveau i forhold til implementering af
logmanagement/SIEM
\end{list2}

Final word is the Studieordning which can be downloaded from\\
{\footnotesize \link{https://kompetence.kea.dk/uddannelser/it-digitalt/diplom-i-it-sikkerhed}\\
\link{Studieordning_for_Diplomuddannelsen_i_IT-sikkerhed_Aug_2018.pdf}}


\slide{Some keywords relating to this course}

%\hlkimage{}{}

\begin{quote}\large
    Incident Response Logs Events

Analysis Visualization Dashboards Data-driven Security

SIEM architectures frameworks acquire process Zeek

log formats data types databases JSON XML Security Operations Center

(Incident Response) Intelligence R and Python fundamentals

Practical application Building Infosec \hskip 1cm Ansible Playbooks

Collect, mine, organize, and analyze relevant data sources

Sort data create reporting and monitoring

IP-address Netflow nfdump Elasticsearch real-world knowledge
\end{quote}

\begin{list2}
\item Lots of new terms, technologies and tools
\end{list2}


\slide{Primary literature}

\hlkrightpic{5cm}{0cm}{old_book_lumen_design_st_01.png}
Primary literature -- used in courses:
\begin{list2}
\item \emph{Data-Driven Security: Analysis, Visualization and Dashboards} Jay Jacobs, Bob Rudis\\
ISBN: 978-1-118-79372-5 February 2014 \url{https://datadrivensecurity.info/} - short DDS
\item \emph{Crafting the InfoSec Playbook: Security Monitoring and Incident Response Master Plan}\\
 by Jeff Bollinger, Brandon Enright, and Matthew Valites ISBN: 9781491949405 - short CIP
\item \emph{Intelligence-Driven Incident Response} \\
 Scott Roberts ISBN: 9781491934944 - short IDI
\item \emph{Security Operations Center: Building, Operating, and Maintaining your SOC}\\
ISBN: 9780134052014 Joseph Muniz - short SOC
\end{list2}


\slide{Data-Driven Security: Analysis, Visualization and Dashboards}

\hlkimage{6cm}{book-data-driven-security.jpg}
\emph{Data-Driven Security: Analysis, Visualization and Dashboards} Jay Jacobs, Bob Rudis\\
ISBN: 978-1-118-79372-5 February 2014 \url{https://datadrivensecurity.info/} - short DDS

Our main book for this course. We will read a lot from this one.

\slide{Crafting the InfoSec Playbook}

\hlkimage{6cm}{book-crafting-infosec-playbook.jpg}

\emph{Crafting the InfoSec Playbook: Security Monitoring and Incident Response Master Plan}\\
 by Jeff Bollinger, Brandon Enright, and Matthew Valites ISBN: 9781491949405 - short CIP


\slide{Intelligence-Driven Incident Response}

\hlkimage{6cm}{book-intelligence-driven-incident-response.jpg}

\emph{Intelligence-Driven Incident Response} \\
 Scott Roberts ISBN: 9781491934944 - short IDI



\slide{Security Operations Center}

\hlkimage{6cm}{ book-security-operations-center-cisco.jpg}

\emph{Security Operations Center: Building, Operating, and Maintaining your SOC}\\
ISBN: 9780134052014 Joseph Muniz - short SOC




\slide{What is Infrastructure}


\hlkimage{10cm}{alexander-schimmeck-SeeM4AnkEHE-unsplash.jpg}

\begin{list2}
\item Enterprises today have a lot of computing systems supporting the business needs
\item These are very diverse and often discrete systems
\end{list2}

\hfill Photo by Alexander Schimmeck on Unsplash


\slide{Business Challenges}

\hlkimage{7cm}{adam-bignell-9tI2z5VZIZg-unsplash.jpg}

\begin{list2}
\item Accumulation of software
\item Legacy systems
\item Partners
\item Various types of data
\item Employee churn, replacement \hfill Photo by Adam Bignell on Unsplash
\end{list2}



\slide{Software Challenges}

\hlkimage{7cm}{john-barkiple-l090uFWoPaI-unsplash.jpg}

\begin{list2}
\item Complexity
\item Various languages
\item Various programming paradigms, client server, monolith, Model View Controller
\item Conflicting data types and available structures
\item Steam train vs electric train \hfill Photo by John Barkiple on Unsplash

\end{list2}


\slide{SIEM}

%\hlkimage{}{}

\begin{quote}
{\bf Security information and event management (SIEM)} is a subsection within the field of computer security, where software products and services combine security information management (SIM) and security event management (SEM). They provide real-time analysis of security alerts generated by applications and network hardware.

  Vendors sell SIEM as software, as appliances, or as managed services; these products are also used to log security data and generate reports for compliance purposes.[1]

  The term and the initialism SIEM was coined by Mark Nicolett and Amrit Williams of Gartner in 2005.[2]
\end{quote}
Source: \link{https://en.wikipedia.org/wiki/Security_information_and_event_management}

\begin{list2}
  \item Note: there are alerting examples towards the bottom of the page, with sources
  \item Closely related to log management, incident response
\end{list2}


\slide{SOC}

%\hlkimage{}{}

\begin{quote}
An information security operations center (ISOC or SOC) is a facility where enterprise information systems (web sites, applications, databases, data centers and servers, networks, desktops and other endpoints) are monitored, assessed, and defended.

...

A {\bf security operations center (SOC)} can also be called a security defense center (SDC), security analytics center (SAC), network security operations center (NSOC),[3] security intelligence center, cyber security center, threat defense center, security intelligence and operations center (SIOC). In the Canadian Federal Government the term, infrastructure protection center (IPC), is used to describe a SOC.
\end{quote}
Source: \link{https://en.wikipedia.org/wiki/Information_security_operations_center}

\begin{list2}
  \item We have a whole book about SOCs, but I skipped the introductory chapters!
  \item If you need to build a SOC, that is great source of information
\end{list2}




\slide{Subjects: Incident Response}

\hlkimage{5cm}{homer-end-is-near.jpg}

\begin{list1}
\item Context, what are the threats, what are the answers we want from the SIEM and Logs

\item What are the common cases, where we use the data?

\begin{list2}
\item Incident Response
\item Computer Emergency Response Team (CERT) and Computer Security Incident Response Teams (CSIRT)
\item Security Departments
\item GDPR Data protection
\item Computer Forensics
\end{list2}

\end{list1}


\slide{Incident Handling, phases}

The procedures developed for incident response must cover the complete life-cycle

\begin{list2}
\item  Preparation for an attack, establish procedures and mechanisms for detecting and responding to attacks
\item  Identification of an attack, notice the attack is ongoing
\item  Containment (confinement) of the attack, limit effects of the attack as much as possible
\item  Eradication of the attack, stop attacker, block further similar attacks
\item  Recovery from the attack, restore system to a secure state
\item  Follow-up to the attack, include lessons learned  improve environment
\end{list2}


\slide{Crafting the InfoSec Playbook}


This book will help you to answer common questions:
\begin{list2}
\item How do I find bad actors on my network?
\item How do I find persistent attackers?
\item How can I deal with the pervasive malware threat?
\item How do I detect system compromises?
\item How do I find an owner or responsible parties for systems under my protection?
\item How can I practically use and develop threat intelligence?
\item How can I possibly manage all my log data from all my systems?
\item How will I benefit from increased logging—and not drown in all the noise?
\item How can I use metadata for detection?
\end{list2}
Source: \emph{Crafting the InfoSec Playbook: Security Monitoring and Incident Response Master Plan}\\
 by Jeff Bollinger, Brandon Enright, and Matthew Valites ISBN: 9781491949405


\slide{MITRE ATT\&CK framework}

\hlkimage{14cm}{mitre-attack.png}

Source: \link{https://attack.mitre.org/} Great resource for attack categorization

\slide{Incident Response Checklists}
\hlkimage{9cm}{incident-handling-checklist.png}

This checklist is from the NIST document
\emph{Computer Security Incident Handling Guide: Recommendations of the National Institute
of Standards and Technology}, NIST Special Publication 800-61
Revision 2, August 2012.

\slide{CIS Controls also recommend Incident Response}

\begin{quote}{\bf
CIS Control 19:}\\
Incident Response and Management Protect the organization’s information, as well as its reputation, by developing and implementing an incident response infrastructure (e.g., plans, defined roles, training, communications, management oversight) for quickly discovering an attack and then effectively containing the damage, eradicating the attacker’s presence, and restoring the integrity of the network and systems.
\end{quote}

Source:
Center for Internet Security CIS Controls 7.1 CIS-Controls-Version-7-1.pdf
from \link{https://www.cisecurity.org/controls/}



\slide{Anatomy of an Auditing System}


Sample logs from login with Secure Shell (SSH) and performing the command \verb+sudo su -+
\begin{alltt}
Jun  5 11:53:15 pumba sshd[64505]: Accepted publickey for hlk from 79.142.233.18 port 43902
 ssh2: ED25519 SHA256:l8OJMcywyBcraJiCWJ06uZ2yzHfu0VuiArqVvlVyfEI

Jun  5 11:53:19 pumba sudo:      hlk : TTY=ttyp2 ; PWD=/home/hlk ; USER=root ; COMMAND=/usr/bin/su -
\end{alltt}

\begin{list1}
\item Example systems: Unix syslog, IBM main frame RACF and Windows Event Logs service
\item Logs should be protected and considered confidential information
\end{list1}



\slide{Anatomy of an Auditing System}

When data has been gathered it should be analyzed.

\begin{itemize}
\item {\bf Logger functions} - collect
\item {\bf Analyzer} - analyze it, creating dashboard can provide some insights
\item {\bf Notifier} - report results by email or other means
\item Example systems Windows Event Logs service can inform of successful and failed logins, both should be collected
\item Logs should be protected and considered confidential information, by sending it to a centralized system with a high security level protects it
\end{itemize}

Modern systems exist to take all data from logging and provide high capacity storage, searching and sorting.

\slide{Why Elasticsearch}

\hlkimage{8cm}{illustrated-screenshot-hero-siem-500x730.png}
Screenshot from \url{https://www.elastic.co/siem}

Recommend building a proof-of-concept infrastructure using the Elastic stack and gather experience with logging. This can be done without a license fee and the organization can then see what works and doesn't. Then using the experiences as input an informed decision can be made, to continue with this as a home grown logging and auditing solution, or buy a premade one.


\slide{Technologies used in this course}

The following tools and environments are examples that may be introduced in this course:

\begin{list2}
\item Programming languages and frameworks Java, Python, regular expressions
\item Development environments -- choose your own IDE / Editor -- I use {\bf Atom}
\item Networking and network protocols: TCP/IP, HTTP, DNS, Netflow
\item Formats XML, JSON, CSV, raw text, web scraping
\item Web technologies and services: REST, API, HTML5, CSS, JavaScript
\item Tools like cURL, Zeek, Git and Github
\item Message queueing systems: MQ and Redis could be added
\item Aggregated example platforms: Elastic stack, logstash, elasticsearch, kibana, grafana, Filebeat
\item Cloud and virtualisation Docker, Kubernetes, Azure, AWS, microservices -- can be added
\end{list2}

\centerline{This list is not complete or a promise }


\slide{OSI and Internet}

\hlkimage{10cm,angle=90}{images/compare-osi-ip.pdf}

\centerline{Data on all layers}

\slide{Networking in TCP/IP}

\hlkimage{10cm}{arp-basic.pdf}

\begin{list2}
\item Everything uses TCP/IP today, more or less.
\item Clients make requests, receives responses
\item HyperText Transfer Protocol (HTTP) is an example
\item All devices shown can produce logs and events
\end{list2}


\slide{Sources: Network overview}

\hlkimage{15cm}{sample-ip-network.pdf}

\begin{quote}

\end{quote}

\begin{list2}
\item Internet, routers, firewalls, switches, clients and servers (Wi-Fi not shown)
\end{list2}


\slide{Sources: Strategy for implementing identification and detection}

We recommend that the following strategy is used for implementing identification and detection -- logging:
\begin{enumerate}
\item[\faSquareO] Enable system logging from servers
\item[\faSquareO] Enable system logging from network devices
\item[\faSquareO] Enable logging from client devices
\item[\faSquareO] Centralize logging
\item[\faSquareO] Add search facilities and dashboards
\item[\faSquareO] Perform system audits manually or automatically
\item[\faSquareO] Setup alerting and notification with procedures
\end{enumerate}




\slide{Intrusion Kill Chains}

\hlkimage{13cm}{crafting-cip-kill-chain.png}

\begin{list2}
\item See also \emph{Intelligence-Driven Computer Network Defense Informed by Analysis of Adversary Campaigns and Intrusion Kill Chains}, Eric M. Hutchins , Michael J. Cloppert, Rohan M. Amin, Ph.D. Lockheed Martin Corporation\\{\footnotesize
 \link{https://www.lockheedmartin.com/content/dam/lockheed-martin/rms/documents/cyber/LM-White-Paper-Intel-Driven-Defense.pdf}}
\end{list2}



\slide{Detection Capabilities}


Security incidents happen, but what happens. One of the actions to reduce impact of incidents are done in preparing for incidents.

\begin{itemize}
\item \emph{Preparation} for an attack, establish procedures and mechanisms for detecting and responding to attacks
\end{itemize}

Preparation will enable easy {\bf identification} of affected systems, better {\bf containment} which systems are likely to be infected, {\bf eradication} what happened -- how to do the {\bf eradication} and {\bf recovery}.

\slide{Data Analysis Skills}

\begin{quote}
Although we could spend an entire book creating an exhaustive list of skills needed to be a good security data scientist, this chapter covers the following skills/domains that a data scientist will benefit from
knowing within information security:
\begin{list2}
\item Domain expertise—Setting and maintaining a purpose to the analysis
\item Data management—Being able to prepare, store, and maintain data
\item Programming—The glue that connects data to analysis
\item Statistics—To learn from the data
\item Visualization—Communicating the results effectively
\end{list2}
It might be easy to label any one of these skills as the most important, but in reality, the whole is greater than the sum of its parts. Each of these contributes a significant and important piece to the workings of
security data science.
\end{quote}

Source: \emph{Data-Driven Security: Analysis, Visualization and Dashboards} Jay Jacobs, Bob Rudis\\
ISBN: 978-1-118-79372-5 February 2014 \url{https://datadrivensecurity.info/} - short DDS



\slide{Don't use spreadsheets!}

%\hlkimage{}{}

\begin{quote}

\end{quote}

\begin{list2}
\item Spreadsheets are great for some tasks, but ...
\item They don't scale
\item The model can be broken -- edit a single formula
\item Rounding errors accumulate
\item Input and output are limited
\item Most functions require manual work
\end{list2}





\slide{Data overview JSON}

\begin{quote}
JavaScript Object Notation (JSON, pronounced /ˈdʒeɪsən/; also /ˈdʒeɪˌsɒn/[note 1]) is an open-standard file format or data interchange format that uses {\bf human-readable text} to transmit data objects consisting of attribute–value pairs and array data types (or any other serializable value). It is a very common data format, with a diverse range of applications, such as serving as replacement for XML in AJAX systems.[6]
\end{quote}
Source: \url{https://en.wikipedia.org/wiki/JSON}

\begin{list2}
\item I like JSON much better than XML
\item Many web services can supply data in JSON format
\end{list2}

\slide{JSON example}

\begin{minted}[fontsize=\footnotesize]{json}
{
  "first name": "John",
  "last name": "Smith",
  "age": 25,
  "address": {
    "street address": "21 2nd Street",
    "city": "New York",
    "state": "NY",
    "postal code": "10021"
  },
  "phone numbers": [
    {
      "type": "home",
      "number": "212 555-1234"
    },
  ],
}
\end{minted}

\begin{list2}
\item This is a basic JSON document, new data attribute-value pairs can be added\\
Source: \url{https://en.wikipedia.org/wiki/JSON}
\end{list2}




\slide{Python and REST}

\inputminted{python}{programs/rest-1.py}

\begin{list2}
\item  Lets try to use some Python to access a REST service.
\item  We will use the JSONPlaceholder which is a free online REST API:
\link{https://jsonplaceholder.typicode.com/}
\item Start at the site: \link{https://jsonplaceholder.typicode.com/guide.html} and try running a few of the examples with your browser
\item Then try using the same URLS in the Requests HTTP library from Python,\\
\link{https://requests.readthedocs.io/en/master/}
\end{list2}


\slide{Note about frameworks and libraries}

\begin{minted}[fontsize=\footnotesize]{python}
import xml.etree.ElementTree as ET
tree = ET.parse('testfile.xml')
root = tree.getroot()

print(root.tag)
print('Nmap version: \t\t{:s} '.format(root.attrib['version']))
print('Nmap started: \t\t{:s} '.format(root.attrib['startstr']))
print('Nmap command line: \t{:s} '.format(root.attrib['args']))

hosts = tree.findall('./host')
for host in hosts:
    print(host.tag)
    print(host.attrib)
    for hostvalues in host:
        print(hostvalues.tag)
        print(hostvalues.attrib)
\end{minted}

\begin{list2}
\item Dont import JSON or XML using home made programs
\item Example uses xml.etree.ElementTree from Python \url{https://docs.python.org/3.7/library/xml.etree.elementtree.html}
\end{list2}

\slide{Convert XML to JSON}

\begin{minted}[fontsize=\footnotesize]{python}
import xml.etree.ElementTree as ET
import json
def etree_to_dict(t):
    d = {t.tag : map(etree_to_dict, t.getchildren())}
    d.update(('@' + k, v) for k, v in t.attrib.items())
    d['text'] = t.text
    return d

tree = ET.parse('testfile.xml')
root = tree.getroot()
mydict = etree_to_dict(root)
print(type(tree))
print(type(root))
print(type(mydict))

print(mydict)

with open('testfile.json', 'w') as json_file:
  json.dump(mydict, json_file)
\end{minted}

Converting using Python is easy



\slide{Side note: Zeek Security Monitor handles formats differently}

Zeek has files formatted with a header:
\begin{alltt}\footnotesize
#fields ts      uid     id.orig_h       id.orig_p       id.resp_h       id.resp_p       proto   trans_id
        rtt     query   qclass  qclass_name     qtype   qtype_name      rcode   rcode_name      AA
        TC      RD      RA      Z       answers TTLs    rejected

1538982372.416180	CD12Dc1SpQm42QW4G3	10.xxx.0.145	57476	10.x.y.141	53	udp	20383
	0.045021	www.dr.dk	1	C_INTERNET	1	A	0	NOERROR	F	F	T	T	0
   www.dr.dk-v1.edgekey.net,e16198.b.akamaiedge.net,2.17.212.93	60.000000,20409.000000,20.000000	F
\end{alltt}

Note: this show ALL the fields captured and dissected by Zeek, there is a nice utility program bro-cut which can select specific fields:

\begin{alltt}\small
root@NMS-VM:/var/spool/bro/bro# cat dns.log | bro-cut -d ts query answers | grep dr.dk
2018-10-08T09:06:12+0200	www.dr.dk	www.dr.dk-v1.edgekey.net,e16198.b.akamaiedge.net,2.17.212.93
\end{alltt}

Can also just use JSON now via Filebeat


\slide{Metadata -- enrichment}

\hlkimage{10cm}{crafting-security-playbook-metadata.png}

Source: picture from Crafting the InfoSec Playbook, CIP

Metadata + Context


\slide{Example plot 6-17 }

\hlkimage{10cm}{jay-service-discovery.png}
Source: DDS 6. Visualizing Security Data

\begin{list2}
\item Interesting graph, and interesting results Changing away from standard ports delay attackers!
\end{list2}


\slide{Applied Security Visualization examples}


\hlkimage{13cm}{applied-security-visualization-firewall.png}

Source: Firewall Report in \emph{Applied security visualization}, Rafael Marty, 2009

\slide{Applied Security Visualization examples}


\hlkimage{10cm}{applied-security-visualization-flow.png}

Source: Network Flow Data in \emph{Applied security visualization}, Rafael Marty, 2009



\slide{Drill down process}

%\hlkimage{}{}

\begin{quote}

\end{quote}

\begin{enumerate}
\item Get an overview
\item Research top talkers,
\item When identified and handled, remove with filter \verb+not host 10.1.2.3+
\item Look at the next ones
\end{enumerate}

Look into details, lookup hostnames -- hopefully your tool allows some help



\slide{Elasticsearch example systems}

%\hlkimage{}{}

\begin{quote}\small
ElasticSearch consumes practically anything you give it and provides straightforward ways to ask it questions and get data out of it. You just need to feed it semi- or unstructured data and fold in some domain intelligence to enable smart indexing. It works its multi-node NoSQL magic in conjunction with a layer of full-text searching to give you almost instantaneous query results even for large amounts of data.\\
Source: DDS 8. Breaking Up with Your Relational Database
\end{quote}

\begin{list2}
\item Elasticsearch SIEM -- from Elastic, including Elastic Common Schema (ECS)\\
  \link{https://github.com/elastic/ecs}
\item Wazuh -- agent for clients, log events, integrity protection etc.
\item HELK -- all-in one hunting system
\item ElastiFlow -- netflow system
\item Arkime (renamed recently from Moloch) -- packet capture
\end{list2}

Lots of commercial systems, and lots of companies providing cloud logging platform

\slide{Architecture}

\hlkimage{17cm}{elastic-stack-buffered.png}
\begin{list2}
\item Real production environments often add some buffering in between
\item Allows the ingestion to become more smooth, no lost messages
\end{list2}


\slide{ElastiFlow}

\hlkimage{10cm}{elastiflow.png}

\begin{quote}
  ElastiFlow™ provides network flow data collection and visualization using the Elastic Stack (Elasticsearch, Logstash and Kibana). It supports Netflow v5/v9, sFlow and IPFIX flow types (1.x versions support only Netflow v5/v9).
\end{quote}
Source: Picture and text from \link{https://github.com/robcowart/elastiflow}



\slide{Packetbeat}

\hlkimage{10cm}{demo-siem-setup-packetbeat.pdf}


\begin{list2}
\item By installing packetbeat and doing network mirroring from the network switch, we can gather a lot of information
\item Packetbeat supports Elastic Common Schema (ECS) \link{https://www.elastic.co/beats/packetbeat}
\item ICMP (v4 and v6)
DHCP (v4)
DNS
HTTP
AMQP 0.9.1
Cassandra
Mysql
PostgreSQL
Redis
Thrift-RPC
MongoDB
Memcache
NFS
TLS
SIP/SDP (beta)
\end{list2}




\slide{Attack Lifecycle}

\hlkimage{8cm}{socbook-targeted-attack.png}

\begin{quote}
Many breaches discovered are caused by a system having a known vulnerability exploited before it is properly patched.
\end{quote}

Source: Mandiant’s Targeted Attack Lifecycle, SOC chapter 7. vuln management

\begin{list2}
\item Chapter contains lots of references
\item Also chapter links inventory controls with active discovery tools and mitigating
\item Mentions Threat Feeds, which should be integrated into SIEM and/or organizations
\item You need to stay up-to-date on current threats, and be able to search for signs in your own network
\end{list2}





% Summary and conclusion


\slide{Conklusion: chaos and and panic}

\hlkimage{5cm}{dont-panic.png}
\begin{list2}
\item We started out fine, structured approach!
\item We got interrupted ... which happens a lot
\item We didn't finish! Again this is very common in real life
\item Microsoft alone release patches and updates for more than 100 vulnerabilities a month
\item All software has security problems, and need updates!
\end{list2}


\slide{Incident Log and financial}

Take a piece of paper or a computer
\begin{list2}
\item We just got interrupted in our important job with the CIS controls
\item We now need to fill out an Incident Log and calculate the cost
\item When an incident happens it must be dealth with effeciently
\item If you don't have security procedures it will often be longer lasting and more expensive
\end{list2}

\slide{March 2021: ProxyLogon/ProxyShell CVE-2021-26855 CVSS:3.0 9.1 / 8.4}
\begin{quote}
In March 2021, both Microsoft and IT Professionals had a major headache in the form of an Exchange zero-day commonly known as ProxyLogon. The vulnerability, widely considered the {\bf most critical to ever hit Microsoft Exchange}, was quickly exploited in the wild by suspected state-sponsored threat actors, with US government and military systems identified as the most targeted sectors. {\bf Ransomware variants such as DoejoCrypt were soon actively exploiting unpatched Exchange instances}, attempting to monetise the vulnerability.

A follow-up exploit, dubbed ProxyShell, was evolutionary in nature and targeted on-premise Client Access Servers (CAS) in {\bf all supported versions of Exchange Server.} Due to the {\bf remotely accessible nature of Exchange CAS, any unpatched instances would be vulnerable to Remote Code Execution. High profile victims included the European Banking Authority and the Norwegian Parliament.}
\end{quote}
Source - for this description:\\
\link{https://chessict.co.uk/resources/blog/posts/2022/january/2021-top-security-vulnerabilities/}


\slide{ProxyLogon CVE-2021-26855 CVSS:3.0 9.1 / 8.4}

\begin{quote}
ProxyLogon is the formally generic name for CVE-2021-26855, a vulnerability on Microsoft Exchange Server that allows an attacker bypassing the authentication and impersonating as the admin. We have also chained this bug with another post-auth arbitrary-file-write vulnerability, CVE-2021-27065, to get code execution. All affected components are vulnerable by default!

As a result, {\bf an unauthenticated attacker can execute arbitrary commands on Microsoft Exchange Server through an only opened 443 port!}
\end{quote}

Sources: \link{https://proxylogon.com/}\\
\link{https://msrc.microsoft.com/update-guide/vulnerability/CVE-2021-26855}





\slide{June 2021: PrintNightmare CVE-2021-34527 CVSS:3.0 8.8 / 8.2}
\begin{quote} \small
In June, Microsoft released a critical security update to address weaknesses in the Printer Spooler service on Windows desktop and server platforms. Unfortunately, it was released out-of-band outside of the standard patch Tuesdays due to the severity. Microsoft even released patches for Windows 7, an supported operating system that does not normally receive updates.

Initially categorised by Microsoft as a local privilege escalation on Windows, security researchers subsequently identified an additional {\bf Remote Code Execution (RCE)} vector resulting in an updated advisory from Microsoft. As ever, the ability to test and deploy patches in a time-sensitive manner is key to minimising the impact of such vulnerabilities.

Additionally, PrintNightmare had the additional horror factor of dropping during the {\bf summer holiday season in the northern hemisphere}. Our consultants continue to see systems vulnerable to PrintNightmare on client engagements, which can be trivially leveraged to obtain privilege escalation on unpatched Windows systems.
\end{quote}

Source - for this description:\\
\link{https://chessict.co.uk/resources/blog/posts/2022/january/2021-top-security-vulnerabilities/}

See also \link{https://msrc.microsoft.com/update-guide/vulnerability/CVE-2021-34527}

Note: this incident happened during summer time, vacations etc, double the cost.


\slide{September 2021: ForcedEntry}
\begin{quote}\small
{\bf Apple didn’t escape the wrath of critical zero-day vulnerabilities in 2021}, with ForcedEntry made public in September. The concern was not just that it could escape in-built sandbox controls and be leveraged against {\bf almost all iOS versions at the time}, but also that it was in the form of a {\bf one-click exploit meaning that no user interaction was needed}. A threat actor would simply require the target victim’s phone number or email address to send a weaponised GIF. {\bf Furthermore, iMessage was affected on macOS and watchOS, giving the exploit a significant attack surface of well over a billion devices.}

An analysis released at the end of 2021 confirmed a highly complex exploit which is believed to have been created by the NSO Group, creators of the Pegasus platform, albeit with the sophistication of nation-state actors. Given the nature of the attack and the level of complexity, high profile individuals are likely to be the intended targets of such exploits, only used sparingly against targeted victims.
\end{quote}

Source - for this description:\\
\link{https://chessict.co.uk/resources/blog/posts/2022/january/2021-top-security-vulnerabilities/}

See also
\link{https://en.wikipedia.org/wiki/FORCEDENTRY}



\slide{November 2021: Log4Shell}
\begin{quote}\small
It would not be possible to discuss 2021 in the context of vulnerabilities without the mention of Log4Shell. {\bf A widely used Java-based logging library caused headaches for Security professionals worldwide}. Many scrambled to quantify their use of Log4j within their estates.

A zero-day exploit quickly followed, confirming the worst - {\bf Remote Code Execution (RCE) was indeed possible.} However, what made the nature of the vulnerability even more challenging was the ability to exploit a backend logging system from an unaffected front end host. For example, an attacker can craft a weaponised log entry on a mobile app or webserver not running Log4j. The attacker could make their way through to backend middleware itself running Log4j, which significantly extends the attack surface of the vulnerability.

The NCSC even took the step of recommending the update was immediately applied, whether or not Log4Shell was known to be in use. As is commonly the case with critical vulnerabilities, two successive Log4j patches were subsequently released in the week following the original addressing Denial of Service (DoS) and a further RCE. This further increased workloads of Security and IT teams just as they thought the worst of 2021 had been and gone.
\end{quote}
Source - for this description:\\
\link{https://chessict.co.uk/resources/blog/posts/2022/january/2021-top-security-vulnerabilities/}

See also \link{https://en.wikipedia.org/wiki/Log4Shell}



\slide{March 2022: Dirty pipe Linux CVE-2022-0847}

%\hlkimage{}{}

\begin{quote}
This is the story of CVE-2022-0847, a vulnerability in the {\bf Linux kernel since 5.8} which allows overwriting data in arbitrary read-only files. This leads to {\bf privilege escalation because unprivileged processes can inject code into root processes.}

It is similar to CVE-2016-5195 “Dirty Cow” but is easier to exploit.

The vulnerability was fixed in Linux 5.16.11, 5.15.25 and 5.10.102.
\end{quote}
Sources:
\link{https://dirtypipe.cm4all.com/}
\link{https://thestack.technology/dirty-pipe-exploited-linux-vulnerability-cve-2022-0847/}
\link{https://access.redhat.com/security/cve/CVE-2022-0847}




\slide{April 2022: Lenovo UEFI }
\begin{quote}
ESET researchers have discovered and analyzed three vulnerabilities affecting various Lenovo consumer laptop models. The first two of these vulnerabilities – CVE-2021-3971, CVE-2021-3972 – affect UEFI firmware drivers originally meant to be used only during the manufacturing process of Lenovo consumer notebooks. Unfortunately, they were mistakenly included also in the production BIOS images without being properly deactivated. These affected firmware drivers can be activated by attacker to directly disable SPI flash protections (BIOS Control Register bits and Protected Range registers) or the UEFI Secure Boot feature from a privileged user-mode process during OS runtime. It means that exploitation of these vulnerabilities would allow attackers to deploy and successfully execute SPI flash or ESP implants, like LoJax or our latest UEFI malware discovery ESPecter, on the affected devices.
\end{quote}

Source:

also:
\link{https://www.welivesecurity.com/2022/04/19/when-secure-isnt-secure-uefi-vulnerabilities-lenovo-consumer-laptops/}

See also:
\link{https://www.bleepingcomputer.com/news/security/lenovo-uefi-firmware-driver-bugs-affect-over-100-laptop-models/}



\end{document}
