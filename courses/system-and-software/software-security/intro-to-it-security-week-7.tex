\documentclass[Screen16to9,17pt]{foils}
\usepackage{kea-slides}
\externaldocument{intro-to-it-security-exercises}
\selectlanguage{english}

% OB2 Softwaresikkerhed (10 ECTS)
% Indhold
% Modulet fokuserer på sikkerhedsperspektivet i software, blandt andet programkvalitet og
% fejlhåndterings samt datahåndterings betydning for en software arkitekturs sårbarheder.
% Elementet introducerer også til forskellige designprincipper, herunder ”security by design”.
% Læringsmål
% Viden
% Den studerende har viden om:
% Hvilken betydning programkvalitet har for it-sikkerhed ift.:
%* Trusler mod software
%* Kriterier for programkvalitet
%* Fejlhåndtering i programmer
%* Forståelse for security design principles, herunder:
% * security by design
% * privacy by design
% Færdigheder
% Den studerende kan:
% Tage højde for sikkerhedsaspekter ved at:
% * Programmere håndtering af forventede og uventede fejl
% * Definere lovlige og ikke-lovlige input data, bl.a. til test
%* Bruge et API og/eller standard biblioteker
% * Opdage og forhindre sårbarheder i programkoder
% * Sikkerhedsvurdere et givet software arkitektur
% Kompetencer
% Den studerende kan:
% * Håndtere risikovurdering af programkode for sårbarheder.
%* Håndtere udvalgte krypteringstiltag

\begin{document}

\mytitlepage{Introduction to Software Security}
{Introduction to IT-Security;\\Dat Introduction to IT Security 2024 Week 7}

\hlkprofiluk

\slide{Goals for today}

\hlkimage{9cm}{nesa-by-makers-IgUR1iX0mqM-unsplash.jpg}

\begin{list2}
\item Introduction to Software Security
\item Create a good starting point for learning about \emph{softsec}
\item Learn to find resources, files and programs/libraries
\item Prepare tools for the exercises \hfill {\small Photo by NESA by Makers on Unsplash}
\end{list2}


\slide{Goals: Software Security}

\hlkimage{12cm}{software.pdf}

\begin{list1}
\item Software security - because software is everywhere
\end{list1}


\slide{Plan for software security in this course}

\hlkimage{4cm}{Blob.jpg}
Picture from the OpenBSD project, software blobs

\begin{list2}
\item Get started looking at this big subject
\item 3 days with mix of teaching and doing exercises
\item Last day dedicated to catching up and playing with OWASP JuiceShop
\end{list2}


\slide{Plan for today}

\begin{list2}
\item Software Security intro -- why do we have problems
\item Software is complex
\item All software have security vulnerabilities
\end{list2}

Exercises
\begin{list2}
\item Git getting started
\item UFW firewall -- you cannot hack what you cannot talk to!
\item OWASP JuiceShop -- tool installation, docker and an insecure software application
\item Django tutorial -- not doing it, but reading
\end{list2}


\slide{Exercises}

\hlkimage{7cm}{Shaking-hands_web.jpg}

Since we are going to be doing exercises, each team will need a laptop

The following are two recommended systems:
\begin{list2}
\item One able to run Docker software servers and web applications
%\item Setup instructions and help \url{https://github.com/kramse/kramse-labs}
\end{list2}

Docker is a toolbox we will use and participants will use a project called OWASP JuiceShop



\slide{Course Materials}

\begin{list1}
\item This material is in multiple parts:
\begin{list2}
%\item Introduktionsmateriale med baggrundsinformation
\item Slide shows - presentation - this file
\item Exercises booklet
\item Pwning OWASP Juice Shop Official companion guide to the OWASP Juice Shop
Can be found online for free, but recommend buying the PDF from \link{https://leanpub.com/juice-shop} - suggested price USD 5.99

\end{list2}
\item Additional resources from the internet
\item Note: the presentation slides are not a substitute for reading and doing exercises, many details are not shown
\end{list1}

\slide{Confidentiality, Integrity and Availability}

\hlkimage{8cm}{cia-triad-uk.pdf}

\begin{list1}
\item We want to protect something
\item Confidentiality - data kept a secret
\item Integrity - data is not subjected to unauthorized changes
\item Availability - data and systems are available when needed
\end{list1}


\slide{What is Infrastructure -- Software}


\hlkimage{10cm}{alexander-schimmeck-SeeM4AnkEHE-unsplash.jpg}

\begin{list2}
\item Enterprises today have a lot of computing systems supporting the business needs
\item These are very diverse and often discrete systems
\end{list2}

\hfill Photo by Alexander Schimmeck on Unsplash

\slide{Business Challenges}

\hlkimage{7cm}{adam-bignell-9tI2z5VZIZg-unsplash.jpg}

\begin{list2}
\item Accumulation of software
\item Legacy systems
\item Partners
\item Various types of data
\item Employee churn, replacement \hfill Photo by Adam Bignell on Unsplash
\end{list2}


\slide{Software Challenges}

\hlkimage{7cm}{john-barkiple-l090uFWoPaI-unsplash.jpg}

\begin{list2}
\item Complexity
\item Various languages
\item Various programming paradigms, client server, monolith, Model View Controller
\item Conflicting data types and available structures
\item Steam train vs electric train \hfill Photo by John Barkiple on Unsplash

\end{list2}




\slide{Developers Challenges}

\hlkimage{10cm}{kelly-sikkema-YK0HPwWDJ1I-unsplash.jpg}

\begin{list2}
\item Work in teams across organisation - and partners, vendors, sub-contractors
\item Work with legacy systems, old technology
\item Learn new Technologies \hfill Photo by Kelly Sikkema on Unsplash
\end{list2}




\slide{Integration Challenges}

% hands

\hlkimage{10cm}{thomas-drouault-IBUcu_9vXJc-unsplash.jpg}

\begin{list2}
\item Enable communication between components
\item Need mediator, interpreter, translator
\item Recognize standard patterns \hfill Photo by Thomas Drouault on Unsplash
\end{list2}

\slide{Leap Years}

%\hlkimage{}{}

Who can remember the rules for leap years?

\begin{quote}
The leap year problem (also known as the leap year bug or the leap day bug) is a problem for both digital (computer-related) and non-digital documentation and data storage situations which results from errors in the calculation of which years are leap years, or from manipulating dates without regard to the difference between leap years and common years.
\end{quote}
Source: \link{https://en.wikipedia.org/wiki/Leap_year} and
\link{https://en.wikipedia.org/wiki/Leap_year_problem}

\begin{list2}
\item Something with February
\item Something with every fourth year
\end{list2}

\slide{Leap Years, the rules}

%\hlkimage{}{}

\begin{quote}
For example, in the Gregorian calendar, each leap year has 366 days instead of 365, by extending February to 29 days rather than the common 28. These extra days occur in {\bf each year that is an integer multiple of 4 (except for years evenly divisible by 100, but not by 400).} The leap year of 366 days has 52 weeks and two days, hence the year following a leap year will start later by two days of the week.
\end{quote}
Source: \link{https://en.wikipedia.org/wiki/Leap_year}

\begin{list2}
\item So 1900 was not a leap year
\item 2000 was a leap year!
\end{list2}


\slide{Falsehoods programmers believe about time}

%\hlkimage{}{}

\begin{quote}
I have repeatedly been confounded to discover just how many mistakes in both test and application code stem from misunderstandings or misconceptions about time. By this I mean both the interesting way in which computers handle time, and the fundamental gotchas inherent in how we humans have constructed our calendar – daylight savings being just the tip of the iceberg.
\end{quote}
Source: \link{https://infiniteundo.com/post/25326999628/falsehoods-programmers-believe-about-time}

All of these assumptions are wrong
\begin{list2}
\item 1 There are always 24 hours in a day.
\item 2. Months have either 30 or 31 days.
\item 3. Years have 365 days.
\item ...
\end{list2}

\centerline{Lesson, calculations with time are complex, don't implement time software -- use libraries!}

\slide{Assumptions}

\begin{quote}
Any security policy, mechanism, or procedure is based on assumptions that, if incorrect, destroy the superstructure on which it is built.\\
Matt Bishop, Computer Security 2019
\end{quote}

\begin{list1}
\item Example, vendor patches
\item Important points:
\begin{list2}
\item Is patch correct? Example Spectre and heartbleed
\item Vendor test environments equal to intended environments
\item Installed correctly - including operator skills
\end{list2}
\end{list1}


\slide{Unix Manual system}

\hlkimage{7cm}{images/Unix-command-1.pdf}

\begin{quote}
 It is a book about a Spanish guy called Manual. You should read it.
       -- Dilbert
\end{quote}

\begin{list1}
\item Manual system in Unix is always there!
\item Key word search \verb+man -k+ see also \verb+apropos+
\item Different sections, can be chosen
\end{list1}

See \verb+man crontab+ the command vs the file format in section 5 \verb+man 5 crontab+



\slide{A manual page}

\begin{alltt}\footnotesize
\small
NAME
     cal - displays a calendar
SYNOPSIS
     cal [-jy] [[month]  year]
DESCRIPTION
   cal displays a simple calendar.  If arguments are not specified, the cur-
   rent month is displayed.  The options are as follows:
   -j      Display julian dates (days one-based, numbered from January 1).
   -y      Display a calendar for the current year.

The Gregorian Reformation is assumed to have occurred in 1752 on the 3rd
of September.  By this time, most countries had recognized the reforma-
tion (although a few did not recognize it until the early 1900's.)  Ten
days following that date were eliminated by the reformation, so the cal-
endar for that month is a bit unusual.
\end{alltt}

\slide{The year 1752}

\begin{alltt}\footnotesize
  user@Projects:$ cal 1752
...
         April                  May                   June
  Su Mo Tu We Th Fr Sa  Su Mo Tu We Th Fr Sa  Su Mo Tu We Th Fr Sa
            1  2  3  4                  1  2      1  2  3  4  5  6
   5  6  7  8  9 10 11   3  4  5  6  7  8  9   7  8  9 10 11 12 13
  12 13 14 15 16 17 18  10 11 12 13 14 15 16  14 15 16 17 18 19 20
  19 20 21 22 23 24 25  17 18 19 20 21 22 23  21 22 23 24 25 26 27
  26 27 28 29 30        24 25 26 27 28 29 30  28 29 30
                        31
          July                 August              September
  Su Mo Tu We Th Fr Sa  Su Mo Tu We Th Fr Sa  Su Mo Tu We Th Fr Sa
            1  2  3  4                     1  {\bf        1  2 14 15 16}
   5  6  7  8  9 10 11   2  3  4  5  6  7  8  17 18 19 20 21 22 23
  12 13 14 15 16 17 18   9 10 11 12 13 14 15  24 25 26 27 28 29 30
  19 20 21 22 23 24 25  16 17 18 19 20 21 22
  26 27 28 29 30 31     23 24 25 26 27 28 29
                        30 31
...
\end{alltt}

\slide{Hackerlab Setup}

\hlkimage{6cm}{hacklab-1.png}

\begin{list2}
\item Hardware: modern laptop CPU with virtualisation\\
Dont forget to enable hardware virtualisation in the BIOS
\item Virtualisation software: Docker
\item Setup instructions can be found in the exercise booklet
\end{list2}

\centerline{It is enough if teams each have one laptop with Docker}

\slide{About the exercises}

\hlkimage{10cm}{thomas-drouault-IBUcu_9vXJc-unsplash.jpg}

You will need in your group -- 2-3 persons is recommended:
Docker or similar container technology,Browser -- Firefox comes to mind and Burp Suite -- Java program

I will use a virtual machine with Debian 12 Bookworm for this! Most exercises can be executed from this VM. You may be able to run exercises from your normal operating system, but it may take longer than just adding a Debian VM and doing it.

\slide{Technologies used}

The following tools and environments are examples that may be introduced in this course:

\begin{list2}
\item Programming languages and frameworks Java, Python, regular expressions
\item Development environments -- choose your own IDE / Editor -- I use Atom
\item Networking and network protocols: TCP/IP, HTTP, DNS
\item Formats XML, JSON, CSV
\item Web technologies and services: REST, API, HTML5, CSS, JavaScript
\item Tools like cURL, Git and Github
\item Cloud and virtualisation {\bf Docker}\\
Kubernetes, Azure, AWS, microservices -- cloud is a big part of security today
\end{list2}

\centerline{This list is not complete or a promise }

\slide{OWASP Juice Shop Project}

\hlkimage{3cm}{JuiceShop_Logo_100px.png}

We will also use the OWASP Juice Shop Tool Project as a running example. This is an application which is modern AND designed to have security flaws.

Read more about this project at: \link{https://www.owasp.org/index.php/OWASP_Juice_Shop_Project}\\ \link{https://github.com/bkimminich/juice-shop}

It is recommended to buy the Pwning OWASP Juice Shop Official companion guide to the OWASP Juice Shop from \link{https://leanpub.com/juice-shop} - suggested price USD 5.99


\slide{Aftale om test af netværk}

\vskip 1cm
{\bfseries Straffelovens paragraf 263 Stk. 2. Med bøde eller fængsel
  indtil 6 måneder
straffes den, som uberettiget skaffer sig adgang til en andens
oplysninger eller programmer, der er bestemt til at bruges i et anlæg
til elektronisk databehandling.}

Hacking kan betyde:
\begin{list2}
\item At man skal betale erstatning til personer eller virksomheder
\item At man får konfiskeret sit udstyr af politiet
\item At man, hvis man er over 15 år og bliver dømt for hacking, kan
  få en bøde - eller fængselsstraf i alvorlige tilfælde
\item At man, hvis man er over 15 år og bliver dømt for hacking, får
en plettet straffeattest. Det kan give problemer, hvis man skal finde
et job eller hvis man skal rejse til visse lande, fx USA og
Australien
\item Frit efter: \link{http://www.stophacking.dk} lavet af Det
  Kriminalpræventive Råd
\item Frygten for terror har forstærket ovenstående - så lad være!
\end{list2}


\slide{Exercise CHAOS: Don't Panic -- have fun learning}

\hlkimage{6cm}{dont-panic.png}

\begin{quote}
“It is said that despite its many glaring (and occasionally fatal) inaccuracies, the Hitchhiker’s Guide to the Galaxy itself has outsold the Encyclopedia Galactica because it is slightly cheaper, and because it has the words ‘DON’T PANIC’ in large, friendly letters on the cover.”
\end{quote}
Hitchhiker’s Guide to the Galaxy, Douglas Adams

\slide{Your lab setup}

\begin{list2}
\item Go to GitHub, Find user Kramse, click through security-courses, find the software-security folder\\
\link{https://github.com/kramse/security-courses/tree/master/courses/system-and-software/software-security}
\item Look into the files named: \verb+intro-to-it-security-week-7.pdf+, \verb+intro-to-it-security-week-8.pdf+ and \verb+intro-to-it-security-exercises.pdf+
\item Install Docker on a laptop in your team -- you might already have it
\end{list2}

Hint: If you have a Debian virtual machine you can install the docker software as a package using the Ansible tool, by checking out a repo \verb+kramse-labs+ and running Ansible:
\begin{alltt}
sudo apt install ansible git
git clone https://github.com/kramse/kramse-labs.git
cd docker-install
ansible-playbook -v 1-dependencies.yml
\end{alltt}

\slide{Why introduce Git and Github?}

We introduce Git here also because Github, one of the most popular places to store Git repositories has added security tools which you can use.

An example of a security feature at Github is the use of dependencies, when a project is stored on Github they can scan for outdated dependencies which have security issues.

You can read more about the features available, and some common problems in software in their article:
\emph{How GitHub secures open source software}
Feb 23, 2021 // 10 min read\\
\link{https://resources.github.com/security/open-source/how-github-secures-open-source-software/}


\exercise{ex:git-tutorial}

\exercise{ex:debian-firewall}


\slide{Design vs Implementation}

Software vulnerabilities can be divided into two major categories:
\begin{list2}
\item Design vulnerabilities
\item Implementation vulnerabilities
\end{list2}

Even with a well-thought-out security design a program can contain implementation flaws.

\slide{Common Secure Design Issues}

\begin{list2}
\item Design must specify the security model's structure\\
Not having this written down is a common problem
\item Common problem AAA Authentication, Authorization, Accounting (book uses audited)
\item Weak or Missing Session Management
\item Weak or Missing Authentication
\item Weak or Missing Authorization
\end{list2}


\slide{Input Validation}

Missing or flawed input validation is the number one cause of many of the most severe vulnerabilities:
\begin{list2}
\item Buffer overflows - writing into control structures of programs, taking over instructions and program flow
\item SQL injection - executing commands and queries in database systems
\item Cross-site scripting - web based attack type
\item Recommend centralizing validation routines
\item Perform validation in secure context, controller on server
\item Secure component boundaries
\end{list2}

\slide{Weak Structural Security}

Our book describes more design flaws:
\begin{list2}
\item Large Attack surface
\item Running a Process at Too High a Privilege Level, dont run everything as root or administrator
\item No Defense in Depth, use more controls, make a strong chain
\item Not Failing Securely
\item Mixing Code and Data
\item Misplaced trust in External Systems
\item Insecure Defaults
\item Missing Audit Logs
\end{list2}

\slide{Secure Programming for Linux and Unix Howto}

\begin{list1}
\item More information about systems design and implementation can be found in the free resource:
\item Secure Programming for Linux and Unix HOWTO, David Wheeler
\item \link{https://dwheeler.com/secure-programs/Secure-Programs-HOWTO.pdf}
\item Chapter 5. Validate All Input details input validation in the context of Unix programs
\item Chapter 6. Restrict Operations to Buffer Bounds (Avoid Buffer Overflow)
\item Chapter 7. Design Your Program for Security
\end{list1}

This is included to show that software security has been around for decades. Linux is also a very common platform today, being part of both Android, Cloud and Internet of Things (IoT)

\slide{Principle of Least Privilege}

\begin{list1}
\item {\bf Definition 14-1} The \emph{principle of least privilege} states that a subject should be given only those privileges that it needs in order to complete the task.
\item Also drop privileges when not needed anymore, relinquish rights immediately
\item Example, need to read a document - but not write.
\item Database systems can often provide very fine grained access to data
\end{list1}

\slide{Principle of Fail-Safe defaults}

\begin{list1}
\item {\bf Definition 14-3} The \emph{principle of fail-safe defaults} states that, unless a subject is given explicit access to an object, it should be denied access to that object.
\item Default access \emph{none}
\item In firewalls default-deny - that which is not allowed is prohibited
\item Newer devices today can come with no administrative users, while older devices often came with default admin/admin users
\item Real world example, OpenSSH config files that come with \verb+PermitRootLogin no+
\end{list1}

\exercise{ex:sw-startjuice}

\exercise{ex:js-burp}


\slidenext{}


\end{document}
