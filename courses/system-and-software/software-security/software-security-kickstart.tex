\documentclass[a4paper,11pt,notitlepage, landscape]{report}
% Henrik Kramselund, February 2001
% hlk@security6.net,
% My standard packages
\usepackage{zencurity-one-page}

\begin{document}

\rm
\selectlanguage{english}

\newcommand{\subject}[1]{Software Security course}

%\mytitle{SIEM and Log Analysis}{exercises}
\lhead{\fancyplain{}{\color{titlecolor}\bfseries\LARGE Kickstart: \subject}}


\normal

This material is prepared for use in \emph{\subject} and was prepared by
Henrik Kramselund, \url{xhek@kea.dk}.
It contains the very basic information to get started!

These course and exercises are expected to be performed in a training setting with network connected systems. The exercises use a number of tools which can be copied and reused after training. A lot is described about setting up your workstation in the Github repositories.

{\bf The main site is: \link{https://github.com/kramse/}} I try to gather all information there!\\
To get kickstarted in this course:

\begin{list2}
\item[\faSquareO] Make sure you can login to Fronter \link{https://kea-fronter.itslearning.com/}\\
Electronic version of this document will be uploaded here!
\item[\faSquareO] Lecture plan for this course will be in Fronter\\
Source is also in Git \link{https://github.com/kramse/kea-it-sikkerhed/blob/master/softwaresikkerhed/lektionsplan.md}
\item[\faSquareO] Bookmark the main Github page: \link{https://github.com/kramse/}\\
Note: there are two pinned repositories \verb+security-courses+ and \verb+kramse-labs+
\item[\faSquareO] Slides and exercises booklet -- PDF will be in Fronter\\
Source is in Github -- feel free clone or download single files\\
{\footnotesize\link{https://github.com/kramse/security-courses/tree/master/courses/system-and-software/software-security}}
\item[\faSquareO] We will use virtual machines in teams of two persons, so check BIOS settings\\
-- make sure CPU settings have virtualisation turned ON
\item[\faSquareO] Select and install virtualisation software
\item[\faSquareO] Read about setup of exercise systems here\\
\link{https://github.com/kramse/kramse-labs}
\item[\faSquareO] Get the books! Either on paper or PDF
\item[\faSquareO] Get the supporting resources, to be found in Fronter and also linked in lecture plan
\end{list2}

I hope we will have a fun and enjoyable time in this course.

Best regards

Henrik Kramselund, he/him
\end{document}
