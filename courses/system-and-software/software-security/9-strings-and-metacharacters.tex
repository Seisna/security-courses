\documentclass[Screen16to9,17pt]{foils}
\usepackage{zencurity-slides}
\externaldocument{software-security-exercises}
\selectlanguage{english}


\begin{document}

\mytitlepage
{9. Strings and Metacharacters}
{KEA Kompetence OB2 Software Security}

\slide{Goals: }

\hlkimage{5cm}{unicode-1.pdf}

\begin{list1}
\item Strings are used in most programs,
\item Problems with strings go back decades like Microsoft IIS 4.0/5.0 Unicode bug CVE-2000-0884
\item Handling letters, numbers, sentences, filenames, ... - string data
\item Multiple data formats, from American Standard Code for Information Interchange (ASCII), Extended Binary Coded Decimal Interchange Code (EBCDIC), ISO 8859-1 / ISO-8859-15 EURO €
\item From 7-bit ASCII, 8-bit ASCII to multibyte symbols in Unicode
\item Lots of opportunity for errors, seaching on google for \emph{unicode bug CVE} gave 500.000 hits!
\end{list1}

\slide{Plan for today}

\begin{list1}
\item Subjects
\begin{list2}
\item Processing strings
\item C String handling
\item Metacharacters
\item Character sets and unicode
\end{list2}
\item Exercises
\begin{list2}
\item Recommendations for handling strings, how does Python help, how does Django handle strings, and input validation
\item Truncate and Encoding Attacks JuiceShop
\item Django String Handling
\end{list2}
\end{list1}

\slide{Reading Summary}

\begin{list1}
\item \emph{Gray Hat Hacking} chapters 1-2, 11-13 - browse if you need to, many pages.
\item Browse: \emph{Return-Oriented Programming:Systems, Languages, and Applications} and \emph{Removing ROP Gadgets from OpenBSD}
\end{list1}

Also checkout \url{https://en.wikipedia.org/wiki/C_string_handling}
for use when you dont have the books with you.

Wikipedia calls format string vulnerabilities Uncontrolled format string \link{https://en.wikipedia.org/wiki/Uncontrolled_format_string}



\slide{Processing strings}

\begin{quote}
Many of the most significant security vulnerabilities of the last decade, (1997-2007) are the result of memory corruption due to mishandling textual data, or logical flaws due to the misinterpretation of the content on the textual data
\end{quote}
Source: \emph{The Art of Software Security Assessment Identifying and Preventing
Software Vulnerabilities} 2007

\begin{list1}
\item Spoiler, the problems didn't end in 2007
\item Major areas of string handling:
\begin{list2}
\item memory corruption due to string mishandling
\item Vulnerabilities due to in-band control data in the form of metacharacters
\item Vulnerabilities resulting from conversions between character encodings in different languages
\end{list2}
\item By understanding the {\bf common patterns} associated with these vulnerabilities, you can identify and prevent their occurence
\end{list1}

\slide{C String handling}


\begin{minted}[fontsize=\footnotesize]{c}
#include <stdio.h>
#include <stdlib.h>
#include <string.h>
int main(int argc, char **argv)
{      char buf[10];
        strcpy(buf, argv[1]);
        printf("%s\n",buf);
}
\end{minted}


\begin{list2}
\item In C there is no native type for strings; strings are formed by constructing arrays of the char data type, with the null character (0x00) marking the end of a string
\item The string is then represented by the pointer to the beginning, \verb+buf+
\item C++ standard library has a string class, a little safer
\item Converting between C++ string class and C strings may result in vulnerabilities
\item Many systems use C at the bottom, C APIs etc.
\end{list2}


\slide{Unbounded String Functions}

 Unsafe group of functions:
\begin{list2}
\item {\bf scanf()} read data from somewhere, multiple variants
\item {\bf sprintf()} print formatted into string/buffer - overflow\\
Changing the format string is a whole group in itself
\item {\bf strcpy()} family is notorious for causing a large number of security vulnerabilities
\item {\bf strcat()} string concatenation, combining strings can be problemtatic
\end{list2}

They will continue whatever they do, until they meet a null-terminator or get an error, segmentation faults
\vskip 1cm
\centerline{These were the ones people used in the beginning, and forever}

\slide{Many Years ago around 1988 }

\begin{minted}[fontsize=\footnotesize]{c}
/usr/src/etc/fingerd.c from 4.3BSD:
main(argc, argv)
	char *argv[];
{
	register char *sp;
	char line[512];
	struct sockaddr_in sin;
...
	line[0] = '\0';
	gets(line);
\end{minted}

Source code link \url{https://www.tuhs.org/cgi-bin/utree.pl?file=4.3BSD/usr/src/etc/fingerd.c}

More description in the articles:\\
{\footnotesize\url{https://spaf.cerias.purdue.edu/tech-reps/823.pdf}} \emph{The Internet Worm Program: An Analysis}
Purdue Technical Report CSD-TR-823
Eugene H. Spafford\\ {\footnotesize\url{https://blog.rapid7.com/2019/01/02/the-ghost-of-exploits-past-a-deep-dive-into-the-morris-worm/}}\\ The Ghost of Exploits Past: A Deep Dive into the Morris Worm


\slide{Exim CVE-2019-15846 git diff exim-4.92.1 exim-4.92.2}

\begin{minted}[fontsize=\footnotesize]{c}
diff --git a/src/src/string.c b/src/src/string.c
@@ -224,6 +224,8 @@ interpreted in strings.
 Arguments:
   pp       points a pointer to the initiating \ in the string;
            the pointer gets updated to point to the final character
+           If the backslash is the last character in the string, it
+           is not interpreted.
 Returns:   the value of the character escape
 */

@@ -236,6 +238,7 @@ const uschar *hex_digits= CUS"0123456789abcdef";
 int ch;
 const uschar *p = *pp;
 ch = *(++p);
+if (ch == '\0') return **pp;
\end{minted}

The vulnerability is exploitable by sending a SNI ending in a
backslash-null sequence during the initial TLS handshake. The exploit
exists as a POC. More details see doc/doc-txt/cve-2019-15846/ in the source code repository.


Internet worm 1988 -- Exim 2019, about 30 years of \emph{progress}


\slide{Printf -- formatted printing}

\begin{quote}
The printf() function can be used to print more than just fixed strings. This
function can also use format strings to print variables in many different for-
mats. A format string is just a character string with special escape sequences
that tell the function to insert variables printed in a specific format in place
of the escape sequence. The way the printf() function has been used in the
previous programs, the \verb+"Hello, world!\n"+ string technically is the format string;
however, it is devoid of special escape sequences. These escape sequences are
also called format parameters, and for each one found in the format string, the
function is expected to take an additional argument. Each format parameter
begins with a percent sign ( % ) and uses a single-character shorthand very
similar to formatting characters used by GDB’s e x amine command.
\end{quote}

Source: \emph{Hacking, 2nd Edition: The Art of Exploitation}, Jon Erickson

\slide{Examples printf}

\begin{minted}{c}
// Example of printing with different format string
printf("[A] Dec: %d, Hex: %x, Unsigned: %u\n", A, A, A);
printf("[B] Dec: %d, Hex: %x, Unsigned: %u\n", B, B, B);
printf("[field width on B] 3: '%3u', 10: '%10u', '%08u'\n", B, B, B);
printf("[string] %s Address %08x\n", string, string);
// Example of unary address operator (dereferencing) and a %x format string
printf("variable A is at address: %08x\n", &A);
\end{minted}

Can also print into another string (buffers) with functions like \verb+sprintf()+ -- string printf

Source: \emph{Hacking, 2nd Edition: The Art of Exploitation}, Jon Erickson

\slide{0x350 Format Strings}

\begin{quote}
{\bf Exploiting format string vulnerabilities}\\

A format string exploit is another technique you can use to gain control of
a privileged program. Like buffer overflow exploits, format string exploits also
depend on programming mistakes that may not appear to have an obvious
impact on security. Luckily for programmers, once the technique is known,
it’s fairly easy to spot format string vulnerabilities and eliminate them.
Although format string vulnerabilities aren’t very common anymore, the
following techniques can also be used in other situations.
\end{quote}

Summary:
\begin{list2}
\item 0x353 Reading from Arbitrary Memory Addresses
\item 0x354 Writing to Arbitrary Memory Addresses
\item Plus some tips and hints for making it easier
\end{list2}

\slide{Bounded String Functions}

Adding a maximum length to the functions should help:
\begin{list2}
\item {\bf snprintf()} copies a maximum number of bytes!
\item Different semantics on Windows and Unix.
\item Windows does not guarantee null-termination, returns -1
\item Unix guarantee null-termination, returns number of chars that would have been written had there been enough room
\item {\bf strncpy()} does accept a maximum number of bytes to be copied into the destination, but does not guarantee null termination
\item {\bf strncat()} size to provide is the space left in the buffer, not the size of the whole buffer
\item Easy to result in off-by-one vulnerabilities
\end{list2}

\vskip 1cm

\centerline{Programming is easy, right \smiley}

\slide{Better Functions from BSD}

\begin{list2}
\item strlcpy, strlcat  size-bounded string copying and concatenation
\item {\bf strlcpy()} a variant of strcpy that truncates the result to fit in the destination buffer
\item {\bf strlcat()} a variant of strcat that truncates the result to fit in the destination buffer
\item Originally OpenBSD 2.4 in December, 1998
\item These functions always write one null to the destination buffer
\item May truncate the result, return size of buffer needed, programmer must check return code and handle this
\end{list2}

Example references:\\
\link{https://en.wikipedia.org/wiki/C_string_handling}\\
\link{https://en.wikibooks.org/wiki/C_Programming/C_Reference/nonstandard/strlcpy}

\slide{Parsing String Data}


\begin{minted}[fontsize=\normalsize]{c}

char t[1000];
char tt[100];

// Get data into t from somewhere

while (*t != ':') {
  *tt++ = *t++;
}
*tt = 0;
\end{minted}

\begin{list2}
\item Example, if the input is larger than destination pointed to by \verb+tt+ then problems can arise
\item Character expansion, making output bigger can overflow
\item Other examples can be found across the internet over 30+ years
\end{list2}

\slide{Metacharacters}

\begin{list2}
\item Null 0x00, special in C, but just another char in other languages
\item Space
\item / used as filename delimiters, and \textbackslash{} in Windows
\item . dot used in various ways for domain names, file types etc.
\item Comma-seperated files, using \verb+, . ; :+ etc.
\item Special characters for syntax purposes, \verb+* % & | ?+ etc. Searching for everything or wild card search
\end{list2}



\slide{File Name Canonicalization}

\verb+C:\WINDOWS\system32\calc.exe+

or
\begin{list2}
\item \verb+C:\WINDOWS\system32\drivers\..\calc.exe+
\item \verb+calc.exe+
\item \verb+.\calc.exe+
\item \verb+..\calc.exe+
\item \verb+\\?\WINDOWS\system32\calc.exe+
\item Attacks are called path or directory traversal, using \verb+..+ to enter paths not expected by the application, ref Microsoft IIS Unicode vulnerabilities

\item Linux allows you to reference a file like \verb+wc -l /etc/passwd+ or \verb+wc -l ///etc/////passwd+
\end{list2}



\slide{Shell Metacharacters}

\begin{alltt}
<pre>
<?php passthru("{\bf{}ping $HOST}"); ?>
</pre>
\end{alltt}


\begin{list2}
\item Misc dangerous shell characters:
\item \verb+;+ seperator, execute multiple commands, \verb+|+ pipe, execute multiple commands
\item \verb+` `+ back ticks, or \verb+$( )+ execute a command and insert result
\item \verb+< > + redirect input, output etc.
\item Perl: \verb+print `/usr/bin/finger $input{'command'}`;+
\item UNIX shell: \verb+`echo hello`+
\item Microsoft SQL: \verb+exec master..xp_cmdshell 'net user test testpass /ADD'+
\item I prefer explicit allow filters (white lists) for filtering metacharacters, if at all possible. Easier for a phone number than name, YMMV
\end{list2}



\slide{HTML and XML encoding, plus serialization}

\begin{list2}
\item HTML and XML can contain encoded data \verb+%20+ is a space
\item Requests sent over HTTP can contain serialization and de-serialization, basically sending code
\item Multiple layers of decoding can result in problems, like double-decode Microsoft IIS vulnerability CVE-2001-0333
\end{list2}


\slide{Character sets and unicode}

\verb;GET /..%c0%af..%c0%afwinnt/system32/cmd.exe?/c+dir;

\begin{list2}
\item UTF-8 becoming the standard used, example from CVE-2000-0884
\item Calls \verb+cmd.exe+ with any command from URL
\item Example encoding for \verb+/+
\item 0x2f
\item 0xC0 0xAF - the one used above
\item 0xE0 0x80 0xAF
\item 0xF0 0x80 0x80 0xAF
\end{list2}


\slide{Django: Unicode data}

\begin{quote}
{\bf Django supports Unicode data everywhere.}\\

This document tells you what you need to know if you’re writing applications that use data or templates that are encoded in something other than ASCII.
...
{\bf Useful utility functions}\\
Because some string operations come up again and again, Django ships with a few useful functions that should make working with string and bytestring objects a bit easier.
\end{quote}


Source:\\
\link{https://docs.djangoproject.com/en/3.1/ref/unicode/}

\exercise{ex:django-string}
\exercise{ex:django-orm}
\exercise{ex:django-email}

\exercise{ex:truncate-encoding}




\slidenext{}


\end{document}
