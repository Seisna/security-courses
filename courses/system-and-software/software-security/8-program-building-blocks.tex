\documentclass[Screen16to9,17pt]{foils}
\usepackage{zencurity-slides}
\externaldocument{software-security-exercises}
\selectlanguage{english}

\begin{document}

\mytitlepage
{8. Program Building Blocks}
{KEA Kompetence OB2 Software Security}


\slide{Goals for today}

\hlkimage{6cm}{thomas-galler-hZ3uF1-z2Qc-unsplash.jpg}

Todays goals:
\begin{list2}
\item Talk about the Design Patterns concept
\item Present some of the ones often found in programs
\item Patterns can also be found in other areas, like Patterns in Network Architecture
\end{list2}

  Photo by Thomas Galler on Unsplash

\slide{Plan for today}

\begin{list1}
\item Subjects
\begin{list2}
\item Common constructs
\item Recurring code patterns
\item Example programs with flaws: OpenSSH, OpenSSL, Windows MS-RPC DCOM, Linux teardrop
\end{list2}
\item Exercises
\begin{list2}
\item Static analysis with PMD
\item Git hook example
\item JuiceShop SQL login
\item Use a XML library in Python
\end{list2}
\end{list1}

\slide{Reading Summary}

\begin{list1}
\item \emph{Gray Hat Hacking} chapters 1-2, 11-13 - browse if you need to, many pages.
\end{list1}



\slide{Design Patterns}

\begin{list2}
\item \emph{Design Patterns: Elements of Reusable Object-Oriented Software} (1994), Erich Gamma, Richard Helm, Ralph Johnson, and John Vlissides

\item The book describes 23 classic software design patterns

\item \url{https://en.wikipedia.org/wiki/Design_Patterns}

\item Ideas of patterns precede this book, but became a more popular subject
\end{list2}



\slide{Common constructs}


\begin{list2}
\item Programs exhibit the same patterns, some examples:
\item Solve problems in the same domains
\item Need to store lists of strings / characters etc.
\item Data structures becomes useful in other programs
\item Sorting routines needed in many programs
\end{list2}





\slide{Auditing code patterns}

Various pitfalls, and areas
\begin{list2}
\item {\bf Variable Relationship}
\item Examples presented are C buffers
\item A variable-pointer to the buffer and the variable with the length
\item Their relationship can easily get invalidated, leading to vulnerabilities
\item Question: would object oriented programming help?
\end{list2}



\slide{Relevant examples}


\begin{list2}
\item Apache was the worlds most popular web server!
\item \verb+mod_dav+ implements a very complex protocol, file server features using http!
\item Bind and the resolver libraries were the de facto, and still is, for DNS resolving in most of the open source software world!
\item Sendmail was the most popular mail server for many years, but replaced mostly by Qmail, Postfix and Exim. Sendmail was used on both open source and commercial Unix operating systems like SunOS, AIX, HP-UX etc.
\item OpenSSH mentioned next is also the most popular SSH protocol implementation, found in almost all Linux distributions and a wide range of network devices and other internet conected things
\item OpenSSL is of course the most popular open source crypto library ... famous for the heartbleed bug and other hits
\end{list2}


\slide{Structure and Object Mismanagement}


\begin{list2}
\item Structures in C can help group related data elements
\item Both auditors and attackers can benefit ...
\item Object oriented programs and languages encapsulate this
\item Responsibility is on the object implementation, good!
\item We saw structures last time
\end{list2}



\slide{Uninitialized Variables}

\begin{minted}[fontsize=\footnotesize]{c}
    Packet *p = SCMalloc(SIZE_OF_PACKET);
    if (unlikely(p == NULL)) {
         return 0;
    }
    ThreadVars tv;
    DecodeThreadVars dtv;

    memset(&dtv, 0, sizeof(DecodeThreadVars));
    memset(&tv,  0, sizeof(ThreadVars));
    memset(p, 0, SIZE_OF_PACKET);
\end{minted}

\begin{list2}
\item Always make sure variables have well defined values.
\item Defensive program above use memset to clear contents of buffers
\item Example from Suricata allocating memory, checking it got the memory, clearing contents
\end{list2}


\slide{Manipulating Lists - and other data structures}


\begin{list2}
\item Storing data in a list is nice, can add and remove elements
\item Single linked list, start from head and go through list ... slow
\item Double linked list can go back again from element
\item More work updating, moving more pointers ... complex, may introduce errors
\item Example data structure double linked list\\
\url{https://algorithms.tutorialhorizon.com/doubly-linked-list-complete-implementation/}
\item Multiple problems can arise, also using the wrong structure for something can result in vulnerabilities
\end{list2}

\vskip 1cm
\centerline{Dont write your data structure libraries yourself}

\slide{Linux Teardrop}

\begin{minted}[fontsize=\footnotesize]{python}
#! /usr/bin/env python3
if attack == '0':
  print "Using attack 0"
  size=36    offset=3   load1="\x00"*size
  i=IP()   i.dst=target   i.flags="MF"   i.proto=17

  size=4     offset=18  load2="\x00"*size
  j=IP()   j.dst=target   j.flags=0   j.proto=17   j.frag=offset
  send(i/load1)
  send(j/load2)
\end{minted}

\begin{list2}
\item IP fragments, packets are split when crossing a link with lower MTU
\item If fragments created by an attacker are overlapping it creates a problem
\item Scapy example code by Sam Bowne, full code can be found at:\\
\url{https://samsclass.info/123/proj10/teardrop.htm}
\item what is a Maximum Transmission Unit (MTU)\\
See \url{https://en.wikipedia.org/wiki/Maximum_transmission_unit} for a description,\\ related/similar attack setting source=destination \url{https://en.wikipedia.org/wiki/LAND}
\end{list2}


\slide{Other problems}


\begin{list2}
\item Hashing algorithms
\item There have been multiple problems with hashing algorithms
\item Denial of Service and arbitrary code can be the result
\item Example vulns from popular programming languages, others have similar!\\
\url{https://www.cvedetails.com/vulnerability-list/vendor_id-74/product_id-128/cvssscoremin-7/cvssscoremax-7.99/PHP-PHP.html} search for hash \\
\url{https://github.com/bk2204/php-hash-dos} specific example in PHP 7.0.0~rc3-3\\
\url{https://www.ruby-lang.org/en/security/} Ruby has some
\item also when programmers selected wrong, or weak, hashing algorithms for passwords
\end{list2}

\slide{Control Flow}

\hlkimage{18cm}{ssl-tls-breaks-timeline.png}

\begin{list2}
\item If constructs, make sure to exercise each path
\item Case/switch constructs, make sure to catch a default
\item Loops being subverted to create buffer overflow, terminating conditions
\item Forgetting a \verb+break+ in a case
\item Picture selected due to the Apple "goto fail" - certificate validation\\
\link{https://dwheeler.com/essays/apple-goto-fail.html}
\end{list2}


\slide{Apple Goto Fail CVE-2014-1266}

\begin{minted}[fontsize=\footnotesize]{python}
if ((err = SSLHashSHA1.update(&hashCtx, &clientRandom)) != 0)
    goto fail;
if ((err = SSLHashSHA1.update(&hashCtx, &serverRandom)) != 0)
    goto fail;
if ((err = SSLHashSHA1.update(&hashCtx, &signedParams)) != 0)
    goto fail;
    goto fail;  /* MISTAKE THIS LINE SHOULD NOT BE HERE */
if ((err = SSLHashSHA1.final(&hashCtx, &hashOut)) != 0)
    goto fail;
\end{minted}

\begin{list2}
\item Always going to goto fail, second goto is always active
\item Leading to later checks not performed
\item Example code from\emph{Anatomy of a “goto fail” – Apple’s SSL bug explained}\\ {\footnotesize\url{https://nakedsecurity.sophos.com/2014/02/24/anatomy-of-a-goto-fail-apples-ssl-bug-explained-plus-an-unofficial-patch/}}

\end{list2}


\slide{Loop running over buffer}

\begin{list2}
\item MS-RPC DCOM buffer overflow
\item Ended up in the Blaster worm infecting : \url{https://en.wikipedia.org/wiki/Blaster_(computer_worm)}
\item notice the timeline, even if patches ARE available people didn't patch
\item This is why we have \emph{patch tuesdays}
\item Blaster was not as fast as SQL Slammer, which infected the internet in just minutes
\item Other example from book, NTPD, which is also a common and thought to be safe service to run
\item And \verb+mod_php+ nonterminating buffer vulnerability
\item Plus a few off-by-one errors in \verb+mod_rewrite+ and OpenBSD ftpd
\item ...
\end{list2}


\slide{Side effects, corner cases and 32-bit vs 64-bit}

\begin{list2}
%\item Chapter lists a couple of corner cases
\item Functions can change variables in global scope while doing something
\item Allocating memory can go wrong, check return values
\item Asking for 0 bytes of memory is technically legal but may cause problems
\item Doing 64-bit check with if and then being truncated to 32-bit when doing actual memory allocation function, result in unintended behaviour
\item Double free is also mentioned, freeing an already free location may exploit heap management routines\\
Check malloc and free very carefully
\end{list2}


\slide{Who do you trust?}


\begin{list2}
\item Example programs shown with flaws in this chapter: OpenSSH, OpenSSL, Windows MS-RPC DCOM, Linux teardrop
\item Who do you trust?
\item Can we trust any software?

\item What about Suricata? - CVE list\\
{\footnotesize\url{https://www.cvedetails.com/vulnerability-list/vendor_id-13364/product_id-27771/Openinfosecfoundation-Suricata.html}}
\item Spoiler: Bypass vulnerabilities, getting data past the IDS, Denial of Service crashing the IDS
\end{list2}


\slide{Example applications from Microsoft}

How to get ahead? - use existing good examples!

Microsoft has released sample applications.

\begin{quote}
Secure Development Documentation
Learn how to develop and deploy secure applications on Azure with our sample apps, best practices, and guidance.

Get started
Develop a secure web application on Azure
\end{quote}

Source:
\url{https://docs.microsoft.com/en-us/azure/security/develop/}

Yes, this describes how to run Alpine Linux on their Azure Cloud.



\slide{Resources}

\begin{quote}\footnotesize
Microsoft Security Development Lifecycle (SDL) – The SDL is a software development process from Microsoft that helps developers build more secure software. It helps you address security compliance requirements while reducing development costs.

Open Web Application Security Project (OWASP) – OWASP is an online community that produces freely available articles, methodologies, documentation, tools, and technologies in the field of web application security.

Pushing Left, Like a Boss – A series of online articles that outlines different types of application security activities that developers should complete to create more secure code.

Microsoft identity platform – The Microsoft identity platform is an evolution of the Azure AD identity service and developer platform. It’s a full-featured platform that consists of an authentication service, open-source libraries, application registration and configuration, full developer documentation, code samples, and other developer content. The Microsoft identity platform supports industry-standard protocols like OAuth 2.0 and OpenID Connect.

Security best practices for Azure solutions – A collection of security best practices to use when you design, deploy, and manage cloud solutions by using Azure. This paper is intended to be a resource for IT pros. This might include designers, architects, developers, and testers who build and deploy secure Azure solutions.

Security and Compliance Blueprints on Azure – Azure Security and Compliance Blueprints are resources that can help you build and launch cloud-powered applications that comply with stringent regulations and standards.

Next steps\\
In the following articles, we recommend security controls and activities that can help you design, develop, and deploy secure applications.

* Design secure applications
* Develop secure applications
* Deploy secure applications
\end{quote}

Source:
\url{https://docs.microsoft.com/en-us/azure/security/develop/secure-dev-overview}


\slide{Good security}

\hlkimage{12cm}{god-sikkerhed.pdf}

\begin{list1}
\item You always have limited resources for protection - use them as best as possible
\end{list1}


\slide{Balanced security}

\hlkimage{21cm}{afbalanceret-sikkerhed.pdf}

\begin{list1}
\item Better to have the same level of security
\item If you have bad security in some part - guess where attackers will end up
\item Hackers are not required to take the hardest path into the network
\item Realize there is no such thing as 100\% security
\end{list1}



\slide{Jumping through networks and systems}

\begin{list1}
\item Hackers break into systems they can reach
\item and hackers break into systems they can reach from hacked systems \smiley
\item In real life networks, a remote root exploit from the internet to the main database is rare
\item but jumping through others can possibly break the whole network
\item There are a lot of hackers which have presented how they hacked companies, example\\
{\tiny\url{https://arstechnica.com/security/2016/04/how-hacking-team-got-hacked-phineas-phisher/}}
\end{list1}

\slide{Example hacks we have done during testing}

\begin{list1}
\item My own examples include pentest assignments where we:
\begin{list2}
\item Two servers under test, Solaris\\
had NFS world export, we could mount, found a backup of the other system, extracted /etc/passwd from backip, user logins with empty password, Solaris Telnet allows login, your password is empty - enter one
\item Another test had SRX firewall as a core component. Due to misconfiguration we could access with SNMP, and thereby found the complete network structure revealed, how many network segments, subnets, netmask, ARP, interface counters - which lead to access files with password pictures, it was a bank
\item Countless times we have found either a password file for a database, and used the same passwords for the operating system, or vice versa
\item Countless times we have found either a password file in test systems, and people have used the same password in production, or vice versa
\end{list2}
\end{list1}


\slide{First advice use the modern operating systems}

\begin{list1}
\item Newer versions of Microsoft Windows, Mac OS X and Linux
\begin{list2}
\item Buffer overflow protection
\item Stack protection, non-executable stack
\item Heap protection, non-executable heap
\item \emph{Randomization of parameters} stack gap m.v.
\end{list2}
\item Note: these still have errors and bugs, but are better than older versions
\item OpenBSD has shown the way in many cases\\ \link{http://www.openbsd.org/papers/}
\end{list1}

\vskip 1cm

\centerline{Always try to make life worse and more costly for attackers}

\slide{EIP Patterns}

\hlkimage{14cm}{eip-patterns.png}

Following slides are mostly inspiration, and not specifically curriculum

\slide{Microservices for Java chapter 1}

\hlkimage{3cm}{microservices-for-java-developers.jpg}

Microservices for Java Developers , Christian Posta, 2016 O’Reilly\\
 \link{https://www.oreilly.com/programming/free/files/microservices-for-java-developers.pdf}

We will use the introduction, which is recommended.

\slide{Whats in the book}

\begin{quote}
This book is for Java developers and architects interested in developing microservices. We start the book with the high-level understanding and fundamental prerequisites that should be in place to be successful with a microservice architecture.
\end{quote}

Introducing some Java frameworks
\begin{list2}
\item The Spring ecosystem, Dropwizard and WildFly Swarm, we’ll use JBoss Forge CLI
\item Finally, when we build and deploy our microservices as Docker containers running inside of Kubernetes
\item They use VirtualBox with Docker and Kubernetes, YMMV
\item With source on Github, not updated in years though!\\
\link{https://github.com/redhat-developer/microservices-by-example-source}
\end{list2}
Source: {\footnotesize\\
Microservices for Java Developers , Christian Posta, 2016 O’Reilly}



\slide{Open source is also leading the charge in the technology space}

\begin{quote}\small
Open source is also leading the charge in the technology space. Following the commoditization curves, open source is a place developers can go to challenge proprietary vendors by building and
  innovating on software that was once only available (without source
  no less) with high license costs. This drives communities to build
  things like operating systems (Linux), programming languages (Go),
  message queues (Apache ActiveMQ), and web servers ( httpd ). Even
  companies that originally rejected open source are starting to come
  around by open sourcing their technologies and contributing to
  existing communities. As open source and open ecosystems have
  become the norm, we’re starting to see a lot of the innovation in
  software technology coming directly from open source communities
  (e.g., Apache Spark, Docker, and Kubernetes).
\end{quote}

\begin{list2}
\item We have used Linux and multiple products from the Apache website/foundation
\end{list2}
Source: {\footnotesize\\
Microservices for Java Developers , Christian Posta, 2016 O’Reilly}




\slide{Building distributed systems is hard}

\hlkimage{8cm}{microservices-java-book.png}

\begin{quote}\small
  Microservice architecture (MSA) is an approach to building software systems that decomposes business domain models into smaller,
  consistent, bounded-contexts implemented by services. These services are isolated and autonomous yet communicate to provide some
  piece of business functionality. Microservices are typically implemented and operated by small teams with enough autonomy that
  each team and service can change its internal implementation
  details (including replacing it outright!) with minimal impact across
  the rest of the system.
\end{quote}

Source: {\footnotesize\\
Microservices for Java Developers , Christian Posta, 2016 O’Reilly}





\slide{Teamwork}


\begin{list2}
\item Teams communicate through promises
\item specify these promises with interfaces
of their services and via wikis that document their services
\item Each team would be responsible for designing the service, picking
the right technology for the problem set, and deploying, managing
and waking up at 2 a.m. for any issues
\item Understand what the service is doing without being tangled into
other concerns in a larger application
\item Quickly build the service locally
\item Pick the right technology for the problem (lots of writes? lots of
queries? low latency? bursty?)
\item Test the service
\item Build/deploy/release at a cadence necessary for the business,
which may be independent of other services
\item Identify and horizontally scale parts of the architecture where
needed
\item Improve resiliency of the system as a whole



\end{list2}
Source: {\footnotesize\\
Microservices for Java Developers , Christian Posta, 2016 O’Reilly}




\slide{Challenges}

\begin{list2}
\item Microservices may not be efficient. It can be more resource intensive.
\item You may end up with
what looks like duplication.
\item Operational complexity is a lot higher.
\item It becomes very difficult to understand the system holistically.
\item It becomes significantly harder to debug problems.
\item In some areas you may have to relax the notion of transaction.
\end{list2}
Source: {\footnotesize\\
Microservices for Java Developers , Christian Posta, 2016 O’Reilly}




\slide{Design for Faults}

\begin{quote}
  Things will fail, so we must develop our applications to be resilient and handle failure, not just prevent it. We should
  be able to deal with faults gracefully and not let faults propagate to
  total failure of the system.

  Building distributed systems is different from building shared-
memory, single process, monolithic applications. One glaring differ‐
ence is that communication over a network is not the same as a local
call with shared memory.
\end{quote}

\begin{list2}
\item Networks are inherently unreliable
\end{list2}
Source: {\footnotesize\\
Microservices for Java Developers , Christian Posta, 2016 O’Reilly}


\slide{Design with Dependencies in Mind}

\begin{quote}
  To be able to move fast and be agile from an organization or
  distributed-systems standpoint, we have to design systems with
  dependency thinking in mind; we need loose coupling in our teams,
  in our technology, and our governance.
\end{quote}

Source: {\footnotesize\\
Microservices for Java Developers , Christian Posta, 2016 O’Reilly}



\slide{Design with Promises in Mind}

\begin{quote}
  In a microservice environment with autonomous teams and serv‐
  ices, it’s very important to keep in mind the relationship between
  service provider and service consumer. As an autonomous service
  team, you cannot place obligations on other teams and services
  because you do not own them; they’re autonomous by definition. All
  you can do is choose whether or not to accept their promises of
  functionality or behavior. As a provider of a service to others, all you
  can do is promise them a certain behavior.
\end{quote}

\begin{list2}
\item Promises as published by APIs and versions in those!
\item References \emph{Consumer-Driven Contracts: A Service Evolution Pattern}\\
\link{https://martinfowler.com/articles/consumerDrivenContracts.html}
\end{list2}
Source: {\footnotesize\\
Microservices for Java Developers , Christian Posta, 2016 O’Reilly}

\slide{Chapter 10: Service API and Contract Versioning\\
with Web Services and REST Services}

\begin{quote}
After a service contract is deployed, consumer programs will naturally begin forming dependencies on it. When we are subsequently forced to make changes
 to the contract, we need to figure out:
\begin{list2}
\item Whether the changes will negatively impact existing (and potentially future) service consumers
\item How changes that will and will not impact consumers should be implemented and communicated
\end{list2}
\end{quote}

Included in the chapter are references to:
\begin{list2}
\item Versioning Web Services
\item Versioning REST Services
\item Fine and Coarse-Grained Constraints
\end{list2}
Source: {\footnotesize\\
\emph{Service‑Oriented Architecture: Analysis and Design for Services and Microservices}, Thomas Erl, 2017}

\slide{Versioning and Compatibility}

\begin{list2}
\item Backwards Compatibility\\
A backwards-compatible change to a REST-compliant service contract might involve adding some new resources or adding new capabilities to existing resources. In each of these cases the existing service consumers will only invoke the old methods on the old resources, which continue to work as they previously did.
\item Forwards Compatibility\\
When a service contract is designed in such a manner so that it can support a range of future consumer programs, it is considered to have an extent of forwards compatibility. This means that the contract can essentially accommodate how consumer programs will evolve over time.
\end{list2}

Source: {\footnotesize\\
\emph{Service‑Oriented Architecture: Analysis and Design for Services and Microservices}, Thomas Erl, 2017}


\slide{REST Service Capability Granularity}

\hlkimage{12cm}{soabook-7-19-rest-service.png}

\begin{list2}
\item REST using HTTP has the standard HTTP methods available (e.g., GET, POST, PUT, DELETE);
\item See also \link{https://en.wikipedia.org/wiki/Representational_state_transfer}
\end{list2}
Source: {\footnotesize\\
\emph{Service‑Oriented Architecture: Analysis and Design for Services and Microservices}, Thomas Erl, 2017}



\slide{Docker }

%\hlkimage{}{}

\begin{quote}
  Package Software into Standardized Units for Development, Shipment and Deployment
  A container is a standard unit of software that packages up code and all its dependencies so the application runs quickly and reliably from one computing environment to another. A Docker container image is a lightweight, standalone, executable package of software that includes everything needed to run an application: code, runtime, system tools, system libraries and settings.
\end{quote}
Source: \\{\footnotesize
\url{https://www.docker.com/resources/what-container}}

\begin{list2}
  \item One of the most popular deployment methods today
\end{list2}

\slide{Containerized Applications}

\hlkimage{8cm}{container-what-is-container.png}

Source: {\footnotesize
\url{https://www.docker.com/resources/what-container}}

\begin{list2}
  \item See also \link{https://en.wikipedia.org/wiki/Linux_namespaces}\\
   \emph{Various container software use Linux namespaces in combination with cgroups to isolate their processes, including Docker[11] and LXC.}
\end{list2}

\slide{Docker Images and layers}

\hlkimage{10cm}{container-layers.jpg}

\begin{quote}
A Docker image is built up from a series of layers. Each layer represents an instruction in the image’s Dockerfile. Each layer except the very last one is read-only.
\end{quote}

Source: {\footnotesize
\link{https://docs.docker.com/storage/storagedriver/}}



\slide{Kubernetes}

%\hlkimage{}{}

\begin{quote}
  Kubernetes (K8s) is an open-source system for automating deployment, scaling, and management of containerized applications.
  It groups containers that make up an application into logical units for easy management and discovery. Kubernetes builds upon 15 years of experience of running production workloads at Google, combined with best-of-breed ideas and practices from the community.
\end{quote}
Source: {\footnotesize
\link{https://kubernetes.io/}}

Key points:
\begin{list2}
\item Open source originally from Google
\item Scalable
\item Uses containers inside
\item Infrastructure as code
\end{list2}


\slide{Infrastructure as code}

%\hlkimage{}{}

\begin{quote}
{\bf Infrastructure as code (IaC)} is the process of managing and provisioning computer data centers through machine-readable definition files, rather than physical hardware configuration or interactive configuration tools.[1] The IT infrastructure managed by this comprises both physical equipment such as bare-metal servers as well as virtual machines and associated configuration resources. The definitions may be in a version control system. It can use either scripts or declarative definitions, rather than manual processes, but the term is more often used to promote declarative approaches.
\end{quote}
Source: {\footnotesize
\link{https://en.wikipedia.org/wiki/Infrastructure_as_code}}

\begin{list2}
  \item Has become the norm in many places
\end{list2}

\slide{Getting started with Kubernetes: Minikube}

\centerline{Not running it now, but do on your own, if you like}
\begin{alltt}
  minikube start --cpus 2 --memory 2048 --disk-­size 10g
\end{alltt}
The last parameter is important; it specifies which VM driver to use (see Minikube documentation for details). After the installation is complete, you can get the status of Minikube:
\begin{alltt}
  $ minikube status
  minikubeVM: Running
  localkube: Running
  kubectl: Correctly Configured: pointing to minikube-­vm at 192.168.64.2
\end{alltt}

This means the local Kubernetes cluster is up and running.

This is a nice way to try Kubernetes -- can be done locally or in the cloud:\\
\link{https://kubernetes.io/docs/tutorials/hello-minikube/}

\exercise{ex:pmd-static}

\exercise{ex:git-hook}

\exercise{ex:juice-shop-login}

\exercise{ex-python-library}

\slidenext{}

\end{document}
