\documentclass[Screen16to9,17pt]{foils}
\usepackage{zencurity-slides}
\externaldocument{siem-log-analysis-exercises}
\selectlanguage{english}


\begin{document}

\mytitlepage
{1. Initial Overview of SIEM Terms}
{KEA Kompetence SIEM and Log Analysis}


\slide{Goals for today}

\hlkimage{6cm}{thomas-galler-hZ3uF1-z2Qc-unsplash.jpg}

Todays goals:
\begin{list2}
\item Introduce some terms and technologies
\item Make it possible to run Python -- easy on Debian
\item Learn to find resources, data files and programs/libraries
\item Put SIEM into context of the information security picture
\end{list2}

  Photo by Thomas Galler on Unsplash


\slide{Plan for today}

\begin{list1}
\item Subjects
\begin{list2}
\item Introduction to R, Python, using the book
\item Github/Git
\item SIEM and SOC
\item Context, what are the threats, what are the answers we want from the SIEM and Logs
\end{list2}
\item Exercise theme: Building a house requires materials
\begin{list2}
\item Python Libraries - git clone
\item Date formats ISO 8601
\item Grok patterns, regular expressions
\item Fetching resources from the internet -- http-resources
\end{list2}
\end{list1}


\slide{Time schedule}

\begin{list2}
\item 17:00 - 18:15\\

\item 30min break\\

\item 18:45 - 19:30\\

\item 15min break\\

\item 19:45 -20:30 45min\\
\end{list2}



\slide{Reading Summary}

\begin{list1}
\item DDS 1. The Journey to Data-Driven Security
\item DDS 2. Building Your Analytics Toolbox: R and Python
\item CIP 1 Incident Response Fundamentals
\item CIP 2 What Are You Trying to Protect?
\vskip 1cm
\item Skim CIP 3 What Are the Threats?

\end{list1}



\slide{Reading Summary, continued}

%\hlkimage{}{}

\begin{quote}
Data-Driven Security: Analysis, Visualization and Dashboards has been designed to take you on a journey
into the world of security data science. The start of the journey looks a bit like the word cloud shown in Figure 1, which was created from the text in the chapters of this book. You have a great deal of information available to you, and may be able to pick out a signal or two within the somewhat hazy noise on your own.

However, it’s like looking for a needle in a haystack without a magnet.
\end{quote}
Source: DDS 1. The Journey to Data-Driven Security

\begin{list2}
\item A history of Learning from Data
\item Data Analysis Skills
\item Exploring Data
\item Tools needed vary, I recommend the editor Atom, install using Ansible \smiley
\end{list2}

\slide{Reading Summary, continued}

%\hlkimage{}{}

\begin{quote}
A discussion of which programming language is better than another for a certain set of tasks often turns (quickly) into a religious war of words that rarely wins converts and never becomes fully resolved. As a security data scientist, you will find that you do not have the luxury of language bias. There will be times when one language shines in one area while a different one shines in another, and you need the skills of a diplomat to bring them both together to solve real problems.
\end{quote}
Source: DDS 2. Building Your Analytics Toolbox: R and Python

\begin{list2}
\item Some tools are needed, some might be nice to know
\item Python is a need to know
\item R is a nice to know
\end{list2}


\slide{Reading Summary, continued}

%\hlkimage{}{}

\begin{quote}
\begin{list2}
\item Keeping an organization safe from attack, as well as having a talented team available to respond quickly, minimizes damage to your reputation and business.
\item Fostering and developing relationships with IT, HR, legal, executives, and others is critical to the success of a CSIRT.
\item Sharing incident and threat data with external groups improves everyone’s security and gives your organization credibility and trust with groups that might be able to help in the future.
\item A good team relies on good tools, and a great team optimizes their operations.
\item A solid and well-socialized InfoSec policy gives the incident response team the authority and charter to protect networks and data.
\end{list2}
\end{quote}
Source: CIP 1 Incident Response Fundamentals

\slide{Reading Summary, continued}

%\hlkimage{}{}

\begin{quote}
\begin{list2}
\item You can’t properly protect your network if you don’t know what to protect.
\item Define and understand your critical assets and what’s most important to your
  organization.
\item Ensure that you can attribute ownership or responsibility for all systems on your
  network.
\item Understand and leverage the log data that can help you determine host owner‐
  ship.
\item A complex network is difficult to protect, unless you understand it well.
\end{list2}
\end{quote}

Source: CIP 2 What Are You Trying to Protect?





\slide{Reading Summary, continued}

%\hlkimage{}{}

\begin{quote}
  \begin{list2}
\item To protect your organization, you must understand the threats it faces.
   \item If you don’t think you have something to lose, you haven’t thought about it
  enough.
   \item Crime evolves with culture and society. Online crime will increase as more things
  of value are digitally stored and globally accessible.
   \item Malicious activity can come from a number of sources, but the most common
  source is organized crime, followed by targeted attackers and trusted insiders.
   \item Different organizations face different threats; focus your efforts on protecting the
  high-value assets and make sure you monitor them closely.
\end{list2}

\end{quote}
Source: CIP 3 What Are the Threats?





\slide{A history of Learning from Data}



%\hlkimage{}{}

\begin{quote}

\end{quote}

\begin{list2}
\item All our books contain a lot of example data
\item Often you start from a few sources, to prove the worth
\end{list2}


\slide{What is Infrastructure}


\hlkimage{10cm}{alexander-schimmeck-SeeM4AnkEHE-unsplash.jpg}

\begin{list2}
\item Enterprises today have a lot of computing systems supporting the business needs
\item These are very diverse and often discrete systems
\end{list2}

\hfill Photo by Alexander Schimmeck on Unsplash


\slide{Business Challenges}

\hlkimage{7cm}{adam-bignell-9tI2z5VZIZg-unsplash.jpg}

\begin{list2}
\item Accumulation of software
\item Legacy systems
\item Partners
\item Various types of data
\item Employee churn, replacement \hfill Photo by Adam Bignell on Unsplash
\end{list2}



\slide{Software Challenges}

\hlkimage{7cm}{john-barkiple-l090uFWoPaI-unsplash.jpg}

\begin{list2}
\item Complexity
\item Various languages
\item Various programming paradigms, client server, monolith, Model View Controller
\item Conflicting data types and available structures
\item Steam train vs electric train \hfill Photo by John Barkiple on Unsplash

\end{list2}




\slide{Developers Challenges}

\hlkimage{10cm}{kelly-sikkema-YK0HPwWDJ1I-unsplash.jpg}

\begin{list2}
\item Work in teams across organisation - and partners, vendors, sub-contractors
\item Work with legacy systems, old technology
\item Learn new Technologies \hfill Photo by Kelly Sikkema on Unsplash
\end{list2}




\slide{Integration Challenges}

% hands

\hlkimage{10cm}{thomas-drouault-IBUcu_9vXJc-unsplash.jpg}

\begin{list2}
\item Enable communication between components
\item Need mediator, interpreter, translator
\item Recognize standard patterns \hfill Photo by Thomas Drouault on Unsplash
\end{list2}


\slide{SIEM}

%\hlkimage{}{}

\begin{quote}
{\bf Security information and event management (SIEM)} is a subsection within the field of computer security, where software products and services combine security information management (SIM) and security event management (SEM). They provide real-time analysis of security alerts generated by applications and network hardware.

  Vendors sell SIEM as software, as appliances, or as managed services; these products are also used to log security data and generate reports for compliance purposes.[1]

  The term and the initialism SIEM was coined by Mark Nicolett and Amrit Williams of Gartner in 2005.[2]
\end{quote}
Source: \link{https://en.wikipedia.org/wiki/Security_information_and_event_management}

\begin{list2}
  \item Note: there are alerting examples towards the bottom of the page, with sources
  \item Closely related to log management, incident response
\end{list2}


\slide{SOC}

%\hlkimage{}{}

\begin{quote}
An information security operations center (ISOC or SOC) is a facility where enterprise information systems (web sites, applications, databases, data centers and servers, networks, desktops and other endpoints) are monitored, assessed, and defended.

...

A security operations center (SOC) can also be called a security defense center (SDC), security analytics center (SAC), network security operations center (NSOC),[3] security intelligence center, cyber security center, threat defense center, security intelligence and operations center (SIOC). In the Canadian Federal Government the term, infrastructure protection center (IPC), is used to describe a SOC.
\end{quote}
Source: \link{https://en.wikipedia.org/wiki/Information_security_operations_center}

\begin{list2}
  \item We have a whole book about SOCs, but I skipped the introductory chapters!
  \item If you need to build a SOC, that is great source of information
\end{list2}




\slide{Subjects: }

\hlkimage{5cm}{homer-end-is-near.jpg}

\begin{list1}
\item Context, what are the threats, what are the answers we want from the SIEM and Logs

\item What are the common cases, where we use the data?

\begin{list2}
\item Incident Response
\item Computer Emergency Response Team (CERT) and Computer Security Incident Response Teams (CSIRT)
\item Security Departments
\item GDPR Data protection
\item Computer Forensics
\end{list2}

\end{list1}


\slide{Incident Handling, phases}

The procedures developed for incident response must cover the complete life-cycle

\begin{list2}
\item  Preparation for an attack, establish procedures and mechanisms for detecting and responding to attacks
\item  Identification of an attack, notice the attack is ongoing
\item  Containment (confinement) of the attack, limit effects of the attack as much as possible
\item  Eradication of the attack, stop attacker, block further similar attacks
\item  Recovery from the attack, restore system to a secure state
\item  Follow-up to the attack, include lessons learned  improve environment
\end{list2}


\slide{Crafting the InfoSec Playbook}


This book will help you to answer common questions:
\begin{list2}
\item How do I find bad actors on my network?
\item How do I find persistent attackers?
\item How can I deal with the pervasive malware threat?
\item How do I detect system compromises?
\item How do I find an owner or responsible parties for systems under my protection?
\item How can I practically use and develop threat intelligence?
\item How can I possibly manage all my log data from all my systems?
\item How will I benefit from increased logging—and not drown in all the noise?
\item How can I use metadata for detection?
\end{list2}
Source: \emph{Crafting the InfoSec Playbook: Security Monitoring and Incident Response Master Plan}\\
 by Jeff Bollinger, Brandon Enright, and Matthew Valites ISBN: 9781491949405


\slide{MITRE ATT\&CK framework}

\hlkimage{14cm}{mitre-attack.png}

Source: \link{https://attack.mitre.org/} Great resource for attack categorization

\slide{Incident Response Checklists}
\hlkimage{9cm}{incident-handling-checklist.png}

This checklist is from the NIST document
\emph{Computer Security Incident Handling Guide: Recommendations of the National Institute
of Standards and Technology}, NIST Special Publication 800-61
Revision 2, August 2012.

\slide{CIS Controls also recommend Incident Response}

\begin{quote}{\bf
CIS Control 19:}\\
Incident Response and Management Protect the organization’s information, as well as its reputation, by developing and implementing an incident response infrastructure (e.g., plans, defined roles, training, communications, management oversight) for quickly discovering an attack and then effectively containing the damage, eradicating the attacker’s presence, and restoring the integrity of the network and systems.
\end{quote}

Source:
Center for Internet Security CIS Controls 7.1 CIS-Controls-Version-7-1.pdf
from \link{https://www.cisecurity.org/controls/}



\slide{Anatomy of an Auditing System}


Sample logs from login with Secure Shell (SSH) and performing \verb+sudo su -+
\begin{alltt}\footnotesize
Jun  5 11:53:15 pumba sshd[64505]: Accepted publickey for hlk from 79.142.233.18 port 43902
 ssh2: ED25519 SHA256:l8OJMcywyBcraJiCWJ06uZ2yzHfu0VuiArqVvlVyfEI

Jun  5 11:53:19 pumba sudo:      hlk : TTY=ttyp2 ; PWD=/home/hlk ; USER=root ; COMMAND=/usr/bin/su -
\end{alltt}

\begin{list1}
\item Example systems: Unix syslog, IBM main frame RACF and Windows Event Logs service
\item \emph{swatchdog} is an old skool, but simple tool that works
\item Logs should be protected and considered confidential information
\end{list1}



\slide{Anatomy of an Auditing System}

When data has been gathered it should be analyzed.

\begin{itemize}
\item {\bf Logger functions} - collect
\item {\bf Analyzer} - analyze it, creating dashboard can provide some insights
\item {\bf Notifier} - report results by email or other means
\item Example systems Windows Event Logs service can inform of successful and failed logins, both should be collected
\item Logs should be protected and considered confidential information, by sending it to a centralized system with a high security level protects it
\end{itemize}

Modern systems exist to take all data from logging and provide high capacity storage, searching and sorting.

\slide{Why Elasticsearch}

\hlkimage{8cm}{illustrated-screenshot-hero-siem-500x730.png}
Screenshot from \url{https://www.elastic.co/siem}

Recommend building a proof-of-concept infrastructure using the Elastic stack and gather experience with logging. This can be done without a license fee and the organization can then see what works and doesn't. Then using the experiences as input an informed decision can be made, to continue with this as a home grown logging and auditing solution, or buy a premade one.


\slide{Technologies used in this course}

The following tools and environments are examples that may be introduced in this course:

\begin{list2}
\item Programming languages and frameworks Java, Python, regular expressions
\item Development environments -- choose your own IDE / Editor -- I use {\bf Atom}
\item Networking and network protocols: TCP/IP, HTTP, DNS, Netflow
\item Formats XML, JSON, CSV, raw text, web scraping
\item Web technologies and services: REST, API, HTML5, CSS, JavaScript
\item Tools like cURL, Zeek, Git and Github
\item Message queueing systems: MQ and Redis could be added
\item Aggregated example platforms: Elastic stack, logstash, elasticsearch, kibana, grafana, Filebeat
\item Cloud and virtualisation Docker, Kubernetes, Azure, AWS, microservices -- can be added
\end{list2}

\centerline{This list is not complete or a promise }

\slide{Book: Applied Network Security Monitoring (ANSM)}

\hlkimage{4cm}{ansm-book.png}

\emph{Applied Network Security Monitoring: Collection, Detection, and Analysis}
1st Edition

Chris Sanders, Jason Smith ISBN: 9780124172081 496 pp.
Imprint: Syngress, December 2013

{\footnotesize\link{https://www.elsevier.com/books/applied-network-security-monitoring/unknown/978-0-12-417208-1}}

We use this in the course Network and Communication Security at KEA

\slide{Baseline Skills}

\begin{list2}\small
\item Threat-Centric Security, NSM, and the NSM Cycle
\item TCP/IP Protocols
\item Common Application Layer Protocols
\item Packet Analysis
\item Windows Architecture
\item Linux Architecture
\item Basic Data Parsing (BASH, Grep, SED, AWK, etc)
\item IDS Usage (Snort, Suricata, etc.)
\item Indicators of Compromise and IDS Signature Tuning
\item Open Source Intelligence Gathering
\item Basic Analytic Diagnostic Methods
\item Basic Malware Analysis
\end{list2}

Source: \emph{Applied Network Security Monitoring Collection, Detection, and Analysis}, Chris Sanders and Jason Smith


\slide{OSI and Internet}

\hlkimage{10cm,angle=90}{images/compare-osi-ip.pdf}

\centerline{Data on all layers}

\slide{Networking in TCP/IP}

\hlkimage{12cm}{arp-basic.pdf}

\begin{list2}
\item Everything uses TCP/IP today, more or less.
\item Clients make requests, receives responses
\item HyperText Transfer Protocol (HTTP) is an example
\end{list2}

\slide{Detection Capabilities}


Security incidents happen, but what happens. One of the actions to reduce impact of incidents are done in preparing for incidents.

\begin{itemize}
\item \emph{Preparation} for an attack, establish procedures and mechanisms for detecting and responding to attacks
\end{itemize}

Preparation will enable easy {\bf identification} of affected systems, better {\bf containment} which systems are likely to be infected, {\bf eradication} what happened -- how to do the {\bf eradication} and {\bf recovery}.

\slide{Strategy for implementing identification and detection}

We recommend that the following strategy is used for implementing identification and detection.

We have the following recommendations and actions points for logging:
\begin{enumerate}
\item[\faSquareO] Enable system logging from servers
\item[\faSquareO] Enable system logging from network devices
\item[\faSquareO] Centralize logging
\item[\faSquareO] Add search facilities and dashboards
\item[\faSquareO] Perform system audits manually or automatically
\item[\faSquareO] Setup notification and notification procedures
\end{enumerate}

\slide{Extended Sources}
When a basic logging infrastructure is setup, it can be expanded to increase coverage, by
adding more sources:

\begin{list2}
\item DNS query logging -- will enable multiple cases to be resolved, example malware identification and tracing, when was a malware domain queried, when was the first infection
\item Session data from Firewalls, Netflow -- traffic patterns can be investigated and both attacks and cases like exfiltration can likely be seen
\end{list2}

Hint: Take the sources available first, make a proof-of-concept, expand coverage

\slide{Data Analysis Skills}

\begin{quote}
Although we could spend an entire book creating an exhaustive list of skills needed to be a good security data scientist, this chapter covers the following skills/domains that a data scientist will benefit from
knowing within information security:
\begin{list2}
\item Domain expertise—Setting and maintaining a purpose to the analysis
\item Data management—Being able to prepare, store, and maintain data
\item Programming—The glue that connects data to analysis
\item Statistics—To learn from the data
\item Visualization—Communicating the results effectively
\end{list2}
It might be easy to label any one of these skills as the most important, but in reality, the whole is greater than the sum of its parts. Each of these contributes a significant and important piece to the workings of
security data science.
\end{quote}

Source: \emph{Data-Driven Security: Analysis, Visualization and Dashboards} Jay Jacobs, Bob Rudis\\
ISBN: 978-1-118-79372-5 February 2014 \url{https://datadrivensecurity.info/} - short DDS



\slide{Don't use spreadsheets!}

%\hlkimage{}{}

\begin{quote}

\end{quote}

\begin{list2}
\item Spreadsheets are great for some tasks, but ...
\item They don't scale
\item The model can be broken -- edit a single formula
\item Rounding errors accumulate
\item Input and output are limited
\item Most functions require manual work
\end{list2}

\slide{Start programming}

%\hlkimage{}{}
Example program, reading from HTTP, into Python, out using JSON
\inputminted{python}{programs/rest-1.py}

\begin{list2}
\item Think of programing in this course as as reading, processing and outputting data
\item Read in any format
\item Process with any function
\item Output in any format
\item And reuse existing software
\end{list2}



\slide{DDS: Chapter 2 exercises }

\hlkimage{5cm}{book-data-driven-security.jpg}


\begin{list2}
\item Lets get R and Python running
\item Try as many of the scripts and examples as we want
\end{list2}


\slide{Data overview XML data, JSON}

\hlkimage{15cm}{chris-lawton-5IHz5WhosQE-unsplash.jpg}

Photo by Chris Lawton on Unsplash

\slide{XML data}

\begin{quote}
  Extensible Markup Language (XML) is a markup language that defines a set of rules for encoding documents in a format that is both human-readable and machine-readable. The World Wide Web Consortium's XML 1.0 Specification[2] of 1998[3] and several other related specifications[4]—all of them free open standards—define XML.[5]

  The design goals of XML emphasize simplicity, generality, and usability across the Internet.[6] It is a textual data format with strong support via Unicode for different human languages. Although the design of XML focuses on documents, the language is widely used for the representation of arbitrary data structures[7] such as those used in web services.
\end{quote}
Source: \url{https://en.wikipedia.org/wiki/XML}

\begin{list2}
\item We have seen XML used for configuration in Apache Tomcat and Camel
\item Perfect for computers, less for humans
\end{list2}

\slide{XML data example - Nmap output}

\begin{minted}[fontsize=\footnotesize]{xml}
  <?xml version="1.0" encoding="UTF-8"?>
  <!DOCTYPE nmaprun>
  <?xml-stylesheet href="file:///usr/bin/../share/nmap/nmap.xsl" type="text/xsl"?>
  <!-- Nmap 7.70 scan initiated Sat Feb 22 23:35:53 2020 as: nmap -oA router -sP 10.0.42.1 -->
  <nmaprun scanner="nmap" args="nmap -oA router -sP 10.0.42.1" start="1582410953"
   startstr="Sat Feb 22 23:35:53 2020" version="7.70" xmloutputversion="1.04">
  <verbose level="0"/>
  <debugging level="0"/>
  <host><status state="up" reason="echo-reply" reason_ttl="62"/>
  <address addr="10.0.42.1" addrtype="ipv4"/>
  <hostnames>
  </hostnames>
  <times srtt="2235" rttvar="5000" to="100000"/>
  </host>
  <runstats><finished time="1582410953" timestr="Sat Feb 22 23:35:53 2020" elapsed="0.32"
   summary="Nmap done at Sat Feb 22 23:35:53 2020; 1 IP address (1 host up)
   scanned in 0.32 seconds" exit="success"/><hosts up="1" down="0" total="1"/>
  </runstats>
  </nmaprun>
\end{minted}


\slide{XML data - documents}

\begin{quote}
Hundreds of document formats using XML syntax have been developed,[8] including RSS, Atom, SOAP, SVG, and XHTML. XML-based formats have become the default for many office-productivity tools, including Microsoft Office (Office Open XML), OpenOffice.org and LibreOffice (OpenDocument), and Apple's iWork[citation needed]. XML has also provided the base language for communication protocols such as XMPP. Applications for the Microsoft .NET Framework use XML files for configuration, and property lists are an implementation of configuration storage built on XML.[9]
\end{quote}
Source: \url{https://en.wikipedia.org/wiki/XML}

\begin{list2}
\item Document formats using XML may still be proprietary!
\item Documents using XML can be validated, are they well-formed according to the Document Type Definition (DTD)
\end{list2}


\slide{Transforming XML using XSLT}
\begin{quote}


XSLT (Extensible Stylesheet Language Transformations) is a language for transforming XML documents into other XML documents,[1] or other formats such as HTML for web pages, plain text or XSL Formatting Objects, which may subsequently be converted to other formats, such as PDF, PostScript and PNG.[2] XSLT 1.0 is widely supported in modern web browsers.[3]
\end{quote}
Source: \url{https://en.wikipedia.org/wiki/XSLT}

\begin{list2}
\item Can be seen as a mapping between formats, different XML schemas
\item Also is Turing complete, is a programming language
\end{list2}

\slide{XSLT example}

\begin{minted}[fontsize=\footnotesize]{xml}
<?xml version="1.0" encoding="UTF-8"?>
<xsl:stylesheet xmlns:xsl="http://www.w3.org/1999/XSL/Transform" version="1.0">
  <xsl:output method="xml" indent="yes"/>
  <xsl:template match="/persons">
    <root> <xsl:apply-templates select="person"/> </root>
  </xsl:template>
  <xsl:template match="person">
    <name username="{@username}"> <xsl:value-of select="name" /> </name>
  </xsl:template>
</xsl:stylesheet>
\end{minted}

\begin{list2}
\item XSLT uses XPath to identify subsets of the source document tree and perform calculations. XPath also provides a range of functions
\item XSLT functionalities overlap with those of XQuery, which was initially conceived as a query language for large collections of XML documents\\
Source: \url{https://en.wikipedia.org/wiki/XSLT}
\end{list2}

\slide{xsltproc example using Nmap}

\begin{alltt}\footnotesize
$ su -
# apt install nmap xsltproc
# nmap -sP -oA /tmp/router 91.102.91.18
# exit
$ xsltproc /tmp/router.xml > /tmp/router.html
$ firefox /tmp/router.html
\end{alltt}


\begin{list2}
\item We can use the command line tool \verb+xlstproc+ for transforming documents
\item \verb+apt install xsltproc+
\item Its part of the package Libxslt \url{https://en.wikipedia.org/wiki/Libxslt}
\vskip 2cm
\item Try installing the tools Nmap and \verb+xsltproc+ and reproduce the above
\item This is an easy tool to try, and very useful too
\end{list2}





\slide{Data overview JSON}

\begin{quote}
JavaScript Object Notation (JSON, pronounced /ˈdʒeɪsən/; also /ˈdʒeɪˌsɒn/[note 1]) is an open-standard file format or data interchange format that uses {\bf human-readable text} to transmit data objects consisting of attribute–value pairs and array data types (or any other serializable value). It is a very common data format, with a diverse range of applications, such as serving as replacement for XML in AJAX systems.[6]
\end{quote}
Source: \url{https://en.wikipedia.org/wiki/JSON}

\begin{list2}
\item I like JSON much better than XML
\item Many web services can supply data in JSON format
\end{list2}

\slide{JSON example}

\begin{minted}[fontsize=\footnotesize]{json}
{
  "first name": "John",
  "last name": "Smith",
  "age": 25,
  "address": {
    "street address": "21 2nd Street",
    "city": "New York",
    "state": "NY",
    "postal code": "10021"
  },
  "phone numbers": [
    {
      "type": "home",
      "number": "212 555-1234"
    },
  ],
}
\end{minted}

\begin{list2}
\item This is a basic JSON document, new data attribute-value pairs can be added\\
Source: \url{https://en.wikipedia.org/wiki/JSON}
\end{list2}




\slide{Python and REST}

\inputminted{python}{programs/rest-1.py}

\begin{list2}
\item  Lets try to use some Python to access a REST service.
\item  We will use the JSONPlaceholder which is a free online REST API:
\link{https://jsonplaceholder.typicode.com/}
\item Start at the site: \link{https://jsonplaceholder.typicode.com/guide.html} and try running a few of the examples with your browser
\item Then try using the same URLS in the Requests HTTP library from Python,\\
\link{https://requests.readthedocs.io/en/master/}
\end{list2}


\slide{Note about frameworks and libraries}

\begin{minted}[fontsize=\footnotesize]{python}
import xml.etree.ElementTree as ET
tree = ET.parse('testfile.xml')
root = tree.getroot()

print(root.tag)
print('Nmap version: \t\t{:s} '.format(root.attrib['version']))
print('Nmap started: \t\t{:s} '.format(root.attrib['startstr']))
print('Nmap command line: \t{:s} '.format(root.attrib['args']))

hosts = tree.findall('./host')
for host in hosts:
    print(host.tag)
    print(host.attrib)
    for hostvalues in host:
        print(hostvalues.tag)
        print(hostvalues.attrib)
\end{minted}

\begin{list2}
\item Dont import JSON or XML using home made programs
\item Example uses xml.etree.ElementTree from Python \url{https://docs.python.org/3.7/library/xml.etree.elementtree.html}
\end{list2}

\slide{Convert XML to JSON}

\begin{minted}[fontsize=\footnotesize]{python}
import xml.etree.ElementTree as ET
import json
def etree_to_dict(t):
    d = {t.tag : map(etree_to_dict, t.getchildren())}
    d.update(('@' + k, v) for k, v in t.attrib.items())
    d['text'] = t.text
    return d

tree = ET.parse('testfile.xml')
root = tree.getroot()
mydict = etree_to_dict(root)
print(type(tree))
print(type(root))
print(type(mydict))

print(mydict)

with open('testfile.json', 'w') as json_file:
  json.dump(mydict, json_file)
\end{minted}

Converting using Python is easy



\slide{Side note: Zeek Security Monitor handles formats differently}

Zeek has files formatted with a header:
\begin{alltt}\footnotesize
#fields ts      uid     id.orig_h       id.orig_p       id.resp_h       id.resp_p       proto   trans_id
        rtt     query   qclass  qclass_name     qtype   qtype_name      rcode   rcode_name      AA
        TC      RD      RA      Z       answers TTLs    rejected

1538982372.416180	CD12Dc1SpQm42QW4G3	10.xxx.0.145	57476	10.x.y.141	53	udp	20383
	0.045021	www.dr.dk	1	C_INTERNET	1	A	0	NOERROR	F	F	T	T	0
   www.dr.dk-v1.edgekey.net,e16198.b.akamaiedge.net,2.17.212.93	60.000000,20409.000000,20.000000	F
\end{alltt}

Note: this show ALL the fields captured and dissected by Zeek, there is a nice utility program zeek-cut which can select specific fields:

\begin{alltt}\small
root@NMS-VM:/var/spool/bro/bro# cat dns.log | zeek-cut -d ts query answers | grep dr.dk
2018-10-08T09:06:12+0200	www.dr.dk	www.dr.dk-v1.edgekey.net,e16198.b.akamaiedge.net,2.17.212.93
\end{alltt}

Can also just use JSON now via Filebeat


\exercise{ex-python-library}



\slide{Functions found on the internet, or in libraries}


\begin{list2}
\item Probably every function we require in this course is already implemented
\item Reuse libraries
\item Import math libraries to get statistics
\item Parsing of structures -- import parsing library
\item Need reading a file format, find the package for it
\item Even machine learning -- import it
\item Visualize using a framework
\end{list2}



\slide{Git intro}

\begin{quote}
Git (/ɡɪt/)[7] is a distributed version-control system for tracking changes in source code during software development.[8] It is designed for coordinating work among programmers, but it can be used to track changes in any set of files. Its goals include speed,[9] data integrity,[10] and support for distributed, non-linear workflows.[11]
\end{quote}

Source: \url{https://en.wikipedia.org/wiki/Git}

\begin{list2}
\item We will introduce Git and Github - commercial Git hosting company\\
\url{https://en.wikipedia.org/wiki/Git}
\item Try creating a Git repository, remember to add a README
\item Checkout the repository
\item Edit a file
\item add and commit it
\end{list2}

Use the Github Hello World example: \url{https://guides.github.com/activities/hello-world/}


\slide{My daily job -- Security engineering a job role}

\begin{alltt}\footnotesize
On any given day, you may be challenged to:
        Create new ways to solve existing production security issues
        Configure and install firewalls and intrusion detection systems
        Perform vulnerability testing, risk analyses and security assessments
        Develop automation scripts to handle and track incidents
        Investigate intrusion incidents, conduct forensic investigations and incident responses
        Collaborate with colleagues on authentication, authorization and encryption solutions
        Evaluate new technologies and processes that enhance security capabilities
        Test security solutions using industry standard analysis criteria
        Deliver technical reports and formal papers on test findings
        Respond to information security issues during each stage of a project’s lifecycle
        Supervise changes in software, hardware, facilities, telecommunications and user needs
        Define, implement and maintain corporate security policies
        Analyze and advise on new security technologies and program conformance
        Recommend modifications in legal, technical and regulatory areas that affect IT security
\end{alltt}

Source: \url{https://www.cyberdegrees.org/jobs/security-engineer/}\\
also
\url{https://en.wikipedia.org/wiki/Security_engineering}




\exercise{ex:git-clone-XX}




\slide{A warning about dates!}

%\hlkimage{}{}

\begin{quote}

\end{quote}

\begin{list2}
\item Remember September 1752
\item Dates can be tricky
\item Use a standard date format ISO 8601
\item \emph{Falsehoods programmers believe about time}\\
\link{https://infiniteundo.com/post/25326999628/falsehoods-programmers-believe-about-time}
\item Updated with more:\\
\link{https://infiniteundo.com/post/25509354022/more-falsehoods-programmers-believe-about-time}
\end{list2}


\exercise{ex:dateformats}

\slide{Grok expresssions}

{\footnotesize
\begin{verbatim}
  filter {
    if [type] == "syslog" {
      grok {
        match => { "message" => "%{SYSLOGTIMESTAMP:syslog_timestamp}
        %{SYSLOGHOST:syslog_hostname} %{DATA:syslog_program}
        (?:\[%{POSINT:syslog_pid}\])?: %{GREEDYDATA:syslog_message}" }
        add_field => [ "received_at", "%{@timestamp}" ]
        add_field => [ "received_from", "%{host}" ]
      }
      syslog_pri { }
      date {
        match => [ "syslog_timestamp", "MMM  d HH:mm:ss", "MMM dd HH:mm:ss" ]
      }
    }
  }
\end{verbatim}
}

\begin{list2}
\item Logstash filter expressions grok can normalize and split data into fields
\end{list2}

Source:
Config snippet from recommended link\\
{\small\link{http://logstash.net/docs/1.4.1/tutorials/getting-started-with-logstash}}


\slide{Grok expresssions, sample from my archive}

{\footnotesize
\begin{verbatim}
filter {
# decode some SSHD
if [syslog_program] == "sshd" {
  grok {
# May 20 10:27:08 odn1-nsm-01 sshd[4554]: Accepted publickey for hlk from
10.50.11.17 port 50365 ssh2: DSA 9e:fd:3b:3d:fc:11:0e:b9:bd:22:71:a9:36:d8:06:c7

match => { "message" => "%{SYSLOGTIMESTAMP:timestamp} %{HOSTNAME:host_target}
sshd\[%{BASE10NUM}\]: Accepted publickey for %{USERNAME:username} from
  %{IP:src_ip} port %{BASE10NUM:port} ssh2" }

# "May 20 10:27:08 odn1-nsm-01 sshd[4554]: pam_unix(sshd:session):
session opened for user hlk by (uid=0)"
match => { "message" => "%{SYSLOGTIMESTAMP:timestamp} %{HOSTNAME:host_target}
sshd\[%{BASE10NUM}\]: pam_unix\(sshd:session\): session opened for user
%{USERNAME:username}" }
\end{verbatim}
}

\begin{list2}
\item Logstash filter expressions grok can normalize and split data into fields
\end{list2}


\exercise{ex:grokdebugger1}



\slidenext{}


\end{document}
