\documentclass[a4paper,11pt,notitlepage,landscape]{report}
% Henrik Kramselund  , February 2001
% hlk@security6.net,
% My standard packages
\usepackage{zencurity-one-page}
\usepackage{graphicx}
%\usepackage{lscape}

\begin{document}

\rm
\selectlanguage{english}

\newcommand{\subject}[1]{SIEM and Log Analysis course}

%\mytitle{SIEM and Log Analysis}{exercises}
%{\LARGE Kickstart: SIEM and Log Analysis}{}
\lhead{\fancyplain{}{\color{titlecolor}\bfseries\LARGE Kickstart 2: SIEM and Log Analysis -- SELKS}}


\normal

This material is prepared for use in \emph{\subject} and was prepared by
Henrik Kramselund, \url{xhek@kea.dk} \url{hkj@zencurity.dk}.
It contains the very basic information to get started!

The course had some problems with Elastic stack version 8 -- which is updated on multiple fronts, like HTTPS/TLS. This is giving us a lot of headache. The students meet with different obstacles, so this kickstart 2 document is a way out!

I would like for you to install Docker and try out SELKS \url{https://www.stamus-networks.com/selks}

If you want to use the same as me with Debian VM, which installs in less than 30minutes:
\begin{list1}
\item[\faSquareO] Install a basic Debian 12 Bookworm with Sudo configured
\item[\faSquareO] Install git and Ansible, see our exercise:\\
\verb+sudo apt install git ansible+
\item[\faSquareO] Clone the Github repo: \link{https://github.com/kramse/kramse-labs}\\
\verb+git clone https://github.com/kramse/kramse-labs+
\item[\faSquareO] Go into this repository and install Docker, there is a small README.md too:\\
\verb+cd kramse-labs/docker-install+ and then \verb+ansible-playbook 1-dependencies.yml+
\item[\faSquareO] Enable Docker: \verb+systemctl enable docker+ and reboot the VM
\item[\faSquareO] Check docker, \verb+docker run hello-world+
\item[\faSquareO] Clone the SELKS repository:\\
\verb+git clone https://github.com/StamusNetworks/SELKS.git+
\item[\faSquareO] Go into this and run docker-compose as described in the instructions:\\
\url{https://github.com/StamusNetworks/SELKS/wiki/Docker}\\
{\bf  make sure to select the right network interface, so Suricata can sniff packets}
{\bf I did NOT install Portainer}
\item[\faSquareO] Use a browser to access the platform on \url{https://127.0.0.1}
\item[\faSquareO] Relax
\end{list1}

This will provide a basic Elasticsearch version 7, with Kibana and Suricata

\eject

Basic Debian with Sudo:
\hlkimage{12cm}{image/selks-0.png}

Install git and ansible:
\hlkimage{12cm}{image/selks-1.png}
\eject

Git clone kramse-labs:
\hlkimage{12cm}{image/selks-2.png}

Use Ansible to install Docker:
\hlkimage{12cm}{image/selks-3.png}
\eject

Wait for docker to be installed:
\hlkimage{12cm}{image/selks-4.png}

Enable it for reboot and reboot:
\hlkimage{12cm}{image/selks-5.png}
\eject

Check docker -- if it only works for root it is also OK to use that:
\hlkimage{12cm}{image/selks-6.png}

Git clone SELKS repository:
\hlkimage{12cm}{image/selks-7.png}
\eject

Run the ./easy-setup script:
\hlkimage{12cm}{image/selks-8.png}

Answer questions about network interface
\hlkimage{12cm}{image/selks-9.png}
\eject

Docker will start fetching the images -- took about 5 minutes on 4G router:
\hlkimage{12cm}{image/selks-10.png}

\hlkimage{12cm}{image/selks-11.png}
\eject

Start the docker containers:
\hlkimage{12cm}{image/selks-13.png}


%\hlkimage{12cm}{image/selks-14.png}

Start a browser and accept the self-signed certificate:
\hlkimage{12cm}{image/selks-15.png}

\eject

Success -- hopefully, login with username: \verb+selks-user+ and password: \verb+selks-user+:
\hlkimage{12cm}{image/selks-16.png}

After browsing to a few sites::
\hlkimage{12cm}{image/selks-17.png}




\end{document}
