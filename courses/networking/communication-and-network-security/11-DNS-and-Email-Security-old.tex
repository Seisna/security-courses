\documentclass[Screen16to9,17pt]{foils}
\usepackage{zencurity-slides}

\externaldocument{communication-and-network-security-exercises}
\selectlanguage{english}

\begin{document}

\mytitlepage
{11. DNS and Email Security}
{Communication and Network Security \the\year}




\slide{Goals for today}

%\hlkimage{8cm}{}

Todays goals:
\begin{list2}
\item Talk about DNS and email standards
\item We already discussed DNSSEC
\item Try running DNS servers!
\end{list2}

Trying to make today less heavy with information.


\slide{Plan for today}

\begin{list1}
\item Subjects
\begin{list2}
\item DNS introduction
\item SMTP introduction
\item SMTP TLS
\item SPF, DKIM, DMARC
\item DNSSEC - DNS integrity
\item DNS over TLS vs DNS over HTTPS - DNS encryption

Exercises
\end{list2}
\item Exercises
\begin{list2}
\item Check some examples like how danish banks are using DMARC, and how your own companies can start using it
\item SSLscan with SMTP TLS
\item Run a DNS server Unbound
\end{list2}
\end{list1}


\slide{Reading Summary}

\begin{list1}
\item Re-read PPA DNS pages 173-183
\item \link{https://en.wikipedia.org/wiki/Sender_Policy_Framework}
\item \link{https://en.wikipedia.org/wiki/DMARC}
\item \link{https://en.wikipedia.org/wiki/DomainKeys_Identified_Mail}
\item ANSM chapter 14,15 - 66 pages
\end{list1}




\slide{Fokus \the\year: DNS og email}

%\hlkrightpic{6cm}{0cm}{brian-patrick-tagalog-680954-unsplash.jpg}
{~}

\begin{list2}
\item Vi er afhængige af email, modtagelse og afsendelse
\item Når vi modtager skal det helst gå hurtigt
\item Når vi sender skal vi ikke ende i spam mappen
\item Phishing, hvem kan sende \emph{fra vores domæne}
\end{list2}


\slide{Various key attack types, clients and employees}

\begin{list2}
\item Phishing - sending fake emails, to collect credentials
\item Spear phishing - targetted attacks
\item Person in the middle - sniffing and changing data in transit
\item Drive-by attacks - web pages infected with malware, often ad servers
\item Malware transferred via USB or email
\item Credential Stuffing, Password related, like re-use of password, see slide about being pwned
\end{list2}

\vskip 1cm
Hackers try to create "urgency", click this or loose money


\slide{Domain Name System}

\hlkimage{10cm}{dns-1.pdf}

\begin{list1}
\item Gennem DHCP får man typisk også information om DNS servere
\item En DNS server kan slå navne, domæner og adresser op
\item Foregår via query og response med datatyper kaldet resource records
\item DNS er en distribueret database, så opslag kan resultere i flere opslag
\end{list1}

\slide{DNS systemet}

\begin{list1}
\item navneopslag på Internet
\item tidligere brugte man en {\bfseries hosts} fil\\
hosts filer bruges stadig lokalt til serveren - IP-adresser
\item UNIX: /etc/hosts
\item Windows \verb+c:\windows\system32\drivers\etc\hosts+
\item Eksempel: www.zencurity.com har adressen 185.129.60.130
\item skrives i database filer, zone filer
\end{list1}

\begin{alltt}
ns1     IN      A       185.129.60.130
        IN      AAAA    2a06:d380:0:3065::53
www     IN      A       185.129.60.130
        IN      AAAA    2a06:d380:0:3065::80
\end{alltt}


\slide{DNS er mere end navneopslag}

\begin{list1}
  \item består af resource records med en type:
    \begin{list2}
\item adresser A-records
\item IPv6 adresser AAAA-records
\item autoritative navneservere NS-records
\item post, mail-exchanger MX-records
\item flere andre: md ,  mf ,  cname ,  soa ,
                  mb , mg ,  mr ,  null ,  wks ,  ptr ,
                  hinfo ,  minfo ,  mx ....
\end{list2}
\end{list1}
\begin{alltt}
        IN      MX      10      mail.zencurity.dk.
        IN      MX      20      mail2.zencurity.dk.
\end{alltt}

\slide{Basal DNS opsætning}

\begin{alltt}
domain zencurity.net
nameserver 91.239.100.100
nameserver 2001:67c:28a4::
nameserver 89.233.43.71
nameserver 2a01:3a0:53:53::
\end{alltt}

\begin{list1}
\item \verb+/etc/resolv.conf+ angiver navneservere og søgedomæner
\item typisk indhold er domænenavn og IP-adresser for navneservere
\item Filen opdateres også automatisk på DHCP klienter
\item {\bf Husk at man godt kan slå AAAA records op over IPv4}
\item De viste servere er fra censurfridns.dk og kan benyttes frit
\end{list1}

\slide{DNS root servere}

\hlkimage{20cm}{root-servers.png}

\link{http://root-servers.org/}

\slide{DK-hostmaster}

\begin{list1}
\item bestyrer .dk TLD - top level domain

\item man registrerer ikke .dk-domæner hos DK-hostmaster, men hos en
  registrator
\item Et domæne bør have flere navneservere og flere postservere
\item autoritativ navneserver - ved autoritativt om IP-adresse for
  maskine.domæne.dk findes
\item ikke-autoritativ - har på vegne af en klient slået en adresse op
\item Det anbefales at overveje en service som
  \link{http://www.gratisdns.dk} der har flere navneservere distribueret
  over stor geografisk afstand
\end{list1}



\slide{DNS problems}

\begin{quote}
The Domain Name System (DNS) [32][33] provides for a distributed database mapping host names to IP
addresses. An intruder who interferes with the proper operation of the DNS can mount a variety of
attacks, including denial of service and password collection. There are a number of vulnerabilities.
\end{quote}

\begin{list1}
\item We have a lot of the same problems in DNS today
\item Plus some more caused by middle-boxes, NAT, DNS size, DNS inspection
\begin{list2}
\item DNS must allow both UDP and TCP port 53
\item Your DNS servers must have updated software, see DNS flag day\\ https://dnsflagday.net/ after which kludges will be REMOVED!
\item DNS is unencrypted
\end{list2}
\end{list1}



\slide{DNS over TLS vs DNS over HTTPS - DNS encryption}

\begin{list2}
\item Protocols exist that encrypt DNS data, like dnscrypt which is not RFC\\ standard \link{https://dnscrypt.info/} \link{https://en.wikipedia.org/wiki/DNSCrypt}
\item Today we have competing standards:
\item
\emph{Specification for DNS over Transport Layer Security (TLS)} (DoT), RFC 7858 MAY 2016\\
\link{https://en.wikipedia.org/wiki/DNS_over_TLS}

\item \emph{DNS Queries over HTTPS (DoH)} RFC 8484

\item How to cofigure DoT \link{https://dnsprivacy.org/wiki/display/DP/DNS+Privacy+Clients}
\end{list2}

\slide{DNS attacks, Your registrar}

\hlkimage{10cm}{krebs-lenovo-google-dns-hack.png}
\begin{list1}
%\item DNS is the Domain Name System, \link{https://en.wikipedia.org/wiki/Dns}
\item DNS insecurity has huge impact on your security!
\item Most are denial of service, by may create Mitm or confidentiality concerns
\item Select DNS providers with care
\end{list1}


Sources:\\
{\tiny
\link{https://krebsonsecurity.com/2015/02/webnic-registrar-blamed-for-hijack-of-lenovo-google-domains/}\\
\link{http://www.version2.dk/artikel/google-og-lenovo-defaced-som-foelge-af-overset-sikkerhedsproblemstilling-91295}}


\slide{DNSSEC get started now}

\hlkimage{12cm}{cz-nic-dnssec-tlsa-validator.png}

\begin{quote}
"TLSA records store hashes of remote server TLS/SSL certificates. The authenticity of a TLS/SSL certificate for a domain name is verified by DANE protocol (RFC 6698). DNSSEC and TLSA validation results are displayer by using several icons."
\end{quote}


\slide{DNSSEC and DANE}

\begin{quote}
"Objective:

Specify mechanisms and techniques that allow Internet applications to
establish cryptographically secured communications by using information
distributed through DNSSEC for discovering and authenticating public
keys which are associated with a service located at a domain name."
\end{quote}

\begin{list1}
\item DNS-based Authentication of Named Entities (dane)
\end{list1}

\slide{Email security \the\year - Goals}

\begin{list2}
\item SPF Sender Policy Framework\\ {\footnotesize\link{https://en.wikipedia.org/wiki/Sender_Policy_Framework}}
\item DKIM DomainKeys Identified Mail\\
{\footnotesize\link{https://en.wikipedia.org/wiki/DomainKeys_Identified_Mail}}
\item DMARC Domain-based Message Authentication, Reporting and Conformance\\
{\footnotesize\link{https://en.wikipedia.org/wiki/DMARC}}
\item DANE DNS-based Authentication of Named Entities\\ {\footnotesize\link{https://en.wikipedia.org/wiki/DNS-based_Authentication_of_Named_Entities}}
\item Brug allesammen, check efter ændringer!
\end{list2}




\slide{Hvordan bruger man IPv6}

\begin{center}
\hlkbig
\vskip 2 cm
www.zencurity.com

hlk@zencurity.com
\end{center}

DNS AAAA record tilføjes
\begin{alltt}
www     IN A    91.102.91.17
        IN AAAA 2001:16d8:ff00:12f::2

mail.zencurity.com has address 91.102.91.22
mail.zencurity.com has IPv6 address 2a02:9d0:3000:1::216
\end{alltt}




\slide{BIND DNS server}

\begin{list1}
\item Berkeley Internet Name Daemon server
\item Mange bruger BIND fra Internet Systems Consortium
   - altså Open Source
\item konfigureres gennem \verb+named.conf+
\item det anbefales at bruge BIND version 9
\end{list1}

\begin{list2}
\item Biblen omkring DNS og BIND er:\\
\emph{DNS and BIND}, Paul Albitz \& Cricket Liu, O'Reilly, 5th
  edition Maj 2006
\end{list2}

\slide{BIND konfiguration - et udgangspunkt}

\begin{alltt}\tiny
acl internals \{ 127.0.0.1; ::1; 10.0.0.0/24; \};
options \{
        // the random device depends on the OS !
        random-device "/dev/random"; directory "/namedb";
        {\bf listen-on-v6 { any; };}
        port 53; version "Dont know"; allow-query \{ any; \};
\};
view "internal" \{
   match-clients \{ internals; \}; recursion yes;
   zone "." \{
       type hint;   file "root.cache"; \};
   // localhost forward lookup
   zone "localhost." \{ type master; file "internal/db.localhost";   \};
   // localhost reverse lookup from IPv4 address
   zone "0.0.127.in-addr.arpa" \{ type master; file "internal/db.127.0.0"; notify no;   \};
...
\}
\end{alltt}

BIND is still used a lot, but I recommend Unbound for recursive servers, and outsourcing authoritative DNS


\slide{Unbound and NSD}

\begin{quote}
Unbound is a validating, recursive, caching DNS resolver. It is designed to be fast and lean and incorporates modern features based on open standards.

To help increase online privacy, Unbound supports DNS-over-TLS which allows clients to encrypt their communication. In addition, it supports various modern standards that limit the amount of data exchanged with authoritative servers.
\end{quote}

\link{https://www.nlnetlabs.nl/projects/unbound/about/}

My preferred local DNS server. We will now stop and look at this configuration file and function.

Also check out uncensored DNS and his DNS over TLS setup!\\
Even has pinning information available:\\ {\small\link{https://blog.censurfridns.dk/blog/32-dns-over-tls-pinning-information-for-unicastcensurfridnsdk/}}

\exercise{ex:dns-server-test}



\slide{DNSSEC DNS integrity}

\hlkimage{9cm}{cz-nic-dnssec-tlsa-validator.png}

\begin{quote}
"TLSA records store hashes of remote server TLS/SSL certificates. The authenticity of a TLS/SSL certificate for a domain name is verified by DANE protocol (RFC 6698). DNSSEC and TLSA validation results are displayer by using several icons."
\end{quote}

DNSSEC is something you should enable ASAP where possible

\slide{DNSSEC}

\hlkimage{20cm}{wwwdnsseckeys_02.png}

\centerline{DNSSEC - nu også i Danmark}

Du kan sikre dit domæne med DNSSEC - wooohooo!

Det betyder en tillid til DNS som muliggør alskens services.

Kilde:\\
\link{https://www.dk-hostmaster.dk/english/tech-notes/dnssec/}




\slide{SMTP Simple Mail Transfer Protocol}

\begin{alltt}\tiny
hlk@bigfoot:hlk$ telnet mail.kramse.dk 25
Connected to sunny.
220 sunny.kramse.dk ESMTP Postfix
HELO bigfoot
250 sunny.kramse.dk
MAIL FROM: Henrik
250 Ok
RCPT TO: hlk@kramse.dk
250 Ok
DATA
354 End data with <CR><LF>.<CR><LF>
hejsa
.
250 Ok: queued as 749193BD2
QUIT
221 Bye
\end{alltt}

\begin{list2}
\item RFC-821 SMTP Simple Mail Transfer Protocol fra 1982
\item \link{http://en.wikipedia.org/wiki/Simple_Mail_Transfer_Protocol}
\end{list2}



\slide{e-mail servere}

\begin{list1}
\item Example software Sendmail og Postfix
\begin{list2}
\item Sendmail - den ældste
\item Postfix en modulært og sikkerhedsmæssigt god e-mail server\\
er ligeledes nem at konfigurere
\end{list2}
\item Dertil kommer diverse andre mailservere:
\item Microsoft Exchange på Windows servere
\item Related software Roundcube, Dovecot IMAP server, Thunderbird, etc.
\item also need anti-spam solutions
\item I use Spamhaus lists, \link{https://www.spamhaus.org/}
\end{list1}


\slide{Postfix postserveren}

\hlkimage{6cm}{postfix-mouse.png}

\begin{list1}
\item Lavet af Wietse Venema for IBM
\item Nem at konfigurere og sikker
\item \verb+main.cf+ findes typisk i kataloget \verb+/etc/postfix+
\end{list1}

\slide{Audit af postservere}

\begin{list1}
\item Typisk findes konfigurationsfilerne til postservere under
  \verb+/etc+
\begin{list2}
\item \verb+/etc/mail+
\item \verb+/etc/postfix+
\end{list2}
\item Det vigtigste er at den er opdateret og IKKE tillader relaying
\item Der findes diverse test-scripts til relaycheck på internet
\item Husk også at checke domæne records, MX og A
\end{list1}

\slide{Test af e-mail server}

\begin{alltt}\tiny
[hlk]$ {\bfseries telnet localhost 25}
Connected.
Escape character is '^]'.
220 server ESMTP Postfix
{\bfseries helo test}
250 server
{\bfseries mail from: postmaster@pentest.dk}
250 Ok
{\bfseries rcpt to: root@pentest.dk}
250 Ok
{\bfseries data}
354 End data with <CR><LF>.<CR><LF>
{\bfseries skriv en kort besked}
.
250 Ok: queued as 91AA34D18
{\bfseries quit}
\end{alltt}
%$

Skal ikke tillade relaying

Idag benyttes ofte en stjålet brugerkonto med brugernavn og kodeord til at sende spam.



\slide{sslscan - can do STARTTLS with SMTP too}

\begin{alltt}\footnotesize
$ sslscan --starttls-smtp mail.kramse.org
Connected to 91.102.91.22
Testing SSL server mail.kramse.org on port 25 using SNI name mail.kramse.org
...
  Supported Server Cipher(s):
Preferred TLSv1.2  256 bits  ECDHE-RSA-AES256-GCM-SHA384   Curve 25519 DHE 253
Accepted  TLSv1.2  256 bits  DHE-RSA-AES256-GCM-SHA384     DHE 2048 bits
Accepted  TLSv1.2  256 bits  DHE-RSA-AES256-SHA256         DHE 2048 bits
Accepted  TLSv1.2  256 bits  DHE-RSA-AES256-SHA            DHE 2048 bits
Accepted  TLSv1.2  256 bits  ECDHE-RSA-CHACHA20-POLY1305   Curve 25519 DHE 253
Accepted  TLSv1.2  256 bits  DHE-RSA-CHACHA20-POLY1305     DHE 2048 bits
Accepted  TLSv1.2  256 bits  DHE-RSA-CAMELLIA256-SHA256    DHE 2048 bits
Accepted  TLSv1.2  256 bits  DHE-RSA-CAMELLIA256-SHA       DHE 2048 bits
...
Subject:  mail.kramse.org
Altnames: DNS:mail.kramse.org
Issuer:   Let's Encrypt Authority X3
Not valid before: Mar 24 15:05:20 2020 GMT
Not valid after:  Jun 22 15:05:20 2020 GMT
\end{alltt}

Source:
Originally sslscan from http://www.titania.co.uk
 but use the version on Kali


\slide{Postservere til klienter}

\begin{list1}
\item SMTP som vi har gennemgået er til at sende mail mellem servere
\item Når vi skal hente post sker det typisk med POP3 eller IMAP
\begin{list2}
\item POP3 Post Office Protocol version 3 RFC-1939
\item Internet Message Access Protocol (typisk IMAPv4) RFC-3501
\end{list2}
\item Forskellen mellem de to er at man typisk med POP3 henter posten, hvor man med IMAP lader den ligge på serveren
\item POP3 er bedst hvis kun en klient skal hente
\item IMAP er bedst hvis du vil tilgå din post fra flere systemer
\item Jeg bruger selv IMAPS, IMAP over SSL/TLS kryptering - idet kodeord ellers sendes i klartekst
\end{list1}


\slide{IMAP/POP3 i Danmark}

\hlkimage{12cm}{images/pop3-1.pdf}

\begin{list1}
\item Man har tillid til sin ISP - der administrerer såvel net som server
\end{list1}

\slide{IMAP/POP3 i Danmark - trådløst}

\hlkimage{9cm}{images/pop3-wlan.pdf}
\begin{list1}
\item Har man tillid til andre ISP'er? Alle ISP'er?
\item Deler man et netværksmedium med andre?
\item {\color{security6blue}Brug de rigtige protokoller!} POP3S, IMAPS, SMTPTLS, ...
\end{list1}

Meget lig File Transfer Protocol (FTP)

\slide{SMTP TLS}

\begin{quote}
The STARTTLS command for IMAP and POP3 is defined in RFC 2595, for SMTP in RFC 3207, for XMPP in RFC 6120 and for NNTP in RFC 4642. For IRC, the IRCv3 Working Group has defined the STARTTLS extension. FTP uses the command "AUTH TLS" defined in RFC 4217 and LDAP defines a protocol extension OID in RFC 2830. HTTP uses upgrade header.
\end{quote}

\begin{list1}
\item SMTP was extended with support for Transport Layer Security TLS
\item Also called {\bf Opportunistic TLS}, where the quote is also from:\\ \link{https://en.wikipedia.org/wiki/Opportunistic_TLS}
\end{list1}




\slide{SPF, DKIM, DMARC}

\begin{list2}
\item SPF Sender Policy Framework - which IPs can send for my domain\\ {\footnotesize\link{https://en.wikipedia.org/wiki/Sender_Policy_Framework}}
\item DKIM DomainKeys Identified Mail - when sending we use these keys\\
{\footnotesize\link{https://en.wikipedia.org/wiki/DomainKeys_Identified_Mail}}
\item DMARC Domain-based Message Authentication, Reporting and Conformance\\
{\footnotesize\link{https://en.wikipedia.org/wiki/DMARC}}\\
Who tried sending spam which does not match our SPF and DKIM, tell us!
\item DANE DNS-based Authentication of Named Entities\\ {\footnotesize\link{https://en.wikipedia.org/wiki/DNS-based_Authentication_of_Named_Entities}}\\
Put everything in DNS and secure it with DNSSEC
\item Brug allesammen, check efter ændringer!
\end{list2}

\vskip 1cm
\centerline{\hlkbig Email security \the\year - Goals}




\slidenext

\end{document}
