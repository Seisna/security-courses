\slide{Dag 4 Netværkssikkerhed og firewalls}

\hlkimage{18cm}{server-owned.pdf}



\slide{Kryptografi}

\hlkimage{18cm}{images/crypto-rot13.pdf}

\begin{list1}
\item Kryptografi er læren om, hvordan man kan kryptere data
\item Kryptografi benytter algoritmer som sammen med nøgler giver en
  ciffertekst - der kun kan læses ved hjælp af den tilhørende nøgle
\end{list1}

\slide{Public key kryptografi - 1}

\hlkimage{18cm}{images/crypto-public-key.pdf}

\begin{list1}
\item privat-nøgle kryptografi (eksempelvis AES) benyttes den samme
  nøgle til kryptering og dekryptering 
\item offentlig-nøgle kryptografi (eksempelvis RSA) benytter to
  separate nøgler til kryptering og dekryptering
\end{list1}

\slide{Public key kryptografi - 2}

\hlkimage{18cm}{images/crypto-public-key-2.pdf}

\begin{list1}

\item offentlig-nøgle kryptografi (eksempelvis RSA) bruger den private
  nøgle til at dekryptere
\item man kan ligeledes bruge offentlig-nøgle kryptografi til at
  signere dokumenter - som så verificeres med den offentlige nøgle
\end{list1}


\slide{Kryptografiske principper}

\begin{list1}
\item Algoritmerne er kendte
\item Nøglerne er hemmelige
\item Nøgler har en vis levetid - de skal skiftes ofte
\item Et successfuldt angreb på en krypto-algoritme er enhver genvej
  som kræver mindre arbejde end en gennemgang af alle nøglerne 
\item Nye algoritmer, programmer, protokoller m.v. skal gennemgås nøje!
\item Se evt. Snake Oil Warning Signs:
Encryption Software to Avoid 
\link{http://www.interhack.net/people/cmcurtin/snake-oil-faq.html}
\end{list1}

\slide{DES, Triple DES og AES}

\hlkimage{15cm}{images/AES_head.png}

\begin{list1}
\item DES kryptering baseret på den IBM udviklede Lucifer algoritme
  har været benyttet gennem mange år. 
\item Der er vedtaget en ny standard algoritme Advanced Encryption
  Standard (AES) som afløser Data Encryption Standard (DES)
\item Algoritmen hedder Rijndael og er udviklet
af Joan Daemen og Vincent Rijmen.
%\item \emph{Rijndael is available for free. You can use it for
%whatever purposes  you want, irrespective of whether
%it is accepted as AES or not.}

\item Kilde:
\link{http://csrc.nist.gov/encryption/aes/}\\
\href{http://www.esat.kuleuven.ac.be/~rijmen/rijndael/}
{http://www.esat.kuleuven.ac.be/\~{}rijmen/rijndael/}
\end{list1}


\slide{Formålet med kryptering}

\vskip 3 cm
\centerline{\hlkbig kryptering er den eneste måde at sikre:}
\vskip 3 cm
\centerline{\hlkbig fortrolighed}
\vskip 3 cm
\centerline{\hlkbig autenticitet / integritet}


\slide{e-mail og forbindelser}

\begin{list1}
\item Kryptering af e-mail
\begin{list2}
\item Pretty Good Privacy - Phil Zimmermann
\item PGP = mail sikkerhed
\end{list2}
\item Kryptering af sessioner SSL/TLS
\begin{list2}
\item Secure Sockets Layer SSL / Transport Layer Services TLS
\item krypterer data der sendes mellem webservere og klienter
\item SSL kan bruges generelt til mange typer sessioner, eksempelvis
  POP3S, IMAPS, SSH m.fl.
\end{list2}
\vskip 1 cm 
\item Sender I kreditkortnummeret til en webserver der kører uden https?
\end{list1}

\slide{MD5 message digest funktion}

\hlkimage{16cm}{images/message-digest-1.pdf}

\begin{list1}
\item HASH algoritmer giver en unik værdi baseret på input
%\item output fra algoritmerne kaldes også message digest
%\item MD5 er et eksempel på en meget brugt algoritme
%\item MD5 algoritmen har følgende egenskaber:
%  \begin{list2}
%  \item output er 128-bit "fingerprint" uanset længden af input
\item værdien ændres radikalt selv ved små ændringer i input
%  \end{list2}
\item MD5 er blandt andet beskrevet i RFC-1321: The MD5 Message-Digest
  Algorithm 
%\item Algoritmen MD5 er baseret på MD4, begge udviklet af Ronald
%  L. Rivest kendt fra blandt andet RSA Data Security, Inc
\item Både MD5 og SHA-1 undersøges nøje og der er fundet kollisioner
  som kan påvirke vores brug i fremtiden - \emph{stay tuned}
\end{list1} 

%%% Local Variables: 
%%% mode: latex
%%% TeX-master: "tcpip-security"
%%% End: 



\slide{kryptering, OpenPGP}

\begin{list1}
  \item kryptering er den eneste måde at sikre:
    \begin{list2}
      \item fortrolighed
      \item autenticitet
    \end{list2}
\item kryptering består af:
  \begin{list2}
    \item Algoritmer - eksempelvis RSA
    \item \emph{protokoller} - måden de bruges på
\item programmer - eksempelvis PGP
\end{list2}
\item fejl eller sårbarheder i en af komponenterne kan formindske
  sikkerheden  
\item PGP = mail sikkerhed, se eksempelvis Enigmail plugin til Mozilla Thunderbird

\end{list1}

\slide{PGP/GPG verifikation af integriteten}

\begin{list1}
\item Pretty Good Privacy PGP
\item Gnu Privacy Guard GPG
\item Begge understøtter OpenPGP - fra IETF RFC-2440
\item Når man har hentet og verificeret en nøgle kan man fremover nemt
checke integriteten af software pakker
\end{list1}


\begin{alltt}
\small
hlk@bigfoot:postfix$ gpg --verify  postfix-2.1.5.tar.gz.sig
gpg: Signature made Wed Sep 15 17:36:03 2004 CEST using RSA key ID D5327CB9
gpg: Good signature from "wietse venema <wietse@porcupine.org>"
gpg:                 aka "wietse venema <wietse@wzv.win.tue.nl>"  
\end{alltt}
%$

\slide{Make and install programs from source}

\begin{list1}
\item Mange open source programmer kommer som en tar-fil  
\item De fleste C programmer benytter sig så af følgende kommando
\begin{list2}
\item konfigurer softwaren - undersøg hvilket operativsystem det er
\item byg software ved hjælp af en Makefile - kompilerer og linker
\item installer software - ofte i \verb+/usr/local/bin+    
\end{list2}
\end{list1}

\begin{alltt}
./configure;make;make install  
\end{alltt}

\slide{SSL og TLS}

\hlkimage{18cm}{ehandel-https.pdf}

\begin{list1}
\item Oprindeligt udviklet af Netscape Communications Inc.
\item Secure Sockets Layer SSL er idag blevet adopteret af IETF og kaldes
derfor også for Transport Layer Security TLS
TLS er baseret på SSL Version 3.0
\item RFC-2246 The TLS Protocol Version 1.0 fra Januar 1999
\end{list1}

\slide{SSL/TLS udgaver af protokoller}
\hlkimage{16cm}{imap-ssl.png}

\begin{list1}
\item Mange protokoller findes i udgaver hvor der benyttes SSL
\item HTTPS vs HTTP
\item IMAPS, POP3S, osv.
\item Bemærk: nogle protokoller benytter to porte IMAP 143/tcp vs IMAPS 993/tcp
\item Andre benytter den samme port men en kommando som starter:
\item SMTP STARTTLS RFC-3207
\end{list1}

\slide{Secure Shell - SSH og SCP}

%\begin{center}
%\colorbox{white}{\includegraphics[width=12cm]{images/tshirt-9b.jpg}}  
%\end{center}

\hlkimage{16cm}{images/openssh-banner.png}

\begin{list1}
\item Hvad er Secure Shell SSH?  
\item Oprindeligt udviklet af Tatu Ylönen i Finland,\\
se \link{http://www.ssh.com}
\item SSH afløser en række protokoller som er usikre:
  \begin{list2}
  \item Telnet til terminal adgang
  \item r* programmerne, rsh, rcp, rlogin, ...
  \item FTP med brugerid/password
  \end{list2}
\end{list1}


\slide{SSH - de nye kommandoer er}
\begin{list1}
\item kommandoerne er:
\begin{list2}
  \item ssh - Secure Shell
  \item scp - Secure Copy
  \item sftp - secure FTP 
  \end{list2}
\item Husk: SSH er både navnet på protokollerne - version 1 og 2 samt
  programmet \verb+ssh+ til at logge ind på andre systemer
\item SSH tillader også port-forward, tunnel til usikre protokoller,
  eksempelvis X protokollen til UNIX grafiske vinduer
\item {\bfseries NB: Man bør idag bruge SSH protokol version 2!}
\end{list1}


\slide{SSH nøgler}

I praksis benytter man nøgler fremfor kodeord
\begin{list1}
\item I kan lave jeres egne SSH nøgler med programmerne i Putty
\item Hvilken del skal jeg have for at kunne give jer adgang til en
  server?
\item Hvordan får jeg smartest denne nøgle?
\end{list1}

\slide{Installation af SSH nøgle}
\begin{list1}
\item Vi bruger login med password på kurset, men for
  fuldstændighedens skyld beskrives her hvordan nøgle installeres:

\begin{list2}
\item først skal der genereres et nøglepar {\bfseries id\_dsa og id\_dsa.pub}
\item Den offentlige del, filen id\_dsa.pub, kopieres til serveren
\item Der logges ind på serveren 
\item Der udføres følgende kommandoer:
\end{list2}
\end{list1}
\begin{alltt}
$ cd  \emph{skift til dit hjemmekatalog}
$ mkdir .ssh  \emph{lav et katalog til ssh-nøgler}
$ cat id\_dsa.pub >> .ssh/authorized\_keys  \emph{kopierer nøglen}
$ chmod -R go-rwx .ssh  \emph{skift rettigheder på nøglen}
\end{alltt}


\slide{OpenSSH konfiguration}

\begin{list1}
\item Sådan anbefaler jeg at konfigurere OpenSSH SSHD
\item Det gøres i filen \verb+sshd_config+ typisk \verb+/etc/ssh/sshd_config+  
\end{list1}

\begin{alltt}
\small
Port 22780
Protocol 2

PermitRootLogin no
PubkeyAuthentication yes
AuthorizedKeysFile      .ssh/authorized_keys
# To disable tunneled clear text passwords, change to no here!
PasswordAuthentication no

#X11Forwarding no
#X11DisplayOffset 10
#X11UseLocalhost yes
\end{alltt}

Det er en smagssag om man vil tillade \emph{X11 forwarding}



\slide{VLAN Virtual LAN}

\hlkimage{10cm}{vlan-portbased.pdf}

\begin{list1}
\item Nogle switche tillader at man opdeler portene
\item Denne opdeling kaldes VLAN og portbaseret er det mest simple
\item Port 1-4 er et LAN
\item De resterende er et andet LAN
\item Data skal omkring en firewall eller en router for at krydse fra VLAN1 til VLAN2
\end{list1}

\slide{IEEE 802.1q}

\hlkimage{20cm}{vlan-8021q.pdf}

\begin{list1}
\item Nogle switche tillader at man opdeler portene, men tillige benytter 802.1q
\item Med 802.1q tillades VLAN tagging på Ethernet niveau
\item Data skal omkring en firewall eller en router for at krydse fra VLAN1 til VLAN2
\item VLAN trunking giver mulighed for at dele VLANs ud på flere switches
\item Der findes administrationsværktøjer der letter dette arbejde: OpenNAC FreeNAC, Cisco VMPS
\end{list1}





\slide{IEEE 802.1x  Port Based Network Access Control}

\hlkimage{15cm}{osx-8021x.png}

\begin{list1}
\item Nogle switche tillader at man benytter 802.1x
\item Denne protokol sikrer at man valideres før der gives adgang til porten
\item Når systemet skal have adgang til porten afleveres brugernavn og kodeord/certifikat
\item Denne protokol indgår også i WPA Enterprise
\end{list1}


\slide{802.1x og andre teknologier}

\begin{list1}
\item 802.1x i forhold til MAC filtrering giver væsentlige fordele
\item MAC filtrering kan spoofes, hvor 802.1x kræver det rigtige kodeord
\item Typisk benyttes RADIUS og 802.1x integrerer således mod både LDAP og Active Directory
\end{list1}









%XXX \slide{input fra firewallskursus}




\slide{Hvad er en firewall}

\vskip 4 cm
\centerline{\hlkbig En firewall er noget som {\color{green}blokerer}
  traffik på Internet}  

\vskip 1 cm
\pause

\centerline{\hlkbig En firewall er noget som {\color{red}tillader}
  traffik på Internet}

\slide{Firewallrollen idag}

\begin{list1}
\item Idag skal en firewall være med til at:
\begin{list2}
\item Forhindre angribere i at komme ind
\item Forhindre angribere i at sende traffik ud
\item Forhindre virus og orme i at sprede sig i netværk
\item Indgå i en samlet løsning med ISP, routere, firewalls, switchede
  strukturer, intrusion detectionsystemer samt andre dele af infrastrukturen
\end{list2}
\item Det kræver overblik!
\end{list1}


\slide{firewalls}

\begin{itemize}
\item Basalt set et netværksfilter - det yderste fæstningsværk
\item Indeholder typisk:
  \begin{list2}
   \item Grafisk brugergrænseflade til konfiguration - er det en
   fordel?
\item TCP/IP filtermuligheder - pakkernes afsender, modtager, retning
  ind/ud, porte, protokol, ...
\item Kun IPv4 for de fleste kommercielle firewalls
\item Både IPv4 og IPv6 for Open Source firewalls: IPF, OpenBSD PF,
  Linux firewalls, ...
\item Foruddefinerede regler/eksempler - er det godt hvis det er nemt
  at tilføje/åbne en usikker protokol?
\item Typisk NAT funktionalitet indbygget
\item Typisk mulighed for nogle serverfunktioner: kan agere
  DHCP-server, DNS caching server og lignende
  \end{list2}
\item En router med Access Control Lists - ACL kaldes ofte
  netværksfilter, mens en dedikeret maskine kaldes firewall -
  funktionen er reelt den samme - der filtreres trafik
\end{itemize}


\slide{Packet filtering}

\begin{alltt}
\small
0                   1                   2                   3   
0 1 2 3 4 5 6 7 8 9 0 1 2 3 4 5 6 7 8 9 0 1 2 3 4 5 6 7 8 9 0 1 
+-+-+-+-+-+-+-+-+-+-+-+-+-+-+-+-+-+-+-+-+-+-+-+-+-+-+-+-+-+-+-+-+
|Version|  IHL  |Type of Service|          Total Length         |
+-+-+-+-+-+-+-+-+-+-+-+-+-+-+-+-+-+-+-+-+-+-+-+-+-+-+-+-+-+-+-+-+
|         Identification        |Flags|      Fragment Offset    |
+-+-+-+-+-+-+-+-+-+-+-+-+-+-+-+-+-+-+-+-+-+-+-+-+-+-+-+-+-+-+-+-+
|  Time to Live |    Protocol   |         Header Checksum       |
+-+-+-+-+-+-+-+-+-+-+-+-+-+-+-+-+-+-+-+-+-+-+-+-+-+-+-+-+-+-+-+-+
|                       Source Address                          |
+-+-+-+-+-+-+-+-+-+-+-+-+-+-+-+-+-+-+-+-+-+-+-+-+-+-+-+-+-+-+-+-+
|                    Destination Address                        |
+-+-+-+-+-+-+-+-+-+-+-+-+-+-+-+-+-+-+-+-+-+-+-+-+-+-+-+-+-+-+-+-+
|                    Options                    |    Padding    |
+-+-+-+-+-+-+-+-+-+-+-+-+-+-+-+-+-+-+-+-+-+-+-+-+-+-+-+-+-+-+-+-+  
\end{alltt}

\begin{list1}
\item Packet filtering er firewalls der filtrerer på IP niveau
\item Idag inkluderer de fleste statefull inspection 
\end{list1}

\slide{Kommercielle firewalls}
\begin{list2}
\item Checkpoint Firewall-1 \link{http://www.checkpoint.com}
\item Nokia appliances - Nokia IPSO \link{http://www.nokia.com}
\item Cisco PIX \link{http://www.cisco.com}
\item Clavister firewalls \link{http://www.clavister.com}
\item Netscreen - nu ejet af Juniper
  \link{http://www.juniper.net}
\end{list2}

Ovenstående er dem som jeg oftest ser ude hos mine kunder

\slide{Open source baserede firewalls}
\begin{list2} 
\item Linux firewalls - fra begyndelsen til det nuværende netfilter
  til kerner version 2.4 og 2.6\\
\link{http://www.netfilter.org}
\item Firewall GUIs ovenpå Linux - mange! IPcop, Guarddog, Watchguard
nogle Linux firewalls er kommercielle produkter
\item IP Filter (IPF) \link{http://coombs.anu.edu.au/~avalon/}
\item OpenBSD PF - findes idag på andre operativsystemer
\link{http://www.openbsd.org} 
\item FreeBSD IPFW og IPFW2 \link{http://www.freebsd.org}
\item Mac OS X benytter IPFW
\item FreeBSD inkluderer også OpenBSD PF
\item NetBSD - bruger IPF og er ved at inkludere OpenBSD PF
\end{list2}

NB: kun eksempler og dem jeg selv har brugt


\slide{Hardware eller software}


\begin{list1}
\item Man hører indimellem begrebet \emph{hardware firewall}  
\item Det er dog et faktum at en firewall består af:
\begin{list2}
\item Netværkskort - som er hardware
\item Filtreringssoftware - som er \emph{software}!    
\end{list2}
\item Det giver ikke mening at kalde en Zyxel 10 en hardware firewall
  og en Soekris med OpenBSD for en software firewall!
\item Det er efter min mening et marketingtrick
\vskip 1 cm
\item Man kan til gengæld godt argumentere for at en dedikeret
  firewall som en separat enhed kan give bedre sikkerhed
\end{list1}

\slide{TCP three way handshake}

\hlkimage{7cm}{images/tcp-three-way.pdf}

\begin{list2}
\item {\bfseries TCP SYN half-open} scans
\item Tidligere loggede systemer kun når der var etableret en fuld TCP
  forbindelse - dette kan/kunne udnyttes til \emph{stealth}-scans
\item Hvis en maskine modtager mange SYN pakker kan dette fylde
  tabellen over connections op - og derved afholde nye forbindelser
  fra at blive oprette - {\bfseries SYN-flooding}
\end{list2}

\slide{firewall regelsæt eksempel}

\begin{alltt}
\tiny 
# hosts
router="217.157.20.129"
webserver="217.157.20.131"
# Networks
homenet="{ 192.168.1.0/24, 1.2.3.4/24 }"
wlan="10.0.42.0/24"
wireless=wi0

# things not used
spoofed="{ 127.0.0.0/8, 172.16.0.0/12, 10.0.0.0/16, 255.255.255.255/32 }"

block in all # default block anything
# loopback and other interface rules
pass out quick on lo0 all
pass in quick on lo0 all

# egress and ingress filtering - disallow spoofing, and drop spoofed
block in quick from $spoofed to any
block out quick from any to $spoofed

pass in on $wireless proto tcp from $wlan to any port = 22
pass in on $wireless proto tcp from $homenet to any port = 22
pass in on $wireless proto tcp from any to $webserver port = 80

pass out quick proto tcp  from $homenet to any flags S/S keep state
pass out quick proto udp  from $homenet to any         keep state
pass out quick proto icmp from $homenet to any         keep state
\end{alltt}
%$



\slide{netdesign - med firewalls - 100\% sikkerhed?}

\begin{center}
\colorbox{white}{\includegraphics[width=12cm]{images/kut.jpg}}  
\end{center}

\begin{list1}
\item Hvor skal en firewall placeres for at gøre størst nytte?
\item Hvad er forudsætningen for at en firewall virker?\\
At der er konfigureret et sæt fornuftige regler!
\item Hvor kommer reglerne fra? Sikkerhedspolitikken!
%\item Kan man lave en 100\% sikker firewall? Ja selvfølgelig, se!
\end{list1}


\centerline{\small Kilde: Billedet er fra Marcus Ranum The ULTIMATELY
  Secure Firewall} 


\slide{Firewall er ikke alene}

\begin{list1}
\item Firewalls er ikke alene
\begin{list2}
\item anti-virus på klienter og postsystemer
\item IDS systemer
\item Backupsystemer
\item Adgangskontrol
\item ... mange andre ting er mindst ligeså vigtige
\end{list2}
\end{list1}

\centerline{\hlkbig Forsvaret er som altid - flere lag af sikkerhed! }


\slide{Firewall historik}


\hlkimage{6cm}{images/cheswick-cover2e.jpg}

\begin{list1}
\item Firewalls har været kendt siden starten af 90'erne
\item Den første bog \emph{Firewalls and Internet Security} udkom i
  1994 men der findes mange akademiske artikler om firewalls 
\item Bogen \emph{Firewalls and Internet Security} anbefales,  
William R. Cheswick, Steven M. Bellovin, Aviel D. Rubin,
Addison-Wesley, 2nd edition, 2003  
\end{list1}

\slide{An Evening with Berferd}


\begin{list1}
\item Artikel om en hacker der lokkes, vurderes, overvåges
\item Et tidligt eksempel på en honeypot
\item Idag anbefales The Honeynet Project hvis man vil vide mere
\\\link{http://www.honeynet.org}
\end{list1}




\slide{m0n0wall}

\hlkimage{20cm}{images/m0n0wall-1.pdf}


\slide{First or Last match firewall?}

\hlkimage{20cm}{images/first-last-match-1.pdf}


\slide{firewall koncepter}

\begin{list1}
\item Rækkefølgen af regler betyder noget!
\begin{list2}
\item To typer af firewalls:
 First match - når en regel matcher, gør det som angives block/pass
 Last match  - marker pakken hvis den matcher, til sidst afgøres block/pass
\end{list2}
\item {\bf Det er ekstremt vigtigt at vide hvilken type firewall
    man bruger!} 
\item OpenBSD PF er last match
\item FreeBSD IPFW er first match  
\item Linux iptables/netfilter er last match
\end{list1}

\slide{First or Last match firewall?}

\hlkimage{20cm}{images/first-last-match-1.pdf}
\begin{list2}
\item To typer af firewalls:
 First match - eksempelvis IPFW,
 Last match - eksempelvis PF
%\item {\bf Det er ekstremt vigtigt at vide hvilken type firewall
%    man bruger!} 
\end{list2}


\slide{First match - IPFW}

\begin{alltt}
\hlksmall
00100 16389  1551541 allow ip from any to any via lo0
00200     0        0 deny log ip from any to 127.0.0.0/8
00300     0        0 check-state
...
{\bfseries 
65435    36     5697 deny log ip from any to any}
65535   865    54964 allow ip from any to any
\end{alltt}

\vskip 2 cm

\centerline{Den sidste regel nås aldrig!}

\slide{Last match - OpenBSD PF}

\begin{alltt}
\small
ext_if="ext0"
int_if="int0"

block in
pass out keep state

pass quick on \{ lo $int_if \}

# Tillad forbindelser ind på port 80=http og port 53=domain
# på IP-adressen for eksterne netkort ($ext_if) syntaksen
pass in on $ext_if proto tcp to ($ext_if) port http keep state
pass in on $ext_if proto \{ tcp, udp \} to ($ext_if) port domain keep state
\end{alltt}

\vskip 2 cm
\centerline{Pakkerne markeres med block eller pass indtil sidste
  regel}
\centerline{nøgleordet \emph{quick} afslutter match - god til store
  regelsæt} 

\slide{Linux iptables/netfilter eksempel}

\begin{alltt}
\footnotesize
# Firewall configuration written by system-config-securitylevel
# Manual customization of this file is not recommended.
*filter
:INPUT ACCEPT [0:0]
:FORWARD ACCEPT [0:0]
:OUTPUT ACCEPT [0:0]
:RH-Firewall-1-INPUT - [0:0]
-A INPUT -j RH-Firewall-1-INPUT
-A FORWARD -j RH-Firewall-1-INPUT
-A RH-Firewall-1-INPUT -i lo -j ACCEPT
-A RH-Firewall-1-INPUT -p icmp --icmp-type any -j ACCEPT
-A RH-Firewall-1-INPUT -p 50 -j ACCEPT
-A RH-Firewall-1-INPUT -p 51 -j ACCEPT
-A RH-Firewall-1-INPUT -p udp --dport 5353 -d 224.0.0.251 -j ACCEPT
-A RH-Firewall-1-INPUT -p udp -m udp --dport 631 -j ACCEPT
-A RH-Firewall-1-INPUT -m state --state ESTABLISHED,RELATED -j ACCEPT
-A RH-Firewall-1-INPUT -m state --state NEW -m tcp -p tcp --dport 443 -j ACCEPT
-A RH-Firewall-1-INPUT -m state --state NEW -m tcp -p tcp --dport 22 -j ACCEPT
-A RH-Firewall-1-INPUT -j REJECT --reject-with icmp-host-prohibited
COMMIT
\end{alltt}

\centerline{NB: husk at aktivere IP forwarding}

\slide{Firewall GUI}

\hlkimage{24cm}{images/fwbuilder-screenshot1.png}

\begin{list1}
\item Der findes mange GUI programmer til Open Source firewalls
\end{list1}

Kilde: billede fra \link{http://www.fwbuilder.org}


\slide{m0n0wall}

\hlkimage{20cm}{images/m0n0wall-1.pdf}

Kilde: billede fra \link{http://m0n0.ch/wall/}

\slide{Firewalls og ICMP}


\begin{alltt}
ipfw add allow icmp from any to any icmptypes 3,4,11,12
\end{alltt}

\begin{list1}
\item Ovenstående er IPFW syntaks for at tillade de interessant ICMP beskeder igennem
\item Tillad ICMP types:
\begin{list2}
\item 3 Destination Unreachable
\item 4 Source Quench Message
\item 11 Time Exceeded
\item 12 Parameter Problem Message
\end{list2}
\end{list1}

\slide{Firewall konfiguration}

\begin{list1}
\item Den bedste firewall konfiguration starter med:
\begin{list2}
\item Papir og blyant
\item En fornuftig adressestruktur
\end{list2}
\item Brug dernæst en firewall med GUI første gang!
\item Husk dernæst:
\begin{list2}
\item En firewall skal passes
\item En firewall skal opdateres
\item Systemerne bagved skal hærdes!    
\end{list2}
\end{list1}

\slide{Bloker indefra og ud}

\begin{list1}
\item Der er porte og services som altid bør blokeres
\item Det kan være kendte sårbare services
\begin{list2}
\item Windows SMB filesharing - ikke til brug på Internet!
\item UNIX NFS - ikke til brug på Internet!
\end{list2}
\item Kendte problemer:
\begin{list2}
\item KaZaA og andre P2P programmer - hvis muligt!
\item Portmapper - port 111    
\end{list2}
\end{list1}

\slide{Firewall konfiguration}

\begin{list1}
\item Den bedste firewall konfiguration starter med:
\begin{list2}
\item Papir og blyant
\item En fornuftig adressestruktur
\end{list2}
\item Brug dernæst en firewall med GUI første gang!
\item Husk dernæst:
\begin{list2}
\item En firewall skal passes
\item En firewall skal opdateres
\item Systemerne bagved skal hærdes!    
\end{list2}
\end{list1}


\slide{En typisk firewall konfiguration}

\hlkimage{22cm}{images/firma-netvaerk.pdf}

\centerline{Opdeling i separate netværkssegmenter!}

\slide{personlige firewalls}

\begin{list1}
\item Personlige firewalls:  

\begin{list2}
\item Microsoft Windows XP
\item ZoneAlarm \link{http://www.zonelabs.com}  
\end{list2}
\item Personlige firewalls til Microsoft Windows inkluderer ofte
blokering af hvilket programmer der må tilgå netværk
\end{list1}

\centerline{\color{titlecolor}Det anbefales at bruge en personlig firewall}

Note: Lad være med at stille spørgsmål om logfilen i diverse fora!

{\bfseries Hvis du ikke forstår loggen så lad den ligge!}




\slide{Firewallværktøjer}
% måske til reference afsnit?

\begin{list1}
\item Der benyttes på kurset en del værktøjer:
\begin{list2}
\item nmap - \link{http://www.insecure.org} portscanner
\item Nessus - \link{http://www.nessus.org} automatiseret testværktøj
%\item libnet m.fl. - \link{http://www.packetfactory.net} - diverse projekter
%  relateret til pakker og IP netværk
%\item l0phtcrack - \link{http://www.atstake.com/research/lc/} - The Password
%  Auditing and Recovery Application
\item Ethereal - \link{http://www.ethereal.com} avanceret netværkssniffer
%\item F.I.R.E -  \link{http://biatchux.dmzs.com/} - en cd-rom der indeholder en 
%  bootable Linux del.
\item OpenBSD - \link{http://www.openbsd.org} operativsystem med fokus
  på sikkerhed 
\item m0n0wall - \link{http://www.m0n0.ch} gratis firewall baseret på FreeBSD

%\item \link{http://www.isecom.org/} - Open Source Security Testing
%  Methodology Manual - gennemgang af elementer der bør indgå i en struktureret test 
\end{list2}
\end{list1}

\slide{Specielle features}

\begin{list2}
\item Network Address Translation - NAT
\item IPv6 funktionalitet

\item Båndbredde håndtering
\item VLAN funktionalitet - mere udbredt i forbindelse med VoIP
\item Redundante firewalls - pfsync og CARP
% pfsync giver et indblik i hvordan den slags kan laves, hvor de
% kommercielle ``bare kan det''
\item IPsec og Andre VPN features
\end{list2}

\slide{Proxy servers}

\begin{list1}
\item Filtrering på højere niveauer i OSI modellen er muligt
\item Idag findes proxy applikationer til de mest almindelige
  funktioner
\item Den typiske proxy er en caching webproxy der kan foretage HTTP
  request på vegne af arbejdsstationer og gemme resultatet 
\item NB: nogle protokoller egner sig ikke til proxy servere
\item SSL forbindelser til \emph{secure websites} kan per design ikke proxies
\end{list1}

\slide{IPsec og Andre VPN features}

\begin{list1}
\item De fleste firewalls giver mulighed for at lave krypterede
  tunneler
\item Nyttigt til fjernkontorer der skal have usikker traffik henover
  usikre netværk som Internet 
\item Konceptet kaldes Virtual Private Network VPN
\item IPsec er de facto standarden for VPN og beskrevet i RFC'er 
\end{list1}


\slide{IPsec}

\begin{itemize}
\item Sikkerhed i netv�rket
\item RFC-2401 Security Architecture for the Internet Protocol
\item RFC-2402 IP Authentication Header (AH)
\item RFC-2406 IP Encapsulating Security Payload (ESP)
\item RFC-2409 The Internet Key Exchange (IKE) - dynamisk keying
\item B�de til IPv4 og IPv6
\item {\bfseries MANDATORY} i IPv6! - et krav hvis man implementerer
  fuld IPv6 support
\item god pr�sentation p� \link{http://www.hsc.fr/presentations/ike/}
\item Der findes IKEscan til at scanne efter IKE
  porte/implementationer\\
\link{http://www.nta-monitor.com/ike-scan/index.htm}
\end{itemize}

\slide{IPsec er ikke simpelt!}

\hlkimage{16cm}{images/ipsec-hsc.png}
\centerline{Kilde: \link{http://www.hsc.fr/presentations/ike/}}


\slide{RFC-2402 IP AH}

\begin{alltt}
\small
    0                   1                   2                   3
    0 1 2 3 4 5 6 7 8 9 0 1 2 3 4 5 6 7 8 9 0 1 2 3 4 5 6 7 8 9 0 1
   +-+-+-+-+-+-+-+-+-+-+-+-+-+-+-+-+-+-+-+-+-+-+-+-+-+-+-+-+-+-+-+-+
   | Next Header   |  Payload Len  |          RESERVED             |
   +-+-+-+-+-+-+-+-+-+-+-+-+-+-+-+-+-+-+-+-+-+-+-+-+-+-+-+-+-+-+-+-+
   |                 Security Parameters Index (SPI)               |
   +-+-+-+-+-+-+-+-+-+-+-+-+-+-+-+-+-+-+-+-+-+-+-+-+-+-+-+-+-+-+-+-+
   |                    Sequence Number Field                      |
   +-+-+-+-+-+-+-+-+-+-+-+-+-+-+-+-+-+-+-+-+-+-+-+-+-+-+-+-+-+-+-+-+
   |                                                               |
   +                Authentication Data (variable)                 |
   |                                                               |
   +-+-+-+-+-+-+-+-+-+-+-+-+-+-+-+-+-+-+-+-+-+-+-+-+-+-+-+-+-+-+-+-+
\end{alltt}

\slide{RFC-2402 IP AH}

Indpakning - pakkerne f�r og efter Authentication Header:
\begin{alltt}
\small
                BEFORE APPLYING AH
            ----------------------------
      IPv4  |orig IP hdr  |     |      |
            |(any options)| TCP | Data |
            ----------------------------

                  AFTER APPLYING AH
            ---------------------------------
      IPv4  |orig IP hdr  |    |     |      |
            |(any options)| AH | TCP | Data |
            ---------------------------------
            |<------- authenticated ------->|
                 except for mutable fields
\end{alltt}

\slide{RFC-2406 IP ESP}

Pakkerne f�r og efter:
\begin{alltt}
\small
               BEFORE APPLYING ESP
         ---------------------------------------
   IPv6  |             | ext hdrs |     |      |
         | orig IP hdr |if present| TCP | Data |
         ---------------------------------------



               AFTER APPLYING ESP
         ---------------------------------------------------------
   IPv6  | orig |hop-by-hop,dest*,|   |dest|   |    | ESP   | ESP|
         |IP hdr|routing,fragment.|ESP|opt*|TCP|Data|Trailer|Auth|
         ---------------------------------------------------------
                                   |<---- encrypted ---->|
                               |<---- authenticated ---->|
\end{alltt}

\slide{ipsec konfigurationsfiler}

\begin{list1}
%\item Der er f�lgende dokumenter til IPsec p� websitet\\
% \link{www.security.net/courses/ipsec}:
\item Der er f�lgende filer tilg�ngelige\\
  \begin{list2}
  \item konfigurationsfiler i NetBSD/FreeBSD/Mac OS X format - med
    \verb+setkey+ kommandoen
  \item konfigurationsfil til OpenBSD server - med \verb+ipsecadm+
    kommandoen
%  \item IKE.pdf \emph{Dynamic Management of the IPsec Parameters:
%      The IKE Protocol}, fra Herve Schauer Consultants
%\item NetBSD IPsec dokumentation
%\item Cisco \emph{Introduction to IP security}
  \end{list2}
\end{list1}


\slide{IPsec setup}

%\hlkimage{}{images/}

\begin{list1}
  \item Client: Mac OS X/NetBSD/FreeBSD - samme syntaks\\
\verb+rc.ipsec.client+

\item Server: OpenBSD - bruger ipsecadm kommando\\
\verb+rc.ipsec.server+

\item �velse til l�seren: lav samme i Cisco IOS
\item Det vil ofte v�re relevant at se p� IOS og IPsec i laboratoriet
\item Dette setup n�r vi ikke at demonstrere
\end{list1}

\slide{rc.ipsec.client - client setup - adresser}

\begin{verbatim}
#!/bin/sh
# /etc/rc.ipsec.client - IPsec client configuration
# built from http://rt.fm/~jcs/ipsec_wep.phtml
# FreeBSD/NetBSD syntaks! - used on Mac OS X
# IPv4
SECSERVER=10.0.42.1
SECCLIENT=10.0.42.53
# IPv6
#SECSERVER=2001:618:433:101::1
#SECCLIENT=2001:618:433:101::153
ESPKEY=`cat ipsec.esp.key`
AHKEY=`cat ipsec.ah.key`

# Flush IPsec SAs in case we get called more than once
setkey -F
setkey -F -P
\end{verbatim}

\slide{rc.ipsec.client - client setup - SAs}

\begin{verbatim}
# Establish Security Associations
# 1000 is from the server to the client
# 1001 is from the client to the server
setkey -c <<EOF

add $SECSERVER $SECCLIENT esp 0x1000 \
-m tunnel -E blowfish-cbc 0x$ESPKEY  -A hmac-sha1 0x$AHKEY;

add $SECCLIENT $SECSERVER esp 0x1001 \
-m tunnel -E blowfish-cbc 0x$ESPKEY -A hmac-sha1 0x$AHKEY;

spdadd $SECCLIENT $SECSERVER any -P out \
ipsec esp/tunnel/$SECCLIENT-$SECSERVER/default;

spdadd $SECSERVER $SECCLIENT any -P in \
ipsec esp/tunnel/$SECSERVER-$SECCLIENT/default;
EOF
\end{verbatim}

\slide{rc.ipsec.server - server setup - adresser}

\begin{verbatim}
#!/bin/sh
#
# /etc/rc.ipsec - IPsec server configuration
# built from http://rt.fm/~jcs/ipsec_wep.phtml
# OpenBSD syntaks!
SECSERVER=10.0.42.1
SECCLIENT=10.0.42.53
#SECSERVER6=2001:618:433:101::1
#SECCLIENT6=2001:618:433:101::153

ESPKEY=`cat ipsec.esp.key`
AHKEY=`cat ipsec.ah.key`

# Flush IPsec SAs in case we get called more than once
ipsecadm flush
\end{verbatim}


\slide{rc.ipsec.server - server setup - SAs}

\begin{verbatim}
# Establish Security Associations
#
# 1000 is from the server to the client
ipsecadm new esp -spi 1000 -src $SECSERVER -dst $SECCLIENT \
-forcetunnel -enc blf -key $ESPKEY \
-auth sha1 -authkey $AHKEY

# 1001 is from the client to the server
ipsecadm new esp -spi 1001 -src $SECCLIENT -dst $SECSERVER \
-forcetunnel -enc blf -key $ESPKEY \
-auth sha1 -authkey $AHKEY
\end{verbatim}


\slide{rc.ipsec.server - server setup - flows}

\small
\begin{verbatim}
# Create flows
#
# Data going from the outside to the client
ipsecadm flow -out -src $SECSERVER -dst $SECCLIENT -proto esp \
-addr 0.0.0.0 0.0.0.0 $SECCLIENT 255.255.255.255 -dontacq
# IPv6
#ipsecadm flow -out -src $SECSERVER -dst $SECCLIENT -proto esp \
#-addr :: :: $SECCLIENT ffff:ffff:ffff:ffff:ffff:ffff:ffff:ffff -dontacq

# Data going from the client to the outside
ipsecadm flow -in -src $SECSERVER -dst $SECCLIENT -proto esp \
-addr $SECCLIENT 255.255.255.255 0.0.0.0 0.0.0.0 -dontacq
# IPv6
#ipsecadm flow -in -src $SECSERVER -dst $SECCLIENT -proto esp \
#-addr :: :: $SECCLIENT ffff:ffff:ffff:ffff:ffff:ffff:ffff:ffff -dontacq
\end{verbatim}



%%% Local Variables:
%%% mode: latex
%%% TeX-master: t
%%% End:


\slide{OpenVPN / OpenSSL VPN}

\begin{quote}
OpenVPN is a full-featured SSL VPN solution which can accomodate a
wide range of configurations, including remote access, site-to-site
VPNs, WiFi security, and enterprise-scale remote access solutions with
load balancing, failover, and fine-grained access-controls (articles)
(examples) (security overview) (non-english languages).   
\end{quote}

\begin{list1}
\item Et andet populært VPN produkt er OpenVPN
\item Bemærk dog at hvis der benyttes TCP i TCP risikerer man at støde ind i 
et problem som kaldes \emph{TCP in TCP meltdown} 
\item Kilde: \link{http://openvpn.net/}  
\end{list1}



\exercise{ex:unix-basic-firewall}










\slide{Portscan, TCP, UDP og ICMP}

Forskellen mellem TCP og UDP i forbindelse med portscan, og effekten af en firewall der dropper pakker

\slide{Basal Portscanning}

\begin{list1}
  \item Hvad er portscanning
\item afprøvning af alle porte fra 0/1 og op til 65535
\item målet er at identificere åbne porte - sårbare services
\item typisk TCP og UDP scanning
\item TCP scanning er ofte mere pålidelig end UDP scanning
\end{list1}

{\hlkbig TCP handshake er nemmere at identificere

UDP applikationer svarer forskelligt - hvis overhovedet}

\slide{TCP three way handshake}

\hlkimage{7cm}{images/tcp-three-way.pdf}

\begin{list2}
\item {\bfseries TCP SYN half-open} scans
\item Tidligere loggede systemer kun når der var etableret en fuld TCP
  forbindelse - dette kan/kunne udnyttes til \emph{stealth}-scans
\item Hvis en maskine modtager mange SYN pakker kan dette fylde
  tabellen over connections op - og derved afholde nye forbindelser
  fra at blive oprette - {\bfseries SYN-flooding}
\end{list2}


\slide{Ping og port sweep}

\begin{list1}
\item scanninger på tværs af netværk kaldes for sweeps 
\item Scan et netværk efter aktive systemer med PING
\item Scan et netværk efter systemer med en bestemt port åben
\item Er som regel nemt at opdage:
  \begin{list2}
    \item konfigurer en maskine med to IP-adresser som ikke er i brug
\item hvis der kommer trafik til den ene eller anden er det portscan
\item hvis der kommer trafik til begge IP-adresser er der nok
  foretaget et sweep - bedre hvis de to adresser ligger et stykke fra hinanden
  \end{list2}

\end{list1}

\slide{nmap port sweep efter port 80/TCP}

\begin{list1}
  \item Port 80 TCP er webservere
\end{list1}

\begin{alltt}
\small # {\bfseries nmap  -p 80 217.157.20.130/28}

Starting nmap V. 3.00 ( www.insecure.org/nmap/ )
Interesting ports on router.kramse.dk (217.157.20.129):
Port       State       Service
80/tcp     filtered    http                    

Interesting ports on www.kramse.dk (217.157.20.131):
Port       State       Service
80/tcp     open        http                    

Interesting ports on  (217.157.20.139):
Port       State       Service
80/tcp     open        http                    

\end{alltt}

\slide{nmap port sweep efter port 161/UDP}

\begin{list1}
  \item Port 161 UDP er SNMP
\end{list1}

\begin{alltt}  
\small # {\bfseries nmap -sU -p 161 217.157.20.130/28}

Starting nmap V. 3.00 ( www.insecure.org/nmap/ )
Interesting ports on router.kramse.dk (217.157.20.129):
Port       State       Service
161/udp    open        snmp                    

The 1 scanned port on mail.kramse.dk (217.157.20.130) is: closed

Interesting ports on www.kramse.dk (217.157.20.131):
Port       State       Service
161/udp    open        snmp                    

The 1 scanned port on  (217.157.20.132) is: closed
\end{alltt}

\slide{OS detection}
\begin{alltt}
\footnotesize
# nmap -O ip.adresse.slet.tet \emph{scan af en gateway}
Starting nmap 3.48 ( http://www.insecure.org/nmap/ ) at 2003-12-03 11:31 CET
Interesting ports on gw-int.security6.net (ip.adresse.slet.tet):
(The 1653 ports scanned but not shown below are in state: closed)
PORT     STATE SERVICE
22/tcp   open  ssh
80/tcp   open  http
1080/tcp open  socks
5000/tcp open  UPnP
Device type: general purpose
Running: FreeBSD 4.X
OS details: FreeBSD 4.8-STABLE
Uptime 21.178 days (since Wed Nov 12 07:14:49 2003)
Nmap run completed -- 1 IP address (1 host up) scanned in 7.540 seconds
\end{alltt}

\begin{list2}
  \item lavniveau måde at identificere operativsystemer på
\item send pakker med \emph{anderledes} indhold
\item Reference: \emph{ICMP Usage In Scanning} Version 3.0,
  Ofir Arkin\\ \link{http://www.sys-security.com/html/projects/icmp.html}
\end{list2}

\slide{Top 75 Security Tools}

\begin{list1}
%  \item I er meget ivrige efter at afprøve en masse
\item listen over 75 top security
  tools - nogle værktøjer springes over, nogle har vi brugt
\item Den er samlet af Fyodor og findes på:\\
\link{http://www.insecure.org/tools.html}
\end{list1}


\slide{Hvad skal der ske?}

\begin{list1}
\item Tænk som en hacker
\item Rekognoscering
\begin{list2}
\item ping sweep, port scan
\item OS detection - TCP/IP eller banner grab
\item Servicescan - rpcinfo, netbios, ...
\item telnet/netcat interaktion med services
\end{list2}
\item Udnyttelse/afprøvning: Nessus, nikto, exploit programs
\item Oprydning vises ikke på kurset, men I bør i praksis:
\begin{list2}
\item Lav en rapport
\item Gennemgå rapporten, registrer ændringer
\item Opdater programmer, konfigurationer, arkitektur, osv. 
\end{list2}
\item I skal jo også VISE andre at I gør noget ved sikkerheden.
\end{list1}


\exercise{ex:nmap-sweep}
\exercise{ex:nmap-portscan}
\exercise{ex:nmap-service}
\exercise{ex:nmap-os}



\slide{Firewalls og IPv6}

\begin{list1}
\item Læg mærke til forskellen mellem ARP og ICMPv6  
\item Hvis det er muligt lav een regel der tillader adgang til services uanset protokol
\item NB: husk at aktivere IP forwarding når I skal lave en firewall
\end{list1}


\slide{OpenBSD PF}
\begin{alltt}
\footnotesize
# Macros: define common values, so they can be referenced and changed easily.
int_if=vr0
ext_if=vr2
tunnel_if=gif0
table <homenet6> { 2001:16d8:ffd2:cf0f::/64 }
set skip on lo0
scrub in all
# Filtering: the implicit first two rules are
block in all
block out all
# allow ICMPv6 for NDP
pass in inet6 proto ipv6-icmp all icmp6-type neighbradv keep state
# server with configured IP address and router advertisement daemon running
pass out inet6 proto ipv6-icmp all icmp6-type routersol keep state
# client which uses autoconfiguration would use this instead
#pass in inet6 proto ipv6-icmp all icmp6-type routeradv keep state
#pass out inet6 proto ipv6-icmp all icmp6-type neighbrsol keep state
table <sixxspop> { 82.96.56.14 2001:16d8:ff00:155::1 }
pass in on $ext_if proto icmp from <sixxspop6> to ($ext_if)
pass in on $tunnel_if proto icmp6 from <sixxspop6> to any
pass in on $int_if all
pass out on $int_if all keep state
...  probably not working AS IS
\end{alltt}


\slide{Redundante firewalls}

\hlkimage{8cm}{images/pfsync-carp-1.jpg}

\begin{list2}
\item OpenBSD Common Address Redundancy Protocol CARP - både IPv4 og IPv6\\
overtagelse af adresse både IPv4 og IPv6
\item pfsync - sender opdateringer om firewall states mellem de to systemer  
\item Kilde:
\link{http://www.countersiege.com/doc/pfsync-carp/}
\end{list2}

\slide{Redundante forbindelser hardware}

\begin{alltt}
root@azumi:# cat hostname.fxp0
up
root@azumi:# cat hostname.fxp1 
up
root@azumi:# cat /etc/hostname.trunk0
trunkproto failover trunkport fxp0 trunkport fxp1
dhcp
\end{alltt}

\begin{list1}
\item OpenBSD trunk interface
\item Linux bonding, 
\item Etherchannel Cisco
\item Idag anbefales IEEE 802.3ad LACP som er en åben standard
\item \link{http://en.wikipedia.org/wiki/EtherChannel}
\end{list1}

\slide{LACP Link Aggregation Control Protocol}

\hlkimage{7cm}{lacp-1.pdf}

\begin{list1}
\item IEEE 802.3ad standardiseret bundling/failover
\item Målet er at give:
\begin{list2}
\item mere båndbredde end en enkelt port
\item failover - hvis et link falder ud
\end{list2}
\item En server med to netinterfaces kan med fordel forbindes til to porte
\item Er ikke generelt understøttet i alle operativsystemer, men det kommer
\item \link{http://en.wikipedia.org/wiki/Link_Aggregation_Control_Protocol}
\end{list1}


\slide{Redundante forbindelser IP-niveau}

\hlkimage{12cm}{router-redundancy-1.pdf}

\begin{list1}
\item HSRP Hot Standby Router Protocol, Cisco protokol, RFC-2281
\item VRRP Virtual Router Redundancy Protocol, IETF RFC-3768, åben standard - ikke fri
\item CARP Common Address Redundancy Protocol, findes på OpenBSD og FreeBSD
\item \link{http://en.wikipedia.org/wiki/Common_Address_Redundancy_Protocol}
\end{list1}








\slide{Mobile IP}

\begin{list1}
\item Mobility er ved at blive et krav, idet enheder idag er mobile
\item Specielt ønsker vi at håndholdte computere og laptops kan modtage data
\item Tidligere skiftede man blot adresse undervejs
\item Idag ønsker man at enheden kan kontaktes nemmere, selv udenfor \emph{huset}
\item RFC-3344 IP Mobility Support for IPv4
\item RFC-4721 Mobile IPv4 Challenge/Response Extensions (Revised)
\item RFC-3775
\item \link{http://en.wikipedia.org/wiki/Mobile_IP}
\item Bemærk at Mobile IP ikke altid er nødvendig eller benyttes, mange protokoller som eksempelvis POP3/IMAP virker fint ved at enheden kalder tilbage til serveren
\end{list1}

\slide{Mobile IP begreber}

\begin{list1}
\item Definitioner - fra RFC-3344:
\begin{list2}
\item Mobile Node A host or router that changes its point of attachment from one
         network or subnetwork to another. 
\item Home Agent A router on a mobile node's home network which tunnels
         datagrams for delivery to the mobile node when it is away from
         home, and maintains current location information for the mobile
         node.
\item Foreign Agent A router on a mobile node's visited network which provides
         routing services to the mobile node while registered.  The
         foreign agent detunnels and delivers datagrams to the mobile
         node that were tunneled by the mobile node's home agent.  For
         datagrams sent by a mobile node, the foreign agent may serve as
         a default router for registered mobile nodes.
\end{list2}
\item Selve funktionen:\\
   A mobile node is given a long-term IP address on a home network.
   This home address is administered in the same way as a "permanent" IP
   address is provided to a stationary host.  When away from its home
   network, a "care-of address" is associated with the mobile node and
   reflects the mobile node's current point of attachment. 
\end{list1}

\slide{Oversigt Mobile IP}

\hlkimage{23cm}{mobile-ip-1.pdf}

Se også Mobile IPv6 A short introduction \link{http://www.hznet.de/ipv6/mipv6-intro.pdf}


\slide{VoIP Voice over IP}

\begin{list1}
\item Tidligere havde vi adskilte netværk, nu samles de 
\item Idag er det meget normalt at både firmaer og private bruger IP-telefoni
\item Fordele er primært billigere og mere fleksibelt
\item Eksempler på IP telefoni:
\begin{list2}
\item Skype benytter IP, men egenudviklet protokol
\item Cisco IP-telefoner benyttes ofte i firmaer
\item Cybercity telefoni kører over IP, med analog adapter
\end{list2}
\item Det anbefales at se på Asterisk telefoniserver, hvis man har mod på det :-)
\item \link{http://www.asterisk.org/}
\end{list1}

\slide{VoIP bekymringer}

\begin{list1}
\item Der er generelt problemer med:
\begin{list2}
\item Stabilitet - quality of service, netværket skal være bygget til det
\item Sikkerhed - hvem lytter med, hvem kan afbryde forbindelsen\\
Se evt. \link{http://www.voipsa.org/}
\item Spam over VoIP, connect, send WAV fil med spam kaldes SPIT
\item Kompatabilitet - hvilke protokoller, codecs, standarder, ...
\end{list2}
\item Der er flere store spillere
\end{list1}

\slide{VoIP protokoller}

\begin{list1}
\item SIP Session Initiation Protocol, IETF standard signaleringsprotokol
\item H.323 ITU-T standard signaleringsprotokol
\item IAX Inter-Asterisk Exchange Protocol, Asterisk protokol
\item SSCP Cisco protokol
\item ZRTP Phil Zimmermann, zfone - sikker kommunikation\\ 
\link{http://zfoneproject.com/}
\end{list1}