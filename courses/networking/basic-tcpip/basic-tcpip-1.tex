\slide{Dag 1 Basale begreber og mindre netværk}


\hlkimage{22cm}{images/kursus-netvaerk.pdf}


\slide{Netværk til routning}

\hlkimage{27cm}{basic-ipv6-network.pdf}

\vskip 2cm


\slide{Internet idag}


\hlkimage{12cm}{images/server-client.pdf}

\begin{list1}
\item Klienter og servere
\item Rødder i akademiske miljøer
\item Protokoller der er op til 20 år gamle
\item Meget lidt kryptering, mest på http til brug ved e-handel 
\item Kurset omhandler udelukkende netværk baseret på IP protokollerne
\end{list1}

\slide{Internet er åbne standarder!}

{\hlkbig \color{titlecolor}
We reject kings, presidents, and voting.\\
We believe in rough consensus and running code.\\
-- The IETF credo Dave Clark, 1992.}

\begin{list1}
\item Request for comments - RFC - er en serie af dokumenter
\item RFC, BCP, FYI, informational\\
de første stammer tilbage fra 1969
\item Ændres ikke, men får status Obsoleted når der udkommer en nyere
  version af en standard
\item Standards track:\\
Proposed Standard $\rightarrow$ Draft Standard $\rightarrow$ Standard
\item  Åbne standarder = åbenhed, ikke garanti for sikkerhed
\end{list1}


\slide{Hvad er Internet}

\begin{list1}
\item Kommunikation mellem mennesker!
\item Baseret på TCP/IP
\begin{list2}
\item best effort
\item packet switching (IPv6 kalder det packets, ikke datagram)
\item forbindelsesorienteret, \emph{connection-oriented}
\item forbindelsesløs, \emph{connection-less}
\end{list2}
\end{list1}

RFC-1958:
\begin{quote}
 A good analogy for the development of the Internet is that of
 constantly renewing the individual streets and buildings of a city,
 rather than razing the city and rebuilding it. The architectural
 principles therefore aim to provide a framework for creating
 cooperation and standards, as a small "spanning set" of rules that
 generates a large, varied and evolving space of technology.
\end{quote}


\slide{IP netværk: Internettet historisk set}

\begin{list2}  
\item[1961]  L. Kleinrock, MIT packet-switching teori
\item[1962]  J. C. R. Licklider, MIT - notes 
\item[1964]  Paul Baran: On Distributed Communications
\item[1969]  ARPANET startes 4 noder
\item[1971]  14 noder
\item[1973]  Arbejde med IP startes
\item[1973]  Email er ca. 75\% af ARPANET traffik
\item[1974]  TCP/IP: Cerf/Kahn: A protocol for Packet
        Network Interconnection
\item[1983]  EUUG $\rightarrow$ DKUUG/DIKU forbindelse
\item[1988]  ca. 60.000 systemer på Internettet
        The Morris Worm rammer ca. 10\%
\item[2000]  Maj I LOVE YOU ormen rammer
%\item[2001]  August Code Red ~600.000 servere 
\item[2002]  Ialt ca. 130 millioner på Internet
\end{list2}

\slide{Internet historisk set -  anno 1969}
\hlkimage{10cm}{1969_4-node_map.png}
%size 2

\begin{list2}
\item Node 1: University of California Los Angeles
\item Node 2: Stanford Research Institute
\item Node 3: University of California Santa Barbara
\item Node 4: University of Utah
%\item Kilde: \link{http://www.zakon.org/robert/internet/timeline/}
\end{list2}

\slide{De tidlige notater om Internet}

\begin{list1}
\item L. Kleinrock \emph{Information Flow in Large Communication nets}, 1961
\item J.C.R. Licklider, MIT noter fra 1962 \emph{On-Line Man Computer
  Communication} 
\item Paul Baran, 1964 \emph{On distributed Communications}
12-bind serie af rapporter\\
\link{http://www.rand.org/publications/RM/baran.list.html}
\item V. Cerf og R. Kahn, 1974 
\emph{A protocol for Packet Network Interconnection}
IEEE Transactions on Communication, vol. COM-22, pp. 637-648, May 1974
\item De tidlige notater kan findes på nettet!
\end{list1}

Læs evt. mere i mit speciale \link{http://www.inet6.dk/thesis.pdf}

\slide{BSD UNIX}

\hlkimage{4cm}{implementation_freebsd.jpg}

\begin{list1}
  \item UNIX kildeteksten var nem at få fat i for universiteter og
  mange andre
\item Bell Labs/AT\&T var et telefonselskab - ikke et software hus
\item På Berkeley Universitetet blev der udviklet en del på UNIX og
  det har givet anledning til en hel gren kaldet BSD UNIX
\item BSD står for Berkeley Software Distribution
\item BSD UNIX har blandt andet resulteret i virtual memory management
  og en masse TCP/IP relaterede applikationer
\end{list1}

\slide{Open Source definitioner - uddrag}

\begin{list1}
\item Free Redistribution - der må ikke lægges begrænsninger på om
  softwaren gives væk eller sælges
\item Source Code - kildeteksten skal være tilgængelig 
\item Derived Works - det skal være muligt at arbejde videre på 
\item Integrity of The Author's Source Code - det skal være muligt at
  beskytte sit navn og rygte, ved at kræve ændret navn for
  afledte projekter
\item Softwaren kaldes ofte også Free Software, nogle bruger endda Libre
\item Eksempler er BSD licensen, Apache, GNU GPL og mange andre
\item Kilder: \link{http://www.opensource.org/}\\
\link{http://en.wikipedia.org/wiki/FLOSS} Free/Libre/Open-Source Software  
\end{list1}

\slide{BSD licensen er pragmatisk}

\begin{list1}
 \item BSD licensen kræver ikke at man offentliggør sine ændringer,
 man kan altså bruge BSD kildetekst og stadig lave et kommercielt
 produkt!
\item GNU GPL bliver af nogle omtalt som en virus - der
  \emph{inficerer} softwaren, og afledte projekter
\end{list1}



\slide{Hvad er Internet}

\begin{list1}
\item 80'erne IP/TCP starten af 80'erne
\item 90'erne IP version 6 udarbejdes
  \begin{list2}
  \item IPv6 ikke brugt i Europa og US
  \item IPv6 er ekstremt vigtigt i Asien 
  \item historisk få adresser tildelt til 3.verdenslande
  \item Større Universiteter i USA har ofte større allokering end Kina!
  \end{list2}
\item 1991 WWW "opfindes" af Tim Berners-Lee hos CERN
\item E-mail var hovedparten af traffik
  - siden overtog web/http førstepladsen
\end{list1}

\slide{Hvad er Internet}

\vskip 1 cm

\centerline{Antallet af hosts på Internet}

\hlkimage{16cm}{images/Count_Host.png}

\begin{list1}
\item Kilde: 
Hobbes' Internet Timeline v5.6\\
\link{http://www.zakon.org/robert/internet/timeline/}
\end{list1}

\slide{Hvad er Internet}

\vskip 1 cm

\centerline{Antallet af World Wide Web servere}

\hlkimage{16cm}{images/Count_WWW.png}  

\begin{list1}
\item Kilde: Hobbes' Internet Timeline v5.6\\
\link{http://www.zakon.org/robert/internet/timeline/}
\end{list1}


% IP-adresser

\slide{Fælles adresserum}

\vskip 2 cm
\hlkimage{17cm}{IP-address.pdf}

\begin{list1}
\item Hvad kendetegner internet idag
\item Der er et fælles adresserum baseret på 32-bit adresser
\item En IP-adresse kunne være 10.0.0.1
\end{list1}

\slide{IPv4 addresser og skrivemåde}

\begin{alltt}
hlk@bigfoot:hlk$ ipconvert.pl 127.0.0.1
Adressen er: 127.0.0.1
Adressen er: 2130706433
hlk@bigfoot:hlk$ ping 2130706433
PING 2130706433 (127.0.0.1): 56 data bytes
64 bytes from 127.0.0.1: icmp_seq=0 ttl=64 time=0.135 ms
64 bytes from 127.0.0.1: icmp_seq=1 ttl=64 time=0.144 ms
\end{alltt}

\begin{list1}
\item IP-adresser skrives typisk som decimaltal adskilt af punktum
\item Kaldes {\bf dot notation}: 10.1.2.3
\item Kan også skrive som oktal eller heksadecimale tal
\end{list1}



\slide{IP-adresser som bits}

\begin{alltt}
IP-adresse: 127.0.0.1
Heltal:	2130706433
Binary:	1111111000000000000000000000001
\end{alltt}

\begin{list1}
\item IP-adresser kan også konverteres til bits
\item Computeren regner binært, vi bruger dot-notationen
\end{list1}

\slide{Internet ABC}

\begin{list1}
\item Tidligere benyttede man klasseinddelingen af IP-adresser: A, B, C, D og E
\item Desværre var denne opdeling ufleksibel:
\begin{list2}
\item A-klasse kunne potentielt indeholde 16 millioner hosts
\item B-klasse kunne potentielt indeholder omkring 65.000 hosts
\item C-klasse kunne indeholde omkring 250 hosts
\end{list2}
\item Derfor bad de fleste om adresser i B-klasser - så de var ved at løbe tør!
\item D-klasse benyttes til multicast
\item E-klasse er blot reserveret
\item Se evt. \link{http://en.wikipedia.org/wiki/Classful\_network}
\end{list1}


\slide{CIDR Classless Inter-Domain Routing}

\hlkimage{15cm}{CIDR-aggregation.pdf}

\begin{list1}
\item Subnetmasker var oprindeligt indforstået
\item Dernæst var det noget man brugte til at opdele sit A, B eller C net med
\item Ved at tildele flere C-klasser kunne man spare de resterende B-klasser - men det betød en routing table explosion
\item Idag er subnetmaske en sammenhængende række 1-bit der angiver størrelse på nettet
\item 10.0.0.0/24 betyder netværket 10.0.0.0 med subnetmaske 255.255.255.0
\item Nogle få steder kaldes det tillige supernet, supernetting
\end{list1}


\slide{Subnet calculator, CIDR calculator}

\hlkimage{10cm}{subnet-calculator.png}

\begin{list1}
\item Der findes et væld af programmer som kan hjælpe med at udregne
subnetmasker til IPv4
\item Screenshot fra \link{http://www.subnet-calculator.com/}
\end{list1}


\slide{RFC-1918 private netværk}

\begin{list1}
\item Der findes et antal adresserum som alle må benytte frit:
\begin{list2}
\item 10.0.0.0    -  10.255.255.255  (10/8 prefix)
\item 172.16.0.0  -  172.31.255.255  (172.16/12 prefix)
\item 192.168.0.0 -  192.168.255.255 (192.168/16 prefix)
\end{list2}
\item Address Allocation for Private Internets RFC-1918 adresserne!
\item NB: man må ikke sende pakker ud på internet med disse som afsender, giver ikke mening
\end{list1}

\slide{IPv4 addresser opsummering}

\begin{list2}
\item Altid 32-bit adresser
\item Skrives typisk med 4 decimaltal dot notation 10.1.2.3
\item Netværk angives med CIDR Classless Inter-Domain Routing RFC-1519
\item CIDR notation 10.0.0.0/8 -
  fremfor 10.0.0.0 med subnet maske 255.0.0.0
\item Specielle adresser\\
127.0.0.1 localhost/loopback\\
0.0.0.0  default route
\item RFC-1918 angiver private adresser som alle kan bruge

\end{list2}


\slide{Stop - netværket idag}

\begin{list1}
\item Bemærk hvilket netværk vi bruger idag
\item Primære server fiona har IP-adressen 10.0.45.36
\item Primære router luffe har IP-adressen 10.0.45.2 (og flere andre)
\item Sekundære router idag er Bianca som har IP-adressen 10.0.46.2 (og flere andre)  
\item Hvis du kender til IP i forvejen så udforsk gerne på egen hånd netværket
\item Det er tilladt at logge ind på alle systemer, undtagen Henrik's laptop bigfoot :-)
\item {\bf Det er forbudt at ændre IP-konfiguration og passwords}
\item Nu burde I kunne forbinde jer til netværket fysisk, check med \verb+ping 10.0.45.2+
\item Det er nok at en PC i hver gruppe er på kursusnetværket
\end{list1}

\centerline{Pause for dem hvor det virker, mens vi ordner resten}


\slide{OSI og Internet modellerne}

\hlkimage{14cm,angle=90}{images/compare-osi-ip.pdf}


\slide{Netværkshardware}

\begin{list1}
\item Der er mange muligheder med IP netværk, IP kræver meget lidt
\item Ofte benyttede idag er:
\begin{list2}
\item Ethernet - varianter 10mbit, 100mbit, gigabit, 10 Gigabit findes, men er dyrt
\item Wireless 802.11 teknologier
\item ADSL/ATM teknologier til WAN forbindelser
\item MPLS ligeledes til WAN forbindelser
\end{list2}
\item Ethernet kan bruge kobberledninger eller fiber
\item WAN forbindelser er typisk fiber på grund af afstanden mellem routere
\item Tidligere benyttede inkluderer: X.25, modem, FDDI, ATM, Token-Ring
\end{list1}

\slide{Ethernet stik, kabler og dioder}

\hlkimage{20cm}{ethernetLights.jpg}

\centerline{Dioder viser typisk om der er link, hastighed samt aktivitet}

\slide{Trådløse teknologier}

\hlkimage{10cm}{WCG200v2_med.jpg}

\begin{list1}
\item Et typisk 802.11 Access-Point (AP) der har Wireless og Ethernet stik/switch
\end{list1}

\slide{MAC adresser}
%\hlkimage{10cm}{apple-oui.png}

\begin{alltt}
00-03-93   (hex)        Apple Computer, Inc.
000393     (base 16)    Apple Computer, Inc.
                        20650 Valley Green Dr.
                        Cupertino CA 95014
                        UNITED STATES
\end{alltt}
\begin{list1}
\item Netværksteknologierne benytter adresser på lag 2
\item Typisk svarende til 48-bit MAC adresser som kendes fra Ethernet MAC-48/EUI-48
\item Første halvdel af adresserne er Organizationally Unique Identifier (OUI)
\item Ved hjælp af OUI kan man udlede hvilken producent der har produceret netkortet
\item \link{http://standards.ieee.org/regauth/oui/index.shtml}
\end{list1}

\slide{Half/full-duplex og speed}

\hlkimage{20cm}{half-full-duplex.pdf}

\begin{list1}
\item Hvad hastighed overføres data med?
\item De fleste nyere Ethernet netkort kan køre i fuld-duplex
\item med full-duplex kan der både sendes og modtages data samtidigt
\item Ethernet kan benytte auto-negotiation - der ofte virker\\
Klart bedre i gigabitnetkort men pas på
\end{list1}





\slide{Broer og routere}

\hlkimage{20cm}{wan-network.pdf}
\centerline{Fysisk er der en begrænsing for hvor lange ledningerne må være}

\slide{Bridges}

\begin{list1}
\item Ethernet er broadcast teknologi, hvor data sendes ud på et delt medie - Æteren
\item Broadcast giver en grænse for udbredningen vs hastighed
\item Ved hjælp af en bro kan man forbinde to netværkssegmenter på layer-2
\item Broen kopierer data mellem de to segmenter
\item Virker som en forstærker på signalet, men mere intelligent
\item Den intelligente bro kender MAC adresserne på hver side
\item Broen kopierer kun hvis afsender og modtager er på hver sin side
\end{list1}

Kilde: For mere information søg efter Aloha-net\\ \link{http://en.wikipedia.org/wiki/ALOHAnet}


\slide{En switch}

\hlkimage{15cm}{switch-1.pdf}

\begin{list1}
\item Ved at fortsætte udviklingen kunne man samle broer til en switch
\item En switch idag kan sende og modtage på flere porte samtidig, og med full-duplex
\item Bemærk performance begrænses af backplane i switchen
\end{list1}

\slide{Topologier og Spanning Tree Protocol}

\hlkimage{18cm}{switch-STP.pdf}

Se mere i bogen af Radia Perlman, \emph{Interconnections: Bridges, Routers, Switches, and Internetworking Protocols}


\slide{Core, Distribution og Access net}

\hlkimage{20cm}{core-dist.pdf}

\centerline{Det er ikke altid man har præcis denne opdeling, men den er ofte brugt}




\slide{Pakker i en datastrøm}

\hlkimage{23cm}{ethernet-frame-1.pdf}
\begin{list1}
\item Ser vi data som en datastrøm er pakkerne blot et mønster lagt henover data 
\item Netværksteknologien definerer start og slut på en frame
\item Fra et lavere niveau modtager vi en pakke, eksempelvis 1500-bytes fra Ethernet driver
\end{list1}



\slide{IPv4 pakken - header - RFC-791}

\begin{alltt}
\small  
    0                   1                   2                   3   
    0 1 2 3 4 5 6 7 8 9 0 1 2 3 4 5 6 7 8 9 0 1 2 3 4 5 6 7 8 9 0 1 
   +-+-+-+-+-+-+-+-+-+-+-+-+-+-+-+-+-+-+-+-+-+-+-+-+-+-+-+-+-+-+-+-+
   |Version|  IHL  |Type of Service|          Total Length         |
   +-+-+-+-+-+-+-+-+-+-+-+-+-+-+-+-+-+-+-+-+-+-+-+-+-+-+-+-+-+-+-+-+
   |         Identification        |Flags|      Fragment Offset    |
   +-+-+-+-+-+-+-+-+-+-+-+-+-+-+-+-+-+-+-+-+-+-+-+-+-+-+-+-+-+-+-+-+
   |  Time to Live |    Protocol   |         Header Checksum       |
   +-+-+-+-+-+-+-+-+-+-+-+-+-+-+-+-+-+-+-+-+-+-+-+-+-+-+-+-+-+-+-+-+
   |                       Source Address                          |
   +-+-+-+-+-+-+-+-+-+-+-+-+-+-+-+-+-+-+-+-+-+-+-+-+-+-+-+-+-+-+-+-+
   |                    Destination Address                        |
   +-+-+-+-+-+-+-+-+-+-+-+-+-+-+-+-+-+-+-+-+-+-+-+-+-+-+-+-+-+-+-+-+
   |                    Options                    |    Padding    |
   +-+-+-+-+-+-+-+-+-+-+-+-+-+-+-+-+-+-+-+-+-+-+-+-+-+-+-+-+-+-+-+-+

                    Example Internet Datagram Header
\end{alltt}


\slide{IP karakteristik}

\begin{list1}
\item Fælles adresserum
\item Best effort - kommer en pakke fra er det fint, hvis ikke må højere lag klare det
\item Kræver ikke mange services fra underliggende teknologi \emph{dumt netværk}
\item Defineret gennem åben standardiseringsprocess og RFC-dokumenter
\end{list1}



\slide{Fragmentering og PMTU}

\hlkimage{20cm}{fragments-1.pdf}
\begin{list1}
\item Hidtil har vi antaget at der blev brugt Ethernet med pakkestørrelse på 1500 bytes
\item Pakkestørrelsen kaldes MTU Maximum Transmission Unit 
\item Skal der sendes mere data opdeles i pakker af denne størrelse, fra afsender
\item Men hvad hvis en router på vejen ikke bruger 1500 bytes, men kun 1000
\end{list1}

\slide{ICMP Internet Control Message Protocol}

\begin{list1}
\item Kontrolprotokol og fejlmeldinger
\item Nogle af de mest almindelige beskedtyper
\begin{list2}
\item echo
\item netmask
\item info
\end{list2}
\item Bruges generelt til \emph{signalering}
\item Defineret i RFC-792
\end{list1}

\centerline{\bf NB: nogle firewall-administratorer blokerer alt ICMP - det er forkert!}

\slide{ICMP beskedtyper}

\begin{list1}
\item Type
\begin{list2}
\item 0 = net unreachable;
\item 1 = host unreachable;
\item 2 = protocol unreachable;
\item 3 = port unreachable;
\item 4 = fragmentation needed and DF set;
\item 5 = source route failed.
\end{list2}
\item Ved at fjerne ALT ICMP fra et net fjerner man nødvendig funktionalitet!
\item Tillad ICMP types:
\begin{list2}
\item 3 Destination Unreachable
\item 4 Source Quench Message
\item 11 Time Exceeded
\item 12 Parameter Problem Message
\end{list2}
\end{list1}

\slide{Hvordan virker ARP?}

\begin{center}
\colorbox{white}{\includegraphics[width=18cm]{images/arp-basic.pdf}}  
\end{center}

%server 00:30:65:22:94:a1\\
%client 00:40:70:12:95:1c\\
%hacker 00:02:03:04:05:06\\

\slide{Hvordan virker ARP? - 2}
\begin{list1}
\item {\bfseries ping 10.0.0.2} udført på server medfører
\item ARP Address Resolution Protocol request/reply:
  \begin{list2}
  \item ARP request i broadcast - Who has 10.0.0.2 Tell 10.0.0.1
  \item ARP reply (fra 10.0.0.2) 10.0.0.2 is at 00:40:70:12:95:1c
  \end{list2}
\item IP ICMP request/reply:
  \begin{list2}
    \item Echo (ping) request fra 10.0.0.1 til 10.0.0.2
\item Echo (ping) reply fra 10.0.0.2 til 10.0.0.1
\item ...
  \end{list2}
\item ARP udføres altid på Ethernet før der kan sendes IP trafik
\item (kan være RARP til udstyr der henter en adresse ved boot)
\end{list1}


\slide{ARP cache}

\begin{alltt}
\small
hlk@bigfoot:hlk$ arp -an        
? (10.0.42.1) at 0:0:24:c8:b2:4c on en1 [ethernet]
? (10.0.42.2) at 0:c0:b7:6c:19:b on en1 [ethernet]
\end{alltt}

\begin{list1}
\item ARP cache kan vises med kommandoen \verb+arp -an+
\item -a viser alle
\item -n viser kun adresserne, prøver ikke at slå navne op - typisk hurtigere
\item ARP cache er dynamisk og adresser fjernes automatisk efter 5-20 minutter hvis de ikke bruges mere
\item Læs mere med \verb+man 4 arp+
\end{list1}


\slide{Manualsystemet}

\begin{quote}
 It is a book about a Spanish guy called Manual. You should read it.
       -- Dilbert
\end{quote}

\begin{list1}
\item Manualsystemet i UNIX er utroligt stærkt!
\item Det SKAL altid installeres sammen med værktøjerne!
\item Det er næsten identisk på diverse UNIX varianter!  
\item \verb+man -k+ søger efter keyword, se også \verb+apropos+
\end{list1}

Prøv \verb+man crontab+ og \verb+man 5 crontab+

\hlkimage{10cm}{images/unix-command-1.pdf}

\slide{En manualside}

\begin{alltt}
\small
CAL(1)                BSD General Commands Manual                CAL(1)
NAME
     cal - displays a calendar
SYNOPSIS
     cal [-jy] [[month]  year]
DESCRIPTION
   cal displays a simple calendar.  If arguments are not specified, the cur-
   rent month is displayed.  The options are as follows:
   -j      Display julian dates (days one-based, numbered from January 1).
   -y      Display a calendar for the current year.

The Gregorian Reformation is assumed to have occurred in 1752 on the 3rd
of September.  By this time, most countries had recognized the reforma-
tion (although a few did not recognize it until the early 1900's.)  Ten
days following that date were eliminated by the reformation, so the cal-
endar for that month is a bit unusual.

HISTORY
     A cal command appeared in Version 6 AT&T UNIX.  
\end{alltt}

\slide{Kommandolinien på UNIX}

\begin{list1}
\item Shells kommandofortolkere:
  \begin{list2}
    \item sh - Bourne Shell
\item bash - Bourne Again Shell
\item ksh - Korn shell, lavet af David Korn
\item csh - C shell, syntaks der minder om C sproget
\item flere andre, zsh, tcsh 
  \end{list2}
\item Svarer til command.com og cmd.exe på Windows
\item Kan bruges som komplette programmeringssprog
\end{list1}

\slide{Kommandoprompten}


\begin{alltt}    
\small
[hlk@fischer hlk]$ id
uid=6000(hlk) gid=20(staff) groups=20(staff), 
0(wheel), 80(admin), 160(cvs) 
[hlk@fischer hlk]$ 

[root@fischer hlk]# id
uid=0(root) gid=0(wheel) groups=0(wheel), 1(daemon),
2(kmem), 3(sys), 4(tty), 5(operator), 20(staff), 
31(guest), 80(admin) 
[root@fischer hlk]#
\end{alltt}

\begin{list1}  
\item typisk viser et dollartegn at man er logget ind som almindelig bruge
\item mens en havelåge at man er root - superbruger
\end{list1}

\slide{Kommandoliniens opbygning}


\begin{alltt}
echo [-n] [string ...]  
\end{alltt}

\begin{list1}
\item Kommandoerne der skrives på kommandolinien skrives sådan:
\begin{list2}
\item Starter altid med kommandoen, man kan ikke skrive \verb+henrik echo+
\item Options skrives typisk med bindestreg foran, eksempelvis \verb+-n+
\item Flere options kan sættes sammen, \verb+tar -cvf+ eller \verb+tar cvf+
\item I manualsystemet kan man se valgfrie options i firkantede
  klammer \verb+[]+
\item Argumenterne til kommandoen skrives typisk til sidst (eller der
  bruges redirection)
\end{list2}
\end{list1}


\slide{Adgang til UNIX}

\begin{center}
\includegraphics[width=4cm]{images/kde.png}
\includegraphics[width=4cm]{images/gnome-logo-large.png}
\end{center}

\begin{list1}
%\item Systemer der minder om UNIX kan idag nemt skaffes
\item Adgang til UNIX kan ske via grafiske brugergrænseflader
  \begin{list2}
%  \item X11 \link{http://www.x.org}
  \item KDE \link{http://www.kde.org}
  \item GNOME \link{http://www.gnome.org}
  \end{list2}
\item eller kommandolinien
\end{list1}
\centerline{\includegraphics[width=17cm]{images/unix-cmdline.pdf}}


\exercise{ex:putty-install}


\exercise{ex:winscp-install}

\exercise{ex:unix-login}

\exercise{ex:unix-cal}

\exercise{ex:sudo}



\exercise{ex:unix-boot-cd}




\exercise{ex:unix-basic-commands}



\slide{TCP/IP basiskonfiguration}

\begin{alltt}
ifconfig en0 10.0.42.1 netmask 255.255.255.0
route add default gw 10.0.42.1 
\end{alltt}

\begin{list1}
\item konfiguration af interfaces og netværk på UNIX foregår med:
\item \verb+ifconfig+, \verb+route+ og \verb+netstat+  
\item - ofte pakket ind i konfigurationsmenuer m.v.
\item fejlsøgning foregår typisk med \verb+ping+ og \verb+traceroute+
\item På Microsoft Windows benyttes ikke \verb+ifconfig+\\
men kommandoerne \verb+ipconfig+ og \verb+ipv6+
\end{list1}


\slide{Små forskelle}

\begin{alltt}
$ route add default 10.0.42.1
\emph{uden gw keyword!}

$ route add default gw 10.0.42.1 
\emph{Linux kræver gw med}
\end{alltt}

\vskip 1cm

\centerline{\bf NB: UNIX varianter kan indbyrdes være forskellige!}



\slide{Flere små forskelle}

\vskip 1cm 
\centerline{ping eller ping6}

\begin{list1}
\item Nogle systemer vælger at ping kommandoen kan ping'e både IPv4 og Ipv6
\item Andre vælger at \verb+ping+ kun benyttes til IPv4, mens IPv6 ping kaldes for \verb+ping6+
\item Læg også mærke til jargonen \emph{at pinge}
\end{list1}


\slide{OpenBSD}

Netværkskonfiguration på OpenBSD:
\begin{alltt}
# cat /etc/hostname.sk0
inet 10.0.0.23 0xffffff00 NONE
# cat /etc/mygate
10.0.0.1
# cat /etc/resolv.conf    
domain security6.net
lookup file bind
nameserver 212.242.40.3
nameserver 212.242.40.51
\end{alltt}

\slide{FreeBSD}

Netværkskonfiguration på FreeBSD \verb+/etc/rc.conf+:
\begin{alltt}
\small
# This file now contains just the overrides from /etc/defaults/rc.conf.
hostname="freebsd.security6.net
#ifconfig_vr0="DHCP"
ifconfig_vr0="inet 10.20.30.75 netmask 255.255.255.0"
router_enable="NO"
defaultrouter="10.20.30.65"
keyrate="fast"
moused_enable="YES"
ntpdate_enable="NO"
ntpdate_flags="none"
saver="blank"
sshd_enable="YES"
usbd_enable="YES"
...
\end{alltt}


\slide{GUI værktøjer - autoconfiguration}

\hlkimage{20cm}{osx-network-automatic.png}

\slide{GUI værktøjer - manuel konfiguration}

\hlkimage{20cm}{osx-network-manual.png}

\slide{ifconfig output}

\begin{alltt}\small
hlk@bigfoot:hlk$ ifconfig -a
lo0: flags=8049<UP,LOOPBACK,RUNNING,MULTICAST> mtu 16384
        inet 127.0.0.1 netmask 0xff000000 
        inet6 ::1 prefixlen 128 
        inet6 fe80::1%lo0 prefixlen 64 scopeid 0x1 
gif0: flags=8010<POINTOPOINT,MULTICAST> mtu 1280
stf0: flags=0<> mtu 1280
en0: flags=8863<UP,BROADCAST,SMART,RUNNING,SIMPLEX,MULTICAST> mtu 1500
        ether 00:0a:95:db:c8:b0 
        media: autoselect (none) status: inactive
        supported media: none autoselect 10baseT/UTP <half-duplex> 10baseT/UTP <full-duplex> 10baseT/UTP <full-duplex,hw-loopback> 100baseTX <half-duplex> 100baseTX <full-duplex> 100baseTX <full-duplex,hw-loopback> 1000baseT <full-duplex> 1000baseT <full-duplex,hw-loopback> 1000baseT <full-duplex,flow-control> 1000baseT <full-duplex,flow-control,hw-loopback>
en1: flags=8863<UP,BROADCAST,SMART,RUNNING,SIMPLEX,MULTICAST> mtu 1500
        ether 00:0d:93:86:7c:3f 
        media: autoselect (<unknown type>) status: inactive
        supported media: autoselect
\end{alltt}
%$
\vskip 1 cm
\centerline{ifconfig output er næsten ens på tværs af UNIX}




\slide{Vigtigste protokoller}


\begin{list1}
\item ARP Address Resolution Protocol
\item IP og ICMP Internet Control Message Protocol
\item UDP User Datagram Protocol
\item TCP Transmission Control Protocol
\item DHCP Dynamic Host Configuration Protocol 
\item DNS Domain Name System
\end{list1}
\vskip 1cm
\centerline{Ovenstående er omtrent minimumskrav for at komme på internet}

% allerede gennemgået ovenfor
%\slide{ICMP}

%\begin{list1}
%\item 	Internet Control Message Protocol 
%	Defineret i RFC-792

%\end{list1}


\slide{UDP User Datagram Protocol}
\hlkimage{20cm}{udp-1.pdf}
\begin{list1}
\item Forbindelsesløs RFC-768, \emph{connection-less} - der kan tabes pakker
\item Kan benyttes til multicast/broadcast - flere modtagere
\end{list1}



\slide{TCP Transmission Control Protocol}
\hlkimage{20cm}{tcp-1.pdf}

\begin{list1}
\item Forbindelsesorienteret RFC-791 September 1981, \emph{connection-oriented}
\item Enten overføres data eller man får fejlmeddelelse
\end{list1}




\slide{TCP three way handshake}

\hlkimage{7cm}{images/tcp-three-way.pdf}

\begin{list2}
\item {\bfseries TCP SYN half-open} scans
\item Tidligere loggede systemer kun når der var etableret en fuld TCP
  forbindelse - dette kan/kunne udnyttes til \emph{stealth}-scans
\item Hvis en maskine modtager mange SYN pakker kan dette fylde
  tabellen over connections op - og derved afholde nye forbindelser
  fra at blive oprette - {\bfseries SYN-flooding}
\end{list2}

\slide{Well-known port numbers}

\hlkimage{10cm}{iana1.jpg}

\begin{list1}
\item IANA vedligeholder en liste over magiske konstanter i IP
\item De har lister med hvilke protokoller har hvilke protokol ID m.v.
\item En liste af interesse er port numre, hvor et par eksempler er:
\begin{list2}
\item Port 25 SMTP Simple Mail Transfer Protocol
\item Port 53 DNS Domain Name System
\item Port 80 HTTP Hyper Text Transfer Protocol over TLS/SSL
\item Port 443 HTTP over TLS/SSL
\end{list2}
\item Se flere på \link{http://www.iana.org}
\end{list1}

\slide{Hierarkisk routing}

\hlkimage{18cm}{routing-1.pdf}
Hvordan kommer pakkerne frem til modtageren

\slide{IP default gateway}

\hlkimage{13cm}{routing-2.pdf}

\begin{list1}
\item IP routing er nemt
\item En host kender en default gateway i nærheden
\item En router har en eller flere upstream routere, få adresser den sender videre til
\item Core internet har default free zone, kender \emph{alle netværk} 
\end{list1}



\slide{DHCP Dynamic Host Configuration Protocol}

\hlkimage{13cm}{dhcp-1.pdf}

\begin{list1}
\item Hvordan får man information om default gateway
\item Man sender et DHCP request og modtager et svar fra en DHCP server
\item Dynamisk konfiguration af klienter fra en centralt konfigureret server
\item Bruges til IP adresser og meget mere
\end{list1}


\slide{Routing}


\begin{list1}
  \item routing table - tabel over netværkskort og tilhørende adresser
\item default gateway - den adresse hvortil man sender
  \emph{non-local} pakker\\kaldes også default route, gateway of last
  resort
\item routing styres enten manuelt - opdatering af route tabellen,
  eller konfiguration af adresser og subnet maske på netkort
\item eller automatisk ved brug af routing protocols - interne og
  eksterne route protokoller
\item de lidt ældre routing protokoller har ingen sikkerhedsmekanismer
\item {\bf IP benytter longest match i routing tabeller!}
\item Den mest specifikke route gælder for forward af en pakke!
\end{list1}


\slide{Routing forståelse}

\begin{alltt}
\small
$ netstat -rn
Routing tables

Internet:
Destination    Gateway         Flags  Refs      Use  Netif 
default        10.0.0.1        UGSc    23        7    en0
10/24          link#4          UCS      1        0    en0
10.0.0.1       0:0:24:c1:58:ac UHLW    24       18    en0  
10.0.0.33      127.0.0.1       UHS      0        1    lo0
10.0.0.63      127.0.0.1       UHS      0        0    lo0
127            127.0.0.1       UCS      0        0    lo0
127.0.0.1      127.0.0.1       UH       4     7581    lo0
169.254        link#4          UCS      0        0    en0  
\end{alltt}

\vskip 1 cm
\centerline{Start med kun at se på Destination, Gateway og Netinterface}


\exercise{ex:network-ifconfig}
\exercise{ex:network-netstat}
\exercise{ex:network-lsof}


\slide{whois systemet}

\begin{list1}
\item IP adresserne administreres i dagligdagen af et antal Internet
  registries, hvor de største er:
\begin{list2}
\item RIPE (Réseaux IP Européens)  \link{http://ripe.net}
\item ARIN American Registry for Internet Numbers \link{http://www.arin.net}
\item Asia Pacific Network Information Center \link{http://www.apnic.net}
\item LACNIC (Regional Latin-American and Caribbean IP Address Registry) - Latin America and some Caribbean Islands
\end{list2}
\item disse fire kaldes for Regional Internet Registries (RIRs) i
  modsætning til Local Internet Registries (LIRs) og National Internet
  Registry (NIR) 
\end{list1}

\slide{whois systemet-2}

\begin{list1}
\item ansvaret for Internet IP adresser ligger hos ICANN The Internet
  Corporation for Assigned Names and Numbers\\
\link{http://www.icann.org}
\item NB: ICANN må ikke forveksles med IANA Internet Assigned Numbers
  Authority \link{http://www.iana.org/} som bestyrer portnumre m.v.
\end{list1}

\exercise{ex:whois}


% basic ping og traceroute

\slide{Ping}

\begin{list1}
\item ICMP - Internet Control Message Protocol
\item Benyttes til fejlbeskeder og til diagnosticering af forbindelser
\item ping programmet virker ved hjælp af ICMP ECHO request og
  forventer ICMP ECHO reply
\item 
\begin{alltt}
\small {\bfseries 
$ ping 192.168.1.1}
PING 192.168.1.1 (192.168.1.1): 56 data bytes
64 bytes from 192.168.1.1: icmp_seq=0 ttl=150 time=8.849 ms
64 bytes from 192.168.1.1: icmp_seq=1 ttl=150 time=0.588 ms
64 bytes from 192.168.1.1: icmp_seq=2 ttl=150 time=0.553 ms
\end{alltt}
\end{list1}

\slide{traceroute}

\begin{list1}
  \item traceroute programmet virker ved hjælp af TTL
\item levetiden for en pakke tælles ned i hver router på vejen og ved at sætte denne lavt
  opnår man at pakken \emph{timer ud} - besked fra hver router på vejen
\item default er UDP pakker, men på UNIX systemer er der ofte mulighed
  for at bruge ICMP
\item 
\begin{alltt}
\small{\bfseries 
$ traceroute 217.157.20.129}
traceroute to 217.157.20.129 (217.157.20.129),
30 hops max, 40 byte packets
 1  safri (10.0.0.11)  3.577 ms  0.565 ms  0.323 ms
 2  router (217.157.20.129)  1.481 ms  1.374 ms  1.261 ms
\end{alltt}
\end{list1}

%DNS
\slide{Domain Name System}

\hlkimage{12cm}{dns-1.pdf}

\begin{list1}
\item Gennem DHCP får man typisk også information om DNS servere
\item En DNS server kan slå navne, domæner og adresser op
\item Foregår via query og response med datatyper kaldet resource records
\item DNS er en distribueret database, så opslag kan resultere i flere opslag
\end{list1}


\slide{DNS systemet}

\begin{list1}
\item navneopslag på Internet  
\item tidligere brugte man en {\bfseries hosts} fil\\
hosts filer bruges stadig lokalt til serveren - IP-adresser
\item UNIX: /etc/hosts
\item Windows \verb+c:\windows\system32\drivers\etc\hosts+
\item Eksempel: www.security6.net har adressen 217.157.20.131
\item skrives i database filer, zone filer
\end{list1}

\begin{alltt}
ns1     IN      A       217.157.20.130
        IN      AAAA    2001:618:433::1
www     IN      A       217.157.20.131
        IN      AAAA    2001:618:433::14
\end{alltt}

\slide{Mere end navneopslag}

\begin{list1}
  \item består af resource records med en type:
    \begin{list2}
\item adresser A-records
\item IPv6 adresser AAAA-records
\item autoritative navneservere NS-records
\item post, mail-exchanger MX-records
\item flere andre: md ,  mf ,  cname ,  soa ,
                  mb , mg ,  mr ,  null ,  wks ,  ptr ,
                  hinfo ,  minfo ,  mx ....
\end{list2}
\end{list1}
\begin{alltt}
        IN      MX      10      mail.security6.net.
        IN      MX      20      mail2.security6.net.
\end{alltt}

\slide{Basal DNS opsætning på klienter}

\begin{list1}    
\item \verb+/etc/resolv.conf+
\item NB: denne fil kan hedde noget andet på UNIX varianter!
\item eksempelvis \verb+/etc/netsvc.conf+
\item typisk indhold er domænenavn og IP-adresser for navneservere
\end{list1}

\begin{alltt}
domain security6.net
nameserver 212.242.40.3
nameserver 212.242.40.51
\end{alltt}

\slide{DNS root servere}
\begin{list1}
  \item Root-servere - 13 stk geografisk distribueret fordelt på Internet
\end{list1}

\begin{alltt}
I.ROOT-SERVERS.NET.     3600000 A       192.36.148.17
E.ROOT-SERVERS.NET.     3600000 A       192.203.230.10
D.ROOT-SERVERS.NET.     3600000 A       128.8.10.90
A.ROOT-SERVERS.NET.     3600000 A       198.41.0.4
H.ROOT-SERVERS.NET.     3600000 A       128.63.2.53
C.ROOT-SERVERS.NET.     3600000 A       192.33.4.12
G.ROOT-SERVERS.NET.     3600000 A       192.112.36.4
F.ROOT-SERVERS.NET.     3600000 A       192.5.5.241
B.ROOT-SERVERS.NET.     3600000 A       128.9.0.107
J.ROOT-SERVERS.NET.     3600000 A       198.41.0.10
K.ROOT-SERVERS.NET.     3600000 A       193.0.14.129
L.ROOT-SERVERS.NET.     3600000 A       198.32.64.12
M.ROOT-SERVERS.NET.     3600000 A       202.12.27.33  
\end{alltt}

\slide{DK-hostmaster}

\begin{list1}
\item bestyrer .dk TLD - top level domain
  
\item man registrerer ikke .dk-domæner hos DK-hostmaster, men hos en
  registrator 
\item Et domæne bør have flere navneservere og flere postservere
\item autoritativ navneserver - ved autoritativt om IP-adresse for
  maskine.domæne.dk findes 
\item ikke-autoritativ - har på vegne af en klient slået en adresse op
\item Det anbefales at overveje en service som
  \link{http://www.gratisdns.dk} der har 5 navneservere distribueret
  over stor geografisk afstand - en udenfor Danmark
\end{list1}

\slide{Navngivning af servere}

\begin{list1}
  \item Hvordan skal vi kunne huske og administrere servere?
\item Det er ikke nemt at navngive hverken brugere eller servere!
\item Selvom det lyder smart med A01S13, som forkortelse af Afdeling
  01's Server nr 13, er det umuligt at huske
\item ... men måske nødvendigt i de største netværk
  \begin{list2}
  
\item Windows serveren er domænecontroller - skal hedde:
\item Linux server som er terminalserver - skal hedde:
\item PC-system med NetBSD skal måske være vores ene server - skal hedde: ?
\item PC-system 1 med en Linux server - skal hedde: 
\item PC-system 2 med en Linux server - skal hedde:
  \end{list2}
\end{list1}



% NAT
\slide{NAT Network Address Translation}
\hlkimage{20cm}{nat-1.pdf}


\vskip 2 cm
\begin{list2}
\item NAT bruges til at forbinde et privat net (RFC-1918 adresser) med internet
\item NAT gateway udskifter afsender adressen med sin egen
\item En quick and dirty fix der vil forfølge os for resten af vores
  liv 
\item Ødelægger en del protokoller :-(
\item Lægger state i netværket - ødelægger fate sharing  
\end{list2}




\slide{NAT is BAD}


\hlkimage{20cm}{nat-is-bad.pdf}


\begin{list2}
\item NAT ødelægger end-to-end transparency!
\item Problemer med servere bagved NAT
\item "løser" problemet "godt nok" (tm) for mange
\item Men idag ser vi multilevel NAT! - eeeeeeewwwwww!
\item Se RFC-2775 Internet Transparency for mere om dette emne
\end{list2}


\exercise{ex:ping}
\exercise{ex:icmpush}
\exercise{ex:basic-dns-lookup}

