\slide{Dag 2 IPv6, Management, diagnosticering}

%\hlkimage{18cm}{nagios-status-overview.jpg}
\hlkimage{18cm}{cricket-mini-graph.png}


% IPv6
\slide{IPv4 Adresserummet er ved at løbe ud}

\begin{list1}
\item Adresserummet er ved at løbe ud! faktum!
\item 32-bit - der ikke kan udnyttes fuldt ud
\item Tidligere brugte man begreberne A,B og C klasser af IP-adresser
\begin{list2}
\item 10.0.0.0    -  10.255.255.255  (10/8 prefix)
\item 172.16.0.0  -  172.31.255.255  (172.16/12 prefix)
\item 192.168.0.0 -  192.168.255.255 (192.168/16 prefix)
\end{list2}
\item Address Allocation for Private Internets RFC-1918 adresserne!
\item Husk at idag benyttes Classless Inter-Domain Routing CIDR\\
\link{http://en.wikipedia.org/wiki/Classless_Inter-Domain_Routing}
\item Notation: 192.168.1.0/24\\
det sædvanlige hjemmenet med subnet maske 255.255.255.0
\end{list1}

\slide{Status idag}
\hlkimage{10cm}{map_of_the_internet.jpg}

\slide{Tidslinie for IPv6 (forkortet)}

\begin{list2}
\item 1990 Vancouver IETF meeting
det estimeres at klasse B vil løbe ud ca. marts 1994

\item 1990 ultimo 
initiativer til at finde en afløser for IPv4

\item 1995 januar 
RFC-1752 Recommendation for the IP NG Protocol

\item 1995 september 
RFC-1883, RFC-1884, RFC-1885, RFC-1886 1. generation 

\item 1998 10. august 
"core" IPv6 dokumenter bliver Draft Standard
\item 
Kilde: RFC-2460, RFC-2461, RFC-2463, RFC-1981 - m.fl.
\end{list2}

\slide{IPv6: Internet redesigned? - nej!}
 
\begin{list1}
\item Målet var at bevare de gode egenskaber
\begin{list2}
\item basalt set Internet i gamle dage
\item back to basics!
\item fate sharing
\item kommunikationen afhænger ikke af state i netværket
\item end-to-end transparency
\end{list2}
\item Idag er Internet blevet en nødvendighed for mange!
\end{list1}

\centerline{\bf IP er en forretningskritisk ressource}

IPv6 basis i RFC-1752 The Recommendation for the IP Next Generation Protocol




\slide{KAME - en IPv6 reference implementation}

\hlkimage{6cm}{kame-noanime-small.png}

%center
\centerline{\link{http://www.kame.net}}

\begin{list2}
\item Er idag at betragte som en reference implementation\\
- i stil med BSD fra Berkeley var det
\item KAME har været på forkant med implementation af draft dokumenter
\item KAME er inkluderet i OpenBSD, NetBSD, FreeBSD og BSD/OS
- har været det siden version 2.7, 1.5, 4.0 og 4.2

\item Projektet er afsluttet, men nye projekter fortsætter i
WIDE regi \link{http://www.wide.ad.jp/}
\item Der er udkommet to bøger som i detaljer gennemgår IPv6 protokollerne i KAME
\end{list2}

\slide{Hvordan bruger man IPv6}


\begin{center}
\hlkbig
\vskip 2 cm
www.inet6.dk

hlk@inet6.dk

\end{center}

\pause
DNS AAAA record tilføjes

\begin{alltt}
www     IN A    91.102.91.17
        IN AAAA 2001:16d8:ff00:12f::2
mail    IN A    91.102.91.17
        IN AAAA 2001:16d8:ff00:12f::2
\end{alltt}

\slide{IPv6 addresser og skrivemåde}

\hlkimage{20cm}{ipv6-address-1.pdf}

\begin{list2}
\item 128-bit adresser, subnet prefix næsten altid 64-bit
\item skrives i grupper af 4 hexcifre ad gangen adskilt af kolon :
\item foranstillede 0 i en gruppe kan udelades, en række 0 kan erstattes med ::
\item dvs 0:0:0:0:0:0:0:0 er det samme som \\
0000:0000:0000:0000:0000:0000:0000:0000
\item Dvs min webservers IPv6 adresse kan skrives som: 
2001:16d8:ff00:12f::2
\item Specielle adresser:
::1 localhost/loopback og
::  default route
\item Læs mere i RFC-3513
\end{list2}

\slide{IPv6 addresser - prefix notation}


\begin{list1}
\item CIDR Classless Inter-Domain Routing RFC-1519
\item Aggregatable Global Unicast
\item 2001::/16 RIR subTLA space
\begin{list2}
\item 2001:200::/23 APNIC
\item 2001:400::/23 ARIN
\item 2001:600::/23 RIPE
\end{list2}
\item 2002::/16 6to4 prefix
\item 3ffe::/16 6bone allocation
\item link-local unicast addresses\\
fe80::/10 genereres udfra MAC addresserne EUI-64
\end{list1}



%%%%%%%%%%%%%%%%%%%%%%%%%%%%%%%%%%%%%%%%%%%%%%%%%%%%%%%%%%%%%%%%%%%%%%%
\slide{IPv6 addresser - multicast}

\begin{list1}
\item Unicast - identificerer ét interface 
pakker sendes til en modtager

\item Multicast - identificerer flere interfaces
pakker sendes til flere modtagere

\item Anycast - indentificerer en "gruppe"
en pakke sendes til et vilkårligt interface med denne adresse typisk det nærmeste

\item Broadcast?
er væk, udeladt, finito, gone!
	
\item Husk også at site-local er deprecated, se RFC-3879
\end{list1}

%%%%%%%%%%%%%%%%%%%%%%%%%%%%%%%%%%%%%%%%%%%%%%%%%%%%%%%%%%%%%%%%%%%%%%%
\slide{IPv6 pakken - header - RFC-2460}

\begin{list2}
\item Simplere - fixed size - 40 bytes
\item Sjældent brugte felter (fra v4) udeladt (kun 6 vs 10 i IPv4)
\item Ingen checksum!
\item Adresser 128-bit
\item 64-bit aligned, alle 6 felter med indenfor første 64
\end{list2}

Mindre kompleksitet for routere på vejen medfører
mulighed for flere pakker på en given router

%%%%%%%%%%%%%%%%%%%%%%%%%%%%%%%%%%%%%%%%%%%%%%%%%%%%%%%%%%%%%%%%%%%%%%%
\slide{IPv6 pakken - header - RFC-2460}


\begin{alltt}
\footnotesize

   +-+-+-+-+-+-+-+-+-+-+-+-+-+-+-+-+-+-+-+-+-+-+-+-+-+-+-+-+-+-+-+-+
   |Version| Traffic Class |           Flow Label                  |
   +-+-+-+-+-+-+-+-+-+-+-+-+-+-+-+-+-+-+-+-+-+-+-+-+-+-+-+-+-+-+-+-+
   |         Payload Length        |  Next Header  |   Hop Limit   |
   +-+-+-+-+-+-+-+-+-+-+-+-+-+-+-+-+-+-+-+-+-+-+-+-+-+-+-+-+-+-+-+-+
   |                                                               |
   +                                                               +
   |                                                               |
   +                         Source Address                        +
   |                                                               |
   +                                                               +
   |                                                               |
   +-+-+-+-+-+-+-+-+-+-+-+-+-+-+-+-+-+-+-+-+-+-+-+-+-+-+-+-+-+-+-+-+
   |                                                               |
   +                                                               +
   |                                                               |
   +                      Destination Address                      +
   |                                                               |
   +                                                               +
   |                                                               |
   +-+-+-+-+-+-+-+-+-+-+-+-+-+-+-+-+-+-+-+-+-+-+-+-+-+-+-+-+-+-+-+-+
\end{alltt}


\slide{IPv4 pakken - header - RFC-791}


\begin{alltt}
\small  
    0                   1                   2                   3   
    0 1 2 3 4 5 6 7 8 9 0 1 2 3 4 5 6 7 8 9 0 1 2 3 4 5 6 7 8 9 0 1 
   +-+-+-+-+-+-+-+-+-+-+-+-+-+-+-+-+-+-+-+-+-+-+-+-+-+-+-+-+-+-+-+-+
   |Version|  IHL  |Type of Service|          Total Length         |
   +-+-+-+-+-+-+-+-+-+-+-+-+-+-+-+-+-+-+-+-+-+-+-+-+-+-+-+-+-+-+-+-+
   |         Identification        |Flags|      Fragment Offset    |
   +-+-+-+-+-+-+-+-+-+-+-+-+-+-+-+-+-+-+-+-+-+-+-+-+-+-+-+-+-+-+-+-+
   |  Time to Live |    Protocol   |         Header Checksum       |
   +-+-+-+-+-+-+-+-+-+-+-+-+-+-+-+-+-+-+-+-+-+-+-+-+-+-+-+-+-+-+-+-+
   |                       Source Address                          |
   +-+-+-+-+-+-+-+-+-+-+-+-+-+-+-+-+-+-+-+-+-+-+-+-+-+-+-+-+-+-+-+-+
   |                    Destination Address                        |
   +-+-+-+-+-+-+-+-+-+-+-+-+-+-+-+-+-+-+-+-+-+-+-+-+-+-+-+-+-+-+-+-+
   |                    Options                    |    Padding    |
   +-+-+-+-+-+-+-+-+-+-+-+-+-+-+-+-+-+-+-+-+-+-+-+-+-+-+-+-+-+-+-+-+

                    Example Internet Datagram Header
\end{alltt}


\slide{IPv6 pakken - extension headers RFC-2460}

\begin{list1}
\item Fuld IPv6 implementation indeholder:
\begin{list2}
\item Hop-by-Hop Options
\item Routing (Type 0)
\item Fragment - fragmentering KUN i end-points!
\item Destination Options
\item Authentication
\item Encapsulating Security Payload
\end{list2}
\item Ja, IPsec er en del af IPv6!
\end{list1}

\slide{Placering af extension headers}

\begin{alltt}
\small
  +---------------+----------------+------------------------
  |  IPv6 header  | Routing header | TCP header + data
  |               |                |
  | Next Header = |  Next Header = |
  |    Routing    |      TCP       |
  +---------------+----------------+------------------------


  +---------------+----------------+-----------------+-----------------
  |  IPv6 header  | Routing header | Fragment header | fragment of TCP
  |               |                |                 |  header + data
  | Next Header = |  Next Header = |  Next Header =  |
  |    Routing    |    Fragment    |       TCP       |
  +---------------+----------------+-----------------+-----------------
\end{alltt}


%%%%%%%%%%%%%%%%%%%%%%%%%%%%%%%%%%%%%%%%%%%%%%%%%%%%%%%%%%%%%%%%%%%%%%%
\slide{IPv6 configuration - kom igang}

\begin{list1}
\item Router bagved NAT
	skal blot kunne forwarde protokoltype 0x41\\
	Cisco 677: \verb+set nat entry add 10.1.2.3 0 41+

\item Teredo - the Shipworm er også en mulighed og benyttes aktivt på Windows Vista idag

\item Officiel IPv4 addresse kan bruges med 6to4 til at lave prefix og router

\item DNS nameserver anbefales!!
	tænk på om den skal svare IPv6 AAAA record
	OG svare over IPv6 sockets - er måske ikke nødvendigt

\item IPv6-only netværk er sikkert sjældne indtil videre men det er
  muligt at lave dem nu 
\item Jeg bruger \link{http://www.sixxs.net} som har vejledninger  til diverse operativsystemer
\end{list1}


%%%%%%%%%%%%%%%%%%%%%%%%%%%%%%%%%%%%%%%%%%%%%%%%%%%%%%%%%%%%%%%%%%%%%%%
\slide{IPv6 configuration - klienter}

\begin{alltt}
\footnotesize$ ping6 ::1
PING6(56=40+8+8 bytes) ::1 --> ::1
16 bytes from ::1, icmp_seq=0 hlim=64 time=0.254 ms
16 bytes from ::1, icmp_seq=1 hlim=64 time=0.23 ms
^C
	--- ::1 ping6 statistics ---
2 packets transmitted, 2 packets received, 0% packet loss
round-trip min/avg/max = 0.230/0.242/0.254 ms
\end{alltt}

\begin{list1}
  

\item Microsoft Windows XP
\verb+ipv6 install+ fra kommandolinien eller brug kontrolpanelet

\item ipv6 giver mulighed for at konfigurere tunnel
svarer omtrent til 'ifconfig' på Unix

\item Migrering er vigtigt i IPv6!
Hvis I aktiverer IPv6 nu på en router, vil I 
pludselig have IPv6 på alle klienter ;-)


\item Se evt. appendix F Enabling IPv6 functionality i 
\link{http://inet6.dk/thesis.pdf}
\end{list1}


\slide{ifconfig med ipv6 - Unix}

Næsten ingen forskel på de sædvanlige kommandoer ifconfig, netstat, 
\begin{alltt}
\small
root# ifconfig en1 inet6 2001:1448:81:beef::1
root# ifconfig en1
en1: flags=8863<UP,BROADCAST,SMART,RUNNING,SIMPLEX,MULTICAST> mtu 1500{\color{security6blue}
        inet6 fe80::230:65ff:fe17:94d1 prefixlen 64 scopeid 0x5 
        inet6 2001:1448:81:beef::1 prefixlen 64 }
        inet 169.254.32.125 netmask 0xffff0000 broadcast 169.254.255.255
        ether 00:30:65:17:94:d1 
        media: autoselect status: active
        supported media: autoselect
\end{alltt}

%size 4
Fjernes igen med:\\
\verb+ifconfig en1 inet6 -alias 2001:1448:81:beef::1+\\
Prøv også:\\ \verb+ifconfig en1 inet6+


\slide{GUI værktøjer - autoconfiguration}

\hlkimage{20cm}{osx-network-automatic.png}

\centerline{De fleste moderne operativsystemer er efterhånden opdateret med menuer til IPv6}

\slide{GUI værktøjer - manuel konfiguration}

\hlkimage{20cm}{osx-network-manual.png}

\centerline{Bemærk hvorledes subnetmaske nu blot er en prefix length}

\slide{ping til IPv6 adresser}


\begin{alltt}
\small 
root# ping6 ::1
PING6(56=40+8+8 bytes) ::1 --> ::1
16 bytes from ::1, icmp_seq=0 hlim=64 time=0.312 ms
16 bytes from ::1, icmp_seq=1 hlim=64 time=0.319 ms
^C
--- localhost ping6 statistics ---
2 packets transmitted, 2 packets received, 0% packet loss
round-trip min/avg/max = 0.312/0.316/0.319 ms
\end{alltt}

Nogle operativsystemer kalder kommandoen ping6, andre bruger blot ping

\slide{ping6 til global unicast adresse}


\begin{alltt}
\footnotesize
root# ping6 2001:1448:81:beef:20a:95ff:fef5:34df
PING6(56=40+8+8 bytes) 2001:1448:81:beef::1 --> 2001:1448:81:beef:20a:95ff:fef5:34df
16 bytes from 2001:1448:81:beef:20a:95ff:fef5:34df, icmp_seq=0 hlim=64 time=10.639 ms
16 bytes from 2001:1448:81:beef:20a:95ff:fef5:34df, icmp_seq=1 hlim=64 time=1.615 ms
16 bytes from 2001:1448:81:beef:20a:95ff:fef5:34df, icmp_seq=2 hlim=64 time=2.074 ms
^C
--- 2001:1448:81:beef:20a:95ff:fef5:34df ping6 statistics ---
3 packets transmitted, 3 packets received, 0% packet loss
round-trip min/avg/max = 1.615/4.776/10.639 ms
\end{alltt}


\slide{ ping6 til specielle adresser}


\begin{alltt}
\small
root# ping6 -I en1 ff02::1
PING6(56=40+8+8 bytes) fe80::230:65ff:fe17:94d1 --> ff02::1
16 bytes from fe80::230:65ff:fe17:94d1, icmp_seq=0 hlim=64 time=0.483 ms
16 bytes from fe80::20a:95ff:fef5:34df, icmp_seq=0 hlim=64 time=982.932 ms
16 bytes from fe80::230:65ff:fe17:94d1, icmp_seq=1 hlim=64 time=0.582 ms
16 bytes from fe80::20a:95ff:fef5:34df, icmp_seq=1 hlim=64 time=9.6 ms
16 bytes from fe80::230:65ff:fe17:94d1, icmp_seq=2 hlim=64 time=0.489 ms
16 bytes from fe80::20a:95ff:fef5:34df, icmp_seq=2 hlim=64 time=7.636 ms
^C
--- ff02::1 ping6 statistics ---
4 packets transmitted, 4 packets received, +4 duplicates, 0% packet loss
round-trip min/avg/max = 0.483/126.236/982.932 ms
\end{alltt}

\begin{list2}
%\item ff00::0         ipv6-mcastprefix
\item ff02::1         ipv6-allnodes
\item ff02::2         ipv6-allrouters
\item ff02::3         ipv6-allhosts
\end{list2}

\slide{Stop - tid til leg}

\begin{list1}
\item Der findes et trådløst netværk med IPv6
\item Join med en laptop og prøv at pinge lidt
\begin{enumerate}
\item Virker \verb+ping6 ::1+ eller \verb+ping ::1+, fortsæt
\item Virker kommando svarende til: \verb+ping6 -I en1 ff02::1+\\
- burde vise flere maskiner
\item Kig på dine egne adresser med: \verb+ipv6+ (Windows) eller \verb+ifconfig+ (Unix)
\item Prøv at trace i netværket
\end{enumerate}
\item Hvordan fik I IPv6 adresser?
\end{list1}

\slide{ router advertisement daemon}


\begin{alltt}
/etc/rtadvd.conf:
en0:
      :addrs#1:addr="2001:1448:81:b00f::":prefixlen#64:
en1:
      :addrs#1:addr="2001:1448:81:beef::":prefixlen#64:

root# /usr/sbin/rtadvd -Df en0 en1
root# sysctl -w net.inet6.ip6.forwarding=1
net.inet6.ip6.forwarding: 0 -> 1
\end{alltt}

\begin{list1}
\item Stateless autoconfiguration er en stor ting i IPv6
\item Kommandoen starter den i debug-mode og i forgrunden\\
- normalt vil man starte den fra et script 
\item Typisk skal forwarding aktiveres, som vist med BSD sysctl kommando
\end{list1}





\slide{IPv6 og andre services}


\begin{alltt}
\small
root# netstat -an | grep -i listen

tcp46  0  0  *.80             *.*    LISTEN
tcp4   0  0  *.6000           *.*    LISTEN
tcp4   0  0  127.0.0.1.631    *.*    LISTEN
tcp4   0  0  *.25             *.*    LISTEN
tcp4   0  0  *.20123          *.*    LISTEN
tcp46  0  0  *.20123          *.*    LISTEN
tcp4   0  0  127.0.0.1.1033   *.*    LISTEN
\end{alltt}

ovenstående er udført på Mac OS X


\slide{IPv6 output fra kommandoer - inet6 family}


\begin{alltt}
\small
root# netstat -an -f inet6

Active Internet connections (including servers)
Proto Recv Send  Local  Foreign   (state)
tcp46  0   0     *.80     *.*     LISTEN
tcp46  0   0     *.22780  *.*     LISTEN
udp6   0   0     *.5353   *.*                    
udp6   0   0     *.5353   *.*                    
udp6   0   0     *.514    *.*                    
icm6   0   0     *.*      *.*                    
icm6   0   0     *.*      *.*                    
icm6   0   0     *.*      *.* 
\end{alltt}

ovenstående er udført på Mac OS X og tilrettet lidt


\slide{IPv6 er default for mange services}


\begin{alltt}
\small
root# telnet localhost 80

{\color{security6blue}Trying ::1...
Connected to localhost.}
Escape character is '^]'.
GET / HTTP/1.0

HTTP/1.1 200 OK
Date: Thu, 19 Feb 2004 09:22:34 GMT
Server: Apache/2.0.43 (Unix)
Content-Location: index.html.en
Vary: negotiate,accept-language,accept-charset
...
\end{alltt}

\slide{IPv6 er default i OpenSSH}

\begin{alltt}
\small
hlk$ ssh -v localhost -p 20123
OpenSSH_3.6.1p1+CAN-2003-0693, SSH protocols 1.5/2.0, OpenSSL 0x0090702f
debug1: Reading configuration data /Users/hlk/.ssh/config
debug1: Applying options for *
debug1: Reading configuration data /etc/ssh_config
debug1: Rhosts Authentication disabled, originating port will not be trusted.{\color{security6blue}
debug1: Connecting to localhost [::1] port 20123.}
debug1: Connection established.
debug1: identity file /Users/hlk/.ssh/id_rsa type -1
debug1: identity file /Users/hlk/.ssh/id_dsa type 2
debug1: Remote protocol version 2.0, remote software version OpenSSH_3.6.1p1+CAN-2003-0693
debug1: match: OpenSSH_3.6.1p1+CAN-2003-0693 pat OpenSSH*
debug1: Enabling compatibility mode for protocol 2.0
debug1: Local version string SSH-2.0-OpenSSH_3.6.1p1+CAN-2003-0693
\end{alltt}



\slide{Apache access log}


\begin{alltt}
\footnotesize
root# tail -f access_log 
::1 - - [19/Feb/2004:09:05:33 +0100] "GET /images/IPv6ready.png 
HTTP/1.1" 304 0
::1 - - [19/Feb/2004:09:05:33 +0100] "GET /images/valid-html401.png
HTTP/1.1" 304 0
::1 - - [19/Feb/2004:09:05:33 +0100] "GET /images/snowflake1.png 
HTTP/1.1" 304 0
::1 - - [19/Feb/2004:09:05:33 +0100] "GET /~hlk/security6.net/images/logo-1.png
HTTP/1.1" 304 0
2001:1448:81:beef:20a:95ff:fef5:34df - - [19/Feb/2004:09:57:35 +0100] 
"GET / HTTP/1.1" 200 1456
2001:1448:81:beef:20a:95ff:fef5:34df - - [19/Feb/2004:09:57:35 +0100] 
"GET /apache_pb.gif HTTP/1.1" 200 2326
2001:1448:81:beef:20a:95ff:fef5:34df - - [19/Feb/2004:09:57:36 +0100]
"GET /favicon.ico HTTP/1.1" 404 209
2001:1448:81:beef:20a:95ff:fef5:34df - - [19/Feb/2004:09:57:36 +0100] 
"GET /favicon.ico HTTP/1.1" 404 209
\end{alltt}
\vskip 1cm
\centerline{Apache konfigureres nemt til at lytte på IPv6}

\slide{Apache HTTPD server}

\begin{list1}
\item 
\item Mange bruger HTTPD fra Apache projektet\\
  \link{http://httpd.apache.org} - netcraft siger omkring 70\%
\item konfigureres gennem \verb+httpd.conf+
\end{list1}

\begin{alltt}
Listen 0.0.0.0:80
Listen [::]:80
...
Allow from 127.0.0.1
Allow from 2001:1448:81:0f:2d:9ff:f86:3f 
Allow from 217.157.20.133
\end{alltt}





\slide{OpenBSD fast IPv6 adresse}

Netværkskonfiguration på OpenBSD - flere filer:
\begin{alltt}
\small
# cat /etc/hostname.sk0
inet 10.0.0.23 0xffffff00 NONE
inet6 2001:1448:81:30::2
# cat /etc/mygate
10.0.0.1
# grep 2001 /etc/rc.local
route add -inet6 default 2001:1448:81:30::1
# cat /etc/resolv.conf    
domain security6.net
lookup file bind
nameserver 212.242.40.3
nameserver 212.242.40.51
nameserver 2001:1448:81:30::10
\end{alltt}

\slide{Basal DNS opsætning}



\begin{alltt}
domain security6.net
nameserver 212.242.40.3
nameserver 212.242.40.51
nameserver 2001:1448:81:30::2
\end{alltt}

\begin{list1}    
\item \verb+/etc/resolv.conf+ angiver navneservere og søgedomæner
\item typisk indhold er domænenavn og IP-adresser for navneservere
\item Filen opdateres også automatisk på DHCP klienter
\item {\bf Husk at man godt kan slå AAAA records op over IPv4} 
\item NB: denne fil kan hedde noget andet på UNIX varianter!
\item eksempelvis \verb+/etc/netsvc.conf+
\end{list1}

\slide{DNS systemet}

\begin{list1}
\item Navneopslag på Internet - centralt for IPv6
\item Tidligere brugte man en {\bfseries hosts} fil\\
hosts filer bruges stadig lokalt til serveren - IP-adresser
\item UNIX: /etc/hosts
\item Windows \verb+c:\windows\system32\drivers\etc\hosts+
\item Eksempel: www.security6.net har adressen 217.157.20.131
\item skrives i database filer, zone filer
\end{list1}

\begin{alltt}
ns1     IN      A       217.157.20.130
        IN      AAAA    2001:618:433::1
www     IN      A       217.157.20.131
        IN      AAAA    2001:618:433::14
\end{alltt}

\slide{Mere end navneopslag}

\begin{list1}
  \item består af resource records med en type:
    \begin{list2}
\item adresser A-records
\item IPv6 adresser AAAA-records
\item autoritative navneservere NS-records
\item post, mail-exchanger MX-records
\item flere andre: md ,  mf ,  cname ,  soa ,
                  mb , mg ,  mr ,  null ,  wks ,  ptr ,
                  hinfo ,  minfo ,  mx ....
\end{list2}
\end{list1}
\begin{alltt}
        IN      MX      10      mail.security6.net.
        IN      MX      20      mail2.security6.net.
\end{alltt}


\slide{BIND DNS server}

\begin{list1}
\item Berkeley Internet Name Daemon server
\item Mange bruger BIND fra Internet Systems Consortium
   - altså Open Source
\item konfigureres gennem \verb+named.conf+
\item det anbefales at bruge BIND version 9
\end{list1}

\begin{list2}
\item \emph{DNS and BIND}, Paul Albitz \& Cricket Liu, O'Reilly, 4th
  edition Maj 2001 
\item \emph{DNS and BIND cookbook}, Cricket Liu, O'Reilly, 4th
  edition Oktober 2002 
\end{list2}

Kilde: \link{http://www.isc.org}

\slide{BIND konfiguration - et udgangspunkt}

\begin{alltt}
\small 
acl internals \{ 127.0.0.1; ::1; 10.0.0.0/24; \};
options \{
        // the random device depends on the OS !
        random-device "/dev/random"; directory "/namedb";
        {\bf listen-on-v6 { any; };}
        port 53; version "Dont know"; allow-query \{ any; \};
\};
view "internal" \{
   match-clients \{ internals; \}; recursion yes;
   zone "." \{
       type hint;   file "root.cache"; \};
   // localhost forward lookup
   zone "localhost." \{
        type master; file "internal/db.localhost";   \};
   // localhost reverse lookup from IPv4 address
   zone "0.0.127.in-addr.arpa" \{
        type master; file "internal/db.127.0.0"; notify no;   \};
...
\}
\end{alltt}



\slide{Routing forståelse - IPv6}
\begin{alltt}
\small
$ netstat -f inet6 -rn 
Routing tables

Internet6:
Destination                 Gateway           Flags      Netif 
default             fe80::200:24ff:fec1:58ac  UGc         en0
::1                         ::1               UH          lo0
2001:1448:81:cf0f::/64      link#4            UC          en0
2001:1448:81:cf0f::1        0:0:24:c1:58:ac   UHLW        en0
fe80::/64                   fe80::1           Uc          lo0
fe80::1                     link#1            UHL         lo0
fe80::/64                   link#4            UC          en0
fe80::20d:93ff:fe28:2812    0:d:93:28:28:12   UHL         lo0
fe80::/64                   link#5            UC          en1
fe80::20d:93ff:fe86:7c3f    0:d:93:86:7c:3f   UHL         lo0
ff01::/32                   ::1               U           lo0
ff02::/32                   ::1               UC          lo0
ff02::/32                   link#4            UC          en0
ff02::/32                   link#5            UC          en1
\end{alltt}



\slide{ IPv6 neighbor discovery protocol (NDP)}

\hlkimage{20cm}{ipv6-arp-ndp.pdf}

\begin{list1}
\item ARP er væk
\item NDP erstatter og udvider ARP, Sammenlign \verb+arp -an+ med \verb+ndp -an+
\item Til dels erstatter ICMPv6 således DHCP i IPv6, DHCPv6 findes dog
\item {\bf NB: bemærk at dette har stor betydning for firewallregler!}
\end{list1}

\slide{ARP vs NDP}

\begin{alltt}
\small
hlk@bigfoot:basic-ipv6-new$ arp -an        
? (10.0.42.1) at{\bf 0:0:24:c8:b2:4c} on en1 [ethernet]
? (10.0.42.2) at 0:c0:b7:6c:19:b on en1 [ethernet]
hlk@bigfoot:basic-ipv6-new$ ndp -an        
Neighbor                      Linklayer Address  Netif Expire    St Flgs Prbs
::1                           (incomplete)         lo0 permanent R      
2001:16d8:ffd2:cf0f:21c:b3ff:fec4:e1b6 0:1c:b3:c4:e1:b6 en1 permanent R      
fe80::1%lo0                   (incomplete)         lo0 permanent R      
fe80::200:24ff:fec8:b24c%en1 {\bf 0:0:24:c8:b2:4c}      en1 8h54m51s  S  R   
fe80::21c:b3ff:fec4:e1b6%en1  0:1c:b3:c4:e1:b6     en1 permanent R      
\end{alltt}







\slide{Fremtiden er nu}
\label{slide:future}

\begin{list1}
\item Det er sagt mange gange at nu skal vi igang med IPv6
\item Der er sket store ændringer fra starten af 2007 til nu
\item Hvor det i starten af 2007 var status quo er flere begyndt at presse på
\item Selv på version2.dk omtales det \link{http://www.version2.dk/artikel/6147}\\
\emph{Seks DNS-rodservere tænder for IPv6
ICANN har nu aktiveret IPv6 på seks af internettets 13 rodservere. Med det rigtige udstyr kan man nu køre helt uden om IPv4.}
\end{list1}

%%%%%%%%%%%%%%%%%%%%%%%%%%%%%%%%%%%%%%%%%%%%%%%%%%%%%%%%%%%%%%%%%%%%%%%
\slide{Hvorfor implementere IPv6 i jeres netværk?}

\begin{list1}
\item Addresserummet
\begin{list2}
\item end-to-end transparancy
\item 	nemmere administration
\end{list2}\item Autoconfiguration
\begin{list2}
\item stateless autoconfiguration
\item automatisk routerconfiguration! 
(router renumbering)
\end{list2}
\item Performance
\begin{list2}
\item simplere format
\item kortere behandlingstid i routere
\end{list2}
\item Fleksibilitet - generelt
\item Sikkerhed
\begin{list2}
\item IPsec er et krav!
\item Afsender IP-adressen ændres ikke igennem NAT!
\end{list2}
\end{list1}

\slide{Hvorfor migrere til IPv6?}

\begin{list1}
\item IPv4 er mere end 25 år gammel - fra 1981 til idag
\item Idag har folk ønsker/krav til kommunikationen
\begin{list2}
\item båndbredde
\item latency
\item Quality-of-service
\item sikkerhed
\end{list2}

\item Meget af dette er, eller kan, implementeres med IPv4 - men det bliver lappeløsninger

\item NB: IPv6 er designet til at løse SPECIFIKKE problemer
\end{list1}

\slide{The Internet has done this before!}


\begin{quote}
   Because all hosts can not be converted to TCP simultaneously, and
   some will implement only IP/TCP, it will be necessary to provide
   temporarily for communication between NCP-only hosts and TCP-only
   hosts.  To do this certain hosts which implement both NCP and IP/TCP
   will be designated as relay hosts.  These relay hosts will support
   Telnet, FTP, and Mail services on both NCP and TCP.  These relay
   services will be provided  beginning in November 1981, and will be
   fully in place in January 1982.

  
   Initially there will be many NCP-only hosts and a few TCP-only hosts,
   and the load on the relay hosts will be relatively light.  As time
   goes by, and the conversion progresses, there will be more TCP
   capable hosts, and fewer NCP-only hosts, plus new TCP-only hosts.
   But, presumably most hosts that are now NCP-only will implement
   IP/TCP in addition to their NCP and become "dual protocol" hosts.
   So, while the load on the relay hosts will rise, it will not be a
   substantial portion of the total traffic.
\end{quote}

\centerline{NCP/TCP Transition Plan November 1981 RFC-801}

%%%%%%%%%%%%%%%%%%%%%%%%%%%%%%%%%%%%%%%%%%%%%%%%%%%%%%%%%%%%%%%%%%%%%%%
\slide{Er IPv6 klar? - Korte svar - ja}

\begin{list1}
\item Det bruges idag aktivt, især i dele af verden der ikke har store dele
af v4 adresserummet 

\item Kernen af IPv6 er stabil

\item IPv6 er inkluderet i mange operativsystemer idag\\
AIX, Solaris, BSD'erne, Linux, Mac OS X og Windows XP
Cisco IOS, Juniper Networks\\
Juniper har haft hardware support for IPv6 i mange år!

\item IPv6 TCP/IP stakke til indlejrede systemer er klar

\item prøv at lave \verb+ping6 ::1+ på jeres maskiner - det er IPv6

\item Se listen over IPv6 implementationer på
http://playground.sun.com/ipv6/ipng-implementations.html
\end{list1}



\slide{IPv6 bruges idag}

\begin{list1}
\item Listen over brugere vokser konstant
\item Store nye netværk designes alle med IPv6
en liste kan eksempelvis ses på addressen:
\link{http://www.ipv6.ac.uk/gtpv6/eu.html}

\item Andre links kan vise statistik for internet og IPv4/IPv6
\item \link{http://www.bgpexpert.com/addrspace2007.php}
\item \link{https://wiki.caida.org/wiki/iic/bin/view/Main/WebHome}
\item \link{http://bgp.he.net/ipv6-progress-report.cgi}
\item Se også:
\link{http://www.eu.ipv6tf.org/}


\end{list1}

\slide{5 dårlige argumenter for ikke at bruge IPv6 nu}

\begin{list1}
\item Det er ikke færdigt\\
- IPv4 har ALDRIG været færdigt ;-)

\item Ikke nødvendigt\\
- man kan stikke hovedet i busken 

\item NAT løser alle problemer og er meget sikkert ...\\
- NAT er en lappeløsning 

\item Udskiftning af HELE infrastrukturen er for dyrt\\
- man opgraderer/udskifter jævnligt udstyr 

\item Vent til det er færdigt!\\
- man mister muligheden for at påvirke resultatet!
\end{list1}

\exercise{ex:ping6}
\exercise{ex:basic-dns-lookup6}



\slide{Færdig med IPv6}

\begin{list1}
\item I resten af kurset vil vi ikke betragte IPv6 eller IPv4 som noget specielt
\item Vi vil indimellem bruge det ene, indimellem det andet
\item Alle subnets er konfigureret ens på IPv4 og IPv6
\item Subnets som i IPv4 hedder prefix.45 vil således i IPv6 hedde noget med prefix:45:
\item At have ens routing på IPv4 og IPv6 vil typisk IKKE være tilfældet i praksis
\item Man kan jo lige så godt forbedre netværket mens man går over til IPv6 :-)
\end{list1}

\slide{Nu skal vi til management og diagnosticering}

\begin{list1}
\item Always check the spark plugs!
\item Når man skal spore fejl i netværk er det essentielt at starte fra bunden:
\begin{list2}
\item Er der link?
\item Er der IP-adresse?
\item Er der route?
\item Modtager systemet pakker
\item Er der en returvej fra systemet! Den her kan snyde mange!
\item Lytter serveren på den port man vil forbinde til, UDP/TCP
\end{list2} 
\item Hvis der ikke er link vil man aldrig få svar fra databasen/webserveren/postserveren
\end{list1}

\slide{Udtræk af netværkskonfigurationen}

\begin{list1}
\item De vigtigste kommandoer til udtræk af netværkskonfigurationen:
\begin{list2}
\item cat - til at vise tekstfiler
\item ifconfig - interface configuration
\item netstat - network statistics
\item lsof - list open files
\end{list2}
\end{list1}

\slide{Basale testværktøjer TCP - Telnet og OpenSSL}

\begin{list1}
\item Telnet blev tidligere brugt til login og er en klartekst forbindelse
over TCP
\item Telnet kan bruges til at teste forbindelsen til mange ældre serverprotokoller som benytter ASCII kommandoer
\begin{list2}
\item \verb+telnet mail.kramse.dk 25+ laver en forbindelse til port 25/tcp
\item \verb+telnet www.kramse.dk 80+ laver en forbindelse til port 80/tcp
\end{list2}
\item Til krypterede forbindelser anbefales det at teste med openssl
\begin{list2}
\item \verb+openssl s_client -host www.kramse.dk -port 443+\\
laver en forbindelse til port 443/tcp med SSL
\item \verb+openssl s_client -host mail.kramse.dk -port 993+\\
 laver en forbindelse til port 993/tcp med SSL
\end{list2}
\item Med OpenSSL i client-mode kan services tilgås med samme tekstkommandoer som med telnet
\end{list1}


\slide{Basale testværktøjer UDP}

\begin{list1}
\item UDP er lidt drilsk, for de fleste services er ikke \emph{ASCII protokoller}
\item Der findes dog en række testprogrammer, a la ping
\begin{list2}
\item nsping - name server ping
\item dhcping - dhcp server ping
\item ...
\end{list2}
\item Derudover kan man bruge de sædvanlige programmer som \verb+host+ til navneopslag osv.
\end{list1}


\slide{IP netværkstuning}

\hlkimage{8cm}{712a.png}

\begin{list1}
\item IP har eksisteret mange år
\item Vi har udskiftet langsommme forbindelser med hurtige forbindelser
\item Vi har udskiftet langsomme MHz maskiner med Quad-core GHz maskiner
\item IP var tidligere meget konservativt, for ikke at overbelaste modtageren
\item Billedet er en HP arbejdsstation med 19" skærm og en 60MHz HP PA-RISC processor
\end{list1}

\slide{Anbefalet netværkstuning - hvad skal tunes}

\begin{list1}
\item Der er visse indstillinger som tidligere var standard, de bør idag slås fra
\item En del er allerede tunet i nyere versioner af IP-stakkene, men check lige
\item Ideer til ting som skal slås fra:
\begin{list2}
\item broadcast ICMP, undgå smurfing
\item Source routing, kan måske omgå firewalls og filtre    
\end{list2}
\item Ideer til ting som skal slås til/ændres:
\begin{list2}
\item Bufferstørrelser - hvorfor have en buffer på 65535 bytes på en maskine med 32GB ram?
\item Nye funktioner som RFC-1323 TCP Extensions for High Performance
\end{list2}
\item Det anbefales at finde leverandørens vejledning til hvad der kan tunes
\end{list1}

\slide{Netværkskonfiguration med sysctl}

\begin{alltt}\small
	# tuning
	net.inet.tcp.recvspace=65535
	net.inet.tcp.sendspace=65535
	net.inet.udp.recvspace=65535
	net.inet.udp.sendspace=32768
	# postgresql tuning
	kern.seminfo.semmni=256
	kern.seminfo.semmns=2048
	kern.shminfo.shmmax=50331648\end{alltt}
\begin{list1}
\item På mange UNIX varianter findes et specielt tuningsprogram,
  sysctl
\item Findes blandt andet på alle BSD'erne: FreeBSD, OpenBSD, NetBSD
  og Darwin/OSX 
\item Ændringerne skrives ind i filen \verb+/etc/sysctl.conf+
\item På Linux erstatter det til dels konfiguration med echo \\
\verb+echo 1 > /proc/net/ip/forwarding+   
\item På AIX benyttes kommandoen network options \verb+no+ 
\end{list1}

\slide{Tuning}

\begin{list1}
\item Hvad er flaskehalsen for programmet?
\item I/O bundet - en enkelt disk eller flere
\item CPU bundet - regnekraften
\item Netværket - 10Mbit half-duplex adapter
\item Memory - begynder systemet at \emph{swappe} eller \emph{thrashe}
\item brug top og andre statistikprogrammer til at se disse data
\end{list1}


\slide{Måling af througput}

\begin{list1}
\item Når der skal tunes er det altid nødvendigt med en baseline
\item Man kan ikke begynde at tune ud fra subjektive målinger 
\item \emph{Det kører langsomt}, \emph{Svartiden er for høj}
\item Målinger der giver præcise tal er nødvendige, før og efter målinger!
\item Der findes et antal værktøjer til, blandt andet Iperf
\end{list1}

\slide{Målinger med Iperf}

\begin{alltt}\small
hlk@fluffy:hlk$ iperf -s
------------------------------------------------------------
Server listening on TCP port 5001
TCP window size: 64.0 KByte (default)
------------------------------------------------------------
[  4] local 10.0.42.23 port 5001 connected with 10.0.42.67 port 51148
[  4]  0.0-10.2 sec  6.95 MBytes  5.71 Mbits/sec
[  4] local 10.0.42.23 port 5001 connected with 10.0.42.67 port 51149
[  4]  0.0-10.2 sec  7.02 MBytes  5.76 Mbits/sec
\end{alltt}

Ovenstående er set fra server, client kaldes med \verb+iperf -c fluffy+

\slide{Stop - vi prøver i fællesskab Iperf}

\begin{list1}
\item Vi prøver lige Iperf sammen
\item hvis alle prøver samtidig giver det stor variation i resultaterne
\end{list1}





\slide{Apache benchmark og andre programmer}

\begin{alltt}
\footnotesize
hlk@bigfoot:hlk$ ab -n 100 http://www.kramse.dk/
This is ApacheBench, Version 2.0.41-dev <$Revision: 1.121.2.12 $> apache-2.0
Copyright (c) 1996 Adam Twiss, Zeus Technology Ltd, http://www.zeustech.net/
Copyright (c) 2006 The Apache Software Foundation, http://www.apache.org/

Benchmarking www.kramse.dk (be patient)...
...
\end{alltt}

\begin{list1}
\item Der findes specialiserede værktøjer til mange protokoller
\item Eksempelvis følger der et apache benchmark med Apache HTTPD serveren
\item Mange andre værktøjer til at simulere flere samtidige brugere
\end{list1}

\slide{Apache Benchmark output - 1 }

\begin{alltt}
\footnotesize
Server Software:        Apache
Server Hostname:        www.kramse.dk
Server Port:            80

Document Path:          /
Document Length:        7547 bytes

Concurrency Level:      1
Time taken for tests:   13.84924 seconds
Complete requests:      100
Failed requests:        0
Write errors:           0
Total transferred:      778900 bytes
HTML transferred:       754700 bytes
Requests per second:    7.64 #/sec (mean)
Time per request:       130.849 ms (mean)
Time per request:       130.849 ms (mean, across all concurrent requests)
Transfer rate:          58.08 Kbytes/sec received
\end{alltt}

\slide{Apache Benchmark output - 3}

\begin{alltt}
\footnotesize
Connection Times (ms)
              min  mean+/-sd median   max
Connect:       22   24   4.0     24      58
Processing:    96  105  33.0     99     421
Waiting:       63   71  32.7     65     386
Total:        119  130  33.5    124     446

Percentage of the requests served within a certain time (ms)
  50%    124
  66%    126
  75%    128
  80%    130
  90%    143
  95%    153
  98%    189
  99%    446
 100%    446 (longest request)
\end{alltt}



\exercise{ex:sysctl}
\exercise{ex:iperf}
\exercise{ex:apache-benchmark}

\slide{Antal pakker per sekund} 

\begin{list1}
\item Til tider er det ikke båndbredden som sådan man vil måle
\item Specielt for routere er det vigtigt at de kan behandle mange pakker per sekund, pps
\item Til dette kan man lege med det indbyggede Ping program i flooding mode
\item Når programmet kaldes (som systemadministrator) med \verb+ping -f server+ vil den sende ping pakker så hurtigt som netkortet tillader
\item Programmer der kan teste pakker per sekund kaldes generelt for blaster tools
\end{list1}

\slide{traceroute}

\begin{list1}
  \item traceroute programmet virker ved hjælp af TTL
\item levetiden for en pakke tælles ned i hver router på vejen og ved at sætte denne lavt
  opnår man at pakken \emph{timer ud} - besked fra hver router på vejen
\item default er UDP pakker, men på UNIX systemer er der ofte mulighed
  for at bruge ICMP
\end{list1}

\begin{alltt}
{\bfseries\$ traceroute 217.157.20.129}
traceroute to 217.157.20.129 (217.157.20.129), 
30 hops max, 40 byte packets
 1  safri (10.0.0.11)  3.577 ms  0.565 ms  0.323 ms
 2  router (217.157.20.129)  1.481 ms  1.374 ms  1.261 ms
\end{alltt}


\slide{traceroute - med UDP}

\begin{alltt}
\tiny
# {\bfseries tcpdump -i en0 host 217.157.20.129 or host 10.0.0.11}
tcpdump: listening on en0
23:23:30.426342 10.0.0.200.33849 > router.33435: udp 12 [ttl 1]
23:23:30.426742 safri > 10.0.0.200: icmp: time exceeded in-transit
23:23:30.436069 10.0.0.200.33849 > router.33436: udp 12 [ttl 1]
23:23:30.436357 safri > 10.0.0.200: icmp: time exceeded in-transit
23:23:30.437117 10.0.0.200.33849 > router.33437: udp 12 [ttl 1]
23:23:30.437383 safri > 10.0.0.200: icmp: time exceeded in-transit
23:23:30.437574 10.0.0.200.33849 > router.33438: udp 12
23:23:30.438946 router > 10.0.0.200: icmp: router udp port 33438 unreachable
23:23:30.451319 10.0.0.200.33849 > router.33439: udp 12
23:23:30.452569 router > 10.0.0.200: icmp: router udp port 33439 unreachable
23:23:30.452813 10.0.0.200.33849 > router.33440: udp 12
23:23:30.454023 router > 10.0.0.200: icmp: router udp port 33440 unreachable
23:23:31.379102 10.0.0.200.49214 > safri.domain:  6646+ PTR? \\
200.0.0.10.in-addr.arpa. (41)
23:23:31.380410 safri.domain > 10.0.0.200.49214:  6646 NXDomain* 0/1/0 (93)
14 packets received by filter
0 packets dropped by kernel
\end{alltt}

\slide{Værdien af traceroute}

\begin{list1}
\item Diagnosticering af netværksproblemer - formålet med traceroute
\item Indblik i netværkets opbygning!
\item Svar fra hosts - en modtaget pakke fremfor et \emph{sort hul}

\item Traceroute er ikke et angreb - det er også vigtigt at kunne
  genkende normal trafik!
\end{list1}

\slide{Network mapping}

\hlkimage{23cm}{images/network-example.pdf}

\begin{list1}
\item Ved brug af traceroute og tilsvarende programmer kan man ofte
  udlede topologien i det netværk man undersøger  
\end{list1}


\slide{Flere traceprogrammer}

\begin{list1}
\item mtr My traceroute - grafisk \link{http://www.bitwizard.nl/mtr/}
\item lft - \emph{layer four trace} benytter TCP SYN og FIN prober
\item trace ved hjælp af TCP og andre protokoller findes
\item paratrace - \emph{Parasitic Traceroute via Established TCP Flows
    and IPID Hopcount} 
\item Der findes webservices hvor man kan trace fra, eksempelvis: \link{http://www.samspade.org}
\end{list1}

\slide{TCPDUMP - protokolanalyse pakkesniffer}

\hlkimage{14cm}{images/tcpdump-manual.pdf}


\centerline{\link{http://www.tcpdump.org}
- både til Windows og UNIX}
\slide{tcpdump - normal brug}

\begin{list2}
  \item tekstmode
\item kan gemme netværkspakker i filer
\item kan læse netværkspakker fra filer
\item er de-facto standarden for at gemme netværksdata i filer
\end{list2}

\begin{alltt}
\tiny [root@otto hlk]# tcpdump -i en0
tcpdump: listening on en0
13:29:39.947037 fe80::210:a7ff:fe0b:8a5c > ff02::1: icmp6: router advertisement
13:29:40.442920 10.0.0.200.49165 > dns1.cybercity.dk.domain:  1189+[|domain]
13:29:40.487150 dns1.cybercity.dk.domain > 10.0.0.200.49165:  1189 NXDomain*[|domain]
13:29:40.514494 10.0.0.200.49165 > dns1.cybercity.dk.domain:  24765+[|domain]
13:29:40.563788 dns1.cybercity.dk.domain > 10.0.0.200.49165:  24765 NXDomain*[|domain]
13:29:40.602892 10.0.0.200.49165 > dns1.cybercity.dk.domain:  36485+[|domain]
13:29:40.648288 dns1.cybercity.dk.domain > 10.0.0.200.49165:  36485 NXDomain*[|domain]
13:29:40.650596 10.0.0.200.49165 > dns1.cybercity.dk.domain:  4101+[|domain]
13:29:40.694868 dns1.cybercity.dk.domain > 10.0.0.200.49165:  4101 NXDomain*[|domain]
13:29:40.805160 10.0.0.200 > mail: icmp: echo request
13:29:40.805670 mail > 10.0.0.200: icmp: echo reply
...
\end{alltt}

\slide{TCPDUMP syntaks - udtryk}

\begin{list1}
\item filtre til husbehov
  \begin{list2}
\item type - host, net og port
\item src pakker med afsender IP eller afsender port
\item dst pakker med modtager IP eller modtager port
\item host - afsender eller modtager 
\item proto - protokol: ether, fddi, tr, ip, ip6, arp,  rarp,  decnet,
tcp og udp
\end{list2}
\item IP adresser kan angives som dotted-decimal eller navne
\item porte kan angives med numre eller navne
\item komplekse udtryk opbygges med logisk and,  or,  not
\end{list1}

\slide{tcpdump udtryk eksempler}

\begin{list1}
  \item Host 10.1.2.3\\
Alle pakker hvor afsender eller modtager er 10.1.2.3
\item host 10.2.3.4 and not host 10.3.4.5\\
Alle pakker til/fra 10.2.3.4 undtagen dem til/fra 10.3.4.5\\
- meget praktisk hvis man er logget ind på 10.2.3.4 via netværk fra 10.3.4.5
\item host foo and not port ftp  and not  port  ftp-data\\
trafik til/fra maskine \emph{foo} undtagen hvis det er FTP trafik
\end{list1}

\slide{Wireshark - grafisk pakkesniffer}

\hlkimage{20cm}{images/wireshark-website.png} 

\centerline{\link{http://www.wireshark.org}}
\centerline{både til Windows og UNIX, tidligere kendt som Ethereal}

\slide{Programhygiejne!}

\begin{list1}
  \item {\color{red}Download, installer - kør!} - farligt!
\item Sådan gøres det:
  \begin{list2}
    \item download program OG signaturfil/MD5
\item verificer signatur eller MD5
\item installer
\item brug programmet
\item hold programmet opdateret!\\
Se eksempelvis teksten på hjemmesiden:\\
\emph{Wireshark 0.99.2 has been released. Several security-related vulnerabilities have been fixed and several new features have been added.}
  \end{list2}
\item NB: ikke alle programmer har signaturer :(
\item MD5 er en envejs hash algoritme - mere om det senere
\end{list1}


\slide{Brug af Wireshark}

\hlkimage{13cm}{images/ethereal-capture-options.png}

\centerline{Man starter med Capture - Options}

\slide{Brug af Wireshark}

\hlkimage{24cm}{images/ethereal-main-window.png}

\centerline{Læg mærke til filtermulighederne}

\exercise{ex:wireshark}

\slide{syslog}

\begin{list1}
\item syslog er system loggen på UNIX og den er effektiv
  \begin{list2}
\item man kan definere hvad man vil se og hvor man vil have det
  dirigeret hen
\item man kan samle det i en fil eller opdele alt efter programmer og
  andre kriterier
\item man kan ligeledes bruge named pipes - dvs filer i filsystemet
  som tunneller fra chroot'ed services til syslog i det centrale system! 
\item man kan nemt sende data til andre systemer
  \end{list2}
\item Hvis man vil lave en centraliseret løsning er følgende link
  vigtigt: \\
Tina Bird, Counterpane\\
\link{http://loganalysis.org}
\end{list1}

\slide{syslogd.conf eksempel}
\begin{alltt}
\small
*.err;kern.debug;auth.notice;authpriv.none;mail.crit    /dev/console
*.notice;auth,authpriv,cron,ftp,kern,lpr,mail,user.none /var/log/messages
kern.debug;user.info;syslog.info                        /var/log/messages
auth.info                                               /var/log/authlog
authpriv.debug                                          /var/log/secure
...
# Uncomment to log to a central host named "loghost".
#*.notice;auth,authpriv,cron,ftp,kern,lpr,mail,user.none        @loghost
#kern.debug,user.info,syslog.info                               @loghost
#auth.info,authpriv.debug,daemon.info                           @loghost
\end{alltt}

\slide{Andre syslogs syslog-ng}

\begin{list1}
\item der findes andre syslog systemer eksempelvis syslog-ng
\item konfigureres gennem \verb+/etc/syslog-ng/syslog-ng.conf+   
\item Eksempel på indholdet af filen kunne være:
\end{list1}

\begin{alltt}
\small 
options \{ 
        long_hostnames(off); 
        sync(0); 
        stats(43200); 
\};

source src { unix-stream("/dev/log"); internal(); pipe("/proc/kmsg"); };
destination messages { file("/var/log/messages"); };
destination console_all { file("/dev/console"); };
log { source(src); destination(messages); };
log { source(src); destination(console_all); };
\end{alltt}

\exercise{ex:syslogd-basic}

\slide{Logfiler og computer forensics}
\begin{list1}
\item Logfiler er en nødvendighed for at have et transaktionsspor
\item Logfiler er desuden nødvendige for at fejlfinde
\item Det kan være relevant at sammenholde logfiler fra:  
\begin{list2}
\item routere
\item firewalls
\item intrusion detection systemer
\item adgangskontrolsystemer
\item ...
\end{list2}
\item Husk - tiden er vigtig! Network Time Protocol (NTP) anbefales 
\item Husk at logfilerne typisk kan slettes af en angriber -
  hvis denne får kontrol med systemet
\end{list1}

\slide{Simple Network Management Protocol}

\begin{list1}
\item SNMP er en protokol der supporteres af de fleste professionelle
  netværksenheder, såsom switche, routere
\item hosts - skal slås til men følger som regel med
\item SNMP bruges til: 
  \begin{list2}
    \item \emph{network management}
    \item statistik
    \item rapportering af fejl - SNMP traps
  \end{list2}
\item {\bfseries sikkerheden baseres på community strings der sendes
    som klartekst ...}
\item det er nemmere at brute-force en community string end en
  brugerid/kodeord kombination
\end{list1}

\slide{SNMP - \emph{hacking}}

\vskip 2 cm

\begin{list1}
\item Simple Network Management Protocol
\item sikkerheden afhænger alene af en Community string SNMPv2
\item typisk er den nem at gætte:
  \begin{list2}
    \item public - default til at aflæse statistik
\item private - default når man skal ændre på enheden, skrive 
\item cisco
\item ...
  \end{list2}
\item Der findes lister og ordbøger på nettet over kendte default communities
\end{list1}

\slide{Systemer med SNMP}

\begin{list1}
  \item kan være svært at finde ... det er UDP 161
\item Hvis man finder en så prøv at bruge {\bfseries snmpwalk}
  programmet - det kan vise alle tilgængelige SNMP oplysninger fra den
  pågældende host 
\item det kan være en af måderne at identificere uautoriserede WLAN
  Access Points på - sweep efter port 161/UDP 
\item snmpwalk er et af de mest brugte programmer til at hente snmp
  oplysninger - i forbindelse med hackning og penetrationstest
\end{list1}

\slide{snmpwalk}

\begin{list1}
\item Typisk brug er:
\item \verb+snmpwalk -v 1 -c secret switch1+
\item \verb+snmpwalk -v 2c -c secret switch1+
\item Eventuelt bruges \verb+snmpget+ og \verb+snmpset+
\item Ovenstående er en del af Net-SNMP pakken, \link{http://net-snmp.sourceforge.net/}
\end{list1}

\exercise{ex:snmpwalk}

\slide{brute force}

\begin{list1}
\item hvad betyder bruteforcing?\\
afprøvning af alle mulighederne
\end{list1}

\begin{alltt}
\small
Hydra v2.5 (c) 2003 by van Hauser / THC <vh@thc.org>
Syntax: hydra [[[-l LOGIN|-L FILE] [-p PASS|-P FILE]] | [-C FILE]] 
[-o FILE] [-t TASKS] [-g TASKS] [-T SERVERS] [-M FILE] [-w TIME] 
[-f] [-e ns] [-s PORT] [-S] [-vV] server service [OPT]
Options:
  -S        connect via SSL
  -s PORT   if the service is on a different default port, define it here
  -l LOGIN  or -L FILE login with LOGIN name, or load several logins from FILE
  -p PASS   or -P FILE try password PASS, or load several passwords from FILE
  -e ns     additional checks, "n" for null password, "s" try login as pass
  -C FILE   colon seperated "login:pass" format, instead of -L/-P option
  -M FILE   file containing server list (parallizes attacks, see -T)
  -o FILE   write found login/password pairs to FILE instead of stdout
...  
\end{alltt}


\slide{Eksempler på SNMP og management}

\begin{list1}
\item Ofte foregår administration af netværksenheder via HTTP, Telnet eller SSH
\begin{list2}	
\item små dumme enheder er idag ofte web-enablet
\item bedre enheder giver både HTTP og kommandolinieadgang
\item de bedste giver mulighed for SSH, fremfor Telnet
\end{list2}
\end{list1}


\slide{Tobi Oetiker's MRTG The Multi Router Traffic Grapher}

\hlkimage{15cm}{rrdtool-demo.png}

\begin{list1}
\item Monitorering af SNMP enheder og grafer
\item Inkluderer en nem configmaker og benytter idag RRDTool til data
\item Hjemmesiden: \link{http://oss.oetiker.ch/mrtg/}
\end{list1}

\slide{RRDTool Round Robin Database Tool}

\hlkimage{12cm}{rrdtool-demo.png}

\begin{list1}
\item Round Robin Database Tool er en måde at gemme data på
\item Med RRDTool kan man derefter få lavet grafer
\item Typisk bruger man et andet værktøj som benytter RRDTool til data
\item \link{http://oss.oetiker.ch/rrdtool/doc/index.en.html}
\end{list1}

Kan bruges til temperaturmålinger og alt muligt andet

\slide{Smokeping}

\hlkimage{15cm}{smokeping-demo.jpg}

\begin{list1}
\item Måling af latency for netværksservice
\item Understøtter et stort antal prober: ICMP, DNS, HTTP, LDAP, SMTP, ...
\item Min SmokePing server \link{http://pumba.kramse.dk/smokeping/}
\item Hjemmesiden for SmokePing \link{http://oss.oetiker.ch/smokeping/}
\item Lavet af Tobias Oetiker og Niko Tyni
\end{list1}


\slide{Nagios}

\begin{list1}
\item Overvågningsværktøj der giver godt overblik
\begin{list2}	
\item Monitoring af diverse services (SMTP, POP3, HTTP, NNTP, PING, etc.)
\item Monitoring af host resources (processor load, disk and memory usage, running processes, log files, etc.)
\item Monitoring af andre ressourcer som temperatur
\item Simpel plugin design som gør det nemt at udvide
\item Kan sende e-mail, SMS m.v.
\end{list2}
\item Benyttes mange steder
\item Hjemmesiden for Nagios \link{http://www.nagios.org/}
\end{list1}

\slide{Stop - overvågningsværktøjer}

\begin{list1}
\item Brug lidt tid på at se på vores netværk
\item Valgfrit om I vil se på Administrationsinterface på switche, SNMP indstillinger eksempelvis
\item Eller Nagios og SmokePing på mine servere
\end{list1}
