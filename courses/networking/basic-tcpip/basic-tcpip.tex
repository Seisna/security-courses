\documentclass[Screen16to9,17pt]{foils}
\usepackage{zencurity-slides}

\externaldocument{\jobname-exercises}
\selectlanguage{english}

\begin{document}
% Come to this workshop and learn some TCP/IP. (sorry, I forgot to bring stuff to make your own Ethernet 2-3m network cable)

% First I will do a 30min introduction to TCP/IP version 4 and version 6, show and tell with the BornHack network as example. Then afterwards there will be small isolated networks to play with. Learn how DHCP works, configure a router, and connect it to the BornHack network and THE INTERNET!!!!!1111

% I will bring network hardware comparable to what you would use in a normal home network, but no other people are connected except you. You are in control and can experiment.

% Devices will be a range of cheap TP-Link devices, routers, switches etc. Some more capable GL.Inet routers - have wireguard etc. When workshop finishes you can borrow the devices until BornHack ends.

% I will also bring some cheap Ethernet cables for selling 10m/20m/30m - so you can connect your tent/village later


\mytitlepage
{Networking and TCP/IP for beginners}
{BornHack July 2024}


\hlkprofiluk

\slide{Goals for today}

\hlkimage{6cm}{bornhack-camp-2024.jpg}

\begin{list2}
\item Introduce basic TCP/IP terminology
\item Show the BornHack network
\item Describe how you can connect a router or switch to the network
\item Describe basics of TCP/IP in 30 minutes
\item Let you get some hands on with IP protocols
\end{list2}

Photo is NWWC camp at BornHack 2024, come by and say hi


\slide{Time schedule}

\begin{list2}
\item 15:00 - 15:45\\
Introduction and basics

\item 15:45 - 17:00\\
Connect to the network, play with TCP/IP, switches and routers.
\end{list2}

Note: even though I talk a lot about Unix and Linux, you can definitely run a lot of tools on Windows and Mac OS X. The basic tools are available like the built-in ones and Nmap

Command line tools are sometimes used in the slides, as they only show text where a GUI screenshot can be cluttered with a lot of information, feel free to find GUI tools and web sites with same functionality

\slide{Exercises}

Exercises are completely optional

\begin{list2}
\item Try ping and traceroute
\item See your own IP settings
\item Borrow a USB Ethernet and connect to a switch or router
\end{list2}

Linux is a toolbox I will use and participants are recommended to research virtual machines



\slide{Course Materials}

\begin{list2}
\item This material is in multiple parts:

\item Slide show - presentation - this file
\item Exercises - PDF which is used for this and other workshops
\item Additional resources from the internet are linked throughout
\item Wikipedia has a LOT of nice pages about IP protocols, for example:
\end{list2}

\begin{quote}\small
Transport Layer Security (TLS) is a cryptographic protocol designed to provide communications security over a computer network. The protocol is widely used in applications such as email, instant messaging, and voice over IP, but its use in securing HTTPS remains the most publicly visible.
\end{quote}
Source: \url{https://en.wikipedia.org/wiki/Transport_Layer_Security}



\slide{Prerequisites}

If you are interested in TCP/IP you are welcome

If you want to be an expert in IP and network security I recommend doing exercises

\begin{list1}
\item It is recommended to use virtual machines for the exercises
\item Network security and most internet related security work has the following requirements:
\begin{list2}
\item Network experience
\item TCP/IP principles - often in more detail than a common user
\item Programming is an advantage, for automating things
\item Some Linux and Unix knowledge is in my opinion a {\bf necessary skill} for infosec work\\
-- too many new tools to ignore, and lots found at sites like Github and Open Source written for Linux
\end{list2}
\end{list1}



\slide{Wifi Hardware}

If you want to do sniffing of wireless it will be an advantage to have a wireless USB network card. Make sure to play nice, and dont abuse knowledge!

\begin{list2}
\item The following are two recommended models:
\item TP-link TL-WN722N hardware version 2.0 cheap but only support 2.4GHz
\item Alfa AWUS036ACH 2.4GHz + 5GHz Dual-Band and high performing
\item Both work great in Kali Linux for our purposes, but are older models by now
\end{list2}

Unfortunately the vendors change models often enough that the above are hard to find. I recommend using your favourite search engine and research which cards work with Kali Linux and airodump-ng.

I have some available you can borrow


\slide{Book: Practical Packet Analysis (PPA)}


\hlkimage{6cm}{PracticalPacketAnalysis3E_cover.png}

\emph{Practical Packet Analysis,
Using Wireshark to Solve Real-World Network Problems}
by Chris Sanders, 3rd Edition
April 2017, 368 pp.
ISBN-13:
978-1-59327-802-1
\link{https://nostarch.com/packetanalysis3}

I recommend this book for people new to networking, it has been in HumbleBundle book bundles multiple times



\slide{Internet Today}

\hlkimage{10cm}{images/server-client.pdf}

\begin{list1}
\item Clients and servers, roots in the academic world
\item Protocols are old, some more than 20 years
\item Very few protocols where encrypted, today a lot has switched to HTTPS and TLS
\end{list1}

\slide{Internet is Open Standards!}

{\hlkbig \color{titlecolor}
We reject kings, presidents, and voting.\\
We believe in rough consensus and running code.\\
-- The IETF credo Dave Clark, 1992.}

\begin{list1}
\item Request for comments (RFC) -- a series of documents spanning decades
\item RFC, BCP, FYI, informational -- first ones from 1969!
\item Are not updated but status is changed to Obsoleted when new versions are published
\item Standards track:\\
Proposed Standard $\rightarrow$ Draft Standard $\rightarrow$ Standard
\end{list1}

\slide{Internetworking: history}

\begin{list2}
\item[1961]  L. Kleinrock, MIT packet-switching theory
\item[1962]  J. C. R. Licklider, MIT - notes
\item[1964]  Paul Baran: On Distributed Communications
\item[1969]  ARPANET 4 nodes
\item[1971]  14 nodes
\item[1973]  Design of Internet Protocols started
\item[1973]  Email is about 75\% of all ARPANET traffic
\item[1974]  TCP/IP: Cerf/Kahn: A protocol for Packet
        Network Interconnection
\item[1983]  EUUG $\rightarrow$ DKUUG/DIKU Denmark
\item[1988]  About 60.000 systems on the internet -
        The Morris Worm hits about 10\%
\item[2002] About 130 million Internet hosts
\item[2010] IANA reserved blocks 7\% (Maj 2010) - \link{http://www.potaroo.net/tools/ipv4/}
\end{list2}


\slide{What is the Internet}

\begin{list1}
\item Communication between humans - currently!
\item Based on TCP/IP
\begin{list2}
\item best effort
\item packet switching (IPv6 calls it packets, not datagram)
\item \emph{connection-oriented} TCP
\item \emph{connection-less} UDP
\end{list2}
\end{list1}

RFC-1958:
\begin{quote}
 A good analogy for the development of the Internet is that of
 constantly renewing the individual streets and buildings of a city,
 rather than razing the city and rebuilding it. The architectural
 principles therefore aim to provide a framework for creating
 cooperation and standards, as a small "spanning set" of rules that
 generates a large, varied and evolving space of technology.
\end{quote}



\slide{Common Address Space}

\vskip 2 cm
\hlkimage{13cm}{IP-address.pdf}

\begin{list2}
\item Internet is defined by the address space
\item IPv4 based on 32-bit addresses, example dotted decimal format 10.0.0.1
\item IPv6 very similar to IPv4 without NAT, 128-bit addresses in hex ::1, 2a06:d380:0:101::80
\end{list2}


\slide{How to use the Internet Protocols (IP)}

Names are used by humans
\begin{center}\hlkbig
www.kramse.org

hlk@kramse.org
\end{center}

Computers use the addresses

\begin{alltt}
www     IN      A       185.129.63.130
        IN      AAAA    2a06:d380:0:102::80
mail    IN      A       217.157.63.115
        IN      AAAA    2a06:d380:0:102::25
\end{alltt}





\slide{Documentation Prefix, IPv6 updates etc.}

Even documentation has its own prefix, RFC5737:
\begin{alltt}
The blocks 192.0.2.0/24 (TEST-NET-1), 198.51.100.0/24 (TEST-NET-2),
and 203.0.113.0/24 (TEST-NET-3) are provided for use in
documentation.
\end{alltt}

IPv6 listed in RFC3849 \verb+2001:DB8::/32+

See RFC3330 \emph{Special-Use IPv4 Addresses} which is updated by
RFC6890 \emph{Special-Purpose IP Address Registries} which in turn is updated by RFC8190\\
Use the web version of RFCs to surf back and forth \link{https://www.rfc-editor.org/rfc/rfc8190}


\slide{CIDR Classless Inter-Domain Routing}

\hlkimage{15cm}{CIDR-aggregation.pdf}

\begin{list2}
\item Subnet mask originally inferred by the class
\item Started to allocate multiple C-class networks - save remaining B-class\\
Resulted in routing table explosion - btw Stop using A, B, C
\item A subnet mask today is a row of 1-bit
\item Supernet, supernetting
\item 10.0.0.0/24 means the network 10.0.0.0 with 24 subnet bits (mask 255.255.255.0)
\item 2a06:d380:0:101::80/64 means the network with 64-bit prefix length
\end{list2}

\slide{Protocols: OSI and Internet models}

\hlkimage{11cm,angle=90}{images/compare-osi-ip.pdf}


\slide{Ethernet, cables}

\hlkimage{17cm}{ethernetLights.jpg}

\centerline{Show link, and activity -- blinkenlights}



\slide{MAC address}
%\hlkimage{10cm}{apple-oui.png}

\begin{alltt}
00-03-93   (hex)        Apple Computer, Inc.
000393     (base 16)    Apple Computer, Inc.
                        20650 Valley Green Dr.
                        Cupertino CA 95014
                        UNITED STATES
\end{alltt}
\begin{list1}
\item Network technologies use a layer 2 hardware address
\item Typically using 48-bit MAC addresses known from Ethernet MAC-48/EUI-48
\item First half is assigned to companies -- Organizationally Unique Identifier (OUI)
\item Using the OUI you can see which producer and roughly when a network chip was produced
\item \link{http://standards.ieee.org/regauth/oui/index.shtml}
\end{list1}


\slide{Bridges}

\begin{list1}
\item Ethernet is a broadcast technology data transmitted into the ether -- a cable
\item This limits how many devices to connect
\item Using bridges we can connect segments -- which copy between them if needed
\item It learns the devices on each side (MAC address)
\end{list1}

See also \link{http://en.wikipedia.org/wiki/ALOHAnet}


\slide{A switch}

\hlkimage{10cm}{switch-1.pdf}

\begin{list1}
\item Today we use switches, Don't buy a hub, not even for experimenting or sniffing
\item A switch can receive and send data on multiple ports at the same time
\item Performance only limited by the backplane and switching chips
\item Can also often route with the same speed and mirror packets
\end{list1}


\slide{Wireless }

\hlkimage{10cm}{WCG200v2_med.jpg}

\begin{list1}
\item A typical home router would have built-in 802.11 Access-Point (AP) and some Ethernet LAN ports
\end{list1}


\slide{Topologier og Spanning Tree Protocol}

\hlkimage{13cm}{switch-STP.pdf}

Se mere i bogen af Radia Perlman, \emph{Interconnections: Bridges, Routers, Switches, and Internetworking Protocols}

\slide{Core, Distribution og Access net}

\hlkimage{17cm}{core-dist.pdf}

\centerline{Det er ikke altid man har præcis denne opdeling, men den er ofte brugt}

\slide{Bridges and routers}

\hlkimage{17cm}{wan-network.pdf}


\slide{Packets of data}

\hlkimage{18cm}{ethernet-frame-1.pdf}
\begin{list1}
\item Looking into the hardware we see that data is laid out according to a structure -- frames and packets
\item Often a start and end signal of a frame -- like Ethernet
\item Today we talk about packets of 1500 bytes which is common in Ethernet
\end{list1}



\slide{IPv4 header - RFC-791 september 1981}

\begin{alltt}\footnotesize
    0                   1                   2                   3
    0 1 2 3 4 5 6 7 8 9 0 1 2 3 4 5 6 7 8 9 0 1 2 3 4 5 6 7 8 9 0 1
   +-+-+-+-+-+-+-+-+-+-+-+-+-+-+-+-+-+-+-+-+-+-+-+-+-+-+-+-+-+-+-+-+
   |Version|  IHL  |Type of Service|          Total Length         |
   +-+-+-+-+-+-+-+-+-+-+-+-+-+-+-+-+-+-+-+-+-+-+-+-+-+-+-+-+-+-+-+-+
   |         Identification        |Flags|      Fragment Offset    |
   +-+-+-+-+-+-+-+-+-+-+-+-+-+-+-+-+-+-+-+-+-+-+-+-+-+-+-+-+-+-+-+-+
   |  Time to Live |    Protocol   |         Header Checksum       |
   +-+-+-+-+-+-+-+-+-+-+-+-+-+-+-+-+-+-+-+-+-+-+-+-+-+-+-+-+-+-+-+-+
   |                       Source Address                          |
   +-+-+-+-+-+-+-+-+-+-+-+-+-+-+-+-+-+-+-+-+-+-+-+-+-+-+-+-+-+-+-+-+
   |                    Destination Address                        |
   +-+-+-+-+-+-+-+-+-+-+-+-+-+-+-+-+-+-+-+-+-+-+-+-+-+-+-+-+-+-+-+-+
   |                    Options                    |    Padding    |
   +-+-+-+-+-+-+-+-+-+-+-+-+-+-+-+-+-+-+-+-+-+-+-+-+-+-+-+-+-+-+-+-+

                    Example Internet Datagram Header
\end{alltt}
Source: \url{https://datatracker.ietf.org/doc/html/rfc791} and updated later

\slide{IPv6 header - RFC-1883 December 1995}
\vskip -1cm
\begin{alltt}\footnotesize
   +-+-+-+-+-+-+-+-+-+-+-+-+-+-+-+-+-+-+-+-+-+-+-+-+-+-+-+-+-+-+-+-+
   |Version| Traffic Class |           Flow Label                  |
   +-+-+-+-+-+-+-+-+-+-+-+-+-+-+-+-+-+-+-+-+-+-+-+-+-+-+-+-+-+-+-+-+
   |         Payload Length        |  Next Header  |   Hop Limit   |
   +-+-+-+-+-+-+-+-+-+-+-+-+-+-+-+-+-+-+-+-+-+-+-+-+-+-+-+-+-+-+-+-+
   |                                                               |
   +                                                               +
   |                                                               |
   +                         Source Address                        +
   |                                                               |
   +                                                               +
   |                                                               |
   +-+-+-+-+-+-+-+-+-+-+-+-+-+-+-+-+-+-+-+-+-+-+-+-+-+-+-+-+-+-+-+-+
   |                                                               |
   +                                                               +
   |                                                               |
   +                      Destination Address                      +
   |                                                               |
   +                                                               +
   |                                                               |
   +-+-+-+-+-+-+-+-+-+-+-+-+-+-+-+-+-+-+-+-+-+-+-+-+-+-+-+-+-+-+-+-+
\end{alltt}
%Source: \url{https://datatracker.ietf.org/doc/html/rfc1883} and obsoleted by\\
%\url{https://datatracker.ietf.org/doc/html/rfc8200}




\slide{Windows - ipconfig}

\hlkimage{14cm}{win-ipconfig-ipv6.png}

\slide{Windows -- control panel with DHCP}
\hlkimage{10cm}{win-control-panel-ipv6.png}

DHCP is responsible for giving you a dynamic address

\slide{Unix - practical examples ifconfig and ping}

\begin{alltt}\small
$ ifconfig en0
en0: flags=8863<UP,BROADCAST,SMART,RUNNING,SIMPLEX,MULTICAST> mtu 1500
        inet6 {\bf fe80::216:cbff:feac:1d9f%en0} prefixlen 64 scopeid 0x4
        inet {\bf 10.0.42.15} netmask 0xffffff00 broadcast 10.0.42.255
        inet6 {\bf 2001:16d8:dd0f:cf0f:216:cbff:feac:1d9f} prefixlen 64 autoconf
        ether 00:16:cb:ac:1d:9f
        media: autoselect (1000baseT <full-duplex>) status: active

$ ping6 ::1
PING6(56=40+8+8 bytes) ::1 --> ::1
16 bytes from ::1, icmp_seq=0 hlim=64 time=0.089 ms
16 bytes from ::1, icmp_seq=1 hlim=64 time=0.155 ms

$ traceroute6 2001:16d8:dd0f:cf0f::1
traceroute6 to 2001:16d8:dd0f:cf0f::1 (2001:16d8:dd0f:cf0f::1)
from 2001:16d8:dd0f:cf0f:216:cbff:feac:1d9f, 64 hops max, 12 byte packets
 1  2001:16d8:dd0f:cf0f::1  0.399 ms  0.371 ms  0.294 ms
\end{alltt}


\slide{The basic tools for countering threats}

{\Large Knowledge and insight}
\begin{list2}
\item Networks have end-points and conversations on multiple layers
\item Wireshark is advanced, try right-clicking different places
\item Name resolution includes low level MAC addresses, and IP - names
\end{list2}

\begin{list2}
\item Tcpdump format, built-in to many network devices
\item Remote packet dumps, like \verb+tcpdump –i eth0 –w packets.pcap+
\item Story: tcpdump was originally written in 1988 by Van Jacobson, Sally Floyd, Vern Paxson and Steven McCanne who were, at the time, working in the Lawrence Berkeley Laboratory Network Research Group\\
 \link{https://en.wikipedia.org/wiki/Tcpdump}
\end{list2}

\vskip 5mm

\centerline{\Large Great network security comes from knowing networks!}



\slide{Network Knowledge Needed}

To work with network security the following protocols are the bare minimum to know about.

\begin{list2}
\item ARP Address Resolution Protocol for IPv4
\item NDP Neighbor Discovery Protocol for IPv6
\item IPv4 \& IPv6 -- the basic packet fields source, destination,
\item ICMPv4 \& ICMPv6 Internet Control Message Protocol
\item UDP User Datagram Protocol
\item TCP Transmission Control Protocol
\item DHCP Dynamic Host Configuration Protocol
\item DNS Domain Name System
\end{list2}

\centerline{A little Linux knowledge is also {\bf highly recommended}}


\slide{UDP User Datagram Protocol}
\hlkimage{16cm}{udp-1.pdf}
Connectionless \url{https://en.wikipedia.org/wiki/User_Datagram_Protocol}


\slide{TCP Transmission Control Protocol}
\hlkimage{14cm}{tcp-1.pdf}

Connection-oriented \url{https://en.wikipedia.org/wiki/Transmission_Control_Protocol}


\slide{Well-Known Port Numbers}

\hlkimage{6cm}{iana1.jpg}

\begin{list1}
\item IANA maintains a list of magical numbers in TCP/IP
\item Lists of protocl numbers, port numers etc.
\item A few notable examples:
\begin{list2}
\item Port 25/tcp Simple Mail Transfer Protocol (SMTP)
\item Port 53/udp and 53/tcp Domain Name System (DNS)
\item Port 80/tcp Hyper Text Transfer Protocol (HTTP)
\item Port 443/tcp HTTP over TLS/SSL (HTTPS)
\end{list2}
\item Source: \link{http://www.iana.org}
\end{list1}


\slide{Whois -- Where do IP addresses come from}

\begin{list1}
\item A these magical numbers we use on the internet are administered by IANA \url{https://www.iana.org/}
\item They have handed out portions to the Region Internet Registries (RIR)
\begin{list2}
\item RIPE (Réseaux IP Européens)  \link{http://ripe.net}
\item ARIN American Registry for Internet Numbers \link{http://www.arin.net}
\item Asia Pacific Network Information Center \link{http://www.apnic.net}
\item LACNIC (Regional Latin-American and Caribbean IP Address Registry) - Latin America and some Caribbean Islands
\end{list2}
\item AFRINIC \url{https://afrinic.net/}
\item They are memberbased, and members are called Local Internet Registries (LIRs) og National Internet Registry (NIR)
\end{list1}


\slide{Ping}

\begin{list1}
\item ICMP - Internet Control Message Protocol
\item Benyttes til fejlbeskeder og til diagnosticering af forbindelser
\item ping programmet virker ved hjælp af ICMP ECHO request og
  forventer ICMP ECHO reply
\item
\begin{alltt}
\small {\bfseries
$ ping 192.168.1.1}
PING 192.168.1.1 (192.168.1.1): 56 data bytes
64 bytes from 192.168.1.1: icmp_seq=0 ttl=150 time=8.849 ms
64 bytes from 192.168.1.1: icmp_seq=1 ttl=150 time=0.588 ms
64 bytes from 192.168.1.1: icmp_seq=2 ttl=150 time=0.553 ms
\end{alltt}
\end{list1}

\slide{traceroute}

\begin{list1}
  \item traceroute programmet virker ved hjælp af TTL
\item levetiden for en pakke tælles ned i hver router på vejen og ved at sætte denne lavt
  opnår man at pakken \emph{timer ud} - besked fra hver router på vejen
\item default er UDP pakker, men på UNIX systemer er der ofte mulighed
  for at bruge ICMP
\item
\begin{alltt}
\small{\bfseries
$ traceroute 185.129.60.129}
traceroute to 185.129.60.129 (185.129.60.129),
30 hops max, 40 byte packets
 1  safri (10.0.0.11)  3.577 ms  0.565 ms  0.323 ms
 2  router (185.129.60.129)  1.481 ms  1.374 ms  1.261 ms
\end{alltt}
\end{list1}


\slide{NAT Network Address Translation}
\hlkimage{16cm}{nat-1.pdf}


\vskip 2 cm
\begin{list2}
\item NAT is used for connecting private networks to the Internet
\item NAT gateway replaces source address and forwards packets
\item A quick and dirty fix that keeps messing up networks and protocols
\item The NAT router/firewall has state tables
\end{list2}

\slide{RFC-1918 Private Networks}

\begin{list1}
\item There is a list of network prefixes anyone can use, for private networks:
\begin{list2}
\item 10.0.0.0    -  10.255.255.255  (10/8 prefix)
\item 172.16.0.0  -  172.31.255.255  (172.16/12 prefix)
\item 192.168.0.0 -  192.168.255.255 (192.168/16 prefix)
\end{list2}
\item Address Allocation for Private Internets RFC-1918 adresserne!
\item To use these typically there will be a NAT device in front
\end{list1}

\begin{alltt}
The blocks 192.0.2.0/24 (TEST-NET-1), 198.51.100.0/24 (TEST-NET-2),
and 203.0.113.0/24 (TEST-NET-3) are provided for use in
documentation.

169.254.0.0/16 has been ear-marked as the IP range to use for end node
auto-configuration when a DHCP server may not be found
\end{alltt}



\slide{Course Network}
.
\hlkrightpic{85mm}{-1cm}{sample-network.png}

\begin{list1}
\item I have a course network with me -- the whole of BornHack \smiley\\
which has the following information:
\begin{list2}
\item Juniper MX240 router -- in the basement, we can go see it
\item Juniper switches
\item Aruba APs wireless access-points
\item IPv4 addresses: 151.216.32.0/21 total 2048
\item IPv6 addresses route6: 2001:678:9ec::/48
\item Autonomous system number: AS208647 BornHack\\
\url{https://en.wikipedia.org/wiki/Autonomous_system_(Internet)}

\end{list2}
\end{list1}

You are encouraged to use the network

\slide{Wireshark - graphical network sniffer}

\hlkimage{13cm}{images/wireshark-http.png}

\centerline{Capture - Options, select a network interface}
\centerline{\link{http://www.wireshark.org}}


\slide{Your Privacy }

\hlkimage{18cm}{images/internet-browsing.pdf}

\begin{list2}
\item Your data travels far
\item Often crossing borders, virtually and literally
\end{list2}


\slide{Defense in depth}

%\hlkimage{10cm}{Bartizan.png}
\hlkimage{15cm}{medieval-clipart-5}
\centerline{Picture originally from: \url{http://karenswhimsy.com/public-domain-images}}


\slide{Network Segmentation -- Firewalls}

\begin{quote}\small
\$ firewall\\

1. (I) {\bf An internetwork gateway that restricts data communication traffic to and from one of the connected networks} (the one said to be "inside" the firewall) and thus protects that network's system resources against threats from the other network (the one that is said to be "outside" the firewall). (See: guard, security gateway.)

2. (O) {\bf A device or system that controls the flow of traffic between networks using differing security postures.} Wack, J. et al (NIST), "Guidelines on Firewalls and Firewall Policy", Special Publication 800-41, January 2002.

Tutorial: A firewall typically protects a smaller, secure network (such as a corporate LAN, or even just one host) from a larger network (such as the Internet). The firewall is installed at the point where the networks connect, and the firewall applies policy rules to control traffic that flows in and out of the protected network.
\end{quote}
{\footnotesize Source: RFC4949 \emph{Internet Security Glossary, Version 2}\\
\link{https://datatracker.ietf.org/doc/html/rfc4949} 2007}

\slide{Continued}
\begin{quote}\small
{\bf A firewall is not always a single computer.} For example, a firewall may consist of a pair of filtering routers and one or more proxy servers running on one or more bastion hosts, all connected to a small, dedicated LAN (see: buffer zone) between the two routers. The external router blocks attacks that use IP to break security (IP address spoofing, source routing, packet fragments), while proxy servers block attacks that would exploit a vulnerability in a higher-layer protocol or service. The internal router blocks traffic from leaving the protected network except through the proxy servers. The difficult part is defining criteria by which packets are denied passage through the firewall, because a firewall not only needs to keep unauthorized traffic (i.e., intruders) out, but usually also needs to let authorized traffic pass both in and out.
\end{quote}
{\footnotesize Source: RFC4949 \emph{Internet Security Glossary, Version 2}\\
\link{https://datatracker.ietf.org/doc/html/rfc4949} 2007}

\begin{list2}
\item Network layer, packet filters, application level, stateless, stateful
\item Firewalls are by design a choke point, natural place \\
to do network security monitoring!
\item Older but still interesting Cheswick chapter 2 PDF
\emph{A Security Review of Protocols:
Lower Layers}\\
\link{http://www.wilyhacker.com/}
\end{list2}






\slide{Modern Firewall Infrastructures}


\centerline{\hlkbig A firewall {\color{security6blue}blocks traffic} on a network}

\vskip 1 cm
\pause

\centerline{\hlkbig A firewall {\color{red}allows traffic} on a network}
{\small The interesting part is typically what it allows!}

\begin{list1}
\item A firewall infrastructure must:
\begin{list2}
\item Prevent attackers from entering
\item Prevent data exfiltration
\item Prevent worms, malware, virus from spreading in networks
\item Be part of an overall solution with ISP, routers, other firewalls, switched infrastructures,\\
  intrusion detection systems and the rest of the infrastructure
\item ...
\end{list2}
\end{list1}

\vskip 5mm
\centerline{Difficult -- and requires design and secure operations}



\slide{Open source based firewalls}
\begin{list2}
\item Linux firewalls IP tables, use command line tool ufw Uncomplicated Firewall!
\item Firewall GUIs on top of Linux -- lots! Some are also available as commercial ones
\item OpenBSD PF
\link{http://www.openbsd.org}
\item FreeBSD IPFW og IPFW2 \link{http://www.freebsd.org}
\item Mac OS X uses OpenBSD PF
\item FreeBSD has an older version of the OpenBSD PF, should really be renamed now
\end{list2}



\slide{Uncomplicated Firewall (UFW)}

\begin{alltt}\small
root@debian01:~# apt install ufw
...
root@debian01:~# ufw allow 22/tcp
Rules updated
Rules updated (v6)
root@debian01:~# ufw enable
Command may disrupt existing ssh connections. Proceed with operation (y|n)? y
Firewall is active and enabled on system startup
root@debian01:~# ufw status numbered
Status: active

     To                         Action      From
     --                         ------      ----
[ 1] 22/tcp                     ALLOW IN    Anywhere
[ 2] 22/tcp (v6)                ALLOW IN    Anywhere (v6)
\end{alltt}

\begin{list2}
\item Extremely easy to use -- I recommend and use the (Uncomplicated Firewall) UFW
\end{list2}



\slide{Firewalls are NOT Alone}

\hlkimage{15cm}{network-layers-1.png}

\centerline{Use Defense in Depth -- all layers have features}


\slide{Together with Firewalls - Virtual LAN (VLAN)}

\hlkimage{8cm}{vlan-portbased.pdf}

\begin{list1}
\item Managed switches often allow splitting into zones called virtual LANs
\item Most simple version is port based
\item Like putting ports 1-4 into one LAN and remaining in another LAN
\item Packets must traverse a router or firewall to cross between VLANs
\end{list1}

\slide{Virtual LAN (VLAN) IEEE 802.1q}

\hlkimage{16cm}{vlan-8021q.pdf}

\begin{list1}
\item Using IEEE 802.1q  VLAN tagging on Ethernet frames
\item Virtual LAN, to pass from one to another, must use a router/firewall
\item Allows separation/segmentation and protects traffic from many security issues
\end{list1}


\slide{ARP in IPv4}

\begin{center}
\colorbox{white}{\includegraphics[width=18cm]{images/arp-basic.pdf}}
\end{center}

%server 00:30:65:22:94:a1\\
%client 00:40:70:12:95:1c\\
%hacker 00:02:03:04:05:06\\


\slide{ARP request and reply}
\begin{list1}
\item {\bfseries ping 10.0.0.2} from server
\item ARP Address Resolution Protocol request/reply:
  \begin{list2}
  \item ARP request broadcasted on layer 2 - Who has 10.0.0.2 Tell 10.0.0.1
  \item ARP reply (from 10.0.0.2) 10.0.0.2 is at 00:40:70:12:95:1c
  \end{list2}
\item IP ICMP request/reply:
  \begin{list2}
    \item Echo (ping) request from 10.0.0.1 to 10.0.0.2
\item Echo (ping) reply from 10.0.0.2 to 10.0.0.1
\item ...
  \end{list2}
\item ARP is performed on Ethernet before IP can be transmitted
\end{list1}

\slide{ IPv6 neighbor discovery protocol (NDP)}

\hlkimage{20cm}{ipv6-arp-ndp.pdf}

\begin{list1}
\item NDP replaces ARP, compare \verb+arp -an+ and \verb+ndp -an+
\item RFC4861 Neighbor Discovery for IP version 6 (IPv6)
\end{list1}

\slide{Hello neighbors}

\begin{alltt}\small
$ ping6 -w -I en1 ff02::1
PING6(72=40+8+24 bytes) fe80::223:6cff:fe9a:f52c%en1 --> ff02::1
30 bytes from fe80::223:6cff:fe9a:f52c%en1: bigfoot
36 bytes from fe80::216:cbff:feac:1d9f%en1: mike.kramse.dk.
38 bytes from fe80::200:aaff:feab:9f06%en1: xrx0000aaab9f06
34 bytes from fe80::20d:93ff:fe4d:55fe%en1: harry.local
36 bytes from fe80::200:24ff:fec8:b24c%en1: kris.kramse.dk.
31 bytes from fe80::21b:63ff:fef5:38df%en1: airport5
32 bytes from fe80::216:cbff:fec4:403a%en1: main-base
44 bytes from fe80::217:f2ff:fee4:2156%en1: Base Station Koekken
35 bytes from fe80::21e:c2ff:feac:cd17%en1: arnold.local
\end{alltt}

\exercise{ex:debian-firewall}

\end{document}



\slide{Er TCP/IP interessant?}
\hlkimage{12cm}{kame-noanime-small.png}

\begin{list1}
\item IP er med i alle de gængse operativsystemer UNIX og Windows
\item Internet er overalt
\end{list1}

\slide{Formål: TCP/IP grundkursus}
\hlkimage{12cm}{images/sample-network.png}
\centerline{IP-baserede netværk}

\slide{Formål: mere specifikt}

\begin{list1}
\item At introducere IP familien af protokoller
\item Kendskab til almindeligt brugte programmer i disse miljøer\\
 - ping, traceroute, samt serverfunktioner Apache HTTP, BIND DNS m.v.
\item Gennemgang af netværksdesign ved hjælp af almindeligt brugte setups\\ - en skalamodel af internet
\end{list1}

\slide{Forudsætninger}

\begin{list1}
\item Dette er en workshop og fuldt udbytte kræver at
  deltagerne udfører praktiske øvelser
\item Kurset anvender OpenBSD til øvelser, men UNIX kendskab
er ikke nødvendigt
\item De fleste øvelser kan udføres fra en Windows PC
\item Øvelserne foregår via login til UNIX maskinen
\begin{list2}
\item Til penetrationstest og det meste Internet-sikkerhedsarbejde er der
følgende forudsætninger
\item Netværkserfaring
\item TCP/IP principper - ofte i detaljer
\item Programmmeringserfaring er en fordel
\item UNIX kendskab er ofte en {\bfseries nødvendighed}\\
- fordi de nyeste værktøjer er skrevet til UNIX i form af Linux og BSD
\vskip 3 mm
\end{list2}
\end{list1}


\slide{Kursusfaciliteter}

\begin{list1}
\item Der er opbygget et kursusnetværk med følgende primære systemer:
\begin{list2}
\item UNIX server Fiona med HTTP server og værktøjer
%\item Sun Solaris PC server ved navn Flaffy, Athlon 64 X2 med 2GB ram
\item UNIX boot CD'er eller VMware images - jeres systemer
\end{list2}
\item På UNIX serveren tillades login diverse kursusbrugere - kursus1,
  kursus2, kursus3, ... kodeordet er {\bf kursus}
\item Det er en fordel at benytte hver sin bruger, så man kan gemme scripts
\item På de resterende systemer kan benyttes brugeren {\bf kursus}
\end{list1}

\begin{alltt}
Login: {\bf kursus}
Password: {\bf kursus42}
\end{alltt}

\slide{Knoppix og BackTrack boot CD'er}

%\hlkimage{5cm}{images/auditor.jpg}

\begin{list1}
\item Vi bruger UNIX og SSH på kurset
\item I kan bruge en udleveret CD til at boote Linux på jeres
  arbejdsstation og derfra arbejde, eller I kan benytte Fiona
\item Brug CD'en eller VMware player til de grafiske værktøjer som Wireshark
\item CD'en er under en åben licens - må kopieres frit :-)
\item ISO image kan hentes fra mirrors
\item BackTrack \link{http://www.remote-exploit.org/backtrack.html}
%\item Knoppix Dansk \link{http://tyge.sslug.dk/knoppix/}
\item Til begyndere indenfor Linux anbefales Ubuntu eller Kubuntu til
  arbejdsstationer
\end{list1}

\slide{Stop - tid til check}

\begin{list1}
\item Er alle kommet
\item Har alle en PC med
\item Har alle et kabel eller trådløst netkort som virker
\item Der findes et trådløst netværk ved navn {\bf kamenet}
\item Mangler der strømkabler
\item Mangler noget af ovenstående, sæt nogen igang med at finde det
\end{list1}



\slide{UNIX starthjælp}
\begin{list1}
\item Da UNIX indgår er her et lille \emph{cheat sheet} til UNIX
\begin{list2}
\item DOS/Windows kommando - tilsvarende UNIX, og forklaring
\item dir - ls - står for list files, viser filnavne
\item del - rm - står for remove, sletter filer
%\item ren - mv - move flytter filer til nyt navn, rename
%\item md - mkdir - make directory, lav en mappe/katalog
\item cd - cd - change directory, skifter katalog
\item type - cat - concatenate, viser indholdet af tekstfiler
\item more - less - viser tekstfiler en side af gangen
\item attrib - chmod - change mode, ændrer rettighederne på filer
\end{list2}
\item Prøv bare:
  \begin{list2}
    \item {\bfseries ls} list, eller long listing med {\bfseries ls -l}
    \item {\bfseries cat /etc/hosts} viser hosts filen
\item {\bfseries chmod +x head.sh} - sæt execute bit på en fil så den
  kan udføres som et program med kommandoen \verb+./head.sh+
  \end{list2}
\end{list1}

\slide{Aftale om test af netværk}

{\bfseries Straffelovens paragraf 263 Stk. 2. Med bøde eller fængsel
  indtil 6 måneder
straffes den, som uberettiget skaffer sig adgang til en andens
oplysninger eller programmer, der er bestemt til at bruges i et anlæg
til elektronisk databehandling.}

Hacking kan betyde:
\begin{list2}
\item At man skal betale erstatning til personer eller virksomheder
\item At man får konfiskeret sit udstyr af politiet
\item At man, hvis man er over 15 år og bliver dømt for hacking, kan
  få en bøde - eller fængselsstraf i alvorlige tilfælde
\item At man, hvis man er over 15 år og bliver dømt for hacking, får
en plettet straffeattest. Det kan give problemer, hvis man skal finde
et job eller hvis man skal rejse til visse lande, fx USA og
Australien
\item Frit efter: \link{http://www.stophacking.dk} lavet af Det
  Kriminalpræventive Råd
\item Frygten for terror har forstærket ovenstående - så lad være!
\end{list2}


\slide{Agenda - dag 1 Basale begreber og mindre netværk}

\begin{list1}
\item Opstart - hvad er IP og TCP/IP
\item Adresser
\item Subnets og CIDR
\item TCP og UDP
\item Basal DNS
\item Lidt om hardware half/full-duplex

\end{list1}



\slide{Agenda - dag 2 IPv6, Management, diagnosticering}

\begin{list1}
\item IP version 6
\item ARP og NDP
\item Ping
\item Traceroute
\item Snifferprogrammer Tcpdump og Wireshark
\item Management
\item Tuning og perfomancemålinger
\item RRDTool og Smokeping
\item Overvågning og Nagios
\item Wireless 802.11
\end{list1}


\slide{Agenda - dag 3 Dynamiske protokoller og services}

\begin{list1}
\item Netværksservices og serverfunktioner
\item DNS protokoller og servere
\item HTTP protokoller og servere
\item Dynamisk routing: BGP og OSPF
\item Produktionsmodning af netværk
\item Netværksprogrammering: små utilityprogrammer og scripts
\end{list1}

\slide{Agenda - dag 4  Netværkssikkerhed og firewalls}

\begin{list1}
\item SSL Secure Sockets Layer
\item VLAN 802.1q
\item 802.1x portbaseret autentifikation
\item WPA Wi-Fi Protected Access
\item VPN protokoller og IPSec
\item VoIP introduktion
\item Mobile IP introduktion
\end{list1}

\slide{Agenda - dag 5 Netværksdesign og templates}

\begin{list1}
\item Netværksdesign
\item Infrastrukturer i praksis
\item Templates til almindeligt forekommende setups
\item Afslutning og opsummering på kursus
\vskip 1 cm
\item Udfyld meget gerne evalueringsskemaerne, tak
\end{list1}

% agenda slut


% days 1-5
\slide{Dag 1 Basale begreber og mindre netværk}


\hlkimage{22cm}{images/kursus-netvaerk.pdf}


\slide{Netværk til routning}

\hlkimage{27cm}{basic-ipv6-network.pdf}

\vskip 2cm


\slide{Internet idag}


\hlkimage{12cm}{images/server-client.pdf}

\begin{list1}
\item Klienter og servere
\item Rødder i akademiske miljøer
\item Protokoller der er op til 20 år gamle
\item Meget lidt kryptering, mest på http til brug ved e-handel 
\item Kurset omhandler udelukkende netværk baseret på IP protokollerne
\end{list1}

\slide{Internet er åbne standarder!}

{\hlkbig \color{titlecolor}
We reject kings, presidents, and voting.\\
We believe in rough consensus and running code.\\
-- The IETF credo Dave Clark, 1992.}

\begin{list1}
\item Request for comments - RFC - er en serie af dokumenter
\item RFC, BCP, FYI, informational\\
de første stammer tilbage fra 1969
\item Ændres ikke, men får status Obsoleted når der udkommer en nyere
  version af en standard
\item Standards track:\\
Proposed Standard $\rightarrow$ Draft Standard $\rightarrow$ Standard
\item  Åbne standarder = åbenhed, ikke garanti for sikkerhed
\end{list1}


\slide{Hvad er Internet}

\begin{list1}
\item Kommunikation mellem mennesker!
\item Baseret på TCP/IP
\begin{list2}
\item best effort
\item packet switching (IPv6 kalder det packets, ikke datagram)
\item forbindelsesorienteret, \emph{connection-oriented}
\item forbindelsesløs, \emph{connection-less}
\end{list2}
\end{list1}

RFC-1958:
\begin{quote}
 A good analogy for the development of the Internet is that of
 constantly renewing the individual streets and buildings of a city,
 rather than razing the city and rebuilding it. The architectural
 principles therefore aim to provide a framework for creating
 cooperation and standards, as a small "spanning set" of rules that
 generates a large, varied and evolving space of technology.
\end{quote}


\slide{IP netværk: Internettet historisk set}

\begin{list2}  
\item[1961]  L. Kleinrock, MIT packet-switching teori
\item[1962]  J. C. R. Licklider, MIT - notes 
\item[1964]  Paul Baran: On Distributed Communications
\item[1969]  ARPANET startes 4 noder
\item[1971]  14 noder
\item[1973]  Arbejde med IP startes
\item[1973]  Email er ca. 75\% af ARPANET traffik
\item[1974]  TCP/IP: Cerf/Kahn: A protocol for Packet
        Network Interconnection
\item[1983]  EUUG $\rightarrow$ DKUUG/DIKU forbindelse
\item[1988]  ca. 60.000 systemer på Internettet
        The Morris Worm rammer ca. 10\%
\item[2000]  Maj I LOVE YOU ormen rammer
%\item[2001]  August Code Red ~600.000 servere 
\item[2002]  Ialt ca. 130 millioner på Internet
\end{list2}

\slide{Internet historisk set -  anno 1969}
\hlkimage{10cm}{1969_4-node_map.png}
%size 2

\begin{list2}
\item Node 1: University of California Los Angeles
\item Node 2: Stanford Research Institute
\item Node 3: University of California Santa Barbara
\item Node 4: University of Utah
%\item Kilde: \link{http://www.zakon.org/robert/internet/timeline/}
\end{list2}

\slide{De tidlige notater om Internet}

\begin{list1}
\item L. Kleinrock \emph{Information Flow in Large Communication nets}, 1961
\item J.C.R. Licklider, MIT noter fra 1962 \emph{On-Line Man Computer
  Communication} 
\item Paul Baran, 1964 \emph{On distributed Communications}
12-bind serie af rapporter\\
\link{http://www.rand.org/publications/RM/baran.list.html}
\item V. Cerf og R. Kahn, 1974 
\emph{A protocol for Packet Network Interconnection}
IEEE Transactions on Communication, vol. COM-22, pp. 637-648, May 1974
\item De tidlige notater kan findes på nettet!
\end{list1}

Læs evt. mere i mit speciale \link{http://www.inet6.dk/thesis.pdf}

\slide{BSD UNIX}

\hlkimage{4cm}{implementation_freebsd.jpg}

\begin{list1}
  \item UNIX kildeteksten var nem at få fat i for universiteter og
  mange andre
\item Bell Labs/AT\&T var et telefonselskab - ikke et software hus
\item På Berkeley Universitetet blev der udviklet en del på UNIX og
  det har givet anledning til en hel gren kaldet BSD UNIX
\item BSD står for Berkeley Software Distribution
\item BSD UNIX har blandt andet resulteret i virtual memory management
  og en masse TCP/IP relaterede applikationer
\end{list1}

\slide{Open Source definitioner - uddrag}

\begin{list1}
\item Free Redistribution - der må ikke lægges begrænsninger på om
  softwaren gives væk eller sælges
\item Source Code - kildeteksten skal være tilgængelig 
\item Derived Works - det skal være muligt at arbejde videre på 
\item Integrity of The Author's Source Code - det skal være muligt at
  beskytte sit navn og rygte, ved at kræve ændret navn for
  afledte projekter
\item Softwaren kaldes ofte også Free Software, nogle bruger endda Libre
\item Eksempler er BSD licensen, Apache, GNU GPL og mange andre
\item Kilder: \link{http://www.opensource.org/}\\
\link{http://en.wikipedia.org/wiki/FLOSS} Free/Libre/Open-Source Software  
\end{list1}

\slide{BSD licensen er pragmatisk}

\begin{list1}
 \item BSD licensen kræver ikke at man offentliggør sine ændringer,
 man kan altså bruge BSD kildetekst og stadig lave et kommercielt
 produkt!
\item GNU GPL bliver af nogle omtalt som en virus - der
  \emph{inficerer} softwaren, og afledte projekter
\end{list1}



\slide{Hvad er Internet}

\begin{list1}
\item 80'erne IP/TCP starten af 80'erne
\item 90'erne IP version 6 udarbejdes
  \begin{list2}
  \item IPv6 ikke brugt i Europa og US
  \item IPv6 er ekstremt vigtigt i Asien 
  \item historisk få adresser tildelt til 3.verdenslande
  \item Større Universiteter i USA har ofte større allokering end Kina!
  \end{list2}
\item 1991 WWW "opfindes" af Tim Berners-Lee hos CERN
\item E-mail var hovedparten af traffik
  - siden overtog web/http førstepladsen
\end{list1}

\slide{Hvad er Internet}

\vskip 1 cm

\centerline{Antallet af hosts på Internet}

\hlkimage{16cm}{images/Count_Host.png}

\begin{list1}
\item Kilde: 
Hobbes' Internet Timeline v5.6\\
\link{http://www.zakon.org/robert/internet/timeline/}
\end{list1}

\slide{Hvad er Internet}

\vskip 1 cm

\centerline{Antallet af World Wide Web servere}

\hlkimage{16cm}{images/Count_WWW.png}  

\begin{list1}
\item Kilde: Hobbes' Internet Timeline v5.6\\
\link{http://www.zakon.org/robert/internet/timeline/}
\end{list1}


% IP-adresser

\slide{Fælles adresserum}

\vskip 2 cm
\hlkimage{17cm}{IP-address.pdf}

\begin{list1}
\item Hvad kendetegner internet idag
\item Der er et fælles adresserum baseret på 32-bit adresser
\item En IP-adresse kunne være 10.0.0.1
\end{list1}

\slide{IPv4 addresser og skrivemåde}

\begin{alltt}
hlk@bigfoot:hlk$ ipconvert.pl 127.0.0.1
Adressen er: 127.0.0.1
Adressen er: 2130706433
hlk@bigfoot:hlk$ ping 2130706433
PING 2130706433 (127.0.0.1): 56 data bytes
64 bytes from 127.0.0.1: icmp_seq=0 ttl=64 time=0.135 ms
64 bytes from 127.0.0.1: icmp_seq=1 ttl=64 time=0.144 ms
\end{alltt}

\begin{list1}
\item IP-adresser skrives typisk som decimaltal adskilt af punktum
\item Kaldes {\bf dot notation}: 10.1.2.3
\item Kan også skrive som oktal eller heksadecimale tal
\end{list1}



\slide{IP-adresser som bits}

\begin{alltt}
IP-adresse: 127.0.0.1
Heltal:	2130706433
Binary:	1111111000000000000000000000001
\end{alltt}

\begin{list1}
\item IP-adresser kan også konverteres til bits
\item Computeren regner binært, vi bruger dot-notationen
\end{list1}

\slide{Internet ABC}

\begin{list1}
\item Tidligere benyttede man klasseinddelingen af IP-adresser: A, B, C, D og E
\item Desværre var denne opdeling ufleksibel:
\begin{list2}
\item A-klasse kunne potentielt indeholde 16 millioner hosts
\item B-klasse kunne potentielt indeholder omkring 65.000 hosts
\item C-klasse kunne indeholde omkring 250 hosts
\end{list2}
\item Derfor bad de fleste om adresser i B-klasser - så de var ved at løbe tør!
\item D-klasse benyttes til multicast
\item E-klasse er blot reserveret
\item Se evt. \link{http://en.wikipedia.org/wiki/Classful\_network}
\end{list1}


\slide{CIDR Classless Inter-Domain Routing}

\hlkimage{15cm}{CIDR-aggregation.pdf}

\begin{list1}
\item Subnetmasker var oprindeligt indforstået
\item Dernæst var det noget man brugte til at opdele sit A, B eller C net med
\item Ved at tildele flere C-klasser kunne man spare de resterende B-klasser - men det betød en routing table explosion
\item Idag er subnetmaske en sammenhængende række 1-bit der angiver størrelse på nettet
\item 10.0.0.0/24 betyder netværket 10.0.0.0 med subnetmaske 255.255.255.0
\item Nogle få steder kaldes det tillige supernet, supernetting
\end{list1}


\slide{Subnet calculator, CIDR calculator}

\hlkimage{10cm}{subnet-calculator.png}

\begin{list1}
\item Der findes et væld af programmer som kan hjælpe med at udregne
subnetmasker til IPv4
\item Screenshot fra \link{http://www.subnet-calculator.com/}
\end{list1}


\slide{RFC-1918 private netværk}

\begin{list1}
\item Der findes et antal adresserum som alle må benytte frit:
\begin{list2}
\item 10.0.0.0    -  10.255.255.255  (10/8 prefix)
\item 172.16.0.0  -  172.31.255.255  (172.16/12 prefix)
\item 192.168.0.0 -  192.168.255.255 (192.168/16 prefix)
\end{list2}
\item Address Allocation for Private Internets RFC-1918 adresserne!
\item NB: man må ikke sende pakker ud på internet med disse som afsender, giver ikke mening
\end{list1}

\slide{IPv4 addresser opsummering}

\begin{list2}
\item Altid 32-bit adresser
\item Skrives typisk med 4 decimaltal dot notation 10.1.2.3
\item Netværk angives med CIDR Classless Inter-Domain Routing RFC-1519
\item CIDR notation 10.0.0.0/8 -
  fremfor 10.0.0.0 med subnet maske 255.0.0.0
\item Specielle adresser\\
127.0.0.1 localhost/loopback\\
0.0.0.0  default route
\item RFC-1918 angiver private adresser som alle kan bruge

\end{list2}


\slide{Stop - netværket idag}

\begin{list1}
\item Bemærk hvilket netværk vi bruger idag
\item Primære server fiona har IP-adressen 10.0.45.36
\item Primære router luffe har IP-adressen 10.0.45.2 (og flere andre)
\item Sekundære router idag er Bianca som har IP-adressen 10.0.46.2 (og flere andre)  
\item Hvis du kender til IP i forvejen så udforsk gerne på egen hånd netværket
\item Det er tilladt at logge ind på alle systemer, undtagen Henrik's laptop bigfoot :-)
\item {\bf Det er forbudt at ændre IP-konfiguration og passwords}
\item Nu burde I kunne forbinde jer til netværket fysisk, check med \verb+ping 10.0.45.2+
\item Det er nok at en PC i hver gruppe er på kursusnetværket
\end{list1}

\centerline{Pause for dem hvor det virker, mens vi ordner resten}


\slide{OSI og Internet modellerne}

\hlkimage{14cm,angle=90}{images/compare-osi-ip.pdf}


\slide{Netværkshardware}

\begin{list1}
\item Der er mange muligheder med IP netværk, IP kræver meget lidt
\item Ofte benyttede idag er:
\begin{list2}
\item Ethernet - varianter 10mbit, 100mbit, gigabit, 10 Gigabit findes, men er dyrt
\item Wireless 802.11 teknologier
\item ADSL/ATM teknologier til WAN forbindelser
\item MPLS ligeledes til WAN forbindelser
\end{list2}
\item Ethernet kan bruge kobberledninger eller fiber
\item WAN forbindelser er typisk fiber på grund af afstanden mellem routere
\item Tidligere benyttede inkluderer: X.25, modem, FDDI, ATM, Token-Ring
\end{list1}

\slide{Ethernet stik, kabler og dioder}

\hlkimage{20cm}{ethernetLights.jpg}

\centerline{Dioder viser typisk om der er link, hastighed samt aktivitet}

\slide{Trådløse teknologier}

\hlkimage{10cm}{WCG200v2_med.jpg}

\begin{list1}
\item Et typisk 802.11 Access-Point (AP) der har Wireless og Ethernet stik/switch
\end{list1}

\slide{MAC adresser}
%\hlkimage{10cm}{apple-oui.png}

\begin{alltt}
00-03-93   (hex)        Apple Computer, Inc.
000393     (base 16)    Apple Computer, Inc.
                        20650 Valley Green Dr.
                        Cupertino CA 95014
                        UNITED STATES
\end{alltt}
\begin{list1}
\item Netværksteknologierne benytter adresser på lag 2
\item Typisk svarende til 48-bit MAC adresser som kendes fra Ethernet MAC-48/EUI-48
\item Første halvdel af adresserne er Organizationally Unique Identifier (OUI)
\item Ved hjælp af OUI kan man udlede hvilken producent der har produceret netkortet
\item \link{http://standards.ieee.org/regauth/oui/index.shtml}
\end{list1}

\slide{Half/full-duplex og speed}

\hlkimage{20cm}{half-full-duplex.pdf}

\begin{list1}
\item Hvad hastighed overføres data med?
\item De fleste nyere Ethernet netkort kan køre i fuld-duplex
\item med full-duplex kan der både sendes og modtages data samtidigt
\item Ethernet kan benytte auto-negotiation - der ofte virker\\
Klart bedre i gigabitnetkort men pas på
\end{list1}





\slide{Broer og routere}

\hlkimage{20cm}{wan-network.pdf}
\centerline{Fysisk er der en begrænsing for hvor lange ledningerne må være}

\slide{Bridges}

\begin{list1}
\item Ethernet er broadcast teknologi, hvor data sendes ud på et delt medie - Æteren
\item Broadcast giver en grænse for udbredningen vs hastighed
\item Ved hjælp af en bro kan man forbinde to netværkssegmenter på layer-2
\item Broen kopierer data mellem de to segmenter
\item Virker som en forstærker på signalet, men mere intelligent
\item Den intelligente bro kender MAC adresserne på hver side
\item Broen kopierer kun hvis afsender og modtager er på hver sin side
\end{list1}

Kilde: For mere information søg efter Aloha-net\\ \link{http://en.wikipedia.org/wiki/ALOHAnet}


\slide{En switch}

\hlkimage{15cm}{switch-1.pdf}

\begin{list1}
\item Ved at fortsætte udviklingen kunne man samle broer til en switch
\item En switch idag kan sende og modtage på flere porte samtidig, og med full-duplex
\item Bemærk performance begrænses af backplane i switchen
\end{list1}

\slide{Topologier og Spanning Tree Protocol}

\hlkimage{18cm}{switch-STP.pdf}

Se mere i bogen af Radia Perlman, \emph{Interconnections: Bridges, Routers, Switches, and Internetworking Protocols}


\slide{Core, Distribution og Access net}

\hlkimage{20cm}{core-dist.pdf}

\centerline{Det er ikke altid man har præcis denne opdeling, men den er ofte brugt}




\slide{Pakker i en datastrøm}

\hlkimage{23cm}{ethernet-frame-1.pdf}
\begin{list1}
\item Ser vi data som en datastrøm er pakkerne blot et mønster lagt henover data 
\item Netværksteknologien definerer start og slut på en frame
\item Fra et lavere niveau modtager vi en pakke, eksempelvis 1500-bytes fra Ethernet driver
\end{list1}



\slide{IPv4 pakken - header - RFC-791}

\begin{alltt}
\small  
    0                   1                   2                   3   
    0 1 2 3 4 5 6 7 8 9 0 1 2 3 4 5 6 7 8 9 0 1 2 3 4 5 6 7 8 9 0 1 
   +-+-+-+-+-+-+-+-+-+-+-+-+-+-+-+-+-+-+-+-+-+-+-+-+-+-+-+-+-+-+-+-+
   |Version|  IHL  |Type of Service|          Total Length         |
   +-+-+-+-+-+-+-+-+-+-+-+-+-+-+-+-+-+-+-+-+-+-+-+-+-+-+-+-+-+-+-+-+
   |         Identification        |Flags|      Fragment Offset    |
   +-+-+-+-+-+-+-+-+-+-+-+-+-+-+-+-+-+-+-+-+-+-+-+-+-+-+-+-+-+-+-+-+
   |  Time to Live |    Protocol   |         Header Checksum       |
   +-+-+-+-+-+-+-+-+-+-+-+-+-+-+-+-+-+-+-+-+-+-+-+-+-+-+-+-+-+-+-+-+
   |                       Source Address                          |
   +-+-+-+-+-+-+-+-+-+-+-+-+-+-+-+-+-+-+-+-+-+-+-+-+-+-+-+-+-+-+-+-+
   |                    Destination Address                        |
   +-+-+-+-+-+-+-+-+-+-+-+-+-+-+-+-+-+-+-+-+-+-+-+-+-+-+-+-+-+-+-+-+
   |                    Options                    |    Padding    |
   +-+-+-+-+-+-+-+-+-+-+-+-+-+-+-+-+-+-+-+-+-+-+-+-+-+-+-+-+-+-+-+-+

                    Example Internet Datagram Header
\end{alltt}


\slide{IP karakteristik}

\begin{list1}
\item Fælles adresserum
\item Best effort - kommer en pakke fra er det fint, hvis ikke må højere lag klare det
\item Kræver ikke mange services fra underliggende teknologi \emph{dumt netværk}
\item Defineret gennem åben standardiseringsprocess og RFC-dokumenter
\end{list1}



\slide{Fragmentering og PMTU}

\hlkimage{20cm}{fragments-1.pdf}
\begin{list1}
\item Hidtil har vi antaget at der blev brugt Ethernet med pakkestørrelse på 1500 bytes
\item Pakkestørrelsen kaldes MTU Maximum Transmission Unit 
\item Skal der sendes mere data opdeles i pakker af denne størrelse, fra afsender
\item Men hvad hvis en router på vejen ikke bruger 1500 bytes, men kun 1000
\end{list1}

\slide{ICMP Internet Control Message Protocol}

\begin{list1}
\item Kontrolprotokol og fejlmeldinger
\item Nogle af de mest almindelige beskedtyper
\begin{list2}
\item echo
\item netmask
\item info
\end{list2}
\item Bruges generelt til \emph{signalering}
\item Defineret i RFC-792
\end{list1}

\centerline{\bf NB: nogle firewall-administratorer blokerer alt ICMP - det er forkert!}

\slide{ICMP beskedtyper}

\begin{list1}
\item Type
\begin{list2}
\item 0 = net unreachable;
\item 1 = host unreachable;
\item 2 = protocol unreachable;
\item 3 = port unreachable;
\item 4 = fragmentation needed and DF set;
\item 5 = source route failed.
\end{list2}
\item Ved at fjerne ALT ICMP fra et net fjerner man nødvendig funktionalitet!
\item Tillad ICMP types:
\begin{list2}
\item 3 Destination Unreachable
\item 4 Source Quench Message
\item 11 Time Exceeded
\item 12 Parameter Problem Message
\end{list2}
\end{list1}

\slide{Hvordan virker ARP?}

\begin{center}
\colorbox{white}{\includegraphics[width=18cm]{images/arp-basic.pdf}}  
\end{center}

%server 00:30:65:22:94:a1\\
%client 00:40:70:12:95:1c\\
%hacker 00:02:03:04:05:06\\

\slide{Hvordan virker ARP? - 2}
\begin{list1}
\item {\bfseries ping 10.0.0.2} udført på server medfører
\item ARP Address Resolution Protocol request/reply:
  \begin{list2}
  \item ARP request i broadcast - Who has 10.0.0.2 Tell 10.0.0.1
  \item ARP reply (fra 10.0.0.2) 10.0.0.2 is at 00:40:70:12:95:1c
  \end{list2}
\item IP ICMP request/reply:
  \begin{list2}
    \item Echo (ping) request fra 10.0.0.1 til 10.0.0.2
\item Echo (ping) reply fra 10.0.0.2 til 10.0.0.1
\item ...
  \end{list2}
\item ARP udføres altid på Ethernet før der kan sendes IP trafik
\item (kan være RARP til udstyr der henter en adresse ved boot)
\end{list1}


\slide{ARP cache}

\begin{alltt}
\small
hlk@bigfoot:hlk$ arp -an        
? (10.0.42.1) at 0:0:24:c8:b2:4c on en1 [ethernet]
? (10.0.42.2) at 0:c0:b7:6c:19:b on en1 [ethernet]
\end{alltt}

\begin{list1}
\item ARP cache kan vises med kommandoen \verb+arp -an+
\item -a viser alle
\item -n viser kun adresserne, prøver ikke at slå navne op - typisk hurtigere
\item ARP cache er dynamisk og adresser fjernes automatisk efter 5-20 minutter hvis de ikke bruges mere
\item Læs mere med \verb+man 4 arp+
\end{list1}


\slide{Manualsystemet}

\begin{quote}
 It is a book about a Spanish guy called Manual. You should read it.
       -- Dilbert
\end{quote}

\begin{list1}
\item Manualsystemet i UNIX er utroligt stærkt!
\item Det SKAL altid installeres sammen med værktøjerne!
\item Det er næsten identisk på diverse UNIX varianter!  
\item \verb+man -k+ søger efter keyword, se også \verb+apropos+
\end{list1}

Prøv \verb+man crontab+ og \verb+man 5 crontab+

\hlkimage{10cm}{images/unix-command-1.pdf}

\slide{En manualside}

\begin{alltt}
\small
CAL(1)                BSD General Commands Manual                CAL(1)
NAME
     cal - displays a calendar
SYNOPSIS
     cal [-jy] [[month]  year]
DESCRIPTION
   cal displays a simple calendar.  If arguments are not specified, the cur-
   rent month is displayed.  The options are as follows:
   -j      Display julian dates (days one-based, numbered from January 1).
   -y      Display a calendar for the current year.

The Gregorian Reformation is assumed to have occurred in 1752 on the 3rd
of September.  By this time, most countries had recognized the reforma-
tion (although a few did not recognize it until the early 1900's.)  Ten
days following that date were eliminated by the reformation, so the cal-
endar for that month is a bit unusual.

HISTORY
     A cal command appeared in Version 6 AT&T UNIX.  
\end{alltt}

\slide{Kommandolinien på UNIX}

\begin{list1}
\item Shells kommandofortolkere:
  \begin{list2}
    \item sh - Bourne Shell
\item bash - Bourne Again Shell
\item ksh - Korn shell, lavet af David Korn
\item csh - C shell, syntaks der minder om C sproget
\item flere andre, zsh, tcsh 
  \end{list2}
\item Svarer til command.com og cmd.exe på Windows
\item Kan bruges som komplette programmeringssprog
\end{list1}

\slide{Kommandoprompten}


\begin{alltt}    
\small
[hlk@fischer hlk]$ id
uid=6000(hlk) gid=20(staff) groups=20(staff), 
0(wheel), 80(admin), 160(cvs) 
[hlk@fischer hlk]$ 

[root@fischer hlk]# id
uid=0(root) gid=0(wheel) groups=0(wheel), 1(daemon),
2(kmem), 3(sys), 4(tty), 5(operator), 20(staff), 
31(guest), 80(admin) 
[root@fischer hlk]#
\end{alltt}

\begin{list1}  
\item typisk viser et dollartegn at man er logget ind som almindelig bruge
\item mens en havelåge at man er root - superbruger
\end{list1}

\slide{Kommandoliniens opbygning}


\begin{alltt}
echo [-n] [string ...]  
\end{alltt}

\begin{list1}
\item Kommandoerne der skrives på kommandolinien skrives sådan:
\begin{list2}
\item Starter altid med kommandoen, man kan ikke skrive \verb+henrik echo+
\item Options skrives typisk med bindestreg foran, eksempelvis \verb+-n+
\item Flere options kan sættes sammen, \verb+tar -cvf+ eller \verb+tar cvf+
\item I manualsystemet kan man se valgfrie options i firkantede
  klammer \verb+[]+
\item Argumenterne til kommandoen skrives typisk til sidst (eller der
  bruges redirection)
\end{list2}
\end{list1}


\slide{Adgang til UNIX}

\begin{center}
\includegraphics[width=4cm]{images/kde.png}
\includegraphics[width=4cm]{images/gnome-logo-large.png}
\end{center}

\begin{list1}
%\item Systemer der minder om UNIX kan idag nemt skaffes
\item Adgang til UNIX kan ske via grafiske brugergrænseflader
  \begin{list2}
%  \item X11 \link{http://www.x.org}
  \item KDE \link{http://www.kde.org}
  \item GNOME \link{http://www.gnome.org}
  \end{list2}
\item eller kommandolinien
\end{list1}
\centerline{\includegraphics[width=17cm]{images/unix-cmdline.pdf}}


\exercise{ex:putty-install}


\exercise{ex:winscp-install}

\exercise{ex:unix-login}

\exercise{ex:unix-cal}

\exercise{ex:sudo}



\exercise{ex:unix-boot-cd}




\exercise{ex:unix-basic-commands}



\slide{TCP/IP basiskonfiguration}

\begin{alltt}
ifconfig en0 10.0.42.1 netmask 255.255.255.0
route add default gw 10.0.42.1 
\end{alltt}

\begin{list1}
\item konfiguration af interfaces og netværk på UNIX foregår med:
\item \verb+ifconfig+, \verb+route+ og \verb+netstat+  
\item - ofte pakket ind i konfigurationsmenuer m.v.
\item fejlsøgning foregår typisk med \verb+ping+ og \verb+traceroute+
\item På Microsoft Windows benyttes ikke \verb+ifconfig+\\
men kommandoerne \verb+ipconfig+ og \verb+ipv6+
\end{list1}


\slide{Små forskelle}

\begin{alltt}
$ route add default 10.0.42.1
\emph{uden gw keyword!}

$ route add default gw 10.0.42.1 
\emph{Linux kræver gw med}
\end{alltt}

\vskip 1cm

\centerline{\bf NB: UNIX varianter kan indbyrdes være forskellige!}



\slide{Flere små forskelle}

\vskip 1cm 
\centerline{ping eller ping6}

\begin{list1}
\item Nogle systemer vælger at ping kommandoen kan ping'e både IPv4 og Ipv6
\item Andre vælger at \verb+ping+ kun benyttes til IPv4, mens IPv6 ping kaldes for \verb+ping6+
\item Læg også mærke til jargonen \emph{at pinge}
\end{list1}


\slide{OpenBSD}

Netværkskonfiguration på OpenBSD:
\begin{alltt}
# cat /etc/hostname.sk0
inet 10.0.0.23 0xffffff00 NONE
# cat /etc/mygate
10.0.0.1
# cat /etc/resolv.conf    
domain security6.net
lookup file bind
nameserver 212.242.40.3
nameserver 212.242.40.51
\end{alltt}

\slide{FreeBSD}

Netværkskonfiguration på FreeBSD \verb+/etc/rc.conf+:
\begin{alltt}
\small
# This file now contains just the overrides from /etc/defaults/rc.conf.
hostname="freebsd.security6.net
#ifconfig_vr0="DHCP"
ifconfig_vr0="inet 10.20.30.75 netmask 255.255.255.0"
router_enable="NO"
defaultrouter="10.20.30.65"
keyrate="fast"
moused_enable="YES"
ntpdate_enable="NO"
ntpdate_flags="none"
saver="blank"
sshd_enable="YES"
usbd_enable="YES"
...
\end{alltt}


\slide{GUI værktøjer - autoconfiguration}

\hlkimage{20cm}{osx-network-automatic.png}

\slide{GUI værktøjer - manuel konfiguration}

\hlkimage{20cm}{osx-network-manual.png}

\slide{ifconfig output}

\begin{alltt}\small
hlk@bigfoot:hlk$ ifconfig -a
lo0: flags=8049<UP,LOOPBACK,RUNNING,MULTICAST> mtu 16384
        inet 127.0.0.1 netmask 0xff000000 
        inet6 ::1 prefixlen 128 
        inet6 fe80::1%lo0 prefixlen 64 scopeid 0x1 
gif0: flags=8010<POINTOPOINT,MULTICAST> mtu 1280
stf0: flags=0<> mtu 1280
en0: flags=8863<UP,BROADCAST,SMART,RUNNING,SIMPLEX,MULTICAST> mtu 1500
        ether 00:0a:95:db:c8:b0 
        media: autoselect (none) status: inactive
        supported media: none autoselect 10baseT/UTP <half-duplex> 10baseT/UTP <full-duplex> 10baseT/UTP <full-duplex,hw-loopback> 100baseTX <half-duplex> 100baseTX <full-duplex> 100baseTX <full-duplex,hw-loopback> 1000baseT <full-duplex> 1000baseT <full-duplex,hw-loopback> 1000baseT <full-duplex,flow-control> 1000baseT <full-duplex,flow-control,hw-loopback>
en1: flags=8863<UP,BROADCAST,SMART,RUNNING,SIMPLEX,MULTICAST> mtu 1500
        ether 00:0d:93:86:7c:3f 
        media: autoselect (<unknown type>) status: inactive
        supported media: autoselect
\end{alltt}
%$
\vskip 1 cm
\centerline{ifconfig output er næsten ens på tværs af UNIX}




\slide{Vigtigste protokoller}


\begin{list1}
\item ARP Address Resolution Protocol
\item IP og ICMP Internet Control Message Protocol
\item UDP User Datagram Protocol
\item TCP Transmission Control Protocol
\item DHCP Dynamic Host Configuration Protocol 
\item DNS Domain Name System
\end{list1}
\vskip 1cm
\centerline{Ovenstående er omtrent minimumskrav for at komme på internet}

% allerede gennemgået ovenfor
%\slide{ICMP}

%\begin{list1}
%\item 	Internet Control Message Protocol 
%	Defineret i RFC-792

%\end{list1}


\slide{UDP User Datagram Protocol}
\hlkimage{20cm}{udp-1.pdf}
\begin{list1}
\item Forbindelsesløs RFC-768, \emph{connection-less} - der kan tabes pakker
\item Kan benyttes til multicast/broadcast - flere modtagere
\end{list1}



\slide{TCP Transmission Control Protocol}
\hlkimage{20cm}{tcp-1.pdf}

\begin{list1}
\item Forbindelsesorienteret RFC-791 September 1981, \emph{connection-oriented}
\item Enten overføres data eller man får fejlmeddelelse
\end{list1}




\slide{TCP three way handshake}

\hlkimage{7cm}{images/tcp-three-way.pdf}

\begin{list2}
\item {\bfseries TCP SYN half-open} scans
\item Tidligere loggede systemer kun når der var etableret en fuld TCP
  forbindelse - dette kan/kunne udnyttes til \emph{stealth}-scans
\item Hvis en maskine modtager mange SYN pakker kan dette fylde
  tabellen over connections op - og derved afholde nye forbindelser
  fra at blive oprette - {\bfseries SYN-flooding}
\end{list2}

\slide{Well-known port numbers}

\hlkimage{10cm}{iana1.jpg}

\begin{list1}
\item IANA vedligeholder en liste over magiske konstanter i IP
\item De har lister med hvilke protokoller har hvilke protokol ID m.v.
\item En liste af interesse er port numre, hvor et par eksempler er:
\begin{list2}
\item Port 25 SMTP Simple Mail Transfer Protocol
\item Port 53 DNS Domain Name System
\item Port 80 HTTP Hyper Text Transfer Protocol over TLS/SSL
\item Port 443 HTTP over TLS/SSL
\end{list2}
\item Se flere på \link{http://www.iana.org}
\end{list1}

\slide{Hierarkisk routing}

\hlkimage{18cm}{routing-1.pdf}
Hvordan kommer pakkerne frem til modtageren

\slide{IP default gateway}

\hlkimage{13cm}{routing-2.pdf}

\begin{list1}
\item IP routing er nemt
\item En host kender en default gateway i nærheden
\item En router har en eller flere upstream routere, få adresser den sender videre til
\item Core internet har default free zone, kender \emph{alle netværk} 
\end{list1}



\slide{DHCP Dynamic Host Configuration Protocol}

\hlkimage{13cm}{dhcp-1.pdf}

\begin{list1}
\item Hvordan får man information om default gateway
\item Man sender et DHCP request og modtager et svar fra en DHCP server
\item Dynamisk konfiguration af klienter fra en centralt konfigureret server
\item Bruges til IP adresser og meget mere
\end{list1}


\slide{Routing}


\begin{list1}
  \item routing table - tabel over netværkskort og tilhørende adresser
\item default gateway - den adresse hvortil man sender
  \emph{non-local} pakker\\kaldes også default route, gateway of last
  resort
\item routing styres enten manuelt - opdatering af route tabellen,
  eller konfiguration af adresser og subnet maske på netkort
\item eller automatisk ved brug af routing protocols - interne og
  eksterne route protokoller
\item de lidt ældre routing protokoller har ingen sikkerhedsmekanismer
\item {\bf IP benytter longest match i routing tabeller!}
\item Den mest specifikke route gælder for forward af en pakke!
\end{list1}


\slide{Routing forståelse}

\begin{alltt}
\small
$ netstat -rn
Routing tables

Internet:
Destination    Gateway         Flags  Refs      Use  Netif 
default        10.0.0.1        UGSc    23        7    en0
10/24          link#4          UCS      1        0    en0
10.0.0.1       0:0:24:c1:58:ac UHLW    24       18    en0  
10.0.0.33      127.0.0.1       UHS      0        1    lo0
10.0.0.63      127.0.0.1       UHS      0        0    lo0
127            127.0.0.1       UCS      0        0    lo0
127.0.0.1      127.0.0.1       UH       4     7581    lo0
169.254        link#4          UCS      0        0    en0  
\end{alltt}

\vskip 1 cm
\centerline{Start med kun at se på Destination, Gateway og Netinterface}


\exercise{ex:network-ifconfig}
\exercise{ex:network-netstat}
\exercise{ex:network-lsof}


\slide{whois systemet}

\begin{list1}
\item IP adresserne administreres i dagligdagen af et antal Internet
  registries, hvor de største er:
\begin{list2}
\item RIPE (Réseaux IP Européens)  \link{http://ripe.net}
\item ARIN American Registry for Internet Numbers \link{http://www.arin.net}
\item Asia Pacific Network Information Center \link{http://www.apnic.net}
\item LACNIC (Regional Latin-American and Caribbean IP Address Registry) - Latin America and some Caribbean Islands
\end{list2}
\item disse fire kaldes for Regional Internet Registries (RIRs) i
  modsætning til Local Internet Registries (LIRs) og National Internet
  Registry (NIR) 
\end{list1}

\slide{whois systemet-2}

\begin{list1}
\item ansvaret for Internet IP adresser ligger hos ICANN The Internet
  Corporation for Assigned Names and Numbers\\
\link{http://www.icann.org}
\item NB: ICANN må ikke forveksles med IANA Internet Assigned Numbers
  Authority \link{http://www.iana.org/} som bestyrer portnumre m.v.
\end{list1}

\exercise{ex:whois}


% basic ping og traceroute

\slide{Ping}

\begin{list1}
\item ICMP - Internet Control Message Protocol
\item Benyttes til fejlbeskeder og til diagnosticering af forbindelser
\item ping programmet virker ved hjælp af ICMP ECHO request og
  forventer ICMP ECHO reply
\item 
\begin{alltt}
\small {\bfseries 
$ ping 192.168.1.1}
PING 192.168.1.1 (192.168.1.1): 56 data bytes
64 bytes from 192.168.1.1: icmp_seq=0 ttl=150 time=8.849 ms
64 bytes from 192.168.1.1: icmp_seq=1 ttl=150 time=0.588 ms
64 bytes from 192.168.1.1: icmp_seq=2 ttl=150 time=0.553 ms
\end{alltt}
\end{list1}

\slide{traceroute}

\begin{list1}
  \item traceroute programmet virker ved hjælp af TTL
\item levetiden for en pakke tælles ned i hver router på vejen og ved at sætte denne lavt
  opnår man at pakken \emph{timer ud} - besked fra hver router på vejen
\item default er UDP pakker, men på UNIX systemer er der ofte mulighed
  for at bruge ICMP
\item 
\begin{alltt}
\small{\bfseries 
$ traceroute 217.157.20.129}
traceroute to 217.157.20.129 (217.157.20.129),
30 hops max, 40 byte packets
 1  safri (10.0.0.11)  3.577 ms  0.565 ms  0.323 ms
 2  router (217.157.20.129)  1.481 ms  1.374 ms  1.261 ms
\end{alltt}
\end{list1}

%DNS
\slide{Domain Name System}

\hlkimage{12cm}{dns-1.pdf}

\begin{list1}
\item Gennem DHCP får man typisk også information om DNS servere
\item En DNS server kan slå navne, domæner og adresser op
\item Foregår via query og response med datatyper kaldet resource records
\item DNS er en distribueret database, så opslag kan resultere i flere opslag
\end{list1}


\slide{DNS systemet}

\begin{list1}
\item navneopslag på Internet  
\item tidligere brugte man en {\bfseries hosts} fil\\
hosts filer bruges stadig lokalt til serveren - IP-adresser
\item UNIX: /etc/hosts
\item Windows \verb+c:\windows\system32\drivers\etc\hosts+
\item Eksempel: www.security6.net har adressen 217.157.20.131
\item skrives i database filer, zone filer
\end{list1}

\begin{alltt}
ns1     IN      A       217.157.20.130
        IN      AAAA    2001:618:433::1
www     IN      A       217.157.20.131
        IN      AAAA    2001:618:433::14
\end{alltt}

\slide{Mere end navneopslag}

\begin{list1}
  \item består af resource records med en type:
    \begin{list2}
\item adresser A-records
\item IPv6 adresser AAAA-records
\item autoritative navneservere NS-records
\item post, mail-exchanger MX-records
\item flere andre: md ,  mf ,  cname ,  soa ,
                  mb , mg ,  mr ,  null ,  wks ,  ptr ,
                  hinfo ,  minfo ,  mx ....
\end{list2}
\end{list1}
\begin{alltt}
        IN      MX      10      mail.security6.net.
        IN      MX      20      mail2.security6.net.
\end{alltt}

\slide{Basal DNS opsætning på klienter}

\begin{list1}    
\item \verb+/etc/resolv.conf+
\item NB: denne fil kan hedde noget andet på UNIX varianter!
\item eksempelvis \verb+/etc/netsvc.conf+
\item typisk indhold er domænenavn og IP-adresser for navneservere
\end{list1}

\begin{alltt}
domain security6.net
nameserver 212.242.40.3
nameserver 212.242.40.51
\end{alltt}

\slide{DNS root servere}
\begin{list1}
  \item Root-servere - 13 stk geografisk distribueret fordelt på Internet
\end{list1}

\begin{alltt}
I.ROOT-SERVERS.NET.     3600000 A       192.36.148.17
E.ROOT-SERVERS.NET.     3600000 A       192.203.230.10
D.ROOT-SERVERS.NET.     3600000 A       128.8.10.90
A.ROOT-SERVERS.NET.     3600000 A       198.41.0.4
H.ROOT-SERVERS.NET.     3600000 A       128.63.2.53
C.ROOT-SERVERS.NET.     3600000 A       192.33.4.12
G.ROOT-SERVERS.NET.     3600000 A       192.112.36.4
F.ROOT-SERVERS.NET.     3600000 A       192.5.5.241
B.ROOT-SERVERS.NET.     3600000 A       128.9.0.107
J.ROOT-SERVERS.NET.     3600000 A       198.41.0.10
K.ROOT-SERVERS.NET.     3600000 A       193.0.14.129
L.ROOT-SERVERS.NET.     3600000 A       198.32.64.12
M.ROOT-SERVERS.NET.     3600000 A       202.12.27.33  
\end{alltt}

\slide{DK-hostmaster}

\begin{list1}
\item bestyrer .dk TLD - top level domain
  
\item man registrerer ikke .dk-domæner hos DK-hostmaster, men hos en
  registrator 
\item Et domæne bør have flere navneservere og flere postservere
\item autoritativ navneserver - ved autoritativt om IP-adresse for
  maskine.domæne.dk findes 
\item ikke-autoritativ - har på vegne af en klient slået en adresse op
\item Det anbefales at overveje en service som
  \link{http://www.gratisdns.dk} der har 5 navneservere distribueret
  over stor geografisk afstand - en udenfor Danmark
\end{list1}

\slide{Navngivning af servere}

\begin{list1}
  \item Hvordan skal vi kunne huske og administrere servere?
\item Det er ikke nemt at navngive hverken brugere eller servere!
\item Selvom det lyder smart med A01S13, som forkortelse af Afdeling
  01's Server nr 13, er det umuligt at huske
\item ... men måske nødvendigt i de største netværk
  \begin{list2}
  
\item Windows serveren er domænecontroller - skal hedde:
\item Linux server som er terminalserver - skal hedde:
\item PC-system med NetBSD skal måske være vores ene server - skal hedde: ?
\item PC-system 1 med en Linux server - skal hedde: 
\item PC-system 2 med en Linux server - skal hedde:
  \end{list2}
\end{list1}



% NAT
\slide{NAT Network Address Translation}
\hlkimage{20cm}{nat-1.pdf}


\vskip 2 cm
\begin{list2}
\item NAT bruges til at forbinde et privat net (RFC-1918 adresser) med internet
\item NAT gateway udskifter afsender adressen med sin egen
\item En quick and dirty fix der vil forfølge os for resten af vores
  liv 
\item Ødelægger en del protokoller :-(
\item Lægger state i netværket - ødelægger fate sharing  
\end{list2}




\slide{NAT is BAD}


\hlkimage{20cm}{nat-is-bad.pdf}


\begin{list2}
\item NAT ødelægger end-to-end transparency!
\item Problemer med servere bagved NAT
\item "løser" problemet "godt nok" (tm) for mange
\item Men idag ser vi multilevel NAT! - eeeeeeewwwwww!
\item Se RFC-2775 Internet Transparency for mere om dette emne
\end{list2}


\exercise{ex:ping}
\exercise{ex:icmpush}
\exercise{ex:basic-dns-lookup}



\slide{Dag 2 IPv6, Management, diagnosticering}

%\hlkimage{18cm}{nagios-status-overview.jpg}
\hlkimage{18cm}{cricket-mini-graph.png}


% IPv6
\slide{IPv4 Adresserummet er ved at løbe ud}

\begin{list1}
\item Adresserummet er ved at løbe ud! faktum!
\item 32-bit - der ikke kan udnyttes fuldt ud
\item Tidligere brugte man begreberne A,B og C klasser af IP-adresser
\begin{list2}
\item 10.0.0.0    -  10.255.255.255  (10/8 prefix)
\item 172.16.0.0  -  172.31.255.255  (172.16/12 prefix)
\item 192.168.0.0 -  192.168.255.255 (192.168/16 prefix)
\end{list2}
\item Address Allocation for Private Internets RFC-1918 adresserne!
\item Husk at idag benyttes Classless Inter-Domain Routing CIDR\\
\link{http://en.wikipedia.org/wiki/Classless_Inter-Domain_Routing}
\item Notation: 192.168.1.0/24\\
det sædvanlige hjemmenet med subnet maske 255.255.255.0
\end{list1}

\slide{Status idag}
\hlkimage{10cm}{map_of_the_internet.jpg}

\slide{Tidslinie for IPv6 (forkortet)}

\begin{list2}
\item 1990 Vancouver IETF meeting
det estimeres at klasse B vil løbe ud ca. marts 1994

\item 1990 ultimo 
initiativer til at finde en afløser for IPv4

\item 1995 januar 
RFC-1752 Recommendation for the IP NG Protocol

\item 1995 september 
RFC-1883, RFC-1884, RFC-1885, RFC-1886 1. generation 

\item 1998 10. august 
"core" IPv6 dokumenter bliver Draft Standard
\item 
Kilde: RFC-2460, RFC-2461, RFC-2463, RFC-1981 - m.fl.
\end{list2}

\slide{IPv6: Internet redesigned? - nej!}
 
\begin{list1}
\item Målet var at bevare de gode egenskaber
\begin{list2}
\item basalt set Internet i gamle dage
\item back to basics!
\item fate sharing
\item kommunikationen afhænger ikke af state i netværket
\item end-to-end transparency
\end{list2}
\item Idag er Internet blevet en nødvendighed for mange!
\end{list1}

\centerline{\bf IP er en forretningskritisk ressource}

IPv6 basis i RFC-1752 The Recommendation for the IP Next Generation Protocol




\slide{KAME - en IPv6 reference implementation}

\hlkimage{6cm}{kame-noanime-small.png}

%center
\centerline{\link{http://www.kame.net}}

\begin{list2}
\item Er idag at betragte som en reference implementation\\
- i stil med BSD fra Berkeley var det
\item KAME har været på forkant med implementation af draft dokumenter
\item KAME er inkluderet i OpenBSD, NetBSD, FreeBSD og BSD/OS
- har været det siden version 2.7, 1.5, 4.0 og 4.2

\item Projektet er afsluttet, men nye projekter fortsætter i
WIDE regi \link{http://www.wide.ad.jp/}
\item Der er udkommet to bøger som i detaljer gennemgår IPv6 protokollerne i KAME
\end{list2}

\slide{Hvordan bruger man IPv6}


\begin{center}
\hlkbig
\vskip 2 cm
www.inet6.dk

hlk@inet6.dk

\end{center}

\pause
DNS AAAA record tilføjes

\begin{alltt}
www     IN A    91.102.91.17
        IN AAAA 2001:16d8:ff00:12f::2
mail    IN A    91.102.91.17
        IN AAAA 2001:16d8:ff00:12f::2
\end{alltt}

\slide{IPv6 addresser og skrivemåde}

\hlkimage{20cm}{ipv6-address-1.pdf}

\begin{list2}
\item 128-bit adresser, subnet prefix næsten altid 64-bit
\item skrives i grupper af 4 hexcifre ad gangen adskilt af kolon :
\item foranstillede 0 i en gruppe kan udelades, en række 0 kan erstattes med ::
\item dvs 0:0:0:0:0:0:0:0 er det samme som \\
0000:0000:0000:0000:0000:0000:0000:0000
\item Dvs min webservers IPv6 adresse kan skrives som: 
2001:16d8:ff00:12f::2
\item Specielle adresser:
::1 localhost/loopback og
::  default route
\item Læs mere i RFC-3513
\end{list2}

\slide{IPv6 addresser - prefix notation}


\begin{list1}
\item CIDR Classless Inter-Domain Routing RFC-1519
\item Aggregatable Global Unicast
\item 2001::/16 RIR subTLA space
\begin{list2}
\item 2001:200::/23 APNIC
\item 2001:400::/23 ARIN
\item 2001:600::/23 RIPE
\end{list2}
\item 2002::/16 6to4 prefix
\item 3ffe::/16 6bone allocation
\item link-local unicast addresses\\
fe80::/10 genereres udfra MAC addresserne EUI-64
\end{list1}



%%%%%%%%%%%%%%%%%%%%%%%%%%%%%%%%%%%%%%%%%%%%%%%%%%%%%%%%%%%%%%%%%%%%%%%
\slide{IPv6 addresser - multicast}

\begin{list1}
\item Unicast - identificerer ét interface 
pakker sendes til en modtager

\item Multicast - identificerer flere interfaces
pakker sendes til flere modtagere

\item Anycast - indentificerer en "gruppe"
en pakke sendes til et vilkårligt interface med denne adresse typisk det nærmeste

\item Broadcast?
er væk, udeladt, finito, gone!
	
\item Husk også at site-local er deprecated, se RFC-3879
\end{list1}

%%%%%%%%%%%%%%%%%%%%%%%%%%%%%%%%%%%%%%%%%%%%%%%%%%%%%%%%%%%%%%%%%%%%%%%
\slide{IPv6 pakken - header - RFC-2460}

\begin{list2}
\item Simplere - fixed size - 40 bytes
\item Sjældent brugte felter (fra v4) udeladt (kun 6 vs 10 i IPv4)
\item Ingen checksum!
\item Adresser 128-bit
\item 64-bit aligned, alle 6 felter med indenfor første 64
\end{list2}

Mindre kompleksitet for routere på vejen medfører
mulighed for flere pakker på en given router

%%%%%%%%%%%%%%%%%%%%%%%%%%%%%%%%%%%%%%%%%%%%%%%%%%%%%%%%%%%%%%%%%%%%%%%
\slide{IPv6 pakken - header - RFC-2460}


\begin{alltt}
\footnotesize

   +-+-+-+-+-+-+-+-+-+-+-+-+-+-+-+-+-+-+-+-+-+-+-+-+-+-+-+-+-+-+-+-+
   |Version| Traffic Class |           Flow Label                  |
   +-+-+-+-+-+-+-+-+-+-+-+-+-+-+-+-+-+-+-+-+-+-+-+-+-+-+-+-+-+-+-+-+
   |         Payload Length        |  Next Header  |   Hop Limit   |
   +-+-+-+-+-+-+-+-+-+-+-+-+-+-+-+-+-+-+-+-+-+-+-+-+-+-+-+-+-+-+-+-+
   |                                                               |
   +                                                               +
   |                                                               |
   +                         Source Address                        +
   |                                                               |
   +                                                               +
   |                                                               |
   +-+-+-+-+-+-+-+-+-+-+-+-+-+-+-+-+-+-+-+-+-+-+-+-+-+-+-+-+-+-+-+-+
   |                                                               |
   +                                                               +
   |                                                               |
   +                      Destination Address                      +
   |                                                               |
   +                                                               +
   |                                                               |
   +-+-+-+-+-+-+-+-+-+-+-+-+-+-+-+-+-+-+-+-+-+-+-+-+-+-+-+-+-+-+-+-+
\end{alltt}


\slide{IPv4 pakken - header - RFC-791}


\begin{alltt}
\small  
    0                   1                   2                   3   
    0 1 2 3 4 5 6 7 8 9 0 1 2 3 4 5 6 7 8 9 0 1 2 3 4 5 6 7 8 9 0 1 
   +-+-+-+-+-+-+-+-+-+-+-+-+-+-+-+-+-+-+-+-+-+-+-+-+-+-+-+-+-+-+-+-+
   |Version|  IHL  |Type of Service|          Total Length         |
   +-+-+-+-+-+-+-+-+-+-+-+-+-+-+-+-+-+-+-+-+-+-+-+-+-+-+-+-+-+-+-+-+
   |         Identification        |Flags|      Fragment Offset    |
   +-+-+-+-+-+-+-+-+-+-+-+-+-+-+-+-+-+-+-+-+-+-+-+-+-+-+-+-+-+-+-+-+
   |  Time to Live |    Protocol   |         Header Checksum       |
   +-+-+-+-+-+-+-+-+-+-+-+-+-+-+-+-+-+-+-+-+-+-+-+-+-+-+-+-+-+-+-+-+
   |                       Source Address                          |
   +-+-+-+-+-+-+-+-+-+-+-+-+-+-+-+-+-+-+-+-+-+-+-+-+-+-+-+-+-+-+-+-+
   |                    Destination Address                        |
   +-+-+-+-+-+-+-+-+-+-+-+-+-+-+-+-+-+-+-+-+-+-+-+-+-+-+-+-+-+-+-+-+
   |                    Options                    |    Padding    |
   +-+-+-+-+-+-+-+-+-+-+-+-+-+-+-+-+-+-+-+-+-+-+-+-+-+-+-+-+-+-+-+-+

                    Example Internet Datagram Header
\end{alltt}


\slide{IPv6 pakken - extension headers RFC-2460}

\begin{list1}
\item Fuld IPv6 implementation indeholder:
\begin{list2}
\item Hop-by-Hop Options
\item Routing (Type 0)
\item Fragment - fragmentering KUN i end-points!
\item Destination Options
\item Authentication
\item Encapsulating Security Payload
\end{list2}
\item Ja, IPsec er en del af IPv6!
\end{list1}

\slide{Placering af extension headers}

\begin{alltt}
\small
  +---------------+----------------+------------------------
  |  IPv6 header  | Routing header | TCP header + data
  |               |                |
  | Next Header = |  Next Header = |
  |    Routing    |      TCP       |
  +---------------+----------------+------------------------


  +---------------+----------------+-----------------+-----------------
  |  IPv6 header  | Routing header | Fragment header | fragment of TCP
  |               |                |                 |  header + data
  | Next Header = |  Next Header = |  Next Header =  |
  |    Routing    |    Fragment    |       TCP       |
  +---------------+----------------+-----------------+-----------------
\end{alltt}


%%%%%%%%%%%%%%%%%%%%%%%%%%%%%%%%%%%%%%%%%%%%%%%%%%%%%%%%%%%%%%%%%%%%%%%
\slide{IPv6 configuration - kom igang}

\begin{list1}
\item Router bagved NAT
	skal blot kunne forwarde protokoltype 0x41\\
	Cisco 677: \verb+set nat entry add 10.1.2.3 0 41+

\item Teredo - the Shipworm er også en mulighed og benyttes aktivt på Windows Vista idag

\item Officiel IPv4 addresse kan bruges med 6to4 til at lave prefix og router

\item DNS nameserver anbefales!!
	tænk på om den skal svare IPv6 AAAA record
	OG svare over IPv6 sockets - er måske ikke nødvendigt

\item IPv6-only netværk er sikkert sjældne indtil videre men det er
  muligt at lave dem nu 
\item Jeg bruger \link{http://www.sixxs.net} som har vejledninger  til diverse operativsystemer
\end{list1}


%%%%%%%%%%%%%%%%%%%%%%%%%%%%%%%%%%%%%%%%%%%%%%%%%%%%%%%%%%%%%%%%%%%%%%%
\slide{IPv6 configuration - klienter}

\begin{alltt}
\footnotesize$ ping6 ::1
PING6(56=40+8+8 bytes) ::1 --> ::1
16 bytes from ::1, icmp_seq=0 hlim=64 time=0.254 ms
16 bytes from ::1, icmp_seq=1 hlim=64 time=0.23 ms
^C
	--- ::1 ping6 statistics ---
2 packets transmitted, 2 packets received, 0% packet loss
round-trip min/avg/max = 0.230/0.242/0.254 ms
\end{alltt}

\begin{list1}
  

\item Microsoft Windows XP
\verb+ipv6 install+ fra kommandolinien eller brug kontrolpanelet

\item ipv6 giver mulighed for at konfigurere tunnel
svarer omtrent til 'ifconfig' på Unix

\item Migrering er vigtigt i IPv6!
Hvis I aktiverer IPv6 nu på en router, vil I 
pludselig have IPv6 på alle klienter ;-)


\item Se evt. appendix F Enabling IPv6 functionality i 
\link{http://inet6.dk/thesis.pdf}
\end{list1}


\slide{ifconfig med ipv6 - Unix}

Næsten ingen forskel på de sædvanlige kommandoer ifconfig, netstat, 
\begin{alltt}
\small
root# ifconfig en1 inet6 2001:1448:81:beef::1
root# ifconfig en1
en1: flags=8863<UP,BROADCAST,SMART,RUNNING,SIMPLEX,MULTICAST> mtu 1500{\color{security6blue}
        inet6 fe80::230:65ff:fe17:94d1 prefixlen 64 scopeid 0x5 
        inet6 2001:1448:81:beef::1 prefixlen 64 }
        inet 169.254.32.125 netmask 0xffff0000 broadcast 169.254.255.255
        ether 00:30:65:17:94:d1 
        media: autoselect status: active
        supported media: autoselect
\end{alltt}

%size 4
Fjernes igen med:\\
\verb+ifconfig en1 inet6 -alias 2001:1448:81:beef::1+\\
Prøv også:\\ \verb+ifconfig en1 inet6+


\slide{GUI værktøjer - autoconfiguration}

\hlkimage{20cm}{osx-network-automatic.png}

\centerline{De fleste moderne operativsystemer er efterhånden opdateret med menuer til IPv6}

\slide{GUI værktøjer - manuel konfiguration}

\hlkimage{20cm}{osx-network-manual.png}

\centerline{Bemærk hvorledes subnetmaske nu blot er en prefix length}

\slide{ping til IPv6 adresser}


\begin{alltt}
\small 
root# ping6 ::1
PING6(56=40+8+8 bytes) ::1 --> ::1
16 bytes from ::1, icmp_seq=0 hlim=64 time=0.312 ms
16 bytes from ::1, icmp_seq=1 hlim=64 time=0.319 ms
^C
--- localhost ping6 statistics ---
2 packets transmitted, 2 packets received, 0% packet loss
round-trip min/avg/max = 0.312/0.316/0.319 ms
\end{alltt}

Nogle operativsystemer kalder kommandoen ping6, andre bruger blot ping

\slide{ping6 til global unicast adresse}


\begin{alltt}
\footnotesize
root# ping6 2001:1448:81:beef:20a:95ff:fef5:34df
PING6(56=40+8+8 bytes) 2001:1448:81:beef::1 --> 2001:1448:81:beef:20a:95ff:fef5:34df
16 bytes from 2001:1448:81:beef:20a:95ff:fef5:34df, icmp_seq=0 hlim=64 time=10.639 ms
16 bytes from 2001:1448:81:beef:20a:95ff:fef5:34df, icmp_seq=1 hlim=64 time=1.615 ms
16 bytes from 2001:1448:81:beef:20a:95ff:fef5:34df, icmp_seq=2 hlim=64 time=2.074 ms
^C
--- 2001:1448:81:beef:20a:95ff:fef5:34df ping6 statistics ---
3 packets transmitted, 3 packets received, 0% packet loss
round-trip min/avg/max = 1.615/4.776/10.639 ms
\end{alltt}


\slide{ ping6 til specielle adresser}


\begin{alltt}
\small
root# ping6 -I en1 ff02::1
PING6(56=40+8+8 bytes) fe80::230:65ff:fe17:94d1 --> ff02::1
16 bytes from fe80::230:65ff:fe17:94d1, icmp_seq=0 hlim=64 time=0.483 ms
16 bytes from fe80::20a:95ff:fef5:34df, icmp_seq=0 hlim=64 time=982.932 ms
16 bytes from fe80::230:65ff:fe17:94d1, icmp_seq=1 hlim=64 time=0.582 ms
16 bytes from fe80::20a:95ff:fef5:34df, icmp_seq=1 hlim=64 time=9.6 ms
16 bytes from fe80::230:65ff:fe17:94d1, icmp_seq=2 hlim=64 time=0.489 ms
16 bytes from fe80::20a:95ff:fef5:34df, icmp_seq=2 hlim=64 time=7.636 ms
^C
--- ff02::1 ping6 statistics ---
4 packets transmitted, 4 packets received, +4 duplicates, 0% packet loss
round-trip min/avg/max = 0.483/126.236/982.932 ms
\end{alltt}

\begin{list2}
%\item ff00::0         ipv6-mcastprefix
\item ff02::1         ipv6-allnodes
\item ff02::2         ipv6-allrouters
\item ff02::3         ipv6-allhosts
\end{list2}

\slide{Stop - tid til leg}

\begin{list1}
\item Der findes et trådløst netværk med IPv6
\item Join med en laptop og prøv at pinge lidt
\begin{enumerate}
\item Virker \verb+ping6 ::1+ eller \verb+ping ::1+, fortsæt
\item Virker kommando svarende til: \verb+ping6 -I en1 ff02::1+\\
- burde vise flere maskiner
\item Kig på dine egne adresser med: \verb+ipv6+ (Windows) eller \verb+ifconfig+ (Unix)
\item Prøv at trace i netværket
\end{enumerate}
\item Hvordan fik I IPv6 adresser?
\end{list1}

\slide{ router advertisement daemon}


\begin{alltt}
/etc/rtadvd.conf:
en0:
      :addrs#1:addr="2001:1448:81:b00f::":prefixlen#64:
en1:
      :addrs#1:addr="2001:1448:81:beef::":prefixlen#64:

root# /usr/sbin/rtadvd -Df en0 en1
root# sysctl -w net.inet6.ip6.forwarding=1
net.inet6.ip6.forwarding: 0 -> 1
\end{alltt}

\begin{list1}
\item Stateless autoconfiguration er en stor ting i IPv6
\item Kommandoen starter den i debug-mode og i forgrunden\\
- normalt vil man starte den fra et script 
\item Typisk skal forwarding aktiveres, som vist med BSD sysctl kommando
\end{list1}





\slide{IPv6 og andre services}


\begin{alltt}
\small
root# netstat -an | grep -i listen

tcp46  0  0  *.80             *.*    LISTEN
tcp4   0  0  *.6000           *.*    LISTEN
tcp4   0  0  127.0.0.1.631    *.*    LISTEN
tcp4   0  0  *.25             *.*    LISTEN
tcp4   0  0  *.20123          *.*    LISTEN
tcp46  0  0  *.20123          *.*    LISTEN
tcp4   0  0  127.0.0.1.1033   *.*    LISTEN
\end{alltt}

ovenstående er udført på Mac OS X


\slide{IPv6 output fra kommandoer - inet6 family}


\begin{alltt}
\small
root# netstat -an -f inet6

Active Internet connections (including servers)
Proto Recv Send  Local  Foreign   (state)
tcp46  0   0     *.80     *.*     LISTEN
tcp46  0   0     *.22780  *.*     LISTEN
udp6   0   0     *.5353   *.*                    
udp6   0   0     *.5353   *.*                    
udp6   0   0     *.514    *.*                    
icm6   0   0     *.*      *.*                    
icm6   0   0     *.*      *.*                    
icm6   0   0     *.*      *.* 
\end{alltt}

ovenstående er udført på Mac OS X og tilrettet lidt


\slide{IPv6 er default for mange services}


\begin{alltt}
\small
root# telnet localhost 80

{\color{security6blue}Trying ::1...
Connected to localhost.}
Escape character is '^]'.
GET / HTTP/1.0

HTTP/1.1 200 OK
Date: Thu, 19 Feb 2004 09:22:34 GMT
Server: Apache/2.0.43 (Unix)
Content-Location: index.html.en
Vary: negotiate,accept-language,accept-charset
...
\end{alltt}

\slide{IPv6 er default i OpenSSH}

\begin{alltt}
\small
hlk$ ssh -v localhost -p 20123
OpenSSH_3.6.1p1+CAN-2003-0693, SSH protocols 1.5/2.0, OpenSSL 0x0090702f
debug1: Reading configuration data /Users/hlk/.ssh/config
debug1: Applying options for *
debug1: Reading configuration data /etc/ssh_config
debug1: Rhosts Authentication disabled, originating port will not be trusted.{\color{security6blue}
debug1: Connecting to localhost [::1] port 20123.}
debug1: Connection established.
debug1: identity file /Users/hlk/.ssh/id_rsa type -1
debug1: identity file /Users/hlk/.ssh/id_dsa type 2
debug1: Remote protocol version 2.0, remote software version OpenSSH_3.6.1p1+CAN-2003-0693
debug1: match: OpenSSH_3.6.1p1+CAN-2003-0693 pat OpenSSH*
debug1: Enabling compatibility mode for protocol 2.0
debug1: Local version string SSH-2.0-OpenSSH_3.6.1p1+CAN-2003-0693
\end{alltt}



\slide{Apache access log}


\begin{alltt}
\footnotesize
root# tail -f access_log 
::1 - - [19/Feb/2004:09:05:33 +0100] "GET /images/IPv6ready.png 
HTTP/1.1" 304 0
::1 - - [19/Feb/2004:09:05:33 +0100] "GET /images/valid-html401.png
HTTP/1.1" 304 0
::1 - - [19/Feb/2004:09:05:33 +0100] "GET /images/snowflake1.png 
HTTP/1.1" 304 0
::1 - - [19/Feb/2004:09:05:33 +0100] "GET /~hlk/security6.net/images/logo-1.png
HTTP/1.1" 304 0
2001:1448:81:beef:20a:95ff:fef5:34df - - [19/Feb/2004:09:57:35 +0100] 
"GET / HTTP/1.1" 200 1456
2001:1448:81:beef:20a:95ff:fef5:34df - - [19/Feb/2004:09:57:35 +0100] 
"GET /apache_pb.gif HTTP/1.1" 200 2326
2001:1448:81:beef:20a:95ff:fef5:34df - - [19/Feb/2004:09:57:36 +0100]
"GET /favicon.ico HTTP/1.1" 404 209
2001:1448:81:beef:20a:95ff:fef5:34df - - [19/Feb/2004:09:57:36 +0100] 
"GET /favicon.ico HTTP/1.1" 404 209
\end{alltt}
\vskip 1cm
\centerline{Apache konfigureres nemt til at lytte på IPv6}

\slide{Apache HTTPD server}

\begin{list1}
\item 
\item Mange bruger HTTPD fra Apache projektet\\
  \link{http://httpd.apache.org} - netcraft siger omkring 70\%
\item konfigureres gennem \verb+httpd.conf+
\end{list1}

\begin{alltt}
Listen 0.0.0.0:80
Listen [::]:80
...
Allow from 127.0.0.1
Allow from 2001:1448:81:0f:2d:9ff:f86:3f 
Allow from 217.157.20.133
\end{alltt}





\slide{OpenBSD fast IPv6 adresse}

Netværkskonfiguration på OpenBSD - flere filer:
\begin{alltt}
\small
# cat /etc/hostname.sk0
inet 10.0.0.23 0xffffff00 NONE
inet6 2001:1448:81:30::2
# cat /etc/mygate
10.0.0.1
# grep 2001 /etc/rc.local
route add -inet6 default 2001:1448:81:30::1
# cat /etc/resolv.conf    
domain security6.net
lookup file bind
nameserver 212.242.40.3
nameserver 212.242.40.51
nameserver 2001:1448:81:30::10
\end{alltt}

\slide{Basal DNS opsætning}



\begin{alltt}
domain security6.net
nameserver 212.242.40.3
nameserver 212.242.40.51
nameserver 2001:1448:81:30::2
\end{alltt}

\begin{list1}    
\item \verb+/etc/resolv.conf+ angiver navneservere og søgedomæner
\item typisk indhold er domænenavn og IP-adresser for navneservere
\item Filen opdateres også automatisk på DHCP klienter
\item {\bf Husk at man godt kan slå AAAA records op over IPv4} 
\item NB: denne fil kan hedde noget andet på UNIX varianter!
\item eksempelvis \verb+/etc/netsvc.conf+
\end{list1}

\slide{DNS systemet}

\begin{list1}
\item Navneopslag på Internet - centralt for IPv6
\item Tidligere brugte man en {\bfseries hosts} fil\\
hosts filer bruges stadig lokalt til serveren - IP-adresser
\item UNIX: /etc/hosts
\item Windows \verb+c:\windows\system32\drivers\etc\hosts+
\item Eksempel: www.security6.net har adressen 217.157.20.131
\item skrives i database filer, zone filer
\end{list1}

\begin{alltt}
ns1     IN      A       217.157.20.130
        IN      AAAA    2001:618:433::1
www     IN      A       217.157.20.131
        IN      AAAA    2001:618:433::14
\end{alltt}

\slide{Mere end navneopslag}

\begin{list1}
  \item består af resource records med en type:
    \begin{list2}
\item adresser A-records
\item IPv6 adresser AAAA-records
\item autoritative navneservere NS-records
\item post, mail-exchanger MX-records
\item flere andre: md ,  mf ,  cname ,  soa ,
                  mb , mg ,  mr ,  null ,  wks ,  ptr ,
                  hinfo ,  minfo ,  mx ....
\end{list2}
\end{list1}
\begin{alltt}
        IN      MX      10      mail.security6.net.
        IN      MX      20      mail2.security6.net.
\end{alltt}


\slide{BIND DNS server}

\begin{list1}
\item Berkeley Internet Name Daemon server
\item Mange bruger BIND fra Internet Systems Consortium
   - altså Open Source
\item konfigureres gennem \verb+named.conf+
\item det anbefales at bruge BIND version 9
\end{list1}

\begin{list2}
\item \emph{DNS and BIND}, Paul Albitz \& Cricket Liu, O'Reilly, 4th
  edition Maj 2001 
\item \emph{DNS and BIND cookbook}, Cricket Liu, O'Reilly, 4th
  edition Oktober 2002 
\end{list2}

Kilde: \link{http://www.isc.org}

\slide{BIND konfiguration - et udgangspunkt}

\begin{alltt}
\small 
acl internals \{ 127.0.0.1; ::1; 10.0.0.0/24; \};
options \{
        // the random device depends on the OS !
        random-device "/dev/random"; directory "/namedb";
        {\bf listen-on-v6 { any; };}
        port 53; version "Dont know"; allow-query \{ any; \};
\};
view "internal" \{
   match-clients \{ internals; \}; recursion yes;
   zone "." \{
       type hint;   file "root.cache"; \};
   // localhost forward lookup
   zone "localhost." \{
        type master; file "internal/db.localhost";   \};
   // localhost reverse lookup from IPv4 address
   zone "0.0.127.in-addr.arpa" \{
        type master; file "internal/db.127.0.0"; notify no;   \};
...
\}
\end{alltt}



\slide{Routing forståelse - IPv6}
\begin{alltt}
\small
$ netstat -f inet6 -rn 
Routing tables

Internet6:
Destination                 Gateway           Flags      Netif 
default             fe80::200:24ff:fec1:58ac  UGc         en0
::1                         ::1               UH          lo0
2001:1448:81:cf0f::/64      link#4            UC          en0
2001:1448:81:cf0f::1        0:0:24:c1:58:ac   UHLW        en0
fe80::/64                   fe80::1           Uc          lo0
fe80::1                     link#1            UHL         lo0
fe80::/64                   link#4            UC          en0
fe80::20d:93ff:fe28:2812    0:d:93:28:28:12   UHL         lo0
fe80::/64                   link#5            UC          en1
fe80::20d:93ff:fe86:7c3f    0:d:93:86:7c:3f   UHL         lo0
ff01::/32                   ::1               U           lo0
ff02::/32                   ::1               UC          lo0
ff02::/32                   link#4            UC          en0
ff02::/32                   link#5            UC          en1
\end{alltt}



\slide{ IPv6 neighbor discovery protocol (NDP)}

\hlkimage{20cm}{ipv6-arp-ndp.pdf}

\begin{list1}
\item ARP er væk
\item NDP erstatter og udvider ARP, Sammenlign \verb+arp -an+ med \verb+ndp -an+
\item Til dels erstatter ICMPv6 således DHCP i IPv6, DHCPv6 findes dog
\item {\bf NB: bemærk at dette har stor betydning for firewallregler!}
\end{list1}

\slide{ARP vs NDP}

\begin{alltt}
\small
hlk@bigfoot:basic-ipv6-new$ arp -an        
? (10.0.42.1) at{\bf 0:0:24:c8:b2:4c} on en1 [ethernet]
? (10.0.42.2) at 0:c0:b7:6c:19:b on en1 [ethernet]
hlk@bigfoot:basic-ipv6-new$ ndp -an        
Neighbor                      Linklayer Address  Netif Expire    St Flgs Prbs
::1                           (incomplete)         lo0 permanent R      
2001:16d8:ffd2:cf0f:21c:b3ff:fec4:e1b6 0:1c:b3:c4:e1:b6 en1 permanent R      
fe80::1%lo0                   (incomplete)         lo0 permanent R      
fe80::200:24ff:fec8:b24c%en1 {\bf 0:0:24:c8:b2:4c}      en1 8h54m51s  S  R   
fe80::21c:b3ff:fec4:e1b6%en1  0:1c:b3:c4:e1:b6     en1 permanent R      
\end{alltt}







\slide{Fremtiden er nu}
\label{slide:future}

\begin{list1}
\item Det er sagt mange gange at nu skal vi igang med IPv6
\item Der er sket store ændringer fra starten af 2007 til nu
\item Hvor det i starten af 2007 var status quo er flere begyndt at presse på
\item Selv på version2.dk omtales det \link{http://www.version2.dk/artikel/6147}\\
\emph{Seks DNS-rodservere tænder for IPv6
ICANN har nu aktiveret IPv6 på seks af internettets 13 rodservere. Med det rigtige udstyr kan man nu køre helt uden om IPv4.}
\end{list1}

%%%%%%%%%%%%%%%%%%%%%%%%%%%%%%%%%%%%%%%%%%%%%%%%%%%%%%%%%%%%%%%%%%%%%%%
\slide{Hvorfor implementere IPv6 i jeres netværk?}

\begin{list1}
\item Addresserummet
\begin{list2}
\item end-to-end transparancy
\item 	nemmere administration
\end{list2}\item Autoconfiguration
\begin{list2}
\item stateless autoconfiguration
\item automatisk routerconfiguration! 
(router renumbering)
\end{list2}
\item Performance
\begin{list2}
\item simplere format
\item kortere behandlingstid i routere
\end{list2}
\item Fleksibilitet - generelt
\item Sikkerhed
\begin{list2}
\item IPsec er et krav!
\item Afsender IP-adressen ændres ikke igennem NAT!
\end{list2}
\end{list1}

\slide{Hvorfor migrere til IPv6?}

\begin{list1}
\item IPv4 er mere end 25 år gammel - fra 1981 til idag
\item Idag har folk ønsker/krav til kommunikationen
\begin{list2}
\item båndbredde
\item latency
\item Quality-of-service
\item sikkerhed
\end{list2}

\item Meget af dette er, eller kan, implementeres med IPv4 - men det bliver lappeløsninger

\item NB: IPv6 er designet til at løse SPECIFIKKE problemer
\end{list1}

\slide{The Internet has done this before!}


\begin{quote}
   Because all hosts can not be converted to TCP simultaneously, and
   some will implement only IP/TCP, it will be necessary to provide
   temporarily for communication between NCP-only hosts and TCP-only
   hosts.  To do this certain hosts which implement both NCP and IP/TCP
   will be designated as relay hosts.  These relay hosts will support
   Telnet, FTP, and Mail services on both NCP and TCP.  These relay
   services will be provided  beginning in November 1981, and will be
   fully in place in January 1982.

  
   Initially there will be many NCP-only hosts and a few TCP-only hosts,
   and the load on the relay hosts will be relatively light.  As time
   goes by, and the conversion progresses, there will be more TCP
   capable hosts, and fewer NCP-only hosts, plus new TCP-only hosts.
   But, presumably most hosts that are now NCP-only will implement
   IP/TCP in addition to their NCP and become "dual protocol" hosts.
   So, while the load on the relay hosts will rise, it will not be a
   substantial portion of the total traffic.
\end{quote}

\centerline{NCP/TCP Transition Plan November 1981 RFC-801}

%%%%%%%%%%%%%%%%%%%%%%%%%%%%%%%%%%%%%%%%%%%%%%%%%%%%%%%%%%%%%%%%%%%%%%%
\slide{Er IPv6 klar? - Korte svar - ja}

\begin{list1}
\item Det bruges idag aktivt, især i dele af verden der ikke har store dele
af v4 adresserummet 

\item Kernen af IPv6 er stabil

\item IPv6 er inkluderet i mange operativsystemer idag\\
AIX, Solaris, BSD'erne, Linux, Mac OS X og Windows XP
Cisco IOS, Juniper Networks\\
Juniper har haft hardware support for IPv6 i mange år!

\item IPv6 TCP/IP stakke til indlejrede systemer er klar

\item prøv at lave \verb+ping6 ::1+ på jeres maskiner - det er IPv6

\item Se listen over IPv6 implementationer på
http://playground.sun.com/ipv6/ipng-implementations.html
\end{list1}



\slide{IPv6 bruges idag}

\begin{list1}
\item Listen over brugere vokser konstant
\item Store nye netværk designes alle med IPv6
en liste kan eksempelvis ses på addressen:
\link{http://www.ipv6.ac.uk/gtpv6/eu.html}

\item Andre links kan vise statistik for internet og IPv4/IPv6
\item \link{http://www.bgpexpert.com/addrspace2007.php}
\item \link{https://wiki.caida.org/wiki/iic/bin/view/Main/WebHome}
\item \link{http://bgp.he.net/ipv6-progress-report.cgi}
\item Se også:
\link{http://www.eu.ipv6tf.org/}


\end{list1}

\slide{5 dårlige argumenter for ikke at bruge IPv6 nu}

\begin{list1}
\item Det er ikke færdigt\\
- IPv4 har ALDRIG været færdigt ;-)

\item Ikke nødvendigt\\
- man kan stikke hovedet i busken 

\item NAT løser alle problemer og er meget sikkert ...\\
- NAT er en lappeløsning 

\item Udskiftning af HELE infrastrukturen er for dyrt\\
- man opgraderer/udskifter jævnligt udstyr 

\item Vent til det er færdigt!\\
- man mister muligheden for at påvirke resultatet!
\end{list1}

\exercise{ex:ping6}
\exercise{ex:basic-dns-lookup6}



\slide{Færdig med IPv6}

\begin{list1}
\item I resten af kurset vil vi ikke betragte IPv6 eller IPv4 som noget specielt
\item Vi vil indimellem bruge det ene, indimellem det andet
\item Alle subnets er konfigureret ens på IPv4 og IPv6
\item Subnets som i IPv4 hedder prefix.45 vil således i IPv6 hedde noget med prefix:45:
\item At have ens routing på IPv4 og IPv6 vil typisk IKKE være tilfældet i praksis
\item Man kan jo lige så godt forbedre netværket mens man går over til IPv6 :-)
\end{list1}

\slide{Nu skal vi til management og diagnosticering}

\begin{list1}
\item Always check the spark plugs!
\item Når man skal spore fejl i netværk er det essentielt at starte fra bunden:
\begin{list2}
\item Er der link?
\item Er der IP-adresse?
\item Er der route?
\item Modtager systemet pakker
\item Er der en returvej fra systemet! Den her kan snyde mange!
\item Lytter serveren på den port man vil forbinde til, UDP/TCP
\end{list2} 
\item Hvis der ikke er link vil man aldrig få svar fra databasen/webserveren/postserveren
\end{list1}

\slide{Udtræk af netværkskonfigurationen}

\begin{list1}
\item De vigtigste kommandoer til udtræk af netværkskonfigurationen:
\begin{list2}
\item cat - til at vise tekstfiler
\item ifconfig - interface configuration
\item netstat - network statistics
\item lsof - list open files
\end{list2}
\end{list1}

\slide{Basale testværktøjer TCP - Telnet og OpenSSL}

\begin{list1}
\item Telnet blev tidligere brugt til login og er en klartekst forbindelse
over TCP
\item Telnet kan bruges til at teste forbindelsen til mange ældre serverprotokoller som benytter ASCII kommandoer
\begin{list2}
\item \verb+telnet mail.kramse.dk 25+ laver en forbindelse til port 25/tcp
\item \verb+telnet www.kramse.dk 80+ laver en forbindelse til port 80/tcp
\end{list2}
\item Til krypterede forbindelser anbefales det at teste med openssl
\begin{list2}
\item \verb+openssl s_client -host www.kramse.dk -port 443+\\
laver en forbindelse til port 443/tcp med SSL
\item \verb+openssl s_client -host mail.kramse.dk -port 993+\\
 laver en forbindelse til port 993/tcp med SSL
\end{list2}
\item Med OpenSSL i client-mode kan services tilgås med samme tekstkommandoer som med telnet
\end{list1}


\slide{Basale testværktøjer UDP}

\begin{list1}
\item UDP er lidt drilsk, for de fleste services er ikke \emph{ASCII protokoller}
\item Der findes dog en række testprogrammer, a la ping
\begin{list2}
\item nsping - name server ping
\item dhcping - dhcp server ping
\item ...
\end{list2}
\item Derudover kan man bruge de sædvanlige programmer som \verb+host+ til navneopslag osv.
\end{list1}


\slide{IP netværkstuning}

\hlkimage{8cm}{712a.png}

\begin{list1}
\item IP har eksisteret mange år
\item Vi har udskiftet langsommme forbindelser med hurtige forbindelser
\item Vi har udskiftet langsomme MHz maskiner med Quad-core GHz maskiner
\item IP var tidligere meget konservativt, for ikke at overbelaste modtageren
\item Billedet er en HP arbejdsstation med 19" skærm og en 60MHz HP PA-RISC processor
\end{list1}

\slide{Anbefalet netværkstuning - hvad skal tunes}

\begin{list1}
\item Der er visse indstillinger som tidligere var standard, de bør idag slås fra
\item En del er allerede tunet i nyere versioner af IP-stakkene, men check lige
\item Ideer til ting som skal slås fra:
\begin{list2}
\item broadcast ICMP, undgå smurfing
\item Source routing, kan måske omgå firewalls og filtre    
\end{list2}
\item Ideer til ting som skal slås til/ændres:
\begin{list2}
\item Bufferstørrelser - hvorfor have en buffer på 65535 bytes på en maskine med 32GB ram?
\item Nye funktioner som RFC-1323 TCP Extensions for High Performance
\end{list2}
\item Det anbefales at finde leverandørens vejledning til hvad der kan tunes
\end{list1}

\slide{Netværkskonfiguration med sysctl}

\begin{alltt}\small
	# tuning
	net.inet.tcp.recvspace=65535
	net.inet.tcp.sendspace=65535
	net.inet.udp.recvspace=65535
	net.inet.udp.sendspace=32768
	# postgresql tuning
	kern.seminfo.semmni=256
	kern.seminfo.semmns=2048
	kern.shminfo.shmmax=50331648\end{alltt}
\begin{list1}
\item På mange UNIX varianter findes et specielt tuningsprogram,
  sysctl
\item Findes blandt andet på alle BSD'erne: FreeBSD, OpenBSD, NetBSD
  og Darwin/OSX 
\item Ændringerne skrives ind i filen \verb+/etc/sysctl.conf+
\item På Linux erstatter det til dels konfiguration med echo \\
\verb+echo 1 > /proc/net/ip/forwarding+   
\item På AIX benyttes kommandoen network options \verb+no+ 
\end{list1}

\slide{Tuning}

\begin{list1}
\item Hvad er flaskehalsen for programmet?
\item I/O bundet - en enkelt disk eller flere
\item CPU bundet - regnekraften
\item Netværket - 10Mbit half-duplex adapter
\item Memory - begynder systemet at \emph{swappe} eller \emph{thrashe}
\item brug top og andre statistikprogrammer til at se disse data
\end{list1}


\slide{Måling af througput}

\begin{list1}
\item Når der skal tunes er det altid nødvendigt med en baseline
\item Man kan ikke begynde at tune ud fra subjektive målinger 
\item \emph{Det kører langsomt}, \emph{Svartiden er for høj}
\item Målinger der giver præcise tal er nødvendige, før og efter målinger!
\item Der findes et antal værktøjer til, blandt andet Iperf
\end{list1}

\slide{Målinger med Iperf}

\begin{alltt}\small
hlk@fluffy:hlk$ iperf -s
------------------------------------------------------------
Server listening on TCP port 5001
TCP window size: 64.0 KByte (default)
------------------------------------------------------------
[  4] local 10.0.42.23 port 5001 connected with 10.0.42.67 port 51148
[  4]  0.0-10.2 sec  6.95 MBytes  5.71 Mbits/sec
[  4] local 10.0.42.23 port 5001 connected with 10.0.42.67 port 51149
[  4]  0.0-10.2 sec  7.02 MBytes  5.76 Mbits/sec
\end{alltt}

Ovenstående er set fra server, client kaldes med \verb+iperf -c fluffy+

\slide{Stop - vi prøver i fællesskab Iperf}

\begin{list1}
\item Vi prøver lige Iperf sammen
\item hvis alle prøver samtidig giver det stor variation i resultaterne
\end{list1}





\slide{Apache benchmark og andre programmer}

\begin{alltt}
\footnotesize
hlk@bigfoot:hlk$ ab -n 100 http://www.kramse.dk/
This is ApacheBench, Version 2.0.41-dev <$Revision: 1.121.2.12 $> apache-2.0
Copyright (c) 1996 Adam Twiss, Zeus Technology Ltd, http://www.zeustech.net/
Copyright (c) 2006 The Apache Software Foundation, http://www.apache.org/

Benchmarking www.kramse.dk (be patient)...
...
\end{alltt}

\begin{list1}
\item Der findes specialiserede værktøjer til mange protokoller
\item Eksempelvis følger der et apache benchmark med Apache HTTPD serveren
\item Mange andre værktøjer til at simulere flere samtidige brugere
\end{list1}

\slide{Apache Benchmark output - 1 }

\begin{alltt}
\footnotesize
Server Software:        Apache
Server Hostname:        www.kramse.dk
Server Port:            80

Document Path:          /
Document Length:        7547 bytes

Concurrency Level:      1
Time taken for tests:   13.84924 seconds
Complete requests:      100
Failed requests:        0
Write errors:           0
Total transferred:      778900 bytes
HTML transferred:       754700 bytes
Requests per second:    7.64 #/sec (mean)
Time per request:       130.849 ms (mean)
Time per request:       130.849 ms (mean, across all concurrent requests)
Transfer rate:          58.08 Kbytes/sec received
\end{alltt}

\slide{Apache Benchmark output - 3}

\begin{alltt}
\footnotesize
Connection Times (ms)
              min  mean+/-sd median   max
Connect:       22   24   4.0     24      58
Processing:    96  105  33.0     99     421
Waiting:       63   71  32.7     65     386
Total:        119  130  33.5    124     446

Percentage of the requests served within a certain time (ms)
  50%    124
  66%    126
  75%    128
  80%    130
  90%    143
  95%    153
  98%    189
  99%    446
 100%    446 (longest request)
\end{alltt}



\exercise{ex:sysctl}
\exercise{ex:iperf}
\exercise{ex:apache-benchmark}

\slide{Antal pakker per sekund} 

\begin{list1}
\item Til tider er det ikke båndbredden som sådan man vil måle
\item Specielt for routere er det vigtigt at de kan behandle mange pakker per sekund, pps
\item Til dette kan man lege med det indbyggede Ping program i flooding mode
\item Når programmet kaldes (som systemadministrator) med \verb+ping -f server+ vil den sende ping pakker så hurtigt som netkortet tillader
\item Programmer der kan teste pakker per sekund kaldes generelt for blaster tools
\end{list1}

\slide{traceroute}

\begin{list1}
  \item traceroute programmet virker ved hjælp af TTL
\item levetiden for en pakke tælles ned i hver router på vejen og ved at sætte denne lavt
  opnår man at pakken \emph{timer ud} - besked fra hver router på vejen
\item default er UDP pakker, men på UNIX systemer er der ofte mulighed
  for at bruge ICMP
\end{list1}

\begin{alltt}
{\bfseries\$ traceroute 217.157.20.129}
traceroute to 217.157.20.129 (217.157.20.129), 
30 hops max, 40 byte packets
 1  safri (10.0.0.11)  3.577 ms  0.565 ms  0.323 ms
 2  router (217.157.20.129)  1.481 ms  1.374 ms  1.261 ms
\end{alltt}


\slide{traceroute - med UDP}

\begin{alltt}
\tiny
# {\bfseries tcpdump -i en0 host 217.157.20.129 or host 10.0.0.11}
tcpdump: listening on en0
23:23:30.426342 10.0.0.200.33849 > router.33435: udp 12 [ttl 1]
23:23:30.426742 safri > 10.0.0.200: icmp: time exceeded in-transit
23:23:30.436069 10.0.0.200.33849 > router.33436: udp 12 [ttl 1]
23:23:30.436357 safri > 10.0.0.200: icmp: time exceeded in-transit
23:23:30.437117 10.0.0.200.33849 > router.33437: udp 12 [ttl 1]
23:23:30.437383 safri > 10.0.0.200: icmp: time exceeded in-transit
23:23:30.437574 10.0.0.200.33849 > router.33438: udp 12
23:23:30.438946 router > 10.0.0.200: icmp: router udp port 33438 unreachable
23:23:30.451319 10.0.0.200.33849 > router.33439: udp 12
23:23:30.452569 router > 10.0.0.200: icmp: router udp port 33439 unreachable
23:23:30.452813 10.0.0.200.33849 > router.33440: udp 12
23:23:30.454023 router > 10.0.0.200: icmp: router udp port 33440 unreachable
23:23:31.379102 10.0.0.200.49214 > safri.domain:  6646+ PTR? \\
200.0.0.10.in-addr.arpa. (41)
23:23:31.380410 safri.domain > 10.0.0.200.49214:  6646 NXDomain* 0/1/0 (93)
14 packets received by filter
0 packets dropped by kernel
\end{alltt}

\slide{Værdien af traceroute}

\begin{list1}
\item Diagnosticering af netværksproblemer - formålet med traceroute
\item Indblik i netværkets opbygning!
\item Svar fra hosts - en modtaget pakke fremfor et \emph{sort hul}

\item Traceroute er ikke et angreb - det er også vigtigt at kunne
  genkende normal trafik!
\end{list1}

\slide{Network mapping}

\hlkimage{23cm}{images/network-example.pdf}

\begin{list1}
\item Ved brug af traceroute og tilsvarende programmer kan man ofte
  udlede topologien i det netværk man undersøger  
\end{list1}


\slide{Flere traceprogrammer}

\begin{list1}
\item mtr My traceroute - grafisk \link{http://www.bitwizard.nl/mtr/}
\item lft - \emph{layer four trace} benytter TCP SYN og FIN prober
\item trace ved hjælp af TCP og andre protokoller findes
\item paratrace - \emph{Parasitic Traceroute via Established TCP Flows
    and IPID Hopcount} 
\item Der findes webservices hvor man kan trace fra, eksempelvis: \link{http://www.samspade.org}
\end{list1}

\slide{TCPDUMP - protokolanalyse pakkesniffer}

\hlkimage{14cm}{images/tcpdump-manual.pdf}


\centerline{\link{http://www.tcpdump.org}
- både til Windows og UNIX}
\slide{tcpdump - normal brug}

\begin{list2}
  \item tekstmode
\item kan gemme netværkspakker i filer
\item kan læse netværkspakker fra filer
\item er de-facto standarden for at gemme netværksdata i filer
\end{list2}

\begin{alltt}
\tiny [root@otto hlk]# tcpdump -i en0
tcpdump: listening on en0
13:29:39.947037 fe80::210:a7ff:fe0b:8a5c > ff02::1: icmp6: router advertisement
13:29:40.442920 10.0.0.200.49165 > dns1.cybercity.dk.domain:  1189+[|domain]
13:29:40.487150 dns1.cybercity.dk.domain > 10.0.0.200.49165:  1189 NXDomain*[|domain]
13:29:40.514494 10.0.0.200.49165 > dns1.cybercity.dk.domain:  24765+[|domain]
13:29:40.563788 dns1.cybercity.dk.domain > 10.0.0.200.49165:  24765 NXDomain*[|domain]
13:29:40.602892 10.0.0.200.49165 > dns1.cybercity.dk.domain:  36485+[|domain]
13:29:40.648288 dns1.cybercity.dk.domain > 10.0.0.200.49165:  36485 NXDomain*[|domain]
13:29:40.650596 10.0.0.200.49165 > dns1.cybercity.dk.domain:  4101+[|domain]
13:29:40.694868 dns1.cybercity.dk.domain > 10.0.0.200.49165:  4101 NXDomain*[|domain]
13:29:40.805160 10.0.0.200 > mail: icmp: echo request
13:29:40.805670 mail > 10.0.0.200: icmp: echo reply
...
\end{alltt}

\slide{TCPDUMP syntaks - udtryk}

\begin{list1}
\item filtre til husbehov
  \begin{list2}
\item type - host, net og port
\item src pakker med afsender IP eller afsender port
\item dst pakker med modtager IP eller modtager port
\item host - afsender eller modtager 
\item proto - protokol: ether, fddi, tr, ip, ip6, arp,  rarp,  decnet,
tcp og udp
\end{list2}
\item IP adresser kan angives som dotted-decimal eller navne
\item porte kan angives med numre eller navne
\item komplekse udtryk opbygges med logisk and,  or,  not
\end{list1}

\slide{tcpdump udtryk eksempler}

\begin{list1}
  \item Host 10.1.2.3\\
Alle pakker hvor afsender eller modtager er 10.1.2.3
\item host 10.2.3.4 and not host 10.3.4.5\\
Alle pakker til/fra 10.2.3.4 undtagen dem til/fra 10.3.4.5\\
- meget praktisk hvis man er logget ind på 10.2.3.4 via netværk fra 10.3.4.5
\item host foo and not port ftp  and not  port  ftp-data\\
trafik til/fra maskine \emph{foo} undtagen hvis det er FTP trafik
\end{list1}

\slide{Wireshark - grafisk pakkesniffer}

\hlkimage{20cm}{images/wireshark-website.png} 

\centerline{\link{http://www.wireshark.org}}
\centerline{både til Windows og UNIX, tidligere kendt som Ethereal}

\slide{Programhygiejne!}

\begin{list1}
  \item {\color{red}Download, installer - kør!} - farligt!
\item Sådan gøres det:
  \begin{list2}
    \item download program OG signaturfil/MD5
\item verificer signatur eller MD5
\item installer
\item brug programmet
\item hold programmet opdateret!\\
Se eksempelvis teksten på hjemmesiden:\\
\emph{Wireshark 0.99.2 has been released. Several security-related vulnerabilities have been fixed and several new features have been added.}
  \end{list2}
\item NB: ikke alle programmer har signaturer :(
\item MD5 er en envejs hash algoritme - mere om det senere
\end{list1}


\slide{Brug af Wireshark}

\hlkimage{13cm}{images/ethereal-capture-options.png}

\centerline{Man starter med Capture - Options}

\slide{Brug af Wireshark}

\hlkimage{24cm}{images/ethereal-main-window.png}

\centerline{Læg mærke til filtermulighederne}

\exercise{ex:wireshark}

\slide{syslog}

\begin{list1}
\item syslog er system loggen på UNIX og den er effektiv
  \begin{list2}
\item man kan definere hvad man vil se og hvor man vil have det
  dirigeret hen
\item man kan samle det i en fil eller opdele alt efter programmer og
  andre kriterier
\item man kan ligeledes bruge named pipes - dvs filer i filsystemet
  som tunneller fra chroot'ed services til syslog i det centrale system! 
\item man kan nemt sende data til andre systemer
  \end{list2}
\item Hvis man vil lave en centraliseret løsning er følgende link
  vigtigt: \\
Tina Bird, Counterpane\\
\link{http://loganalysis.org}
\end{list1}

\slide{syslogd.conf eksempel}
\begin{alltt}
\small
*.err;kern.debug;auth.notice;authpriv.none;mail.crit    /dev/console
*.notice;auth,authpriv,cron,ftp,kern,lpr,mail,user.none /var/log/messages
kern.debug;user.info;syslog.info                        /var/log/messages
auth.info                                               /var/log/authlog
authpriv.debug                                          /var/log/secure
...
# Uncomment to log to a central host named "loghost".
#*.notice;auth,authpriv,cron,ftp,kern,lpr,mail,user.none        @loghost
#kern.debug,user.info,syslog.info                               @loghost
#auth.info,authpriv.debug,daemon.info                           @loghost
\end{alltt}

\slide{Andre syslogs syslog-ng}

\begin{list1}
\item der findes andre syslog systemer eksempelvis syslog-ng
\item konfigureres gennem \verb+/etc/syslog-ng/syslog-ng.conf+   
\item Eksempel på indholdet af filen kunne være:
\end{list1}

\begin{alltt}
\small 
options \{ 
        long_hostnames(off); 
        sync(0); 
        stats(43200); 
\};

source src { unix-stream("/dev/log"); internal(); pipe("/proc/kmsg"); };
destination messages { file("/var/log/messages"); };
destination console_all { file("/dev/console"); };
log { source(src); destination(messages); };
log { source(src); destination(console_all); };
\end{alltt}

\exercise{ex:syslogd-basic}

\slide{Logfiler og computer forensics}
\begin{list1}
\item Logfiler er en nødvendighed for at have et transaktionsspor
\item Logfiler er desuden nødvendige for at fejlfinde
\item Det kan være relevant at sammenholde logfiler fra:  
\begin{list2}
\item routere
\item firewalls
\item intrusion detection systemer
\item adgangskontrolsystemer
\item ...
\end{list2}
\item Husk - tiden er vigtig! Network Time Protocol (NTP) anbefales 
\item Husk at logfilerne typisk kan slettes af en angriber -
  hvis denne får kontrol med systemet
\end{list1}

\slide{Simple Network Management Protocol}

\begin{list1}
\item SNMP er en protokol der supporteres af de fleste professionelle
  netværksenheder, såsom switche, routere
\item hosts - skal slås til men følger som regel med
\item SNMP bruges til: 
  \begin{list2}
    \item \emph{network management}
    \item statistik
    \item rapportering af fejl - SNMP traps
  \end{list2}
\item {\bfseries sikkerheden baseres på community strings der sendes
    som klartekst ...}
\item det er nemmere at brute-force en community string end en
  brugerid/kodeord kombination
\end{list1}

\slide{SNMP - \emph{hacking}}

\vskip 2 cm

\begin{list1}
\item Simple Network Management Protocol
\item sikkerheden afhænger alene af en Community string SNMPv2
\item typisk er den nem at gætte:
  \begin{list2}
    \item public - default til at aflæse statistik
\item private - default når man skal ændre på enheden, skrive 
\item cisco
\item ...
  \end{list2}
\item Der findes lister og ordbøger på nettet over kendte default communities
\end{list1}

\slide{Systemer med SNMP}

\begin{list1}
  \item kan være svært at finde ... det er UDP 161
\item Hvis man finder en så prøv at bruge {\bfseries snmpwalk}
  programmet - det kan vise alle tilgængelige SNMP oplysninger fra den
  pågældende host 
\item det kan være en af måderne at identificere uautoriserede WLAN
  Access Points på - sweep efter port 161/UDP 
\item snmpwalk er et af de mest brugte programmer til at hente snmp
  oplysninger - i forbindelse med hackning og penetrationstest
\end{list1}

\slide{snmpwalk}

\begin{list1}
\item Typisk brug er:
\item \verb+snmpwalk -v 1 -c secret switch1+
\item \verb+snmpwalk -v 2c -c secret switch1+
\item Eventuelt bruges \verb+snmpget+ og \verb+snmpset+
\item Ovenstående er en del af Net-SNMP pakken, \link{http://net-snmp.sourceforge.net/}
\end{list1}

\exercise{ex:snmpwalk}

\slide{brute force}

\begin{list1}
\item hvad betyder bruteforcing?\\
afprøvning af alle mulighederne
\end{list1}

\begin{alltt}
\small
Hydra v2.5 (c) 2003 by van Hauser / THC <vh@thc.org>
Syntax: hydra [[[-l LOGIN|-L FILE] [-p PASS|-P FILE]] | [-C FILE]] 
[-o FILE] [-t TASKS] [-g TASKS] [-T SERVERS] [-M FILE] [-w TIME] 
[-f] [-e ns] [-s PORT] [-S] [-vV] server service [OPT]
Options:
  -S        connect via SSL
  -s PORT   if the service is on a different default port, define it here
  -l LOGIN  or -L FILE login with LOGIN name, or load several logins from FILE
  -p PASS   or -P FILE try password PASS, or load several passwords from FILE
  -e ns     additional checks, "n" for null password, "s" try login as pass
  -C FILE   colon seperated "login:pass" format, instead of -L/-P option
  -M FILE   file containing server list (parallizes attacks, see -T)
  -o FILE   write found login/password pairs to FILE instead of stdout
...  
\end{alltt}


\slide{Eksempler på SNMP og management}

\begin{list1}
\item Ofte foregår administration af netværksenheder via HTTP, Telnet eller SSH
\begin{list2}	
\item små dumme enheder er idag ofte web-enablet
\item bedre enheder giver både HTTP og kommandolinieadgang
\item de bedste giver mulighed for SSH, fremfor Telnet
\end{list2}
\end{list1}


\slide{Tobi Oetiker's MRTG The Multi Router Traffic Grapher}

\hlkimage{15cm}{rrdtool-demo.png}

\begin{list1}
\item Monitorering af SNMP enheder og grafer
\item Inkluderer en nem configmaker og benytter idag RRDTool til data
\item Hjemmesiden: \link{http://oss.oetiker.ch/mrtg/}
\end{list1}

\slide{RRDTool Round Robin Database Tool}

\hlkimage{12cm}{rrdtool-demo.png}

\begin{list1}
\item Round Robin Database Tool er en måde at gemme data på
\item Med RRDTool kan man derefter få lavet grafer
\item Typisk bruger man et andet værktøj som benytter RRDTool til data
\item \link{http://oss.oetiker.ch/rrdtool/doc/index.en.html}
\end{list1}

Kan bruges til temperaturmålinger og alt muligt andet

\slide{Smokeping}

\hlkimage{15cm}{smokeping-demo.jpg}

\begin{list1}
\item Måling af latency for netværksservice
\item Understøtter et stort antal prober: ICMP, DNS, HTTP, LDAP, SMTP, ...
\item Min SmokePing server \link{http://pumba.kramse.dk/smokeping/}
\item Hjemmesiden for SmokePing \link{http://oss.oetiker.ch/smokeping/}
\item Lavet af Tobias Oetiker og Niko Tyni
\end{list1}


\slide{Nagios}

\begin{list1}
\item Overvågningsværktøj der giver godt overblik
\begin{list2}	
\item Monitoring af diverse services (SMTP, POP3, HTTP, NNTP, PING, etc.)
\item Monitoring af host resources (processor load, disk and memory usage, running processes, log files, etc.)
\item Monitoring af andre ressourcer som temperatur
\item Simpel plugin design som gør det nemt at udvide
\item Kan sende e-mail, SMS m.v.
\end{list2}
\item Benyttes mange steder
\item Hjemmesiden for Nagios \link{http://www.nagios.org/}
\end{list1}

\slide{Stop - overvågningsværktøjer}

\begin{list1}
\item Brug lidt tid på at se på vores netværk
\item Valgfrit om I vil se på Administrationsinterface på switche, SNMP indstillinger eksempelvis
\item Eller Nagios og SmokePing på mine servere
\end{list1}


\slide{Dag 3 Dynamiske protokoller og services}

\hlkimage{20cm}{openbgpd-network-2.pdf}

% 802.11 start
%input fra wireless kursus, wireless præsentationer, wireless foredrag pentest III
%Husk WDS og bridging

\slide{Trådløse teknologier 802.11}

\begin{list1}
\item 802.11 er arbejdsgruppen under IEEE 
\item De mest kendte standarder idag indenfor trådløse teknologier:
\begin{list2}
\item 802.11b 11Mbps versionen
\item 802.11g 54Mbps versionen
\item 802.11n endnu hurtigere, og draft
\item 802.11i Security enhancements
\end{list2}
\item Der er proprietære versioner 22Mbps og den slags\\
- det anbefales IKKE at benytte disse da det giver vendor lock-in -
man bliver låst fast
\end{list1}

Kilde: \link{http://grouper.ieee.org/groups/802/11/index.html}

\slide{802.11 modes og frekvenser}

\begin{list1}
\item Access point kører typisk i \emph{access point mode} også kaldet
  infrastructure mode - al trafik går via AP
\item Alternativt kan wireless kort oprette ad-hoc netværk - hvor
  trafikken går direkte mellem netkort
\item Frekvenser op til kanal 11 og 12+13 i DK/EU
\item Helst 2 kanaler spring for 802.11b AP der placeres indenfor rækkevidde
\item Helst 4 kanaler spring for 802.11g AP der placeres indenfor rækkevidde
\end{list1}

\slide{Eksempel på netværk med flere AP'er}

\hlkimage{20cm}{images/wireless-multi-ap.pdf}

\slide{Eksempel på netværk med flere AP'er}

\hlkimage{20cm}{images/wireless-multi-ap-2.pdf}


\slide{Wireless Distribution System WDS}

\hlkimage{18cm}{images/wireless-multi-ap-wds.pdf}

\begin{list1}
\item Se også:
\link{http://en.wikipedia.org/wiki/Wireless_Distribution_System}
\end{list1}


\slide{\hskip 1 cm Er trådløse netværk interessante?}

\begin{list1}
\item Sikkerhedsproblemer i de trådløse netværk er mange
  \begin{list2}
  \item Fra lavt niveau - eksempelvis ARP, 802.11
  \item dårlige sikringsmekanismer - WEP
  \item dårligt udstyr - mange fejl
  \item usikkkerhed om implementering og overvågning
  \end{list2}
\item Trådløst udstyr er blevet meget billigt!
\item Det er et krav fra brugerne - trådløst er lækkert
\end{list1}


\slide{Konsekvenserne}

\hlkimage{10cm}{images/wireless-daekning.pdf}

\begin{list2}
\item Værre end Internetangreb - anonymt
\item Kræver ikke fysisk adgang til lokationer
%\emph{spioneres imod}
\item Konsekvenserne ved sikkerhedsbrud er generelt større
\item Typisk får man direkte LAN eller Internet adgang!
\end{list2}

\slide{Værktøjer}

\begin{list1}
\item Alle bruger nogenlunde de samme værktøjer, måske forskellige
  mærker
\begin{list2}
\item Wirelessscanner - Kismet og netstumbler
\item Wireless Injection - typisk på Linux
\item ...
\item Aircrack-ng
\end{list2}
\item Jeg anbefaler Auditor Security Collection og BackTrack boot CD'erne
\end{list1}


\slide{Konsulentens udstyr wireless}

\begin{list1}
\item Laptop med PC-CARD slot
\item Trådløse kort Atheros, de indbyggede er ofte ringe ;-)
\item Access Points - jeg anbefaler Airport Express
\item Antenner hvis man har lyst
\item Bøger: 
\begin{list2}
\item \emph{Real 802.11 security}
\item Se oversigter over bøger og værktøjer igennem præsentationen:
\end{list2}
\item Internetressourcer:
\begin{list2}
\item BackTrack - CD image med Linux+værktøjer    
\item Packetstorm wireless tools
\link{http://packetstormsecurity.org/wireless/}
\item \emph{Beginner's Guide to Wireless Auditing}
David Maynor 
\link{http://www.securityfocus.com/infocus/1877?ref=rss}
\end{list2}
\end{list1}


\slide{Typisk brug af 802.11 udstyr}

\begin{center}
\colorbox{white}{\includegraphics[width=20cm]{images/wlan-accesspoint-1.pdf}}
\end{center}

\centerline{\hlkbig et access point - forbindes til netværket}

\slide{Basal konfiguration}

\begin{list1}
\item Når man tager fat på udstyr til trådløse netværk opdager man:
\item SSID - nettet skal have et navn
\item frekvens / kanal - man skal vælge en kanal, eller udstyret
  vælger en automatisk
\item der er nogle forskellige metoder til sikkerhed  
\end{list1}

\slide{Trådløs sikkerhed}

\hlkimage{14cm}{images/apple-wireless-security.png}

\begin{list2}
\item Trådløs sikkerhed - WPA og WPA2
\item Nem konfiguration
\item Nem konfiguration af Access Point  
\end{list2}

\slide{Wireless networking sikkerhed i 802.11b}

\hlkimage{10cm}{images/wlan-accesspoint-1.pdf}

\begin{list1}
\item Sikkerheden er baseret på nogle få forudsætninger 
  \begin{list2}
  \item SSID - netnavnet
  \item WEP \emph{kryptering} - Wired Equivalent Privacy
  \item måske MAC flitrering, kun bestemte kort må tilgå accesspoint 
  \end{list2}
\item Til gengæld er disse forudsætninger ofte ikke tilstrækkelige ... 
%\item Hvorfor hader du WEP?
  \begin{list2}
  \item WEP er måske \emph{ok} til visse små hjemmenetværk
  \item WEP er baseret på en DELT hemmelighed som alle stationer kender
  \item nøglen ændres sjældent, og det er svært at distribuere en ny
  \end{list2}
  
\end{list1}


\slide{Forudsætninger}

\begin{list1}
\item Til gengæld er disse forudsætninger ofte ikke tilstrækkelige ... 
\item Hvad skal man beskytte?
\item Hvordan kan sikkerheden omgås?
\item Mange firmaer og virksomheder stille forskellige krav til
  sikkerheden - der er ikke en sikkerhedsmekanisme der passer til alle
\end{list1}

\slide{SSID - netnavnet}

\begin{list1}
\item Service Set Identifier (SSID) - netnavnet
\item 32 ASCII tegn eller 64 hexadecimale cifre
\item Udstyr leveres typisk med et standard netnavn
\begin{list2}
\item Cisco - tsunami
\item Linksys udstyr - linksys
\item Apple Airport, 3Com m.fl. - det er nemt at genkende dem  
\end{list2}
\item SSID kaldes også for NWID - network id
\item SSID broadcast - udstyr leveres oftest med broadcast af SSID
\end{list1}


%wardriving her
\slide{Demo: wardriving med stumbler programmer}

\hlkimage{17cm}{images/macstumbler.png}  

\begin{list1}
\item man tager et trådløst netkort og en bærbar computer og noget software:
\begin{list2}
\item Netstumbler - Windows \link{http://www.netstumbler.com}
\item dstumbler - UNIX \link{http://www.dachb0den.com/projects/dstumbler.html}
\item iStumbler - Mac \link{http://www.istumbler.net/}
\item Kismet ... mange andre
  \end{list2}
\end{list1}

\slide{Start på demo - wardriving}

\hlkimage{15cm}{images/server-client-wlan.pdf}

\begin{list2}
\item Almindelige laptops bruges til demo
\item Der startes et \emph{access point}
\end{list2}

\slide{MAC filtrering}

\begin{list1}
\item De fleste netkort tillader at man udskifter sin MAC adresse
\item MAC adressen på kortene er med i alle pakker der sendes
\item MAC adressen er aldrig krypteret, for hvordan skulle pakken så
  nå frem?
\item MAC adressen kan derfor overtages, når en af de tilladte
  stationer forlader området ...
\end{list1}

\slide{Resultater af wardriving}

\begin{list1}
\item Hvad opdager man ved wardriving?
\begin{list2}
\item at WEP IKKE krypterer hele pakken
\item at alle pakker indeholder MAC adressen
\item WEP nøglen skifter sjældent
\item ca. 2/3 af de netværk man finder har ikke WEP slået til - og der
  er fri og uhindret adgang til Internet
\end{list2}
\item {\color{red}
Man kan altså lytte med på et netværk med WEP, genbruge en anden
maskines MAC adresse - og måske endda bryde WEP krypteringen.}
\item 
Medmindre man kender virksomheden og WEP nøglen ikke er skiftet ...
det er besværligt at skifte den, idet alle stationer skal opdateres.
\end{list1}

\slide{Storkøbenhavn}

\begin{center}
\colorbox{white}{\includegraphics[width=20cm]{images/20030830-kbh.png}}  
\end{center}



\slide{Informationsindsamling}

\begin{list1}
\item Det vi har udført er informationsindsamling
\item Indsamlingen kan være aktiv eller passiv indsamling i forhold
  til målet for angrebet
\item passiv kunne være at lytte med på trafik eller søge i databaser
  på Internet
\item aktiv indsamling er eksempelvis at sende ICMP pakker og
  registrere hvad man får af svar
\end{list1}

\slide{WEP kryptering} 

%\begin{center}
%\colorbox{white}{\includegraphics[width=12cm]{images/airsnort.pdf}}  
%\end{center}
\begin{list1}
\item WEP \emph{kryptering} - med nøgler der specificeres som tekst
  eller hexadecimale cifre
\item typisk 40-bit, svarende til 5 ASCII tegn eller 10 hexadecimale
  cifre eller 104-bit 13 ASCII tegn eller 26 hexadecimale cifre
\item WEP er baseret på RC4 algoritmen der er en \emph{stream cipher}
  lavet af Ron Rivest for RSA Data Security
\end{list1}


\slide{De første fejl ved WEP}
\begin{list1}
\item Oprindeligt en dårlig implementation i mange Access Points
\item Fejl i krypteringen - rettet i nyere firmware
\item WEP er baseret på en DELT hemmelighed som alle stationer kender
\item Nøglen ændres sjældent, og det er svært at distribuere en ny
\end{list1}

\slide{WEP som sikkerhed}

\hlkimage{6cm}{images/no-wep.pdf}
\begin{list1}
\item WEP er \emph{ok} til et privat hjemmenetværk
\item WEP er for simpel til et større netværk - eksempelvis 20 brugere
\item Firmaer bør efter min mening bruge andre
  sikkerhedsforanstaltninger 
\item Hvordan udelukker man en bestemt bruger?
\end{list1}

\slide{WEP sikkerhed}

\hlkimage{12cm}{images/airsnort.pdf}  

\begin{quote}
AirSnort is a wireless LAN (WLAN) tool which recovers encryption
keys. AirSnort operates by passively monitoring transmissions,
computing the encryption key when enough packets have been gathered.  

802.11b, using the Wired Equivalent Protocol (WEP), is crippled with
numerous security flaws. Most damning of these is the weakness
described in " Weaknesses in the Key Scheduling Algorithm of RC4 "
by Scott Fluhrer, Itsik Mantin and Adi Shamir. Adam Stubblefield
was the first to implement this attack, but he has not made his
software public. AirSnort, along with WEPCrack, which was released
about the same time as AirSnort, are the first publicly available
implementaions of this attack.  \link{http://airsnort.shmoo.com/}
\end{quote}

%\begin{list1}
%\item i dag er firmware opdateret hos de fleste producenter
%\item men sikkerheden baseres stadig på een delt hemmelighed
%\end{list1}

\slide{major cryptographic errors}

\begin{list1}
\item weak keying - 24 bit er allerede kendt - 128-bit = 104 bit i praksis
\item small IV - med kun 24 bit vil hver IV blive genbrugt oftere
\item CRC-32 som integritetscheck er ikke \emph{stærkt} nok
  kryptografisk set
\item Authentication gives pad - giver fuld adgang - hvis der bare
  opdages \emph{encryption pad} for en bestemt IV. Denne IV kan så
  bruges til al fremtidig kommunikation
\end{list1}

{\hlkbig Konklusion: Kryptografi er svært}

\slide{WEP cracking - airodump og aircrack}

\hlkimage{3cm}{images/no-wep.pdf}

\begin{list1}
\item airodump - opsamling af krypterede pakker
\item aircrack - statistisk analyse og forsøg på at finde WEP nøglen
\item Med disse værktøjer er det muligt at knække \emph{128-bit nøgler}!
\item Blandt andet fordi det reelt er 104-bit nøgler \smiley
\item tommelfingerregel - der skal opsamles mange pakker ca. 100.000
  er godt
\item Links:\\
\link{http://www.cr0.net:8040/code/network/aircrack/} aircrack\\
\link{http://www.securityfocus.com/infocus/1814} WEP: Dead Again
\end{list1}

\slide{airodump afvikling}

\begin{list1}
\item Når airodump kører opsamles pakkerne
\item samtidig vises antal initialisationsvektorer IV's:
\end{list1}

\vskip 1 cm

\begin{alltt}
\hlktiny
   BSSID              CH  MB  ENC  PWR  Packets   LAN IP / # IVs   ESSID

   00:03:93:ED:DD:8D   6  11       209   {\bf 801963                  540180}   wanlan    
\end{alltt}

\vskip 2 cm

\begin{list1}
\item NB: dataopsamlingen er foretaget på 100\% opdateret Mac udstyr
\end{list1}


\slide{aircrack - WEP cracker}

\begin{alltt}
\hlktiny
   $ aircrack -n 128 -f 2 aftendump-128.cap
                                 aircrack 2.1
   * Got  540196! unique IVs | fudge factor = 2
   * Elapsed time [00:00:22] | tried 12 keys at 32 k/m
   KB    depth   votes
    0    0/  1   CE(  45) A1(  20) 7E(  15) 98(  15) 72(  12) 82(  12) 
    1    0/  2   62(  43) 1D(  24) 29(  15) 67(  13) 94(  13) F7(  13) 
    2    0/  1   B6( 499) E7(  18) 8F(  15) 14(  13) 1D(  12) E5(  10) 
    3    0/  1   4E( 157) EE(  40) 29(  39) 15(  30) 7D(  28) 61(  20) 
    4    0/  1   93( 136) B1(  28) 0C(  15) 28(  15) 76(  15) D6(  15) 
    5    0/  2   E1(  75) CC(  45) 39(  31) 3B(  30) 4F(  16) 49(  13) 
    6    0/  2   3B(  65) 51(  42) 2D(  24) 14(  21) 5E(  15) FC(  15) 
    7    0/  2   6A( 144) 0C(  96) CF(  34) 14(  33) 16(  33) 18(  27) 
    8    0/  1   3A( 152) 73(  41) 97(  35) 57(  28) 5A(  27) 9D(  27) 
    9    0/  1   F1(  93) 2D(  45) 51(  29) 57(  27) 59(  27) 16(  26) 
   10    2/  3   5B(  40) 53(  30) 59(  24) 2D(  15) 67(  15) 71(  12) 
   11    0/  2   F5(  53) C6(  51) F0(  21) FB(  21) 17(  15) 77(  15) 
   12    0/  2   E6(  88) F7(  81) D3(  36) E2(  32) E1(  29) D8(  27) 
         {\color{red}\bf KEY FOUND! [ CE62B64E93E13B6A3AF15BF5E6 ]}
\end{alltt}
%$


\slide{Hvor lang tid tager det?}

\begin{list1}
\item Opsamling a data - ca. en halv time på 802.11b ved optimale forhold
\item Tiden for kørsel af aircrack fra auditor CD 
på en Dell CPi 366MHz Pentium II laptop:
\end{list1}
\begin{alltt}

   $ time aircrack -n 128 -f 2 aftendump-128.cap
   ...
   real    5m44.180s   user  0m5.902s     sys  1m42.745s
   \end{alltt}
   %$

\pause
\begin{list1}
\item Tiden for kørsel af aircrack på en moderne 1.6GHz CPU med
  almindelig laptop disk tager typisk mindre end 60 sekunder
\end{list1}

\slide{Erstatning for WEP- WPA}

\begin{list1}
\item Det anbefales at bruge:
%\begin{list2}
\item Kendte VPN teknologier eller WPA
\item baseret på troværdige algoritmer
\item implementeret i professionelt udstyr
\item fra troværdige leverandører
\item udstyr der vedligeholdes og opdateres
%\end{list2}
\item Man kan måske endda bruge de eksisterende løsninger - fra
  hjemmepc adgang, mobil adgang m.v.
\end{list1}


\slide{RADIUS}
\begin{list1}
\item RADIUS er en protokol til autentificering af brugere op mod en
  fælles server 
\item Remote Authentication Dial In User Service (RADIUS)
\item RADIUS er beskrevet i RFC-2865
\item RADIUS kan være en fordel i større netværk med 
\begin{list2}
\item dial-in
\item administration af netværksudstyr
\item trådløse netværk
\item andre RADIUS kompatible applikationer
\end{list2}
\end{list1}


\slide{Erstatninger for WEP}
\begin{list1}
\item Der findes idag andre metoder til sikring af trådløse netværk
\item 802.1x Port Based Network Access Control 
\item WPA - Wi-Fi Protected Access)\\
WPA = 802.1X + EAP + TKIP + MIC
\item nu WPA2
\begin{quote}
WPA2 is based on the final IEEE 802.11i amendment to the 802.11
standard and is eligible for FIPS 140-2 compliance.
\end{quote}
\item Kilde: 
\href{http://www.wifialliance.org/OpenSection/protected_access.asp}
{http://www.wifialliance.org/OpenSection/protected\_access.asp}
\end{list1}

\slide{WPA eller WPA2?}

\begin{quote}
WPA2 is based upon the Institute for Electrical and Electronics
Engineers (IEEE) 802.11i amendment to the 802.11 standard, which was
ratified on July 29, 2004.  
\end{quote}

\begin{quote}
Q: How are WPA and WPA2 similar?\\
A: Both WPA and WPA2 offer a high level of assurance for end-users and network
administrators that their data will remain private and access to their
network restricted to authorized users.
Both utilize 802.1X and Extensible Authentication Protocol (EAP) for
authentication. Both have Personal and Enterprise modes of operation
that meet the distinct needs of the two different consumer and
enterprise market segments. 

Q: How are WPA and WPA2 different?\\
A: WPA2 provides a {\bf stronger encryption mechanism} through {\bf
  Advanced Encryption Standard (AES)}, which is a requirement for some
corporate and government users. 
\end{quote}

\centerline{Kilde: http://www.wifialliance.org WPA2 Q and A}

\slide{WPA Personal eller Enterprise}

\begin{list1}
\item Personal - en delt hemmelighed, preshared key
\item Enterprise - brugere valideres op mod fælles server
\item Hvorfor er det bedre?
\begin{list2}
\item Flere valgmuligheder - passer til store og små
\item WPA skifter den faktiske krypteringsnøgle jævnligt - TKIP 
\item Initialisationsvektoren (IV) fordobles 24 til 48 bit   
\item Imødekommer alle kendte problemer med WEP!
\item Integrerer godt med andre teknologier - RADIUS

\vskip 1 cm
\item EAP - Extensible Authentication Protocol - individuel autentifikation
\item TKIP - Temporal Key Integrity Protocol - nøgleskift og integritet
\item MIC - Message Integrity Code - Michael, ny algoritme til integritet
\end{list2}

\end{list1}


\slide{WPA cracking}

\begin{list1}
\item Nu skifter vi så til WPA og alt er vel så godt?  
\pause
\item Desværre ikke!
\item Du skal vælge en laaaaang passphrase, ellers kan man sniffe WPA
  handshake når en computer går ind på netværket!
\item Med et handshake kan man med aircrack igen lave off-line
  bruteforce angreb!
\end{list1}

\slide{WPA cracking demo}

\begin{list1}
\item Vi konfigurerer AP med Henrik42 som WPA-PSK/passhrase  
\item Vi finder netværk kismet eller airodump
\item Vi starter airodump mod specifik kanal
\item Vi spoofer deauth og opsamler WPA handshake
\item Vi knækker WPA :-)
\end{list1}

\centerline{Brug manualsiderne for programmerne i aircrack-ng pakken!}

\slide{WPA cracking med aircrack - start}

\begin{alltt}
\small
slax ~ # aircrack-ng -w dict wlan-test.cap
Opening wlan-test.cap
Read 1082 packets.

#  BSSID              ESSID           Encryption

1  00:11:24:0C:DF:97  wlan            WPA (1 handshake)
2  00:13:5F:26:68:D0  Noea            No data - WEP or WPA
3  00:13:5F:26:64:80  Noea            No data - WEP or WPA
4  00:00:00:00:00:00                  Unknown

Index number of target network ? {\bf 1}
\end{alltt}

\slide{WPA cracking med aircrack - start}

\begin{alltt}
\small
          [00:00:00] 0 keys tested (0.00 k/s)

                    KEY FOUND! [ Henrik42 ]

Master Key     : 8E 61 AB A2 C5 25 4D 3F 4B 33 E6 AD 2D 55 6F 76
                 6E 88 AC DA EF A3 DE 30 AF D8 99 DB F5 8F 4D BD
Transcient Key : C5 BB 27 DE EA 34 8F E4 81 E7 AA 52 C7 B4 F4 56
                 F2 FC 30 B4 66 99 26 35 08 52 98 26 AE 49 5E D7
                 9F 28 98 AF 02 CA 29 8A 53 11 EB 24 0C B0 1A 0D
                 64 75 72 BF 8D AA 17 8B 9D 94 A9 31 DC FB 0C ED

EAPOL HMAC     : 27 4E 6D 90 55 8F 0C EB E1 AE C8 93 E6 AC A5 1F

\end{alltt}

\vskip 1 cm

\centerline{Min Thinkpad X31 med 1.6GHz Pentium M knækker ca. 150 Keys/sekund}

\slide{Encryption key length}

\hlkimage{19cm}{encryption-crack.png}

Kilde: \link{http://www.mycrypto.net/encryption/encryption_crack.html}

\slide{WPA cracking med Pyrit}

\begin{quote}
\emph{Pyrit} takes a step ahead in attacking WPA-PSK and WPA2-PSK, the protocol that today de-facto protects public WIFI-airspace. The project's goal is to estimate the real-world security provided by these protocols. Pyrit does not provide binary files or wordlists and does not encourage anyone to participate or engage in any harmful activity. {\bf This is a research project, not a cracking tool.}

\emph{Pyrit's} implementation allows to create massive databases, pre-computing part of the WPA/WPA2-PSK authentication phase in a space-time-tradeoff. The performance gain for real-world-attacks is in the range of three orders of magnitude which urges for re-consideration of the protocol's security. Exploiting the computational power of GPUs, \emph{Pyrit} is currently by far the most powerful attack against one of the world's most used security-protocols. 
\end{quote}

\begin{list1}
\item sloooow, plejede det at være -  ~150 keys/s på min Thinkpad X31
\item Kryptering afhænger af SSID! Så check i tabellen er ~minutter.
\item \link{http://pyrit.wordpress.com/about/} 
\end{list1}

\slide{Tired of WoW?}

\hlkimage{22cm}{pyritperfaa3.png}

Kilde: \link{http://code.google.com/p/pyrit/}


\slide{Tools man bør kende}

\begin{list2}
\item Aircrack {http://www.aircrack-ng.org/}
\item Kismet \link{http://www.kismetwireless.net/}
\item Airsnort \link{http://airsnort.shmoo.com/} læs pakkerne med WEP
  kryptering  
\item Airsnarf \link{http://airsnarf.shmoo.com/} - lav dit eget AP
  parallelt med det rigtige og snif hemmeligheder
\item Wireless Scanner \link{http://www.iss.net/} - kommercielt 
%\item wepcrack \link{http://wepcrack.sourceforge.net/} - knæk
  krypteringen i WEP
%\item BSD Airtools \link{http://www.dachb0den.com/projects/bsd-airtools.html}
\item Dette er et lille uddrag af programmer\\
Se også \link{http://packetstormsecurity.org/wireless/}
\end{list2}

\slide{Når adgangen er skabt}

\begin{list1}
\item Så går man igang med de almindelige værktøjer
\item Fyodor Top 100 Network Security Tools \link{http://www.sectools.org}  
\end{list1}
\vskip 2 cm

\centerline{\hlkbig Forsvaret er som altid - flere lag af sikkerhed! }

\slide{Infrastrukturændringer}

\begin{center}
\colorbox{white}{\includegraphics[height=11cm]{images/wlan-accesspoint-2.pdf}}
\end{center}

\centerline{\hlkbig Sådan bør et access point forbindes til netværket}


\slide{Anbefalinger mht. trådløse netværk}

\begin{minipage}{10cm}
\includegraphics[width=10cm]{images/wlan-accesspoint-2.pdf}
\end{minipage}
\begin{minipage}{\linewidth-10cm}
\begin{list2}
\item Brug noget tilfældigt som SSID - netnavnet
\item Brug ikke WEP til virksomhedens netværk\\
- men istedet en VPN løsning med individuel
  autentificering eller WPA
\item NB: WPA Personal/PSK kræver passphrase på +40 tegn!
\item Placer de trådløse adgangspunkter hensigtsmæssigt i netværket -
  så de kan overvåges
\item Lav et sæt regler for brugen af trådløse netværk - hvor må 
  medarbejdere bruge det?
\item Se eventuelt pjecerne \emph{Beskyt dit trådløse Netværk} fra
Ministeriet for Videnskab, Teknologi og Udvikling \\
\link{http://www.videnskabsministeriet.dk/}
\end{list2}
\end{minipage} 


\slide{Hjemmenetværk for nørder}

\begin{list1}
\item Lad være med at bruge et wireless-kort i en PC til at lave AP, brug et AP
\item Husk et AP kan være en router, men den kan ofte også blot være en bro
\item Brug WPA og overvej at lave en decideret DMZ til WLAN
\item Placer AP hensigtsmæddigt og gerne højt, oppe på et skab eller lignende
\end{list1}

%\exercise{ex:AirPort-AP}

%\exercise{ex:wardriving-windows}
%\exercise{ex:wardriving-kismet}
%\exercise{ex:aircrack-ng}



\slide{Dynamisk routing} 

\begin{list1}
\item Når netværkene vokser bliver det administrativt svært at vedligeholde
\item Det skalerer dårligt med statiske routes til netværk
\item Samtidig vil man gerne have redundante forbindelser
\item Til dette brug har man STP på switch niveau og dynamisk routing på IP niveau
\end{list1}



% OpenBGPD

\slide{BGP Border Gateway Protocol}

\begin{list1}
\item Er en dynamisk routing protocol som benyttes eksternt
\item Netværk defineret med AS numre annoncerer hvilke netværk de er forbundet til
\item Autonomous System (AS) er en samling netværk
\item BGP version 4 er beskrevet i RFC-4271
\item BGP routere forbinder sig til andre BGP routere og snakker sammen, \emph{peering}
\item \link{http://en.wikipedia.org/wiki/Border_Gateway_Protocol}
\item Vores setup svarer til dette:
\item \link{http://www.kramse.dk/projects/network/openbgpd-basic_en.html}
\end{list1}

\slide{RIP Routing Information Protocol} 

\begin{list1}
\item Gammel routingprotokol som ikke benyttes mere
\item RIP er en distance vector routing protokol, tæller antal hops
\item \link{http://en.wikipedia.org/wiki/Routing_Information_Protocol}
\end{list1}


% OpenOSPFD

\slide{OSPF Open Shortest Path First}

\begin{list1}
\item Er en dynamisk routing protocol som benyttes til intern routing
\item OSPF version 3 er beskrevet i RFC-2740
\item OSPF bruger hverken TCP eller UDP, men sin egen protocol med ID 89
\item OSPF bruger en metric/cost pr link for at udregne smart routing 
\item \link{http://en.wikipedia.org/wiki/Open_Shortest_Path_First}
\item Vores setup svarer til OpenBGPD setup, blot med OpenOSPFD
\end{list1}


\slide{EIGRP}

\begin{list1}
\item Cisco protokol til intern routing, hvis man udelukkende har Cisco udstyr
\item \link{http://www.cisco.com}
\end{list1}

\slide{Stop - vi gennemgår og tester vores dynamiske routing}

\begin{list1}
\item Vi gennemgår hvordan vores setup ser ud
\item Vi laver traceroute før og efter:
\item Vi fjerner en ledning \emph{link down}
\item Vi stopper en router og ser de annoncerede netværk forsvinder
\item Vi booter en router og ser de annoncerede netværk igen
\end{list1}

\slide{Båndbreddestyring og policy based routing}

\begin{list1}
\item Mange routere og firewalls idag kan lave båndbredde allokering til
  protokoller, porte og derved bestemte services
\item Mest kendte er i Open Source:
\begin{list2}
\item ALTQ bruges på OpenBSD - integreret i PF
%  \link{http://www.csl.sony.co.jp/person/kjc/kjc/software.html}
\item FreeBSD har dummynet
\item Linux har tilsvarende\\
ADSL-Bandwidth-Management-HOWTO, ADSL Bandwidth Management HOWTO\\
Adv-Routing-HOWTO, Linux Advanced Routing \& Traffic Control HOWTO\\
\link{http://www.knowplace.org/shaper/resources.html} Linux resources
\end{list2}
\item Det kaldes også traffic shaping
\end{list1}


\slide{Routingproblemer, angreb}

\begin{list1}
  \item falske routing updates til protokollerne
\item sende redirect til maskiner
\item source routing - mulighed for at specificere en ønsket vej for
  pakken 
\item Der findes (igen) specialiserede programmer til at teste og
  forfalske routing updates, svarende til icmpush programmet
\item Det anbefales at sikre routere bedst muligt - eksempelvis 
Secure IOS template der findes på adressen:\\
{\small \link{http://www.cymru.com/Documents/secure-ios-template.html}}
\item Med UNIX systemer generelt anbefales opdaterede systemer og netværkstuning
\end{list1}


\slide{Source routing}

\begin{list1}
\item Hvis en angriber kan fortælle hvilken vej en pakke skal følge
  kan det give anledning til sikkerhedsproblemer
\item maskiner idag bør ikke lytte til source routing, evt. skal de
  droppe pakkerne
\end{list1}


\slide{Formålet med resten af dagen }

\hlkimage{15cm}{unix-servers.png}

\centerline{Vi skal gennemgå gængse internet-serverfunktioner}

\slide{Network Services}

\begin{list1}
\item Flere UNIX varianter har fået mere moderne strukturer til at
  starte services
\item SystemV start/stop af services er stadig meget udbredt rc.d
  katalogstrukturer 
\item Solaris: Service Management Facility SMF
\item AIX: Subsystem Ressource Controller  
\item Mac OS X: launchd
\item Windows: services, net stop/start m.fl.
\end{list1}

\slide{daemoner}

%chuckie billede
%\hlkimage{}{}

\begin{list1}
\item Hjælper med til at køre systemet
\item udfører jobs
\item typiske daemoner er:
  \begin{list2}
  \item ftpd - FTP daemonen giver FTP adgang til filoverførsler
\item Telnetd - giver login adgang - NB: ukrypteret!
\item tftpd - Trivial file transfer protocol daemon, bruges til boot
  og opgradering af netværksudstyr - kræver ikke password
\item pop3d - POP3 post office protocol, elektronisk post
\item sshd - SSH protokol daemonen giver adgang til login via SSH
  \end{list2}
\end{list1}

\slide{inetd en super server}

\begin{list1}
  \item inetd har mange funktioner
\item istedet for at have 10 programmer der lytter på diverse porte
  kan inetd lytte på en hel masse, og så give forbindelsen videre til
  programmerne når der er brug for det:
\item \verb+/etc/inetd.conf+
\end{list1}

\begin{alltt}
\small
finger stream  tcp     nowait  nobody  /usr/libexec/tcpd fingerd -s
ftp    stream  tcp     nowait  root    /usr/libexec/tcpd ftpd -l
login  stream  tcp     nowait  root    /usr/libexec/tcpd rlogind
nntp   stream  tcp     nowait  usenet  /usr/libexec/tcpd nntpd
ntalk  dgram   udp     wait    root    /usr/libexec/tcpd ntalkd
shell  stream  tcp     nowait  root    /usr/libexec/tcpd rshd
telnet stream  tcp     nowait  root    /usr/libexec/tcpd telnetd
uucpd  stream  tcp     nowait  root    /usr/libexec/tcpd uucpd
comsat dgram   udp     wait    root    /usr/libexec/tcpd comsat
tftp   dgram   udp     wait    nobody  /usr/libexec/tcpd tftpd /tftpboot
\end{alltt}

\slide{xinetd}

\begin{list1}
\item konfigureres med separate filer pr service i kataloget
\verb+/etc/xinetd.d+ eksempelvis: \verb+/etc/xinetd.d/cups-lpd+:
\end{list1}

\begin{alltt}
service printer
\{
        socket_type = stream
        protocol    = tcp
        wait        = no
        user        = lp
        server      = /usr/lib/cups/daemon/cups-lpd
        disable     = yes
\}  
\end{alltt}

\slide{UNIX Print systemer}

\hlkimage{6cm}{images/cups-happy.pdf}

\begin{list1}
\item De fleste benytter idag standard kommandoerne:
\begin{list2}
\item \verb+lp+ og \verb+lpr+ - print files
\item \verb+lpq+ - show printer queue status
\item \verb+lprm+ - cancel print jobs
\item Mange bruger softwaren Common UNIX Printing System fra
  \link{http://www.cups.org} 
\item Gamle UNIX systemer bruger stadig konfiguration via
  \verb+/etc/printcap+ 
\item remote print sker gennem Line Printer Daemon LPD protokollen port 515/tcp
\item nyere printere understøtter Internet Printing Protocol IPP port 80/tcp
\end{list2}
\end{list1}

Kilde: billede er fra CUPS


\slide{TFTP Trivial File Transfer Protocol}

\begin{list1}
\item Trivial File Transfer Protocol - uautentificerede filoverførsler
\item De bruges især til:
  \begin{list2}
\item TFTP bruges til boot af netværksklienter uden egen harddisk
\item TFTP benytter UDP og er derfor ikke garanteret at data overføres korrekt
  \end{list2}
\item TFTP sender alt i klartekst, hverken password \\ 
{\bfseries USER brugernavn} og \\
{\bfseries PASS hemmeligt-kodeord} 
\end{list1}


\slide{FTP File Transfer Protocol}

\begin{list1}
\item File Transfer Protocol - filoverførsler
\item Bruges især til:
  \begin{list2}
    \item FTP - drivere, dokumenter, rettelser - Windows Update? er
    enten HTTP eller FTP
  \end{list2}
\item FTP sender i klartekst\\ 
{\bfseries USER brugernavn} og \\
{\bfseries PASS hemmeligt-kodeord} 
\item Der findes varianter som tillader kryptering, men brug istedet SCP/SFTP over Secure Shell protokollen
\end{list1}


\slide{FTP Daemon konfiguration}

\begin{list1}
\item Meget forskelligt!
\item WU-FTPD er meget udbredt
\item BSD FTPD ligeså meget anvendt
\item \emph{anonym ftp} er når man tillader alle at logge ind\\
men husk så ikke at tillade upload af filer!
\item På BSD oprettes blot en bruger med navnet \verb+ftp+ så er der åbent!
\end{list1}


\slide{NTP Network Time Protocol}

\begin{list1}
\item NTP opsætning
\item foregår typisk i \verb+/etc/ntp.conf+ eller \verb+/etc/ntpd.conf+   
\item det vigtigste er navnet på den server man vil bruge som tidskilde
\item Brug enten en NTP server hos din udbyder eller en fra \link{http://www.pool.ntp.org/}
\item Eksempelvis:
\end{list1}

\begin{alltt}
server ntp.cybercity.dk

server 0.dk.pool.ntp.org
server 0.europe.pool.ntp.org
server 3.europe.pool.ntp.org

\end{alltt}

\slide{What time is it?}

\hlkimage{8cm}{images/xclock.pdf}

\begin{list1}
\item Hvad er klokken?
\item Hvad betydning har det for sikkerheden?
\item Brug NTP Network Time Protocol på produktionssystemer
\end{list1}


\slide{What time is it? - spørg ICMP}

\vskip 1 cm

\begin{list1}
  \item ICMP timestamp option - request/reply
\item hvad er klokken på en server
\item Slayer icmpush - er installeret på server
\item viser tidstempel
\end{list1}

\begin{alltt}
# {\bfseries icmpush -v -tstamp 10.0.0.12}
ICMP Timestamp Request packet sent to 10.0.0.12 (10.0.0.12)

Receiving ICMP replies ...
fischer         -> 21:27:17
icmpush: Program finished OK
\end{alltt}

\slide{Stop - NTP Konfigurationseksempler}

\hlkimage{12cm}{osx-ntp.png}

\begin{list1}
\item Vi har en masse udstyr, de meste kan NTP, men hvordan
\item Vi gennemgår, eller I undersøger selv:
\begin{list2}
\item Airport
\item Switche (managed)
\item Mac OS X
\item OpenBSD - check \verb+man rdate+ og \verb+man ntpd+
\end{list2}
\end{list1}

\slide{BIND DNS server}

\begin{list1}
\item Berkeley Internet Name Daemon server
\item Mange bruger BIND fra Internet Systems Consortium
   - altså Open Source
\item konfigureres gennem \verb+named.conf+
\item det anbefales at bruge BIND version 9
\end{list1}

\begin{list2}
\item \emph{DNS and BIND}, Paul Albitz \& Cricket Liu, O'Reilly, 4th
  edition Maj 2001 
\item \emph{DNS and BIND cookbook}, Cricket Liu, O'Reilly, 4th
  edition Oktober 2002 
\end{list2}

Kilde: \link{http://www.isc.org}




\slide{BIND konfiguration - et udgangspunkt}

\begin{alltt}
\small 
acl internals \{ 127.0.0.1; ::1; 10.0.0.0/24; \};
options \{
        // the random device depends on the OS !
        random-device "/dev/random"; directory "/namedb";
        port 53; version "Dont know"; allow-query \{ any; \};
\};
view "internal" \{
   match-clients \{ internals; \};
   recursion yes;
   zone "." \{
       type hint;   file "root.cache"; \};
   // localhost forward lookup
   zone "localhost." \{
        type master; file "internal/db.localhost";   \};
   // localhost reverse lookup from IPv4 address
   zone "0.0.127.in-addr.arpa" \{
        type master; file "internal/db.127.0.0"; notify no;   \};
...
\}
\end{alltt}

\exercise{ex:bind-version}

\exercise{ex:bind-config}

\exercise{ex:bind-dnszone}

\slide{Små DNS tools bind-version - Shell script}

\begin{alltt}\small
#! /bin/sh
# Try to get version info from BIND server
PROGRAM=`basename $0`
. `dirname $0`/functions.sh
if [ $# -ne 1 ]; then
   echo "get name server version, need a target! "
   echo "Usage: $0 target"
   echo "example $0 10.1.2.3"
   exit 0
fi
TARGET=$1
# using dig 
start_time
dig @$1 version.bind chaos txt
echo Authors BIND er i versionerne 9.1 og 9.2 - måske ...
dig @$1 authors.bind chaos txt
stop_time
\end{alltt}
\centerline{\link{http://www.kramse.dk/files/tools/dns/bind-version}}

\slide{Små DNS tools dns-timecheck - Perl script}

\begin{alltt}\small
#!/usr/bin/perl
# modified from original by Henrik Kramshøj, hlk@kramse.dk
# 2004-08-19
#
# Original from: http://www.rfc.se/fpdns/timecheck.html
use Net::DNS;

my $resolver = Net::DNS::Resolver->new;
$resolver->nameservers($ARGV[0]);

my $query = Net::DNS::Packet->new;
$query->sign_tsig("n","test");

my $response = $resolver->send($query);
foreach my $rr ($response->additional) {
  print "localtime vs nameserver $ARGV[0] time difference: ";
  print$rr->time_signed - time() if $rr->type eq "TSIG";
}  
\end{alltt}
% inserting stupid $ to stop EMACS from
\centerline{\link{http://www.kramse.dk/files/tools/dns/dns-timecheck}}


\slide{DHCPD server}

\begin{list1}
\item Dynamic Host Configuration Protocol Server
\item Mange bruger DHCPD fra Internet Systems Consortium\\
  \link{http://www.isc.org} - altså Open Source
\item konfigureres gennem \verb+dhcpd.conf+ - næsten samme syntaks som BIND 
\item DHCP er en efterfølger til BOOTP protokollen
\end{list1}

\begin{alltt}
\small
ddns-update-style ad-hoc;

shared-network LOCAL-NET \{
    option  domain-name "security6.net";
    option  domain-name-servers 212.242.40.3, 212.242.40.51;
    subnet 10.0.42.0 netmask 255.255.255.0 \{
            option routers 10.0.42.1;
            range 10.0.42.32 10.0.42.127;
    \}
\}  
\end{alltt}

\exercise{ex:dhcpd-config}





\slide{Logfiler}
\begin{list1}
\item Logfiler er en nødvendighed for at have et transaktionsspor
\item Logfiler giver mulighed for statistik
\item Logfiler er desuden nødvendige for at fejlfinde
\item Det kan være relevant at sammenholde logfiler fra:  
\begin{list2}
\item routere
\item firewalls
\item webservere
\item intrusion detection systemer
\item adgangskontrolsystemer
\item ...
\end{list2}
\item Husk - tiden er vigtig! Network Time Protocol (NTP) anbefales 
\item Husk at logfilerne typisk kan slettes af en angriber -
  hvis denne får kontrol med systemet
\end{list1}

% apache HTTPD

\slide{World Wide Web fødes}

\hlkimage{15cm}{images/tim-berners-lee-2001-europaeum-eighth.jpg}

\begin{list1}
\item Tim Berners-Lee opfinder WWW 1989 og den første webbrowser og
  server i 1990 mens han arbejder for CERN
\end{list1}

Kilde:
\link{http://www.w3.org/People/Berners-Lee/}

\slide{World Wide Web udviklingen}

\hlkimage{20cm}{images/Count_WWW.png}

\begin{list1}
\item Udviklingen på world wide web bliver en stor kommerciel success
\end{list1}

Kilde: Hobbes Internet time-line\\
\link{http://www.zakon.org/robert/internet/timeline/}

\slide{Nogle HTTP og webrelaterede RFC'er}

\begin{list2}
\item[1945] Hypertext Transfer Protocol -- HTTP/1.0. T. Berners-Lee, R.
     Fielding, H. Frystyk. May 1996.
\item[2068] Hypertext Transfer Protocol -- HTTP/1.1. R. Fielding, J. Gettys,
     J. Mogul, H. Frystyk, T. Berners-Lee. January 1997. (Obsoleted by
     RFC2616)
\item[2069] An Extension to HTTP : Digest Access Authentication. J. Franks,
     P. Hallam-Baker, J. Hostetler, P. Leach, A. Luotonen, E. Sink, L.
     Stewart. January 1997. (Obsoleted by
     RFC2617)
\item[2145] Use and Interpretation of HTTP Version Numbers. J. C. Mogul, R.
     Fielding, J. Gettys, H. Frystyk. May 1997. 
\item[2518] HTTP Extensions for Distributed Authoring -- WEBDAV. Y. Goland,
     E. Whitehead, A. Faizi, S. Carter, D. Jensen. February 1999.
\item[2616] Hypertext Transfer Protocol -- HTTP/1.1. R. Fielding, J. Gettys,
     J. Mogul, H. Frystyk, L. Masinter, P. Leach, T. Berners-Lee. June
     1999. (Obsoletes
     RFC2068) (Updated by RFC2817)
\item[2818] HTTP Over TLS. E. Rescorla. May 2000. 
\end{list2}

\begin{quote}
HTTP er basalt set en sessionsløs protokol bestående at individuelle
HTTP forespørgsler via TCP forbindelser  
\end{quote}

\slide{Infokager og state management}
\begin{list2}
\item[2109] HTTP State Management Mechanism. D. Kristol, L. Montulli.
     February 1997. (Format: TXT=43469 bytes) (Obsoleted by RFC2965)
     (Status: PROPOSED STANDARD)
\item[2965] HTTP State Management Mechanism. D. Kristol, L. Montulli. October
     2000. (Format: TXT=56176 bytes) (Obsoletes RFC2109) (Status: PROPOSED
     STANDARD)
\end{list2}
\begin{quote}
1.  ABSTRACT
   This document specifies a way to create a stateful session with HTTP
   requests and responses.  It describes two new headers, Cookie and
   Set-Cookie, which carry state information between participating
   origin servers and user agents.  The method described here differs
   from Netscape's Cookie proposal, but it can interoperate with
   HTTP/1.0 user agents that use Netscape's method.  (See the HISTORICAL
   section.)
\end{quote}

(Citatet er fra RFC-2109)

\slide{Apache HTTP serveren}

\hlkimage{14cm}{images/httpd_logo_wide.pdf}
\vskip 1 cm
{\hlkbig Hvorfor skrive Apache HTTP server?}

\vskip 1 cm
\begin{quote}
Fordi Apache idag er en organisation med mange delprojekter - hvoraf
mange relateres til web og webløsninger  
\end{quote}

\slide{Er Apache HTTP server interessant?}

\hlkimage{20cm}{images/netcraft-2004.pdf}

\begin{list1}
\item Apache HTTP server er iflg. netcraft og andre den mest benyttede
  HTTP server på Internet!
\item Apache er grand old man i Internet sammenhæng - bygget udfra
  NCSA HTTP serveren
\item Mange løsninger bygges på Apache
\end{list1}

Kilde: \link{http://www.netcraft.com}

\slide{Hvad er Apache?}

\hlkimage{8cm}{images/apache_pb.png}

\begin{quote}
The Apache HTTP Server Project is an effort to develop and maintain an
open-source HTTP server for modern operating systems including UNIX
and Windows NT. The goal of this project is to provide a secure,
efficient and extensible server that provides HTTP services in sync
with the current HTTP standards.   
\end{quote}

Kilde: Apache HTTPD FAQ \link{http://httpd.apache.org}


\slide{Fordele ved Apache HTTPD}

\begin{list1}
\item En HTTP webserver oprindeligt baseret på NCSA webserveren,
(National Center for Supercomputing Applications)\\
- Apache is "A PAtCHy server" 
\item Konfigurerbar og fleksibel
\item Understøtter moduler
\item Open Source og kildeteksten er tilgængelig med en fri licens
\item Understøtter HTTP/1.1
\item allestedsnærværende - UNIX: Linux, IBM AIX, BSD, Sun Solaris...
\end{list1}

Kilde: Apache HTTPD FAQ \link{http://httpd.apache.org}

\slide{Men Apache er også ...}
\begin{list1}
\item Apache Software Foundation med mange andre spændende projekter
\begin{list2}
\item Cocoon som er et komponentbaseret \emph{web development framework} 
\item Apache Tomcat som er en servlet container der bruges som den officielle
referenceimplementation for Java Servlet og JavaServer Pages
teknologierne
\item Apache-SSL SSL delen af webserveren
\item FOP print formatter drevet af XSL formatting objects (XSL-FO)
\item Xerces XML parser
\item Xalan XSLT processor
\end{list2}
\item XML og Web services er buzz-words idag
\end{list1}


\slide{Apache varianter}

\begin{list2}
\item Stronghold - sælges ikke mere\\
\link{http://www.redhat.com/software/stronghold/} 
\item IBM HTTP server\\
\link{http://www-306.ibm.com/software/webservers/httpservers/}  
\item Oracle HTTP Server
\item HP Secure Web Server for OpenVMS Alpha (based on Apache)
\item findes sikkert flere 
\end{list2}

\slide{Brug dokumentationen }
\hlkimage{14cm}{images/apache-docs.pdf}
\centerline{\link{http://httpd.apache.org/docs-2.0/}}


\slide{Apache kogebogen}

\hlkimage{6cm}{images/apacheckbk.png}

\begin{list2}
\item Vi bruger bogen Apache cookbook på kurset
\item Både som opgavehæfte og opslagsværk
\item \emph{Apache Cookbook} af Ken Coar, Rich Bowen, November 2003,
ISBN: 0-596-00191-6
\end{list2}

\slide{Apache Security bogen}

\hlkimage{6cm}{images/apachesc.png}

\begin{list2}
\item Vi bruger bogen Apache Security bogen på kurset
\item Primært som opslagsværk
\item \emph{Apache Security} af Ivan Ristic, February 2005,ISBN: 0-596-00724-8
\end{list2}

\slide{Start og stop af apache}

\begin{list1}
\item Apache bruger programmet \verb+apachectl+  
\item Dette program kan bruges til flere formål:
\begin{list2}
\item \verb+apachectl start+ - opstart af apache    
\item \verb+apachectl stop+ - stop af apache
\item \verb+apachectl restart+ - genstart af apache    
\item \verb+apachectl configtest+ - test af apache konfigurationen - syntaks!
\end{list2}
\item husk at der kan være flere versioner af apache på systemet!
\item Det kan være en fordel enten at lave et alias eller ændre PATH i
  jeres SHELL profil!\\
\verb+alias apachectl="/home/hlk/apache2/bin/apachectl"+ 
\end{list1}

\slide{ServerAdmin, ServerRoot, DocumentRoot}

\begin{alltt}\small
ServerAdmin webmaster@security6.net
ServerRoot "/usr/local/apache2"
DocumentRoot "/userdata/sites" 
User www
Group www
ServerName fluffy:80
\end{alltt}

\begin{list2}
%\item Nogle basale indstillinger for en Apache server
\item Indsættes i \verb+httpd.conf+
\item ServerAdmin - administratoren for denne webserver, bør sættes
  til eksempelvis webmaster@security6.net
\item ServerRoot - roden af serveren, mange andre referencer sker
  relativt til denne
\item DocumentRoot - den primære placering for filer der skal serveres
  fra denne server 
\item User og Group - hvilket brugerid skal serveren afvikles under,
  efter opstarten som root
\item Servername - hvad hedder denne server
\end{list2}


\slide{Apache Logfiler, konfiguration af logfiler}

\begin{verbatim}
LogLevel warn 
ErrorLog /usr/local/apache2/logs/error_log 
LogFormat "%h %l %u %t \"%r\" %>s %b \"%{Referer}i\" 
    %\"%{User-Agent}i\"" combined
CustomLog /usr/local/apache2/logs/access_log combined   
\end{verbatim}

\begin{list1}
\item Logning i Apache styres med log-direktiver til to slags - access
  og error 
\item De vigtigste direktiver i httpd.conf er:
\begin{list2}
\item LogLevel - hvad skal logges af beskeder i error log, fra emerg til debug
\item ErrorLog - default fejlbeskeder fra Apache 
\item LogFormat - hvordan skal access log se ud
\item CustomLog - hvor skal access log gemmes - default    
\end{list2}
\end{list1}

\slide{Tuning af Apache opstartsparametre}

\begin{list1}
\item Det anbefales at indstille Apache opstarten som noget af det
  første
\item UNIX bruger som standard prefork modellen hvor Apache starter et
  antal processer der forventes at være nogenlunde OK til den
  forventede belastning 
\end{list1}

\begin{alltt}
\small
# prefork MPM
# StartServers: number of server processes to start
# MinSpareServers: minimum number of server processes which are kept spare
# MaxSpareServers: maximum number of server processes which are kept spare
# MaxClients: maximum number of server processes allowed to start
# MaxRequestsPerChild: maximum number of requests a server process serves
<IfModule prefork.c>
StartServers         5
MinSpareServers      5
MaxSpareServers     10
MaxClients         150
MaxRequestsPerChild  0
</IfModule>
\end{alltt}

\slide{Virtuelle hosts}

\begin{alltt}
<VirtualHost *:80>
    ServerAdmin webmaster@security6.net
    ServerName www.security6.net
    ServerAlias security6.net
    ServerAlias www.security6.dk
    DocumentRoot /userdata/sites/security6.net
    ErrorLog logs/security6.net-error_log
    CustomLog logs/security6.net-access_log combined
...
</VirtualHost>
\end{alltt}

\begin{list1}
\item Apache HTTPD tillader at man benytter virtuelle hosts
\item Bemærk at det er klienten der overfører hostnavnet i HTTP request
\end{list1}

\exercise{ex:apache-httpd.conf}
\exercise{ex:apache-virtual}


\slide{Grundlæggende Apache CGI}

\begin{list1}
\item Common Gateway Interface - standard metode til programmer
\item \verb+ScriptAlias+ er det direktiv der angiver at CGI må
 afvikles
\item Der følger to eksempler med Apache 2 i ServerRoot/cgi-bin:
\begin{list2}
\item printenv - viser en del information om serveren
\item test-cgi - viser hvordan man kan bruge parametre  
\end{list2}
\item {\bf NB: husk at fjerne x-bit efter test!}
\end{list1}

\begin{verbatim}
ScriptAlias /cgi-bin/ "/usr/local/apache2/cgi-bin/"
<Directory "/usr/local/apache2/cgi-bin">
        AllowOverride None
        Options None
        Order allow,deny
        Allow from all
</Directory>
\end{verbatim}

\slide{Hello World CGI - Insecure programming}

\vskip 2 cm

\begin{list1}
\item Problem:\\
Ønsker et simpelt CGI program, en web udgave af finger
\item Formål:\\
Vise oplysningerne om brugere på systemet
\end{list1}

\slide{review af nogle muligheder}

\begin{list1}
\item ASP
\begin{list2}
\item server scripting, meget generelt - man kan alt
\end{list2}

\item SQL
\begin{list2}
\item databasesprog - meget kraftfuldt
\item mange databasesystemer giver mulighed for specifik tildeling af
  privilegier "grant" 
  \end{list2}
\item JAVA
\begin{list2}
\item generelt programmeringssprog
\item bytecode verifikation
\item indbygget sandbox funktionalitet 
\end{list2}
\item Perl og andre generelle programmeringssprog
\item Pas på shell escapes!!!
\end{list1}

\slide{Hello world of insecure web CGI}

\begin{list1}
\item Demo af et sårbart system - badfinger
\item Løsning:
\begin{list2}
\item Kalde finger kommandoen
\item et Perl script
\item afvikles som CGI 
\item standard Apache HTTPD 1.3 server
\end{list2}
\end{list1}

\slide{De vitale - og usikre dele}

{\small
\begin{verbatim}
print "Content-type: text/html\n\n<html>";
print "<body bgcolor=#666666 leftmargin=20 topmargin=20"; 
print "marginwidth=20 marginheight=20>";
print <<XX;
<h1>Bad finger command!</h1>
<HR COLOR=#000>
<form method="post" action="bad_finger.cgi">
Enter userid: <input type="text" size="40" name="command">
</form>
<HR COLOR=#000>
XX
if(&ReadForm(*input)){
    print "<pre>\n";
    print "will execute:\n/usr/bin/finger $input{'command'}\n";
    print "<HR COLOR=#000>\n";
    print `/usr/bin/finger $input{'command'}`;
    print "<pre>\n";
}
\end{verbatim}}

\slide{Almindelige problemer}

\begin{list1}
\item validering af forms
\item validering på klient er godt\\
- godt for brugervenligheden, hurtigt feedback
\item validering på clientside gør intet for sikkerheden
\item serverside validering er nødvendigt
\item generelt er input validering det største problem!
\end{list1}

Brug \emph{Open Web Application Security
Project} \link{http://www.owasp.org}

\slide{Apache HTTPD sikkerhedshuller}

\begin{list1}
\item En Apache installation er ikke bare en HTTPD server
\item ofte inkluderes:
  \begin{list2}
  \item OpenSSL til SSL, dvs HTTPS forbindelser
  \item PHP - et web programmeringssprog
  \item Perl - et programmeringssprog som ofte benyttes til web
  \end{list2}
\item Hver eneste komponent kan have sikkerhedsproblemer!
\end{list1}


\slide{Web løsninger før}

\hlkimage{15cm}{images/web-static.png}

\begin{list1}
\item Statiske hjemmesider i HTML
\item Overskueligt
\item Få regler i firewall
\item Ikke behov for adgang til data indenfor firewall
\end{list1}


\slide{Web løsninger idag}

\hlkimage{18cm}{images/web-dynamic.png}

\slide{Web løsninger idag}

\vskip 2 cm

\begin{list1}
\item Dynamiske hjemmesider - ASP, PHP m.fl.
\item Høj kompleksitet - flere muligheder for fejl
\item Mange regler i firewall - flere DMZ områder/net
\item Behov for adgang til ordredata m.v. indenfor firewall
\end{list1}


\slide{Gode Råd til dynamiske webmiljøer}

\vskip 2 cm

\begin{list1}
\item Brug databaser - der er gode muligheder for finmasket adgangskontrol
\item Brug versionsstyring - hvem lavede hvilket program, hvornår
\item Brug ressourcer på opdatering af medarbejdere
\item Lav retningslinier for webudvikling
\item Overvåg alle systemerne!
\end{list1}

\slide{Typisk fejl på webservere}

\begin{list1}
\item De mest alvorlige:
  \begin{list2}  
\item Ingen hærdning
\item Ingen opdatering efter idriftsættelse
\end{list2}
\item Medium eller kritiske
  \begin{list2}
  \item Adgang til eksempel-programmer (eng: sample programs)
    - kan til tider være meget kritisk!
  \end{list2}
\item De mindre alvorlige
  \begin{list2}
  \item Informationsindsamling
  \item Netmaske - \emph{icmpush -mask}
  \item Navne på udviklere, firmaer, datoer
  \end{list2}
\end{list1}


\slide{chroot og jails}

\begin{list1}
  \item Chroot står for change root, og betyder at processen som
  kalder chroot systemkaldet udskifter sin \emph{filsystemsrod-/} med
  et andet katalog på systemet
\item Oprindeligt blev denne funktion lavet til at teste nye UNIX
  releases uden at overskrive det oprindelige miljø man havde på
  systemet
\item men det kan bruges til at give mere sikkerhed
\item En daemon eller service der kører chroot'ed er sværere at
  udnytte - simpelthen fordi den kun har adgang til en lille del af
  systemet
\item FreeBSD har en endnu mere avanceret version af chroot som giver
  endnu mere kontrol over det miljø som programmerne ser
\end{list1}

\slide{brug af chroot}

\begin{list1}
\item De services man typisk vil chroot'e er BIND, Apache og andre
  udsatte services
\item Der findes heldigvis udførlige beskrivelser af hvordan man
chroote de mest almindelige services 
\item NB: husk at løsninger med Apache ofte kræver PHP, Perl,
  databaser osv. 
\end{list1}

\slide{Gennemgang af chroot konceptet}

\hlkimage{15cm}{images/named-pipe.pdf}
\centerline{named pipes og chroot}

\begin{list2}
\item Husk: Apachedelen kan være i chroot, mens databasesystem er udenfor -
forbindelser via TCP-sockets til localhost
\end{list2}


\slide{Produktionsmodning af miljøer}

\begin{list1}
\item Tænk på det miljø som servere og services skal udsættes for
\item Sørg for hærdning
\end{list1}

\slide{BIND sikring}
\begin{list1}
  \item Nedenstående kan bruges mod andre typer servere!
\item Sikringsforanstaltninger:
  \begin{list2}
  \item Opdateret software - ingen kendte sikkerhedshuller eller
  sårbarheder
\item fjern {\bfseries single points of failure} - er man afhængig af
  en ressource skal man ofte have en backup mulighed, redundant strøm
  eller lignende
\item adskilte servere - interne og eksterne til forskellige formål\\
Eksempelvis den interne postserver hvor alle e-mail opbevares og en
DMZ-postserver hvor ekstern post opbevares kortvarigt
\item lav filtre på netværket, eller på data - firewalls og proxy
  funktioner 
\item begræns adgangen til at læse information
\item begræns adgangen til at skrive information - dynamic updates på
  BIND, men samme princip til webløsninger og opdatering af databaser
\item {\bfseries least privileges} - sørg for at programmer og brugere
  kun har de nødvendige rettigheder til at kunne udføre opgaver
\item følg med på områderne der har relevans for virksomheden og
  \emph{jeres} installation - Windows, UNIX, BIND, Oracle, ...
  \end{list2}
\end{list1}

\slide{Change management}

\begin{list1}
\item Er der tilstrækkeligt med fokus på software i produktion
\item Kan en vilkårlig server nemt reetableres
\item Foretages rettelser direkte på produktionssystemer
\item Er der fall-back plan
\item Burde være god systemadministrator praksis
\end{list1}



\slide{Fundamentet skal være iorden}

\begin{list1}
\item Sørg for at den infrastruktur som I bygger på er sikker:
\begin{list2}
 \item redundans
       \item opdateret
        \item dokumenteret
        \item nem at vedligeholde
\end{list2}

\item  Husk tilgængelighed er også en sikkerhedsparameter
\end{list1}

\slide{CVS til konfigurationsfiler}

\begin{list1}
\item Det anbefales at bruge versionsstyring som eksempelvis CVS til
  konfigurationsfiler 
\item Det kan eksempelvis gøres på følgende måde:
\begin{list2}
\item \verb+mkdirhier /security6.net/cvshome /security6.net/etc+
\item \verb+export CVSROOT=/security6.net/cvshome+
\item \verb+cvs init+ - for at initialisere CVS repository
\item derefter kan man tilføje filer og lave CVS checkin og checkout
\item CVS bruger standard filrettigheder - så opret eventuelt en
  speciel gruppe til CVS brugere
\end{list2}
\item Læs mere om CVS eksempelvis på:\\
\link{http://cvsbook.red-bean.com/cvsbook.html}
\end{list1}

\slide{CVS eksempel med /etc/fstab}
\begin{alltt}
\small
# cd /security6.net/etc
# cvs import -m "initial CVS af etc" etc hlk start
# cd ..;rm -rf etc
# cvs co etc
cvs checkout: Updating etc
# cp /etc/fstab etc
# cvs add etc/fstab 
cvs add: scheduling file `fstab' for addition
cvs add: use 'cvs commit' to add this file permanently
# cvs commit -m "fstab initial"
cvs commit: Examining .
RCS file: /security6.net/cvshome/etc/fstab,v
done
Checking in fstab;
/security6.net/cvshome/etc/fstab,v  <--  fstab
initial revision: 1.1
done
\end{alltt}

\centerline{filer i /etc bør ikke flyttes, andre kan flyttes og sym-linkes}

\slide{individuel autentificering!}

\hlkimage{7cm}{images/ssh-root.pdf}

\begin{list1}
\item Mange UNIX systemer administreres fejlagtigt ved brug af
  root-login
\item Undgå direkte root-login
\item Insister på \verb+sudo+ eller \verb+su+  
\item Hvorfor?
\begin{list2}
\item Sporbarheden mistes hvis brugere logger direkte ind som root
\item Hvis et kodeord til root gættes er der direkte adgang til alt!    
\end{list2}
\end{list1}

\slide{SMTP Simple Mail Transfer Protocol}

\begin{alltt}\tiny
hlk@bigfoot:hlk$ telnet mail.kramse.dk 25
Connected to sunny.
220 sunny.kramse.dk ESMTP Postfix
HELO bigfoot
250 sunny.kramse.dk
MAIL FROM: Henrik
250 Ok
RCPT TO: hlk@kramse.dk
250 Ok
DATA
354 End data with <CR><LF>.<CR><LF>
hejsa
.
250 Ok: queued as 749193BD2
QUIT
221 Bye
\end{alltt}

\begin{list1}
\item RFC-821 SMTP Simple Mail Transfer Protocol fra 1982
\item RFC-2821 fra 2001 og flere andre er idag gældende 
\item \link{http://en.wikipedia.org/wiki/Simple_Mail_Transfer_Protocol}
\item Vedhæftede filer kodes i MIME Multipurpose Internet Mail Extensions
\item Bemærk at MIME encoding forøger størrelsen med ca. 30\%!
%\item Lad VÆRE med at sende store filer, dvs over 7-8MB via e-mail
\end{list1}

\slide{e-mail servere}

\begin{list1}
  \item Sendmail, qmail og postfix
\item Tre meget brugte e-mail systemer
  \begin{list2}
    \item Sendmail - den ældste og mest benyttede
\item Postfix en modulært og sikkerhedsmæssigt god e-mail server\\
er ligeledes nem at konfigurere
\item Qmail - en underlig mailserver lavet af Dan J Bernstein, med en
  speciel licens - ligesom programmøren
  \end{list2}
\item Dertil kommer diverse andre mailservere:
\item Microsoft Exchange på Windows servere
\item Jeg anbefaler at man har en postserver mod internet, der kun sender og modtager ekstern post, og en intern postserver der opbevarer al posten
\end{list1}


\slide{Sendmail postserveren}

\begin{alltt}\small
# "Smart" relay host (may be null)
DS
...
# strip group: syntax (not inside angle brackets!) and trailing semicolon
R$*                     $: $1 <@>                    mark addresses
R$* < $* > $* <@>       $: $1 < $2 > $3              unmark <addr>
R@ $* <@>               $: @ $1                      unmark @host:...
R$* [ IPv6 : $+ ] <@>   $: $1 [ IPv6 : $2 ]          unmark IPv6 addr
R$* :: $* <@>           $: $1 :: $2                  unmark node::addr
R:include: $* <@>       $: :include: $1              unmark :include:...
R$* : $* [ $* ]         $: $1 : $2 [ $3 ] <@>        remark if leading colon
R$* : $* <@>            $: $2                        strip colon if marked
\end{alltt}

\begin{list1}
\item mange konfigurerer Sendmail med \verb+sendmail.cf+ - det er
  {\color{red} forkert} 
\item man bør bruge M4 konfigurationen
\item  - desværre følger M4 filerne sjældent med :-(  
\item Sendmail er oprindeligt lavet af Eric Allman 
\end{list1}


\slide{Postfix postserveren}

\hlkimage{6cm}{postfix-mouse.png}

\begin{list1}
\item Lavet af Wietse Venema for IBM
\item Nem at konfigurere og sikker
\item \verb+main.cf+ findes typisk i kataloget \verb+/etc/postfix+  
\end{list1}

\slide{Audit af postservere}

\begin{list1}
\item Typisk findes konfigurationsfilerne til postservere under
  \verb+/etc+
\begin{list2}
\item \verb+/etc/mail+
\item \verb+/etc/postfix+    
\end{list2}
\item Det vigtigste er at den er opdateret og IKKE tillader relaying
\item Der findes diverse test-scripts til relaycheck på internet
\item Husk også at checke domæne records, MX og A
\end{list1}

\slide{Test af e-mail server}

\begin{alltt}
[hlk]$ {\bfseries telnet localhost 25} 
Connected.
Escape character is '^]'.
220 server ESMTP Postfix
{\bfseries helo test}
250 server
{\bfseries mail from: postmaster@pentest.dk}
250 Ok
{\bfseries rcpt to: root@pentest.dk}
250 Ok
{\bfseries data}
354 End data with <CR><LF>.<CR><LF>
{\bfseries skriv en kort besked}
.
250 Ok: queued as 91AA34D18
{\bfseries quit}
\end{alltt}
%$
\exercise{ex:email-server-config}

\slide{Postservere til klienter}

\begin{list1}
\item SMTP  som vi har gennemgået er til at sende mail mellem servere
\item Når vi skal hente post sker det typisk med POP3 eller IMAP
\begin{list2}
\item POP3 Post Office Protocol version 3 RFC-1939
\item Internet Message Access Protocol (typisk IMAPv4) RFC-3501
\end{list2}
\item Forskellen mellem de to er at man typisk med POP3 henter posten, hvor man med IMAP lader den ligge på serveren
\item POP3 er bedst hvis kun en klient skal hente
\item IMAP er bedst hvis du vil tilgå din post fra flere systemer
\item Jeg bruger selv IMAPS, IMAP over SSL kryptering - idet kodeord ellers sendes i klartekst
\end{list1}


\slide{POP3 i Danmark}

\hlkimage{15cm}{images/pop3-1.pdf}

\begin{list1}
\item Man har tillid til sin ISP - der administrerer såvel net som server
\end{list1}

\slide{POP3 i Danmark - trådløst}

\hlkimage{13cm}{images/pop3-wlan.pdf}
\begin{list1}
\item Har man tillid til andre ISP'er? Alle ISP'er?
\item Deler man et netværksmedium med andre?
\item {\color{green}Brug de rigtige protokoller!}
\end{list1}

\slide{Normal WLAN brug}

\hlkimage{22cm}{images/wlan-airpwn-1.pdf}

\slide{Packet injection - airpwn}

\hlkimage{22cm}{images/wlan-airpwn-2.pdf}

\slide{Airpwn teknikker}

\begin{list1}
\item Klienten sender forespørgsel
\item Hackerens program airpwn lytter og sender så falske pakker
\item Hvordan kan det lade sig gøre?
\begin{list2}
\item Normal forespørgsel og svar på Internet tager 50ms
\item Airpwn kan svare på omkring 1ms angives det
\item Airpwn har alle informationer til rådighed      
\end{list2}
\item Airpwn på Defcon 2004 - findes på Sourceforge\\
\link{http://airpwn.sourceforge.net/}
\item NB: Airpwn som demonstreret er begrænset til TCP og ukrypterede
  forbindelser 
\end{list1}


\slide{Dovenskab er en dyd}

\hlkimage{19cm}{images/unix-dovenskab.pdf}

\slide{Distribuerede filsystemer}

\begin{list1}
\item Til lokalnetværk:
  \begin{list2}
\item Windows filesharing - tidligere et stort sikkerhedshul
\item UNIX NFS - ikke beregnet til nutidens usikre netværk
\end{list2}
\item Over Internet: 
\item AFS - Andrew filesystem\\
  \link{http://www-2.cs.cmu.edu/afs/andrew.cmu.edu/usr/shadow/www/afs.html} 
\item CODA \link{http://www.coda.cs.cmu.edu/}
\item Tænk på de forudsætninger som et program har og forventer er til
  stede! 
\end{list1}


\slide{NFS - netværksfilsystem}

\begin{alltt}
 # sample /etc/exports file
 /               master(rw) trusty(rw,no_root_squash)
 /projects       proj*.local.domain(rw)
 /usr            *.local.domain(ro) @trusted(rw)
 /home/joe       pc001(rw,all_squash,anonuid=150,anongid=100)
 /pub            (ro,insecure,all_squash)  
\end{alltt}

\begin{list2}
\item UNIX NFS er netværksfilsystemet som alle UNIX varianter understøtter
\item Adgangen styres ved brug af \verb+/etc/exports+, eksempel fra
  manualen på Red Hat
\item De fleste bruger version 3 over UDP eller TCP selvom version 4
  burde have bedre sikkerhed 
\item Adgangen gives pr IP-adresse! IP adressebaseret autentifikation
  er pr definition dårlig!
\item Pas på - det er nemt at give root-adgang til andre maskiner!
\end{list2}

\exercise{ex:unix-rpcinfo}
\exercise{ex:unix-nfsinfo}

\slide{Samba og SMB/CIFS}

\begin{list1}
\item Microsoft Server Message Block bruges til netværksdrev og netværksprint i Windows miljøer
\item Samba er en open source implementation, som altid halter bagefter MS
\item De gamle implementationer overfører password i en uheldig version, som kan knækkes 7 tegn ad gangen - hurtigt
\item Microsoft har gennem tiden opdateret protokollen
\item Idag forsøger man at gøre det til en standard som Common Internet File System Protocol CIFS
\item Man kan læse om Samba på hjemmesiden \link{http://www.samba.org/}
\end{list1}



\slide{Dag 4 Netværkssikkerhed og firewalls}

\hlkimage{18cm}{server-owned.pdf}



\slide{Kryptografi}

\hlkimage{18cm}{images/crypto-rot13.pdf}

\begin{list1}
\item Kryptografi er læren om, hvordan man kan kryptere data
\item Kryptografi benytter algoritmer som sammen med nøgler giver en
  ciffertekst - der kun kan læses ved hjælp af den tilhørende nøgle
\end{list1}

\slide{Public key kryptografi - 1}

\hlkimage{18cm}{images/crypto-public-key.pdf}

\begin{list1}
\item privat-nøgle kryptografi (eksempelvis AES) benyttes den samme
  nøgle til kryptering og dekryptering 
\item offentlig-nøgle kryptografi (eksempelvis RSA) benytter to
  separate nøgler til kryptering og dekryptering
\end{list1}

\slide{Public key kryptografi - 2}

\hlkimage{18cm}{images/crypto-public-key-2.pdf}

\begin{list1}

\item offentlig-nøgle kryptografi (eksempelvis RSA) bruger den private
  nøgle til at dekryptere
\item man kan ligeledes bruge offentlig-nøgle kryptografi til at
  signere dokumenter - som så verificeres med den offentlige nøgle
\end{list1}


\slide{Kryptografiske principper}

\begin{list1}
\item Algoritmerne er kendte
\item Nøglerne er hemmelige
\item Nøgler har en vis levetid - de skal skiftes ofte
\item Et successfuldt angreb på en krypto-algoritme er enhver genvej
  som kræver mindre arbejde end en gennemgang af alle nøglerne 
\item Nye algoritmer, programmer, protokoller m.v. skal gennemgås nøje!
\item Se evt. Snake Oil Warning Signs:
Encryption Software to Avoid 
\link{http://www.interhack.net/people/cmcurtin/snake-oil-faq.html}
\end{list1}

\slide{DES, Triple DES og AES}

\hlkimage{15cm}{images/AES_head.png}

\begin{list1}
\item DES kryptering baseret på den IBM udviklede Lucifer algoritme
  har været benyttet gennem mange år. 
\item Der er vedtaget en ny standard algoritme Advanced Encryption
  Standard (AES) som afløser Data Encryption Standard (DES)
\item Algoritmen hedder Rijndael og er udviklet
af Joan Daemen og Vincent Rijmen.
%\item \emph{Rijndael is available for free. You can use it for
%whatever purposes  you want, irrespective of whether
%it is accepted as AES or not.}

\item Kilde:
\link{http://csrc.nist.gov/encryption/aes/}\\
\href{http://www.esat.kuleuven.ac.be/~rijmen/rijndael/}
{http://www.esat.kuleuven.ac.be/\~{}rijmen/rijndael/}
\end{list1}


\slide{Formålet med kryptering}

\vskip 3 cm
\centerline{\hlkbig kryptering er den eneste måde at sikre:}
\vskip 3 cm
\centerline{\hlkbig fortrolighed}
\vskip 3 cm
\centerline{\hlkbig autenticitet / integritet}


\slide{e-mail og forbindelser}

\begin{list1}
\item Kryptering af e-mail
\begin{list2}
\item Pretty Good Privacy - Phil Zimmermann
\item PGP = mail sikkerhed
\end{list2}
\item Kryptering af sessioner SSL/TLS
\begin{list2}
\item Secure Sockets Layer SSL / Transport Layer Services TLS
\item krypterer data der sendes mellem webservere og klienter
\item SSL kan bruges generelt til mange typer sessioner, eksempelvis
  POP3S, IMAPS, SSH m.fl.
\end{list2}
\vskip 1 cm 
\item Sender I kreditkortnummeret til en webserver der kører uden https?
\end{list1}

\slide{MD5 message digest funktion}

\hlkimage{16cm}{images/message-digest-1.pdf}

\begin{list1}
\item HASH algoritmer giver en unik værdi baseret på input
%\item output fra algoritmerne kaldes også message digest
%\item MD5 er et eksempel på en meget brugt algoritme
%\item MD5 algoritmen har følgende egenskaber:
%  \begin{list2}
%  \item output er 128-bit "fingerprint" uanset længden af input
\item værdien ændres radikalt selv ved små ændringer i input
%  \end{list2}
\item MD5 er blandt andet beskrevet i RFC-1321: The MD5 Message-Digest
  Algorithm 
%\item Algoritmen MD5 er baseret på MD4, begge udviklet af Ronald
%  L. Rivest kendt fra blandt andet RSA Data Security, Inc
\item Både MD5 og SHA-1 undersøges nøje og der er fundet kollisioner
  som kan påvirke vores brug i fremtiden - \emph{stay tuned}
\end{list1} 

%%% Local Variables: 
%%% mode: latex
%%% TeX-master: "tcpip-security"
%%% End: 



\slide{kryptering, OpenPGP}

\begin{list1}
  \item kryptering er den eneste måde at sikre:
    \begin{list2}
      \item fortrolighed
      \item autenticitet
    \end{list2}
\item kryptering består af:
  \begin{list2}
    \item Algoritmer - eksempelvis RSA
    \item \emph{protokoller} - måden de bruges på
\item programmer - eksempelvis PGP
\end{list2}
\item fejl eller sårbarheder i en af komponenterne kan formindske
  sikkerheden  
\item PGP = mail sikkerhed, se eksempelvis Enigmail plugin til Mozilla Thunderbird

\end{list1}

\slide{PGP/GPG verifikation af integriteten}

\begin{list1}
\item Pretty Good Privacy PGP
\item Gnu Privacy Guard GPG
\item Begge understøtter OpenPGP - fra IETF RFC-2440
\item Når man har hentet og verificeret en nøgle kan man fremover nemt
checke integriteten af software pakker
\end{list1}


\begin{alltt}
\small
hlk@bigfoot:postfix$ gpg --verify  postfix-2.1.5.tar.gz.sig
gpg: Signature made Wed Sep 15 17:36:03 2004 CEST using RSA key ID D5327CB9
gpg: Good signature from "wietse venema <wietse@porcupine.org>"
gpg:                 aka "wietse venema <wietse@wzv.win.tue.nl>"  
\end{alltt}
%$

\slide{Make and install programs from source}

\begin{list1}
\item Mange open source programmer kommer som en tar-fil  
\item De fleste C programmer benytter sig så af følgende kommando
\begin{list2}
\item konfigurer softwaren - undersøg hvilket operativsystem det er
\item byg software ved hjælp af en Makefile - kompilerer og linker
\item installer software - ofte i \verb+/usr/local/bin+    
\end{list2}
\end{list1}

\begin{alltt}
./configure;make;make install  
\end{alltt}

\slide{SSL og TLS}

\hlkimage{18cm}{ehandel-https.pdf}

\begin{list1}
\item Oprindeligt udviklet af Netscape Communications Inc.
\item Secure Sockets Layer SSL er idag blevet adopteret af IETF og kaldes
derfor også for Transport Layer Security TLS
TLS er baseret på SSL Version 3.0
\item RFC-2246 The TLS Protocol Version 1.0 fra Januar 1999
\end{list1}

\slide{SSL/TLS udgaver af protokoller}
\hlkimage{16cm}{imap-ssl.png}

\begin{list1}
\item Mange protokoller findes i udgaver hvor der benyttes SSL
\item HTTPS vs HTTP
\item IMAPS, POP3S, osv.
\item Bemærk: nogle protokoller benytter to porte IMAP 143/tcp vs IMAPS 993/tcp
\item Andre benytter den samme port men en kommando som starter:
\item SMTP STARTTLS RFC-3207
\end{list1}

\slide{Secure Shell - SSH og SCP}

%\begin{center}
%\colorbox{white}{\includegraphics[width=12cm]{images/tshirt-9b.jpg}}  
%\end{center}

\hlkimage{16cm}{images/openssh-banner.png}

\begin{list1}
\item Hvad er Secure Shell SSH?  
\item Oprindeligt udviklet af Tatu Ylönen i Finland,\\
se \link{http://www.ssh.com}
\item SSH afløser en række protokoller som er usikre:
  \begin{list2}
  \item Telnet til terminal adgang
  \item r* programmerne, rsh, rcp, rlogin, ...
  \item FTP med brugerid/password
  \end{list2}
\end{list1}


\slide{SSH - de nye kommandoer er}
\begin{list1}
\item kommandoerne er:
\begin{list2}
  \item ssh - Secure Shell
  \item scp - Secure Copy
  \item sftp - secure FTP 
  \end{list2}
\item Husk: SSH er både navnet på protokollerne - version 1 og 2 samt
  programmet \verb+ssh+ til at logge ind på andre systemer
\item SSH tillader også port-forward, tunnel til usikre protokoller,
  eksempelvis X protokollen til UNIX grafiske vinduer
\item {\bfseries NB: Man bør idag bruge SSH protokol version 2!}
\end{list1}


\slide{SSH nøgler}

I praksis benytter man nøgler fremfor kodeord
\begin{list1}
\item I kan lave jeres egne SSH nøgler med programmerne i Putty
\item Hvilken del skal jeg have for at kunne give jer adgang til en
  server?
\item Hvordan får jeg smartest denne nøgle?
\end{list1}

\slide{Installation af SSH nøgle}
\begin{list1}
\item Vi bruger login med password på kurset, men for
  fuldstændighedens skyld beskrives her hvordan nøgle installeres:

\begin{list2}
\item først skal der genereres et nøglepar {\bfseries id\_dsa og id\_dsa.pub}
\item Den offentlige del, filen id\_dsa.pub, kopieres til serveren
\item Der logges ind på serveren 
\item Der udføres følgende kommandoer:
\end{list2}
\end{list1}
\begin{alltt}
$ cd  \emph{skift til dit hjemmekatalog}
$ mkdir .ssh  \emph{lav et katalog til ssh-nøgler}
$ cat id\_dsa.pub >> .ssh/authorized\_keys  \emph{kopierer nøglen}
$ chmod -R go-rwx .ssh  \emph{skift rettigheder på nøglen}
\end{alltt}


\slide{OpenSSH konfiguration}

\begin{list1}
\item Sådan anbefaler jeg at konfigurere OpenSSH SSHD
\item Det gøres i filen \verb+sshd_config+ typisk \verb+/etc/ssh/sshd_config+  
\end{list1}

\begin{alltt}
\small
Port 22780
Protocol 2

PermitRootLogin no
PubkeyAuthentication yes
AuthorizedKeysFile      .ssh/authorized_keys
# To disable tunneled clear text passwords, change to no here!
PasswordAuthentication no

#X11Forwarding no
#X11DisplayOffset 10
#X11UseLocalhost yes
\end{alltt}

Det er en smagssag om man vil tillade \emph{X11 forwarding}



\slide{VLAN Virtual LAN}

\hlkimage{10cm}{vlan-portbased.pdf}

\begin{list1}
\item Nogle switche tillader at man opdeler portene
\item Denne opdeling kaldes VLAN og portbaseret er det mest simple
\item Port 1-4 er et LAN
\item De resterende er et andet LAN
\item Data skal omkring en firewall eller en router for at krydse fra VLAN1 til VLAN2
\end{list1}

\slide{IEEE 802.1q}

\hlkimage{20cm}{vlan-8021q.pdf}

\begin{list1}
\item Nogle switche tillader at man opdeler portene, men tillige benytter 802.1q
\item Med 802.1q tillades VLAN tagging på Ethernet niveau
\item Data skal omkring en firewall eller en router for at krydse fra VLAN1 til VLAN2
\item VLAN trunking giver mulighed for at dele VLANs ud på flere switches
\item Der findes administrationsværktøjer der letter dette arbejde: OpenNAC FreeNAC, Cisco VMPS
\end{list1}





\slide{IEEE 802.1x  Port Based Network Access Control}

\hlkimage{15cm}{osx-8021x.png}

\begin{list1}
\item Nogle switche tillader at man benytter 802.1x
\item Denne protokol sikrer at man valideres før der gives adgang til porten
\item Når systemet skal have adgang til porten afleveres brugernavn og kodeord/certifikat
\item Denne protokol indgår også i WPA Enterprise
\end{list1}


\slide{802.1x og andre teknologier}

\begin{list1}
\item 802.1x i forhold til MAC filtrering giver væsentlige fordele
\item MAC filtrering kan spoofes, hvor 802.1x kræver det rigtige kodeord
\item Typisk benyttes RADIUS og 802.1x integrerer således mod både LDAP og Active Directory
\end{list1}









%XXX \slide{input fra firewallskursus}




\slide{Hvad er en firewall}

\vskip 4 cm
\centerline{\hlkbig En firewall er noget som {\color{green}blokerer}
  traffik på Internet}  

\vskip 1 cm
\pause

\centerline{\hlkbig En firewall er noget som {\color{red}tillader}
  traffik på Internet}

\slide{Firewallrollen idag}

\begin{list1}
\item Idag skal en firewall være med til at:
\begin{list2}
\item Forhindre angribere i at komme ind
\item Forhindre angribere i at sende traffik ud
\item Forhindre virus og orme i at sprede sig i netværk
\item Indgå i en samlet løsning med ISP, routere, firewalls, switchede
  strukturer, intrusion detectionsystemer samt andre dele af infrastrukturen
\end{list2}
\item Det kræver overblik!
\end{list1}


\slide{firewalls}

\begin{itemize}
\item Basalt set et netværksfilter - det yderste fæstningsværk
\item Indeholder typisk:
  \begin{list2}
   \item Grafisk brugergrænseflade til konfiguration - er det en
   fordel?
\item TCP/IP filtermuligheder - pakkernes afsender, modtager, retning
  ind/ud, porte, protokol, ...
\item Kun IPv4 for de fleste kommercielle firewalls
\item Både IPv4 og IPv6 for Open Source firewalls: IPF, OpenBSD PF,
  Linux firewalls, ...
\item Foruddefinerede regler/eksempler - er det godt hvis det er nemt
  at tilføje/åbne en usikker protokol?
\item Typisk NAT funktionalitet indbygget
\item Typisk mulighed for nogle serverfunktioner: kan agere
  DHCP-server, DNS caching server og lignende
  \end{list2}
\item En router med Access Control Lists - ACL kaldes ofte
  netværksfilter, mens en dedikeret maskine kaldes firewall -
  funktionen er reelt den samme - der filtreres trafik
\end{itemize}


\slide{Packet filtering}

\begin{alltt}
\small
0                   1                   2                   3   
0 1 2 3 4 5 6 7 8 9 0 1 2 3 4 5 6 7 8 9 0 1 2 3 4 5 6 7 8 9 0 1 
+-+-+-+-+-+-+-+-+-+-+-+-+-+-+-+-+-+-+-+-+-+-+-+-+-+-+-+-+-+-+-+-+
|Version|  IHL  |Type of Service|          Total Length         |
+-+-+-+-+-+-+-+-+-+-+-+-+-+-+-+-+-+-+-+-+-+-+-+-+-+-+-+-+-+-+-+-+
|         Identification        |Flags|      Fragment Offset    |
+-+-+-+-+-+-+-+-+-+-+-+-+-+-+-+-+-+-+-+-+-+-+-+-+-+-+-+-+-+-+-+-+
|  Time to Live |    Protocol   |         Header Checksum       |
+-+-+-+-+-+-+-+-+-+-+-+-+-+-+-+-+-+-+-+-+-+-+-+-+-+-+-+-+-+-+-+-+
|                       Source Address                          |
+-+-+-+-+-+-+-+-+-+-+-+-+-+-+-+-+-+-+-+-+-+-+-+-+-+-+-+-+-+-+-+-+
|                    Destination Address                        |
+-+-+-+-+-+-+-+-+-+-+-+-+-+-+-+-+-+-+-+-+-+-+-+-+-+-+-+-+-+-+-+-+
|                    Options                    |    Padding    |
+-+-+-+-+-+-+-+-+-+-+-+-+-+-+-+-+-+-+-+-+-+-+-+-+-+-+-+-+-+-+-+-+  
\end{alltt}

\begin{list1}
\item Packet filtering er firewalls der filtrerer på IP niveau
\item Idag inkluderer de fleste statefull inspection 
\end{list1}

\slide{Kommercielle firewalls}
\begin{list2}
\item Checkpoint Firewall-1 \link{http://www.checkpoint.com}
\item Nokia appliances - Nokia IPSO \link{http://www.nokia.com}
\item Cisco PIX \link{http://www.cisco.com}
\item Clavister firewalls \link{http://www.clavister.com}
\item Netscreen - nu ejet af Juniper
  \link{http://www.juniper.net}
\end{list2}

Ovenstående er dem som jeg oftest ser ude hos mine kunder

\slide{Open source baserede firewalls}
\begin{list2} 
\item Linux firewalls - fra begyndelsen til det nuværende netfilter
  til kerner version 2.4 og 2.6\\
\link{http://www.netfilter.org}
\item Firewall GUIs ovenpå Linux - mange! IPcop, Guarddog, Watchguard
nogle Linux firewalls er kommercielle produkter
\item IP Filter (IPF) \link{http://coombs.anu.edu.au/~avalon/}
\item OpenBSD PF - findes idag på andre operativsystemer
\link{http://www.openbsd.org} 
\item FreeBSD IPFW og IPFW2 \link{http://www.freebsd.org}
\item Mac OS X benytter IPFW
\item FreeBSD inkluderer også OpenBSD PF
\item NetBSD - bruger IPF og er ved at inkludere OpenBSD PF
\end{list2}

NB: kun eksempler og dem jeg selv har brugt


\slide{Hardware eller software}


\begin{list1}
\item Man hører indimellem begrebet \emph{hardware firewall}  
\item Det er dog et faktum at en firewall består af:
\begin{list2}
\item Netværkskort - som er hardware
\item Filtreringssoftware - som er \emph{software}!    
\end{list2}
\item Det giver ikke mening at kalde en Zyxel 10 en hardware firewall
  og en Soekris med OpenBSD for en software firewall!
\item Det er efter min mening et marketingtrick
\vskip 1 cm
\item Man kan til gengæld godt argumentere for at en dedikeret
  firewall som en separat enhed kan give bedre sikkerhed
\end{list1}

\slide{TCP three way handshake}

\hlkimage{7cm}{images/tcp-three-way.pdf}

\begin{list2}
\item {\bfseries TCP SYN half-open} scans
\item Tidligere loggede systemer kun når der var etableret en fuld TCP
  forbindelse - dette kan/kunne udnyttes til \emph{stealth}-scans
\item Hvis en maskine modtager mange SYN pakker kan dette fylde
  tabellen over connections op - og derved afholde nye forbindelser
  fra at blive oprette - {\bfseries SYN-flooding}
\end{list2}

\slide{firewall regelsæt eksempel}

\begin{alltt}
\tiny 
# hosts
router="217.157.20.129"
webserver="217.157.20.131"
# Networks
homenet="{ 192.168.1.0/24, 1.2.3.4/24 }"
wlan="10.0.42.0/24"
wireless=wi0

# things not used
spoofed="{ 127.0.0.0/8, 172.16.0.0/12, 10.0.0.0/16, 255.255.255.255/32 }"

block in all # default block anything
# loopback and other interface rules
pass out quick on lo0 all
pass in quick on lo0 all

# egress and ingress filtering - disallow spoofing, and drop spoofed
block in quick from $spoofed to any
block out quick from any to $spoofed

pass in on $wireless proto tcp from $wlan to any port = 22
pass in on $wireless proto tcp from $homenet to any port = 22
pass in on $wireless proto tcp from any to $webserver port = 80

pass out quick proto tcp  from $homenet to any flags S/S keep state
pass out quick proto udp  from $homenet to any         keep state
pass out quick proto icmp from $homenet to any         keep state
\end{alltt}
%$



\slide{netdesign - med firewalls - 100\% sikkerhed?}

\begin{center}
\colorbox{white}{\includegraphics[width=12cm]{images/kut.jpg}}  
\end{center}

\begin{list1}
\item Hvor skal en firewall placeres for at gøre størst nytte?
\item Hvad er forudsætningen for at en firewall virker?\\
At der er konfigureret et sæt fornuftige regler!
\item Hvor kommer reglerne fra? Sikkerhedspolitikken!
%\item Kan man lave en 100\% sikker firewall? Ja selvfølgelig, se!
\end{list1}


\centerline{\small Kilde: Billedet er fra Marcus Ranum The ULTIMATELY
  Secure Firewall} 


\slide{Firewall er ikke alene}

\begin{list1}
\item Firewalls er ikke alene
\begin{list2}
\item anti-virus på klienter og postsystemer
\item IDS systemer
\item Backupsystemer
\item Adgangskontrol
\item ... mange andre ting er mindst ligeså vigtige
\end{list2}
\end{list1}

\centerline{\hlkbig Forsvaret er som altid - flere lag af sikkerhed! }


\slide{Firewall historik}


\hlkimage{6cm}{images/cheswick-cover2e.jpg}

\begin{list1}
\item Firewalls har været kendt siden starten af 90'erne
\item Den første bog \emph{Firewalls and Internet Security} udkom i
  1994 men der findes mange akademiske artikler om firewalls 
\item Bogen \emph{Firewalls and Internet Security} anbefales,  
William R. Cheswick, Steven M. Bellovin, Aviel D. Rubin,
Addison-Wesley, 2nd edition, 2003  
\end{list1}

\slide{An Evening with Berferd}


\begin{list1}
\item Artikel om en hacker der lokkes, vurderes, overvåges
\item Et tidligt eksempel på en honeypot
\item Idag anbefales The Honeynet Project hvis man vil vide mere
\\\link{http://www.honeynet.org}
\end{list1}




\slide{m0n0wall}

\hlkimage{20cm}{images/m0n0wall-1.pdf}


\slide{First or Last match firewall?}

\hlkimage{20cm}{images/first-last-match-1.pdf}


\slide{firewall koncepter}

\begin{list1}
\item Rækkefølgen af regler betyder noget!
\begin{list2}
\item To typer af firewalls:
 First match - når en regel matcher, gør det som angives block/pass
 Last match  - marker pakken hvis den matcher, til sidst afgøres block/pass
\end{list2}
\item {\bf Det er ekstremt vigtigt at vide hvilken type firewall
    man bruger!} 
\item OpenBSD PF er last match
\item FreeBSD IPFW er first match  
\item Linux iptables/netfilter er last match
\end{list1}

\slide{First or Last match firewall?}

\hlkimage{20cm}{images/first-last-match-1.pdf}
\begin{list2}
\item To typer af firewalls:
 First match - eksempelvis IPFW,
 Last match - eksempelvis PF
%\item {\bf Det er ekstremt vigtigt at vide hvilken type firewall
%    man bruger!} 
\end{list2}


\slide{First match - IPFW}

\begin{alltt}
\hlksmall
00100 16389  1551541 allow ip from any to any via lo0
00200     0        0 deny log ip from any to 127.0.0.0/8
00300     0        0 check-state
...
{\bfseries 
65435    36     5697 deny log ip from any to any}
65535   865    54964 allow ip from any to any
\end{alltt}

\vskip 2 cm

\centerline{Den sidste regel nås aldrig!}

\slide{Last match - OpenBSD PF}

\begin{alltt}
\small
ext_if="ext0"
int_if="int0"

block in
pass out keep state

pass quick on \{ lo $int_if \}

# Tillad forbindelser ind på port 80=http og port 53=domain
# på IP-adressen for eksterne netkort ($ext_if) syntaksen
pass in on $ext_if proto tcp to ($ext_if) port http keep state
pass in on $ext_if proto \{ tcp, udp \} to ($ext_if) port domain keep state
\end{alltt}

\vskip 2 cm
\centerline{Pakkerne markeres med block eller pass indtil sidste
  regel}
\centerline{nøgleordet \emph{quick} afslutter match - god til store
  regelsæt} 

\slide{Linux iptables/netfilter eksempel}

\begin{alltt}
\footnotesize
# Firewall configuration written by system-config-securitylevel
# Manual customization of this file is not recommended.
*filter
:INPUT ACCEPT [0:0]
:FORWARD ACCEPT [0:0]
:OUTPUT ACCEPT [0:0]
:RH-Firewall-1-INPUT - [0:0]
-A INPUT -j RH-Firewall-1-INPUT
-A FORWARD -j RH-Firewall-1-INPUT
-A RH-Firewall-1-INPUT -i lo -j ACCEPT
-A RH-Firewall-1-INPUT -p icmp --icmp-type any -j ACCEPT
-A RH-Firewall-1-INPUT -p 50 -j ACCEPT
-A RH-Firewall-1-INPUT -p 51 -j ACCEPT
-A RH-Firewall-1-INPUT -p udp --dport 5353 -d 224.0.0.251 -j ACCEPT
-A RH-Firewall-1-INPUT -p udp -m udp --dport 631 -j ACCEPT
-A RH-Firewall-1-INPUT -m state --state ESTABLISHED,RELATED -j ACCEPT
-A RH-Firewall-1-INPUT -m state --state NEW -m tcp -p tcp --dport 443 -j ACCEPT
-A RH-Firewall-1-INPUT -m state --state NEW -m tcp -p tcp --dport 22 -j ACCEPT
-A RH-Firewall-1-INPUT -j REJECT --reject-with icmp-host-prohibited
COMMIT
\end{alltt}

\centerline{NB: husk at aktivere IP forwarding}

\slide{Firewall GUI}

\hlkimage{24cm}{images/fwbuilder-screenshot1.png}

\begin{list1}
\item Der findes mange GUI programmer til Open Source firewalls
\end{list1}

Kilde: billede fra \link{http://www.fwbuilder.org}


\slide{m0n0wall}

\hlkimage{20cm}{images/m0n0wall-1.pdf}

Kilde: billede fra \link{http://m0n0.ch/wall/}

\slide{Firewalls og ICMP}


\begin{alltt}
ipfw add allow icmp from any to any icmptypes 3,4,11,12
\end{alltt}

\begin{list1}
\item Ovenstående er IPFW syntaks for at tillade de interessant ICMP beskeder igennem
\item Tillad ICMP types:
\begin{list2}
\item 3 Destination Unreachable
\item 4 Source Quench Message
\item 11 Time Exceeded
\item 12 Parameter Problem Message
\end{list2}
\end{list1}

\slide{Firewall konfiguration}

\begin{list1}
\item Den bedste firewall konfiguration starter med:
\begin{list2}
\item Papir og blyant
\item En fornuftig adressestruktur
\end{list2}
\item Brug dernæst en firewall med GUI første gang!
\item Husk dernæst:
\begin{list2}
\item En firewall skal passes
\item En firewall skal opdateres
\item Systemerne bagved skal hærdes!    
\end{list2}
\end{list1}

\slide{Bloker indefra og ud}

\begin{list1}
\item Der er porte og services som altid bør blokeres
\item Det kan være kendte sårbare services
\begin{list2}
\item Windows SMB filesharing - ikke til brug på Internet!
\item UNIX NFS - ikke til brug på Internet!
\end{list2}
\item Kendte problemer:
\begin{list2}
\item KaZaA og andre P2P programmer - hvis muligt!
\item Portmapper - port 111    
\end{list2}
\end{list1}

\slide{Firewall konfiguration}

\begin{list1}
\item Den bedste firewall konfiguration starter med:
\begin{list2}
\item Papir og blyant
\item En fornuftig adressestruktur
\end{list2}
\item Brug dernæst en firewall med GUI første gang!
\item Husk dernæst:
\begin{list2}
\item En firewall skal passes
\item En firewall skal opdateres
\item Systemerne bagved skal hærdes!    
\end{list2}
\end{list1}


\slide{En typisk firewall konfiguration}

\hlkimage{22cm}{images/firma-netvaerk.pdf}

\centerline{Opdeling i separate netværkssegmenter!}

\slide{personlige firewalls}

\begin{list1}
\item Personlige firewalls:  

\begin{list2}
\item Microsoft Windows XP
\item ZoneAlarm \link{http://www.zonelabs.com}  
\end{list2}
\item Personlige firewalls til Microsoft Windows inkluderer ofte
blokering af hvilket programmer der må tilgå netværk
\end{list1}

\centerline{\color{titlecolor}Det anbefales at bruge en personlig firewall}

Note: Lad være med at stille spørgsmål om logfilen i diverse fora!

{\bfseries Hvis du ikke forstår loggen så lad den ligge!}




\slide{Firewallværktøjer}
% måske til reference afsnit?

\begin{list1}
\item Der benyttes på kurset en del værktøjer:
\begin{list2}
\item nmap - \link{http://www.insecure.org} portscanner
\item Nessus - \link{http://www.nessus.org} automatiseret testværktøj
%\item libnet m.fl. - \link{http://www.packetfactory.net} - diverse projekter
%  relateret til pakker og IP netværk
%\item l0phtcrack - \link{http://www.atstake.com/research/lc/} - The Password
%  Auditing and Recovery Application
\item Ethereal - \link{http://www.ethereal.com} avanceret netværkssniffer
%\item F.I.R.E -  \link{http://biatchux.dmzs.com/} - en cd-rom der indeholder en 
%  bootable Linux del.
\item OpenBSD - \link{http://www.openbsd.org} operativsystem med fokus
  på sikkerhed 
\item m0n0wall - \link{http://www.m0n0.ch} gratis firewall baseret på FreeBSD

%\item \link{http://www.isecom.org/} - Open Source Security Testing
%  Methodology Manual - gennemgang af elementer der bør indgå i en struktureret test 
\end{list2}
\end{list1}

\slide{Specielle features}

\begin{list2}
\item Network Address Translation - NAT
\item IPv6 funktionalitet

\item Båndbredde håndtering
\item VLAN funktionalitet - mere udbredt i forbindelse med VoIP
\item Redundante firewalls - pfsync og CARP
% pfsync giver et indblik i hvordan den slags kan laves, hvor de
% kommercielle ``bare kan det''
\item IPsec og Andre VPN features
\end{list2}

\slide{Proxy servers}

\begin{list1}
\item Filtrering på højere niveauer i OSI modellen er muligt
\item Idag findes proxy applikationer til de mest almindelige
  funktioner
\item Den typiske proxy er en caching webproxy der kan foretage HTTP
  request på vegne af arbejdsstationer og gemme resultatet 
\item NB: nogle protokoller egner sig ikke til proxy servere
\item SSL forbindelser til \emph{secure websites} kan per design ikke proxies
\end{list1}

\slide{IPsec og Andre VPN features}

\begin{list1}
\item De fleste firewalls giver mulighed for at lave krypterede
  tunneler
\item Nyttigt til fjernkontorer der skal have usikker traffik henover
  usikre netværk som Internet 
\item Konceptet kaldes Virtual Private Network VPN
\item IPsec er de facto standarden for VPN og beskrevet i RFC'er 
\end{list1}


\slide{IPsec}

\begin{itemize}
\item Sikkerhed i netv�rket
\item RFC-2401 Security Architecture for the Internet Protocol
\item RFC-2402 IP Authentication Header (AH)
\item RFC-2406 IP Encapsulating Security Payload (ESP)
\item RFC-2409 The Internet Key Exchange (IKE) - dynamisk keying
\item B�de til IPv4 og IPv6
\item {\bfseries MANDATORY} i IPv6! - et krav hvis man implementerer
  fuld IPv6 support
\item god pr�sentation p� \link{http://www.hsc.fr/presentations/ike/}
\item Der findes IKEscan til at scanne efter IKE
  porte/implementationer\\
\link{http://www.nta-monitor.com/ike-scan/index.htm}
\end{itemize}

\slide{IPsec er ikke simpelt!}

\hlkimage{16cm}{images/ipsec-hsc.png}
\centerline{Kilde: \link{http://www.hsc.fr/presentations/ike/}}


\slide{RFC-2402 IP AH}

\begin{alltt}
\small
    0                   1                   2                   3
    0 1 2 3 4 5 6 7 8 9 0 1 2 3 4 5 6 7 8 9 0 1 2 3 4 5 6 7 8 9 0 1
   +-+-+-+-+-+-+-+-+-+-+-+-+-+-+-+-+-+-+-+-+-+-+-+-+-+-+-+-+-+-+-+-+
   | Next Header   |  Payload Len  |          RESERVED             |
   +-+-+-+-+-+-+-+-+-+-+-+-+-+-+-+-+-+-+-+-+-+-+-+-+-+-+-+-+-+-+-+-+
   |                 Security Parameters Index (SPI)               |
   +-+-+-+-+-+-+-+-+-+-+-+-+-+-+-+-+-+-+-+-+-+-+-+-+-+-+-+-+-+-+-+-+
   |                    Sequence Number Field                      |
   +-+-+-+-+-+-+-+-+-+-+-+-+-+-+-+-+-+-+-+-+-+-+-+-+-+-+-+-+-+-+-+-+
   |                                                               |
   +                Authentication Data (variable)                 |
   |                                                               |
   +-+-+-+-+-+-+-+-+-+-+-+-+-+-+-+-+-+-+-+-+-+-+-+-+-+-+-+-+-+-+-+-+
\end{alltt}

\slide{RFC-2402 IP AH}

Indpakning - pakkerne f�r og efter Authentication Header:
\begin{alltt}
\small
                BEFORE APPLYING AH
            ----------------------------
      IPv4  |orig IP hdr  |     |      |
            |(any options)| TCP | Data |
            ----------------------------

                  AFTER APPLYING AH
            ---------------------------------
      IPv4  |orig IP hdr  |    |     |      |
            |(any options)| AH | TCP | Data |
            ---------------------------------
            |<------- authenticated ------->|
                 except for mutable fields
\end{alltt}

\slide{RFC-2406 IP ESP}

Pakkerne f�r og efter:
\begin{alltt}
\small
               BEFORE APPLYING ESP
         ---------------------------------------
   IPv6  |             | ext hdrs |     |      |
         | orig IP hdr |if present| TCP | Data |
         ---------------------------------------



               AFTER APPLYING ESP
         ---------------------------------------------------------
   IPv6  | orig |hop-by-hop,dest*,|   |dest|   |    | ESP   | ESP|
         |IP hdr|routing,fragment.|ESP|opt*|TCP|Data|Trailer|Auth|
         ---------------------------------------------------------
                                   |<---- encrypted ---->|
                               |<---- authenticated ---->|
\end{alltt}

\slide{ipsec konfigurationsfiler}

\begin{list1}
%\item Der er f�lgende dokumenter til IPsec p� websitet\\
% \link{www.security.net/courses/ipsec}:
\item Der er f�lgende filer tilg�ngelige\\
  \begin{list2}
  \item konfigurationsfiler i NetBSD/FreeBSD/Mac OS X format - med
    \verb+setkey+ kommandoen
  \item konfigurationsfil til OpenBSD server - med \verb+ipsecadm+
    kommandoen
%  \item IKE.pdf \emph{Dynamic Management of the IPsec Parameters:
%      The IKE Protocol}, fra Herve Schauer Consultants
%\item NetBSD IPsec dokumentation
%\item Cisco \emph{Introduction to IP security}
  \end{list2}
\end{list1}


\slide{IPsec setup}

%\hlkimage{}{images/}

\begin{list1}
  \item Client: Mac OS X/NetBSD/FreeBSD - samme syntaks\\
\verb+rc.ipsec.client+

\item Server: OpenBSD - bruger ipsecadm kommando\\
\verb+rc.ipsec.server+

\item �velse til l�seren: lav samme i Cisco IOS
\item Det vil ofte v�re relevant at se p� IOS og IPsec i laboratoriet
\item Dette setup n�r vi ikke at demonstrere
\end{list1}

\slide{rc.ipsec.client - client setup - adresser}

\begin{verbatim}
#!/bin/sh
# /etc/rc.ipsec.client - IPsec client configuration
# built from http://rt.fm/~jcs/ipsec_wep.phtml
# FreeBSD/NetBSD syntaks! - used on Mac OS X
# IPv4
SECSERVER=10.0.42.1
SECCLIENT=10.0.42.53
# IPv6
#SECSERVER=2001:618:433:101::1
#SECCLIENT=2001:618:433:101::153
ESPKEY=`cat ipsec.esp.key`
AHKEY=`cat ipsec.ah.key`

# Flush IPsec SAs in case we get called more than once
setkey -F
setkey -F -P
\end{verbatim}

\slide{rc.ipsec.client - client setup - SAs}

\begin{verbatim}
# Establish Security Associations
# 1000 is from the server to the client
# 1001 is from the client to the server
setkey -c <<EOF

add $SECSERVER $SECCLIENT esp 0x1000 \
-m tunnel -E blowfish-cbc 0x$ESPKEY  -A hmac-sha1 0x$AHKEY;

add $SECCLIENT $SECSERVER esp 0x1001 \
-m tunnel -E blowfish-cbc 0x$ESPKEY -A hmac-sha1 0x$AHKEY;

spdadd $SECCLIENT $SECSERVER any -P out \
ipsec esp/tunnel/$SECCLIENT-$SECSERVER/default;

spdadd $SECSERVER $SECCLIENT any -P in \
ipsec esp/tunnel/$SECSERVER-$SECCLIENT/default;
EOF
\end{verbatim}

\slide{rc.ipsec.server - server setup - adresser}

\begin{verbatim}
#!/bin/sh
#
# /etc/rc.ipsec - IPsec server configuration
# built from http://rt.fm/~jcs/ipsec_wep.phtml
# OpenBSD syntaks!
SECSERVER=10.0.42.1
SECCLIENT=10.0.42.53
#SECSERVER6=2001:618:433:101::1
#SECCLIENT6=2001:618:433:101::153

ESPKEY=`cat ipsec.esp.key`
AHKEY=`cat ipsec.ah.key`

# Flush IPsec SAs in case we get called more than once
ipsecadm flush
\end{verbatim}


\slide{rc.ipsec.server - server setup - SAs}

\begin{verbatim}
# Establish Security Associations
#
# 1000 is from the server to the client
ipsecadm new esp -spi 1000 -src $SECSERVER -dst $SECCLIENT \
-forcetunnel -enc blf -key $ESPKEY \
-auth sha1 -authkey $AHKEY

# 1001 is from the client to the server
ipsecadm new esp -spi 1001 -src $SECCLIENT -dst $SECSERVER \
-forcetunnel -enc blf -key $ESPKEY \
-auth sha1 -authkey $AHKEY
\end{verbatim}


\slide{rc.ipsec.server - server setup - flows}

\small
\begin{verbatim}
# Create flows
#
# Data going from the outside to the client
ipsecadm flow -out -src $SECSERVER -dst $SECCLIENT -proto esp \
-addr 0.0.0.0 0.0.0.0 $SECCLIENT 255.255.255.255 -dontacq
# IPv6
#ipsecadm flow -out -src $SECSERVER -dst $SECCLIENT -proto esp \
#-addr :: :: $SECCLIENT ffff:ffff:ffff:ffff:ffff:ffff:ffff:ffff -dontacq

# Data going from the client to the outside
ipsecadm flow -in -src $SECSERVER -dst $SECCLIENT -proto esp \
-addr $SECCLIENT 255.255.255.255 0.0.0.0 0.0.0.0 -dontacq
# IPv6
#ipsecadm flow -in -src $SECSERVER -dst $SECCLIENT -proto esp \
#-addr :: :: $SECCLIENT ffff:ffff:ffff:ffff:ffff:ffff:ffff:ffff -dontacq
\end{verbatim}



%%% Local Variables:
%%% mode: latex
%%% TeX-master: t
%%% End:


\slide{OpenVPN / OpenSSL VPN}

\begin{quote}
OpenVPN is a full-featured SSL VPN solution which can accomodate a
wide range of configurations, including remote access, site-to-site
VPNs, WiFi security, and enterprise-scale remote access solutions with
load balancing, failover, and fine-grained access-controls (articles)
(examples) (security overview) (non-english languages).   
\end{quote}

\begin{list1}
\item Et andet populært VPN produkt er OpenVPN
\item Bemærk dog at hvis der benyttes TCP i TCP risikerer man at støde ind i 
et problem som kaldes \emph{TCP in TCP meltdown} 
\item Kilde: \link{http://openvpn.net/}  
\end{list1}



\exercise{ex:unix-basic-firewall}










\slide{Portscan, TCP, UDP og ICMP}

Forskellen mellem TCP og UDP i forbindelse med portscan, og effekten af en firewall der dropper pakker

\slide{Basal Portscanning}

\begin{list1}
  \item Hvad er portscanning
\item afprøvning af alle porte fra 0/1 og op til 65535
\item målet er at identificere åbne porte - sårbare services
\item typisk TCP og UDP scanning
\item TCP scanning er ofte mere pålidelig end UDP scanning
\end{list1}

{\hlkbig TCP handshake er nemmere at identificere

UDP applikationer svarer forskelligt - hvis overhovedet}

\slide{TCP three way handshake}

\hlkimage{7cm}{images/tcp-three-way.pdf}

\begin{list2}
\item {\bfseries TCP SYN half-open} scans
\item Tidligere loggede systemer kun når der var etableret en fuld TCP
  forbindelse - dette kan/kunne udnyttes til \emph{stealth}-scans
\item Hvis en maskine modtager mange SYN pakker kan dette fylde
  tabellen over connections op - og derved afholde nye forbindelser
  fra at blive oprette - {\bfseries SYN-flooding}
\end{list2}


\slide{Ping og port sweep}

\begin{list1}
\item scanninger på tværs af netværk kaldes for sweeps 
\item Scan et netværk efter aktive systemer med PING
\item Scan et netværk efter systemer med en bestemt port åben
\item Er som regel nemt at opdage:
  \begin{list2}
    \item konfigurer en maskine med to IP-adresser som ikke er i brug
\item hvis der kommer trafik til den ene eller anden er det portscan
\item hvis der kommer trafik til begge IP-adresser er der nok
  foretaget et sweep - bedre hvis de to adresser ligger et stykke fra hinanden
  \end{list2}

\end{list1}

\slide{nmap port sweep efter port 80/TCP}

\begin{list1}
  \item Port 80 TCP er webservere
\end{list1}

\begin{alltt}
\small # {\bfseries nmap  -p 80 217.157.20.130/28}

Starting nmap V. 3.00 ( www.insecure.org/nmap/ )
Interesting ports on router.kramse.dk (217.157.20.129):
Port       State       Service
80/tcp     filtered    http                    

Interesting ports on www.kramse.dk (217.157.20.131):
Port       State       Service
80/tcp     open        http                    

Interesting ports on  (217.157.20.139):
Port       State       Service
80/tcp     open        http                    

\end{alltt}

\slide{nmap port sweep efter port 161/UDP}

\begin{list1}
  \item Port 161 UDP er SNMP
\end{list1}

\begin{alltt}  
\small # {\bfseries nmap -sU -p 161 217.157.20.130/28}

Starting nmap V. 3.00 ( www.insecure.org/nmap/ )
Interesting ports on router.kramse.dk (217.157.20.129):
Port       State       Service
161/udp    open        snmp                    

The 1 scanned port on mail.kramse.dk (217.157.20.130) is: closed

Interesting ports on www.kramse.dk (217.157.20.131):
Port       State       Service
161/udp    open        snmp                    

The 1 scanned port on  (217.157.20.132) is: closed
\end{alltt}

\slide{OS detection}
\begin{alltt}
\footnotesize
# nmap -O ip.adresse.slet.tet \emph{scan af en gateway}
Starting nmap 3.48 ( http://www.insecure.org/nmap/ ) at 2003-12-03 11:31 CET
Interesting ports on gw-int.security6.net (ip.adresse.slet.tet):
(The 1653 ports scanned but not shown below are in state: closed)
PORT     STATE SERVICE
22/tcp   open  ssh
80/tcp   open  http
1080/tcp open  socks
5000/tcp open  UPnP
Device type: general purpose
Running: FreeBSD 4.X
OS details: FreeBSD 4.8-STABLE
Uptime 21.178 days (since Wed Nov 12 07:14:49 2003)
Nmap run completed -- 1 IP address (1 host up) scanned in 7.540 seconds
\end{alltt}

\begin{list2}
  \item lavniveau måde at identificere operativsystemer på
\item send pakker med \emph{anderledes} indhold
\item Reference: \emph{ICMP Usage In Scanning} Version 3.0,
  Ofir Arkin\\ \link{http://www.sys-security.com/html/projects/icmp.html}
\end{list2}

\slide{Top 75 Security Tools}

\begin{list1}
%  \item I er meget ivrige efter at afprøve en masse
\item listen over 75 top security
  tools - nogle værktøjer springes over, nogle har vi brugt
\item Den er samlet af Fyodor og findes på:\\
\link{http://www.insecure.org/tools.html}
\end{list1}


\slide{Hvad skal der ske?}

\begin{list1}
\item Tænk som en hacker
\item Rekognoscering
\begin{list2}
\item ping sweep, port scan
\item OS detection - TCP/IP eller banner grab
\item Servicescan - rpcinfo, netbios, ...
\item telnet/netcat interaktion med services
\end{list2}
\item Udnyttelse/afprøvning: Nessus, nikto, exploit programs
\item Oprydning vises ikke på kurset, men I bør i praksis:
\begin{list2}
\item Lav en rapport
\item Gennemgå rapporten, registrer ændringer
\item Opdater programmer, konfigurationer, arkitektur, osv. 
\end{list2}
\item I skal jo også VISE andre at I gør noget ved sikkerheden.
\end{list1}


\exercise{ex:nmap-sweep}
\exercise{ex:nmap-portscan}
\exercise{ex:nmap-service}
\exercise{ex:nmap-os}



\slide{Firewalls og IPv6}

\begin{list1}
\item Læg mærke til forskellen mellem ARP og ICMPv6  
\item Hvis det er muligt lav een regel der tillader adgang til services uanset protokol
\item NB: husk at aktivere IP forwarding når I skal lave en firewall
\end{list1}


\slide{OpenBSD PF}
\begin{alltt}
\footnotesize
# Macros: define common values, so they can be referenced and changed easily.
int_if=vr0
ext_if=vr2
tunnel_if=gif0
table <homenet6> { 2001:16d8:ffd2:cf0f::/64 }
set skip on lo0
scrub in all
# Filtering: the implicit first two rules are
block in all
block out all
# allow ICMPv6 for NDP
pass in inet6 proto ipv6-icmp all icmp6-type neighbradv keep state
# server with configured IP address and router advertisement daemon running
pass out inet6 proto ipv6-icmp all icmp6-type routersol keep state
# client which uses autoconfiguration would use this instead
#pass in inet6 proto ipv6-icmp all icmp6-type routeradv keep state
#pass out inet6 proto ipv6-icmp all icmp6-type neighbrsol keep state
table <sixxspop> { 82.96.56.14 2001:16d8:ff00:155::1 }
pass in on $ext_if proto icmp from <sixxspop6> to ($ext_if)
pass in on $tunnel_if proto icmp6 from <sixxspop6> to any
pass in on $int_if all
pass out on $int_if all keep state
...  probably not working AS IS
\end{alltt}


\slide{Redundante firewalls}

\hlkimage{8cm}{images/pfsync-carp-1.jpg}

\begin{list2}
\item OpenBSD Common Address Redundancy Protocol CARP - både IPv4 og IPv6\\
overtagelse af adresse både IPv4 og IPv6
\item pfsync - sender opdateringer om firewall states mellem de to systemer  
\item Kilde:
\link{http://www.countersiege.com/doc/pfsync-carp/}
\end{list2}

\slide{Redundante forbindelser hardware}

\begin{alltt}
root@azumi:# cat hostname.fxp0
up
root@azumi:# cat hostname.fxp1 
up
root@azumi:# cat /etc/hostname.trunk0
trunkproto failover trunkport fxp0 trunkport fxp1
dhcp
\end{alltt}

\begin{list1}
\item OpenBSD trunk interface
\item Linux bonding, 
\item Etherchannel Cisco
\item Idag anbefales IEEE 802.3ad LACP som er en åben standard
\item \link{http://en.wikipedia.org/wiki/EtherChannel}
\end{list1}

\slide{LACP Link Aggregation Control Protocol}

\hlkimage{7cm}{lacp-1.pdf}

\begin{list1}
\item IEEE 802.3ad standardiseret bundling/failover
\item Målet er at give:
\begin{list2}
\item mere båndbredde end en enkelt port
\item failover - hvis et link falder ud
\end{list2}
\item En server med to netinterfaces kan med fordel forbindes til to porte
\item Er ikke generelt understøttet i alle operativsystemer, men det kommer
\item \link{http://en.wikipedia.org/wiki/Link_Aggregation_Control_Protocol}
\end{list1}


\slide{Redundante forbindelser IP-niveau}

\hlkimage{12cm}{router-redundancy-1.pdf}

\begin{list1}
\item HSRP Hot Standby Router Protocol, Cisco protokol, RFC-2281
\item VRRP Virtual Router Redundancy Protocol, IETF RFC-3768, åben standard - ikke fri
\item CARP Common Address Redundancy Protocol, findes på OpenBSD og FreeBSD
\item \link{http://en.wikipedia.org/wiki/Common_Address_Redundancy_Protocol}
\end{list1}








\slide{Mobile IP}

\begin{list1}
\item Mobility er ved at blive et krav, idet enheder idag er mobile
\item Specielt ønsker vi at håndholdte computere og laptops kan modtage data
\item Tidligere skiftede man blot adresse undervejs
\item Idag ønsker man at enheden kan kontaktes nemmere, selv udenfor \emph{huset}
\item RFC-3344 IP Mobility Support for IPv4
\item RFC-4721 Mobile IPv4 Challenge/Response Extensions (Revised)
\item RFC-3775
\item \link{http://en.wikipedia.org/wiki/Mobile_IP}
\item Bemærk at Mobile IP ikke altid er nødvendig eller benyttes, mange protokoller som eksempelvis POP3/IMAP virker fint ved at enheden kalder tilbage til serveren
\end{list1}

\slide{Mobile IP begreber}

\begin{list1}
\item Definitioner - fra RFC-3344:
\begin{list2}
\item Mobile Node A host or router that changes its point of attachment from one
         network or subnetwork to another. 
\item Home Agent A router on a mobile node's home network which tunnels
         datagrams for delivery to the mobile node when it is away from
         home, and maintains current location information for the mobile
         node.
\item Foreign Agent A router on a mobile node's visited network which provides
         routing services to the mobile node while registered.  The
         foreign agent detunnels and delivers datagrams to the mobile
         node that were tunneled by the mobile node's home agent.  For
         datagrams sent by a mobile node, the foreign agent may serve as
         a default router for registered mobile nodes.
\end{list2}
\item Selve funktionen:\\
   A mobile node is given a long-term IP address on a home network.
   This home address is administered in the same way as a "permanent" IP
   address is provided to a stationary host.  When away from its home
   network, a "care-of address" is associated with the mobile node and
   reflects the mobile node's current point of attachment. 
\end{list1}

\slide{Oversigt Mobile IP}

\hlkimage{23cm}{mobile-ip-1.pdf}

Se også Mobile IPv6 A short introduction \link{http://www.hznet.de/ipv6/mipv6-intro.pdf}


\slide{VoIP Voice over IP}

\begin{list1}
\item Tidligere havde vi adskilte netværk, nu samles de 
\item Idag er det meget normalt at både firmaer og private bruger IP-telefoni
\item Fordele er primært billigere og mere fleksibelt
\item Eksempler på IP telefoni:
\begin{list2}
\item Skype benytter IP, men egenudviklet protokol
\item Cisco IP-telefoner benyttes ofte i firmaer
\item Cybercity telefoni kører over IP, med analog adapter
\end{list2}
\item Det anbefales at se på Asterisk telefoniserver, hvis man har mod på det :-)
\item \link{http://www.asterisk.org/}
\end{list1}

\slide{VoIP bekymringer}

\begin{list1}
\item Der er generelt problemer med:
\begin{list2}
\item Stabilitet - quality of service, netværket skal være bygget til det
\item Sikkerhed - hvem lytter med, hvem kan afbryde forbindelsen\\
Se evt. \link{http://www.voipsa.org/}
\item Spam over VoIP, connect, send WAV fil med spam kaldes SPIT
\item Kompatabilitet - hvilke protokoller, codecs, standarder, ...
\end{list2}
\item Der er flere store spillere
\end{list1}

\slide{VoIP protokoller}

\begin{list1}
\item SIP Session Initiation Protocol, IETF standard signaleringsprotokol
\item H.323 ITU-T standard signaleringsprotokol
\item IAX Inter-Asterisk Exchange Protocol, Asterisk protokol
\item SSCP Cisco protokol
\item ZRTP Phil Zimmermann, zfone - sikker kommunikation\\ 
\link{http://zfoneproject.com/}
\end{list1}

\slide{Dag 5 Diverse}

\hlkimage{20cm}{openbgpd-network-1.pdf}




\slide{Opsamling}

\begin{list1}
\item Dagen idag er primært beregnet til opsamling
\item Detaljer som ikke har været gennemgået undervejs, fordi jeg mente det var bedre at skærme imod i den første gennemgang

\end{list1}

\slide{Internet-relaterede organisationer}

\hlkimage{20cm}{IAB_structure.pdf}

\centerline{Oftest er man interesseret i \link{http://www.ietf.org/}}

\slide{Proxy-arp}

\begin{list1}
\item Routere understøtter ofte Proxy ARP
\item Med Proxy ARP svarer de for en adresse bagved routeren
\item Derved kan man få trafik nemt igennem fra internet til adresser
\item Det er smart i visse situationer hvor en subnetting vil spilde for mange adresser
\item Hvis man kun har få adresser er subnetting måske heller ikke muligt
\item \link{http://en.wikipedia.org/wiki/Proxy_ARP}
\end{list1}

\slide{Reverse ARP}

\begin{list1}
\item Tidligere brugte man en protokol kaldet Reverse ARP til at uddele IP-adresser
\item Med Reverse ARP sender en enhed et request og får et Reverse ARP svar tilbage
\item \emph{Jeg har denne MAC adresse, hvad er min IP?}
\item \emph{Hvis du er denne MAC adresse er din IP 10.2.3.1}
\item Det benyttes meget sjældent idag, men var tidligere brugt til netboot af arbejdsstationer m.v.
\end{list1}

\slide{ICMP redirect}

\begin{list1}
\item Routere understøtter ofte ICMP Redirect
\item Med ICMP Redirect kan man til en afsender fortælle en anden vej til destination
\item Den angivne vej kan være smartere eller mere effektiv
\item Det er desværre uheldigt, idet der ingen sikkerhed er
\item Idag bør man ikke lytte til ICMP redirects, ej heller generere dem
\item Det svarer til ARP spoofing, idet trafik omdirigeres
\end{list1}


\slide{Hvordan virker ARP spoofing?}

\begin{center}
\colorbox{white}{\includegraphics[width=15cm]{images/arp-spoof.pdf}}  
\end{center}

\begin{list1}
\item Hackeren sender forfalskede ARP pakker til de to parter
\item De sender derefter pakkerne ud på Ethernet med hackerens MAC
  adresse som modtager - han får alle pakkerne
\end{list1}

\slide{Forsvar mod ARP spoofing}

\begin{list1}
\item Hvad kan man gøre? 
\item låse MAC adresser til porte på switche
\item låse MAC adresser til bestemte IP adresser
\item Efterfølgende administration!
\vskip 1 cm
\item {\bfseries arpwatch er et godt bud} - overvåger ARP
\item bruge protokoller som ikke er sårbare overfor opsamling
\end{list1}


\slide{IGMP Internet Group Management Protocol}

\begin{list1}
\item Der er defineret Multicast protokoller på internet
\item Med multicast kan man sende data til en nærmere angivet gruppe
\item Multicast er tiltænkt ting som radio og video broadcast
\item IPv6 benytter en del multicast adresser, all-nodes, all-routes, ...
\item Hvem der modtager data styres så ved hjælp af IGMP
\item IGMP bruges således til at styre hvem der på et givet tidspunkt er med i IP multicast grupper
\item RFC-3376 Internet Group Management Protocol, Version 3
\item \link{http://en.wikipedia.org/wiki/Internet_Group_Management_Protocol}
\end{list1}


\slide{TCP sequence number prediction}

\begin{list1}
  \item tidligere baserede man ofte login og adgange på de IP adresser
  som folk kom fra
\item det er ikke pålideligt at tro på address based authentication
\item TCP sequence number kan måske gættes
\item Mest kendt er nok Shimomura der blev hacket på den måde, måske
  af Kevin D Mitnick eller en kompagnon
\item I praksis vil det være svært at udføre på moderne operativsystemer
\item Se evt. \link{http://www.takedown.com/}
\item (filmen er ikke så god ;-) ) 
\end{list1}


\slide{Hardware IPv4 checksum offloading}

\begin{list1}
\item IPv4 checksum skal beregnes hvergang man modtager en pakke
\item IPv4 checksum skal beregnes hvergang man sender en pakke
\vskip 1cm
\item Lad en ASIC gøre arbejdet!
\item De fleste servernetkort tilbyder at foretage denne beregning på IPv4
\item IPv6 benytter ikke header checksum, det er unødvendigt
\end{list1}
\vskip 1cm

\centerline{\hlkbig NB: kan resultere i at tcpdump siger checksum er forkert!}


\slide{At være på internet}

\begin{list1}
\item RFC-2142 Mailbox Names for Common Services, Roles and Functions
\item Du BØR konfigurere dit domæne til at modtage post for følgende adresser:
\begin{list2}
\item postmaster@domæne.dk
\item abuse@domæne.dk
\item webmaster@domæne.dk, evt. www@domæne.dk
\end{list2}
\item Du gør det nemmere at rapportere problemer med dit netværk og services
\end{list1}

\slide{E-mail best current practice}

\begin{alltt}
MAILBOX       AREA                USAGE
-----------   ----------------    ---------------------------
ABUSE         Customer Relations  Inappropriate public behaviour
NOC           Network Operations  Network infrastructure
SECURITY      Network Security    Security bulletins or queries  
...
MAILBOX       SERVICE             SPECIFICATIONS
-----------   ----------------    ---------------------------
POSTMASTER    SMTP                [RFC821], [RFC822]
HOSTMASTER    DNS                 [RFC1033-RFC1035]
USENET        NNTP                [RFC977]
NEWS          NNTP                Synonym for USENET
WEBMASTER     HTTP                [RFC 2068]
WWW           HTTP                Synonym for WEBMASTER
UUCP          UUCP                [RFC976]
FTP           FTP                 [RFC959]
\end{alltt}

Kilde: 
RFC-2142 Mailbox Names for Common Services, Roles and Functions. D.
Crocker. May 1997

\slide{Brug krypterede forbindelser}

\hlkimage{18cm}{images/dsniff-comments.pdf}

\begin{list1}
\item Især på utroværdige netværk kan det give problemer at benytte
  sårbare protokoller   
\end{list1}

\slide{Mission 1: Kommunikere sikkert}

\begin{list1}
\item Du må ikke bruge ukrypterede forbindelser til at administrere
  UNIX
\item Du må ikke sende kodeord i ukrypterede e-mail beskeder  
\end{list1}

\centerline{\hlkbig Telnet daemonen - telnetd må og skal dø!}

\pause
\centerline{\hlkbig FTP daemonen - ftpd må og skal dø!}

\pause
\centerline{\hlkbig POP3 daemonen port 110 må og skal dø!}

\pause
\centerline{\hlkbig IMAPD daemonen port 143 må og skal dø!}

\pause
\vskip 1cm 
\centerline{\hlkbig\bf væk med alle de ukrypterede forbindelser!}


\slide{Infrastrukturer i praksis}

\begin{list1}
\item Vi vil nu gennemgå netværksdesign med udgangspunkt i vores setup
\item Vores setup indeholder:
\begin{list2}
\item Routere
\item Firewall
\item Wireless
\item DMZ
\item DHCPD, BIND, BGPD, OSPFD, ...
\end{list2}
\item Den kunne udvides med flere andre teknologier vi har til rådighed:
\begin{list2}
\item VLAN inkl VLAN trunking/distribution
\item WPA Enterprise
\end{list2}
\item Hvad taler for og imod - de næste slides gennemgår nogle standardsetups
\item En slags Patterns for networking
\end{list1}





\slide{Netværksdesign og sikkerhed}

\begin{list1}
\item Hvad kan man gøre for at få bedre netværkssikkerhed?
\begin{list2}
\item Bruge switche - der skal ARP spoofes og bedre performance
\item Opdele med firewall til flere DMZ zoner for at holde
      udsatte servere adskilt fra hinanden, det interne netværk og
      Internet
\item Overvåge, læse logs og reagere på hændelser 
\end{list2}
\item Husk du skal også kunne opdatere dine servere
\end{list1}

\slide{basalt netværk}

\hlkimage{16cm}{images/demo-netvaerk.pdf}

\begin{list1}
\item Du bør opdele dit netværk i segmenter efter traffik
\item Du bør altid holde interne og eksterne systemer adskilt!
\item Du bør isolere farlige services i jails og chroots
\end{list1}



\slide{Intrusion Detection Systems - IDS}

\begin{list1}
  \item angrebsværktøjerne efterlader spor

\item hostbased IDS - kører lokalt på et system og forsøger at
  detektere om der er en angriber inde
\item network based IDS - NIDS - bruger netværket
\item Automatiserer netværksovervågning:
  \begin{list2}
  \item bestemte pakker kan opfattes som en signatur
\item analyse af netværkstrafik - FØR angreb
\item analyse af netværk under angreb - sender en alarm
  \end{list2}
\item \link{http://www.snort.org} - det kan anbefales at se på Snort
\end{list1}

\slide{snort}

\hlkimage{5cm}{images/snort_tm.png}

\begin{list1}
\item Snort er Open Source og derfor godt til undervisning
\item man kan se det som et antivirus system til netværket
\item forsøger at detektere \emph{angreb}, \emph{skadelig} og
  \emph{forkert} traffik
\item pakker der minder om eksempelvis:
  \begin{list2}
    \item nmap portscan
\item nmap OS detection - med underlige pakker
\item fragmenter der overlapper
\item shellcode der sendes til systemer som BIND
  \end{list2}
\end{list1}

\slide{Snort regler}

\begin{alltt}\small
alert icmp $HOME_NET any -> $EXTERNAL_NET any (msg:"ICMP Address Mask
Reply"; icode:0; itype:18; classtype:misc-activity; sid:386; rev:5;)
alert icmp $EXTERNAL_NET any -> $HOME_NET any (msg:"ICMP Address Mask 
Reply undefined code"; icode:>0; itype:18; classtype:misc-activity; 
sid:387; rev:7;)
alert icmp $EXTERNAL_NET any -> $HOME_NET any (msg:"ICMP Address Mask 
Request"; icode:0; itype:17; classtype:misc-activity; sid:388; rev:5;)
alert icmp $EXTERNAL_NET any -> $HOME_NET any (msg:"ICMP Address Mask 
Request undefined code"; icode:>0; itype:17; classtype:misc-activity; 
sid:389; rev:7;)
alert icmp $EXTERNAL_NET any -> $HOME_NET any (msg:"ICMP Alternate 
Host Address"; icode:0; itype:6; classtype:misc-activity; sid:390; rev:5;)
\end{alltt}

\begin{list2}
\item sid - snort rules id - identificerer en signatur  
\item reference - hvor kommer reglen fra
\item icode - ICMP code
\item itype - ICMP type
\item ... se mere i snort manualen
\end{list2}

\slide{Ulemper ved IDS}

\hlkimage{5cm}{images/snort_tm.png}

\begin{list1}
\item snort er baseret på signaturer
\item mange falske alarmer - tuning og vedligehold
\item hvordan sikrer man sig at man har opdaterede signaturer for
  angreb som går verden rundt på et døgn 
\end{list1}

\slide{ Planlægning af IDS miljøer}

\begin{list1}
\item Før installationen
\begin{list2}
\item Hvad er formålet - reaktion eller "statistik"
\item Hvor skal der måles - hele netværket eller specifikke dele
\item Hvad skal måles og hvilke operativsystemer og servere/services
\end{list2}
\item Implementationen
\begin{list2}
\item Er infrastrukturen iorden som den er
\item Er der gode målepunkter - monitorporte
\item Et målepunkt eller flere
\item Hvormeget trafik skal måles
\end{list2}
\item Selve idriftsættelsen
\begin{list2}
\item Ændringer af infrastrukturen
\item Installation af udstyret
\item Test af udstyret udenfor drift
\item Installation i driftsmiljøet
\item Test af udstyret i driftsmiljøet
\end{list2}
\end{list1}

\slide{ Opsætning og konfiguration af IDS miljøer}

\begin{list1}
\item Vælg en simpel installation til at starte med!
\item Undgå for alt i verden for meget information
\begin{list2}
\item Start med en enkelt sensor
\item Byg en server med database og "brugerværktøjer"
\item Start med at overvåge dele af nettet
\item Brug et specifikt regelsæt i starten - eksempelvis kun Windows eller kun UNIX
\item Lav nogle simple rapporter til at starte med
\end{list2}
\item Gør netværket mere sikkert før du lytter på hele netværket
\item Brug tcpdump/Ethereal til at se på trafik, lær IP pakker at
  kende 
\item Brug Snort til at evaluere
\begin{list2}
\item husk at man kan starte med Snort og senere skifte til andre
produkter
\item Erfaring tæller, Snort tillader at man ser de fine detaljer - motoren
\end{list2}
\end{list1}

\slide{ Vedligehold og overvågning af IDS miljøer}

\begin{list1}
\item Uden vedligehold er IDS værdiløst - lad hellere være!
\begin{list2}
\item Vedligehold af software på operativsystemet
\item Vedligehold af IDS softwaren
\item Vedligehold af regelsæt
\end{list2}
\item Overvågning - kører IDS systemet, databaser og sensorer
\item Statistik og brug af IDS systemet
\begin{list2}
\item Vedligehold af rapporter - hvad er vi interesseret i
\item Automatisk rapportgenerering - daglig rapport, rapport pr måned
\item Specielle hændelser - hvad skete der onsdag mellem 11-12
\end{list2}
\item Et IDS kan også blot være en ARPwatch
\item ARPwatch advarer hvis nogen tager adressen fra default gateway
\end{list1}


\slide{Honeypots}

\begin{list1}
\item Man kan udover IDS installere en honeypot
\item En honeypot består typisk af:
  \begin{list2}
    \item Et eller flere sårbare systemer
\item Et eller flere systemer der logger traffik til og fra honeypot
  systemerne 
  \end{list2}
\item Meningen med en honeypot er at den bliver angrebet og brudt ind
  i 
\end{list1}

%\slide{Prelude}

%Måske Prelude i kombination med Nagios, Cricket, MRTG, RRDTool, Smokeping, ARPwatch


%\slide{Oversigt over forsvar mod sårbarheder}

\begin{list1}
\item Hvad muligheder har man
  \begin{list2}
  \item Ændre miljø
  \item forbedre systemerne
  \item undgå standardindstillinger
  \item vær opdateret på sikkerhedsområdet
  \item have retningslinier - ens sikkerhedsniveau
  \item drop kompatibilitet med usikre systemer
  \item en god infrastruktur
  \item brug kryptografi
  \item brug standardbiblioteker
  \item test af systemer
  \end{list2}
\end{list1}

\slide{Ændre miljø}

\begin{list1}
\item Ændre arkitektur sw/hw/netværkstopologi
  \begin{list2}
  \item blokere porte således at en webserver IKKE kan connecte tilbage til hackeren!
  \item blokere de services der IKKE skal tilgås udefra
  \item skifte programmeringssprog
  \end{list2}
\item Husk altid at hackeren også kan gå ind ad hovedøren
\item eksempelvis SAP Internet gateway, hvor man kunne lægge det
  bagvedliggende system ned med loginrequests
\end{list1}
\slide{Forbedre systemerne}

\begin{list1}
\item Operativsystemet
  \begin{list2}
  \item non-executable stack
  \item non-executable heap
  \end{list2}
\item Applikationsservere
  \begin{list2}
  \item filtrering af "dårlige" requests e-Eye sikret IIS
  \item mere "sikker" default opsætning
  \end{list2}
\item Jeg tror vi vil se flere implementere den slags løsninger
\item Eksempelvis:
\begin{list2}
\item Microsoft IIS web server version 6 er mere sikker i default opsætningen  
\item Apache HTTPD web server version 2 er mere modulær og nemmere at bygge sikkert  
\end{list2}
\end{list1}

\slide{Undgå standard indstillinger}

\begin{list1}
\item Giv jer selv mere tid til at patche og opdatere
\item Tiden der går fra en sårbarhed annonceres på bugtraq til den bliver
       udnyttet er meget kort idag!
\item Ved at undgå standard indstillinger kan der
       måske opnås en lidt længere frist - inden ormene kommer
\item NB: ingen garanti
\end{list1}



\slide{Pattern: erstat Telnet med SSH}

\begin{list1}
\item Telnet er død!
\item Brug altid Secure Shell fremfor Telnet
\item Opgrader firmware til en der kan SSH, eller køb bedre udstyr næste gang
\item Selv mine små billige Linksys switche forstår SSH!
\end{list1}

\slide{Pattern: erstat FTP med HTTP}

\begin{list1}
\item Hvis der kun skal distribueres filer kan man ofte benytte HTTP istedet for FTP
\item Hvis der skal overføres med password er SCP/SFTP fra Secure Shell at foretrække
\end{list1}


\slide{Anti-patterns}

\begin{list1}
\item Nu præsenteres et antal setups, som ikke anbefales
\item Faktisk vil jeg advare mod at bruge dem
\item Husk følgende slides er min mening
\end{list1}

\slide{Anti-pattern dobbelt NAT i eget netværk}

\hlkimage{20cm}{nat-double.pdf}

\begin{list1}
\item Det er nødvendigt med NAT for at oversætte traffik der sendes videre
ud på internet.
\vskip 1cm
\item Der er ingen som helst grund til at benytte NAT indenfor eget netværk!
\end{list1}

\slide{Anti-pattern blokering af ALT ICMP}

\begin{alltt}
ipfw add allow icmp from any to any icmptypes 3,4,11,12
\end{alltt}

\begin{list1}
\item Lad være med at blokere for alt ICMP, så ødelægger du funktionaliteten i dit net 
\vskip 1cm
\item \end{list1}

\slide{Anti-pattern blokering af DNS opslag på TCP}

\begin{list1}
\item Det bliver (er) nødvendigt med DNS opslag over TCP på grund af store svar. Det betyder at firewalls skal tillade DNS opslag via TCP
\vskip 1cm
\item 
\item Guide:\\
Brug en caching nameserver, således at det kun er den som kan lave DNS opslag ud i verden

\end{list1}

\slide{Anti-pattern daisy-chain}

\hlkimage{20cm}{daisy-chain-server.pdf}

\begin{list1}
\item Daisy-chain af servere, erstat med firewall, switch og VLAN
\vskip 1cm
\item Det giver et væld af problemer med overvågning, administration, backup og opdatering
\end{list1}

\slide{Anti-pattern WLAN forbundet direkte til LAN}

\hlkimage{10cm}{images/wlan-accesspoint-2.pdf}

\begin{list1}
\item WLAN AP'er forbundet direkte til LAN giver risiko for at sikkerheden brydes, fordi AP falder tilbage på den usikre standardkonfiguration
\vskip 1cm
\item Ved at sætte WLAN direkte på LAN risikerer man at eksterne får direkte adgang
\item Kan selvfølgelig gå an i et privat hjem
\item Det forværres jo flere AP'er man har, har du 100 skal du være sikker på allesammen er sikre!
\end{list1}




\slide{Hackerværktøjer}

\begin{list1}
\item Dan Farmer og Wietse Venema skrev i 1993 artiklen\\
\emph{Improving the Security of Your Site by Breaking Into it}
\item Senere i 1995 udgav de så en softwarepakke med navnet SATAN
\emph{Security Administrator Tool for Analyzing Networks}
 Pakken vagte
 en del furore, idet man jo gav alle på internet mulighed for at hacke
\begin{quote}
We realize that SATAN is a two-edged sword - like
many tools, it can be used for good and for evil
purposes. We also realize that intruders (including
wannabees) have much more capable (read intrusive)
tools than offered with SATAN. 
\end{quote}
\item SATAN og ideerne med automatiseret scanning efter sårbarheder
  blev siden ført videre i programmer som Saint, SARA og idag findes
  mange hackerværktøjer og automatiserede scannere: 
\begin{list2}
\item Nessus, ISS scanner, Fyodor Nmap, Typhoon, ORAscan
\end{list2}
\end{list1}
Kilde:
\link{http://www.fish.com/security/admin-guide-to-cracking.html}



\slide{Brug hackerværktøjer!}

\begin{list1}
\item Hackerværktøjer - bruger I dem? - efter dette kursus gør I 
\item portscannere kan afsløre huller i forsvaret
\item webtestværktøjer som crawler igennem et website og finder alle
  forms kan hjælpe
\item I vil kunne finde mange potentielle problemer proaktivt ved
  regelmæssig brug af disse værktøjer - også potentielle driftsproblemer
\item husk dog penetrationstest er ikke en sølvkugle
\item honeypots kan måske være med til at afsløre angreb og
  kompromitterede systemer hurtigere
\end{list1}


\slide{"I only replaced index.html"}

\begin{list1}
\item Hvad skal man gøre når man bliver hacket ?
\item Hvad koster et indbrud?
\begin{list2}
\item Tid - antal personer der ikke kan arbejde
\item Penge - oprydning, eksterne konsulenter
\item Bøvl - sker altid på det værst mulige tidspunkt
\item Besvær - ALT skal gennemrodes
\item Tab af image/goodwill
\end{list2}
\item Forensic challenge:
I gennemsnit brugte deltagerne 34 timer pr person på
at efterforske i rigtige data fra et indbrud!
angriberen brugte ca. 30 min
\item Kilder:
\link{http://project.honeynet.org/challenge/results/}\\
\link{http://packetstorm.securify.com/docs/hack/i.only.replaced.index.html.txt}
\end{list1}

\slide{Recovering from break-ins}

\begin{list1}
\item {\color{red}\bfseries DU KAN IKKE HAVE TILLID TIL NOGET}
\item På CERT website kan man finde mange gode ressourcer omkring
  sikkerhed og hvad man skal gøre med kompromiterede servere
\item Eksempelvis listen over dokumenter fra adressen:\\
  \link{http://www.cert.org/nav/recovering.html} 
  \begin{list2}
  \item The Intruder Detection Checklist
  \item Windows NT Intruder Detection Checklist 
  \item The UNIX Configuration Guidelines
  \item Windows NT Configuration Guidelines 
  \item The List of Security Tools
  \item Windows NT Security and Configuration Resources 
  \end{list2}
%\item Hvis man mener man står med en kompromitteret server kan
%  følgende være nødvendigt
%  \href{http://www.cert.org/tech_tips/root_compromise.html} 
%{http://www.cert.org/tech\_tips/root\_compromise.html}
\end{list1}





\slide{Opsummering}

\vskip 3 cm

\begin{list1}
\item Husk følgende:
\begin{list2}
\item UNIX og Linux er blot eksempler - navneservice eller HTTP
  server kører fint på Windows
\item DNS er grundlaget for Internet
\item Sikkerheden på internet er generelt dårlig, brug SSL!
\item Procedurerne og vedligeholdelse er essentiel for alle
  operativsystemer!
\item Man skal \emph{hærde} operativsystemer \emph{før} man sætter dem på
  Internet
\item Husk: IT-sikkerhed er ikke kun netværkssikkerhed!
\item God sikkerhed kommer fra langsigtede intiativer\\
\end{list2}
\item Jeg håber I har lært en masse om netværk og kan bruge det i praksis :-)
\end{list1}

\slide{Spørgsmål?}


\vskip 4cm

\begin{center}
\hlkbig

\myname

\myweb
\vskip 2 cm

I er altid velkomne til at sende spørgsmål på e-mail
\end{center}



\slide{Referencer: netværksbøger}

\begin{list2}
\item Stevens, Comer,
\item Network Warrior
\item TCP/IP bogen på dansk
\item KAME bøgerne
\item O'Reilly generelt IPv6 Essentials og IPv6 Network Administration
\item O'Reilly cookbooks: Cisco, BIND og Apache HTTPD
\item Cisco Press og website
\item Firewall bøger, Radia Perlman: IPsec,
\end{list2}

\slide{Bøger om IPv6}

\begin{list1}
\item \emph{IPv6 Network Administration}
af David Malone og Niall Richard Murphy
 - god til real-life admins, typisk
O'Reilly bog
\item \emph{IPv6 Essentials} af Silvia Hagen, O'Reilly 2nd edition (May 17, 2006)
	god reference om emnet
\item \emph{IPv6 Core Protocols Implementation}
af Qing Li, Tatuya Jinmei og Keiichi Shima
\item \emph{IPv6 Advanced Protocols Implementation}
af Qing Li, Jinmei Tatuya og Keiichi Shima
\item - flere andre
\end{list1}


\input{references.tex}



\end{document}
