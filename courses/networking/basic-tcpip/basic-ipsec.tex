\slide{IPsec}

\begin{itemize}
\item Sikkerhed i netv�rket
\item RFC-2401 Security Architecture for the Internet Protocol
\item RFC-2402 IP Authentication Header (AH)
\item RFC-2406 IP Encapsulating Security Payload (ESP)
\item RFC-2409 The Internet Key Exchange (IKE) - dynamisk keying
\item B�de til IPv4 og IPv6
\item {\bfseries MANDATORY} i IPv6! - et krav hvis man implementerer
  fuld IPv6 support
\item god pr�sentation p� \link{http://www.hsc.fr/presentations/ike/}
\item Der findes IKEscan til at scanne efter IKE
  porte/implementationer\\
\link{http://www.nta-monitor.com/ike-scan/index.htm}
\end{itemize}

\slide{IPsec er ikke simpelt!}

\hlkimage{16cm}{images/ipsec-hsc.png}
\centerline{Kilde: \link{http://www.hsc.fr/presentations/ike/}}


\slide{RFC-2402 IP AH}

\begin{alltt}
\small
    0                   1                   2                   3
    0 1 2 3 4 5 6 7 8 9 0 1 2 3 4 5 6 7 8 9 0 1 2 3 4 5 6 7 8 9 0 1
   +-+-+-+-+-+-+-+-+-+-+-+-+-+-+-+-+-+-+-+-+-+-+-+-+-+-+-+-+-+-+-+-+
   | Next Header   |  Payload Len  |          RESERVED             |
   +-+-+-+-+-+-+-+-+-+-+-+-+-+-+-+-+-+-+-+-+-+-+-+-+-+-+-+-+-+-+-+-+
   |                 Security Parameters Index (SPI)               |
   +-+-+-+-+-+-+-+-+-+-+-+-+-+-+-+-+-+-+-+-+-+-+-+-+-+-+-+-+-+-+-+-+
   |                    Sequence Number Field                      |
   +-+-+-+-+-+-+-+-+-+-+-+-+-+-+-+-+-+-+-+-+-+-+-+-+-+-+-+-+-+-+-+-+
   |                                                               |
   +                Authentication Data (variable)                 |
   |                                                               |
   +-+-+-+-+-+-+-+-+-+-+-+-+-+-+-+-+-+-+-+-+-+-+-+-+-+-+-+-+-+-+-+-+
\end{alltt}

\slide{RFC-2402 IP AH}

Indpakning - pakkerne f�r og efter Authentication Header:
\begin{alltt}
\small
                BEFORE APPLYING AH
            ----------------------------
      IPv4  |orig IP hdr  |     |      |
            |(any options)| TCP | Data |
            ----------------------------

                  AFTER APPLYING AH
            ---------------------------------
      IPv4  |orig IP hdr  |    |     |      |
            |(any options)| AH | TCP | Data |
            ---------------------------------
            |<------- authenticated ------->|
                 except for mutable fields
\end{alltt}

\slide{RFC-2406 IP ESP}

Pakkerne f�r og efter:
\begin{alltt}
\small
               BEFORE APPLYING ESP
         ---------------------------------------
   IPv6  |             | ext hdrs |     |      |
         | orig IP hdr |if present| TCP | Data |
         ---------------------------------------



               AFTER APPLYING ESP
         ---------------------------------------------------------
   IPv6  | orig |hop-by-hop,dest*,|   |dest|   |    | ESP   | ESP|
         |IP hdr|routing,fragment.|ESP|opt*|TCP|Data|Trailer|Auth|
         ---------------------------------------------------------
                                   |<---- encrypted ---->|
                               |<---- authenticated ---->|
\end{alltt}

\slide{ipsec konfigurationsfiler}

\begin{list1}
%\item Der er f�lgende dokumenter til IPsec p� websitet\\
% \link{www.security.net/courses/ipsec}:
\item Der er f�lgende filer tilg�ngelige\\
  \begin{list2}
  \item konfigurationsfiler i NetBSD/FreeBSD/Mac OS X format - med
    \verb+setkey+ kommandoen
  \item konfigurationsfil til OpenBSD server - med \verb+ipsecadm+
    kommandoen
%  \item IKE.pdf \emph{Dynamic Management of the IPsec Parameters:
%      The IKE Protocol}, fra Herve Schauer Consultants
%\item NetBSD IPsec dokumentation
%\item Cisco \emph{Introduction to IP security}
  \end{list2}
\end{list1}


\slide{IPsec setup}

%\hlkimage{}{images/}

\begin{list1}
  \item Client: Mac OS X/NetBSD/FreeBSD - samme syntaks\\
\verb+rc.ipsec.client+

\item Server: OpenBSD - bruger ipsecadm kommando\\
\verb+rc.ipsec.server+

\item �velse til l�seren: lav samme i Cisco IOS
\item Det vil ofte v�re relevant at se p� IOS og IPsec i laboratoriet
\item Dette setup n�r vi ikke at demonstrere
\end{list1}

\slide{rc.ipsec.client - client setup - adresser}

\begin{verbatim}
#!/bin/sh
# /etc/rc.ipsec.client - IPsec client configuration
# built from http://rt.fm/~jcs/ipsec_wep.phtml
# FreeBSD/NetBSD syntaks! - used on Mac OS X
# IPv4
SECSERVER=10.0.42.1
SECCLIENT=10.0.42.53
# IPv6
#SECSERVER=2001:618:433:101::1
#SECCLIENT=2001:618:433:101::153
ESPKEY=`cat ipsec.esp.key`
AHKEY=`cat ipsec.ah.key`

# Flush IPsec SAs in case we get called more than once
setkey -F
setkey -F -P
\end{verbatim}

\slide{rc.ipsec.client - client setup - SAs}

\begin{verbatim}
# Establish Security Associations
# 1000 is from the server to the client
# 1001 is from the client to the server
setkey -c <<EOF

add $SECSERVER $SECCLIENT esp 0x1000 \
-m tunnel -E blowfish-cbc 0x$ESPKEY  -A hmac-sha1 0x$AHKEY;

add $SECCLIENT $SECSERVER esp 0x1001 \
-m tunnel -E blowfish-cbc 0x$ESPKEY -A hmac-sha1 0x$AHKEY;

spdadd $SECCLIENT $SECSERVER any -P out \
ipsec esp/tunnel/$SECCLIENT-$SECSERVER/default;

spdadd $SECSERVER $SECCLIENT any -P in \
ipsec esp/tunnel/$SECSERVER-$SECCLIENT/default;
EOF
\end{verbatim}

\slide{rc.ipsec.server - server setup - adresser}

\begin{verbatim}
#!/bin/sh
#
# /etc/rc.ipsec - IPsec server configuration
# built from http://rt.fm/~jcs/ipsec_wep.phtml
# OpenBSD syntaks!
SECSERVER=10.0.42.1
SECCLIENT=10.0.42.53
#SECSERVER6=2001:618:433:101::1
#SECCLIENT6=2001:618:433:101::153

ESPKEY=`cat ipsec.esp.key`
AHKEY=`cat ipsec.ah.key`

# Flush IPsec SAs in case we get called more than once
ipsecadm flush
\end{verbatim}


\slide{rc.ipsec.server - server setup - SAs}

\begin{verbatim}
# Establish Security Associations
#
# 1000 is from the server to the client
ipsecadm new esp -spi 1000 -src $SECSERVER -dst $SECCLIENT \
-forcetunnel -enc blf -key $ESPKEY \
-auth sha1 -authkey $AHKEY

# 1001 is from the client to the server
ipsecadm new esp -spi 1001 -src $SECCLIENT -dst $SECSERVER \
-forcetunnel -enc blf -key $ESPKEY \
-auth sha1 -authkey $AHKEY
\end{verbatim}


\slide{rc.ipsec.server - server setup - flows}

\small
\begin{verbatim}
# Create flows
#
# Data going from the outside to the client
ipsecadm flow -out -src $SECSERVER -dst $SECCLIENT -proto esp \
-addr 0.0.0.0 0.0.0.0 $SECCLIENT 255.255.255.255 -dontacq
# IPv6
#ipsecadm flow -out -src $SECSERVER -dst $SECCLIENT -proto esp \
#-addr :: :: $SECCLIENT ffff:ffff:ffff:ffff:ffff:ffff:ffff:ffff -dontacq

# Data going from the client to the outside
ipsecadm flow -in -src $SECSERVER -dst $SECCLIENT -proto esp \
-addr $SECCLIENT 255.255.255.255 0.0.0.0 0.0.0.0 -dontacq
# IPv6
#ipsecadm flow -in -src $SECSERVER -dst $SECCLIENT -proto esp \
#-addr :: :: $SECCLIENT ffff:ffff:ffff:ffff:ffff:ffff:ffff:ffff -dontacq
\end{verbatim}



%%% Local Variables:
%%% mode: latex
%%% TeX-master: t
%%% End:
