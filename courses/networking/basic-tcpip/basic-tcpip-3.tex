\slide{Dag 3 Dynamiske protokoller og services}

\hlkimage{20cm}{openbgpd-network-2.pdf}

% 802.11 start
%input fra wireless kursus, wireless præsentationer, wireless foredrag pentest III
%Husk WDS og bridging

\slide{Trådløse teknologier 802.11}

\begin{list1}
\item 802.11 er arbejdsgruppen under IEEE 
\item De mest kendte standarder idag indenfor trådløse teknologier:
\begin{list2}
\item 802.11b 11Mbps versionen
\item 802.11g 54Mbps versionen
\item 802.11n endnu hurtigere, og draft
\item 802.11i Security enhancements
\end{list2}
\item Der er proprietære versioner 22Mbps og den slags\\
- det anbefales IKKE at benytte disse da det giver vendor lock-in -
man bliver låst fast
\end{list1}

Kilde: \link{http://grouper.ieee.org/groups/802/11/index.html}

\slide{802.11 modes og frekvenser}

\begin{list1}
\item Access point kører typisk i \emph{access point mode} også kaldet
  infrastructure mode - al trafik går via AP
\item Alternativt kan wireless kort oprette ad-hoc netværk - hvor
  trafikken går direkte mellem netkort
\item Frekvenser op til kanal 11 og 12+13 i DK/EU
\item Helst 2 kanaler spring for 802.11b AP der placeres indenfor rækkevidde
\item Helst 4 kanaler spring for 802.11g AP der placeres indenfor rækkevidde
\end{list1}

\slide{Eksempel på netværk med flere AP'er}

\hlkimage{20cm}{images/wireless-multi-ap.pdf}

\slide{Eksempel på netværk med flere AP'er}

\hlkimage{20cm}{images/wireless-multi-ap-2.pdf}


\slide{Wireless Distribution System WDS}

\hlkimage{18cm}{images/wireless-multi-ap-wds.pdf}

\begin{list1}
\item Se også:
\link{http://en.wikipedia.org/wiki/Wireless_Distribution_System}
\end{list1}


\slide{\hskip 1 cm Er trådløse netværk interessante?}

\begin{list1}
\item Sikkerhedsproblemer i de trådløse netværk er mange
  \begin{list2}
  \item Fra lavt niveau - eksempelvis ARP, 802.11
  \item dårlige sikringsmekanismer - WEP
  \item dårligt udstyr - mange fejl
  \item usikkkerhed om implementering og overvågning
  \end{list2}
\item Trådløst udstyr er blevet meget billigt!
\item Det er et krav fra brugerne - trådløst er lækkert
\end{list1}


\slide{Konsekvenserne}

\hlkimage{10cm}{images/wireless-daekning.pdf}

\begin{list2}
\item Værre end Internetangreb - anonymt
\item Kræver ikke fysisk adgang til lokationer
%\emph{spioneres imod}
\item Konsekvenserne ved sikkerhedsbrud er generelt større
\item Typisk får man direkte LAN eller Internet adgang!
\end{list2}

\slide{Værktøjer}

\begin{list1}
\item Alle bruger nogenlunde de samme værktøjer, måske forskellige
  mærker
\begin{list2}
\item Wirelessscanner - Kismet og netstumbler
\item Wireless Injection - typisk på Linux
\item ...
\item Aircrack-ng
\end{list2}
\item Jeg anbefaler Auditor Security Collection og BackTrack boot CD'erne
\end{list1}


\slide{Konsulentens udstyr wireless}

\begin{list1}
\item Laptop med PC-CARD slot
\item Trådløse kort Atheros, de indbyggede er ofte ringe ;-)
\item Access Points - jeg anbefaler Airport Express
\item Antenner hvis man har lyst
\item Bøger: 
\begin{list2}
\item \emph{Real 802.11 security}
\item Se oversigter over bøger og værktøjer igennem præsentationen:
\end{list2}
\item Internetressourcer:
\begin{list2}
\item BackTrack - CD image med Linux+værktøjer    
\item Packetstorm wireless tools
\link{http://packetstormsecurity.org/wireless/}
\item \emph{Beginner's Guide to Wireless Auditing}
David Maynor 
\link{http://www.securityfocus.com/infocus/1877?ref=rss}
\end{list2}
\end{list1}


\slide{Typisk brug af 802.11 udstyr}

\begin{center}
\colorbox{white}{\includegraphics[width=20cm]{images/wlan-accesspoint-1.pdf}}
\end{center}

\centerline{\hlkbig et access point - forbindes til netværket}

\slide{Basal konfiguration}

\begin{list1}
\item Når man tager fat på udstyr til trådløse netværk opdager man:
\item SSID - nettet skal have et navn
\item frekvens / kanal - man skal vælge en kanal, eller udstyret
  vælger en automatisk
\item der er nogle forskellige metoder til sikkerhed  
\end{list1}

\slide{Trådløs sikkerhed}

\hlkimage{14cm}{images/apple-wireless-security.png}

\begin{list2}
\item Trådløs sikkerhed - WPA og WPA2
\item Nem konfiguration
\item Nem konfiguration af Access Point  
\end{list2}

\slide{Wireless networking sikkerhed i 802.11b}

\hlkimage{10cm}{images/wlan-accesspoint-1.pdf}

\begin{list1}
\item Sikkerheden er baseret på nogle få forudsætninger 
  \begin{list2}
  \item SSID - netnavnet
  \item WEP \emph{kryptering} - Wired Equivalent Privacy
  \item måske MAC flitrering, kun bestemte kort må tilgå accesspoint 
  \end{list2}
\item Til gengæld er disse forudsætninger ofte ikke tilstrækkelige ... 
%\item Hvorfor hader du WEP?
  \begin{list2}
  \item WEP er måske \emph{ok} til visse små hjemmenetværk
  \item WEP er baseret på en DELT hemmelighed som alle stationer kender
  \item nøglen ændres sjældent, og det er svært at distribuere en ny
  \end{list2}
  
\end{list1}


\slide{Forudsætninger}

\begin{list1}
\item Til gengæld er disse forudsætninger ofte ikke tilstrækkelige ... 
\item Hvad skal man beskytte?
\item Hvordan kan sikkerheden omgås?
\item Mange firmaer og virksomheder stille forskellige krav til
  sikkerheden - der er ikke en sikkerhedsmekanisme der passer til alle
\end{list1}

\slide{SSID - netnavnet}

\begin{list1}
\item Service Set Identifier (SSID) - netnavnet
\item 32 ASCII tegn eller 64 hexadecimale cifre
\item Udstyr leveres typisk med et standard netnavn
\begin{list2}
\item Cisco - tsunami
\item Linksys udstyr - linksys
\item Apple Airport, 3Com m.fl. - det er nemt at genkende dem  
\end{list2}
\item SSID kaldes også for NWID - network id
\item SSID broadcast - udstyr leveres oftest med broadcast af SSID
\end{list1}


%wardriving her
\slide{Demo: wardriving med stumbler programmer}

\hlkimage{17cm}{images/macstumbler.png}  

\begin{list1}
\item man tager et trådløst netkort og en bærbar computer og noget software:
\begin{list2}
\item Netstumbler - Windows \link{http://www.netstumbler.com}
\item dstumbler - UNIX \link{http://www.dachb0den.com/projects/dstumbler.html}
\item iStumbler - Mac \link{http://www.istumbler.net/}
\item Kismet ... mange andre
  \end{list2}
\end{list1}

\slide{Start på demo - wardriving}

\hlkimage{15cm}{images/server-client-wlan.pdf}

\begin{list2}
\item Almindelige laptops bruges til demo
\item Der startes et \emph{access point}
\end{list2}

\slide{MAC filtrering}

\begin{list1}
\item De fleste netkort tillader at man udskifter sin MAC adresse
\item MAC adressen på kortene er med i alle pakker der sendes
\item MAC adressen er aldrig krypteret, for hvordan skulle pakken så
  nå frem?
\item MAC adressen kan derfor overtages, når en af de tilladte
  stationer forlader området ...
\end{list1}

\slide{Resultater af wardriving}

\begin{list1}
\item Hvad opdager man ved wardriving?
\begin{list2}
\item at WEP IKKE krypterer hele pakken
\item at alle pakker indeholder MAC adressen
\item WEP nøglen skifter sjældent
\item ca. 2/3 af de netværk man finder har ikke WEP slået til - og der
  er fri og uhindret adgang til Internet
\end{list2}
\item {\color{red}
Man kan altså lytte med på et netværk med WEP, genbruge en anden
maskines MAC adresse - og måske endda bryde WEP krypteringen.}
\item 
Medmindre man kender virksomheden og WEP nøglen ikke er skiftet ...
det er besværligt at skifte den, idet alle stationer skal opdateres.
\end{list1}

\slide{Storkøbenhavn}

\begin{center}
\colorbox{white}{\includegraphics[width=20cm]{images/20030830-kbh.png}}  
\end{center}



\slide{Informationsindsamling}

\begin{list1}
\item Det vi har udført er informationsindsamling
\item Indsamlingen kan være aktiv eller passiv indsamling i forhold
  til målet for angrebet
\item passiv kunne være at lytte med på trafik eller søge i databaser
  på Internet
\item aktiv indsamling er eksempelvis at sende ICMP pakker og
  registrere hvad man får af svar
\end{list1}

\slide{WEP kryptering} 

%\begin{center}
%\colorbox{white}{\includegraphics[width=12cm]{images/airsnort.pdf}}  
%\end{center}
\begin{list1}
\item WEP \emph{kryptering} - med nøgler der specificeres som tekst
  eller hexadecimale cifre
\item typisk 40-bit, svarende til 5 ASCII tegn eller 10 hexadecimale
  cifre eller 104-bit 13 ASCII tegn eller 26 hexadecimale cifre
\item WEP er baseret på RC4 algoritmen der er en \emph{stream cipher}
  lavet af Ron Rivest for RSA Data Security
\end{list1}


\slide{De første fejl ved WEP}
\begin{list1}
\item Oprindeligt en dårlig implementation i mange Access Points
\item Fejl i krypteringen - rettet i nyere firmware
\item WEP er baseret på en DELT hemmelighed som alle stationer kender
\item Nøglen ændres sjældent, og det er svært at distribuere en ny
\end{list1}

\slide{WEP som sikkerhed}

\hlkimage{6cm}{images/no-wep.pdf}
\begin{list1}
\item WEP er \emph{ok} til et privat hjemmenetværk
\item WEP er for simpel til et større netværk - eksempelvis 20 brugere
\item Firmaer bør efter min mening bruge andre
  sikkerhedsforanstaltninger 
\item Hvordan udelukker man en bestemt bruger?
\end{list1}

\slide{WEP sikkerhed}

\hlkimage{12cm}{images/airsnort.pdf}  

\begin{quote}
AirSnort is a wireless LAN (WLAN) tool which recovers encryption
keys. AirSnort operates by passively monitoring transmissions,
computing the encryption key when enough packets have been gathered.  

802.11b, using the Wired Equivalent Protocol (WEP), is crippled with
numerous security flaws. Most damning of these is the weakness
described in " Weaknesses in the Key Scheduling Algorithm of RC4 "
by Scott Fluhrer, Itsik Mantin and Adi Shamir. Adam Stubblefield
was the first to implement this attack, but he has not made his
software public. AirSnort, along with WEPCrack, which was released
about the same time as AirSnort, are the first publicly available
implementaions of this attack.  \link{http://airsnort.shmoo.com/}
\end{quote}

%\begin{list1}
%\item i dag er firmware opdateret hos de fleste producenter
%\item men sikkerheden baseres stadig på een delt hemmelighed
%\end{list1}

\slide{major cryptographic errors}

\begin{list1}
\item weak keying - 24 bit er allerede kendt - 128-bit = 104 bit i praksis
\item small IV - med kun 24 bit vil hver IV blive genbrugt oftere
\item CRC-32 som integritetscheck er ikke \emph{stærkt} nok
  kryptografisk set
\item Authentication gives pad - giver fuld adgang - hvis der bare
  opdages \emph{encryption pad} for en bestemt IV. Denne IV kan så
  bruges til al fremtidig kommunikation
\end{list1}

{\hlkbig Konklusion: Kryptografi er svært}

\slide{WEP cracking - airodump og aircrack}

\hlkimage{3cm}{images/no-wep.pdf}

\begin{list1}
\item airodump - opsamling af krypterede pakker
\item aircrack - statistisk analyse og forsøg på at finde WEP nøglen
\item Med disse værktøjer er det muligt at knække \emph{128-bit nøgler}!
\item Blandt andet fordi det reelt er 104-bit nøgler \smiley
\item tommelfingerregel - der skal opsamles mange pakker ca. 100.000
  er godt
\item Links:\\
\link{http://www.cr0.net:8040/code/network/aircrack/} aircrack\\
\link{http://www.securityfocus.com/infocus/1814} WEP: Dead Again
\end{list1}

\slide{airodump afvikling}

\begin{list1}
\item Når airodump kører opsamles pakkerne
\item samtidig vises antal initialisationsvektorer IV's:
\end{list1}

\vskip 1 cm

\begin{alltt}
\hlktiny
   BSSID              CH  MB  ENC  PWR  Packets   LAN IP / # IVs   ESSID

   00:03:93:ED:DD:8D   6  11       209   {\bf 801963                  540180}   wanlan    
\end{alltt}

\vskip 2 cm

\begin{list1}
\item NB: dataopsamlingen er foretaget på 100\% opdateret Mac udstyr
\end{list1}


\slide{aircrack - WEP cracker}

\begin{alltt}
\hlktiny
   $ aircrack -n 128 -f 2 aftendump-128.cap
                                 aircrack 2.1
   * Got  540196! unique IVs | fudge factor = 2
   * Elapsed time [00:00:22] | tried 12 keys at 32 k/m
   KB    depth   votes
    0    0/  1   CE(  45) A1(  20) 7E(  15) 98(  15) 72(  12) 82(  12) 
    1    0/  2   62(  43) 1D(  24) 29(  15) 67(  13) 94(  13) F7(  13) 
    2    0/  1   B6( 499) E7(  18) 8F(  15) 14(  13) 1D(  12) E5(  10) 
    3    0/  1   4E( 157) EE(  40) 29(  39) 15(  30) 7D(  28) 61(  20) 
    4    0/  1   93( 136) B1(  28) 0C(  15) 28(  15) 76(  15) D6(  15) 
    5    0/  2   E1(  75) CC(  45) 39(  31) 3B(  30) 4F(  16) 49(  13) 
    6    0/  2   3B(  65) 51(  42) 2D(  24) 14(  21) 5E(  15) FC(  15) 
    7    0/  2   6A( 144) 0C(  96) CF(  34) 14(  33) 16(  33) 18(  27) 
    8    0/  1   3A( 152) 73(  41) 97(  35) 57(  28) 5A(  27) 9D(  27) 
    9    0/  1   F1(  93) 2D(  45) 51(  29) 57(  27) 59(  27) 16(  26) 
   10    2/  3   5B(  40) 53(  30) 59(  24) 2D(  15) 67(  15) 71(  12) 
   11    0/  2   F5(  53) C6(  51) F0(  21) FB(  21) 17(  15) 77(  15) 
   12    0/  2   E6(  88) F7(  81) D3(  36) E2(  32) E1(  29) D8(  27) 
         {\color{red}\bf KEY FOUND! [ CE62B64E93E13B6A3AF15BF5E6 ]}
\end{alltt}
%$


\slide{Hvor lang tid tager det?}

\begin{list1}
\item Opsamling a data - ca. en halv time på 802.11b ved optimale forhold
\item Tiden for kørsel af aircrack fra auditor CD 
på en Dell CPi 366MHz Pentium II laptop:
\end{list1}
\begin{alltt}

   $ time aircrack -n 128 -f 2 aftendump-128.cap
   ...
   real    5m44.180s   user  0m5.902s     sys  1m42.745s
   \end{alltt}
   %$

\pause
\begin{list1}
\item Tiden for kørsel af aircrack på en moderne 1.6GHz CPU med
  almindelig laptop disk tager typisk mindre end 60 sekunder
\end{list1}

\slide{Erstatning for WEP- WPA}

\begin{list1}
\item Det anbefales at bruge:
%\begin{list2}
\item Kendte VPN teknologier eller WPA
\item baseret på troværdige algoritmer
\item implementeret i professionelt udstyr
\item fra troværdige leverandører
\item udstyr der vedligeholdes og opdateres
%\end{list2}
\item Man kan måske endda bruge de eksisterende løsninger - fra
  hjemmepc adgang, mobil adgang m.v.
\end{list1}


\slide{RADIUS}
\begin{list1}
\item RADIUS er en protokol til autentificering af brugere op mod en
  fælles server 
\item Remote Authentication Dial In User Service (RADIUS)
\item RADIUS er beskrevet i RFC-2865
\item RADIUS kan være en fordel i større netværk med 
\begin{list2}
\item dial-in
\item administration af netværksudstyr
\item trådløse netværk
\item andre RADIUS kompatible applikationer
\end{list2}
\end{list1}


\slide{Erstatninger for WEP}
\begin{list1}
\item Der findes idag andre metoder til sikring af trådløse netværk
\item 802.1x Port Based Network Access Control 
\item WPA - Wi-Fi Protected Access)\\
WPA = 802.1X + EAP + TKIP + MIC
\item nu WPA2
\begin{quote}
WPA2 is based on the final IEEE 802.11i amendment to the 802.11
standard and is eligible for FIPS 140-2 compliance.
\end{quote}
\item Kilde: 
\href{http://www.wifialliance.org/OpenSection/protected_access.asp}
{http://www.wifialliance.org/OpenSection/protected\_access.asp}
\end{list1}

\slide{WPA eller WPA2?}

\begin{quote}
WPA2 is based upon the Institute for Electrical and Electronics
Engineers (IEEE) 802.11i amendment to the 802.11 standard, which was
ratified on July 29, 2004.  
\end{quote}

\begin{quote}
Q: How are WPA and WPA2 similar?\\
A: Both WPA and WPA2 offer a high level of assurance for end-users and network
administrators that their data will remain private and access to their
network restricted to authorized users.
Both utilize 802.1X and Extensible Authentication Protocol (EAP) for
authentication. Both have Personal and Enterprise modes of operation
that meet the distinct needs of the two different consumer and
enterprise market segments. 

Q: How are WPA and WPA2 different?\\
A: WPA2 provides a {\bf stronger encryption mechanism} through {\bf
  Advanced Encryption Standard (AES)}, which is a requirement for some
corporate and government users. 
\end{quote}

\centerline{Kilde: http://www.wifialliance.org WPA2 Q and A}

\slide{WPA Personal eller Enterprise}

\begin{list1}
\item Personal - en delt hemmelighed, preshared key
\item Enterprise - brugere valideres op mod fælles server
\item Hvorfor er det bedre?
\begin{list2}
\item Flere valgmuligheder - passer til store og små
\item WPA skifter den faktiske krypteringsnøgle jævnligt - TKIP 
\item Initialisationsvektoren (IV) fordobles 24 til 48 bit   
\item Imødekommer alle kendte problemer med WEP!
\item Integrerer godt med andre teknologier - RADIUS

\vskip 1 cm
\item EAP - Extensible Authentication Protocol - individuel autentifikation
\item TKIP - Temporal Key Integrity Protocol - nøgleskift og integritet
\item MIC - Message Integrity Code - Michael, ny algoritme til integritet
\end{list2}

\end{list1}


\slide{WPA cracking}

\begin{list1}
\item Nu skifter vi så til WPA og alt er vel så godt?  
\pause
\item Desværre ikke!
\item Du skal vælge en laaaaang passphrase, ellers kan man sniffe WPA
  handshake når en computer går ind på netværket!
\item Med et handshake kan man med aircrack igen lave off-line
  bruteforce angreb!
\end{list1}

\slide{WPA cracking demo}

\begin{list1}
\item Vi konfigurerer AP med Henrik42 som WPA-PSK/passhrase  
\item Vi finder netværk kismet eller airodump
\item Vi starter airodump mod specifik kanal
\item Vi spoofer deauth og opsamler WPA handshake
\item Vi knækker WPA :-)
\end{list1}

\centerline{Brug manualsiderne for programmerne i aircrack-ng pakken!}

\slide{WPA cracking med aircrack - start}

\begin{alltt}
\small
slax ~ # aircrack-ng -w dict wlan-test.cap
Opening wlan-test.cap
Read 1082 packets.

#  BSSID              ESSID           Encryption

1  00:11:24:0C:DF:97  wlan            WPA (1 handshake)
2  00:13:5F:26:68:D0  Noea            No data - WEP or WPA
3  00:13:5F:26:64:80  Noea            No data - WEP or WPA
4  00:00:00:00:00:00                  Unknown

Index number of target network ? {\bf 1}
\end{alltt}

\slide{WPA cracking med aircrack - start}

\begin{alltt}
\small
          [00:00:00] 0 keys tested (0.00 k/s)

                    KEY FOUND! [ Henrik42 ]

Master Key     : 8E 61 AB A2 C5 25 4D 3F 4B 33 E6 AD 2D 55 6F 76
                 6E 88 AC DA EF A3 DE 30 AF D8 99 DB F5 8F 4D BD
Transcient Key : C5 BB 27 DE EA 34 8F E4 81 E7 AA 52 C7 B4 F4 56
                 F2 FC 30 B4 66 99 26 35 08 52 98 26 AE 49 5E D7
                 9F 28 98 AF 02 CA 29 8A 53 11 EB 24 0C B0 1A 0D
                 64 75 72 BF 8D AA 17 8B 9D 94 A9 31 DC FB 0C ED

EAPOL HMAC     : 27 4E 6D 90 55 8F 0C EB E1 AE C8 93 E6 AC A5 1F

\end{alltt}

\vskip 1 cm

\centerline{Min Thinkpad X31 med 1.6GHz Pentium M knækker ca. 150 Keys/sekund}

\slide{Encryption key length}

\hlkimage{19cm}{encryption-crack.png}

Kilde: \link{http://www.mycrypto.net/encryption/encryption_crack.html}

\slide{WPA cracking med Pyrit}

\begin{quote}
\emph{Pyrit} takes a step ahead in attacking WPA-PSK and WPA2-PSK, the protocol that today de-facto protects public WIFI-airspace. The project's goal is to estimate the real-world security provided by these protocols. Pyrit does not provide binary files or wordlists and does not encourage anyone to participate or engage in any harmful activity. {\bf This is a research project, not a cracking tool.}

\emph{Pyrit's} implementation allows to create massive databases, pre-computing part of the WPA/WPA2-PSK authentication phase in a space-time-tradeoff. The performance gain for real-world-attacks is in the range of three orders of magnitude which urges for re-consideration of the protocol's security. Exploiting the computational power of GPUs, \emph{Pyrit} is currently by far the most powerful attack against one of the world's most used security-protocols. 
\end{quote}

\begin{list1}
\item sloooow, plejede det at være -  ~150 keys/s på min Thinkpad X31
\item Kryptering afhænger af SSID! Så check i tabellen er ~minutter.
\item \link{http://pyrit.wordpress.com/about/} 
\end{list1}

\slide{Tired of WoW?}

\hlkimage{22cm}{pyritperfaa3.png}

Kilde: \link{http://code.google.com/p/pyrit/}


\slide{Tools man bør kende}

\begin{list2}
\item Aircrack {http://www.aircrack-ng.org/}
\item Kismet \link{http://www.kismetwireless.net/}
\item Airsnort \link{http://airsnort.shmoo.com/} læs pakkerne med WEP
  kryptering  
\item Airsnarf \link{http://airsnarf.shmoo.com/} - lav dit eget AP
  parallelt med det rigtige og snif hemmeligheder
\item Wireless Scanner \link{http://www.iss.net/} - kommercielt 
%\item wepcrack \link{http://wepcrack.sourceforge.net/} - knæk
  krypteringen i WEP
%\item BSD Airtools \link{http://www.dachb0den.com/projects/bsd-airtools.html}
\item Dette er et lille uddrag af programmer\\
Se også \link{http://packetstormsecurity.org/wireless/}
\end{list2}

\slide{Når adgangen er skabt}

\begin{list1}
\item Så går man igang med de almindelige værktøjer
\item Fyodor Top 100 Network Security Tools \link{http://www.sectools.org}  
\end{list1}
\vskip 2 cm

\centerline{\hlkbig Forsvaret er som altid - flere lag af sikkerhed! }

\slide{Infrastrukturændringer}

\begin{center}
\colorbox{white}{\includegraphics[height=11cm]{images/wlan-accesspoint-2.pdf}}
\end{center}

\centerline{\hlkbig Sådan bør et access point forbindes til netværket}


\slide{Anbefalinger mht. trådløse netværk}

\begin{minipage}{10cm}
\includegraphics[width=10cm]{images/wlan-accesspoint-2.pdf}
\end{minipage}
\begin{minipage}{\linewidth-10cm}
\begin{list2}
\item Brug noget tilfældigt som SSID - netnavnet
\item Brug ikke WEP til virksomhedens netværk\\
- men istedet en VPN løsning med individuel
  autentificering eller WPA
\item NB: WPA Personal/PSK kræver passphrase på +40 tegn!
\item Placer de trådløse adgangspunkter hensigtsmæssigt i netværket -
  så de kan overvåges
\item Lav et sæt regler for brugen af trådløse netværk - hvor må 
  medarbejdere bruge det?
\item Se eventuelt pjecerne \emph{Beskyt dit trådløse Netværk} fra
Ministeriet for Videnskab, Teknologi og Udvikling \\
\link{http://www.videnskabsministeriet.dk/}
\end{list2}
\end{minipage} 


\slide{Hjemmenetværk for nørder}

\begin{list1}
\item Lad være med at bruge et wireless-kort i en PC til at lave AP, brug et AP
\item Husk et AP kan være en router, men den kan ofte også blot være en bro
\item Brug WPA og overvej at lave en decideret DMZ til WLAN
\item Placer AP hensigtsmæddigt og gerne højt, oppe på et skab eller lignende
\end{list1}

%\exercise{ex:AirPort-AP}

%\exercise{ex:wardriving-windows}
%\exercise{ex:wardriving-kismet}
%\exercise{ex:aircrack-ng}



\slide{Dynamisk routing} 

\begin{list1}
\item Når netværkene vokser bliver det administrativt svært at vedligeholde
\item Det skalerer dårligt med statiske routes til netværk
\item Samtidig vil man gerne have redundante forbindelser
\item Til dette brug har man STP på switch niveau og dynamisk routing på IP niveau
\end{list1}



% OpenBGPD

\slide{BGP Border Gateway Protocol}

\begin{list1}
\item Er en dynamisk routing protocol som benyttes eksternt
\item Netværk defineret med AS numre annoncerer hvilke netværk de er forbundet til
\item Autonomous System (AS) er en samling netværk
\item BGP version 4 er beskrevet i RFC-4271
\item BGP routere forbinder sig til andre BGP routere og snakker sammen, \emph{peering}
\item \link{http://en.wikipedia.org/wiki/Border_Gateway_Protocol}
\item Vores setup svarer til dette:
\item \link{http://www.kramse.dk/projects/network/openbgpd-basic_en.html}
\end{list1}

\slide{RIP Routing Information Protocol} 

\begin{list1}
\item Gammel routingprotokol som ikke benyttes mere
\item RIP er en distance vector routing protokol, tæller antal hops
\item \link{http://en.wikipedia.org/wiki/Routing_Information_Protocol}
\end{list1}


% OpenOSPFD

\slide{OSPF Open Shortest Path First}

\begin{list1}
\item Er en dynamisk routing protocol som benyttes til intern routing
\item OSPF version 3 er beskrevet i RFC-2740
\item OSPF bruger hverken TCP eller UDP, men sin egen protocol med ID 89
\item OSPF bruger en metric/cost pr link for at udregne smart routing 
\item \link{http://en.wikipedia.org/wiki/Open_Shortest_Path_First}
\item Vores setup svarer til OpenBGPD setup, blot med OpenOSPFD
\end{list1}


\slide{EIGRP}

\begin{list1}
\item Cisco protokol til intern routing, hvis man udelukkende har Cisco udstyr
\item \link{http://www.cisco.com}
\end{list1}

\slide{Stop - vi gennemgår og tester vores dynamiske routing}

\begin{list1}
\item Vi gennemgår hvordan vores setup ser ud
\item Vi laver traceroute før og efter:
\item Vi fjerner en ledning \emph{link down}
\item Vi stopper en router og ser de annoncerede netværk forsvinder
\item Vi booter en router og ser de annoncerede netværk igen
\end{list1}

\slide{Båndbreddestyring og policy based routing}

\begin{list1}
\item Mange routere og firewalls idag kan lave båndbredde allokering til
  protokoller, porte og derved bestemte services
\item Mest kendte er i Open Source:
\begin{list2}
\item ALTQ bruges på OpenBSD - integreret i PF
%  \link{http://www.csl.sony.co.jp/person/kjc/kjc/software.html}
\item FreeBSD har dummynet
\item Linux har tilsvarende\\
ADSL-Bandwidth-Management-HOWTO, ADSL Bandwidth Management HOWTO\\
Adv-Routing-HOWTO, Linux Advanced Routing \& Traffic Control HOWTO\\
\link{http://www.knowplace.org/shaper/resources.html} Linux resources
\end{list2}
\item Det kaldes også traffic shaping
\end{list1}


\slide{Routingproblemer, angreb}

\begin{list1}
  \item falske routing updates til protokollerne
\item sende redirect til maskiner
\item source routing - mulighed for at specificere en ønsket vej for
  pakken 
\item Der findes (igen) specialiserede programmer til at teste og
  forfalske routing updates, svarende til icmpush programmet
\item Det anbefales at sikre routere bedst muligt - eksempelvis 
Secure IOS template der findes på adressen:\\
{\small \link{http://www.cymru.com/Documents/secure-ios-template.html}}
\item Med UNIX systemer generelt anbefales opdaterede systemer og netværkstuning
\end{list1}


\slide{Source routing}

\begin{list1}
\item Hvis en angriber kan fortælle hvilken vej en pakke skal følge
  kan det give anledning til sikkerhedsproblemer
\item maskiner idag bør ikke lytte til source routing, evt. skal de
  droppe pakkerne
\end{list1}


\slide{Formålet med resten af dagen }

\hlkimage{15cm}{unix-servers.png}

\centerline{Vi skal gennemgå gængse internet-serverfunktioner}

\slide{Network Services}

\begin{list1}
\item Flere UNIX varianter har fået mere moderne strukturer til at
  starte services
\item SystemV start/stop af services er stadig meget udbredt rc.d
  katalogstrukturer 
\item Solaris: Service Management Facility SMF
\item AIX: Subsystem Ressource Controller  
\item Mac OS X: launchd
\item Windows: services, net stop/start m.fl.
\end{list1}

\slide{daemoner}

%chuckie billede
%\hlkimage{}{}

\begin{list1}
\item Hjælper med til at køre systemet
\item udfører jobs
\item typiske daemoner er:
  \begin{list2}
  \item ftpd - FTP daemonen giver FTP adgang til filoverførsler
\item Telnetd - giver login adgang - NB: ukrypteret!
\item tftpd - Trivial file transfer protocol daemon, bruges til boot
  og opgradering af netværksudstyr - kræver ikke password
\item pop3d - POP3 post office protocol, elektronisk post
\item sshd - SSH protokol daemonen giver adgang til login via SSH
  \end{list2}
\end{list1}

\slide{inetd en super server}

\begin{list1}
  \item inetd har mange funktioner
\item istedet for at have 10 programmer der lytter på diverse porte
  kan inetd lytte på en hel masse, og så give forbindelsen videre til
  programmerne når der er brug for det:
\item \verb+/etc/inetd.conf+
\end{list1}

\begin{alltt}
\small
finger stream  tcp     nowait  nobody  /usr/libexec/tcpd fingerd -s
ftp    stream  tcp     nowait  root    /usr/libexec/tcpd ftpd -l
login  stream  tcp     nowait  root    /usr/libexec/tcpd rlogind
nntp   stream  tcp     nowait  usenet  /usr/libexec/tcpd nntpd
ntalk  dgram   udp     wait    root    /usr/libexec/tcpd ntalkd
shell  stream  tcp     nowait  root    /usr/libexec/tcpd rshd
telnet stream  tcp     nowait  root    /usr/libexec/tcpd telnetd
uucpd  stream  tcp     nowait  root    /usr/libexec/tcpd uucpd
comsat dgram   udp     wait    root    /usr/libexec/tcpd comsat
tftp   dgram   udp     wait    nobody  /usr/libexec/tcpd tftpd /tftpboot
\end{alltt}

\slide{xinetd}

\begin{list1}
\item konfigureres med separate filer pr service i kataloget
\verb+/etc/xinetd.d+ eksempelvis: \verb+/etc/xinetd.d/cups-lpd+:
\end{list1}

\begin{alltt}
service printer
\{
        socket_type = stream
        protocol    = tcp
        wait        = no
        user        = lp
        server      = /usr/lib/cups/daemon/cups-lpd
        disable     = yes
\}  
\end{alltt}

\slide{UNIX Print systemer}

\hlkimage{6cm}{images/cups-happy.pdf}

\begin{list1}
\item De fleste benytter idag standard kommandoerne:
\begin{list2}
\item \verb+lp+ og \verb+lpr+ - print files
\item \verb+lpq+ - show printer queue status
\item \verb+lprm+ - cancel print jobs
\item Mange bruger softwaren Common UNIX Printing System fra
  \link{http://www.cups.org} 
\item Gamle UNIX systemer bruger stadig konfiguration via
  \verb+/etc/printcap+ 
\item remote print sker gennem Line Printer Daemon LPD protokollen port 515/tcp
\item nyere printere understøtter Internet Printing Protocol IPP port 80/tcp
\end{list2}
\end{list1}

Kilde: billede er fra CUPS


\slide{TFTP Trivial File Transfer Protocol}

\begin{list1}
\item Trivial File Transfer Protocol - uautentificerede filoverførsler
\item De bruges især til:
  \begin{list2}
\item TFTP bruges til boot af netværksklienter uden egen harddisk
\item TFTP benytter UDP og er derfor ikke garanteret at data overføres korrekt
  \end{list2}
\item TFTP sender alt i klartekst, hverken password \\ 
{\bfseries USER brugernavn} og \\
{\bfseries PASS hemmeligt-kodeord} 
\end{list1}


\slide{FTP File Transfer Protocol}

\begin{list1}
\item File Transfer Protocol - filoverførsler
\item Bruges især til:
  \begin{list2}
    \item FTP - drivere, dokumenter, rettelser - Windows Update? er
    enten HTTP eller FTP
  \end{list2}
\item FTP sender i klartekst\\ 
{\bfseries USER brugernavn} og \\
{\bfseries PASS hemmeligt-kodeord} 
\item Der findes varianter som tillader kryptering, men brug istedet SCP/SFTP over Secure Shell protokollen
\end{list1}


\slide{FTP Daemon konfiguration}

\begin{list1}
\item Meget forskelligt!
\item WU-FTPD er meget udbredt
\item BSD FTPD ligeså meget anvendt
\item \emph{anonym ftp} er når man tillader alle at logge ind\\
men husk så ikke at tillade upload af filer!
\item På BSD oprettes blot en bruger med navnet \verb+ftp+ så er der åbent!
\end{list1}


\slide{NTP Network Time Protocol}

\begin{list1}
\item NTP opsætning
\item foregår typisk i \verb+/etc/ntp.conf+ eller \verb+/etc/ntpd.conf+   
\item det vigtigste er navnet på den server man vil bruge som tidskilde
\item Brug enten en NTP server hos din udbyder eller en fra \link{http://www.pool.ntp.org/}
\item Eksempelvis:
\end{list1}

\begin{alltt}
server ntp.cybercity.dk

server 0.dk.pool.ntp.org
server 0.europe.pool.ntp.org
server 3.europe.pool.ntp.org

\end{alltt}

\slide{What time is it?}

\hlkimage{8cm}{images/xclock.pdf}

\begin{list1}
\item Hvad er klokken?
\item Hvad betydning har det for sikkerheden?
\item Brug NTP Network Time Protocol på produktionssystemer
\end{list1}


\slide{What time is it? - spørg ICMP}

\vskip 1 cm

\begin{list1}
  \item ICMP timestamp option - request/reply
\item hvad er klokken på en server
\item Slayer icmpush - er installeret på server
\item viser tidstempel
\end{list1}

\begin{alltt}
# {\bfseries icmpush -v -tstamp 10.0.0.12}
ICMP Timestamp Request packet sent to 10.0.0.12 (10.0.0.12)

Receiving ICMP replies ...
fischer         -> 21:27:17
icmpush: Program finished OK
\end{alltt}

\slide{Stop - NTP Konfigurationseksempler}

\hlkimage{12cm}{osx-ntp.png}

\begin{list1}
\item Vi har en masse udstyr, de meste kan NTP, men hvordan
\item Vi gennemgår, eller I undersøger selv:
\begin{list2}
\item Airport
\item Switche (managed)
\item Mac OS X
\item OpenBSD - check \verb+man rdate+ og \verb+man ntpd+
\end{list2}
\end{list1}

\slide{BIND DNS server}

\begin{list1}
\item Berkeley Internet Name Daemon server
\item Mange bruger BIND fra Internet Systems Consortium
   - altså Open Source
\item konfigureres gennem \verb+named.conf+
\item det anbefales at bruge BIND version 9
\end{list1}

\begin{list2}
\item \emph{DNS and BIND}, Paul Albitz \& Cricket Liu, O'Reilly, 4th
  edition Maj 2001 
\item \emph{DNS and BIND cookbook}, Cricket Liu, O'Reilly, 4th
  edition Oktober 2002 
\end{list2}

Kilde: \link{http://www.isc.org}




\slide{BIND konfiguration - et udgangspunkt}

\begin{alltt}
\small 
acl internals \{ 127.0.0.1; ::1; 10.0.0.0/24; \};
options \{
        // the random device depends on the OS !
        random-device "/dev/random"; directory "/namedb";
        port 53; version "Dont know"; allow-query \{ any; \};
\};
view "internal" \{
   match-clients \{ internals; \};
   recursion yes;
   zone "." \{
       type hint;   file "root.cache"; \};
   // localhost forward lookup
   zone "localhost." \{
        type master; file "internal/db.localhost";   \};
   // localhost reverse lookup from IPv4 address
   zone "0.0.127.in-addr.arpa" \{
        type master; file "internal/db.127.0.0"; notify no;   \};
...
\}
\end{alltt}

\exercise{ex:bind-version}

\exercise{ex:bind-config}

\exercise{ex:bind-dnszone}

\slide{Små DNS tools bind-version - Shell script}

\begin{alltt}\small
#! /bin/sh
# Try to get version info from BIND server
PROGRAM=`basename $0`
. `dirname $0`/functions.sh
if [ $# -ne 1 ]; then
   echo "get name server version, need a target! "
   echo "Usage: $0 target"
   echo "example $0 10.1.2.3"
   exit 0
fi
TARGET=$1
# using dig 
start_time
dig @$1 version.bind chaos txt
echo Authors BIND er i versionerne 9.1 og 9.2 - måske ...
dig @$1 authors.bind chaos txt
stop_time
\end{alltt}
\centerline{\link{http://www.kramse.dk/files/tools/dns/bind-version}}

\slide{Små DNS tools dns-timecheck - Perl script}

\begin{alltt}\small
#!/usr/bin/perl
# modified from original by Henrik Kramshøj, hlk@kramse.dk
# 2004-08-19
#
# Original from: http://www.rfc.se/fpdns/timecheck.html
use Net::DNS;

my $resolver = Net::DNS::Resolver->new;
$resolver->nameservers($ARGV[0]);

my $query = Net::DNS::Packet->new;
$query->sign_tsig("n","test");

my $response = $resolver->send($query);
foreach my $rr ($response->additional) {
  print "localtime vs nameserver $ARGV[0] time difference: ";
  print$rr->time_signed - time() if $rr->type eq "TSIG";
}  
\end{alltt}
% inserting stupid $ to stop EMACS from
\centerline{\link{http://www.kramse.dk/files/tools/dns/dns-timecheck}}


\slide{DHCPD server}

\begin{list1}
\item Dynamic Host Configuration Protocol Server
\item Mange bruger DHCPD fra Internet Systems Consortium\\
  \link{http://www.isc.org} - altså Open Source
\item konfigureres gennem \verb+dhcpd.conf+ - næsten samme syntaks som BIND 
\item DHCP er en efterfølger til BOOTP protokollen
\end{list1}

\begin{alltt}
\small
ddns-update-style ad-hoc;

shared-network LOCAL-NET \{
    option  domain-name "security6.net";
    option  domain-name-servers 212.242.40.3, 212.242.40.51;
    subnet 10.0.42.0 netmask 255.255.255.0 \{
            option routers 10.0.42.1;
            range 10.0.42.32 10.0.42.127;
    \}
\}  
\end{alltt}

\exercise{ex:dhcpd-config}





\slide{Logfiler}
\begin{list1}
\item Logfiler er en nødvendighed for at have et transaktionsspor
\item Logfiler giver mulighed for statistik
\item Logfiler er desuden nødvendige for at fejlfinde
\item Det kan være relevant at sammenholde logfiler fra:  
\begin{list2}
\item routere
\item firewalls
\item webservere
\item intrusion detection systemer
\item adgangskontrolsystemer
\item ...
\end{list2}
\item Husk - tiden er vigtig! Network Time Protocol (NTP) anbefales 
\item Husk at logfilerne typisk kan slettes af en angriber -
  hvis denne får kontrol med systemet
\end{list1}

% apache HTTPD

\slide{World Wide Web fødes}

\hlkimage{15cm}{images/tim-berners-lee-2001-europaeum-eighth.jpg}

\begin{list1}
\item Tim Berners-Lee opfinder WWW 1989 og den første webbrowser og
  server i 1990 mens han arbejder for CERN
\end{list1}

Kilde:
\link{http://www.w3.org/People/Berners-Lee/}

\slide{World Wide Web udviklingen}

\hlkimage{20cm}{images/Count_WWW.png}

\begin{list1}
\item Udviklingen på world wide web bliver en stor kommerciel success
\end{list1}

Kilde: Hobbes Internet time-line\\
\link{http://www.zakon.org/robert/internet/timeline/}

\slide{Nogle HTTP og webrelaterede RFC'er}

\begin{list2}
\item[1945] Hypertext Transfer Protocol -- HTTP/1.0. T. Berners-Lee, R.
     Fielding, H. Frystyk. May 1996.
\item[2068] Hypertext Transfer Protocol -- HTTP/1.1. R. Fielding, J. Gettys,
     J. Mogul, H. Frystyk, T. Berners-Lee. January 1997. (Obsoleted by
     RFC2616)
\item[2069] An Extension to HTTP : Digest Access Authentication. J. Franks,
     P. Hallam-Baker, J. Hostetler, P. Leach, A. Luotonen, E. Sink, L.
     Stewart. January 1997. (Obsoleted by
     RFC2617)
\item[2145] Use and Interpretation of HTTP Version Numbers. J. C. Mogul, R.
     Fielding, J. Gettys, H. Frystyk. May 1997. 
\item[2518] HTTP Extensions for Distributed Authoring -- WEBDAV. Y. Goland,
     E. Whitehead, A. Faizi, S. Carter, D. Jensen. February 1999.
\item[2616] Hypertext Transfer Protocol -- HTTP/1.1. R. Fielding, J. Gettys,
     J. Mogul, H. Frystyk, L. Masinter, P. Leach, T. Berners-Lee. June
     1999. (Obsoletes
     RFC2068) (Updated by RFC2817)
\item[2818] HTTP Over TLS. E. Rescorla. May 2000. 
\end{list2}

\begin{quote}
HTTP er basalt set en sessionsløs protokol bestående at individuelle
HTTP forespørgsler via TCP forbindelser  
\end{quote}

\slide{Infokager og state management}
\begin{list2}
\item[2109] HTTP State Management Mechanism. D. Kristol, L. Montulli.
     February 1997. (Format: TXT=43469 bytes) (Obsoleted by RFC2965)
     (Status: PROPOSED STANDARD)
\item[2965] HTTP State Management Mechanism. D. Kristol, L. Montulli. October
     2000. (Format: TXT=56176 bytes) (Obsoletes RFC2109) (Status: PROPOSED
     STANDARD)
\end{list2}
\begin{quote}
1.  ABSTRACT
   This document specifies a way to create a stateful session with HTTP
   requests and responses.  It describes two new headers, Cookie and
   Set-Cookie, which carry state information between participating
   origin servers and user agents.  The method described here differs
   from Netscape's Cookie proposal, but it can interoperate with
   HTTP/1.0 user agents that use Netscape's method.  (See the HISTORICAL
   section.)
\end{quote}

(Citatet er fra RFC-2109)

\slide{Apache HTTP serveren}

\hlkimage{14cm}{images/httpd_logo_wide.pdf}
\vskip 1 cm
{\hlkbig Hvorfor skrive Apache HTTP server?}

\vskip 1 cm
\begin{quote}
Fordi Apache idag er en organisation med mange delprojekter - hvoraf
mange relateres til web og webløsninger  
\end{quote}

\slide{Er Apache HTTP server interessant?}

\hlkimage{20cm}{images/netcraft-2004.pdf}

\begin{list1}
\item Apache HTTP server er iflg. netcraft og andre den mest benyttede
  HTTP server på Internet!
\item Apache er grand old man i Internet sammenhæng - bygget udfra
  NCSA HTTP serveren
\item Mange løsninger bygges på Apache
\end{list1}

Kilde: \link{http://www.netcraft.com}

\slide{Hvad er Apache?}

\hlkimage{8cm}{images/apache_pb.png}

\begin{quote}
The Apache HTTP Server Project is an effort to develop and maintain an
open-source HTTP server for modern operating systems including UNIX
and Windows NT. The goal of this project is to provide a secure,
efficient and extensible server that provides HTTP services in sync
with the current HTTP standards.   
\end{quote}

Kilde: Apache HTTPD FAQ \link{http://httpd.apache.org}


\slide{Fordele ved Apache HTTPD}

\begin{list1}
\item En HTTP webserver oprindeligt baseret på NCSA webserveren,
(National Center for Supercomputing Applications)\\
- Apache is "A PAtCHy server" 
\item Konfigurerbar og fleksibel
\item Understøtter moduler
\item Open Source og kildeteksten er tilgængelig med en fri licens
\item Understøtter HTTP/1.1
\item allestedsnærværende - UNIX: Linux, IBM AIX, BSD, Sun Solaris...
\end{list1}

Kilde: Apache HTTPD FAQ \link{http://httpd.apache.org}

\slide{Men Apache er også ...}
\begin{list1}
\item Apache Software Foundation med mange andre spændende projekter
\begin{list2}
\item Cocoon som er et komponentbaseret \emph{web development framework} 
\item Apache Tomcat som er en servlet container der bruges som den officielle
referenceimplementation for Java Servlet og JavaServer Pages
teknologierne
\item Apache-SSL SSL delen af webserveren
\item FOP print formatter drevet af XSL formatting objects (XSL-FO)
\item Xerces XML parser
\item Xalan XSLT processor
\end{list2}
\item XML og Web services er buzz-words idag
\end{list1}


\slide{Apache varianter}

\begin{list2}
\item Stronghold - sælges ikke mere\\
\link{http://www.redhat.com/software/stronghold/} 
\item IBM HTTP server\\
\link{http://www-306.ibm.com/software/webservers/httpservers/}  
\item Oracle HTTP Server
\item HP Secure Web Server for OpenVMS Alpha (based on Apache)
\item findes sikkert flere 
\end{list2}

\slide{Brug dokumentationen }
\hlkimage{14cm}{images/apache-docs.pdf}
\centerline{\link{http://httpd.apache.org/docs-2.0/}}


\slide{Apache kogebogen}

\hlkimage{6cm}{images/apacheckbk.png}

\begin{list2}
\item Vi bruger bogen Apache cookbook på kurset
\item Både som opgavehæfte og opslagsværk
\item \emph{Apache Cookbook} af Ken Coar, Rich Bowen, November 2003,
ISBN: 0-596-00191-6
\end{list2}

\slide{Apache Security bogen}

\hlkimage{6cm}{images/apachesc.png}

\begin{list2}
\item Vi bruger bogen Apache Security bogen på kurset
\item Primært som opslagsværk
\item \emph{Apache Security} af Ivan Ristic, February 2005,ISBN: 0-596-00724-8
\end{list2}

\slide{Start og stop af apache}

\begin{list1}
\item Apache bruger programmet \verb+apachectl+  
\item Dette program kan bruges til flere formål:
\begin{list2}
\item \verb+apachectl start+ - opstart af apache    
\item \verb+apachectl stop+ - stop af apache
\item \verb+apachectl restart+ - genstart af apache    
\item \verb+apachectl configtest+ - test af apache konfigurationen - syntaks!
\end{list2}
\item husk at der kan være flere versioner af apache på systemet!
\item Det kan være en fordel enten at lave et alias eller ændre PATH i
  jeres SHELL profil!\\
\verb+alias apachectl="/home/hlk/apache2/bin/apachectl"+ 
\end{list1}

\slide{ServerAdmin, ServerRoot, DocumentRoot}

\begin{alltt}\small
ServerAdmin webmaster@security6.net
ServerRoot "/usr/local/apache2"
DocumentRoot "/userdata/sites" 
User www
Group www
ServerName fluffy:80
\end{alltt}

\begin{list2}
%\item Nogle basale indstillinger for en Apache server
\item Indsættes i \verb+httpd.conf+
\item ServerAdmin - administratoren for denne webserver, bør sættes
  til eksempelvis webmaster@security6.net
\item ServerRoot - roden af serveren, mange andre referencer sker
  relativt til denne
\item DocumentRoot - den primære placering for filer der skal serveres
  fra denne server 
\item User og Group - hvilket brugerid skal serveren afvikles under,
  efter opstarten som root
\item Servername - hvad hedder denne server
\end{list2}


\slide{Apache Logfiler, konfiguration af logfiler}

\begin{verbatim}
LogLevel warn 
ErrorLog /usr/local/apache2/logs/error_log 
LogFormat "%h %l %u %t \"%r\" %>s %b \"%{Referer}i\" 
    %\"%{User-Agent}i\"" combined
CustomLog /usr/local/apache2/logs/access_log combined   
\end{verbatim}

\begin{list1}
\item Logning i Apache styres med log-direktiver til to slags - access
  og error 
\item De vigtigste direktiver i httpd.conf er:
\begin{list2}
\item LogLevel - hvad skal logges af beskeder i error log, fra emerg til debug
\item ErrorLog - default fejlbeskeder fra Apache 
\item LogFormat - hvordan skal access log se ud
\item CustomLog - hvor skal access log gemmes - default    
\end{list2}
\end{list1}

\slide{Tuning af Apache opstartsparametre}

\begin{list1}
\item Det anbefales at indstille Apache opstarten som noget af det
  første
\item UNIX bruger som standard prefork modellen hvor Apache starter et
  antal processer der forventes at være nogenlunde OK til den
  forventede belastning 
\end{list1}

\begin{alltt}
\small
# prefork MPM
# StartServers: number of server processes to start
# MinSpareServers: minimum number of server processes which are kept spare
# MaxSpareServers: maximum number of server processes which are kept spare
# MaxClients: maximum number of server processes allowed to start
# MaxRequestsPerChild: maximum number of requests a server process serves
<IfModule prefork.c>
StartServers         5
MinSpareServers      5
MaxSpareServers     10
MaxClients         150
MaxRequestsPerChild  0
</IfModule>
\end{alltt}

\slide{Virtuelle hosts}

\begin{alltt}
<VirtualHost *:80>
    ServerAdmin webmaster@security6.net
    ServerName www.security6.net
    ServerAlias security6.net
    ServerAlias www.security6.dk
    DocumentRoot /userdata/sites/security6.net
    ErrorLog logs/security6.net-error_log
    CustomLog logs/security6.net-access_log combined
...
</VirtualHost>
\end{alltt}

\begin{list1}
\item Apache HTTPD tillader at man benytter virtuelle hosts
\item Bemærk at det er klienten der overfører hostnavnet i HTTP request
\end{list1}

\exercise{ex:apache-httpd.conf}
\exercise{ex:apache-virtual}


\slide{Grundlæggende Apache CGI}

\begin{list1}
\item Common Gateway Interface - standard metode til programmer
\item \verb+ScriptAlias+ er det direktiv der angiver at CGI må
 afvikles
\item Der følger to eksempler med Apache 2 i ServerRoot/cgi-bin:
\begin{list2}
\item printenv - viser en del information om serveren
\item test-cgi - viser hvordan man kan bruge parametre  
\end{list2}
\item {\bf NB: husk at fjerne x-bit efter test!}
\end{list1}

\begin{verbatim}
ScriptAlias /cgi-bin/ "/usr/local/apache2/cgi-bin/"
<Directory "/usr/local/apache2/cgi-bin">
        AllowOverride None
        Options None
        Order allow,deny
        Allow from all
</Directory>
\end{verbatim}

\slide{Hello World CGI - Insecure programming}

\vskip 2 cm

\begin{list1}
\item Problem:\\
Ønsker et simpelt CGI program, en web udgave af finger
\item Formål:\\
Vise oplysningerne om brugere på systemet
\end{list1}

\slide{review af nogle muligheder}

\begin{list1}
\item ASP
\begin{list2}
\item server scripting, meget generelt - man kan alt
\end{list2}

\item SQL
\begin{list2}
\item databasesprog - meget kraftfuldt
\item mange databasesystemer giver mulighed for specifik tildeling af
  privilegier "grant" 
  \end{list2}
\item JAVA
\begin{list2}
\item generelt programmeringssprog
\item bytecode verifikation
\item indbygget sandbox funktionalitet 
\end{list2}
\item Perl og andre generelle programmeringssprog
\item Pas på shell escapes!!!
\end{list1}

\slide{Hello world of insecure web CGI}

\begin{list1}
\item Demo af et sårbart system - badfinger
\item Løsning:
\begin{list2}
\item Kalde finger kommandoen
\item et Perl script
\item afvikles som CGI 
\item standard Apache HTTPD 1.3 server
\end{list2}
\end{list1}

\slide{De vitale - og usikre dele}

{\small
\begin{verbatim}
print "Content-type: text/html\n\n<html>";
print "<body bgcolor=#666666 leftmargin=20 topmargin=20"; 
print "marginwidth=20 marginheight=20>";
print <<XX;
<h1>Bad finger command!</h1>
<HR COLOR=#000>
<form method="post" action="bad_finger.cgi">
Enter userid: <input type="text" size="40" name="command">
</form>
<HR COLOR=#000>
XX
if(&ReadForm(*input)){
    print "<pre>\n";
    print "will execute:\n/usr/bin/finger $input{'command'}\n";
    print "<HR COLOR=#000>\n";
    print `/usr/bin/finger $input{'command'}`;
    print "<pre>\n";
}
\end{verbatim}}

\slide{Almindelige problemer}

\begin{list1}
\item validering af forms
\item validering på klient er godt\\
- godt for brugervenligheden, hurtigt feedback
\item validering på clientside gør intet for sikkerheden
\item serverside validering er nødvendigt
\item generelt er input validering det største problem!
\end{list1}

Brug \emph{Open Web Application Security
Project} \link{http://www.owasp.org}

\slide{Apache HTTPD sikkerhedshuller}

\begin{list1}
\item En Apache installation er ikke bare en HTTPD server
\item ofte inkluderes:
  \begin{list2}
  \item OpenSSL til SSL, dvs HTTPS forbindelser
  \item PHP - et web programmeringssprog
  \item Perl - et programmeringssprog som ofte benyttes til web
  \end{list2}
\item Hver eneste komponent kan have sikkerhedsproblemer!
\end{list1}


\slide{Web løsninger før}

\hlkimage{15cm}{images/web-static.png}

\begin{list1}
\item Statiske hjemmesider i HTML
\item Overskueligt
\item Få regler i firewall
\item Ikke behov for adgang til data indenfor firewall
\end{list1}


\slide{Web løsninger idag}

\hlkimage{18cm}{images/web-dynamic.png}

\slide{Web løsninger idag}

\vskip 2 cm

\begin{list1}
\item Dynamiske hjemmesider - ASP, PHP m.fl.
\item Høj kompleksitet - flere muligheder for fejl
\item Mange regler i firewall - flere DMZ områder/net
\item Behov for adgang til ordredata m.v. indenfor firewall
\end{list1}


\slide{Gode Råd til dynamiske webmiljøer}

\vskip 2 cm

\begin{list1}
\item Brug databaser - der er gode muligheder for finmasket adgangskontrol
\item Brug versionsstyring - hvem lavede hvilket program, hvornår
\item Brug ressourcer på opdatering af medarbejdere
\item Lav retningslinier for webudvikling
\item Overvåg alle systemerne!
\end{list1}

\slide{Typisk fejl på webservere}

\begin{list1}
\item De mest alvorlige:
  \begin{list2}  
\item Ingen hærdning
\item Ingen opdatering efter idriftsættelse
\end{list2}
\item Medium eller kritiske
  \begin{list2}
  \item Adgang til eksempel-programmer (eng: sample programs)
    - kan til tider være meget kritisk!
  \end{list2}
\item De mindre alvorlige
  \begin{list2}
  \item Informationsindsamling
  \item Netmaske - \emph{icmpush -mask}
  \item Navne på udviklere, firmaer, datoer
  \end{list2}
\end{list1}


\slide{chroot og jails}

\begin{list1}
  \item Chroot står for change root, og betyder at processen som
  kalder chroot systemkaldet udskifter sin \emph{filsystemsrod-/} med
  et andet katalog på systemet
\item Oprindeligt blev denne funktion lavet til at teste nye UNIX
  releases uden at overskrive det oprindelige miljø man havde på
  systemet
\item men det kan bruges til at give mere sikkerhed
\item En daemon eller service der kører chroot'ed er sværere at
  udnytte - simpelthen fordi den kun har adgang til en lille del af
  systemet
\item FreeBSD har en endnu mere avanceret version af chroot som giver
  endnu mere kontrol over det miljø som programmerne ser
\end{list1}

\slide{brug af chroot}

\begin{list1}
\item De services man typisk vil chroot'e er BIND, Apache og andre
  udsatte services
\item Der findes heldigvis udførlige beskrivelser af hvordan man
chroote de mest almindelige services 
\item NB: husk at løsninger med Apache ofte kræver PHP, Perl,
  databaser osv. 
\end{list1}

\slide{Gennemgang af chroot konceptet}

\hlkimage{15cm}{images/named-pipe.pdf}
\centerline{named pipes og chroot}

\begin{list2}
\item Husk: Apachedelen kan være i chroot, mens databasesystem er udenfor -
forbindelser via TCP-sockets til localhost
\end{list2}


\slide{Produktionsmodning af miljøer}

\begin{list1}
\item Tænk på det miljø som servere og services skal udsættes for
\item Sørg for hærdning
\end{list1}

\slide{BIND sikring}
\begin{list1}
  \item Nedenstående kan bruges mod andre typer servere!
\item Sikringsforanstaltninger:
  \begin{list2}
  \item Opdateret software - ingen kendte sikkerhedshuller eller
  sårbarheder
\item fjern {\bfseries single points of failure} - er man afhængig af
  en ressource skal man ofte have en backup mulighed, redundant strøm
  eller lignende
\item adskilte servere - interne og eksterne til forskellige formål\\
Eksempelvis den interne postserver hvor alle e-mail opbevares og en
DMZ-postserver hvor ekstern post opbevares kortvarigt
\item lav filtre på netværket, eller på data - firewalls og proxy
  funktioner 
\item begræns adgangen til at læse information
\item begræns adgangen til at skrive information - dynamic updates på
  BIND, men samme princip til webløsninger og opdatering af databaser
\item {\bfseries least privileges} - sørg for at programmer og brugere
  kun har de nødvendige rettigheder til at kunne udføre opgaver
\item følg med på områderne der har relevans for virksomheden og
  \emph{jeres} installation - Windows, UNIX, BIND, Oracle, ...
  \end{list2}
\end{list1}

\slide{Change management}

\begin{list1}
\item Er der tilstrækkeligt med fokus på software i produktion
\item Kan en vilkårlig server nemt reetableres
\item Foretages rettelser direkte på produktionssystemer
\item Er der fall-back plan
\item Burde være god systemadministrator praksis
\end{list1}



\slide{Fundamentet skal være iorden}

\begin{list1}
\item Sørg for at den infrastruktur som I bygger på er sikker:
\begin{list2}
 \item redundans
       \item opdateret
        \item dokumenteret
        \item nem at vedligeholde
\end{list2}

\item  Husk tilgængelighed er også en sikkerhedsparameter
\end{list1}

\slide{CVS til konfigurationsfiler}

\begin{list1}
\item Det anbefales at bruge versionsstyring som eksempelvis CVS til
  konfigurationsfiler 
\item Det kan eksempelvis gøres på følgende måde:
\begin{list2}
\item \verb+mkdirhier /security6.net/cvshome /security6.net/etc+
\item \verb+export CVSROOT=/security6.net/cvshome+
\item \verb+cvs init+ - for at initialisere CVS repository
\item derefter kan man tilføje filer og lave CVS checkin og checkout
\item CVS bruger standard filrettigheder - så opret eventuelt en
  speciel gruppe til CVS brugere
\end{list2}
\item Læs mere om CVS eksempelvis på:\\
\link{http://cvsbook.red-bean.com/cvsbook.html}
\end{list1}

\slide{CVS eksempel med /etc/fstab}
\begin{alltt}
\small
# cd /security6.net/etc
# cvs import -m "initial CVS af etc" etc hlk start
# cd ..;rm -rf etc
# cvs co etc
cvs checkout: Updating etc
# cp /etc/fstab etc
# cvs add etc/fstab 
cvs add: scheduling file `fstab' for addition
cvs add: use 'cvs commit' to add this file permanently
# cvs commit -m "fstab initial"
cvs commit: Examining .
RCS file: /security6.net/cvshome/etc/fstab,v
done
Checking in fstab;
/security6.net/cvshome/etc/fstab,v  <--  fstab
initial revision: 1.1
done
\end{alltt}

\centerline{filer i /etc bør ikke flyttes, andre kan flyttes og sym-linkes}

\slide{individuel autentificering!}

\hlkimage{7cm}{images/ssh-root.pdf}

\begin{list1}
\item Mange UNIX systemer administreres fejlagtigt ved brug af
  root-login
\item Undgå direkte root-login
\item Insister på \verb+sudo+ eller \verb+su+  
\item Hvorfor?
\begin{list2}
\item Sporbarheden mistes hvis brugere logger direkte ind som root
\item Hvis et kodeord til root gættes er der direkte adgang til alt!    
\end{list2}
\end{list1}

\slide{SMTP Simple Mail Transfer Protocol}

\begin{alltt}\tiny
hlk@bigfoot:hlk$ telnet mail.kramse.dk 25
Connected to sunny.
220 sunny.kramse.dk ESMTP Postfix
HELO bigfoot
250 sunny.kramse.dk
MAIL FROM: Henrik
250 Ok
RCPT TO: hlk@kramse.dk
250 Ok
DATA
354 End data with <CR><LF>.<CR><LF>
hejsa
.
250 Ok: queued as 749193BD2
QUIT
221 Bye
\end{alltt}

\begin{list1}
\item RFC-821 SMTP Simple Mail Transfer Protocol fra 1982
\item RFC-2821 fra 2001 og flere andre er idag gældende 
\item \link{http://en.wikipedia.org/wiki/Simple_Mail_Transfer_Protocol}
\item Vedhæftede filer kodes i MIME Multipurpose Internet Mail Extensions
\item Bemærk at MIME encoding forøger størrelsen med ca. 30\%!
%\item Lad VÆRE med at sende store filer, dvs over 7-8MB via e-mail
\end{list1}

\slide{e-mail servere}

\begin{list1}
  \item Sendmail, qmail og postfix
\item Tre meget brugte e-mail systemer
  \begin{list2}
    \item Sendmail - den ældste og mest benyttede
\item Postfix en modulært og sikkerhedsmæssigt god e-mail server\\
er ligeledes nem at konfigurere
\item Qmail - en underlig mailserver lavet af Dan J Bernstein, med en
  speciel licens - ligesom programmøren
  \end{list2}
\item Dertil kommer diverse andre mailservere:
\item Microsoft Exchange på Windows servere
\item Jeg anbefaler at man har en postserver mod internet, der kun sender og modtager ekstern post, og en intern postserver der opbevarer al posten
\end{list1}


\slide{Sendmail postserveren}

\begin{alltt}\small
# "Smart" relay host (may be null)
DS
...
# strip group: syntax (not inside angle brackets!) and trailing semicolon
R$*                     $: $1 <@>                    mark addresses
R$* < $* > $* <@>       $: $1 < $2 > $3              unmark <addr>
R@ $* <@>               $: @ $1                      unmark @host:...
R$* [ IPv6 : $+ ] <@>   $: $1 [ IPv6 : $2 ]          unmark IPv6 addr
R$* :: $* <@>           $: $1 :: $2                  unmark node::addr
R:include: $* <@>       $: :include: $1              unmark :include:...
R$* : $* [ $* ]         $: $1 : $2 [ $3 ] <@>        remark if leading colon
R$* : $* <@>            $: $2                        strip colon if marked
\end{alltt}

\begin{list1}
\item mange konfigurerer Sendmail med \verb+sendmail.cf+ - det er
  {\color{red} forkert} 
\item man bør bruge M4 konfigurationen
\item  - desværre følger M4 filerne sjældent med :-(  
\item Sendmail er oprindeligt lavet af Eric Allman 
\end{list1}


\slide{Postfix postserveren}

\hlkimage{6cm}{postfix-mouse.png}

\begin{list1}
\item Lavet af Wietse Venema for IBM
\item Nem at konfigurere og sikker
\item \verb+main.cf+ findes typisk i kataloget \verb+/etc/postfix+  
\end{list1}

\slide{Audit af postservere}

\begin{list1}
\item Typisk findes konfigurationsfilerne til postservere under
  \verb+/etc+
\begin{list2}
\item \verb+/etc/mail+
\item \verb+/etc/postfix+    
\end{list2}
\item Det vigtigste er at den er opdateret og IKKE tillader relaying
\item Der findes diverse test-scripts til relaycheck på internet
\item Husk også at checke domæne records, MX og A
\end{list1}

\slide{Test af e-mail server}

\begin{alltt}
[hlk]$ {\bfseries telnet localhost 25} 
Connected.
Escape character is '^]'.
220 server ESMTP Postfix
{\bfseries helo test}
250 server
{\bfseries mail from: postmaster@pentest.dk}
250 Ok
{\bfseries rcpt to: root@pentest.dk}
250 Ok
{\bfseries data}
354 End data with <CR><LF>.<CR><LF>
{\bfseries skriv en kort besked}
.
250 Ok: queued as 91AA34D18
{\bfseries quit}
\end{alltt}
%$
\exercise{ex:email-server-config}

\slide{Postservere til klienter}

\begin{list1}
\item SMTP  som vi har gennemgået er til at sende mail mellem servere
\item Når vi skal hente post sker det typisk med POP3 eller IMAP
\begin{list2}
\item POP3 Post Office Protocol version 3 RFC-1939
\item Internet Message Access Protocol (typisk IMAPv4) RFC-3501
\end{list2}
\item Forskellen mellem de to er at man typisk med POP3 henter posten, hvor man med IMAP lader den ligge på serveren
\item POP3 er bedst hvis kun en klient skal hente
\item IMAP er bedst hvis du vil tilgå din post fra flere systemer
\item Jeg bruger selv IMAPS, IMAP over SSL kryptering - idet kodeord ellers sendes i klartekst
\end{list1}


\slide{POP3 i Danmark}

\hlkimage{15cm}{images/pop3-1.pdf}

\begin{list1}
\item Man har tillid til sin ISP - der administrerer såvel net som server
\end{list1}

\slide{POP3 i Danmark - trådløst}

\hlkimage{13cm}{images/pop3-wlan.pdf}
\begin{list1}
\item Har man tillid til andre ISP'er? Alle ISP'er?
\item Deler man et netværksmedium med andre?
\item {\color{green}Brug de rigtige protokoller!}
\end{list1}

\slide{Normal WLAN brug}

\hlkimage{22cm}{images/wlan-airpwn-1.pdf}

\slide{Packet injection - airpwn}

\hlkimage{22cm}{images/wlan-airpwn-2.pdf}

\slide{Airpwn teknikker}

\begin{list1}
\item Klienten sender forespørgsel
\item Hackerens program airpwn lytter og sender så falske pakker
\item Hvordan kan det lade sig gøre?
\begin{list2}
\item Normal forespørgsel og svar på Internet tager 50ms
\item Airpwn kan svare på omkring 1ms angives det
\item Airpwn har alle informationer til rådighed      
\end{list2}
\item Airpwn på Defcon 2004 - findes på Sourceforge\\
\link{http://airpwn.sourceforge.net/}
\item NB: Airpwn som demonstreret er begrænset til TCP og ukrypterede
  forbindelser 
\end{list1}


\slide{Dovenskab er en dyd}

\hlkimage{19cm}{images/unix-dovenskab.pdf}

\slide{Distribuerede filsystemer}

\begin{list1}
\item Til lokalnetværk:
  \begin{list2}
\item Windows filesharing - tidligere et stort sikkerhedshul
\item UNIX NFS - ikke beregnet til nutidens usikre netværk
\end{list2}
\item Over Internet: 
\item AFS - Andrew filesystem\\
  \link{http://www-2.cs.cmu.edu/afs/andrew.cmu.edu/usr/shadow/www/afs.html} 
\item CODA \link{http://www.coda.cs.cmu.edu/}
\item Tænk på de forudsætninger som et program har og forventer er til
  stede! 
\end{list1}


\slide{NFS - netværksfilsystem}

\begin{alltt}
 # sample /etc/exports file
 /               master(rw) trusty(rw,no_root_squash)
 /projects       proj*.local.domain(rw)
 /usr            *.local.domain(ro) @trusted(rw)
 /home/joe       pc001(rw,all_squash,anonuid=150,anongid=100)
 /pub            (ro,insecure,all_squash)  
\end{alltt}

\begin{list2}
\item UNIX NFS er netværksfilsystemet som alle UNIX varianter understøtter
\item Adgangen styres ved brug af \verb+/etc/exports+, eksempel fra
  manualen på Red Hat
\item De fleste bruger version 3 over UDP eller TCP selvom version 4
  burde have bedre sikkerhed 
\item Adgangen gives pr IP-adresse! IP adressebaseret autentifikation
  er pr definition dårlig!
\item Pas på - det er nemt at give root-adgang til andre maskiner!
\end{list2}

\exercise{ex:unix-rpcinfo}
\exercise{ex:unix-nfsinfo}

\slide{Samba og SMB/CIFS}

\begin{list1}
\item Microsoft Server Message Block bruges til netværksdrev og netværksprint i Windows miljøer
\item Samba er en open source implementation, som altid halter bagefter MS
\item De gamle implementationer overfører password i en uheldig version, som kan knækkes 7 tegn ad gangen - hurtigt
\item Microsoft har gennem tiden opdateret protokollen
\item Idag forsøger man at gøre det til en standard som Common Internet File System Protocol CIFS
\item Man kan læse om Samba på hjemmesiden \link{http://www.samba.org/}
\end{list1}

