\documentclass[Screen16to9,17pt]{foils}
\usepackage{zencurity-slides}

\externaldocument{basic-tcpip-and-security-exercises}
\selectlanguage{english}

\begin{document}

% # Praktisk introduktion til TCP/IP, netværk med sikkerhed og IPv6 (18.11)
% En hands-on workshop med fokus på netværks-sikkerhed

% Vi bruger allesammen TCP/IP dagligt, men hvad er det for en størrelse. Det er en familie af protokoller, som bruges til internet og Internet, men hvad er forskellen så på dem?!

% Hvis du er forvirret over de mange netværksbegreber og forkortelser, så kom til dette kursus. Vi vil gennemgå med praktiske eksempler og øvelser.

% * IP adresser
% * Nødvendig hardware, og hvad er det, netkort, switch, router, firewall
% * Routing på internet
% * Konfiguration af en moderne router - via web interface
% * Basal Wi-Fi konfiguration inkl. anbefalet sikkerhed

% De vigtigste protokoller som introduceres er:
% * ARP Address Resolution Protocol
% * IP og ICMP Internet Control Message Protocol
% * TCP Transmission Control Protocol
% * UDP User Datagram Protocol
% * DHCP Dynamic Host Configuration Protocol
% * DNS Domain Name System
% *Hele kursus betragter IPv4 og IPv6 på lige fod og vil synliggøre paraleller og forskelle. På kurset udlånes en lille router til hvert hold, op til 10 hold, og øvelserne laves på fælles netværk, eget lille netværk og med egne computere. Du vil efterfølgende kunne opsætte dit eget netværk hjemme eller i en mindre virksomhed.

% Der anbefales på workshoppen en bog, men det er ikke et krav at købe denne. Practical Packet Analysis - Using Wireshark to Solve Real-World Network Problems, 3rd edition 2017, Chris Sanders ISBN: 9781593278021 - shortened PPA.

% Forudsætninger
% Hvis man vil lave opgaver er det nødvendigt med adgang til en computer. Operativsystemet er mindre vigtigt, men der skal bruges en browser. Der forventes ikke forudgående kendskab til netværk eller TCP/IP.

% Nøgleord: TCP/IP, TCP, UDP, DHCP, DNS, ICMP, TLS, Wi-Fi, WPA2, VLAN


\mytitlepage
{Networking, TCP/IP and Security for Beginners}
{PROSA Nov 2024}


\hlkprofiluk

\slide{Goals for today}

\hlkimage{6cm}{bornhack-camp-2024.jpg}

\begin{list2}
\item Introduce basic TCP/IP terminology
\item Show various network configurations with common protocols
\item Describe how you can connect a router or switch to the network
\item Describe basics of TCP/IP in 30 minutes
\item Let you get some hands on with IP protocols
\end{list2}

Photo is NWWC camp at BornHack 2024, next year 16-23rd of July 2025


\slide{Time schedule}

\hlkimage{5cm}{kame-noanime-small.png}
\begin{list2}
\item 17:00 - 18:15 Introduction and basics -- with my network

\item 18:45 - 21:00 Connect to the network, play with TCP/IP, switches and routers. Mix of presentation and exercises
\end{list2}

Note: even though I talk a lot about Unix and Linux, you can definitely run a lot of tools on Windows and Mac OS X. The basic tools are available like the built-in ones and Nmap. Command line tools are sometimes used in the slides, as they only show text where a GUI screenshot can be cluttered with a lot of information, feel free to find GUI tools and web sites with same functionality

\slide{Exercises are completely optional}

We will use a combination of your systems, my networking hardware and my systems.\\
{\color{red}\Large \bf There might be live sniffing done on traffic!\\
Don't abuse information gathered if you sniff data}

We try to mimic what you would do in your own networks during the exercises.

\begin{list2}
\item Try ping and traceroute
\item See your own IP settings
\item Borrow a USB Ethernet and connect to a switch or router
\item Borrow a router
\end{list2}

Linux is a toolbox I will use and participants are recommended to research virtual machines



\slide{Course Materials}

\begin{list2}
\item This material is in multiple parts:

\item Slide show - presentation - this file
\item Exercises - PDF which is used for this and other workshops
\item Additional resources from the internet are linked throughout
\item Wikipedia has a LOT of nice pages about IP protocols, for example:
\end{list2}

\begin{quote}\small
Transport Layer Security (TLS) is a cryptographic protocol designed to provide communications security over a computer network. The protocol is widely used in applications such as email, instant messaging, and voice over IP, but its use in securing HTTPS remains the most publicly visible.
\end{quote}
Source: \url{https://en.wikipedia.org/wiki/Transport_Layer_Security}



\slide{Prerequisites}

If you are interested in TCP/IP you are welcome

If you want to be an expert in IP and network security I recommend doing exercises

\begin{list1}
\item It is recommended to use virtual machines for the exercises
\item Network security and most internet related security work has the following requirements:
\begin{list2}
\item Network experience
\item TCP/IP principles - often in more detail than a common user
\item Programming is an advantage, for automating things
\item Some Linux and Unix knowledge is in my opinion a {\bf necessary skill} for infosec work\\
-- too many new tools to ignore, and lots found at sites like Github and Open Source written for Linux
\end{list2}
\end{list1}



\slide{Wifi Hardware}

If you want to do sniffing of wireless it will be an advantage to have a wireless USB network card. Make sure to play nice, and dont abuse knowledge!

\begin{list2}
\item The following are two recommended models:
\item TP-link TL-WN722N hardware version 2.0 cheap but only support 2.4GHz
\item Alfa AWUS036ACH 2.4GHz + 5GHz Dual-Band and high performing
\item Both work great in Kali Linux for our purposes, but are older models by now
\end{list2}

Unfortunately the vendors change models often enough that the above are hard to find. I recommend using your favourite search engine and research which cards work with Kali Linux and airodump-ng.

I have some available you can borrow


\slide{Book: Practical Packet Analysis (PPA)}


\hlkimage{6cm}{PracticalPacketAnalysis3E_cover.png}

\emph{Practical Packet Analysis,
Using Wireshark to Solve Real-World Network Problems}
by Chris Sanders, 3rd Edition
April 2017, 368 pp.
ISBN-13:
978-1-59327-802-1
\link{https://nostarch.com/packetanalysis3}

I recommend this book for people new to networking, it has been in HumbleBundle book bundles multiple times



\slide{Internet Today}

\hlkimage{10cm}{images/server-client.pdf}

\begin{list1}
\item Clients and servers, roots in the academic world
\item Protocols are old, some more than 20 years
\item Very few protocols where encrypted, today a lot has switched to HTTPS and TLS
\end{list1}

\slide{Internet is Open Standards!}

{\hlkbig \color{titlecolor}
We reject kings, presidents, and voting.\\
We believe in rough consensus and running code.\\
-- The IETF credo Dave Clark, 1992.}

\begin{list1}
\item Request for comments (RFC) -- a series of documents spanning decades
\item RFC, BCP, FYI, informational -- first ones from 1969!
\item Are not updated but status is changed to Obsoleted when new versions are published
\item Standards track:\\
Proposed Standard $\rightarrow$ Draft Standard $\rightarrow$ Standard
\end{list1}

\slide{Internetworking: history}

\begin{list2}
\item[1961]  L. Kleinrock, MIT packet-switching theory
\item[1962]  J. C. R. Licklider, MIT - notes
\item[1964]  Paul Baran: On Distributed Communications
\item[1969]  ARPANET 4 nodes
\item[1971]  14 nodes
\item[1973]  Design of Internet Protocols started
\item[1973]  Email is about 75\% of all ARPANET traffic
\item[1974]  TCP/IP: Cerf/Kahn: A protocol for Packet
        Network Interconnection
\item[1983]  EUUG $\rightarrow$ DKUUG/DIKU Denmark
\item[1988]  About 60.000 systems on the internet -
        The Morris Worm hits about 10\%
\item[2002] About 130 million Internet hosts
\item[2010] IANA reserved blocks 7\% (Maj 2010) - \link{http://www.potaroo.net/tools/ipv4/}
\end{list2}

\slide{Internet historically set -  anno 1969}
\hlkimage{6cm}{1969_4-node_map.png}
%size 2

\begin{list2}
\item Node 1: University of California Los Angeles
\item Node 2: Stanford Research Institute
\item Node 3: University of California Santa Barbara
\item Node 4: University of Utah
%\item Kilde: \link{http://www.zakon.org/robert/internet/timeline/}
\end{list2}

\slide{What are Internet hosts }

\vskip 1 cm

\centerline{World Wide Web servers}

\hlkimage{13cm}{images/Count_WWW.png}

\begin{list1}
\item Source: Hobbes' Internet Timeline v5.6
\link{http://www.zakon.org/robert/internet/timeline/}
\end{list1}

\slide{What are Internet hosts }

\hlkimage{15cm}{potaroo-ipv4-address-report.png}
\centerline{Cumulative RIR address assignments, per RIR}

\begin{list1}
\item Source:
IPv4 Address Report \link{http://www.potaroo.net/tools/ipv4/}
\end{list1}

\slide{What is the Internet}

\begin{list1}
\item Communication between humans - currently!
\item Based on TCP/IP
\begin{list2}
\item best effort
\item packet switching (IPv6 calls it packets, not datagram)
\item \emph{connection-oriented} TCP
\item \emph{connection-less} UDP
\end{list2}
\end{list1}

RFC-1958:
\begin{quote}
 A good analogy for the development of the Internet is that of
 constantly renewing the individual streets and buildings of a city,
 rather than razing the city and rebuilding it. The architectural
 principles therefore aim to provide a framework for creating
 cooperation and standards, as a small "spanning set" of rules that
 generates a large, varied and evolving space of technology.
\end{quote}



\slide{Common Address Space}

\vskip 2 cm
\hlkimage{13cm}{IP-address.pdf}

\begin{list2}
\item Internet is defined by the address space
\item IPv4 based on 32-bit addresses, example dotted decimal format 10.0.0.1
\item IPv6 very similar to IPv4 without NAT, 128-bit addresses in hex ::1, 2a06:d380:0:101::80
\end{list2}


\slide{How to use the Internet Protocols (IP)}

Names are used by humans
\begin{center}\hlkbig
www.kramse.org

hlk@kramse.org
\end{center}

Computers use the addresses

\begin{alltt}
www     IN      A       185.129.63.130
        IN      AAAA    2a06:d380:0:102::80
mail    IN      A       217.157.63.115
        IN      AAAA    2a06:d380:0:102::25
\end{alltt}




\slide{IPv4 address}

\begin{alltt}
hlk@bigfoot:hlk$ ipconvert.pl 127.0.0.1
Adressen er: 127.0.0.1
Adressen er: 2130706433
hlk@bigfoot:hlk$ ping 2130706433
PING 2130706433 (127.0.0.1): 56 data bytes
64 bytes from 127.0.0.1: icmp_seq=0 ttl=64 time=0.135 ms
64 bytes from 127.0.0.1: icmp_seq=1 ttl=64 time=0.144 ms
\end{alltt}

\begin{list1}
\item IP-adresser typically written as decimal numbers with dots
\item {\bf dot notation}: 10.1.2.3
\end{list1}

\slide{IP-adresser as bits}

\begin{alltt}
IP-adresse: 127.0.0.1
Heltal:	2130706433
Binary:	1111111000000000000000000000001
\end{alltt}

\begin{list1}
\item IP-address converted to bits
\item Computers use bits
\end{list1}

\slide{Internet ABC}

\begin{list1}
\item Previously we used classes: A, B, C, D og E
\item This proved to be a bit inflexible:
\begin{list2}
\item A-klasse has 16 million hosts
\item B-klasse about 65.000 hosts
\item C-klasse only 250 hosts
\end{list2}
\item Most people asked for B-klasser - starting to run out!
\item D-klasse used for multicast
\item E-klasse reserved
\item See \link{http://en.wikipedia.org/wiki/Classful\_network}
\end{list1}

\vskip 5mm
\centerline{\bf Stop saying C, say /24}


\slide{RFC-1918 Private Networks}

\begin{list1}
\item Der findes et antal adresserum som alle må benytte frit:
\begin{list2}
\item 10.0.0.0    -  10.255.255.255  (10/8 prefix)
\item 172.16.0.0  -  172.31.255.255  (172.16/12 prefix)
\item 192.168.0.0 -  192.168.255.255 (192.168/16 prefix)
\end{list2}
\item Address Allocation for Private Internets RFC-1918 adresserne!
\item NB: man må ikke sende pakker ud på internet med disse som afsender, giver ikke mening
\end{list1}

\begin{alltt}
The blocks 192.0.2.0/24 (TEST-NET-1), 198.51.100.0/24 (TEST-NET-2),
and 203.0.113.0/24 (TEST-NET-3) are provided for use in
documentation.

169.254.0.0/16 has been ear-marked as the IP range to use for end node
auto-configuration when a DHCP server may not be found
\end{alltt}


\slide{Documentation Prefix, IPv6 updates etc.}

Even documentation has its own prefix, RFC5737:
\begin{alltt}
The blocks 192.0.2.0/24 (TEST-NET-1), 198.51.100.0/24 (TEST-NET-2),
and 203.0.113.0/24 (TEST-NET-3) are provided for use in
documentation.
\end{alltt}

IPv6 listed in RFC3849 \verb+2001:DB8::/32+

See RFC3330 \emph{Special-Use IPv4 Addresses} which is updated by
RFC6890 \emph{Special-Purpose IP Address Registries} which in turn is updated by RFC8190\\
Use the web version of RFCs to surf back and forth \link{https://www.rfc-editor.org/rfc/rfc8190}


\slide{CIDR Classless Inter-Domain Routing}

\hlkimage{15cm}{CIDR-aggregation.pdf}

\begin{list2}
\item Subnet mask originally inferred by the class
\item Started to allocate multiple C-class networks - save remaining B-class\\
Resulted in routing table explosion - btw Stop using A, B, C
\item A subnet mask today is a row of 1-bit
\item Supernet, supernetting
\item 10.0.0.0/24 means the network 10.0.0.0 with 24 subnet bits (mask 255.255.255.0)
\item 2a06:d380:0:101::80/64 means the network with 64-bit prefix length
\end{list2}



\slide{Protocols: OSI and Internet models}

\hlkimage{11cm,angle=90}{images/compare-osi-ip.pdf}


\slide{Ethernet, cables}

\hlkimage{17cm}{ethernetLights.jpg}

\centerline{Show link, and activity -- blinkenlights}



\slide{MAC address}
%\hlkimage{10cm}{apple-oui.png}

\begin{alltt}
00-03-93   (hex)        Apple Computer, Inc.
000393     (base 16)    Apple Computer, Inc.
                        20650 Valley Green Dr.
                        Cupertino CA 95014
                        UNITED STATES
\end{alltt}
\begin{list1}
\item Network technologies use a layer 2 hardware address
\item Typically using 48-bit MAC addresses known from Ethernet MAC-48/EUI-48
\item First half is assigned to companies -- Organizationally Unique Identifier (OUI)
\item Using the OUI you can see which producer and roughly when a network chip was produced
\item \link{http://standards.ieee.org/regauth/oui/index.shtml}
\end{list1}


\slide{Bridges}

\begin{list1}
\item Ethernet is a broadcast technology data transmitted into the ether -- a cable
\item This limits how many devices to connect
\item Using bridges we can connect segments -- which copy between them if needed
\item It learns the devices on each side (MAC address)
\end{list1}

See also \link{http://en.wikipedia.org/wiki/ALOHAnet}


\slide{A switch}

\hlkimage{10cm}{switch-1.pdf}

\begin{list1}
\item Today we use switches, Don't buy a hub, not even for experimenting or sniffing
\item A switch can receive and send data on multiple ports at the same time
\item Performance only limited by the backplane and switching chips
\item Can also often route with the same speed and mirror packets
\end{list1}


\slide{Wireless }

\hlkimage{10cm}{WCG200v2_med.jpg}

\begin{list1}
\item A typical home router would have built-in 802.11 Access-Point (AP) and some Ethernet LAN ports
\end{list1}

\slide{A modern router}

\hlkimage{7cm}{sft1200_1.png}

\begin{list2}
\item Opal (GL-SFT1200) is a pocket-sized travel router supporting 1200Mbps wireless,
Max. 300 Mbps (2.4GHz) + 867 Mbps (5GHz) Fast Wi-Fi Speeds, Powerful CPU with 3 x Gigabit Ports,
Excellent Security with VPN -- OpenVPN \& WireGuard, IPv6
\end{list2}
Source: \url{https://www.gl-inet.com/products/gl-sft1200/}


\slide{Topologier og Spanning Tree Protocol}

\hlkimage{13cm}{switch-STP.pdf}

Se mere i bogen af Radia Perlman, \emph{Interconnections: Bridges, Routers, Switches, and Internetworking Protocols}

\slide{Core, Distribution og Access net}

\hlkimage{17cm}{core-dist.pdf}

\centerline{Det er ikke altid man har præcis denne opdeling, men den er ofte brugt}

\slide{Bridges and routers}

\hlkimage{17cm}{wan-network.pdf}


\slide{Packets across the wire or wireless}

\hlkimage{20cm}{ethernet-frame-1.pdf}
\begin{list1}
\item Looking at data as a stream the packets are a pattern laid on top
\item Network technology defines the start and end of a frame, example Ethernet
\item From a lower level we receive a packet, example 1500-bytes from Ethernet driver
\item Operating system masks a lot of complexity
\end{list1}





\slide{Windows - ipconfig}

\hlkimage{14cm}{win-ipconfig-ipv6.png}

\slide{Windows -- control panel with DHCP}
\hlkimage{10cm}{win-control-panel-ipv6.png}

DHCP is responsible for giving you a dynamic address



\slide{DHCP Dynamic Host Configuration Protocol}

\hlkimage{10cm}{dhcp-1.pdf}

\begin{list1}
\item How does a system get the address?
\item Typically in  IPv4 they will get them from DHCP
\item System sends a DHCP request, server allocates an address \emph{lease}
\item IPv6 typically uses Router Advertisement -- get prefix, configure their own IPv6 address
\end{list1}



\slide{Unix - practical examples ifconfig and ping}

\begin{alltt}\small
$ ifconfig en0
en0: flags=8863<UP,BROADCAST,SMART,RUNNING,SIMPLEX,MULTICAST> mtu 1500
        inet6 {\bf fe80::216:cbff:feac:1d9f%en0} prefixlen 64 scopeid 0x4
        inet {\bf 10.0.42.15} netmask 0xffffff00 broadcast 10.0.42.255
        inet6 {\bf 2001:16d8:dd0f:cf0f:216:cbff:feac:1d9f} prefixlen 64 autoconf
        ether 00:16:cb:ac:1d:9f
        media: autoselect (1000baseT <full-duplex>) status: active

$ ping6 ::1
PING6(56=40+8+8 bytes) ::1 --> ::1
16 bytes from ::1, icmp_seq=0 hlim=64 time=0.089 ms
16 bytes from ::1, icmp_seq=1 hlim=64 time=0.155 ms

$ traceroute6 2001:16d8:dd0f:cf0f::1
traceroute6 to 2001:16d8:dd0f:cf0f::1 (2001:16d8:dd0f:cf0f::1)
from 2001:16d8:dd0f:cf0f:216:cbff:feac:1d9f, 64 hops max, 12 byte packets
 1  2001:16d8:dd0f:cf0f::1  0.399 ms  0.371 ms  0.294 ms
\end{alltt}


\slide{The basic tools for countering threats}

{\Large Knowledge and insight}
\begin{list2}
\item Networks have end-points and conversations on multiple layers
\item Wireshark is advanced, try right-clicking different places
\item Name resolution includes low level MAC addresses, and IP - names
\end{list2}

\begin{list2}
\item Tcpdump format, built-in to many network devices
\item Remote packet dumps, like \verb+tcpdump –i eth0 –w packets.pcap+
\item Story: tcpdump was originally written in 1988 by Van Jacobson, Sally Floyd, Vern Paxson and Steven McCanne who were, at the time, working in the Lawrence Berkeley Laboratory Network Research Group\\
 \link{https://en.wikipedia.org/wiki/Tcpdump}
\end{list2}

\vskip 5mm

\centerline{\Large Great network security comes from knowing networks!}



\slide{Network Knowledge Needed}

To work with network security the following protocols are the bare minimum to know about.

\begin{list2}
\item ARP Address Resolution Protocol for IPv4
\item NDP Neighbor Discovery Protocol for IPv6
\item IPv4 \& IPv6 -- the basic packet fields source, destination,
\item ICMPv4 \& ICMPv6 Internet Control Message Protocol
\item UDP User Datagram Protocol
\item TCP Transmission Control Protocol
\item DHCP Dynamic Host Configuration Protocol
\item DNS Domain Name System
\end{list2}

\centerline{A little Linux knowledge is also {\bf highly recommended}}


\slide{IPv4 header - RFC-791 september 1981}

\begin{alltt}\footnotesize
    0                   1                   2                   3
    0 1 2 3 4 5 6 7 8 9 0 1 2 3 4 5 6 7 8 9 0 1 2 3 4 5 6 7 8 9 0 1
   +-+-+-+-+-+-+-+-+-+-+-+-+-+-+-+-+-+-+-+-+-+-+-+-+-+-+-+-+-+-+-+-+
   |Version|  IHL  |Type of Service|          Total Length         |
   +-+-+-+-+-+-+-+-+-+-+-+-+-+-+-+-+-+-+-+-+-+-+-+-+-+-+-+-+-+-+-+-+
   |         Identification        |Flags|      Fragment Offset    |
   +-+-+-+-+-+-+-+-+-+-+-+-+-+-+-+-+-+-+-+-+-+-+-+-+-+-+-+-+-+-+-+-+
   |  Time to Live |    Protocol   |         Header Checksum       |
   +-+-+-+-+-+-+-+-+-+-+-+-+-+-+-+-+-+-+-+-+-+-+-+-+-+-+-+-+-+-+-+-+
   |                       Source Address                          |
   +-+-+-+-+-+-+-+-+-+-+-+-+-+-+-+-+-+-+-+-+-+-+-+-+-+-+-+-+-+-+-+-+
   |                    Destination Address                        |
   +-+-+-+-+-+-+-+-+-+-+-+-+-+-+-+-+-+-+-+-+-+-+-+-+-+-+-+-+-+-+-+-+
   |                    Options                    |    Padding    |
   +-+-+-+-+-+-+-+-+-+-+-+-+-+-+-+-+-+-+-+-+-+-+-+-+-+-+-+-+-+-+-+-+

                    Example Internet Datagram Header
\end{alltt}
Source: \url{https://datatracker.ietf.org/doc/html/rfc791} and updated later

\slide{IPv6 header - RFC-1883 December 1995}
\vskip -1cm
\begin{alltt}\footnotesize
   +-+-+-+-+-+-+-+-+-+-+-+-+-+-+-+-+-+-+-+-+-+-+-+-+-+-+-+-+-+-+-+-+
   |Version| Traffic Class |           Flow Label                  |
   +-+-+-+-+-+-+-+-+-+-+-+-+-+-+-+-+-+-+-+-+-+-+-+-+-+-+-+-+-+-+-+-+
   |         Payload Length        |  Next Header  |   Hop Limit   |
   +-+-+-+-+-+-+-+-+-+-+-+-+-+-+-+-+-+-+-+-+-+-+-+-+-+-+-+-+-+-+-+-+
   |                                                               |
   +                                                               +
   |                                                               |
   +                         Source Address                        +
   |                                                               |
   +                                                               +
   |                                                               |
   +-+-+-+-+-+-+-+-+-+-+-+-+-+-+-+-+-+-+-+-+-+-+-+-+-+-+-+-+-+-+-+-+
   |                                                               |
   +                                                               +
   |                                                               |
   +                      Destination Address                      +
   |                                                               |
   +                                                               +
   |                                                               |
   +-+-+-+-+-+-+-+-+-+-+-+-+-+-+-+-+-+-+-+-+-+-+-+-+-+-+-+-+-+-+-+-+
\end{alltt}
%Source: \url{https://datatracker.ietf.org/doc/html/rfc1883} and obsoleted by\\
%\url{https://datatracker.ietf.org/doc/html/rfc8200}

\slide{UDP User Datagram Protocol}
\hlkimage{16cm}{udp-1.pdf}
Connectionless \url{https://en.wikipedia.org/wiki/User_Datagram_Protocol}


\slide{TCP Transmission Control Protocol}
\hlkimage{14cm}{tcp-1.pdf}

Connection-oriented \url{https://en.wikipedia.org/wiki/Transmission_Control_Protocol}


\slide{TCP three way handshake}

\hlkimage{6cm}{images/tcp-three-way.pdf}

\begin{list2}
\item Session setup is used in some protocols
\item Other protocols like HTTP/2 can perform request in the first packet
\end{list2}


\slide{Well-Known Port Numbers}

\hlkimage{6cm}{iana1.jpg}

\begin{list1}
\item IANA maintains a list of magical numbers in TCP/IP
\item Lists of protocl numbers, port numers etc.
\item A few notable examples:
\begin{list2}
\item Port 25/tcp Simple Mail Transfer Protocol (SMTP)
\item Port 53/udp and 53/tcp Domain Name System (DNS)
\item Port 80/tcp Hyper Text Transfer Protocol (HTTP)
\item Port 443/tcp HTTP over TLS/SSL (HTTPS)
\end{list2}
\item Source: \link{http://www.iana.org}
\end{list1}


\slide{Whois -- Where do IP addresses come from}

\begin{list1}
\item A these magical numbers we use on the internet are administered by IANA \url{https://www.iana.org/}
\item They have handed out portions to the Region Internet Registries (RIR)
\begin{list2}
\item RIPE (Réseaux IP Européens)  \link{http://ripe.net}
\item ARIN American Registry for Internet Numbers \link{http://www.arin.net}
\item Asia Pacific Network Information Center \link{http://www.apnic.net}
\item LACNIC (Regional Latin-American and Caribbean IP Address Registry) - Latin America and some Caribbean Islands
\end{list2}
\item AFRINIC \url{https://afrinic.net/}
\item They are memberbased, and members are called Local Internet Registries (LIRs) og National Internet Registry (NIR)
\end{list1}



\slide{Hosting and internet providers}

\hlkimage{16cm}{network-bgp-asn.png}

\begin{list2}
\item BGP networks are used for all of the Internet
\item New standards like Resource Public Key Infrastructure (RPKI) are underway
\item Try RIPE BGPlay \url{https://stat.ripe.net/special/bgplay#bgplay_fetch.resource=185.129.60.1}
\end{list2}


\slide{DNS root -- the server addresses }

\begin{alltt}\small
a.root-servers.net 198.41.0.4, 2001:503:ba3e::2:30 Verisign, Inc.
b.root-servers.net 170.247.170.2, 2801:1b8:10::b University of Southern California,
Information Sciences Institute
c.root-servers.net 192.33.4.12, 2001:500:2::c Cogent Communications
d.root-servers.net 199.7.91.13, 2001:500:2d::d University of Maryland
e.root-servers.net 192.203.230.10, 2001:500:a8::e NASA (Ames Research Center)
f.root-servers.net 192.5.5.241, 2001:500:2f::f Internet Systems Consortium, Inc.
g.root-servers.net 192.112.36.4, 2001:500:12::d0d US Department of Defense (NIC)
h.root-servers.net 198.97.190.53, 2001:500:1::53 US Army (Research Lab)
i.root-servers.net 192.36.148.17, 2001:7fe::53 Netnod
j.root-servers.net 192.58.128.30, 2001:503:c27::2:30 Verisign, Inc.
k.root-servers.net 193.0.14.129, 2001:7fd::1 RIPE NCC
l.root-servers.net 199.7.83.42, 2001:500:9f::42 ICANN
m.root-servers.net 202.12.27.33, 2001:dc3::35 WIDE Project
\end{alltt}
Source: \url{https://www.iana.org/domains/root/servers}, see also \url{https://en.wikipedia.org/wiki/Root_name_server}


\slide{DNS root servers}

\hlkimage{20cm}{root-servers.png}

\link{http://root-servers.org/}



\slide{Metadata -- enrichment}

\hlkimage{10cm}{crafting-security-playbook-metadata.png}

Source: picture from Crafting the InfoSec Playbook, CIP

Metadata + Context



\slide{ICMP and Ping}

\begin{list1}
\item Internet Control Message Protocol (ICMP)
\item Error conditions are signalled using this
\item The ping program sends ICMP ECHO request and
 expect ICMP ECHO reply
\item
\begin{alltt}
\small {\bfseries
$ ping 192.168.1.1}
PING 192.168.1.1 (192.168.1.1): 56 data bytes
64 bytes from 192.168.1.1: icmp_seq=0 ttl=150 time=8.849 ms
64 bytes from 192.168.1.1: icmp_seq=1 ttl=150 time=0.588 ms
64 bytes from 192.168.1.1: icmp_seq=2 ttl=150 time=0.553 ms
\end{alltt}
\end{list1}

\slide{traceroute}

\begin{list1}
  \item traceroute works using the Time to Live (TTL) or Hop-Count fields
\item Each router will subtract one from this, and if zero -- return ICMP message
\item Unix uses UDP and Windows usually uses ICMP Echo
\item
\begin{alltt}
\small{\bfseries
$ traceroute 185.129.60.129}
traceroute to 185.129.60.129 (185.129.60.129),
30 hops max, 40 byte packets
 1  safri (10.0.0.11)  3.577 ms  0.565 ms  0.323 ms
 2  router (185.129.60.129)  1.481 ms  1.374 ms  1.261 ms
\end{alltt}
\end{list1}


\slide{NAT Network Address Translation}
\hlkimage{16cm}{nat-1.pdf}


\vskip 2 cm
\begin{list2}
\item NAT is used for connecting private networks to the Internet
\item NAT gateway replaces source address and forwards packets
\item A quick and dirty fix that keeps messing up networks and protocols
\item The NAT router/firewall has state tables
\end{list2}


\slide{Course Network}
.
\hlkrightpic{65mm}{-1cm}{sample-network.png}

\begin{list1}
\item I have a course network with me which has the following information:
\begin{list2}
\item Router APU24D -- single board computer with OpenBSD
\item Switches Juniper and TP-Link
\item wireless access-points Unifi AP
% \item IPv4 addresses: 151.216.32.0/21 total 2048
% \item IPv6 addresses route6: 2001:678:9ec::/48
% \item Autonomous system number: AS208647 BornHack\\
\url{https://en.wikipedia.org/wiki/Autonomous_system_(Internet)}

\end{list2}
\end{list1}

\centerline{\Large You are encouraged to use the network}


\slide{ARP in IPv4}

\begin{center}
\colorbox{white}{\includegraphics[width=18cm]{images/arp-basic.pdf}}
\end{center}

%server 00:30:65:22:94:a1\\
%client 00:40:70:12:95:1c\\
%hacker 00:02:03:04:05:06\\


\slide{ARP request and reply}
\begin{list1}
\item {\bfseries ping 10.0.0.2} from server
\item ARP Address Resolution Protocol request/reply:
  \begin{list2}
  \item ARP request broadcasted on layer 2 - Who has 10.0.0.2 Tell 10.0.0.1
  \item ARP reply (from 10.0.0.2) 10.0.0.2 is at 00:40:70:12:95:1c
  \end{list2}
\item IP ICMP request/reply:
  \begin{list2}
    \item Echo (ping) request from 10.0.0.1 to 10.0.0.2
\item Echo (ping) reply from 10.0.0.2 to 10.0.0.1
\item ...
  \end{list2}
\item ARP is performed on Ethernet before IP can be transmitted
\end{list1}


\slide{ IPv6 Neighbor Discovery Protocol (NDP)}

\hlkimage{18cm}{ipv6-arp-ndp.pdf}

\begin{list1}
\item Address Resolution Protocol (ARP) is replaced
\item NDP expands on the ARP concept, similar command \verb+arp -an+ compared to \verb+ndp -an+
\item Can do some things we knew from DHCPv4 still DHCPv6 exist
\item {\bf Note ICMPv6 often need to be added to firewall rules for NDP!} {\myalert}
\end{list1}

\slide{ARP vs NDP}

So at the low level, near the hardware we have protocols connecting IP addreses with MAC addresses, Ethernet and Wi-Fi are commonly found

\begin{alltt}
\small
hlk@bigfoot:basic-ipv6-new$ arp -an
? (10.0.42.1) at{\bf 0:0:24:c8:b2:4c} on en1 [ethernet]
? (10.0.42.2) at 0:c0:b7:6c:19:b on en1 [ethernet]

hlk@bigfoot:basic-ipv6-new$ ndp -an
Neighbor                      Linklayer Address  Netif Expire    St Flgs Prbs
::1                           (incomplete)         lo0 permanent R
2001:16d8:ffd2:cf0f:21c:b3ff:fec4:e1b6 0:1c:b3:c4:e1:b6 en1 permanent R
fe80::1%lo0                   (incomplete)         lo0 permanent R
fe80::200:24ff:fec8:b24c%en1 {\bf 0:0:24:c8:b2:4c}      en1 8h54m51s  S  R
fe80::21c:b3ff:fec4:e1b6%en1  0:1c:b3:c4:e1:b6     en1 permanent R
\end{alltt}

\slide{ARP and NDP problems}

\begin{list2}
\item This mapping is used in your operating system, keep a dynamic ARP/neighbor cache -- a table
\item Switches map devices to ports -- tables
\item Routers remember your IP, so it can send responses back -- tables
\item A table has a maximum size! This can cause problems {\myalert}
\item This is all done without ANY security -- you can lie, attackers can lie
\item See ARP spoofing and a sample tool \link{https://en.wikipedia.org/wiki/ARP_spoofing} \\
and \link{https://en.wikipedia.org/wiki/DSniff}
\end{list2}


\slide{IPv6 addressing RFC-4291}

\begin{list1}
\item Addresses are always 128-bit identifiers for interfaces and sets of
   interfaces
\item Unicast:   An identifier for a {\bf single interface}.\\
A packet sent to a
               unicast address is delivered to the interface identified
               by that address.
\item Anycast:   An identifier for a {\bf set of interfaces} (typically
               belonging to different nodes).\\  A packet sent to an
               anycast address is {\bf delivered to one} of the interfaces
               identified by that address (the "nearest" one, according
               to the routing protocols' measure of distance).

\item Multicast: An identifier for a {\bf set of interfaces} (typically
               belonging to different nodes). \\ A packet sent to a
               multicast address is {\bf delivered to all interfaces
               identified by that address}.
\end{list1}

\slide{IPv6 addressing RFC-4291, cont.}

\hlkimage{22cm}{ipv6-address-1.pdf}

\begin{list1}
\item 8 times 4 hex-digits seperated by colon x:x:x:x:x:x:x:x
\item Written as ipv6-address/prefix-length CIDR notation
\item Leading zeros can be removed
\item One or more groups of 16 bits of zeros can be replaced by ::
\end{list1}


\slide{Examples:}
\begin{list2}
\item ABCD:EF01:2345:6789:ABCD:EF01:2345:6789

\item Adddress 2001:DB8:0:0:8:800:200C:417A
\item Address of loopback ::1
\item IPv6 prefix 2a02:09d0:95::1/64, subnet 2a02:09d0:0095:0000::/64
\item Address 2a02:09d0:95::1 or 2a02:09d0:0095:0000:0000:0000:0000:0001
\end{list2}



\slide{IPv6 address - special prefixes }

\begin{list2}
\item link-local unicast addresses\\
fe80::/10 generated from the interface MAC address EUI-64
\item FEC0::/10 site-local - deprecated in RFC-3879

\item 2001:0DB8::/32 NON-ROUTABLE range to be used for documentation purpose RFC-3849.

\item FC00::/7 Unique Local IPv6 Unicast Addresses RFC-4193\\
\link{http://www.simpledns.com/private-ipv6.aspx}\\
If you do not like to put public addresses on internal network - use this instead
\end{list2}



\slide{Hello neighbors}

\begin{alltt}\small
$ ping6 -w -I en1 ff02::1
PING6(72=40+8+24 bytes) fe80::223:6cff:fe9a:f52c%en1 --> ff02::1
30 bytes from fe80::223:6cff:fe9a:f52c%en1: bigfoot
36 bytes from fe80::216:cbff:feac:1d9f%en1: mike.kramse.dk.
38 bytes from fe80::200:aaff:feab:9f06%en1: xrx0000aaab9f06
34 bytes from fe80::20d:93ff:fe4d:55fe%en1: harry.local
36 bytes from fe80::200:24ff:fec8:b24c%en1: kris.kramse.dk.
31 bytes from fe80::21b:63ff:fef5:38df%en1: airport5
32 bytes from fe80::216:cbff:fec4:403a%en1: main-base
44 bytes from fe80::217:f2ff:fee4:2156%en1: Base Station Koekken
35 bytes from fe80::21e:c2ff:feac:cd17%en1: arnold.local
\end{alltt}


\slide{IPv6 autoconfiguration}

\hlkimage{24cm}{modified-eui64.pdf}

\begin{list1}
\item DHCPv6 is available, but {\bfseries stateless autoconfiguration} is king
\item Routers announce subnet prefix via {\bfseries router advertisements}
\item Individual nodes then combine this with their EUI64 identifier
\end{list1}

%\link{http://www.cisco.com/web/about/ac123/ac147/archived_issues/ipj_7-2/ipv6_autoconfig.html}

\slide{Router Advertisement ICMPv6}

\hlkimage{10cm}{ipv6-router-advertisement.png}


\exercise{ex:ping}
\exercise{ex:basic-dns-lookup}
\exercise{ex:whois}
\exercise{ex:ip-address-research}

\slide{Defense in depth}

%\hlkimage{10cm}{Bartizan.png}
\hlkimage{15cm}{medieval-clipart-5}
\centerline{Picture originally from: \url{http://karenswhimsy.com/public-domain-images}}


\slide{Intrusion Kill Chains}

\hlkimage{13cm}{crafting-cip-kill-chain.png}

\begin{list2}
\item See also \emph{Intelligence-Driven Computer Network Defense Informed by Analysis of Adversary Campaigns and Intrusion Kill Chains}, Eric M. Hutchins , Michael J. Cloppert, Rohan M. Amin, Ph.D. Lockheed Martin Corporation\\{\footnotesize
 \link{https://www.lockheedmartin.com/content/dam/lockheed-martin/rms/documents/cyber/LM-White-Paper-Intel-Driven-Defense.pdf}}
\end{list2}


\slide{The Spamhaus Don't Route Or Peer Lists}

\begin{quote}
The Spamhaus Don't Route Or Peer Lists

DROP (Don't Route Or Peer) and EDROP are advisory "drop all traffic" lists, consisting of stolen 'hijacked' netblocks and netblocks controlled entirely by criminals and professional spammers. DROP and EDROP are a tiny subset of the SBL designed for use by firewalls and routing equipment.
\end{quote}

\link{http://www.spamhaus.org/drop/}


\begin{list2}
\item When your SIEM alerts you, you need tools to block and restrict
\item Recommend adding empty blocking access control lists etc. to your network infrastructure
\item Add premade blocking to your name servers, proxy servers, recursive servers
\item Recommend implementing country lists
\end{list2}



\slide{Network Segmentation -- Firewalls}

\begin{quote}\small
\$ firewall\\

1. (I) {\bf An internetwork gateway that restricts data communication traffic to and from one of the connected networks} (the one said to be "inside" the firewall) and thus protects that network's system resources against threats from the other network (the one that is said to be "outside" the firewall). (See: guard, security gateway.)

2. (O) {\bf A device or system that controls the flow of traffic between networks using differing security postures.} Wack, J. et al (NIST), "Guidelines on Firewalls and Firewall Policy", Special Publication 800-41, January 2002.

Tutorial: A firewall typically protects a smaller, secure network (such as a corporate LAN, or even just one host) from a larger network (such as the Internet). The firewall is installed at the point where the networks connect, and the firewall applies policy rules to control traffic that flows in and out of the protected network.
\end{quote}
{\footnotesize Source: RFC4949 \emph{Internet Security Glossary, Version 2}\\
\link{https://datatracker.ietf.org/doc/html/rfc4949} 2007}

\slide{Continued}
\begin{quote}\small
{\bf A firewall is not always a single computer.} For example, a firewall may consist of a pair of filtering routers and one or more proxy servers running on one or more bastion hosts, all connected to a small, dedicated LAN (see: buffer zone) between the two routers. The external router blocks attacks that use IP to break security (IP address spoofing, source routing, packet fragments), while proxy servers block attacks that would exploit a vulnerability in a higher-layer protocol or service. The internal router blocks traffic from leaving the protected network except through the proxy servers. The difficult part is defining criteria by which packets are denied passage through the firewall, because a firewall not only needs to keep unauthorized traffic (i.e., intruders) out, but usually also needs to let authorized traffic pass both in and out.
\end{quote}
{\footnotesize Source: RFC4949 \emph{Internet Security Glossary, Version 2}\\
\link{https://datatracker.ietf.org/doc/html/rfc4949} 2007}

\begin{list2}
\item Network layer, packet filters, application level, stateless, stateful
\item Firewalls are by design a choke point, natural place \\
to do network security monitoring!
\item Older but still interesting Cheswick chapter 2 PDF
\emph{A Security Review of Protocols:
Lower Layers}\\
\link{http://www.wilyhacker.com/}
\end{list2}






\slide{Modern Firewall Infrastructures}


\centerline{\hlkbig A firewall {\color{security6blue}blocks traffic} on a network}

\vskip 1 cm
\pause

\centerline{\hlkbig A firewall {\color{red}allows traffic} on a network}
{\small The interesting part is typically what it allows!}

\begin{list1}
\item A firewall infrastructure must:
\begin{list2}
\item Prevent attackers from entering
\item Prevent data exfiltration
\item Prevent worms, malware, virus from spreading in networks
\item Be part of an overall solution with ISP, routers, other firewalls, switched infrastructures,\\
  intrusion detection systems and the rest of the infrastructure
\item ...
\end{list2}
\end{list1}

\vskip 5mm
\centerline{Difficult -- and requires design and secure operations}



\slide{Open source based firewalls}
\begin{list2}
\item Linux firewalls IP tables, use command line tool ufw Uncomplicated Firewall!
\item Firewall GUIs on top of Linux -- lots! Some are also available as commercial ones
\item OpenBSD PF
\link{http://www.openbsd.org}
\item FreeBSD IPFW og IPFW2 \link{http://www.freebsd.org}
\item Mac OS X uses OpenBSD PF
\item FreeBSD has an older version of the OpenBSD PF, should really be renamed now
\end{list2}



\slide{Uncomplicated Firewall (UFW)}

\begin{alltt}\small
root@debian01:~# apt install ufw
...
root@debian01:~# ufw allow 22/tcp
Rules updated
Rules updated (v6)
root@debian01:~# ufw enable
Command may disrupt existing ssh connections. Proceed with operation (y|n)? y
Firewall is active and enabled on system startup
root@debian01:~# ufw status numbered
Status: active

     To                         Action      From
     --                         ------      ----
[ 1] 22/tcp                     ALLOW IN    Anywhere
[ 2] 22/tcp (v6)                ALLOW IN    Anywhere (v6)
\end{alltt}

\begin{list2}
\item Extremely easy to use -- I recommend and use the (Uncomplicated Firewall) UFW
\end{list2}



\slide{Firewalls are NOT Alone}

\hlkimage{15cm}{network-layers-1.png}

\centerline{Use Defense in Depth -- all layers have features}


\slide{Together with Firewalls - Virtual LAN (VLAN)}

\hlkimage{8cm}{vlan-portbased.pdf}

\begin{list1}
\item Managed switches often allow splitting into zones called virtual LANs
\item Most simple version is port based
\item Like putting ports 1-4 into one LAN and remaining in another LAN
\item Packets must traverse a router or firewall to cross between VLANs
\end{list1}

\slide{Virtual LAN (VLAN) IEEE 802.1q}

\hlkimage{16cm}{vlan-8021q.pdf}

\begin{list1}
\item Using IEEE 802.1q  VLAN tagging on Ethernet frames
\item Virtual LAN, to pass from one to another, must use a router/firewall
\item Allows separation/segmentation and protects traffic from many security issues
\end{list1}

\exercise{ex:debian-firewall}
\exercise{ex:git-tutorial}


\slide{Wireshark - graphical network sniffer}

\hlkimage{13cm}{images/wireshark-http.png}

\centerline{Capture - Options, select a network interface}
\centerline{\link{http://www.wireshark.org}}


\slide{Your Privacy }

\hlkimage{18cm}{images/internet-browsing.pdf}

\begin{list2}
\item Your data travels far
\item Often crossing borders, virtually and literally
\end{list2}



\exercise{ex:wireshark-install}
\exercise{ex:nping-tcp}


\slide{Wi-Fi Introduction and Security}


\slide{Wifi standards IEEE 802.11}

\begin{list1}
\item IEEE 802.11
\begin{list2}
\item 802.11b 11Mbps
\item 802.11g 54Mbps
\item 802.11n faster
\item 802.11i Security enhancements Robust Security Network RSN
\end{list2}
\item New names soon:\\
Wi-Fi 6 to identify devices that support 802.11ax technology\\
Wi-Fi 5 to identify devices that support 802.11ac technology\\
Wi-Fi 4 to identify devices that support 802.11n technology
\end{list1}

Source: \link{http://grouper.ieee.org/groups/802/11/index.html}




\slide{Wi-Fi Labs Network cards}

\hlkimage{16cm}{TL-WN722N.png}

\begin{list1}
\item Laptop or Netbook, I typically use USB wireless cards\\
Cheap and easy -- keep using your built-in with the host OS
\item Access Points - get a small selection for testing
\item Many resources available, books, sites, blogs, etc.
\end{list1}


\slide{IEEE 802.11 Security fast forward }

\begin{quote}
{\bf In 2001}, a group from the University of California, Berkeley presented a paper describing weaknesses in the 802.11 Wired Equivalent Privacy (WEP) security mechanism defined in the original standard; they were followed by {\bf Fluhrer, Mantin, and Shamir's} paper titled "Weaknesses in the Key Scheduling Algorithm of RC4". Not long after, Adam Stubblefield and AT\&T publicly announced the first {\bf verification of the attack}. In the attack, they were able to intercept transmissions and gain unauthorized access to wireless networks.
\end{quote}
Source: \link{http://en.wikipedia.org/wiki/IEEE_802.11}

\slide{IEEE 802.11 Security fast forward }

\begin{quote}
The IEEE set up a dedicated task group to create a replacement security solution, {\bf 802.11i} (previously this work was handled as part of a broader 802.11e effort to enhance the MAC layer). The Wi-Fi Alliance announced an {\bf interim specification called Wi-Fi Protected Access (WPA)} based on a subset of the then current IEEE 802.11i draft. These started to appear in products in {\bf mid-2003}. {\bf IEEE 802.11i (also known as WPA2)} itself was ratified in {\bf June 2004}, and uses government strength encryption in the {\bf Advanced Encryption Standard AES,} instead of RC4, which was used in WEP. The modern recommended encryption for the home/consumer space is {\bf WPA2 (AES Pre-Shared Key) and for the Enterprise space is WPA2 along with a RADIUS authentication server} (or another type of authentication server) and a strong authentication method such as EAP-TLS.
\end{quote}
Source: \link{http://en.wikipedia.org/wiki/IEEE_802.11}

\slide{IEEE 802.11 Security fast forward }

\begin{quote}
In January 2005, the IEEE set up yet another task group "w" to protect management and broadcast frames, which previously were sent unsecured. Its standard was published in 2009.[24]

In {\bf December 2011}, a security flaw was revealed that affects wireless routers with the {\bf optional Wi-Fi Protected Setup (WPS)} feature. While WPS is not a part of 802.11, {\bf the flaw allows a remote attacker to recover the WPS PIN and, with it, the router's 802.11i password in a few hours}.
\end{quote}
Source: \link{http://en.wikipedia.org/wiki/IEEE_802.11}

\vskip 1cm
\centerline{WPS WTF?! Turn it off -- use WPA PSK or WPA Enterprise}


\slide{WPA3 Security}

%\hlkimage{}{}

\begin{quote}\small
{\bf WPA3}\\
In January 2018, the Wi-Fi Alliance announced WPA3 as a replacement to WPA2.[12][13] Certification began in June 2018,[14] and WPA3 support has been mandatory for devices which bear the "Wi-Fi CERTIFIED™" logo since July 2020.[11]

The new standard uses an equivalent 192-bit cryptographic strength in WPA3-Enterprise mode[15] (AES-256 in GCM mode with SHA-384 as HMAC), and still mandates the use of CCMP-128 (AES-128 in CCM mode) as the minimum encryption algorithm in WPA3-Personal mode.

The WPA3 standard also replaces the pre-shared key (PSK) exchange with Simultaneous Authentication of Equals (SAE) exchange, a method originally introduced with IEEE 802.11s, resulting in a more secure initial key exchange in personal mode[16][17] and forward secrecy.[18] The Wi-Fi Alliance also says that WPA3 will mitigate security issues posed by weak passwords and simplify the process of setting up devices with no display interface.[2][19]

Protection of management frames as specified in the IEEE 802.11w amendment is also enforced by the WPA3 specifications.
\end{quote}
Source: \link{https://en.wikipedia.org/wiki/Wi-Fi_Protected_Access}

\begin{list2}
\item Does not seem to be used much, in Denmark, yet -- but coming
\end{list2}

\slide{IS WPA3 supported?}

%\hlkimage{}{}

\begin{quote}
A wireless network adapter that supports Wi-Fi 6. To see if your PC supports it, check the documentation that came with it or check the PC manufacturer’s website.
Tip: You can also check to see if your router supports Wi-Fi 6 by opening the Command Prompt, and then typing the command netsh wlan show drivers. Look next to Radio types supported and see if it includes 802.11ax.
\end{quote}
Source: \link{https://support.microsoft.com/en-us/windows/faster-and-more-secure-wi-fi-in-windows-26177a28-38ed-1a8e-7eca-66f24dc63f09}

\begin{list2}
\item Your devices must support both WPA3 in both operating system and Wi-Fi drivers!
\item Windows 10 and 11 does, so try upgrading drivers
\item Android does since Android 10 -- pretty new still
\item Apple devices have support in recent versions
\end{list2}


\slide{WPA3 on Apple Devices}

%\hlkimage{}{}

\begin{quote}
\begin{list2}
\item iPhone 7 or later.
\item iPad 5th generation or later.
\item Apple TV 4K or later.
\item Apple Watch series 3 or later.
\item Mac computers (late 2013 or later, with 802.11ac or later)
\end{list2}
\end{quote}
Source: \link{https://support.apple.com/da-dk/guide/security/sec8a67fa93d/web}


\slide{Wi-Fi CERTIFIED WPA3™}

%\hlkimage{}{}

\begin{quote}

WPA3™ provides cutting-edge security protocols to the market. Building on the widespread success and adoption of Wi-Fi security, WPA3 adds new features to simplify Wi-Fi security, enable more robust authentication, deliver increased cryptographic strength for highly sensitive data markets, and maintain resiliency of mission critical networks. All WPA3 networks:

\begin{list2}
\item Use the latest security methods
\item Disallow outdated legacy protocols
\item Require use of Protected Management Frames (PMF)
\end{list2}
Since Wi-Fi networks differ in usage purpose and security needs, WPA3 includes additional capabilities specifically for personal and enterprise networks. Users of WPA3-Personal receive increased protections from password guessing attempts, while WPA3-Enterprise users can now take advantage of higher-grade security protocols for sensitive data networks.

WPA3 is a mandatory certification for Wi-Fi CERTIFIED™ devices.
\end{quote}
Source: \url{https://www.wi-fi.org/discover-wi-fi/security}

\slide{WPA3-Personal}

\begin{quote}
WPA3-Personal brings better protections to individual users by providing more robust password-based authentication, even when users choose passwords that fall short of typical complexity recommendations. This capability is enabled through Simultaneous Authentication of Equals (SAE). The technology is resistant to offline dictionary attacks where an adversary attempts to determine a network password by trying possible passwords without further network interaction.

\begin{list2}
\item {\bf Natural password selection}: Allows users to choose passwords that are easier to remember
\item {\bf Ease of use}: Delivers enhanced protections with no change to the way users connect to a network
\item {\bf Forward secrecy}: Protects data traffic even if a password is compromised after the data was transmitted
\end{list2}
\end{quote}

Source: \url{https://www.wi-fi.org/discover-wi-fi/security}


\slide{WPA3-Enterprise}

%\hlkimage{}{}

\begin{quote}
WPA3-Enterprise builds upon the foundation of WPA2-Enterprise with the additional requirement of using Protected Management Frames on all WPA3 connections.

\begin{list2}
\item {\bf Authentication}: multiple Extensible Authentication Protocol (EAP) methods
\item {\bf Authenticated encryption}: minimum 128-bit Advanced Encryption Standard Counter Mode with Cipher Block Chaining Message Authentication (AES-CCMP 128)
\item {\bf Key derivation and confirmation}: minimum 256-bit Hashed Message Authentication Mode (HMAC) with Secure Hash Algorithm (HMAC-SHA256)
\item {\bf Robust management frame protection}: minimum 128-bit Broadcast/Multicast Integrity Protocol Cipher-based Message Authentication Code (BIP-CMAC-128)
\end{list2}
\end{quote}

Source: \url{https://www.wi-fi.org/discover-wi-fi/security}


\slide{WPA3-Enterprise with 192-bit mode}

\begin{quote}

WPA3-Enterprise also offers an optional mode using 192-bit minimum-strength security protocols and cryptographic tools to better protect sensitive data.

\begin{list2}
\item {\bf Authentication}: Extensible Authentication Protocol – Transport Layer Security (EAP-TLS) using Elliptic Curve Diffie-Hellman (ECDH) exchange and Elliptic Curve Digital Signature Algorithm (ECDSA) using a 384-bit elliptic curve
\item {\bf Authenticated encryption}: 256-bit Galois/Counter Mode Protocol (GCMP-256)
\item {\bf Key derivation and confirmation}: 384-bit Hashed Message Authentication Mode (HMAC) with Secure Hash Algorithm (HMAC-SHA384)
\item {\bf Robust management frame protection}: 256-bit Broadcast/Multicast Integrity Protocol Galois Message Authentication Code (BIP-GMAC-256)
\end{list2}
The 192-bit security mode offered by WPA3-Enterprise ensures the right combination of cryptographic tools are used and sets a consistent baseline of security within a WPA3 network.
\end{quote}

Source: \url{https://www.wi-fi.org/discover-wi-fi/security}




\slide{2020: Dragonblood}

%\hlkimage{}{}

\begin{quote}\small
April 2019 — Modern Wi-Fi networks use WPA2 to protect transmitted data. However, because {\bf WPA2 is more than 14 years old}, the Wi-Fi Alliance recently announced the new and more secure WPA3 protocol. One of the supposed advantages of WPA3 is that, thanks to its underlying {\bf Dragonfly handshake}, it's {\bf near impossible to crack the password} of a network. Unfortunately, we found that even with WPA3, an {\bf attacker within range of a victim can still recover the password}. If the victim uses no extra protection such as HTTPS, this allows an attacker to steal sensitive information such as passwords and emails. We hope our disclosure motivates vendors to mitigate our attacks before WPA3 becomes widespread.

...

Fortunately, as a result of our research, both the Wi-Fi standard and EAP-pwd are being updated with a more secure protocol. Although this update is not backwards-compatible with current deployments of WPA3, it does prevent most of our attacks.
\end{quote}
Source: \link{https://wpa3.mathyvanhoef.com/}


\begin{list2}
\item Side-channel leaks
\item Full paper Mathy Vanhoef and Eyal Ronen. 2020. \emph{Dragonblood: Analyzing the Dragonfly Handshake of WPA3 and EAP-pwd}. In IEEE Symposium on Security \& Privacy (SP). IEEE.
 \link{https://eprint.iacr.org/2019/383}

\end{list2}



\slide{Advanced Subjects}


\begin{list2}
\item Nmap
\item Zeek
\item Suricata
\end{list2}

\slide{Basic port scanning}

\begin{list1}
\item What is a port scan
\item Testing all values possible for port number from 0/1 to 65535
\item Goal is to identify open ports, listening and vulnerable services
\item Most often TCP og UDP scan
\item TCP scanning is more realiable than UDP scanning
\item TCP handshake must respond with SYN-ACK packets
\item UDP applications respond differently -- if they even respond\\
so probes with real requests may get response, no firewall they respond withb ICMP on closed ports
\item Use the GUI program Zenmap while learning Nmap
\end{list1}

\slide{TCP three-way handshake}

\hlkimage{4cm}{images/tcp-three-way.pdf}

\begin{list2}
\item {\bfseries TCP SYN half-open} scans
\item in the old days systems would only log when a full TCP connection was setup\\
  -- so doing only half open it was a \emph{stealth}-scans
\item Today system and IDS intrusion detection can easily monitor for this
\item Sending a lot of SYN packets can create a Denial of Service -- {\bfseries SYN-flooding}
\end{list2}

\slide{Scope: select systems for testing}

\hlkimage{10cm}{overview-routing-customer-2015.png}

\begin{list2}
\item Routers in front of critical systems and networks - availability
\item Firewalls -- are traffic flows restricted
\item Mail servers -- open for relaying
\item Web servers -- remote code execution in web systems, data download
\end{list2}

\slide{Ping and port sweep}

\begin{list1}
\item Scans across the network are named sweeps
\item Ping sweeps using ICMP Ping probes
\item Port sweep trying to find a specific service, like port 80 web
\item Quite easy to see in network traffic:
\begin{list2}
\item Selecting two IP-adresser not in use
\item Should not see any traffic, but if it does, its being scanned
\item If traffic is received on both addresses, its a sweep -- if they are a bit apart it is even better, like 10.0.0.100 and 10.0.0.200
  \end{list2}

\vskip 2cm
Pro tip: a Great network intrusion detection engine (IDS), is Suricata \link{suricata-ids.org}
\end{list1}

\slide{what is Nmap today}
\begin{quote}
Nmap ("Network Mapper") is a free and open source (license) utility for network discovery and security auditing.
\end{quote}

\begin{list1}
\item Initial release September 1997; +20 years ago
\item Today a package of programs for Windows, Mac, BSD, Linux, ... source
\item Flexible, powerful, and free! Includes other tools!
\item Lets check release notes, 7.70 pt.\\
http://seclists.org/nmap-announce/2018/0
\end{list1}

Bonus info: you can help Nmap by submitting fingerprints


\slide{Nmap port sweep for web servers}

\begin{alltt}\small
root@cornerstone:~#{\bfseries  nmap -p80,443 172.29.0.0/24}

Starting Nmap 6.47 ( http://nmap.org ) at 2015-02-05 07:31 CET
Nmap scan report for 172.29.0.1
Host is up (0.00016s latency).
PORT    STATE    SERVICE
{\color{darkgreen}80/tcp  open     http}
443/tcp filtered https
MAC Address: 00:50:56:C0:00:08 (VMware)

Nmap scan report for 172.29.0.138
Host is up (0.00012s latency).
PORT    STATE  SERVICE
{\color{darkgreen}80/tcp  open   http}
443/tcp closed https
MAC Address: 00:0C:29:46:22:FB (VMware)

\end{alltt}

\slide{Nmap port sweep for SNMP port 161/UDP}

\begin{alltt}\small
root@cornerstone:~#{\bfseries nmap -sU -p 161 172.29.0.0/24}
Starting Nmap 6.47 ( http://nmap.org ) at 2015-02-05 07:30 CET
Nmap scan report for 172.29.0.1
Host is up (0.00015s latency).
PORT    STATE         SERVICE
{\color{darkgreen}161/udp open|filtered snmp}
MAC Address: 00:50:56:C0:00:08 (VMware)

Nmap scan report for 172.29.0.138
Host is up (0.00011s latency).
PORT    STATE  SERVICE
{\bf{161/udp closed snmp}}
MAC Address: 00:0C:29:46:22:FB (VMware)
...
Nmap done: 256 IP addresses (5 hosts up) scanned in 2.18 seconds
\end{alltt}

\vskip 5mm
\centerline{More reliable to use Nmap script with probes like --script=snmp-info}

\slide{Nmap Advanced OS detection}
\begin{alltt}\footnotesize
root@cornerstone:~#{\bfseries nmap -A -p80,443 172.29.0.0/24}
Starting Nmap 6.47 ( http://nmap.org ) at 2015-02-05 07:37 CET
Nmap scan report for 172.29.0.1
Host is up (0.00027s latency).
PORT    STATE    SERVICE VERSION
80/tcp  open     http    Apache httpd 2.2.26 ((Unix) DAV/2 mod_ssl/2.2.26 OpenSSL/0.9.8zc)
|_http-title: Site doesn't have a title (text/html).
443/tcp filtered https
MAC Address: 00:50:56:C0:00:08 (VMware)
Device type: media device|general purpose|phone
Running: Apple iOS 6.X|4.X|5.X, Apple Mac OS X 10.7.X|10.9.X|10.8.X
OS details: Apple iOS 6.1.3, Apple Mac OS X 10.7.0 (Lion) - 10.9.2 (Mavericks)
or iOS 4.1 - 7.1 (Darwin 10.0.0 - 14.0.0), Apple Mac OS X 10.8 - 10.8.3 (Mountain Lion)
or iOS 5.1.1 - 6.1.5 (Darwin 12.0.0 - 13.0.0)
OS and Service detection performed.
Please report any incorrect results at http://nmap.org/submit/
\end{alltt}

\begin{list2}
\item Low-level way to identify operating systems, also try/use
  \verb+nmap -A+
\item Send probes and observe responses, lookup in table of known OS and responses
\item Techniques known since at least: \emph{ICMP Usage In Scanning} Version 3.0,
  Ofir Arkin, 2001 %\link{https://web.archive.org/web/20050210093427/http://www.sys-security.com/html/projects/icmp.html} % Original side er død
\end{list2}


\exercise{ex:nmap-pingsweep}
\exercise{ex:nmap-synscan}
\exercise{ex:nmap-os}


\slide{Architecture for packet capture}

\hlkimage{4cm}{network-horiz-onion.png}
Source: picture from \link{https://docs.securityonion.net/en/2.3/introduction.html}

\begin{list2}
\item Note the terminology North-South -- from the internet into the systems
\item East-West -- horizontal traffic inside the data center
\item See also from Security Onion \url{https://docs.securityonion.net/en/2.3/architecture.html#architecture}
\end{list2}



\slide{The Zeek Network Security Monitor}

\hlkimage{13cm}{zeek-overview.png}

The Zeek Network Security Monitor is not a single tool, more of a
powerful network analysis framework. Note: the project was renamed from Bro to Zeek in Oct 2018

{\small Source \url{https://www.zeek.org/}}


\slide{Suricata IDS/IPS/NSM}

\hlkimage{6cm}{suricata.png}

\begin{quote}
Suricata is a high performance Network IDS, IPS and Network Security Monitoring engine. Open Source and owned by a community run non-profit foundation, the Open Information Security Foundation (OISF). Suricata is developed by the OISF and its supporting vendors.
\end{quote}

 \link{http://suricata-ids.org/}
 \link{http://openinfosecfoundation.org}

\centerline{I often use Suricata and Zeek together}



\myquestionspage


\end{document}
