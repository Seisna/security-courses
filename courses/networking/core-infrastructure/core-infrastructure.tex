\documentclass[Screen16to9,17pt]{foils}
\usepackage{zencurity-slides}

\externaldocument{core-infrastructure-exercises}
\selectlanguage{english}



% Original description in danish sorry
% Netværk idag er ofte blevet uoverskuelige, men kritiske for vores
% IT-brug. Foredraget indeholder beskrivelse af et baseline setup med
% Ansible opskrifter til konfiguration af et komplet logging setup med
% Suricata, Zeek og passiv DNS opsamling. Formålet er at kunne gå tilbage
% i tiden og identificere inficerede maskiner, angrebsforsøg, men også
% blot dokumentere en baseline for netværket.

% Nøgleord:
% DNS, malware detektion, IDS, Netflow, dashboards, TCP, UDP, ICMP, TLS

% Forudsætninger
% Der bliver opsat testudstyr og netværk, så det største udbytte fås ved at
% medbringe en bærbar computer. Helst med mulighed for at afvikle en virtual
% maskine, med: 30Gb ledig disk, 2 CPU (1 er minimum), 2048Mb memory (1024 er
% minimum). Software til virtualisering kan være Virtual box, VMware eller
% lignende. Det forventes, at deltagerne har dette installeret, testet og selv
% kender det.

% Basic things that we need are below
\selectlanguage{danish}

%\externaldocument{unix-audit-security-oevelser}
\externaldocument{\jobname-exercises}

\begin{document}

% Switch font to dyslexic
\rm
\selectlanguage{english}
\mytitlepage
{Core Infrastructure and BGP Intro}
{Building a production network}

\LogoOn

%\dagsplan


\slide{Goal}

%{\hlkbig En 3 dages workshop, hvor du lærer at angribe dit netværk!}
\hlkimage{2cm}{dont-panic.png}
\centerline{\color{titlecolor}\LARGE Don't Panic!}

Spend some hours setting up a production network:
\begin{list1}
\item Use VLANs, IP subnetting, routing, BGP
\item Use Automated provisioning scripts Ansible
\item Use SNMP and introduce some debugging, monitoring tools
\item Discuss how to Monitor, Mirror data and start an IDS
\end{list1}

\centerline{We try to do a lot, feel free to focus on specific parts}

\slide{Plan for today}

\begin{list2}
\item Design a robust network
\item Isolation and segmentation
\item (Routing Security) removed, see Running a Modern Network slide set
\item Switch and access security, port security
\item (Wireless security) removed
\end{list2}

Think if you could redesign your office network!

\slide{Reading Ideas}

\begin{list2}
\item Read
\item {\small \link{https://nsrc.org/workshops/2018/myren-nsrc-cndo/networking/cndo/en/presentations/Campus_Security_Overview.pdf}}
\item {\small \link{https://nsrc.org/workshops/2018/tenet-nsrc-cndo/networking/cndo/en/presentations/Campus_Operations_BCP.pdf}}
\item Download, but dont read it all\\{\small \link{https://nsrc.org/workshops/2015/apricot2015/raw-attachment/wiki/Track1Agenda/01-ISP-Network-Design.pdf}}
\end{list2}

These are good resources when you come home, and want to build networks!


\slide{Internet today}

\hlkimage{8cm}{images/server-client.pdf}

\begin{list1}
\item Clients and servers
\item Roots in academia
\item Protocols more than 20 years old
\item HTTP is becoming encrypted, but a lot other traffic is not
\end{list1}

\slide{OSI og Internet modellerne}

\hlkimage{10cm,angle=90}{images/compare-osi-ip.pdf}


\slide{Design a robust network Isolation and segmentation}

\begin{list1}
\item Hvad kan man gøre for at få bedre netværkssikkerhed?
\begin{list2}
\item Bruge switche - der skal ARP spoofes og bedre performance
\item Opdele med firewall til flere DMZ zoner for at holde
      udsatte servere adskilt fra hinanden, det interne netværk og
      Internet
\item Overvåge, læse logs og reagere på hændelser
\end{list2}
\item Husk du skal også kunne opdatere dine servere
\end{list1}

\slide{Basic Network Security Pattern Isolate in VLANs}

\hlkimage{8cm}{images/demo-netvaerk.pdf}

\begin{list1}
\item Du bør opdele dit netværk i segmenter efter trafik
\item Du bør altid holde interne og eksterne systemer adskilt!
\item Du bør isolere farlige services i jails og chroots
\item Brug port security til at sikre basale services DHCP, Spanning Tree osv.
\end{list1}


\slide{Our Networks}

We will now configure networks, using our sample switch TP-Link T1500G-10PS

\begin{list1}
\item Core network provides uplink through a switch / internet exchange
\item Each team will need:
\begin{list2}
\item A switch TP-Link T1500G-10PS L2 features - default config
\item USB Ethernet - or VLAN compatible virtualization network
\item Ethernet cables
\end{list2}
\end{list1}


\slide{Exercise in networking VLANs, Routing and RPF}


\begin{list1}
\item Each team will configure:
\begin{list2}
\item Debian VM router-on-a-stick - L3 forwarding
{\small\link{https://en.wikipedia.org/wiki/One-armed_router}}
\item Recommended to serve DHCP service, and possibly NTP etc.
\item Configure Monitoring and LibreNMS - optional
\item Reconfigure uplink from static routing to BGP - optional
\item Connect your IDS - optional, Configure port security - optional
\end{list2}
\end{list1}

Use the guides from:\\
{\small \link{https://www.tp-link.com/uk/support/download/t1500g-10ps/\#Related-Documents}}



\slide{Packet sniffing tools}

\begin{list1}
\item Tcpdump for capturing packets
\item Wireshark for dissecting packets manually with GUI
\item Zeek Network Security Monitor
\item Discuss Suricata, modern robust capable of IDS and IPS (prevention),
ntopng High-speed web-based traffic analysis and Maltrail Malicious traffic detection system \url{https://github.com/stamparm/MalTrail}
\end{list1}

\vskip 5mm
\centerline{Often a combination of tools and methods used in practice}

Full packet capture big data tools also exist


\slide{Using Wireshark}

\hlkimage{13cm}{images/wireshark-http.png}

\centerline{\link{https://www.wireshark.org}}


\slide{Kali Linux the pentest toolbox}

\hlkimage{12cm}{kali-linux.png}

\begin{list1}
\item Kali \link{http://www.kali.org/} brings together 100s of tools
\item 100.000s of videos on youtube alone, searching for kali and \$TOOL
\end{list1}

\vskip 1cm
\centerline{Use this to generate bad traffic}

\slide{Hackerlab setup}

\hlkimage{8cm}{hacklab-1.png}

\begin{list2}
\item Hardware: most modern laptops has CPU with virtualization \\
May need to enale it in BIOS
\item Software: use your favorite operating system, Windows, Mac, Linux
\item Virtualization software: VMware, Virtual box, choose your poison
\item Hackersoftware: Kali as a Virtual Machine \link{https://www.kali.org/}
\item Install sniffing VM - put into \emph{bridge mode}
\end{list2}


\slide{What happens today?}

\begin{list1}
\item Think like a network architect team member
\item Get basic tools running
\item Start rouiting and creating a network
\item Improve situation, think about how to monitor while building
\end{list1}

\centerline{Today focus on the lower parts, but user interfaces are important too}



\exercise{ex:downloadKLR}
\exercise{ex:basicVM}
\exercise{ex:wireshark-install}
\exercise{ex:wireshark-capture}

\exercise{ex:ping}
\exercise{ex:basic-dns-lookup}


\slide{Network mapping}

\hlkimage{8cm}{images/network-example.pdf}

\begin{list1}
\item Using traceroute and similar programs it is often possible to make educated guess to network topology
\item Time to live (TTL) for packets are decreased when crossing a router
\item when it reaches zero the packet is timed out, and ICMP message sent back to source
\item Default Unix traceroute uses UDP, Windows tracert use ICMP
\end{list1}


\slide{traceroute -- UDP}

\begin{alltt}
\footnotesize # {\bfseries tcpdump -i en0 host 10.20.20.129 or host 10.0.0.11}
tcpdump: listening on en0
23:23:30.426342 10.0.0.200.33849 > router.33435: udp 12 {\bf [ttl 1]}
23:23:30.426742 safri > 10.0.0.200: {\bf icmp: time exceeded in-transit}
23:23:30.436069 10.0.0.200.33849 > router.33436: udp 12 {\bf [ttl 1]}
23:23:30.436357 safri > 10.0.0.200: {\bf icmp: time exceeded in-transit}
23:23:30.437117 10.0.0.200.33849 > router.33437: udp 12 {\bf [ttl 1]}
23:23:30.437383 safri > 10.0.0.200: {\bf icmp: time exceeded in-transit}
23:23:30.437574 10.0.0.200.33849 > router.33438: udp 12
23:23:30.438946 router > 10.0.0.200: icmp: router {\bf udp port 33438 unreachable}
23:23:30.451319 10.0.0.200.33849 > router.33439: udp 12
23:23:30.452569 router > 10.0.0.200: icmp: router {\bf udp port 33439 unreachable}
23:23:30.452813 10.0.0.200.33849 > router.33440: udp 12
23:23:30.454023 router > 10.0.0.200: icmp: router {\bf udp port 33440 unreachable}
23:23:31.379102 10.0.0.200.49214 > safri.domain:  6646+ PTR?
200.0.0.10.in-addr.arpa. (41)
23:23:31.380410 safri.domain > 10.0.0.200.49214:  6646 NXDomain* 0/1/0 (93)
14 packets received by filter
0 packets dropped by kernel
\end{alltt}

\vskip 5mm
\centerline{Low TTL, UDP, high ports above 33000 = Unix traceroute signature}


\slide{Experiences gathered}

\begin{list1}
\item Lots of information
\item Reveals a lot about the network, operating systems, services etc.
\item I use a template when getting data
  \begin{list2}
    \item Respond to ICMP: $\Box$\  echo, $\Box$\ mask, $\Box$\ time
\item Respond to traceroute: $\Box$\ ICMP, $\Box$\ UDP
\item Open ports TCP og UDP:
\item Operating system:
\item ... (banner information )
  \end{list2}
\item Beware when doing scans it is possible to make routers, firewalls and devices perform badly or even crash!
\end{list1}

\slide{The Zeek Network Security Monitor}

\hlkimage{13cm}{zeek-overview.png}

The Zeek Network Security Monitor is not a single tool, more of a
powerful network analysis framework (former name Bro)

{\small Source \url{https://www.zeek.org/}, redirects to \url{https://www.zeek.org/zeek.html}}

\slide{Zeek scripts}

\begin{alltt}\small
global dns_A_reply_count=0;
global dns_AAAA_reply_count=0;
...
event dns_A_reply(c: connection, msg: dns_msg, ans: dns_answer, a: addr)
        \{
        ++dns_A_reply_count;
        \}

event dns_AAAA_reply(c: connection, msg: dns_msg, ans: dns_answer, a: addr)
        \{
        ++dns_AAAA_reply_count;
        \}
\end{alltt}

source: dns-fire-count.bro from\\
{\small \link{https://github.com/LiamRandall/bro-scripts/tree/master/fire-scripts}
\url{https://www.zeek.org/sphinx-git/script-reference/scripts.html}}


\exercise{ex:zeekweb}

\slide{Exercise setup}

We will use a combination of your virtual servers, my switch hardware and my virtual systems.

\vskip 1cm
{\Large \bf There will be sniffing done on traffic!\\
Don't abuse information gathered}

We try to mimic what you would do in your own networks during the exercises.


\slide{Get Started with Zeek}

\begin{list1}
\item To run in “base” mode:
 \verb+bro -r traffic.pcap+
\item To run in a “near broctl” mode:
\verb+bro -r traffic.pcap local+
\item To add extra scripts:
\verb+bro -r traffic.pcap myscript.bro+
\end{list1}

\centerline{Note: the project was renamed from Bro to Zeek in Oct 2018}

\exercise{ex:zeekdnsbasic}
\exercise{ex:zeektlsbasic}

\slide{DNS is important}

Another tool that provides a basic SQL-frontend to PCAP-files\\
\url{https://www.dns-oarc.net/tools/packetq}\\
\url{https://github.com/DNS-OARC/PacketQ}

Going back in time and finding systems that visited a specific domain can explain when and where an infection started.

Deciding on which tool to use, Zeek or PacketQ depends on the situation.

\slide{Suricata IDS/IPS/NSM}
\hlkimage{6cm}{suricata.png}

\begin{quote}
Suricata is a high performance Network IDS, IPS and Network Security Monitoring engine. Open Source and owned by a
 community run non-profit foundation, the Open Information Security Foundation (OISF). Suricata is developed by th
e OISF and its supporting vendors.
\end{quote}

 \link{http://suricata-ids.org/}
 \link{http://openinfosecfoundation.org}

 \exercise{ex:basicDebianVM}
 \exercise{ex:vlans-routing-rpf}

 \exercise{ex:dhcpd-config}
 \exercise{ex:dns-config}
 \exercise{ex:bgp-config}
 \exercise{ex:mirrorport}
 \exercise{ex:port-security}


\slide{Summary}

We started from a basic Ubuntu/Debian server, and we now know more about our network.


We know it is possible to create dashboards and visualizing the data.

What are the next steps?


\myquestionspage

\end{document}
