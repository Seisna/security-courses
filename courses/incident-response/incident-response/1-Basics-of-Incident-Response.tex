\documentclass[Screen16to9,17pt]{foils}
\usepackage{kea-slides}
\externaldocument{build/introduction-to-incident-response-exercises}
\selectlanguage{english}

% Input:
% https://www.threathunting.net/reading-list

% Docker image with
% https://hub.docker.com/r/threathuntproj/hunting/
% This image contains a complete threat hunting & data analysis environment built on Python, Pandas, PySpark and Jupyter notebook.

\begin{document}

\mytitlepage
{1. Basics of Incident Response}
{Introduction to Incident Response Elective, KEA}


\slide{Goals for today}

\hlkimage{6cm}{thomas-galler-hZ3uF1-z2Qc-unsplash.jpg}

\begin{list2}
\item Define Incident Response
\item Find some information -- gather data
\item Get started trying  some tools
\end{list2}

{\hfill \small Photo by Thomas Galler on Unsplash}

\slide{Plan for today}

\begin{list2}
\item Basics of Intelligence
\item Basics of Incident Response
\item Get started doing some exercises
\item Incident Response Life cycle
\end{list2}

Exercise theme:
\begin{list2}
\item IP Address Research
\item Mitre ATT\&CK Framework introduction
\item Demo: buffer overflow and kernel information
\item Nginx logging -- web server common log format
\item Packetbeat example logging system
\end{list2}

\slide{Reading Summary}

\emph{Intelligence-Driven Incident Response} (IDIR),\\
Scott Roberts, Rebekah Brown, ISBN: 9781098120689 **2nd edition 2023** - short IDIR

Current status -- try to go through these chapters:
\begin{list2}
\item Foreword and Preface
\item Chapter 1: Introduction
\item Chapter 2: Basics of Intelligence
\item Chapter 3: Basics of Incident Response
\end{list2}

\slide{Foreword}

\hlkimage{2cm}{The_Cuckoos_Egg.jpg}

\begin{quote}
{\bf Threat intelligence was vital to intrusions over 20 years ago}, starting with the story told in {\bf the Cuckoo’s Egg, written by Cliff Stoll}, and has been ever since. But somehow, most organizations are {\bf still learning} to adopt the same principles. ... Lucky for us, this book now exists and steps the reader through {\bf proper threat-intelligence concepts, strategy, and capabilities} that an organization can adopt to evolve their security practice. After reading this book, your operations can grow to become an intelligence-driven operation that is much more efficient than ever in {\bf detecting and reducing the possible impact of breaches that will occur.}
\end{quote}
Source: \emph{Intelligence-Driven Incident Response} (IDIR)\\
 Scott Roberts. Rebekah Brown

\slide{Note mentions!}

%\hlkimage{}{}

Resource and books like these mention a lot of interesting things:
\begin{list2}
\item Authors -- obviously but perhaps check other writing by them
\item Organizations: DoD, NSA, SANS \link{https://www.sans.org/}
\item Persons: Clifford Stoll and the Cuckoo's Egg (book/security breach incident)
\item Company names: Mandiant
\item Tools: Nmap, Tcpdump, Metasploit
\item Web sites
\item Incidents, attacker groups, tactics
\item Term: Cyber Threat Intelligence (CTI)
\item ...
\end{list2}

All of these will enhance your knowledge in this field, so take mental notes along the way.

A related resource is the MITRE ATT\&CK framework \link{https://attack.mitre.org/}

\slide{Preface}

%\hlkimage{}{}

\begin{quote}
The purpose of this book is to {\bf demonstrate how intelligence fits into the incident-response process}, helping responders understand their adversaries in order to {\bf reduce the time it takes to detect, respond to, and remediate intrusions.} Cyber threat intelligence and incident response have long been {\bf closely related}, and in fact are {\bf inextricably linked.} Not only does threat intelligence support and augment incident response, but incident response generates threat intelligence that can be utilized by incident responders. The goal of this book is to help readers {\bf understand, implement, and benefit} from this relationship.
\end{quote}
Source: \emph{Intelligence-Driven Incident Response} (IDIR)\\
 Scott Roberts. Rebekah Brown

\begin{list2}
\item Why, Who and how the book is organized
\end{list2}

\slide{Fundamentals Chapter 1: Introduction}


\begin{quote}
{\large\bf Intelligence as Part of Incident Response}\\
As long as there has been conflict, there have been those who studied, analyzed, and
strove to understand the enemy. Wars have been won and lost based on an ability to
understand the way the enemy thinks and operates, to comprehend their motivations
and identify their tactics ...
\end{quote}
Source: \emph{Intelligence-Driven Incident Response} (IDIR)\\
 Scott Roberts. Rebekah Brown

\begin{list2}
\item We will call it incident response for this course, this is our focus first
\item Later you may pick up the book again and pay more attention to the intelligence gathering and use
\end{list2}

\slide{What do we mean by Intelligence}

%\hlkimage{}{}

\begin{quote}
Intelligence is often defined as information that has been refined and analyzed to
make it actionable. Intelligence, therefore, requires information. In intelligence-
driven incident response, there are multiple ways to gather information that will be
analyzed and used to support incident response.
\end{quote}
Source: \emph{Intelligence-Driven Incident Response} (IDIR)\\
 Scott Roberts. Rebekah Brown


\slide{Chapter 2: Basics of Intelligence}

%\hlkimage{}{}

\begin{quote}
“Joint Publication 2-0,” the US military’s primary joint intelligence doctrine, is one of
the foundational intelligence documents used today. In its introduction, it states,
“Information on its own may be of utility to the commander, but when related to
other information about the operational environment and considered in the light of
past experience, it gives rise to a new understanding of the information, which may
be termed intelligence.”
\end{quote}
Source: \emph{Intelligence-Driven Incident Response} (IDIR)\\
 Scott Roberts. Rebekah Brown

\begin{list2}
\item \emph{Data} is a piece of information, a fact, or a statistic. Data is something that describes something that \emph{is}.
\item \emph{Intelligence} is {\bf derived from a process} of collecting, processing, and analyzing data. Once it has been analyzed, it must be disseminated in order to be useful.
\end{list2}

\slide{Indicators of Compromise }

\begin{quote}
There was a time when many people considered indicators of compromise, or IOCs,
to be synonymous with threat intelligence. IOCs, which we will reference a lot and
cover in depth later in the book, are {\bf things to look for} on a system or in {\bf network logs} that may {\bf indicate that a compromise has taken place}. This includes IP addresses and domains associated with command-and-control servers or malware downloads, hashes of malicious files, and other network- or host-based artifacts that may indicate an intrusion.
\end{quote}
Source: \emph{Intelligence-Driven Incident Response} (IDIR)\\
 Scott Roberts. Rebekah Brown



\slide{Indicators of Compromise and Signatures}

\begin{quote}
An IOC is any piece of information that can be used to objectively describe a
network intrusion, expressed in a platform-independent manner. This could include a simple indicator such as the IP address of a command and control (C2) server or a complex set of behaviors that indicate that a mail server is being used as a malicious SMTP relay.

When an IOC is taken and used in a platform-specific language or format, such as a Snort Rule or a Zeek-formatted file, it becomes part of a signature. A signature can contain one or more IOCs.
\end{quote}

Source: Applied Network Security Monitoring Collection, Detection, and Analysis, 2014 Chris Sanders


\slide{OODA Loop by John Boyd}

\hlkimage{20cm}{OODA.Boyd.png}
Source: Patrick Edwin Moran - Wikipedia \link{https://en.wikipedia.org/wiki/OODA_loop}



\slide{Intelligence Cycle or Intelligence Process}

\hlkimage{9cm}{The_Intelligence_Process_JP_2-0.png}
Source: \link{https://en.wikipedia.org/wiki/Intelligence_cycle}

\begin{list2}
\item I decided to take the more original Intelligence Process picture, which has a bit more details
\end{list2}

\slide{Processing}

Let's look at some processing

\begin{list2}
\item Processing includes normalizing collected data into uniform formats for analysis
\item Indexing -- Large volumes of data need to be made searchable
\item Translation -- for our course we might get multiple input formats but need JSON or XML
\item Enrichment -- Providing additional metadata for a piece of information is important. For example, domain addresses need to be resolved to IP addresses, and {\bf WHOIS registration data fetched}
\item Filtering --
Not all data provides equal value, and analysts can be overwhelmed when presen‐
ted with endless streams of irrelevant data
\item Prioritization --
The data that has been collected may need to be ranked so that analysts can allo‐
cate resources to the most important items\\
Note: this relates to a \emph{baseline}, what errors are normal in your environment
\item Visualization -- Data visualization has advanced significantly and the human eye and brain can often see patterns
\end{list2}


\slide{Metadata -- enrichment}

\hlkimage{10cm}{crafting-security-playbook-metadata.png}

Source: picture from Crafting the InfoSec Playbook, CIP

Metadata + Context


\exercise{ex:ip-address-research}


\exercise{ex:brim-security}


\slide{Chapter 3: Basics of Incident Response}

%\hlkimage{}{}

\begin{quote}
Incident response encompasses the entire process of {\bf identifying intrusions} (whether against a single system or an entire network), developing the information necessary to {\bf fully understand them}, and then {\bf developing and executing the plans} to {\bf remove the intruders}.
\end{quote}
Source: \emph{Intelligence-Driven Incident Response} (IDIR)\\
 Scott Roberts. Rebekah Brown


\begin{list2}
\item An important part of this is \emph{when} to activate the incident response team, what qualifies? Triage of incidents
\end{list2}


\slide{Incident Response Life cycle}

\hlkimage{18cm}{incident-response-life-cycle.png}
Source: \emph{Computer Security Incident Handling Guide}, NIST SP 800-61 Rev. 2




\slide{Preparation}

%\hlkimage{}{}

\begin{quote}
For a defender, the first stage of an incident comes before the attack begins: the Preparation phase. Preparation is the defender’s chance to get ahead of the attacker by deploying new detection systems, creating and updating signatures, and understanding baseline system and network activity.
\end{quote}

\begin{list2}
\item Telemetry -- You can’t find what you can’t see
\item Hardening -- The only thing better than identifying an intrusion quickly is it never happening in the first place
\item Process and documentation -- On the nontechnical side, process is the first line of defense that can be prepared ahead of time\\
You may see the term Standard Operating Procedure (SOP) used in literature
\item Practice -- The last thing preparation allows is the chance to practice your plans. This will speed up future incidents and identify issues that can be corrected
\end{list2}

\slide{Identification}

%\hlkimage{}{}

\begin{quote}
The Identification phase is the moment where the defender identifies the presence of
an attacker impacting their environment. This can occur though a variety of methods:
\begin{list2}
\item Identifying the attacker entering the network, such as a server attack or an
incoming phishing email
\item  Noticing command-and-control traffic from a compromised host
\item  Seeing the massive traffic spike when the attacker begins exfiltrating data
\item  Getting a visit from a special agent at your local FBI field office
\item  And last, but all too often, showing up in an article by Brian Krebs
\end{list2}
\end{quote}



\slide{Detection Capabilities}


Security incidents happen, but what happens. One of the actions to reduce impact of incidents are done in preparing for incidents.

\begin{itemize}
\item \emph{Preparation} for an attack, establish procedures and mechanisms for detecting and responding to attacks
\end{itemize}

Preparation will enable easy {\bf identification} of affected systems, better {\bf containment} which systems are likely to be infected, {\bf eradication} what happened -- how to do the {\bf eradication} and {\bf recovery}.

\slide{Data Analysis Skills}

\begin{quote}
Although we could spend an entire book creating an exhaustive list of skills needed to be a good security data scientist, this chapter covers the following skills/domains that a data scientist will benefit from
knowing within information security:
\begin{list2}
\item Domain expertise—Setting and maintaining a purpose to the analysis
\item Data management—Being able to prepare, store, and maintain data
\item Programming—The glue that connects data to analysis
\item Statistics—To learn from the data
\item Visualization—Communicating the results effectively
\end{list2}
It might be easy to label any one of these skills as the most important, but in reality, the whole is greater than the sum of its parts. Each of these contributes a significant and important piece to the workings of
security data science.
\end{quote}

Source: \emph{Data-Driven Security: Analysis, Visualization and Dashboards} Jay Jacobs, Bob Rudis\\
ISBN: 978-1-118-79372-5 February 2014 \url{https://datadrivensecurity.info/} - short DDS

\slide{The Zeek Network Security Monitor}

\hlkimage{14cm}{zeek-overview.png}

The Zeek Network Security Monitor is not a single tool, more of a
powerful network analysis framework

Zeek is the tool formerly known as Bro, changed name in 2018. \link{https://www.zeek.org/}


\slide{Zeek IDS is}

\hlkimage{14cm}{zeek-ids.png}

\begin{quote}
While focusing on network security monitoring, Zeek provides a comprehensive platform for more general network traffic analysis as well. Well grounded in more than 15 years of research, Zeek has successfully bridged the traditional gap between academia and operations since its inception.
\end{quote}

\link{https://www.Zeek.org/}


\slide{Suricata IDS/IPS/NSM}
\hlkimage{6cm}{suricata.png}

\begin{quote}
Suricata is a high performance Network IDS, IPS and Network Security Monitoring engine.
\end{quote}

 \link{http://suricata-ids.org/}
 \link{http://openinfosecfoundation.org}

In this course we will expect our organisation to have already deployed similar capabilities.

You can find a whole workshop in:\\
{\small\link{https://github.com/kramse/security-courses/tree/master/courses/networking/suricatazeek-workshop}}



\slide{Containment}

%\hlkimage{}{}

\begin{quote}
Common containment options are as follows:
\begin{list2}
\item Disabling the network switch port to which a particular system is connected
\item  Blocking access to malicious network resources such as IPs (at the firewall) and domains or specific URLs (via a network proxy)
\item  Temporarily locking a user account under the control of an intruder
\item  Disabling system services or software an adversary is exploiting
\end{list2}
\end{quote}

\slide{Eradication}

%\hlkimage{}{}

\begin{quote}
Eradication consists of the longer-term mitigation efforts meant to keep an attacker
out for good (unlike the temporary measures in the Containment phase). These
actions should be well thought out and may take a considerable amount of time and
resources to deploy. They are focused on completely obviating as many parts of the
adversary’s plan from ever working in the future.

Common eradication actions are as follows:
\begin{list2}
\item Removing all malware and tools installed by the adversary (see the sidebar “Wiping and Reloading Versus Removal” on page 32)
\item Resetting and remediating all impacted user and service accounts
\item Re-creating secrets that could have been accessed by the attacker, such as shared passwords, certificates, and tokens
\end{list2}
\end{quote}

\slide{Recovery}

%\hlkimage{}{}

\begin{quote}
Containment and eradication often require {\bf drastic action}. Recovery is the process of going back to a nonincident state. In some regards, recovery is less from the attack itself, but more from the actions taken by the incident responders.

For example, if a compromised system is taken from a user for forensic analysis, the
Recovery phase involves returning or replacing the user’s system so that user can
return to previous tasks. If an entire network is compromised, the Recovery phase
involves undoing any actions taken by the attacker across the entire network, and can
be a {\bf lengthy and involved process.}
\end{quote}

\begin{list2}
    \item
\end{list2}

\slide{Lessons Learned -- Follow-Up}

%\hlkimage{}{}

\begin{quote}
Ultimately, the key to Lessons Learned is having the understanding that although early lessons learned will be painful, they will improve—and that’s the point. Early Lessons Learned exercises will call out {\bf flaws, missing technology, missing team members, bad processes, and bad assumptions}. Growing pains with this process are common, but take the time and gut through them. {\bf Few things} will improve an IR team and IR capability as {\bf quickly} as some {\bf tough lessons learned}.
\end{quote}

\begin{list2}
\item Goal is to mature your team, methods, procedures, identification, ...
\item More efficient is cheaper, faster, better
\end{list2}

\slide{Intrusion Kill Chains}

\hlkimage{13cm}{crafting-cip-kill-chain.png}

\begin{list2}
\item See also \emph{Intelligence-Driven Computer Network Defense Informed by Analysis of Adversary Campaigns and Intrusion Kill Chains}, Eric M. Hutchins , Michael J. Cloppert, Rohan M. Amin, Ph.D. Lockheed Martin Corporation\\{\footnotesize
 \link{https://www.lockheedmartin.com/content/dam/lockheed-martin/rms/documents/cyber/LM-White-Paper-Intel-Driven-Defense.pdf}}
\end{list2}

\exercise{ex:mitre-attack}


\exercise{ex:bufferoverflow}


\slide{Python and REST}

\inputminted{python}{programs/rest-1.py}

\begin{list2}
\item  Lets try to use some Python to access a REST service.
\item  We will use the JSONPlaceholder which is a free online REST API:
\link{https://jsonplaceholder.typicode.com/}
\item Start at the site: \link{https://jsonplaceholder.typicode.com/guide.html} and try running a few of the examples with your browser
\item Then try using the same URLS in the Requests HTTP library from Python,\\
\link{https://requests.readthedocs.io/en/master/}
\end{list2}


\slide{Note about frameworks and libraries}

\begin{minted}[fontsize=\footnotesize]{python}
import xml.etree.ElementTree as ET
tree = ET.parse('testfile.xml')
root = tree.getroot()

print(root.tag)
print('Nmap version: \t\t{:s} '.format(root.attrib['version']))
print('Nmap started: \t\t{:s} '.format(root.attrib['startstr']))
print('Nmap command line: \t{:s} '.format(root.attrib['args']))

hosts = tree.findall('./host')
for host in hosts:
    print(host.tag)
    print(host.attrib)
    for hostvalues in host:
        print(hostvalues.tag)
        print(hostvalues.attrib)
\end{minted}

\begin{list2}
\item Dont import JSON or XML using home made programs
\item Example uses xml.etree.ElementTree from Python \url{https://docs.python.org/3.7/library/xml.etree.elementtree.html}
\end{list2}

\slide{Convert XML to JSON}

\begin{minted}[fontsize=\footnotesize]{python}
import xml.etree.ElementTree as ET
import json
def etree_to_dict(t):
    d = {t.tag : map(etree_to_dict, t.getchildren())}
    d.update(('@' + k, v) for k, v in t.attrib.items())
    d['text'] = t.text
    return d

tree = ET.parse('testfile.xml')
root = tree.getroot()
mydict = etree_to_dict(root)
print(type(tree))
print(type(root))
print(type(mydict))

print(mydict)

with open('testfile.json', 'w') as json_file:
  json.dump(mydict, json_file)
\end{minted}

Converting using Python is easy



\slide{Side note: Zeek Security Monitor handles formats differently}

Zeek has files formatted with a header:
\begin{alltt}\footnotesize
#fields ts      uid     id.orig_h       id.orig_p       id.resp_h       id.resp_p       proto   trans_id
        rtt     query   qclass  qclass_name     qtype   qtype_name      rcode   rcode_name      AA
        TC      RD      RA      Z       answers TTLs    rejected

1538982372.416180	CD12Dc1SpQm42QW4G3	10.xxx.0.145	57476	10.x.y.141	53	udp	20383
	0.045021	www.dr.dk	1	C_INTERNET	1	A	0	NOERROR	F	F	T	T	0
   www.dr.dk-v1.edgekey.net,e16198.b.akamaiedge.net,2.17.212.93	60.000000,20409.000000,20.000000	F
\end{alltt}

Note: this show ALL the fields captured and dissected by Zeek, there is a nice utility program zeek-cut which can select specific fields:

\begin{alltt}\small
root@NMS-VM:/var/spool/bro/bro# cat dns.log | zeek-cut -d ts query answers | grep dr.dk
2018-10-08T09:06:12+0200	www.dr.dk	www.dr.dk-v1.edgekey.net,e16198.b.akamaiedge.net,2.17.212.93
\end{alltt}

Can also just use JSON now via Filebeat


\exercise{ex:nginx-logging}

\exercise{ex:packetbeat}

\slidenext{Buy the books! Create your VMs}

\end{document}
