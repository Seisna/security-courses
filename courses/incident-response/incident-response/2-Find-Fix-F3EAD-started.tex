\documentclass[Screen16to9,17pt]{foils}
\usepackage{kea-slides}
\externaldocument{build/introduction-to-incident-response-exercises}
\selectlanguage{english}

% Input:
% https://www.threathunting.net/reading-list

% Docker image with
% https://hub.docker.com/r/threathuntproj/hunting/
% This image contains a complete threat hunting & data analysis environment built on Python, Pandas, PySpark and Jupyter notebook.

\begin{document}

\mytitlepage
{2. Find, Fix, F3EAD started}
{Introduction to Incident Response Elective, KEA}


\slide{Goals for today}

\hlkimage{6cm}{thomas-galler-hZ3uF1-z2Qc-unsplash.jpg}

\begin{list2}
\item Intelligence-driven incident-response process
\item F3EAD process: Find, Fix Finish, Exploit, Analyze, Disseminate
\end{list2}

{\hfill \small Photo by Thomas Galler on Unsplash}

\slide{Plan for today}

\begin{list2}
\item First phases of the F3EAD cycle
\item Find phase, which identifies the starting point for both
intelligence and operational activities
\item Fix phase, tracking down signs of adversary activity on your networks
\item Start using tools -- find tools for the job
\end{list2}

Exercise theme:
\begin{list2}
\item Work with a packet capture -- extract data
\item Choose tools, research tools
\end{list2}

\slide{Time schedule}

\begin{list2}
\item 1) Going over my presentation, first part of F3EAD Find -- first 45min
\item 2) Nitroba packet capture analysis -- 45 min
\item Break 15min
\item 3) Next part of F3EAD Fix -- 2x45min including 4)
\item 4) Looking more into tools -- which tools should we have used
\item At the end summary of today and questions
%\item 4) Introduction to the exam project!
\end{list2}





\slide{Reading Summary}

\emph{Intelligence-Driven Incident Response} (IDIR)
 Scott Roberts. Rebekah Brown, ISBN: 9781098120689

Part II: Practical Application

\begin{quote}
Once you understand the fundamentals, it is time to get down to business. Part 2
steps through the intelligence-driven incident-response process using the F3EAD
process: Find, Fix Finish, Exploit, Analyze, Disseminate. These steps will ensure that
you are gathering and acting on the right information in the right order to get as
much as possible from the intelligence-driven incident-response processes.
\end{quote}

\begin{list2}
\item Chapter 4: Find
\item Chapter 5: Fix
\end{list2}


\slide{Chapter 4: Find}

%\hlkimage{}{}

\begin{quote}
This chapter focuses on the Find phase, which identifies the starting point for both
intelligence and operational activities. In the traditional F3EAD cycle, the Find phase
often identifies high-value targets for special operations teams to target. In
intelligence-driven incident response, the Find phase identifies relevant adversaries
for incident response.

In the case of an ongoing incident, you may have identified or been given some initial
indicators and need to dig for more;
\end{quote}
Source: Source: \emph{Intelligence-Driven Incident Response} (IDIR)
 Scott Roberts. Rebekah Brown, ISBN: 9781098120689


\slide{Actors Versus People}

%\hlkimage{}{}

\begin{quote}
\centerline{\bf\Large Actors Versus People}

Identity is a funny thing. In many cases, when we say they or them or refer to an
adversary, it’s easy to assume we’re referring to the people behind an attack. In some,
rare cases, we are talking about the actual individuals (this is called attribution, some‐
thing we’ll discuss more in the intelligence chapters). But in most cases when we’re
referring to actors, we refer to a persona based on the tactics, techniques, and pro‐
cesses (TTPS) used together to achieve a goal. We mentally group these together and
personify them, since human beings understand stories told that way. This is an
abstraction, because we usually don’t know if it’s one person or a large group. We call
this abstraction of linked TTPs and a goal an actor, regardless of the number of people involved.
\end{quote}
Source: Source: \emph{Intelligence-Driven Incident Response} (IDIR)
 Scott Roberts. Rebekah Brown, ISBN: 9781098120689


\begin{list2}
\item Compare this to the MITRE ATT\&CK framework adversaries and groups\\
\link{https://attack.mitre.org/}
\end{list2}



\slide{Identification}

%\hlkimage{}{}

\begin{quote}
The Identification phase is the moment where the defender identifies the presence of
an attacker impacting their environment. This can occur though a variety of methods:
\begin{list2}
\item Identifying the attacker entering the network, such as a server attack or an
incoming phishing email
\item  Noticing command-and-control traffic from a compromised host
\item  Seeing the massive traffic spike when the attacker begins exfiltrating data
\item  Getting a visit from a special agent at your local FBI field office
\item  And last, but all too often, showing up in an article by Brian Krebs
\end{list2}
\end{quote}

\slide{Starting with Known Information}

%\hlkimage{}{}

\begin{quote}
In {\bf almost every situation}, some {\bf information will be available} on threat actors,
whether that comes from previous incidents or attack attempts within your own envi‐
ronment (internal information) or intelligence reports produced by researchers, ven‐
dors, or other third parties (external information). Ideally, a combination of both
types will be available in order to provide the best overall picture of the threat.
\end{quote}

\begin{list2}
\item Usually something \emph{happens}
\end{list2}

\slide{David J Bianco's Pyramid of Pain}

\hlkimage{12cm}{david-j-bianco-pyramid.png}

\begin{quote}
The most useful information is informa‐
tion that’s hard for the actor to change. Incident responder David J. Bianco captured
this concept, and its impact on the adversary, in his Pyramid of Pain
\end{quote}

\slide{Indicator of Compromise Pieces}

%\hlkimage{}{}
\begin{list2}
\item Filesystem indicators -- File hashes, filenames, strings, paths, sizes, types, signing certificates
\item Memory indicators -- Strings and memory structures
\item Network indicators -- IP addresses, host names, domain names, HTML paths, ports, SSL certificates
\end{list2}
Each type of indicator has unique uses, is visible in different positions (whether mon‐
itoring a single system or a network), and depending on the format it’s in, may be
useful with different tools.

\slide{The Order of Volatility (OOV)}

\hlkimage{12cm}{order_of_volatility.png}

\begin{quote}\small
Certainly care and planning should be used when gathering information
from a running system. Isolating the computer—from other users and the
network—is the first step. And given that some types of data are less
prone to disruption by data collection than others, it’s a good idea to cap-
ture information in accordance with the data’s expected life span.
\end{quote}{\footnotesize
Source: \emph{Forensics Discovery} (FD), Dan Farmer, Wietse Venema 2004}

\slide{Data Formats }

\hlkimage{10cm}{forensic-layers.png}{\footnotesize
Source: \emph{Forensics Discovery} (FD), Dan Farmer, Wietse Venema 2004}


\begin{list2}
\item Data can be stored in many format, or obtained from various sources
\item We should use the correct tools
\item This is also why IoCs should be \emph{expressed in a platform-independent manner}
\end{list2}

\exercise{ex:nitroba-pcap}
\exercise{ex:zeekweb}
\exercise{ex:zeekdnsbasic}
\exercise{ex:zeekioc}





\slide{Chapter 5: Fix}

%\hlkimage{}{}

\begin{quote}
The process of using previously identified intelligence or threat data to identify where
an adversary is, either in your environment or externally, is called a Fix. In the Fix
phase of F3EAD, all the intelligence you gathered in the Find phase is put to work
tracking down signs of adversary activity on your networks.
\end{quote}
Source: Source: \emph{Intelligence-Driven Incident Response} (IDIR)
 Scott Roberts. Rebekah Brown, ISBN: 9781098120689

Chapter describes three ways to \emph{fix} the location of adversary activity:
\begin{list2}
\item Using indicators of compromise,
\item Adversary behavioral indicators, also known as TTPs
\item Adversary goals.
\end{list2}


\slide{Intrusion Detection -- Network Intrusion Detection System (NIDS)}

%\hlkimage{}{}

\begin{quote}
Network alerting involves identifying network traffic that could indicate malicious
activity. Several stages of the kill chain involve network communications between the
attackers and the victim machine, and it is possible to identify this activity by using
intelligence. The activities we can identify by using network traffic include the follow‐
ing:
\begin{list2}
\item Reconnaissance -- network scanning, probes
\item Delivery -- Attachments, Links, Metadata
\item Command and control -- known Command and Control serves used in malware, IRC ports
\item Lateral movement -- unusual traffic patterns internally
\item Actions on target --
\end{list2}
\end{quote}
Source: Source: \emph{Intelligence-Driven Incident Response} (IDIR)
 Scott Roberts. Rebekah Brown, ISBN: 9781098120689

Actions on targets can be hard to identify with network data only,
maybe large data transfers can be spotted, but recommend getting data from systems

\slide{System Alerting -- Host Intrusion Detection System (HIDS)}

%\hlkimage{}{}

\begin{quote}\small
The complement to network monitoring is system monitoring. In the same way that
network alerting is focused on particular aspects of the kill chain, system alerting can
be similarly be divided into the following areas:
\begin{list2}
\item Exploitation
\item Installation
\item Actions over target
\end{list2}
System alerting is always dependent on the operating system. With rare exceptions,
most tools—open source and commercial—are focused on a particular operating sys‐
tem. This is necessary because most security alerting takes place at the lowest levels of
the operating system, requiring deep integration into process management, memory
management, filesystem access, and so forth.
\end{quote}
Source: Source: \emph{Intelligence-Driven Incident Response} (IDIR)
 Scott Roberts. Rebekah Brown, ISBN: 9781098120689

\begin{list2}
\item Linux buffer overflow log messages seen earlier are specific to Linux
\item Windows Defender has become very advanced over the years
\end{list2}


\slide{Zeek on Windows}

%\hlkimage{}{}

\begin{quote}
As we shared at ZeekWeek 2022 in October, we’re thrilled to announce emerging support for Zeek on Windows, thanks to an open-source contribution from Microsoft. Part of its {\bf integration of Zeek into its Defender for Endpoint security platform}, this contribution provides {\bf fully-native build support for Windows platforms} and opens up a range of future technical possibilities in this vast ecosystem. Make sure to check out Microsoft’s talks on the technical aspects of this integration as well as the detection capabilities this move enables.
\end{quote}
Source: \url{https://zeek.org/2022/11/28/zeek-on-windows/}

\begin{list2}
\item Microsoft link\\{\scriptsize
\url{https://techcommunity.microsoft.com/t5/microsoft-defender-for-endpoint/new-network-based-detections-and-improved-device-discovery-using/ba-p/3682111}}
\end{list2}

\slide{Moving to the book}

%\hlkimage{}{}

% Was in the first edition
We will now go through parts of the book, starting at page 85, Road Runner Activity

Notes:
\begin{list2}
\item Books still call Zeek by the old name Bro, name changed in 2018
\item Older tools like SiLK and Argus has mostly been replaced with newer ones Elastiflow or Akvorado\\
\url{https://github.com/robcowart/elastiflow} \url{https://github.com/akvorado/akvorado}
\end{list2}

\exercise{ex:log4shell-iocs}


\slidenext{Buy the books! Create your VMs}

\end{document}
