\documentclass[Screen16to9,17pt]{foils}
\usepackage{kea-slides}
\externaldocument{build/introduction-to-incident-response-exercises}
\selectlanguage{english}

% Input:
% https://www.threathunting.net/reading-list

% Docker image with
% https://hub.docker.com/r/threathuntproj/hunting/
% This image contains a complete threat hunting & data analysis environment built on Python, Pandas, PySpark and Jupyter notebook.

\begin{document}

\mytitlepage
{Introduction to Incident Response Elective, KEA}
{Henrik Kramselund}


\slide{Goals for Course}

\hlkimage{6cm}{thomas-galler-hZ3uF1-z2Qc-unsplash.jpg}

\begin{quote}
Introduction to incident response. what is an incident and a log? We will discuss what happens when someone visits your network. Starting from initial compromise we will demonstrate how we can identify, process and handle incidents in networks.
\end{quote}

{\hfill \small Photo by Thomas Galler on Unsplash}

\slide{Ransomware Attacks are Common}


\hlkimage{13cm}{ransomware_arbor.png}

Make sure to backup your data! Test your backups!\\
Source: link{https://www.netscout.com/threatreport/global-ddos-attack-trends/}


\slide{Course Description}

{\bf Course description}\\
Introduction to Incident Response is a course that will describe the basics of incident response. This will include the terms, tools and processes used by professionals.

Below are the required parts from studieordningen:

{\bf Viden}
\begin{list2}
\item Forskellige cyberangrebs stadier og teknikker
\item Incident-Response cyklus
\item Pricipperne i Event logning
\item Processer i forbindelse med Incident response og Threat hunting
\end{list2}

\slide{Course Description, continued}

{\bf Færdigheder}

\begin{list2}
\item Søge i relevante filer, hukommelse og lignende for indicators of compromise (IoC)
\item Analysere event log, memory og timeline for tegn på security incidents
\item Viderebringe resultater i form af ekspertrapporter
\end{list2}

{\bf Kompetencer}
\begin{list2}
\item Anvende, udvikle og dele Threat Intelligence
\item Anvende og udvikle processer til incident håndtering i en organisation
\end{list2}

\slide{Prerequisites}

\begin{list1}
\item This course includes exercises and getting the most of the course requires the participants to carry out these practical exercises
\item We will use Linux for some exercises but previous Linux and Unix knowledge is not needed
\item It is recommended to use virtual machines for the exercises
\item Security and most internet related security work has the following requirements:
\begin{list2}
\item Network experience
\item Server experience
\item TCP/IP principles - often in more detail than a common user
\item Programming is an advantage, for automating things
\item Some Linux and Unix knowledge is in my opinion a {\bf necessary skill} for infosec work\\
-- too many new tools to ignore, and lots found at sites like Github and Open Source written for Linux
\end{list2}
\end{list1}


\slide{Primary literature}

\hlkrightpic{5cm}{0cm}{old_book_lumen_design_st_01.png}
Primary literature:
\begin{list2}
\item \emph{Intelligence-Driven Incident Response} \\
 Scott Roberts. Rebekah Brown, ISBN: 9781098120689 {\bf 2nd edition}- short IDIR

\item \emph{Forensics Discovery} (FD), Dan Farmer, Wietse Venema 2004, Addison-Wesley 240 pages.\\
ISBN: 9780201634976\\
This book is currently available for "free":\\
\link{http://fish2.com/security/} -- also uploaded to Fronter

\item \emph{
Computer Security Incident Handling Guide}, NIST SP 800-61 Rev. 2, August 2012,\\
\link{https://doi.org/10.6028/NIST.SP.800-61r2} -- also uploaded to Fronter
\end{list2}

{\bf Other papers and resources will also be part of the curriculum!}

\slide{Book: Intelligence-Driven Incident Response}

\hlkimage{6cm}{book-intelligence-driven-incident-response.jpg}

\emph{Intelligence-Driven Incident Response} \\
  Scott Roberts. Rebekah Brown, ISBN: 9781098120689 {\bf 2nd edition}- short IDIR

\slide{Book: Forensics Discovery (FD)}

\hlkimage{8cm}{forensic-discovery.jpg}

\emph{Forensics Discovery}, Dan Farmer, Wietse Venema 2004, Addison-Wesley.

Can be found at \link{http://fish2.com/security/}

\slide{Book: NIST SP800-61rev2}

\hlkimage{12cm}{NIST-SP800-61r2.png}

\link{https://doi.org/10.6028/NIST.SP.800-61r2}


\slide{Incident Handling, phases}

The procedures developed for incident response must cover the complete life-cycle

\begin{list2}
\item  Preparation for an attack, establish procedures and mechanisms for detecting and responding to attacks
\item  Identification of an attack, notice the attack is ongoing
\item  Containment (confinement) of the attack, limit effects of the attack as much as possible
\item  Eradication of the attack, stop attacker, block further similar attacks
\item  Recovery from the attack, restore system to a secure state
\item  Follow-up to the attack, include lessons learned  improve environment
\end{list2}

\slide{Incident Response Checklists}
\hlkimage{9cm}{incident-handling-checklist.png}

This checklist is from the NIST document
\emph{Computer Security Incident Handling Guide: Recommendations of the National Institute
of Standards and Technology}, NIST Special Publication 800-61
Revision 2, August 2012.


\slide{Intrusion Kill Chains}

\hlkimage{13cm}{crafting-cip-kill-chain.png}

\begin{list2}
\item See also \emph{Intelligence-Driven Computer Network Defense Informed by Analysis of Adversary Campaigns and Intrusion Kill Chains}, Eric M. Hutchins , Michael J. Cloppert, Rohan M. Amin, Ph.D. Lockheed Martin Corporation\\{\footnotesize
 \link{https://www.lockheedmartin.com/content/dam/lockheed-martin/rms/documents/cyber/LM-White-Paper-Intel-Driven-Defense.pdf}}
\end{list2}

\slide{Tools on Linux, Mac and Windows }


\begin{alltt}\scriptsize
hlk@debian-lab-11:~/Downloads$ sudo vol -f cridex.vmem windows.pstree.PsTree
Volatility 3 Framework 2.4.1
Progress:  100.00		PDB scanning finished
PID	PPID	ImageFileName	Offset(V)	Threads	Handles	SessionId	Wow64	CreateTime	ExitTime

4	0	System	0x823c89c8	53	240	N/A	False	N/A	N/A
* 368	4	smss.exe	0x822f1020	3	19	N/A	False	2012-07-22 02:42:31.000000 	N/A
** 584	368	csrss.exe	0x822a0598	9	326	0	False	2012-07-22 02:42:32.000000 	N/A
** 608	368	winlogon.exe	0x82298700	23	519	0	False	2012-07-22 02:42:32.000000 	N/A
*** 664	608	lsass.exe	0x81e2a3b8	24	330	0	False	2012-07-22 02:42:32.000000 	N/A
*** 652	608	services.exe	0x81e2ab28	16	243	0	False	2012-07-22 02:42:32.000000 	N/A
**** 1056	652	svchost.exe	0x821dfda0	5	60	0	False	2012-07-22 02:42:33.000000 	N/A
**** 1220	652	svchost.exe	0x82295650	15	197	0	False	2012-07-22 02:42:35.000000 	N/A
...
\end{alltt}

\begin{list2}
\item We will find and run various tools within the incident response space
\item Memory analysis, disk analysis, logging, network dissecting, intelligence feeds
\item After course is done, you will have started a toolbox for incident response and know a few from running them
\end{list2}

\end{document}
