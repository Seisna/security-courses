\documentclass[Screen16to9,17pt]{foils}
\usepackage{kea-slides}
\externaldocument{build/introduction-to-incident-response-exercises}
\selectlanguage{english}

% Input:
% https://www.threathunting.net/reading-list

% Docker image with
% https://hub.docker.com/r/threathuntproj/hunting/
% This image contains a complete threat hunting & data analysis environment built on Python, Pandas, PySpark and Jupyter notebook.

\begin{document}

\mytitlepage
{5. Malware and Incident Response}
{Introduction to Incident Response Elective, KEA}


\slide{Goals for today}

\hlkimage{6cm}{thomas-galler-hZ3uF1-z2Qc-unsplash.jpg}

\begin{list2}
\item Finish part II of the book\\
F3EAD process: Find, Fix Finish, Exploit, Analyze, Disseminate.
\item Continue building your knowledge of basic and popular tools
\item Talk about exam -- so you know what to expect
\end{list2}

{\hfill \small Photo by Thomas Galler on Unsplash}

\slide{Plan for today}

\begin{list2}
\item Look at a few more cases
\item Find tools for helping during incident response
\item Malware and Incident Response
\item Go over chapters 8 and 9 from the book
\end{list2}

Exercise theme:
\begin{list2}
\item Loki - Simple IOC and YARA Scanner\\
\url{https://github.com/Neo23x0/Loki}
\end{list2}

\slide{Time schedule}

\begin{list2}
\item 1) Going over a few cases and Tool Loki scanner -- 45min
\item Break 15min
\item 2) IDIR Chapter 8+9 -- 2x45min
\item 3) Plan your incident response, the mission and tools -- 45min
\item 4) Plan your incident response, processes 45min
\end{list2}

So today we will go through the chapters 8 and 9

\slide{Time schedule}

This day we will be doing a larger project, get started planning incident response
\begin{list2}
\item 1) Going over a few cases -- first 45min
\item 2) Plan your incident response, the mission -- 45 min
\item Break 15min
\item 3) Plan your incident response, tools -- 45min
\item 4) Plan your incident response, processes 45min
\end{list2}

Times are suggested, in real life this process would take months!

\slide{Part 1: Cases}

%\hlkimage{}{}

Goal is to find descriptions of specific malware
\begin{quote}

\end{quote}

Search for IoCs and/or YARA rules
\begin{list2}
\item Hashes
\item Filenames specific for malware
\item Yara rules for process memory
\item C2 Back Connections -- endpoints, IPs, domain names, host names, URLs
\end{list2}


\slide{Part 2: Tools}

%\hlkimage{}{}

We have talked about IoCs a few times, but lets dive deeper

There are many tools for different purposes, but they all have prerequisites, and very often you will need to modify the tools!

You can find lots of information about tools from books, and lists on the internet.

Many things are marked with forensics -- and can be used during incident response.

\begin{list2}
    \item Awesome lists, like: \url{https://github.com/meirwah/awesome-incident-response}
    \item Sigma is another popular format: for more information see \emph{Generic Signature Format for SIEM Systems}
    \url{https://github.com/SigmaHQ/sigma}
\end{list2}

\exercise{ex:loki-ioc-yara}





\slide{Part 3: Reading Summary -- IDIR chapter 8}

\emph{Intelligence-Driven Incident Response} (IDIR)
 Scott Roberts. Rebekah Brown, ISBN: 9781098120689

\begin{quote}
The Analyze phase is where we take data and information and process it into intelligence. This chapter covers the basic principles of analysis, models such as target-centric and structured analysis, and processes to assign confidence levels and address cognitive biases.
\end{quote}

\begin{list2}
\item Chapter 8: Analyze
\end{list2}

\slide{Analyze phase of F3EAD}

\hlkimage{13cm}{f3ead-analysis.png}
Source: \emph{Intelligence-Driven Incident Response} (IDIR)



\slide{Identify, analyze, predict}

%\hlkimage{}{}

\begin{quote}
Rather than {\bf only identifying} the technical details of the attack in order to respond and remediate, we can {\bf analyze} those same domains and IPs to identify patterns that can be used to better {\bf understand the attacker’s tactics}. That involves gathering additional information about the domains and IPs, including who they were registered to and how the attacker used them, in order to determine whether {\bf patterns} can be used to {\bf identify or predict future behaviors}. This new information is then analyzed, intelligence gaps (critical pieces of information that are needed to conduct analysis) are identified, and more information is gathered as needed.
\end{quote}
Source: \emph{Intelligence-Driven Incident Response} (IDIR)


\slide{OPM}

%\hlkimage{}{}

\begin{quote}
One of the most significant breaches in recent history is the breach of the United State’s {\bf Office of Personnel Management (OPM)}, which resulted in the loss of {\bf personal, highly sensitive} information about more than {\bf 20 million individuals} who had undergone a {\bf background investigation for a security clearance}. In addition to the size and sensitivity of the information that was stolen, the OPM breach is notable because
of the {\bf multiple, missed opportunities} for the attack to be {\bf identified and prevented}. The intrusion was a complex campaign that spanned years and included the theft of IT manuals and network maps, the compromise of two contractors with access to OPM’s networks, as well as OPM directly. Even when the individual intrusions were identified, no one connected the dots to identify that a larger threat needed to be addressed.
\end{quote}
Source: \emph{Intelligence-Driven Incident Response} (IDIR)


\slide{What to Analyze?}

%\hlkimage{}{}

\begin{quote}

\begin{list2}
\item Why were we targeted?
\item Who attacked us?
\item How could this have been prevented?
\item How can this be detected?
\item Are there any patterns or trends that can be identified?
\end{list2}

The {\bf output of the analysis} that you conduct in this phase should {\bf enable action},
whether that action is {\bf updating a threat profile, patching systems, or creating rules for detection}.
\end{quote}
Source: \emph{Intelligence-Driven Incident Response} (IDIR)


\slide{Conducting the Analysis}

%\hlkimage{}{}

\begin{quote}

\end{quote}

Enriching Your Data
\begin{list2}
\item Internet WHOIS information given as example
\item DNS -- passive DNS provides historical data
\item Malware information -- often Virus Total is mentioned
\item Sharing groups -- also add public blogs and sites like SANS Internet Storm Center\\
\url{https://isc.sans.edu/}
\end{list2}


\slide{Structured Analysis}

\hlkimage{12cm}{idir-structured-analysis.png }

\slide{Target-centric intelligence analysis}

\hlkimage{14cm}{44-WILSON-fig1-2-large.jpg}

\begin{quote}
In the book Intelligence Analysis, a Target-Centric Approach (CQ Press, 2003), Robert
Clark describes the traditional intelligence cycle as an attempt to give a linear struc‐
ture to a decidedly nonlinear process, and introduces target-centric analysis as an
alternative approach.
\end{quote}



\slide{Part 4:  Reading Summary -- IDIR chapter 9}

\emph{Intelligence-Driven Incident Response} (IDIR)
 Scott Roberts. Rebekah Brown, ISBN: 9781098120689

\begin{quote}
At some point, the investigation {\bf needs to end, or at least pause}, long enough to create outputs useful to other teams or organizations. We call the process of organizing, publishing, and sharing developed {\bf intelligence dissemination}. This is a skill set unto itself and, just like any other skill, has processes and takes time to develop. Good intelligence can be ruined by poor dissemination. Although writing up something after hours of analysis may seem unimportant, it’s worth the time for any intelligence team to focus and build their skills disseminating information.
\end{quote}

\begin{list2}
\item Chapter 9: Disseminate
\end{list2}

\slide{Focus on Writing and Disseminate}

%\hlkimage{}{}

\begin{quote}
Dissemination is such an important skill that in larger intelligence teams, resources
may be dedicated just to the dissemination phase. These dissemination-focused ana‐
lysts need the following:
\begin{itemize}
\item A strong understanding of the overall process and importance of the information
they’re sharing.
\item A firm grasp of the types and needs of stakeholders that the intelligence will be
going to.
\item A disciplined and clear writing style. (Intelligence writing is a little different from typical narrative writing; we’ll get into that later in this chapter.)
\item An eye toward operational security to protect the valuable intelligence products
and materials.
\end{itemize}
\end{quote}

\begin{list2}
\item Similarly an exploit without a \emph{write-up} can seldomly be changed or added to
\end{list2}


\slide{Intelligence Consumer Goals}

%\hlkimage{}{}

\begin{quote}
Also known as consumers, the audience is tied directly into the goal of any intelligence
product. The execution of the goal is intrinsically tied to the stakeholders you’re writing a product for. Every intelligence writer and team must develop an understanding
of the audience they’re writing for, as this understanding leads directly to creating
useful and actionable products. This is never a one-time exercise, as teams you’re
writing for change, evolve, and learn.
\end{quote}

\begin{list2}
\item Audience -- stakeholders, CEO, board of directors, your boss?
\item Internal Technical Consumers -- SOC analysts, incident responders,
cyber threat intelligence analysts, etc.
\item External Technical Consumers -- book PDF links to article about Google’s \emph{Looking into the Aquarium report}\\
\url{https://www.vice.com/en/article/3djn9y/a-glimpse-into-how-much-google-knows-about-russian-government-hackers}\\
\url{https://s3.documentcloud.org/documents/3461560/Google-Aquarium-Clean.pdf}
\end{list2}


\slide{Pause -- lets browse this report}

\hlkimage{8cm}{google-aquarium.png }

\url{https://s3.documentcloud.org/documents/3461560/Google-Aquarium-Clean.pdf}

\begin{list2}
\item You would not be expected to produce this kind of report immediately, but use it for inspiration
\item Book also spend some time discussing \emph{report writing} and have examples
\end{list2}


\slide{Disseminate Conclusion}

%\hlkimage{}{}

\begin{quote}
Great intelligence products generally have the following characteristics:
\begin{list2}
\item Accuracy
\item Audience focused
\item Actionable
\end{list2}

In addition, analysts should ask themselves the following five questions during the
writing process to ensure that the intelligence products that they develop will be well
received and will meet the needs of their intelligence customers:

\begin{list2}
\item What is the goal?
\item Who is the audience?
\item What is the proper length of product?
\item What level of intelligence? (Tactical, operational, strategic?)
\item What tone and type of language can you use? (Technical or nontechnical?)
\end{list2}
\end{quote}


\slidenext{Read the books! Play with tools}

\end{document}
