\documentclass[Screen16to9,17pt]{foils}
\usepackage{kea-slides}
\externaldocument{introduction-to-incident-response-exercises}
\selectlanguage{english}

% Input:
% https://www.threathunting.net/reading-list

% Docker image with
% https://hub.docker.com/r/threathuntproj/hunting/
% This image contains a complete threat hunting & data analysis environment built on Python, Pandas, PySpark and Jupyter notebook.

\begin{document}

\mytitlepage
{8. The Way Forward: Building an Intelligence Program}
{Introduction to Incident Response Elective, KEA}


\slide{Goals for today}

\hlkimage{6cm}{thomas-galler-hZ3uF1-z2Qc-unsplash.jpg}

\begin{list2}
\item Talk about the big picture
\item Summary of the course
\item Strategic Intelligence
\end{list2}

{\hfill \small Photo by Thomas Galler on Unsplash}

\slide{Plan for today}

\begin{list2}
\item
\item
\item
\item
\item Exam subjects, questions, finishing up
\end{list2}

Exercise theme:
\begin{list2}
\item
\item
\item
\end{list2}


\slide{Time schedule}

\begin{list2}
\item 1) Chapter 10: Strategic Intelligence -- 45min
\item 2) Chapter 11: Building an Intelligence Program -- 45 min
\item Break 15min
\item 3) Exam related items -- 45min
\item 4) Summary and finishing up -- 45min
\end{list2}


\slide{Reading Summary}

\emph{Intelligence-Driven Incident Response} (IDIR)
 Scott Roberts. Rebekah Brown, ISBN: 9781491934944

\begin{quote}

\end{quote}

\begin{list2}
\item Chapter 10: Strategic Intelligence
\item Chapter 11: Building an Intelligence Program
\end{list2}


\slide{The Way Forward}

\emph{Intelligence-Driven Incident Response} (IDIR)
 Scott Roberts. Rebekah Brown, ISBN: 9781491934944

\begin{quote}
Intelligence-driven incident response doesn’t end when the final incident report has been delivered; it will become a part of your overall security process. Part 3 covers big-picture aspects of IDIR that are outside individual incident-response investigations. These features include strategic intelligence to continually learn and improve processes, as well as implementation of an intelligence team to support security operations as a whole.
\end{quote}

\begin{list2}
\item What do you know about the \emph{overall security process}?
\item How does this subject incident response fit in?
\end{list2}


\slide{Chapter 10: Strategic Intelligence}

%\hlkimage{}{}

\begin{quote}
Every once in while, an incident responder will start an investigation with a prickling
sensation in the back of his mind. Some call it a premonition, some call it deja vu, but
as the investigation unwinds, it will inevitably hit him: he has done this before. This.
Exact. Same. Investigation.
\end{quote}
Source: \emph{Intelligence-Driven Incident Response} (IDIR)


\begin{list2}
\item Putting out fires takes time, but sometimes you should let the current fire burn, and work on things to prevent and catch future fires
\end{list2}


\slide{What Is Strategic Intelligence?}

%\hlkimage{}{}

\begin{quote}
Strategic intelligence gets its name not only from the subjects that it covers, typically a {\bf high-level analysis of information with long-term implications}, but also from its audience. {\bf Strategic intelligence is geared toward decision makers} with the ability and authority to act, because this type of intelligence should shape policies and strategies moving forward. This doesn’t mean, however, that leadership is the only group that can benefit from these insights. Strategic intelligence is {\bf extremely useful to all levels of personnel} because it can help them understand the surrounding context of the issues that they deal with at their levels.
\end{quote}
Source: \emph{Intelligence-Driven Incident Response} (IDIR)

\begin{list2}
\item Understanding and working together makes a difference
\end{list2}


\slide{The State of Strategic Analysis}

%\hlkimage{}{}

\begin{quote}
In his paper, "The State of Strategic Analysis,” John Heidenrich wrote that “a strategy is not really a plan but the logic driving a plan.” When that logic is present and clearly communicated, analysts can approach problems in a way that supports the overarching goals behind a strategic effort rather than treating each individual situation as its own entity.
\end{quote}
Source:
\emph{The State of Strategic Analysis} John Heidenrich via
\emph{Intelligence-Driven Incident Response} (IDIR)

\begin{list2}
\item Many companies in Denmark does NOT have a clear strategic plan, mission or ideas of how to \emph{do security}
\item Most companies in Denmark consider security an after-thought, burden, cost, annoying
\item Various organisations have tried to do \emph{maturity models} for software and security
\end{list2}

\slide{CIS Controls: Incident Response}

\hlkimage{11cm}{cis-17-incident-response.png}
Source: \url{https://www.cisecurity.org/controls/incident-response-management}


\slide{Developing Target Models}

\hlkimage{12cm}{idir-hierarchial-model.png}

\begin{quote}
Hierarchical models are traditionally used to show personnel or roles, but one unique application of a hierarchical model is to use it to {\bf identify the data that is important to an organization}. A hierarchical model for data includes the broad categories of data, such as financial information, customer information, and sensitive company information.
\end{quote}
Source: \emph{Intelligence-Driven Incident Response} (IDIR)


\slide{Network Models}

\hlkimage{12cm}{idir-network-model.png}

\begin{list2}
\item Process models
\item Timelines -- various uses, tool re-use, spread of attack types, etc.
\end{list2}

\slide{Intelligence Cycle or Intelligence Process}

\hlkimage{9cm}{The_Intelligence_Process_JP_2-0.png}
Source: \link{https://en.wikipedia.org/wiki/Intelligence_cycle}

\begin{list2}
\item Chapter 10 continues applying the Intelligence Cycle/Process to the strategic level \\
-- which we consider high-level for now, we won't be allowed to this in most Danish companies
\end{list2}

\slide{Conclusion Strategic Intelligence}

%\hlkimage{}{}

\begin{quote}
{\Large\bf Conclusion}\\
We consider strategic intelligence to be the {\bf logic behind the plan}, and it is no wonder that many incident responders {\bf struggle with finding the time} to conduct this level of analysis. In many organizations, incident responders would be hard-pressed to find a plan at all, much less understand the logic behind the plan. {\bf Strategic intelligence}, when {\bf properly analyzed and adopted by leadership}, can not only inform leadership of the long-term threats to an organization, but can also provide incident responders with policies and procedures that will {\bf support their ability to meet the needs of their organization}.
\end{quote}
Source: \emph{Intelligence-Driven Incident Response} (IDIR)

\begin{list2}
\item May be hard to convince leadership, so take numbers, collect data, present data
\item ... or leave the organisation
\end{list2}


\slide{Chapter 11: Building an Intelligence Program}

%\hlkimage{}{}

\begin{quote}

\end{quote}

\begin{list2}
    \item
\end{list2}



\slide{Crafting the InfoSec Playbook}
Maybe as a reference look into the book I suggested

\hlkimage{6cm}{book-crafting-infosec-playbook.jpg}

\emph{Crafting the InfoSec Playbook: Security Monitoring and Incident Response Master Plan}\\
 by Jeff Bollinger, Brandon Enright, and Matthew Valites ISBN: 9781491949405 - short CIP


\slide{Crafting the InfoSec Playbook}


This book will help you to answer common questions:
\begin{list2}
\item How do I find bad actors on my network?
\item How do I find persistent attackers?
\item How can I deal with the pervasive malware threat?
\item How do I detect system compromises?
\item How do I find an owner or responsible parties for systems under my protection?
\item How can I practically use and develop threat intelligence?
\item How can I possibly manage all my log data from all my systems?
\item How will I benefit from increased logging—and not drown in all the noise?
\item How can I use metadata for detection?
\end{list2}
Source: \emph{Crafting the InfoSec Playbook: Security Monitoring and Incident Response Master Plan}\\
 by Jeff Bollinger, Brandon Enright, and Matthew Valites ISBN: 9781491949405



\slide{Part 3: Exam related items}

You can now ask questions, or we can walk through all the subjects

Notes: Exam subjects for Introduction to Incident Response
\begin{list2}
\item Exam will be up to 25 min
\item Keywords listed are \emph{ideas} for things to go through not a checklist\\
Consider it example items that fit into this subject.
\item You will only have 10 min to go through your presentation
\item Feel free to include tools and references to tools throughout!
\item Watch out if you do a demo! Better to have a video without audio, and talk over it
\item After this we will do up to 15 min of questions and dialogue in the subject and the course
\end{list2}

\slide{Subjects 1-2}

%\hlkimage{}{}

\begin{alltt}
{\bf 1. Overview of Incident Response}\\
Why do we need Incident Response, a few threats and context
Definitions of Incident Response, Computer Forensics
Main resources we have used, IDIR book, NIST, other books

{\bf 2. Cyber Attack Phases and IR}\\
Main processes and methods, intrusion kill chain, incident response cycle
Overall description of how attacks happen, and how we deal with them
Mitre ATT&CK framework can be used here
\end{alltt}

\slide{Subjects 3-4}

%\hlkimage{}{}

\begin{alltt}
{\bf 3. Incident Response Life Cycle}\\
Detailed about what happens when we have identified an incident
Go through the phases and explain as much as possible within 10minutes
Use NIST SP800-61r2 figure 3-1, IDIR or other books or references as you wish
You can also use the F3EAD process
You can even present overview and compare instead

{\bf 4. Order of Volatility and Tools}\\
Explain the concept of Order of Volatility (OOV) from Forensics Discovery, Farmer Venema 2004
List example tools, explain how they fit with volatility
Why do we run memory tools first, and can wait with disk and storage analysis
Live Response vs cold images and files with data
\end{alltt}

\slide{Subjects 5-6}

%\hlkimage{}{}

\begin{alltt}
{\bf 5. Evidence and IoCs}\\

Anything related to facts we find and use
Definitions of IoCs, examples of IoCs
Domains, DNS, Passive DNS,
File related, Hash values, IP addresses
Enrichment and metadata related to facts
Processing of facts, where do we find them
Ideas Whois, RIPE Stat, RIPE delegated list of IP prefixes etc.
Logging can also be used here


{\bf 6. Tools we have used during the course}\\
Take your favourite tool we have used in classe, create screenshot
Walk us through the process, what it provides, how the tool helps,
what is the output
Beware you CAN do demos, but if something goes wrong ... better
to have a video without sound and present IMHO
Brim, tcpdump, wireshark, Packetbeat
Sysinternals
Zeek, Suricata
Volatility framework
Loki IOC and YARA scanner
MISP Project
... anything we mentioned in class or tried is OK
\end{alltt}

\slide{Subject 7. How to establish an Incident Response Capability}

%\hlkimage{}{}

\begin{alltt}
{\bf 7. How to establish an Incident Response Capability}\\
What are the steps to create this capability in an organisation
Use input from NIST document and IDIR book also provides detailed information
Also the exercise from March 30 may help
Select the parts of this which interest you the most, can be organizational, technical or a mix
Outline steps which you would propose to a CEO when you got hired as CISO
\end{alltt}

\slide{Subject 8. Vulnerability Cases}

%\hlkimage{}{}

\begin{quote}
{\bf 8. Vulnerability Cases}

Ideas\\
a) Take a known vulnerability, like Log4Shell
Go through how it works, what it does, what are prerequsites etc.
What are some IoCs, signs of intrusion with this
Focus on how this relates to Incident Response, how would an organisation react

OR

b) What preperative steps could help CMDB?, how would architecture changes reduce likelihood or prevent this from happening
How can organisations learn from cases in other organisations

OR

c) Budgets and incident response
Talk about steps from the incident response life cycle and how it relates to cost
- usually it is MUCH more efficient to spend a little on preparation, to shorten incidents
\end{quote}


\slide{Part 4: Finishing up in this course }

\hlkimage{7cm}{secure-all-the-things.png}

I would like to use some time to finish up this course.

\begin{list2}
\item Was it useful?
\item Do you want to work within this specific area
\item Do you want to know \emph{about} security or work \emph{in security}
\end{list2}





\end{document}
