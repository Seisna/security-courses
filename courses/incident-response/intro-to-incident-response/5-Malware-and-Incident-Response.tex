\documentclass[Screen16to9,17pt]{foils}
\usepackage{kea-slides}
\externaldocument{introduction-to-incident-response-exercises}
\selectlanguage{english}

% Input:
% https://www.threathunting.net/reading-list

% Docker image with
% https://hub.docker.com/r/threathuntproj/hunting/
% This image contains a complete threat hunting & data analysis environment built on Python, Pandas, PySpark and Jupyter notebook.

\begin{document}

\mytitlepage
{5. Malware and Incident Response}
{Introduction to Incident Response Elective, KEA}


\slide{Goals for today}

\hlkimage{6cm}{thomas-galler-hZ3uF1-z2Qc-unsplash.jpg}

\begin{list2}
\item Finish part II of the book\\
F3EAD process: Find, Fix Finish, Exploit, Analyze, Disseminate.
\item Continue building your knowledge of basic and popular tools
\item Talk about exam -- so you know what to expect
\end{list2}

{\hfill \small Photo by Thomas Galler on Unsplash}

\slide{Plan for today}

\begin{list2}
\item Look at a few more cases
\item Find tools for helping during incident response
\item Malware and Incident Response
\item Go over chapters 8 and 9 from the book
\end{list2}

Exercise theme:
\begin{list2}
\item Loki - Simple IOC and YARA Scanner\\
\url{https://github.com/Neo23x0/Loki}
\end{list2}

\slide{Time schedule -- tuesday March 29.}

\begin{list2}
\item 1) Going over a few cases -- first 45min
\item 2) Tools and Exam subjects -- 45min
\item Break 15min
\item 3) IDIR Chapter 8 -- 45min
\item 4) IDIR Chapter 9 -- 45min
\end{list2}

So today we will go through the chapters 8 and 9, leaving tomorrow open for a larger on-site project, please show up

\slide{Time schedule -- wednesday March 30.}

This day we will be doing a larger project, get started planning incident response
\begin{list2}
\item 1) Going over a few cases -- first 45min
\item 2) Plan your incident response, the mission -- 45 min
\item Break 15min
\item 3) Plan your incident response, tools -- 45min
\item 4) Plan your incident response, processes 45min
\end{list2}

Times are suggested, in real life this process would take months!

\slide{Part 1: Cases}

%\hlkimage{}{}

Goal is to find descriptions of specific malware
\begin{quote}

\end{quote}

Search for IoCs and/or YARA rules
\begin{list2}
\item Hashes
\item Filenames specific for malware
\item Yara rules for process memory
\item C2 Back Connections -- endpoints, IPs, domain names, host names, URLs
\end{list2}


\slide{Part 2: Tools}

%\hlkimage{}{}

We have talked about IoCs a few times, but lets dive deeper
\begin{quote}

\end{quote}

\begin{list2}
    \item Sigma is another popular format: for more information see \emph{Generic Signature Format for SIEM Systems}
    \url{https://github.com/SigmaHQ/sigma}
\end{list2}

\exercise{ex:loki-ioc-yara}





\slide{Part 3: Reading Summary -- IDIR chapter 8}

\emph{Intelligence-Driven Incident Response} (IDIR)
 Scott Roberts. Rebekah Brown, ISBN: 9781491934944

\begin{quote}

\end{quote}

\begin{list2}
\item Chapter 8: Analyze
\end{list2}

\slide{Part 4:  Reading Summary -- IDIR chapter 9}

\emph{Intelligence-Driven Incident Response} (IDIR)
 Scott Roberts. Rebekah Brown, ISBN: 9781491934944

\begin{quote}

\end{quote}

\begin{list2}
\item Chapter 9: Disseminate
\end{list2}




\slidenext{Buy the books! Create your VMs}

\end{document}
