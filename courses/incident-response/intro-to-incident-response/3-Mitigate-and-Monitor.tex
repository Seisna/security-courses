\documentclass[Screen16to9,17pt]{foils}
\usepackage{kea-slides}
\externaldocument{build/introduction-to-incident-response-exercises}
\selectlanguage{english}

% Input:
% https://www.threathunting.net/reading-list

% Docker image with
% https://hub.docker.com/r/threathuntproj/hunting/
% This image contains a complete threat hunting & data analysis environment built on Python, Pandas, PySpark and Jupyter notebook.

\begin{document}

\mytitlepage
{3. Mitigate and Monitor}
{Introduction to Incident Response Elective, KEA}


\slide{Goals for today}

\begin{alltt}\scriptsize
hlk@debian-lab-11:~/Downloads$ sudo vol -f cridex.vmem windows.pstree.PsTree
Volatility 3 Framework 2.4.1
Progress:  100.00		PDB scanning finished
PID	PPID	ImageFileName	Offset(V)	Threads	Handles	SessionId	Wow64	CreateTime	ExitTime

4	0	System	0x823c89c8	53	240	N/A	False	N/A	N/A
* 368	4	smss.exe	0x822f1020	3	19	N/A	False	2012-07-22 02:42:31.000000 	N/A
** 584	368	csrss.exe	0x822a0598	9	326	0	False	2012-07-22 02:42:32.000000 	N/A
** 608	368	winlogon.exe	0x82298700	23	519	0	False	2012-07-22 02:42:32.000000 	N/A
*** 664	608	lsass.exe	0x81e2a3b8	24	330	0	False	2012-07-22 02:42:32.000000 	N/A
*** 652	608	services.exe	0x81e2ab28	16	243	0	False	2012-07-22 02:42:32.000000 	N/A
**** 1056	652	svchost.exe	0x821dfda0	5	60	0	False	2012-07-22 02:42:33.000000 	N/A
**** 1220	652	svchost.exe	0x82295650	15	197	0	False	2012-07-22 02:42:35.000000 	N/A
...
\end{alltt}

\begin{list2}
\item Live Response and Memory analysis
\item Get annoyed -- memory analysis is tricky, and why we practice!
\item Note how the book and processes {\bf break down the problems into smaller bits, FOCUS}
\end{list2}

\slide{Plan for today}

\begin{list2}
\item IDIR chapters F3EAD continued
\item Getting data while observing the Order of Volatility
\item Practice more
\end{list2}

Exercise theme:
\begin{list2}
\item Live Response
\item Memory analysis with Volatility
\item Checklists inspired by NIST SP800-61r2 -- why, who, what, where, when, ...
\end{list2}


\slide{Time schedule}

\begin{list2}
\item 1) Going over my presentation,  -- first 45min
\item 2) Live Response, get to know your systems -- 45 min
\item Break 15min
\item 3) Memory Analysis -- 2x45min including 4)
\item 4) Summary of course and today plus questions
\end{list2}


\slide{Reading Summary}

\emph{Intelligence-Driven Incident Response} (IDIR)
 Scott Roberts. Rebekah Brown, ISBN: 9781098120689

\begin{quote}
Once you have identified the threats that you are facing and investigated how those
threats have accessed and moved through your network, it is time to remove the
threats. This phase is known as Finish and involves not only eradicating the footholds
that malicious actors have put in your network, but also working to remediate whatever enabled them to get access in the first place.
\end{quote}

\begin{list2}
\item Chapter 6: Finish
\end{list2}

Reading Summary, browse \emph{Computer Security Incident Handling Guide} NIST SP800-61r2
\begin{list2}
\item  Chapter 3: Handling an Incident
\end{list2}


\slide{Live Response}

%\hlkimage{}{}

\begin{quote}
One of the less appreciated but often effective analysis methods is live response. Live
response is analysis of a potentially compromised system without taking it offline.

Most forensics analysis requires {\bf turning the system offline}, losing system state information such as active processes. It also risks tipping off the attacker and is widely dis‐
ruptive to users as well.

Live response pulls the following information:
\begin{list2}
\item Configuration information
\item System state
\item Important file and directory information
\item Common persistence mechanisms
\item Installed applications and versions
\end{list2}
\end{quote}

\slide{Live Response tools}

%\hlkimage{}{}

Book list a few tools:
\begin{list2}
\item OSXCollector \url{https://yelp.github.io/osxcollector/}
\item Kansa powershell \url{https://github.com/davehull/Kansa}
\item Tools like these have been around for many years, some wrote to 3,5" floppies
\end{list2}

Pause, let's discuss which ones would suit your environments, the class, your services, mine etc.

This is the common case, you are in the middle of an incident -- which tools do you know, which are available, do you have licenses for commercial ones, etc.

\slide{Other tools that might help with data}

\hlkimage{9cm}{osquery-diskless.png }
Source: \url{https://www.osquery.io/}

Many other tools can help identify problems
\begin{list2}
\item osquery \url{https://www.osquery.io/} if installed collect end point information
\item Lynis \url{https://cisofy.com/lynis/} audit your system
\end{list2}

\exercise{ex:live-response}

\slide{Memory Analysis}

%\hlkimage{}{}

\begin{quote}
Similar to live response, memory analysis focuses on collecting volatile system state
in memory. Given that every process on a system requires memory to run, this tech‐
nique provides an excellent vantage point to gather information, especially from tools
that attempt to run stealthily with limited system footprint.
\end{quote}
Source: Source: \emph{Intelligence-Driven Incident Response} (IDIR)

\begin{list2}
\item FireEye’s Redline memory analysis tool (created by Mandiant)
\end{list2}

\slide{Volatility memory-analysis framework }

%\hlkimage{}{}

\begin{quote}
Volatility is a Python-based, open source memory-analysis framework. Volatility does
not gather memory itself the way Redline does. Instead, it reads the memory formats
from a wide variety of collection tools that run on a wide variety of operating systems.
What Volatility provides is a framework and set of scripts for analyzing memory;
detecting malware running in memory, extracting cryptographic keys—in fact any‐
thing you can find a plug-in to do.
\end{quote}
Source: Source: \emph{Intelligence-Driven Incident Response} (IDIR)

\begin{list2}
\item \url{https://www.volatilityfoundation.org/}
\item Again, more tools exist for Memory Analysis
\item Often virtualisation can provide memory dumps, pause VM, take snapshot and download file -- easy access!
\end{list2}


\slide{Disk Analysis -- not today}

%\hlkimage{}{}

\begin{quote}
Traditional disk forensics typically involves using specialized tools to extract filesys‐
tem information from the raw bits and bytes on a hard drive. The information on a
hard drive is unintelligible at first glance. It contains endlessly nested structures at the
hardware, filesystem, operating system, and data-format level, similar to the OSI
model. Peeling through these layers is a process called file carving.
\end{quote}
Source: Source: \emph{Intelligence-Driven Incident Response} (IDIR)

\begin{list2}
\item Important subject: We will do this another day
\end{list2}

\exercise{ex:volatility-install}


\slide{IDIR Chapter 6: Finish}

%\hlkimage{}{}

\begin{quote}
Finish involves {\bf more than removing malware} from a system, which is why we spend
so much time in the Find and Fix stages. To properly finish an attacker’s activity, it is
critical to {\bf understand how that threat actor operates} and to remove not just malware
or artifacts left behind by an attack, but  {\bf also communications channels, footholds,
redundant access, and any other aspects of an attack} that we uncovered in the Fix
phase. Properly finishing an adversary requires a {\bf deep understanding of the attacker,
their motives, and their actions}, which will allow you to act with confidence as you
secure the systems and regain control of your network.
\end{quote}
Source: Source: \emph{Intelligence-Driven Incident Response} (IDIR)


\begin{list2}
\item This is critical to success
\item When handling real incidents, make sure to work in teams and schedule time off!
\end{list2}


\slide{Stages of Finish}

%\hlkimage{}{}

\begin{quote}
The Finish phase has three stages: {\bf mitigate, remediate, and rearchitect}. These stages acknowledge that you {\bf can’t do everything at once}. Even after a comprehensive investigation, some tactical response actions can take place quickly, but many {\bf strategic response actions, such as rearchitecting, will take longer}. We will discuss the three phases next.
\end{quote}
Source: Source: \emph{Intelligence-Driven Incident Response} (IDIR)

\begin{list2}
\item Mitigate
\item Remediate
\item Rearchitect
\end{list2}

\slide{Mitigate}

%\hlkimage{}{}

\begin{quote}
Mitigation is the process of taking {\bf temporary steps} to keep an intrusion from {\bf getting worse} while {\bf longer-term corrections} are taken. Ideally, mitigation should take place {\bf quickly and in a coordinated fashion} to avoid giving the adversary a chance to react before you have cut off their access. Mitigation takes place at several phases of the {\bf kill chain}, including delivery, command and control, and actions on target.
\end{quote}
Source: Source: \emph{Intelligence-Driven Incident Response} (IDIR)

\begin{list2}
\item Stop the delivery by blocking known entry ways
\item Patch remaining systems -- to avoid new infections
\item Block known bad IP addresses
\end{list2}


\slide{Remediate}

%\hlkimage{}{}

\begin{quote}
Remediation is the process of {\bf removing all adversary capabilities and invalidating any compromised resources} so that they can no longer be used by the adversary to conduct operations. {\bf Remediation} generally focuses on a different set of kill-chain phases
than mitigation does, most notably {\bf exploitation, installation, and actions over target}, which we will break down in this section.
\end{quote}
Source: Source: \emph{Intelligence-Driven Incident Response} (IDIR)

\begin{list2}
\item Clean and Patch systems
\end{list2}


\slide{Rearchitect}

%\hlkimage{}{}

\begin{quote}
One of the most effective uses of {\bf intelligence-driven incident-response data} is an
advanced form of remediation: the incident-response team looks at {\bf past incident
trends, identifies common patterns, and works to mitigate these at a strategic level}.
These mitigations are generally not small changes, and may range from small things
like tweaks to system configurations or additional user training, to massive shifts in
tooling such as the development of a new security tools or even complete network
rearchitecture.
\end{quote}
Source: Source: \emph{Intelligence-Driven Incident Response} (IDIR)

\begin{list2}
\item Why did the attacks succeed in the first place, can we change the environment
\end{list2}

\slide{Taking action}

\begin{quote}
Active defense is frequently equated with the idea of {\bf hack back}, or attempting to attack
a malicious actor directly. Although this qualifies as one aspect of active defense, five
other useful pieces of active defense are {\bf far more common}. This mix-up is based on a
fundamental misunderstanding of the purpose of active defense.
\end{quote}

I do NOT recommend hacking back! Please do inform the administrators responsible for computing resources that attacks you!

Active Defense (5D)
\begin{list2}
\item Deny
\item Disrupt
\item Degrade
\item Deceive
\item Destroy
\end{list2}

\slide{Deny}

\begin{quote}
The idea of denying an adversary is so straightforward and common that most organizations wouldn’t even imagine it’s a type of active defense; most are actively doing deny-style actions. If we go by our traditional definition of disrupting the adversary’s tempo, though, this is a perfect example. {\bf Denying can be simple, such as implementing a new firewall rule to block an adversary’s command and control}, applying a
system patch for a vulnerability, or shutting down access for a compromised email
account. The key to denial is {\bf preemptively excluding} a capability or infrastructure from
the malicious actor.
\end{quote}

\begin{list2}
\item Most likely what you want to do
\item Use intelligence feeds to know what to block
\item Systems like CrowdSec can also be used \url{https://www.crowdsec.net/}
\item The log4j case is an example where a list of systems performing scanning was shared in real time by CrowdSec\\
\url{https://www.crowdsec.net/blog/detect-block-log4j-exploitation-attempts}

\end{list2}


\slide{Disrupt}

%\hlkimage{}{}

\begin{quote}
If the deny action preemptively excludes a capability or infrastructure from the
malicious actor, then {\bf disrupt actively excludes a resource from the malicious actor}. In most cases, disruption requires active observation of an adversary in order to know when they’re active so that they can be disrupted in real time. This could mean
{\bf cutting off a command-and-control channel while it’s being used or interrupting the
exfiltration of a large archive file}.
\end{quote}

\begin{list2}
\item There are systems to actively monitor the network for exfiltration, NDR, XDR \\
\url{https://en.wikipedia.org/wiki/Network_detection_and_response}\\
\url{https://en.wikipedia.org/wiki/Extended_detection_and_response}
\end{list2}

\slide{Degrade}

%\hlkimage{}{}

\begin{quote}
Closely related to disrupting and denying an adversary, {\bf degrade focuses on marginal
reduction} of an adversary’s resources while they’re actively being used. An easily understandable {\bf example is throttling an adversary’s bandwidth during exfiltration}, causing a large file to upload over an extremely slow time frame. This degradation of access attempts to frustrate adversaries, ideally driving them to attempt to access the data in a different way and expose additional infrastructure, tools, or TTPs.
\end{quote}

\begin{list2}
\item If you mostly receive data over some connection, could you implement throttling before attacks start?
\end{list2}

\centerline{\large Active defense conclusion, Implementing multiple tools before they are needed is recommended}

\slide{Deceive}

%\hlkimage{}{}

\begin{quote}
Easily the most advanced of available techniques, the deceive active defense action
is based on the counter-intelligence concept of {\bf deliberately feeding adversaries false
information} with the hopes they’ll treat it as truth and make decisions based on it.
This ranges from planting false documents with incorrect values to hosting honeypot
systems or even networks.
\end{quote}

\begin{list2}
    \item The Cuckoos Egg book has a few examples where the defender tried getting the attacker to spend more time and resources
\end{list2}


\slide{Destroy}

%\hlkimage{}{}

\begin{quote}
Destroy actions do {\bf actual harm, whether kinetic or virtual, to an adversary’s tools,
infrastructure, or operators}. In most cases, this is the purview of law enforcement,
intelligence, or military operators that have the legal authority to commit such acts
(in the US, these are Title 10 and Title 50 organizations). For a {\bf commercial or private
organization to do so is not only generally accepted to be illegal but also dangerous.}
\end{quote}

\begin{list2}
    \item See the LockBit example, currently in the news -- search and you will quickly find it
\end{list2}


\slide{Checklists }

\hlkimage{12cm}{nist-sp800-61r2-checklist-status.png }

\begin{list2}
\item Checklists inspired by NIST SP800-61r2 -- why, who, what, where, when, ...
\end{list2}

\slide{Summary Where are we in the course!}


Questions to ask:
\begin{list2}
\item What is a tool? Are programs the only tools, can a checklist become a tool
\item Which things are generic, which things need to be decided in a specific environment
\end{list2}

\slidenext{Buy the books! Create your VMs}

\slide{Rebuilding a server/services}

%\hlkimage{}{}

There are some benefits if you can quickly re-deploy a server.

\begin{list2}
\item A new server can easily be provisioned
\item If the process is automated most employees can better perform it
\item The server is online quicker
\item The services is provisioned with the right settings, as before the incident
\item Customers and employees can access quicker -- less lost time and money
\end{list2}

The exercise mentioned here is just an example, using Ansible is a way to ensure more people in the organization can deploy services.

\exercise{ex:basicansible}

\end{document}
