\documentclass[Screen16to9,17pt]{foils}
\usepackage{kea-slides}
\externaldocument{build/introduction-to-incident-response-exercises}
\selectlanguage{english}

% Input:
% https://www.threathunting.net/reading-list

% Docker image with
% https://hub.docker.com/r/threathuntproj/hunting/
% This image contains a complete threat hunting & data analysis environment built on Python, Pandas, PySpark and Jupyter notebook.

\begin{document}

\mytitlepage
{6. Reporting on Incident Response}
{Introduction to Incident Response Elective, KEA}


\slide{Goals for today}

\hlkimage{6cm}{thomas-galler-hZ3uF1-z2Qc-unsplash.jpg}

\begin{list2}
\item Connect knowledge from previous days
\item Do a big exercise using knowledge from the class so far
\item Prepare you to work as incident responders in the future
\item Collect a playbook -- get started building your own playbook
\end{list2}

{\hfill \small Photo by Thomas Galler on Unsplash}

\slide{Plan for today}

\begin{list2}
\item Start out with some cases
\item Get you started planning incident response
\item List organizational requirements
\item List preparational steps
\item List specific tools you want to have, know of, buy, gather, ...
\end{list2}

Exercise theme:
\begin{list2}
\item Building your own playbook
\end{list2}

\slide{Time schedule -- wednesday March 30.}

This day we will be doing a larger project, get started planning incident response
\begin{list2}
\item 1) Going over a few cases from Denmark -- first 45min
\item 2) Plan your incident response, the mission -- 45 min
\item Break 15min
\item 3) Plan your incident response, tools -- 45min
\item 4) Plan your incident response, processes 45min
\end{list2}

Times are suggested, in real life this process would take months!

\slide{Part 1: Go over a few more cases}

%\hlkimage{}{}

Let's go over some of the recent cases from Denmark

You are most likely to find jobs in Denmark, and we know danish companies better

\slide{Example cases and Categories from Denmark}

Start by finding a few cases from Denmark, we already talked about Maersk but feel free to re-use this data.

Cases I know are: Forsvaret (2023), Demant (2019), Ecco (2022), kommuner, infrastruktur

Try to put them in categories and find examples of each category:
\begin{list2}
\item DDoS
\item Data leaks -- check datatilsynets perhaps
\item Ransomware Demant -- ransomware, but may stealing data too?
\item ... more categories here, maybe use Mitre ATT\&CK for inspiration
\end{list2}

\slide{Learning from others Incident Response cases}

\begin{list2}
\item What did they do, consider advice from book
\item Could they have responded differently
\item What were they missing, could they learn from this
\item What are some things we definitely would have \emph{ready} for handling incidents
\item What did it cost them, input for security budgets
\end{list2}


\slide{Part 2-4) Plan your incident response}

\hlkimage{8cm}{ks-kyung-784757-unsplash.jpg}

\begin{quote}
Congratulations -- you are now the CISO

And we would like to A) Avoid incidents B) Resolve any incident efficiently
\end{quote}

\begin{list2}
\item Rest of today we will plan our incident response in Company XYX
\item Essentially this will be the start of a playbook
\end{list2}

\slide{Company XYX}

%\hlkimage{}{}

\begin{quote}
This medium sized company with 100 employees produce a cheese cutter and sell them all over the world. They are the number one brand of cheese cutters, loved by chefs around the world.
\end{quote}

\begin{list2}
\item Turnover is in the millions
\item Orders are flowing in through a web shop for customers
\item Another web shop is used by B2B segment for ordering 1.000s of cheese cutters
\end{list2}

Help them, they don't have a CISO, they don't have security people, they are afraid of security incidents -- but don't know anything about them


\slide{Your Goal }

%\hlkimage{}{}

There will be a management and board meeting soon, and you will present

\begin{list2}
\item The XYX Security Organisation
\item The XYX security org mission statement
\item The Basic XYX Incident Response Process
\item The contact list for incident handling, feel free to add external companies
\item A list of systems and tools to put in place before incidents (CMDB?)
\item A list of programs, applications, tools, hardware to use for incidents (external drives and go-bag?)
\end{list2}

\slide{Where to find inspiration}

\hlkimage{10cm}{eugen-str-CrhsIRY3JWY-unsplash.jpg }

We have our main book, and have links to other documents, so feel free to find inspiration in:

\begin{list2}
\item NIST documents SP800 series -- NIST SP800-61r2
\item Awesome lists, like: \url{https://github.com/meirwah/awesome-incident-response}
\end{list2}

\slide{Further inspiration on the next slides}

\hlkimage{8cm}{password-window.png}



\slide{Tools and Resources}

You can find lots of information about tools from books, and lists on the internet.

Many things are marked with forensics -- and can be used during incident response.

\begin{list2}
    \item Awesome lists, like: \url{https://github.com/meirwah/awesome-incident-response}
    \item Sigma is another popular format: for more information see \emph{Generic Signature Format for SIEM Systems}
    \url{https://github.com/SigmaHQ/sigma}
\end{list2}


\slide{Crafting the InfoSec Playbook}

\hlkimage{6cm}{book-crafting-infosec-playbook.jpg}

\emph{Crafting the InfoSec Playbook: Security Monitoring and Incident Response Master Plan}\\
 by Jeff Bollinger, Brandon Enright, and Matthew Valites ISBN: 9781491949405 - short CIP

\emph{Develop your own threat intelligenceand incident detection strategy}

\slide{Crafting the InfoSec Playbook}


This book will help you to answer common questions:
\begin{list2}
\item How do I find bad actors on my network?
\item How do I find persistent attackers?
\item How can I deal with the pervasive malware threat?
\item How do I detect system compromises?
\item How do I find an owner or responsible parties for systems under my protection?
\item How can I practically use and develop threat intelligence?
\item How can I possibly manage all my log data from all my systems?
\item How will I benefit from increased logging—and not drown in all the noise?
\item How can I use metadata for detection?
\end{list2}
Source: \emph{Crafting the InfoSec Playbook: Security Monitoring and Incident Response Master Plan}\\
 by Jeff Bollinger, Brandon Enright, and Matthew Valites ISBN: 9781491949405



\slide{Why Elasticsearch}

\hlkimage{8cm}{illustrated-screenshot-hero-siem-500x730.png}
Screenshot from \url{https://www.elastic.co/siem}

Recommend building a proof-of-concept infrastructure using the Elastic stack and gather experience with logging. This can be done without a license fee and the organization can then see what works and doesn't. Then using the experiences as input an informed decision can be made, to continue with this as a home grown logging and auditing solution, or buy a premade one.


\slide{SIEM}

%\hlkimage{}{}

\begin{quote}
{\bf Security information and event management (SIEM)} is a subsection within the field of computer security, where software products and services combine security information management (SIM) and security event management (SEM). They provide real-time analysis of security alerts generated by applications and network hardware.

  Vendors sell SIEM as software, as appliances, or as managed services; these products are also used to log security data and generate reports for compliance purposes.[1]

  The term and the initialism SIEM was coined by Mark Nicolett and Amrit Williams of Gartner in 2005.[2]
\end{quote}
Source: \link{https://en.wikipedia.org/wiki/Security_information_and_event_management}

\begin{list2}
  \item Note: there are alerting examples towards the bottom of the page, with sources
  \item Closely related to log management, incident response
\end{list2}




\slide{SOC}

%\hlkimage{}{}

\begin{quote}
An information security operations center (ISOC or SOC) is a facility where enterprise information systems (web sites, applications, databases, data centers and servers, networks, desktops and other endpoints) are monitored, assessed, and defended.

...

A security operations center (SOC) can also be called a security defense center (SDC), security analytics center (SAC), network security operations center (NSOC),[3] security intelligence center, cyber security center, threat defense center, security intelligence and operations center (SIOC). In the Canadian Federal Government the term, infrastructure protection center (IPC), is used to describe a SOC.
\end{quote}
Source: \link{https://en.wikipedia.org/wiki/Information_security_operations_center}

\begin{list2}
  \item We have a whole book about SOCs, but I skipped the introductory chapters!
  \item If you need to build a SOC, that is great source of information
\end{list2}

\slide{Baseline Skills}

\begin{list2}\small
\item Threat-Centric Security, NSM, and the NSM Cycle
\item TCP/IP Protocols
\item Common Application Layer Protocols
\item Packet Analysis
\item Windows Architecture
\item Linux Architecture
\item Basic Data Parsing (BASH, Grep, SED, AWK, etc)
\item IDS Usage (Snort, Suricata, etc.)
\item Indicators of Compromise and IDS Signature Tuning
\item Open Source Intelligence Gathering
\item Basic Analytic Diagnostic Methods
\item Basic Malware Analysis
\end{list2}

Source: \emph{Applied Network Security Monitoring Collection, Detection, and Analysis}, Chris Sanders and Jason Smith

\slide{Strategy for implementing identification and detection}

We recommend that the following strategy is used for implementing identification and detection.

We have the following recommendations and actions points for logging:
\begin{enumerate}
\item[\faSquareO] Enable system logging from servers
\item[\faSquareO] Enable system logging from network devices
\item[\faSquareO] Centralize logging
\item[\faSquareO] Add search facilities and dashboards
\item[\faSquareO] Perform system audits manually or automatically
\item[\faSquareO] Setup notification and notification procedures
\end{enumerate}

\slide{Extended Sources}
When a basic logging infrastructure is setup, it can be expanded to increase coverage, by
adding more sources:

\begin{list2}
\item DNS query logging -- will enable multiple cases to be resolved, example malware identification and tracing, when was a malware domain queried, when was the first infection
\item Session data from Firewalls, Netflow -- traffic patterns can be investigated and both attacks and cases like exfiltration can likely be seen
\end{list2}

Hint: Take the sources available first, make a proof-of-concept, expand coverage

\slide{Data Analysis Skills}

\begin{quote}
Although we could spend an entire book creating an exhaustive list of skills needed to be a good security data scientist, this chapter covers the following skills/domains that a data scientist will benefit from
knowing within information security:
\begin{list2}
\item Domain expertise—Setting and maintaining a purpose to the analysis
\item Data management—Being able to prepare, store, and maintain data
\item Programming—The glue that connects data to analysis
\item Statistics—To learn from the data
\item Visualization—Communicating the results effectively
\end{list2}
It might be easy to label any one of these skills as the most important, but in reality, the whole is greater than the sum of its parts. Each of these contributes a significant and important piece to the workings of
security data science.
\end{quote}

Source: \emph{Data-Driven Security: Analysis, Visualization and Dashboards} Jay Jacobs, Bob Rudis\\
ISBN: 978-1-118-79372-5 February 2014 \url{https://datadrivensecurity.info/} - short DDS



\slidenext{}

\end{document}
