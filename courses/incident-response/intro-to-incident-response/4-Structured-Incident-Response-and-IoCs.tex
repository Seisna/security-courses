\documentclass[Screen16to9,17pt]{foils}
\usepackage{kea-slides}
\externaldocument{build/introduction-to-incident-response-exercises}
\selectlanguage{english}

% Input:
% https://www.threathunting.net/reading-list

% Docker image with
% https://hub.docker.com/r/threathuntproj/hunting/
% This image contains a complete threat hunting & data analysis environment built on Python, Pandas, PySpark and Jupyter notebook.

\begin{document}

\mytitlepage
{4. Structured Incident Response and IoCs}
{Introduction to Incident Response Elective, KEA}


\slide{Goals for today}

\hlkimage{6cm}{thomas-galler-hZ3uF1-z2Qc-unsplash.jpg}

\begin{list2}
\item Practical work
\item How to choose between alternatives -- tools and software packages
\item How to get started analysing
\end{list2}

{\hfill \small Photo by Thomas Galler on Unsplash}

\slide{Plan for today}

\begin{list2}
\item Book chapter 7
\item Gathering data
\item Why do we gather data
\item How do we store this incident data
\item Example case: look at the Maersk incident
\item Example tools, including some Windows tools at the end!
\end{list2}

Exercise theme:
\begin{list2}
\item Installing MISP Project
\end{list2}

\slide{Time schedule}

\begin{list2}
\item 1) Going over a few cases -- first 45min
\item 2) Choosing tools -- 45 min
\item Break 15min
\item 3) Threat Information gathering, selecting and presenting -- 45min
\item 4) Buffer and Valentina Palacin book
\end{list2}


\slide{Reading Summary}

\emph{Intelligence-Driven Incident Response} (IDIR)
 Scott Roberts. Rebekah Brown, ISBN: 9781098120689

\begin{quote}
Now we need to {\bf gather all of that data}, analyze it for intelligence value, and integrate it into not only detection and prevention methods, but also more strategic-level initiatives such as risk assessments, prioritization of efforts, and future security investments. To get to the point where you can do all these things, you have to engage the intelligence portion of the F3EAD cycle: Exploit, Analyze, and Disseminate.
\end{quote}

\begin{list2}
\item Chapter 7: Exploit
\end{list2}

\slide{Why did the attackers succeed?}

%\hlkimage{}{}

\begin{quote}
In the Exploit phase, we begin the process that {\bf ensures that we learn} from the incident. We focus on the threat, and not just the enemy. Because of this, it is important that we {\bf not only extract technical indicators} related to the particular attack, such as malware samples and command-and-control IP addresses, but {\bf also the overarching aspects that led to the intrusion} and allowed the attackers to be, at least to some degree, successful.
\end{quote}
Source: Source: \emph{Intelligence-Driven Incident Response} (IDIR)

\begin{list2}
\item Avoiding similar incidents is the goal
\end{list2}


\slide{Gathering Information}

%\hlkimage{}{}

\begin{quote}
{\bf Gathering Information}\\
Depending on how you manage your incident-response data, it is entirely possible
that the most difficult part of the Exploit phase will be finding the important bits of
intelligence from the investigation. When it comes to {\bf gathering incident-response
data}, we have seen it {\bf all—from elaborate systems, to Excel spreadsheets, to Post-It
notes with IP addresses stuck to a whiteboard}. There is no wrong way to gather that
data, but if you want to be able to extract it so that it can be analyzed and used in the
future, there are certainly some ways to make the process easier.
\end{quote}
Source: Source: \emph{Intelligence-Driven Incident Response} (IDIR)


\slide{Storing Threat Information}

Formats listed in the book
\begin{list2}
\item OpenIOC
\item CyBOX, STIX, and TAXII
\item Incident Object Definition and Exchange Format (IODEF) RFC 5070
\item Real-time Inter-network Defense (RID)
\item Vocabulary for Event Recording and Incident Sharing (VERIS)
\item Common Attack Pattern Enumeration and Classification (CAPEC)
\item TLP White/Green/Amber/Red (Traffic Light Protocol)
\end{list2}

Don't worry about the names currently, remember these are often in JSON and XML

\slide{Part 1: Cases and Analyses}

%\hlkimage{}{}

\begin{quote}
Learning from others is easier than doing everything wrong yourself.
\end{quote}

Lets go and find some analysis of the Maersk NotPetya, official and not-official
\begin{list2}
\item Start at \emph{Cyber attack update JUNE 28, 2017}\\
\url{https://investor.maersk.com/news-releases/news-release-details/cyber-attack-update}
\item Lets gather facts about the case -- business facts and technical facts
\item Since our course is incident response, lets not forget how did they recover, did they learn, could this have been avoided altogether
\end{list2}


\slide{Part 2: Tools for Tracking Actions}

%\hlkimage{}{}

\begin{quote}
A variety of tools are available to track your incident data as well as the actions that
have been taken. This section covers ways to organize data, using both publicly avail‐
able and purpose-built tools. When you are just getting started with incident
response and do not have existing systems in place to track information and actions
that have been taken, it is often best to start small and grow into increased capability
and functionality.
\end{quote}
Source: Source: \emph{Intelligence-Driven Incident Response} (IDIR)

\begin{list2}
\item Evaluating tools and software packages.
\item We know we are going to be handling incidents, we need multiple tools
\end{list2}



\slide{Personal note taking apps}

\hlkimage{13cm}{idir-sod.png}

\begin{list2}
\item Zim personal wiki \url{https://zim-wiki.org/}
\item Obsidian \url{https://obsidian.md/}
\item Spreadsheet of Doom via IDIR 118
\end{list2}

\slide{Incident Response Apps -- Threat-Intelligence Platforms}

Let's discuss -- what do we need? Use the book as input, consider the following:

\begin{quote}
As you can probably tell from our coverage of standards and the numerous requirements for managing all the information that you have exploited during an investigation, capturing and analyzing all of this information is no trivial task. A {\bf threat intelligence platform} is often used to {\bf simplify} that process and make {\bf gathering, storing, and searching} this information {\bf easier}.
\end{quote}
Source: Source: \emph{Intelligence-Driven Incident Response} (IDIR)


\begin{list2}
\item Malware Information Sharing Platform (MISP)
\item Collaborative Research into Threats (CRITS)
\item Your Everyday Threat Intelligence (YETI)
\item FIR \url{https://github.com/certsocietegenerale/FIR} -- via IDIR book page 117\\
\emph{FIR is an open source ticketing system built from the
ground up to support intelligence-driven incident response.}
\end{list2}

\exercise{ex:misp-install}


\slide{Part 3: Threat Information gathering, selecting and presenting}

%\hlkimage{}{}
We will continue with the exercise, but also -- importantly -- start to consider what and how to present to management.

\begin{list2}
\item What do we tell the management, what do they need?
\item What should be prioritized?
\item Which feeds do we need, and partners\\
Example \url{https://www.ecrimelabs.com/danish-misp-user-group} \url{https://twitter.com/danishmisp}
\end{list2}


\slide{Part 4:Buffer and Valentina Palacin book}

One of the books in the Cybersecurity bundle is:\\
\emph{Practical Threat Intelligence and Data-Driven Threat Hunting A hands-on guide to threat hunting with the
ATT\&CK Framework and open source tools} Valentina Palacín

\begin{list2}
\item I already owned this book on paper, and consider it a great book
\item Very practically oriented
\item Lots of references to other tools, methods, standards
\item Has exercises within
\end{list2}

\slide{Event Tracing for Windows (ETW)}

%\hlkimage{}{}

\begin{quote}
Event Tracing for Windows (ETW) is a Windows debugging and diagnostic feature
that provides an "efficient kernel-level tracing facility that lets you log kernel or
application-defined events to a log file." ETW allows you to trace events in production
without computer or application restarts.

According to Microsoft, the Event Tracing API is broken into three components:
\begin{list2}
\item Event controllers (start and stop tracing sessions and enable providers)
\item Event providers
\item Event consumers
\end{list2}
\end{quote}

\slide{SilkeETW}

\begin{quote}
Ruben Boonen developed a tool called SilkETW that tries to help with this process and
allows you to download the ETW data in JSON format. This capability makes it really easy
to integrate the data that's been extracted with third-party SIEMs such as Elasticsearch
and Splunk. In addition, the JSON can be converted and exported into PowerShell and
you can combine Yara Rules with SilkETW to enhance your research.
\end{quote}

\begin{list2}
\item \url{https://github.com/fireeye/SilkETW} (also take note of the company FireEye)
\item Let's try it!
\end{list2}

\slide{Sysinternals}

The Valentina book also references the old-skool and always relevant Sysinternals tools!
\begin{quote}
If you have been keeping up with the threat hunting news lately, you may have seen that Sysmon seems to be everyone's favorite. System Monitoring (Sysmon) is part of {\bf Mark Russinovich's Sysinternals Suite} ( \url{https://docs.microsoft.com/en-us/sysinternals/downloads/sysinternals-suite} ). The reason why it gained such attention is because it turned out to be a great way to achieve endpoint visibility without impacting the system's performance.

Sysmon is a system service and device driver that monitors and logs system activity to
the {\bf Windows event log}. Sysmon configuration can be adjusted to better suit our collection
needs since it {\bf provides XML rules that can include and exclude uninteresting items}. The
list of available filter options increases with each Sysmon upgrade.
\end{quote}

\begin{list2}
\item Originally started as ntinternals back in 1996!
\end{list2}

\exercise{ex:dnssec-keytrap}

\slidenext{Buy the books! Create your VMs}

\end{document}
