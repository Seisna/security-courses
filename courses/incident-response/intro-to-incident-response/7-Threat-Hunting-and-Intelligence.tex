\documentclass[Screen16to9,17pt]{foils}
\usepackage{kea-slides}
\externaldocument{build/introduction-to-incident-response-exercises}
\selectlanguage{english}

% Input:
% https://www.threathunting.net/reading-list

% Docker image with
% https://hub.docker.com/r/threathuntproj/hunting/
% This image contains a complete threat hunting & data analysis environment built on Python, Pandas, PySpark and Jupyter notebook.

\begin{document}

\mytitlepage
{7. Threat Hunting and Intelligence}
{Introduction to Incident Response Elective, KEA}


\slide{Goals for today}

\hlkimage{6cm}{thomas-galler-hZ3uF1-z2Qc-unsplash.jpg}

\begin{list2}
\item Adversary emulation
\item Performing disk analysis
\item See what a privilege escalation can look like
\end{list2}

{\hfill \small Photo by Thomas Galler on Unsplash}

\slide{Plan for today}

\begin{list2}
\item Adversary emulation
\item Privilege escalation -- what is it
\item Example privesc on Linux
\item Finding persistence and bad files -- example with Linux
\end{list2}

Exercise theme:
\begin{list2}
\item Privilege escalation with SUID
\item Disk image forensics
\item Cloud environments influence on incident response
\end{list2}

\slide{Time schedule}

\begin{list2}
\item 1) Talk about another resource, Valentina Palacín book  -- first 45min
\item 2) Investigate a few links from this book -- 45 min
\item Break 15min
\item 3-4) Exercises privilege escalation and disk imaging -- 2x45min
\end{list2}


\slide{Reading Summary}



Browse chapter 1 from Practical Threat Intelligence, Valentina Palacín - if you have it

\emph{Practical Threat Intelligence and Data-Driven Threat Hunting}
Valentina Palacín, 2021, ISBN: 978-1-83855-637-2

\begin{list2}
\item Chapter 1: What Is Cyber Threat Intelligence?
\item This book is very much hands on with lots of links, references, tools and names
\item I will now present a bit from the book, since you don't have it
\end{list2}


\slide{Investigate links}

\begin{list2}
\item We cannot go through all of it, but we can get inspired
\item Since this came from real actors, campaigns, threats it is mostly what a real case would be
\end{list2}

This will lead to the later part, doing \emph{an investigation}


\slide{Investigate links 1: OSSEM}

%\hlkimage{}{}

\begin{quote}
OSSEM: To help with the heavy work of creating data dictionaries, the Rodriguez
brothers created the Open Source Security Events Metadata (OSSEM) for
documenting and standardizing security event logs. The project is open source and
can be accessed through the project's GitHub repository\\
\url{https://github.com/hunters-forge/OSSEM}.
\end{quote}


\slide{Investigate links 2: Threat Hunter Playbook}

%\hlkimage{}{}

\begin{quote}
The Threat Hunter Playbook: This open source project is maintained by the
Rodriguez brothers and is meant to help with the documentation project and
sharing threat hunting concepts, developing certain techniques, and building
the hypothesis. You can read more about it in the project's GitHub repository\\
\url{https://github.com/hunters-forge/ThreatHunter-Playbook}
\end{quote}

\slide{Investigate links 3: Adversary emulation}

%\hlkimage{}{}

\begin{quote}
Emulating the adversary: Adversary emulation is a way for red teamers to
replicate adversary behaviors in their organization's environments. In order to
do that, the adversary behaviors need to be mapped and the techniques used by
them need to be chained together to create an action plan. The MITRE ATT\&CK™
Framework provides an example of how to create an emulation plan based on APT3\\
\url{https://attack.mitre.org/resources/adversary-emulation-plans/}
\end{quote}

\slide{MITRE Adversary Emulation Plans}

\hlkimage{10cm}{Mitre-APT3_phase_diagram.png}

\begin{quote}
To showcase the practical use of ATT\&CK for offensive operators and defenders, MITRE created Adversary Emulation Plans. These are prototype documents of what can be done with publicly available threat reports and ATT\&CK.
\end{quote}
Source: \url{https://attack.mitre.org/resources/adversary-emulation-plans/}



\slide{Investigate links 4: Mordor dataset}

%\hlkimage{}{}

\begin{quote}
Mordor: For this stage of the hunt, the Rodriguez brothers created the Mordor
project, which provides
"pre-recorded security events generated by simulated adversarial techniques" in
JSON format.\\
 \url{https://github.com/hunters-forge/mordor}
\end{quote}



\exercise{ex:priv-esc-cron}

\exercise{ex:disk-image-forensics}

\exercise{ex:cloud-incident-response}


\slidenext{Read the books! Play with tools}

\end{document}
