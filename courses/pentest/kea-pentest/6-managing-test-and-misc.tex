\documentclass[Screen16to9,17pt]{foils}
\usepackage{zencurity-slides}

\externaldocument{kea-pentest-exercises}
\selectlanguage{english}

%VF 3 Netværkspenetrationstest (5 ECTS)
% Indhold
%Den studerende lærer om hvordan en penetration test udføres, samt kan indhente oplysninger om de seneste sårbarheder, og kan benytte sig af de relevante værktøjer til dette formål.
%Viden
%Den studerende viden om og forståelse for:
%* Etiske samt kontraktuelle forhold omkring en penetrationstest.
%* Standardiseringorganisationers og myndigheders krav til og om penetrationstesting
%Færdigheder
%Den studerende kan:
%Tage højde for sikkerhedsaspekter ved at:
%* Anvende relevante metoder ved udførsel af en penetrationstest
%* Udarbejde en angrebsplan ud fra indsamlede oplysninger om et mål
%* Finde sårbarheder i et givet system
% * Dokumentere og rapportere fundne sårbarheder
% Kompetencer
%Den studerende kan:
% * Planlægge en penetration test, samt eksekvere den både ved brug af værktøjer og manuelt.


\begin{document}

\mytitlepage
{6. Managing Pentests and Vulnerabilities}
{KEA Kompetence Penetration Testing}




\slide{Plan for today}

\begin{list1}
\item Subjects
\begin{list2}
\item Terminology and methods

\end{list2}
\item Exercises
\begin{list2}
\item JuiceShop attacks
\item a complete test, walk-through
\item Exam trial
\end{list2}
\item  Reading Curriculum:
\begin{list2}
\item Grayhat chapters 1 and 6-9
\end{list2}
\item  Reading Related resources:
\begin{list2}
\item \emph{Policies, governance and best Practice}
\end{list2}
\end{list1}



\slide{Goals for today}
\vskip 2 cm

%{\hlkbig En 3 dages workshop, hvor du lærer at angribe dit netværk!}
\hlkimage{3cm}{dont-panic.png}
\centerline{\color{titlecolor}\LARGE Don't Panic!}


\begin{list1}
\item 1h continue from last time, JuiceShop Attacks
\item 1h a complete test, going through the process of investigating a network, decisions and alternatives during testing
\item 1h Exam preparation, going over questions, trial exam
\end{list1}



\slide{Continue from last time, Web Application Attacks}

\exercise{ex:juiceshop-attack}

\slide{A complete Test}

\begin{list1}
\item 1h a complete test, going through the process of investigating a network, decisions and alternatives during testing
\end{list1}



\slide{Planning a pentest}

\begin{list1}
\item Scope must be agreed before
\begin{list2}
\item Scope -- what is being tested
\item When is the testing done -- time frame, wall clock time
\item Where are the attacks coming from -- log files will contain the attacks\\
but other attacks from other sources are likely during the attacks, which must be blocked
\item We sometimes go broader than the scope -- perhaps checking the router in front with SNMP or doing a small port 80/tcp scan
\item Agree if Denial of Service is to be tested
\item TL;DR Rules of engagement for the project
\end{list2}
\end{list1}

\slide{Selecting systems for testing}

\hlkimage{11cm}{overview-routing-customer-2015.png}

\begin{list2}
\item Routers on the way to critical systems and networks -- especially availability
\item Firewall -- is the environment protected sufficiently, discarding probes
\item Mail servere -- relay testing and also critical data
\item Web servere -- holds data, typically has a lot of functionality
\end{list2}


\slide{Testing Agreement and Scope Example}

\begin{list1}
\item Usually the scope would include targets like these:
\begin{list2}
\item 192.168.1.1 -- firewall/router
\item 192.168.1.2 -- mail server
\item 192.168.1.3 -- web server
\item Test to be done from monday 1st until friday 5th
\item Testing done from 192.0.2.0/28
\end{list2}
\item When testing web servers and sites, especially API -- please include hostname, URLs, documentation. If not included some sites and functionality will NOT be tested!
\end{list1}



\slide{OSI og Internet modellerne}

\hlkimage{10cm,angle=90}{images/compare-osi-ip.pdf}



\slide{Hvad skal der ske?}

\begin{list1}
\item Tænk som en hacker
\item Rekognoscering
\begin{list2}
\item ping sweep, port scan
\item OS detection -- TCP/IP eller banner grab
\item Servicescan -- rpcinfo, netbios, ...
\item telnet/netcat interaktion med services
\end{list2}
\item Udnyttelse/afprøvning: Metasploit, Nikto, exploit programs
\item Oprydning/hærdning vises måske ikke, men I bør i praksis:
\begin{list2}
\item Lav en rapport
\item Ændre, forbedre og hærde systemer
\item Gennemgå rapporten, registrer ændringer
\item Opdater programmer, konfigurationer, arkitektur, osv.
\end{list2}
\item I skal jo også VISE andre at I gør noget ved sikkerheden.
\end{list1}


\slide{Afrapportering -- resultater}

\begin{list1}
\item Hvad indeholder en sikkerhedstest rapport:
\begin{list2}
\item Titel, indholdsfortegnelse, firmanavne -- ca. 15-30 sider for 5 hosts
\item Fortrolighedserklæring -- det er fortrolige oplysninger
\item Executive summary -- ofte i større virksomheder
\item Information om den udførte scanning
\item Omfang/scope
\item Gennemgang af targets -- detaljeret information og med anbefalinger
\item Konklusion -- ofte mere teknisk
\item Bilag -- detaljerede oplysninger og oversigter, checklister
\end{list2}
\item Det er organisationen der selv vælger hvilke anbefalinger der følges
\end{list1}


\slide{Exam preparation}

Primary literature are these books and papers:
\begin{list2}
\item \emph{Gray Hat Hacking: The Ethical Hacker's Handbook}, fifth edition
Allen Harper and others
ISBN: 978-1-260-10841-5, May 2018, 640 pp.- shortened grayhat
\item \emph{Linux Basics for Hackers Getting Started with Networking, Scripting, and Security in Kali}. OccupyTheWeb, December 2018, 248 pp. ISBN-13: 978-1-59327-855-7 - shortened LBfH
\item \emph{Smashing The Stack For Fun And Profit}, by Aleph One
\item \emph{Return-Oriented Programming:Systems, Languages, and Applications}
Ryan Roemer, Erik Buchanan, Hovav Shacam and Stefan Savage University of California, San Diego

\end{list2}


\slide{Deliverables and Exam}


\begin{list2}
\item Exam
\item Individual: Oral based on curriculum
\item Graded (7 scale)
\item Draw a question with no preparation. Question covers a topic
\item Try to discuss the topic, and use practical examples
\item Exam is 30 minutes in total, including pulling the question and grading
\item Count on being able to present talk for about 10 minutes
\item Prepare material (keywords, examples, exercises, wireshark captures) for different topics so that you can use it to help you at the exam

\vskip 5mm
\item Deliverables:
\item 1 Mandatory assignments
\item Mandatory assignments are required in order to be entitled to go to the exam.
\end{list2}

\slide{Eksamensemner}

Eksamensemner til Pentest hos KEA Kompetence

Hjælp og keywords står under hvert emne.

Disse keywords vil IKKE være på sedlen inde i eksamenslokalet!

Pensum:\\
Bogens kapitler 1-3, 6-14, 22-25

Smashing The Stack For Fun And Profit, by Aleph One

Return-Oriented Programming:Systems, Languages, and Applications
Ryan Roemer, Erik Buchanan, Hovav Shacam and Stefan Savage University of California, San Diego

\slide{Emner og keywords}

\begin{list2}
\item {\bf 1. Programming and basic buffer overflows}\\
Trusler mod software, sårbarheder i software opstår, hvad er buffer overflow

\item {\bf 2. Advanced Vulnerabilities}\\
ROP mv. - en lille intro til buffer overflows vil være godt

\item {\bf 3. Network spoofing}\\
ARP spoofing, lav niveau, L2, MAC, og jeg ville tage lidt wifi sikkerhed - aflytningsdele med, Man in the middle angreb

\item {\bf 4. Cracking Passwords and secrets}\\

\item {\bf 5. Ethics and executing pentest}\\
Hvordan udføres pentest, processen
\end{list2}

\centerline{\bf Bemærk der er IKKE keywords inde i eksamenslokalet}

\slide{Emner og keywords, fortsat}

\begin{list2}
\item {\bf 6. Vulnerability disclosure}\\
Hvordan rapporteres vulnerabilities, Responsible disclosure, hvad kan man som firma gøre
for at sikre at man får gode rapporter ind, bl.a. oprette abuse@ email :-D

\item {\bf 7. Web application hacking}\\
Tag udgangspunkt i juice shop måske, oversigter over sårbarheder i Web, OWASP

\item {\bf 8. Metasploit}\\
Forklare hvad Metasploit er og kan. Herunder afvikling af test og udvikling af shell code
- at man ikke selv skal skrive, fordi det er inkluderet

\item {\bf 9. SSL/TLS Secure Sockets Layer / Transport Layer Security}\\
Web sikkerhed - krypteringsdelen,
hvorfor, hvordan man tester, hvordan man fixer, hvordan skal web protokoller sikres i 2019

\item {\bf 10. Kali Linux}\\
Hvad er Kali Linux, hvad er fordelen ved Kali Linux, hvordan kommer man igang med Kali, osv.
\end{list2}

\centerline{\bf Bemærk der er IKKE keywords inde i eksamenslokalet}




\slide{Course Description}

From: STUDIEORDNING Diplomuddannelse i it-sikkerhed August 2018

Indhold:\\
Den studerende lærer om hvordan en penetration test udføres, samt kan indhente oplysninger om de seneste sårbarheder, og kan benytte sig af de relevante værktøjer til dette formål.

Viden\\
Den studerende viden om og forståelse for:
\begin{list2}
\item Etiske samt kontraktuelle forhold omkring en penetrationstest.
\item Standardiseringorganisationers og myndigheders krav til og om penetrationstesting
\end{list2}

Færdigheder\\
Tage højde for sikkerhedsaspekter ved at:
\begin{list2}
\item Anvende relevante metoder ved udførsel af en penetrationstest
\item Udarbejde en angrebsplan ud fra indsamlede oplysninger om et mål
\item Finde sårbarheder i et givet system
 \item Dokumentere og rapportere fundne sårbarheder
\end{list2}

Kompetencer
Den studerende kan:
\begin{list2}
\item Planlægge en penetration test, samt eksekvere den både ved brug af værktøjer og manuelt
\end{list2}

% My translation:\\


Final word is the Studieordning which can be downloaded from\\
{\footnotesize \link{https://kompetence.kea.dk/uddannelser/it-digitalt/diplom-i-it-sikkerhed}\\
\link{Studieordning_for_Diplomuddannelsen_i_IT-sikkerhed_Aug_2018.pdf}}




\end{document}
