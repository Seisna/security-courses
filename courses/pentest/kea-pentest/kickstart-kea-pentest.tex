\documentclass[a4paper,11pt,notitlepage]{report}
% Henrik Lund Kramshoej, February 2001
% hlk@security6.net,
% My standard packages
\usepackage{zencurity-network-exercises}

\begin{document}

\rm
\selectlanguage{english}

\newcommand{\subject}[1]{Network Pentest}

%\mytitle{SIEM and Log Analysis}{exercises}
{\LARGE Kickstart: Network Pentest}{}


\normal

This material is prepared for use in \emph{\subject} and was prepared by Thomas Bach xtba@kea.dk and
Henrik Kramselund  , xhek@kea.dk
It contains the very basic information to get started!

These course and exercises are expected to be performed in a training setting with network connected systems. The exercises use a number of tools which can be copied and reused after training. A lot is described about setting up your workstation in the Github repositories.

{\bf The main site for materials are: \link{https://kea-fronter.itslearning.com/}}

I try to gather all information there!

So to get kickstarted in this course:
\begin{list2}
\item[\faSquareO] Make sure you can login to Fronter\\
\link{https://kea-fronter.itslearning.com/}
\item[\faSquareO] Lecture plan for this course is in Fronter
\item[\faSquareO] Read about setup of exercise systems here\\
\link{https://github.com/kramse/kramse-labs}
\item[\faSquareO] Check BIOS settings - make sure CPU settings have virtualisation turned ON
\item[\faSquareO] Select and install virtualisation software
\item[\faSquareO] Get the books! Either on paper or PDF
\end{list2}

Hopefully we will have a fun and enjoyable time in this course.

\end{document}
